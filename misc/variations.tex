\documentclass[12pt,a4paper]{scrartcl}

% Extended formatting of lists
\usepackage{enumitem}
\newlist{glossdefs}{itemize}{1}
\setlist[glossdefs]{nosep, leftmargin=3em, labelwidth=2.5em, align=left}

% Make multiple columns available in single-column document
\usepackage{multicol}

% Load font stuff for XeTeX
\usepackage{xltxtra}
\usepackage{fontspec}

% Fonts
\setmainfont{FreeSerif}[
 Ligatures=TeX,
]

% Clickable links in footnotes, TOC, etc.
\usepackage[
    xetex,
    hidelinks,
]{hyperref}

% Formatting of table of glossing abbreviations from Leipzig package manual
\usepackage[acronym,nomain,nonumberlist,nopostdot]{glossaries}
\usepackage{glossary-inline}%

\newglossarystyle{mysuper}{%
 \glossarystyle{super}% based on super
 \renewenvironment{theglossary}{%
 \begin{glossdefs}%
 }{%
 \end{glossdefs}%
 }%
 \renewcommand*{\glossaryheader}{}%
 \renewcommand*{\glsgroupheading}[1]{}%
 \renewcommand*{\glossaryentryfield}[5]{%
 \item[\glsentryitem{##1}\glstarget{##1}{##2}]
 \makefirstuc{##3}\glspostdescription{}
 }%
 \renewcommand*{\glsgroupskip}{}%
}%

% Formatting of glosses
\usepackage{expex}
\defineglwlevels{d}
\lingset{everygld=\scriptsize}

% Make ExPex process footnotes correctly.
% cf. http://tex.stackexchange.com/a/320271
\makeatletter
\def\makefootnotehacks#1{\begingroup
\XKV@for@n{#1}\which{%
\edef\temp{fnhack\which}%
\advance\c@footnote by 1
\expandafter\xdef\csname \temp\endcsname
{\fakesuperscript{\the\c@footnote}}}\endgroup\ignorespaces}%
\def\footnotehacktext{\advance\c@footnote by 1 \footnotetext}
\makeatother

\usepackage{leipzig}

\newleipzig{AgtT}{at}{agent topic}
\newleipzig{PatT}{pt}{patient topic}
\newleipzig{DatT}{datt}{dative topic}
\newleipzig{GenT}{gent}{genitive topic}
\newleipzig{LocT}{loct}{locative topic}
\newleipzig{InsT}{inst}{instrumental topic}
\newleipzig{CauT}{caut}{causative topic}
\newleipzig{An}{an}{animate}
\newleipzig{Inan}{inan}{inanimate}
\newleipzig{Hab}{hab}{habitative}
\newleipzig{Ayr}{ayr}{Ayeri}
\newleipzig{NPst}{npst}{near past}
\newleipzig{RPst}{rpst}{remote past}
\newleipzig{NFut}{nfut}{near future}
\newleipzig{RFut}{rfut}{remote future}
\newleipzig{Agtz}{agtz}{agentizer}
\newleipzig{Dim}{dim}{diminutive}
\newleipzig{Comp}{comp}{comparative}
\newleipzig{Supl}{supl}{superlative}

\newleipzig{Dyn}{dyn}{dynamic}
\newleipzig{Hort}{hort}{hortative}
\newleipzig{Iter}{iter}{iterative}
\newleipzig{Aff}{aff}{affirmative}
\newleipzig{Int}{int}{intensifier}

\newleipzig{Adj}{adj}{adjective}
\newleipzig{Adp}{adp}{adposition}
\newleipzig{Lnk}{lnk}{linker}
\newleipzig{Nn}{nn}{noun}
\newleipzig{Post}{post}{postposition}
\newleipzig{Prep}{prep}{preposition}
\newleipzig{Sat}{sat}{satellite}
\newleipzig{Vb}{vb}{verb}

\newleipzig{AF}{af}{argument function}
\newleipzig{DF}{df}{discourse function}
\newleipzig{Adjc}{adj}{adjunct}
\newleipzig{Anim}{anim}{animacy}
\newleipzig{Case}{case}{case}
\newleipzig{Compar}{compar}{comparison}
\newleipzig{Compl}{comp}{complement}
\newleipzig{Fin}{fin}{finiteness}
\newleipzig{Gend}{gend}{gender}
\newleipzig{Index}{index}{index}
\newleipzig{Num}{num}{number}
\newleipzig{Obj}{obj}{object}
\newleipzig{SObj}{obj\fakesubscript{\normalfont\itshape θ}}{secondary object}
\newleipzig{Oblique}{obl}{oblique}
\newleipzig{Pers}{pers}{person}
\newleipzig{Pred}{pred}{predicator}
\newleipzig{Spec}{spec}{specificity}
\newleipzig{Sbj}{subj}{subject}
\newleipzig{Tense}{tense}{tense}
\newleipzig{XCompl}{xcomp}{open complement}

\makeglossaries

\newcommand{\AargI}{{\Aarg}.{\Inan}}
\newcommand{\PargI}{{\Parg}.{\Inan}}
\newcommand{\AgtTI}{{\AgtT}.{\Inan}}
\newcommand{\PatTI}{{\PatT}.{\Inan}}

\newcommand{\TsgM}{{\Tsg}.{\M}}
\newcommand{\TsgF}{{\Tsg}.{\F}}
\newcommand{\TsgN}{{\Tsg}.{\N}}
\newcommand{\TsgI}{{\Tsg}.{\Inan}}
\newcommand{\TplM}{{\Tpl}.{\M}}
\newcommand{\TplF}{{\Tpl}.{\F}}
\newcommand{\TplN}{{\Tpl}.{\N}}
\newcommand{\TplI}{{\Tpl}.{\Inan}}

\newcommand{\Oblq}[1]{\Oblique\fakesubscript{\normalfont\itshape #1}}
\newcommand{\OblqT}{\Oblique\fakesubscript{\normalfont\itshape θ}}

% Ability to draw AVMs
% (Requires avm.sty from http://nlp.stanford.edu/manning/tex/)
\usepackage{avm}
\avmfont{\scshape}
\avmvalfont{\normalfont}

% Other macros
\newcommand{\til}{$\sim$} %{\textasciitilde{}} % Tilde shortcut

\begin{document}

\ex\begingl
\glpreamble \textit{Tahanya ang Yan.} \\
	/taˈhanja aŋ ˈjan/ //

\gla Tahanya ang Yan. //
\glb tahan-ya ang Yan //
\glc write-\TsgM{} \Aarg{} Yan //
\glft `Yan is writing.' //
\endgl\xe

\pex
\a\begingl
\glpreamble \textit{Tahanyāng.} \\
	/tahanˈjaːŋ/ //

\gla Tahanyāng. //
\glb tahan-yāng //
\glc write-\TsgM{}.\Aarg{} //
\glft `He is writing.' //
\endgl

\a\ljudge*\begingl
\gla Tahanya yāng. //
\glb tahan-ya yāng //
\glc write-\TsgM{} \TsgM{}.\Aarg{} //
\endgl

\a\ljudge*\begingl
\gla Tahanya. //
\glb tahan-ya //
\glc write-\TsgM{} //
\endgl

\xe

\pex
\a\begingl
\glpreamble \textit{Ang tahanya Yan tamanley.} \\
	/aŋ taˈhanja ˈjan tamanˈleɪ/ //

\gla Ang tahanya Yan tamanley. //
\glb ang tahan-ya Yan taman-ley //
\glc \AgtT{} write-\TsgM{} Yan letter-\PargI{} //
\glft `Yan is writing a letter.' //
\endgl

\a\ljudge*\begingl
\gla Tahanya ayonang tamanley. //
\glb tahan-ya ang ayon-ang taman-ley //
\glc write-\TsgM{} man-\Aarg{} letter-\PargI{} //
\glft \textit{Intended:} `A man is writing a letter.' //
\endgl

\xe

\pex
\a\begingl
\glpreamble \textit{Le tahanya ang Yan taman.} \\
	/le taˈhanja aŋ ˈjan ˈtaman/ //

\gla Le tahanya ang Yan taman. //
\glb le tahan-ya ang Yan taman //
\glc \PatTI{} write-\TsgM{} \Aarg{} Yan letter //
\glft `The letter, Yan is writing (it).' //
\endgl

\a\ljudge*\begingl
\gla Le tahanara ang Yan taman. //
\glb le tahan-ara Yan taman //
\glc \PatTI{} write-\TsgI{} \Aarg{} Yan letter //
\endgl

\a\ljudge*\begingl
\gla Ang le tahanara ayon taman. //
\glb ang le tahan-ara ayon taman //
\glc \Aarg{} \PatTI{} write-\TsgI{} man letter //
\glft \textit{Intended:} `The man is writing the letter.' //
\endgl

\xe

\pex
\a\begingl
\glpreamble \textit{Tahanara tamanley.} \\
	/tahaˈnara tamanˈleɪ/ //

\gla Tahanara tamanley. //
\glb tahan-ara taman-ley //
\glc write-\TsgI{} letter-\PargI{} //
\glft `A/the letter is being written.' //
\endgl

\a\ljudge*\begingl
\gla Tahanara tamanley ri Yan. //
\glb tahan-ara taman-ley ri Yan //
\glc write-\TsgI{} letter-\PargI{} \Ins{} Yan //
\glft \textit{Intended:} `A/the letter is being written by Yan.' //
\endgl

\a\ljudge*\begingl
\gla Le tahanya ang Yan. //
\glb le tahan-ya ang Yan //
\glc \PatTI{} write-\TsgM{} \Aarg{} Yan //
\endgl
\xe

\pex
\a\begingl
\glpreamble \textit{Ang tahanya Yan adaley.} \\
	/aŋ taˈhanja ˈjan adaˈleɪ/ //

\gla Ang tahanya Yan adaley. //
\glb ang tahan-ya Yan adaley //
\glc \AgtT{} write-\TsgM{} Yan that-\PargI{} //
\glft `Yan is writing it.' //
\endgl

\a\ljudge?\begingl
\gla Ang tahanya Yan rey. //
\glb ang tahan-ya Yan rey //
\glc \AgtT{} write-\TsgM{} Yan \TsgI{}.\Parg{} //
\endgl
\xe

\pex
\a\begingl
\glpreamble \textit{Le tahanya ang Yan adanya.} \\
	/aŋ taˈhanja aŋ ˈjan aˈdanja/ //

\gla Le tahanya ang Yan adanya. //
\glb le tahan-ya ang Yan adaley //
\glc \PatTI{} write-\TsgM{} \Aarg{} Yan that.one //
\glft `Yan is writing it.' //
\endgl

\a\ljudge?\begingl
\gla Le tahanya ang Yan ara. //
\glb le tahan-ya ang Yan ara //
\glc \PatTI{} write-\TsgM{} \Aarg{} Yan \TsgI{} //
\endgl

\a\ljudge*\begingl
\gla Le tahanyāra ang Yan. //
\glb le tahan-ya-ara ang Yan //
\glc \PatTI{} write-\TsgM{}-\TsgI{} \Aarg{} Yan //
\endgl
\xe

\ex\begingl
\glpreamble \textit{Ang ilya Ajān koyās yam Pila.}\\
	/aŋ‿ˈilja aˈdʒaːn koˈjaːs jam ˈpila/ //

\gla Ang ilya Ajān koyās yam Pila. //
\glb ang il-ya Ajān koya-as yam Pila //
\glc \AgtT{} give-\TsgM{} Ajān book-\Parg{} \Dat{} Pila //
\glft `Ajān gives Pila a book.' //
\endgl\xe

\ex\begingl
\glpreamble \textit{Sa ilya Ajān ada-koya yam Pila.}\\
	/sa ˈilja aˈdʒaːn adaˈkoja jam ˈpila/ //

\gla Sa ilya Ajān ada-koya yam Pila. //
\glb sa il-ya Ajān ada-koya yam Pila //
\glc \PatT{} give-\TsgM{} Ajān that-book \Dat{} Pila //
\glft `That book, Ajān gives (it) to Pila.' //
\endgl\xe

\pex
\a\begingl
\glpreamble \textit{Sa delaktang lanya, nay yasa.} \\
	/sa delakˈtang ˈlanja |naɪ ˈjasa/ //

\gla Sa delaktang lanya, nay yasa. //
\glb sa delak-tang lanya nay yasa //
\glc \PatT{} suffer-\TplM{} king and \TsgM{}.\Caus{} //
\glft `From the king they suffer, and due to him.' //
\endgl

\a\ljudge*\begingl
\gla Sa nay sā delaktang lanya. //
\glb sa nay sā delak-tang lanya //
\glc \PatT{} and \CauT{} suffer-\TplM{}.\Aarg{} king //
\glft \textit{Intended:} `From and due to the king they suffer.' //
\endgl

\xe

\ex\ljudge?\begingl
\glpreamble \textit{Sa ilya Ajān eda- nay ada-koya yam Pila.}\\
	/sa ˈilja aˈdʒaːn eda naɪ adaˈkoja jam ˈpila/ //

\gla Sa ilya Ajān eda- nay ada-koya yam Pila. //
\glb sa il-ya Ajān eda- nay ada-koya yam Pila //
\glc \PatT{} give-\TsgM{} Ajān this and that-book \Dat{} Pila //
\glft `This and that book, Ajān gives (them) to Pila.' //
\endgl\xe

\ex\begingl
\glpreamble \textit{Ang ilya Ajān koyās yam ada-Pila.}\\
	/aŋ‿ˈilja aˈdʒaːn koˈjaːs jam‿adaˈpila/ //

\gla Ang ilya Ajān koyās yam ada-Pila. //
\glb ang il-ya Ajān koya-as yam ada-Pila //
\glc \AgtT{} give-\TsgM{} Ajān book-\Parg{} \Dat{} that-Pila //
\glft `Ajān gives that Pila a book.' //
\endgl\xe

\ex\begingl
\glpreamble \textit{Ang ilya Ajān sa `Netuye Karamazov' yam Pila.}\\
	/aŋ‿ˈilja aˈdʒaːn sa neˈtuje karaˈmazov jam ˈpila/ //

\gla Ang ilya Ajān sa `Netuye Karamazov' yam Pila. //
\glb ang il-ya Ajān sa `Netu-ye Karamazov' yam Pila //
\glc \AgtT{} give-\TsgM{} Ajān \Parg{} `Brother-\Pl{} Karamazov' \Dat{} Pila //
\glft `Ajān gives Pila the “Brothers Karamazov”.' //
\endgl\xe

\ex\begingl
\glpreamble \textit{Ang ilya Ajān koyās yam Pila nay Latun.}\\
	/aŋ‿ˈilja aˈdʒaːn koˈjaːs jam ˈpila naɪ ˈlatun/ //

\gla Ang ilya Ajān koyās yam Pila nay Latun. //
\glb ang il-ya Ajān koya-as yam Pila nay Latun //
\glc \AgtT{} give-\TsgM{} Ajān book-\Parg{} \Dat{} Pila and Latun //
\glft `Ajān gives Pila and Latun a book.' //
\endgl\xe

\ex\begingl
\glpreamble \textit{Ang ilya Ajān koyās yam Pila nay netuyam yena.}\\
	/aŋ‿ˈilja aˈdʒaːn koˈjaːs jam ˈpila naɪ neˈtujam ˈjena/ //

\gla Ang ilya Ajān koyās yam Pila nay netuyam yena. //
\glb ang il-ya Ajān koya-as yam Pila nay netu-yam yena //
\glc \AgtT{} give-\TsgM{} Ajān book-\Parg{} \Dat{} Pila and brother-\Dat{} her.\Gen{} //
\glft `Ajān gives Pila and her brother a book.' //
\endgl\xe

\ex\begingl
\glpreamble \textit{Ang ilyan Ajān nay Diyan koyās yam Pila.}\\
	/aŋ‿ˈiljan aˈdʒaːn naɪ ˈdijan koˈjaːs jam ˈpila/ //

\gla Ang ilyan Ajān nay Diyan koyās yam Pila. //
\glb ang il-yan Ajān nay Diyan koya-as yam Pila //
\glc \AgtT{} give-\TplM{} Ajān and Diyan book-\Parg{} \Dat{} Pila //
\glft `Ajān and Diyan give Pila a book.' //
\endgl\xe

\ex\begingl
\glpreamble \textit{Ang ilya edāyon nay ledan koyās yam Pila.}\\
	/aŋ‿ˈilja eˈdaːjon naɪ ˈledan koˈjaːs jam ˈpila/ //

\gla Ang ilya edāyon nay ledan koyās yam Pila. //
\glb ang il-ya eda-ayon nay ledan koya-as yam Pila //
\glc \AgtT{} give-\TsgM{} this-man and friend book-\Parg{} \Dat{} Pila //
\glft `This man and friend gives Pila a book.' //
\endgl\xe

\ex\begingl
\glpreamble \textit{Ang ilyan edāyon nay eda-ledan koyās yam Pila.}\\
	/aŋ‿ˈiljan eˈdaːjon naɪ edaˈledan koˈjaːs jam ˈpila/ //

\gla Ang ilyan edāyon nay eda-ledan koyās yam Pila. //
\glb ang il-yan eda-ayon nay eda-ledan koya-as yam Pila //
\glc \AgtT{} give-\TplM{} this-man and this-friend book-\Parg{} \Dat{} Pila //
\glft `This man and this friend give Pila a book.' //
\endgl\xe

\ex\begingl
\glpreamble \textit{Ang ilya Ajān koyās nay migrayas yam Pila.}\\
	/aŋ‿ˈilja aˈdʒaːn koˈjaːs naɪ miˈgrajas jam ˈpila/ //

\gla Ang ilya Ajān koyās nay migrayas yam Pila. //
\glb ang il-ya Ajān koya-as nay migray-as yam Pila //
\glc \AgtT{} give-\TsgM{} Ajān book-\Parg{} and flower-\Parg{} \Dat{} Pila //
\glft `Ajān gives Pila a book and a pen.' //
\endgl\xe

\ex\begingl
\glpreamble \textit{Ang ilya Ajān koyās nay dadangley yam Pila.}\\
	/aŋ‿ˈilja aˈdʒaːn koˈjaːs naɪ dadaŋˈleɪ jam ˈpila/ //

\gla Ang ilya Ajān koyās nay dadangley yam Pila. //
\glb ang il-ya Ajān koya-as nay dadang-ley yam Pila //
\glc \AgtT{} give-\TsgM{} Ajān book-\Parg{} and pen-\PargI{} \Dat{} Pila //
\glft `Ajān gives Pila a book and a pen.' //
\endgl\xe

\ex\begingl
\glpreamble \textit{Ang ilya Ajān koyās-ikan yam Pila.}\\
	/aŋ‿ˈilja aˈdʒaːn koˈjaːsˌikan jam ˈpila/ //

\gla Ang ilya Ajān koyās-ikan yam Pila. //
\glb ang il-ya Ajān koya-as-ikan yam Pila //
\glc \AgtT{} give-\TsgM{} Ajān book-\Parg{}-many \Dat{} Pila //
\glft `Ajān gives Pila many books.' //
\endgl\xe

\pex
\a\begingl
\glpreamble \textit{Ang ilya nay paya Ajān koyajas māvayam nay yena.} \\
	/aŋ‿ˈilja naɪ ˈpaja aˈdʒaːn koˈjadʒas jam ˌmaːˈvajam naɪ ˈjena/ //

\gla Ang ilya nay paya Ajān koyajas māvayam nay yena. //
\glb ang il-ya nay pa-ya Ajān koya-ye-as māva-yam nay yena //
\glc \AgtT{} give-\TsgM{} and take-\TsgM{} Ajān book-\Pl{}-\Parg{} mother-\Dat{} and \TsgF{}.\Gen{} //
\glft `Ajān gives and takes books to and from mother.' //
\endgl

\a\ljudge*\begingl
\gla Ang ilya nay paya Ajān koyajas mavayam nay na. //
\glb ang il-ya nay pa-ya Ajān koya-ye-as mavayam nay na //
\glc \AgtT{} give-\TsgM{} and take-\TsgM{} Ajān book-\Pl{}-\Parg{} mother-\Dat{} and \Gen{} //
\endgl

\xe

\pex
\a\begingl
\glpreamble \textit{Ang ilya nay paya Ajān koyajas yam Pila nay yena.} \\
	/aŋ‿ˈilja naɪ ˈpaja aˈdʒaːn koˈjadʒas jam ˈpila naɪ ˈjena/ //

\gla Ang ilya nay paya Ajān koyajas yam Pila nay yena. //
\glb ang il-ya nay pa-ya Ajān koya-ye-as yam Pila nay yena //
\glc \AgtT{} give-\TsgM{} and take-\TsgM{} Ajān book-\Pl{}-\Parg{} \Dat{} Pila and \TsgF{}.\Gen{} //
\glft `Ajān gives and takes books to and from Pila.' //
\endgl

\a\ljudge*\begingl
\gla Ang ilya nay paya Ajān koyajas yam nay na Pila. //
\glb ang il-ya nay pa-ya Ajān koya-ye-as yam nay na Pila //
\glc \AgtT{} give-\TsgM{} and take-\TsgM{} Ajān book-\Pl{}-\Parg{} \Dat{} and \Gen{} Pila //
\endgl

\xe

\ex\begingl
\glpreamble \textit{Ang ilya Ajān koyajas nay dadangley-ikan merambay yam Pila.}\\
	/aŋ‿ˈilja aˈdʒaːn koˈjadʒas naɪ dadaŋˈleɪˌikan meramˈbaɪ jam ˈpila/ //

\gla Ang ilya Ajān koyajas nay dadangley-ikan merambay yam Pila. //
\glb ang il-ya Ajān koya-ye-as nay dadang-ley-ikan merambay yam Pila //
\glc \AgtT{} give-\TsgM{} Ajān book-\Pl{}-\Parg{} and pen-\PargI{}-many useful \Dat{} Pila //
\glft `Ajān gives Pila many useful books and pens.' //
\endgl\xe

\ex\begingl
\glpreamble \textit{Ang ilya Ajān koyajas, nay(nay) dadangley-ikan merambay yam Pila.}\\
	/aŋ‿ˈilja aˈdʒaːn koˈjadʒas naɪ(ˈnaɪ) dadaŋˈleɪˌikan meramˈbaɪ jam ˈpila/ //

\gla Ang ilya Ajān koyajas, nay(nay) dadangley-ikan merambay yam Pila. //
\glb ang il-ya Ajān koya-ye-as nay(nay) dadang-ley-ikan merambay yam Pila //
\glc \AgtT{} give-\TsgM{} Ajān book-\Pl{}-\Parg{} and(\til{}also) pen-\PargI{}-many useful \Dat{} Pila //
\glft `Ajān gives Pila books, and (also) many useful pens.' //
\endgl\xe

\ex\begingl
\glpreamble \textit{Ang ilya Ajān koyās-ikan nay dadangley merambay yam Pila.}\\
	/aŋ‿ˈilja aˈdʒaːn koˈjaːsˌikan naɪ dadaŋˌleɪ meramˈbaɪ jam ˈpila/ //

\gla Ang ilya Ajān koyās-ikan nay dadangley merambay yam Pila. //
\glb ang il-ya Ajān koya-as-ikan nay dadang-ley merambay yam Pila //
\glc \AgtT{} give-\TsgM{} Ajān book-\Parg{}-many and pen-\PargI{} useful \Dat{} Pila //
\glft `Ajān gives Pila many books and a useful pen.' //
\endgl\xe

\ex\begingl
\glpreamble \textit{Ang ilya Ajān koyās-ikan nay dadangyeley merambay yam Pila.}\\
	/aŋ‿ˈilja aˈdʒaːn koˈjaːsˌikan naɪ dadaŋˌjeleɪ meramˈbaɪ jam ˈpila/ //

\gla Ang ilya Ajān koyās-ikan nay dadangyeley merambay yam Pila. //
\glb ang il-ya Ajān koya-as-ikan nay dadang-ye-ley merambay yam Pila //
\glc \AgtT{} give-\TsgM{} Ajān book-\Parg{}-many and pen-\Pl{}-\PargI{} useful \Dat{} Pila //
\glft `Ajān gives Pila many books and useful pens.' //
\endgl\xe

\ex\begingl
\glpreamble \textit{Ang ilya Ajān koyajas nay dadangyeley merambay-ikan yam Pila.}\\
	/aŋ‿ˈilja aˈdʒaːn koˈjadʒas naɪ daˌdaŋjeˈleɪ meramˈbaɪˌikan jam ˈpila/ //

\gla Ang ilya Ajān koyajas nay dadangyeley merambay-ikan yam Pila. //
\glb ang il-ya Ajān koya-ye-as nay dadang-ye-ley merambay-ikan yam Pila //
\glc \AgtT{} give-\TsgM{} Ajān book-\Pl{}-\Parg{} and pen-\Pl{}-\PargI{} useful-very \Dat{} Pila //
\glft `Ajān gives Pila very useful books and pens.' //
\endgl\xe

\ex\begingl
\glpreamble \textit{Merambay sikan koyajang nay dadangyeley? — Da-ikan. (*Ikan.)}\\
	/meramˈbaɪ ˈsikan kojaˈdʒaŋ naɪ daˌdaŋjeˈleɪ/ — /daˈikan/ //

\gla Merambay sikan koyajang nay dadangyeley? — Da-ikan. //
\glb merambay sikan koya-ye-ang nay dadang-ye-ley {} da-ikan //
\glc useful how.much book-\Pl{}-\Aarg{} and pen-\Pl{}-\PargI{} {} so-very //
\glft `How useful are the books and pens?' — `Very.' //
\endgl\xe

\ex\begingl
\glpreamble \textit{Ang misya ku-depangas.}\\
	/aŋ ˈmisja kudeˈpaŋas/ //

\gla Ang misya ku-depangas. //
\glb ang mis-ya ku-depang-as //
\glc \AgtT{} act-\TsgM{} like-fool-\Parg{} //
\glft `He acts like a fool.' //
\endgl\xe

\ex\begingl
\glpreamble \textit{Sa misyāng ku-depang.}\\
	/sa misˈjaːŋ kudeˈpaŋ/ //

\gla Sa misya ku-depang. //
\glb sa mis-ya ku-depang //
\glc \PatT{} act-\TsgM{} like-fool //
\glft `Like a fool he acts.' //
\endgl\xe

\ex\begingl
\glpreamble \textit{Ang misya ku-da-depangas.}\\
	/aŋ ˈmisja kudadeˈpaŋas/ //

\gla Ang misya ku-da-depangas. //
\glb ang mis-ya ku-da-depang-as //
\glc \AgtT{} act-\TsgM{}.\Aarg{} like-such-fool-\Parg{} //
\glft `He acts like such a fool.' //
\endgl\xe

\ex\begingl
\glpreamble \textit{Ang misya nay lentaya ku-depangas.}\\
	/aŋ ˈmisja naɪ lenˈtaja kudeˈpaŋas/ //

\gla Ang misya nay lentaya ku-depangas. //
\glb ang mis-ya nay lenta-ya ku-depang-as //
\glc \AgtT{} act-\TsgM{} and sound-\TsgM{} like-fool-\Parg{} //
\glft `He acts and sounds like a fool.' //
\endgl\xe

\ex\begingl
\glpreamble \textit{Sa misyāng nay lentayāng ku-depang.}\\
	/sa misˈjaːng naɪ lentaˈjaːŋ kudeˈpaŋ/ //

\gla Sa misyāng nay lentayāng ku-depangas. //
\glb sa mis-yāng nay lenta-yāng ku-depang-as //
\glc \PatT{} act-\TsgM{}.\Aarg{} and sound-\TsgM{}.\Aarg{} like-fool-\Parg{} //
\glft `Like a fool he acts and sounds.' //
\endgl\xe

\ex\begingl
\glpreamble \textit{Ang misya ku-depangas nay karayās.}\\
	/aŋ ˈmisja kudeˈpaŋas naɪ karaˈjaːs/ //

\gla Ang misya ku-depangas nay karayās //
\glb ang mis-ya ku-depang-as nay karaya-as //
\glc \AgtT{} act-\TsgM{}.\Aarg{} like-fool-\Parg{} and coward-\Parg{} //
\glft `He acts like a fool and coward.' //
\endgl\xe

\ex\begingl
\glpreamble \textit{Ang misya ku-da-depangas nay karayās.}\\
	/aŋ ˈmisja kudadeˈpaŋas naɪ karaˈjaːs/ //

\gla Ang  misyāng ku-da-depangas nay karayās //
\glb ang mis-yāng ku-da-depang-as nay karaya-as //
\glc \AgtT{} act-\TsgM{}.\Aarg{} like-such-fool-\Parg{} and coward-\Parg{} //
\glft `He acts like such a fool and coward.' //
\endgl\xe

\ex\begingl
\glpreamble \textit{Ang silvye ku-māvās yena.}\\
	/aŋ ˈsilvje kuˌmaːˈvaːs ˈjena/ //

\gla Ang silvye ku-māvās yena. //
\glb ang silv-ye ku-māva-as yena //
\glc \AgtT{} look-\TsgF{} like-mother-\Parg{} \TsgF{}.\Gen{} //
\glft `She looks like her mother.' //
\endgl\xe

\ex\begingl
\glpreamble \textit{Ang silvye Pada ku-māvās yena.}\\
	/aŋ ˈsilvje ˈpada kuˌmaːˈvaːs ˈjena/ //

\gla Ang silvye Pada ku-māvās yena. //
\glb ang silv-ye Pada ku-māva-as yena //
\glc \AgtT{} look-\TsgF{} Pada like-mother-\Parg{} \TsgF{}.\Gen{} //
\glft `Pada looks like her mother.' //
\endgl\xe

\pex
\a\begingl
\glpreamble \textit{Ang haruya Pulan sa Linko.} \\
	/aŋ haˈruja ˈpulan sa ˈliŋko/ //

\gla Ang haruya Pulan sa Linko. //
\glb ang haru-ya Pulan sa Linko //
\glc \AgtT{} beat-\TsgM{} Pulan \Parg{} Linko //
\glft `Pulan beats Linko.' //
\endgl

\a\ljudge*\begingl
\gla Ang haruya ang Pulan sa Linko. //
\glb ang haru-ya ang Pulan sa Linko //
\glc \AgtT{} beat-\TsgM{} \Aarg{} Pulan \Parg{} Linko //
\endgl
\xe

\ex\begingl
\glpreamble \textit{Ang haruya Pulan yās.}\\
	/aŋ haˈruja ˈpulan yās/ //

\gla Ang haru-ya Pulan yās. //
\glb ang haru-ya Pulan yās //
\glc \AgtT{} beat-\TsgM{} Pulan \TsgM{}.\Parg{} //
\glft `Pulan beats him.' //
\endgl\xe

\pex
\a\begingl
\glpreamble \textit{Ang haruya sa Linko.} \\
	/aŋ haˈruja sa ˈliŋko/ //

\gla Ang haruya sa Linko. //
\glb ang haru-ya sa Linko //
\glc \AgtT{} beat-\TsgM{} \Parg{} Linko //
\glft `He beats Linko.' //
\endgl

\a\ljudge*\begingl
\gla Ang haruya yāng sa Linko. //
\glb ang haru-ya yāng sa Linko //
\glc \AgtT{} beat-\TsgM{} \TsgM{}.\Aarg{} \Parg{} Linko //
\endgl
\xe

\ex\begingl
\glpreamble \textit{Ang haruya yās.}\\
	/aŋ haˈruja ˈjās/ //

\gla Ang haruya yās. //
\glb ang haru-ya yās //
\glc \AgtT{} beat-\TsgM{} \TsgM{}.\Parg{} //
\glft `He beats him.' //
\endgl\xe

\pex
\a\begingl
\glpreamble \textit{Sa haruya ang Pulan Linko.} \\
	/sa haˈruja aŋ ˈpulan ˈliŋko/ //

\gla Sa haruya ang Pulan Linko. //
\glb sa haru-ya ang Pulan Linko //
\glc \PatT{} beat-\TsgM{} \Aarg{} Pulan Linko //
\glft `Linko, Pulan beats (him).' //
\endgl

\a\ljudge*\begingl
\gla Sa haruya ang Pulan sa Linko. //
\glb sa haru-ya ang Pulan sa Linko //
\glc \PatT{} beat-\TsgM{} \Aarg{} Pulan \Parg{} Linko //
\endgl\xe

\pex
\a\begingl
\glpreamble \textit{Sa haruya ang Pulan Sipra.} \\
	/sa haˈruja aŋ ˈpulan ˈsipra/ //

\gla Sa haruya ang Pulan Sipra. //
\glb sa haru-ya ang Pulan Sipra //
\glc \PatT{} beat-\TsgM{} \Aarg{} Pulan Sipra //
\glft `Sipra, Pulan beats (her).' //
\endgl

\a\ljudge*\begingl
\gla Sa haruye ang Pulan Sipra. //
\glb sa haru-ya ang Pulan Sipra //
\glc \PatT{} beat-\TsgF{} \Aarg{} Pulan Sipra //
\endgl\xe

\pex
\a\begingl
\glpreamble \textit{Sa haruya para ang Pulan ya.} \\
	/sa haˈruja ˈpara aŋ ˈpulan ja/ //

\gla Sa haruya para ang Pulan ya. //
\glb sa haru-ya para ang Pulan ya //
\glc \PatT{} beat-\TsgM{} quickly \Aarg{} Pulan \TsgM{} //
\glft `It is him Pulan quickly beats.' //
\endgl

\a\ljudge*\begingl
\gla Sa haruya para ang Pulan yās. //
\glb sa haru-ya para ang Pulan yās //
\glc \PatT{} beat-\TsgM{} quickly \Aarg{} Pulan \TsgM{}.\Parg{} //
\endgl\xe

\ex\begingl
\glpreamble \textit{Sa haruya para ya ang Pulan.}\\
	/sa haˈruja ˈpara ja aŋ ˈpulan/ //

\gla Sa haruya para ya ang Pulan. //
\glb sa haru-ya para ya ang Pulan //
\glc \PatT{} beat-\TsgM{} quickly \TsgM{} \Aarg{} Pulan //
\glft `It is him Pulan quickly beats.' //
\endgl\xe

\ex\begingl
\glpreamble \textit{Sa haruyāng ya.}\\
	/sa haruˈjaːŋ ja/ //

\gla Sa haruyāng ya. //
\glb sa haru-yāng ya //
\glc \PatT{} beat-\TsgM{}.\Aarg{} \TsgM{} //
\glft `It is him that he beats.' //
\endgl\xe

\ex\begingl
\glpreamble \textit{Haruyās.}\\
	/haruˈjaːs/ //

\gla Haruyās. //
\glb haru-yās //
\glc beat-\TsgM{}.\Parg{} //
\glft `He is (being) beaten.' //
\endgl\xe

\ex\begingl
\glpreamble \textit{Ang haruya Pulan sa Linko lundari.}\\
	/aŋ haˈruja ˈpulan sa ˈliŋko lunˈdari/ //

\gla Ang haruya Pulan sa Linko lundari. //
\glb ang haru-ya Pulan sa Linko lunda-ri //
\glc \AgtT{} beat-\TsgM{} Pulan \Parg{} Linko shoe-\Ins{} //
\glft `Pulan beats Linko with a shoe.' //
\endgl\xe

\ex\begingl
\glpreamble \textit{Ri haruya ang Pulan sa Linko lunda.}\\
	/ri haˈruja aŋ ˈpulan sa ˈliŋko ˈlunda/ //

\gla Ri haruya ang Pulan sa Linko lundari. //
\glb ri haru-ya ang Pulan sa Linko lunda-ri //
\glc \InsT{} beat-\TsgM{} \Aarg{} Pulan \Parg{} Linko shoe-\Ins{} //
\glft `It is with a/the shoe that Pulan beats Linko.' //
\endgl\xe


\ex\begingl
\glpreamble \textit{Ri haruya para ang Pulan sa Linko adanya.}\\
	/ri haˈruja para aŋ ˈpulan sa ˈliŋko aˈdanja/ //

\gla Ri haruya para ang Pulan sa Linko adanya. //
\glb ri haru-ya para ang Pulan sa Linko adanya //
\glc \InsT{} beat-\TsgM{} quickly \Aarg{} Pulan \Parg{} Linko that //
\glft `It is with that that Pulan quickly beats Linko.' //
\endgl\xe

\ex\begingl
\glpreamble \textit{Ri haruya para adanya ang Pulan sa Linko.}\\
	/ri haˈruja para aˈdanja aŋ ˈpulan sa ˈliŋko ˈara/ //

\gla Ri haruya para adanya ang Pulan sa Linko. //
\glb ri haru-ya para adanya ang Pulan sa Linko //
\glc \InsT{} beat-\TsgM{} quickly that \Aarg{} Pulan \Parg{} Linko //
\glft `It is with that that Pulan quickly beats Linko.' //
\endgl\xe

\ex\begingl
\glpreamble \judge?\textit{Ang sarāyn ay nay va kodanya.} \\
	\textit{Ang sarāyn kodanya, yang nay vāng.} \\
	/ja saraˈnaŋ koˈdanja |ˈjaŋ naɪ ˈvaːŋ/ //

\gla Ang sarāyn kodanya,  yang nay vāng. //
\glb Ang sara-ayn kodan-ya yang nay vāng //
\glc \AgtT{} go-\Fpl{} lake-\Loc{} \Second{}.\Aarg{} and \Fsg{}.\Aarg{} //
\glft `You and I are going to the lake.' / `We are going to the lake, you and I.' //
\endgl

\ex\begingl
\glpreamble \judge?\textit{Ya sarāyn yang nay vāng kodan.} \\
	\textit{Ya saranang kodan, yang nay vāng.} \\
	/ja saraˈnaŋ koˈdanja |ˈjaŋ naɪ ˈvaːŋ/ //

\gla Ya saranang kodan, yang nay vāng. //
\glb ya sara-nang kodan yang nay vāng //
\glc \LocT{} go-\Fpl{}.\Aarg{} lake \Second{}.\Aarg{} and \Fsg{}.\Aarg{} //
\glft `You and I are going to the lake.' / `We are going to the lake, you and I.' //
\endgl

\ex\begingl
\glpreamble \textit{Ang koronay sitang-nyama guratanley.}\\
	/aŋ koroˈnaɪ siˈtaŋˌnjama guratanˈleɪ/ //

\gla Ang koronay sitang-nyama guratanley. //
\glb ang koron-ay sitang-nyama guratan-ley //
\glc \AgtT{} know-\Fsg{} self-even answer-\Parg{} //
\glft `Even I know the answer.' //
\endgl\xe

\ex\begingl
\glpreamble \textit{Le koronyang sitang-nyama guratan.}\\
	/le koronˈjaŋ siˈtaŋˌnjama guˈratan/ //

\gla Le koronyang sitang-nyama guratan. //
\glb le koron-yang sitang-nyama guratan //
\glc \PatT{} know-\Fsg{}.\Aarg{} self-even answer //
\glft `The answer, even I know (it).' //
\endgl\xe

\ex\begingl
\glpreamble \textit{Ang kece nay dayungisaye māva yanjas yena.}\\
	/aŋ ˈketʃe naɪ dajuŋiˈsaje ˈmaːva ˈjandʒas ˈjena/ //

\gla Ang kece nay dayungisaye māva yanjas yena. //
\glb ang ket-ye nay dayungisa-ye māva yan-ye-as yena //
\glc \AgtT{} wash-\TsgF{} and dress-\TsgF{} mother boy-\Pl{}-\Parg{} \TsgF{}.\Gen{} //
\glft `The mother washes and dresses her boys.' //
\endgl\xe

\ex\begingl
\glpreamble \textit{Ang kece nay dayungisaye tas.}\\
	/aŋ ˈketʃe naɪ dajuŋiˈsaje tas/ //

\gla Ang kece nay dayungisaye tas. //
\glb ang ket-ye nay dayungisa-ye tas //
\glc \AgtT{} wash-\TsgF{} and dress-\TsgF{} \TplM{}.\Parg{} //
\glft `She washes and dresses them.' //
\endgl\xe

\ex\begingl
\glpreamble \textit{Sa keceng nay dayungisayeng yan.}\\
	/sa keˈtʃeŋ naɪ dajuŋisaˈjeŋ jan/ //

\gla Sa keceng nay dayungisayeng yan. //
\glb sa ket-yeng nay dayungisa-yeng yan //
\glc \PatT{} wash-\TsgF{} and dress-\TsgF{} \TplM{} //
\glft `Them it is whom she washes and dresses.' //
\endgl\xe

\ex\begingl
\glpreamble \textit{Sa ilyāng koyaye nay dadangye yeyam.}\footnotemark{} \\
	/sa ilˈjaːŋ koˈjaje naɪ daˈdaŋje jeˈjam/ //

\gla Sa ilyāng koyaye nay dadangye yeyam. //
\glb sa il-yāng koya-ye nay dadang-ye yeyam //
\glc \PatT{} give-\TsgM{} book-\Pl{} and pen-\Pl{} \TsgF{}.\Dat{} //
\gld %
	\mbox{\begin{avm}\[ \Anim{} & + \]\end{avm}}
	{}
	\mbox{\begin{avm}\[ \Anim{} & + \]\end{avm}}
	{}
	\mbox{\begin{avm}\[ \Anim{} & – \]\end{avm}}
	//
\glft `Books and pens he gives her.' //
\endgl

\footnotetext{I think that due to the way Ayeri deals with verb agreement, it makes more sense for it overall to have nearest-conjunct agreement with regards to gender/animacy than gender resolution, though see below. Number always resolves, though.}

% IDEA:
%
% M + F = M (N attested)
% F + M = M (N attested)
%
% → people have gender-resolution to M
% → resolution to N does occur
%
% M + N = M
% N + M = N
% F + N = M (F possible)
% N + F = N
%
% → people mixed with animate things have nearest-conjunct resolution
% → if F first, resolution to M (though nearest-conjunct resolution possible)
%
% M    + INAN = M
% INAN + M    = INAN
% F    + INAN = M (F possible)
% INAN + F    = INAN
% N    + INAN = N
% INAN + N    = INAN
%
% → people mixed with inanimate things have nearest-conjunct resolution
% → if F first, resolution to M (though nearest-conjunct resolution possible)

\ex\begingl
\glpreamble \textit{Ang sobisayan Mico nay Niva sungkoranas narān.} \\
	/aŋ sobiˈsajan ˈmitʃo naɪ ˈniva suŋkoˈranas naˈra:n/ //

\gla Ang sobisayan Mico nay Niva sungkoranas narān. //
\glb ang sobisa-yan Mico nay Niva sungkoran-as narān //
\glc \AgtT{} study-\TplM{} Mico and Niva science-\Parg{} language //
\gld %
	\mbox{\begin{avm}\[ \Anim{} & + \]\end{avm}}
	\mbox{\begin{avm}\[
		\Anim{} & + \\
		\Gend{} & \M{} \\
		\Num{}  & \Pl{} \\
	\]\end{avm}}
	\mbox{\begin{avm}\[
		\Anim{} & + \\
		\Gend{} & \M{} \\
		\Num{}  & \Sg{} \\
	\]\end{avm}}
	{}
	\mbox{\begin{avm}\[
		\Anim{} & + \\
		\Gend{} & \F{} \\
		\Num{}  & \Sg{} \\
	\]\end{avm}}
	//
\glft `Mico and Niva study linguistics.' //
\endgl\xe

\ex\begingl
\glpreamble \textit{Ang sobisayan Niva nay Mico sungkoranas narān.} \\
	/aŋ sobiˈsajan ˈniva naɪ ˈmitʃo suŋkoˈranas naˈra:n/ //

\gla Ang sobisayan Niva nay Mico sungkoranas narān. //
\glb ang sobisa-yan Niva nay Mico sungkoran-as narān //
\glc \AgtT{} study-\TplM{} Niva and Mico science-\Parg{} language //
\gld %
	\mbox{\begin{avm}\[ \Anim{} & + \]\end{avm}}
	\mbox{\begin{avm}\[
		\Anim{} & + \\
		\Gend{} & \M{} \\
		\Num{}  & \Pl{} \\
	\]\end{avm}}
	\mbox{\begin{avm}\[
		\Anim{} & + \\
		\Gend{} & \F{} \\
		\Num{}  & \Sg{} \\
	\]\end{avm}}
	{}
	\mbox{\begin{avm}\[
		\Anim{} & + \\
		\Gend{} & \M{} \\
		\Num{}  & \Sg{} \\
	\]\end{avm}}
	//
\glft `Niva and Mico study linguistics.' //
\endgl\xe

\ex\begingl
\glpreamble \textit{Toryan ang Yan nay veneyang yana.} \\
	/ˈtorjan aŋ ˈjan naɪ veneˈjang ˈjana/ //

\gla Toryan ang Yan nay veneyang yana. //
\glb tor-yan ang Yan nay veney-ang yana //
\glc sleep-\TplM{} \Aarg{} Yan and dog-\Aarg{} \TsgM{}.\Aarg{} //
\gld %
	\mbox{\begin{avm}\[
		\Anim{} & + \\
		\Gend{} & \M{} \\
		\Num{}  & \Pl{} \\
	\]\end{avm}}
	{}
	\mbox{\begin{avm}\[
		\Anim{} & + \\
		\Gend{} & \M{} \\
		\Num{}  & \Sg{} \\
	\]\end{avm}}
	{}
	\mbox{\begin{avm}\[
		\Anim{} & + \\
		\Gend{} & \N{} \\
		\Num{}  & \Sg{} \\
	\]\end{avm}}
	//
\glft `Yan and his dog are sleeping.' //
\endgl\xe

\ex\begingl
\glpreamble \textit{Toryon veneyang nay badanang.} \\
	/ˈtorjon veneˈjang naɪ badaˈnaŋ/ //

\gla Toryon veneyang nay badanang. //
\glb tor-yon veney-ang nay badan-ang //
\glc sleep-\TplN{} dog-\Aarg{} and father-\Aarg{} //
\gld %
	\mbox{\begin{avm}\[
		\Anim{} & + \\
		\Gend{} & \N{} \\
		\Num{}  & \Pl{} \\
	\]\end{avm}}
	\mbox{\begin{avm}\[
		\Anim{} & + \\
		\Gend{} & \N{} \\
		\Num{}  & \Sg{} \\
	\]\end{avm}}
	{}
	\mbox{\begin{avm}\[
		\Anim{} & + \\
		\Gend{} & \M{} \\
		\Num{}  & \Sg{} \\
	\]\end{avm}}
	//
\glft `The dog and father are sleeping.' //
\endgl\xe

\ex\begingl
\glpreamble \textit{Toryan mavāng nay veneyang.}\footnotemark{} \\
	/ˈtorjan maˈvaːŋ naɪ veneˈjang/ //

\gla Toryan mavāng nay veneyang. //
\glb tor-yan mava-ang nay veney-ang //
\glc sleep-\TplM{} mother-\Aarg{} and dog-\Aarg{} //
\gld %
	\mbox{\begin{avm}\[
		\Anim{} & + \\
		\Gend{} & \M{} \\
		\Num{}  & \Pl{} \\
	\]\end{avm}}
	\mbox{\begin{avm}\[
		\Anim{} & + \\
		\Gend{} & \F{} \\
		\Num{}  & \Sg{} \\
	\]\end{avm}}
	{}
	\mbox{\begin{avm}\[
		\Anim{} & + \\
		\Gend{} & \N{} \\
		\Num{}  & \Sg{} \\
	\]\end{avm}}
	//
\glft `Mother and the dog are sleeping.' //
\endgl\xe

\footnotetext{Lack of gender resolution with regular nearest-conjunct agreement ought to be possible as well.}

\ex\begingl
\glpreamble \textit{Toryon veneyang nay mavāng.} \\
	/ˈtorjon veneˈjang naɪ maˈvaːŋ/ //

\gla Toryon veneyang nay mavāng. //
\glb tor-yon veney-ang nay mava-ang //
\glc sleep-\TplN{} dog-\Aarg{} and mother-\Aarg{} //
\gld %
	\mbox{\begin{avm}\[
		\Anim{} & + \\
		\Gend{} & \N{} \\
		\Num{}  & \Pl{} \\
	\]\end{avm}}
	\mbox{\begin{avm}\[
		\Anim{} & + \\
		\Gend{} & \N{} \\
		\Num{}  & \Sg{} \\
	\]\end{avm}}
	{}
	\mbox{\begin{avm}\[
		\Anim{} & + \\
		\Gend{} & \F{} \\
		\Num{}  & \Sg{} \\
	\]\end{avm}}
	//
\glft `The dog and mother are sleeping.' //
\endgl\xe

\ex\begingl
	\glpreamble \textit{Ang sarayan kadanya Api-Api nay ajam yana tadayen.} \\
		/aŋ saˈrajan kaˈdanja ˈapiˈapi naɪ ˈadʒam ˈjana taˈdajen/ //

	\gla Ang sarayan kadanya Api-Api nay ajam yana tadayen. //
	\glb ang sara-yan kadanya Api-Api nay ajam yana tadayen //
	\glc \AgtT{} go-\TplM{} together Api-Api and toy \TsgM{}.\Gen{} everywhere //
	\gld %
		\mbox{\begin{avm}\[
			\Anim{} & + \\
		\]\end{avm}}
		\mbox{\begin{avm}\[
			\Anim{} & + \\
			\Gend{} & \M{} \\
			\Num{}  & \Pl{} \\
		\]\end{avm}}
		{}
		\mbox{\begin{avm}\[
			\Anim{} & + \\
			\Gend{} & \M{} \\
			\Num{}  & \Sg{} \\
		\]\end{avm}}
		{}
		\mbox{\begin{avm}\[
			\Anim{} & – \\
			\\
			\Num{}  & \Sg{} \\
		\]\end{avm}}
		//
	\glft `Api-Api and his toy are going everywhere together.' //
\endgl\xe

\ex\begingl
	\glpreamble \textit{Eng sarāran kadanya ajam nay Api-Api tadayen.} \\
		/eŋ saˈraːran kaˈdanja ˈadʒam naɪ ˈapiˈapi taˈdajen/ //

	\gla Eng sarāran kadanya ajam nay Api-Api tadayen. //
	\glb Eng sara-aran kadanya ajam nay Api-Api tadayen //
	\glc \AgtTI{} go-\TplI{} together toy and Api-Api everywhere //
	\gld %
		\mbox{\begin{avm}\[
			\Anim{} & – \\
		\]\end{avm}}
		\mbox{\begin{avm}\[
			\Anim{} & – \\
			\\
			\Num{}  & \Pl{} \\
		\]\end{avm}}
		{}
		\mbox{\begin{avm}\[
			\Anim{} & – \\
			\\
			\Num{}  & \Sg{} \\
		\]\end{avm}}
		{}
		\mbox{\begin{avm}\[
			\Anim{} & + \\
			\Gend{} & \M{} \\
			\Num{}  & \Sg{} \\
		\]\end{avm}}
		//
	\glft `The toy and Api-Api are going everywhere together.' //
\endgl\xe

\ex\begingl
	\glpreamble \textit{Ang sarayan kadanya Tavi-Tavi nay ajam yena tadayen.}\footnotemark{} \\
		/aŋ saˈrajan kaˈdanja ˈtaviˈtavi naɪ ˈadʒam ˈjena taˈdajen/ //

	\gla Ang sarayan kadanya Tavi-Tavi nay ajam yena tadayen. //
	\glb ang sara-yan kadanya Tavi-Tavi nay ajam yena tadayen //
	\glc \AgtT{} go-\TplM{} together Tavi-Tavi and toy \TsgF{}.\Gen{} everywhere //
	\gld %
		\mbox{\begin{avm}\[
			\Anim{} & + \\
		\]\end{avm}}
		\mbox{\begin{avm}\[
			\Anim{} & + \\
			\Gend{} & \M{} \\
			\Num{}  & \Pl{} \\
		\]\end{avm}}
		{}
		\mbox{\begin{avm}\[
			\Anim{} & + \\
			\Gend{} & \F{} \\
			\Num{}  & \Sg{} \\
		\]\end{avm}}
		{}
		\mbox{\begin{avm}\[
			\Anim{} & – \\
			\\
			\Num{}  & \Sg{} \\
		\]\end{avm}}
		//
	\glft `Tavi-Tavi and her toy are going everywhere together.' //
\endgl\xe

\footnotetext{Lack of gender resolution with regular nearest-conjunct agreement ought to be possible as well.}

\ex\begingl
	\glpreamble \textit{Eng sarāran kadanya ajam nay Tavi-Tavi tadayen.} \\
		/eŋ saˈraːran kaˈdanja ˈadʒam naɪ ˈtaviˈtavi taˈdajen/ //

	\gla Eng sarāran kadanya ajam nay Tavi-Tavi tadayen. //
	\glb Eng sara-aran kadanya ajam nay Tavi-Tavi tadayen //
	\glc \AgtTI{} go-\TplI{} together toy and Tavi-Tavi everywhere //
	\gld %
		\mbox{\begin{avm}\[
			\Anim{} & – \\
		\]\end{avm}}
		\mbox{\begin{avm}\[
			\Anim{} & – \\
			\\
			\Num{}  & \Pl{} \\
		\]\end{avm}}
		{}
		\mbox{\begin{avm}\[
			\Anim{} & – \\
			\\
			\Num{}  & \Sg{} \\
		\]\end{avm}}
		{}
		\mbox{\begin{avm}\[
			\Anim{} & + \\
			\Gend{} & \F{} \\
			\Num{}  & \Sg{} \\
		\]\end{avm}}
		//
	\glft `The toy and Tavi-Tavi are going everywhere together.' //
\endgl\xe

\ex\begingl
\glpreamble \textit{Ang bitojon sungkoran-narān nay payutān atasas.} \\
	/aŋ biˈtodʒon suŋˌkoranaˈraːn naɪ pajuˈtaːn aˈtasas/ //

\gla Ang bitojon sungkoran-narān nay payutān atasas. //
\glb ang bitog-yon sungkoran-narān nay payutān atas-as //
\glc \AgtT{} tear.apart-\TplN{} science.language and mathematics brain-\Parg{} //
\gld %
	\mbox{\begin{avm}\[ \Anim{} & + \]\end{avm}}
	\mbox{\begin{avm}\[
		\Anim{} & + \\
		\Gend{} & \N{} \\
		\Num{}  & \Pl{} \\
	\]\end{avm}}
	\mbox{\begin{avm}\[
		\Anim{} & + \\
		\Gend{} & \N{} \\
		\Num{}  & \Sg{} \\
	\]\end{avm}}
	{}
	\mbox{\begin{avm}\[
		\Anim{} & – \\
		\\
		\Num{} & \Sg{} \\
	\]\end{avm}}
	//
\glft `Linguistics and math tear the brain apart.' //
\endgl\xe

\ex\begingl
\glpreamble \textit{Eng bitogaran payutān nay sungkoran-narān atasas.} \\
	/eŋ bitoˈgaran pajuˈtaːn naɪ suŋˌkoranaˈraːn aˈtasas/ //

\gla Eng bitogaran payutān nay sungkoran-narān atasas. //
\glb eng bitog-aran payutān nay sungkoran-narān atas-as //
\glc \AgtT{} tear.apart-\TplI{} mathematics and science.language brain-\Parg{} //
\gld %
	\mbox{\begin{avm}\[ \Anim{} & – \]\end{avm}}
	\mbox{\begin{avm}\[
		\Anim{} & – \\
		\\
		\Num{}  & \Pl{} \\
	\]\end{avm}}
	\mbox{\begin{avm}\[
		\Anim{} & – \\
		\\
		\Num{}  & \Sg{} \\
	\]\end{avm}}
	{}
	\mbox{\begin{avm}\[
		\Anim{} & + \\
		\Gend{} & \N{} \\
		\Num{}  & \Sg{} \\
	\]\end{avm}}
	//
\glft `Math and linguistics tear the brain apart.' //
\endgl\xe

\pex
\a\begingl
\glpreamble \textit{Manga nimpye ang Misan.} \\
	/ˌmaŋa ˈnimpje aŋ ˈmisan/ //

\gla Manga nimpye ang Misan. //
\glb manga nimp-ye ang Misan //
\glc \Prog{} run-\TsgF{} \Aarg{} Misan //
\glft `Misan is running.' //
\endgl

\a\ljudge*\begingl
\gla Nimpye manga ang Misan. //
\glb Nimp-ye manga ang Misan //
\glc run-\TsgF{} \Prog{} \Aarg{} Misan //
\endgl
\xe

\ex\begingl
\glpreamble \textit{Ang manga sahaya Tikim rangya.} \\
	/aŋ ˌmaŋa saˈhaja ˈtikim ˈraŋja/ //

\gla Ang manga sahaya Tikim rangya. //
\glb ang manga saha-ya Tikim rang-ya //
\glc \AgtT{} \Prog{} come-\TsgM{} Tikim home-\Loc{} //
\glft `Tikim is coming home.' //
\endgl\xe

\pex
\a\begingl
\glpreamble \textit{Ang manga sahaya rangya nay (manga) nedraya Tikim.} \\
	/aŋ ˌmaŋa saˈhaja ˈraŋja naɪ (ˌmaŋa) neˈdraja ˈtikim / //

\gla Ang manga sahaya rangya nay (manga) nedraya Tikim. //
\glb ang manga saha-ya rang-ya nay (manga) nedra-ya Tikim //
\glc \AgtT{} \Prog{} come-\TsgM{} home-\Loc{} and (\Prog{}) sit-\TsgM{} Tikim //
\glft `Tikim is coming home and sitting down.' //
\endgl

\a\ljudge?\begingl
\glpreamble \textit{Ang manga sahaya nay nedraya Tikim rangya.} \\
	/aŋ ˌmaŋa saˈhaja naɪ neˈdraja ˈtikim ˈraŋja/ //

\gla Ang manga sahaya nay nedraya Tikim rangya. //
\glb ang manga saha-ya nay nedra-ya Tikim rang-ya //
\glc \AgtT{} \Prog{} come-\TsgM{} and sit-\TsgM{} Tikim home-\Loc{} //
\glft `\judge?Tikim is coming and sitting at home.' //
\endgl
\xe

\ex\begingl
\glpreamble \textit{Ang manga sahaya Tikim rangya, nay nedrayāng.} \\
	/aŋ ˌmaŋa saˈhaja ˈtikim ˈraŋja |naɪ nedraˈjaːŋ/ //

\gla Ang manga sahaya Tikim rangya, nay nedrayāng. //
\glb ang manga saha-ya Tikim rang-ya nay nedra-yāng //
\glc \AgtT{} \Prog{} come-\TsgM{} Tikim home-\Loc{} and sit-\TsgM{}.\Aarg{} //
\glft `Tikim is coming home and sits down.' //
\endgl\xe

\ex\begingl
\glpreamble \textit{Sa ming layaye ang Misan koya.} \\
	/sa ˌmiŋ laˈjaje aŋ ˈmisan ˈkoja/ //

\gla Sa ming layaye ang Misan koya. //
\glb Sa ming laya-ye ang Misan koya //
\glc \PatT{} can read-\TsgF{} \Aarg{} Misan book //
\glft `The book, Misan can read (it).' //
\endgl\xe

\pex
\a\begingl
\glpreamble \textit{Ming malyya ang Tikim.} \\
	/miŋ ˈmalja aŋ ˈtikim/ //

\gla Ming malyya ang Tikim. //
\glb ming maly-ya ang Tikim //
\glc can sing-\TsgM{} \Aarg{} Tikim //
\glft `Tikim can sing.' //
\endgl

\a\ljudge*\begingl
\gla Malyya ming ang Tikim. //
\glb maly-ya ming ang Tikim //
\glc sing-\TsgM{} can \Aarg{} Tikim //
\endgl
\xe

\pex
\a\begingl
\glpreamble \textit{Ming malyya veno ang Tikim.} \\
	/miŋ ˈmalja ˈveno aŋ ˈtikim/ //

\gla Ming malyya veno ang Tikim. //
\glb ming maly-ya veno ang Tikim //
\glc can sing-\TsgM{} beautifully \Aarg{} Tikim //
\glft `Tikim can sing beautifully.' //
\endgl

\a\ljudge*\begingl
\gla Ming veno malyya ang Tikim. //
\glb ming veno maly-ya ang Tikim //
\glc can beautifully sing-\TsgM{} \Aarg{} Tikim //
\endgl
\xe

\ex\begingl
\glpreamble \textit{Ming malyya nay tuyaya ang Tikim.} \\
	/miŋ ˈmalja naɪ tuˈjaja aŋ ˈtikim / //

\gla Ming malyya nay tuyaya ang Tikim. //
\glb ming maly-ya nay tuya-ya ang Tikim //
\glc can sing-\TsgM{} and dance-\TsgM{} \Aarg{} Tikim //
\glft `Tikim can sing and dance.' //
\endgl\xe

\ex\begingl
\glpreamble \textit{Ming malyya ang Tikim, naynay tuyayāng.} \\
	/miŋ ˈmalja aŋ ˈtikim |naɪˈnaɪ tujaˈjaːŋ/ //

\gla Ming malyya ang Tikim, naynay tuyayāng. //
\glb ming maly-ya ang Tikim naynay tuya-yāng //
\glc can sing-\TsgM{} \Aarg{} Tikim and.also dance-\TsgM{}.\Aarg{} //
\glft `Tikim can sing and also dances.' //
\endgl\xe

\ex\begingl
\glpreamble \textit{Mya manga nimpongye Sipra edauyi.} \\
	/mja ˌmaŋa nimˈpoŋje ˈsipra eˈdaui/ //

\gla Mya manga nimpongye Sipra edauyi. //
\glb mya manga nimp-ong-ye Sipra edauyi //
\glc be.supposed.to \Prog{} run-\Irr{}-\TsgF{} Sipra now //
\glft `Sipra ought to be running now.' //
\endgl\xe

\pex
\a\begingl
\glpreamble \textit{Le mya ming sidegongya badanang ajam.} \\
	/le mja miŋ sideˈgoŋja badaˈnaŋ ˈadʒam/ //

\gla Le mya ming sidegongya badanang ajam. //
\glb le mya ming sideg-ong-ya badan-ang ajam //
\glc \PatTI{} be.supposed.to can repair-\Irr{}-\TsgM{} father-\Aarg{} toy //
\glft `The toy, father should be able to repair (it).' //
\endgl

\a\ljudge*\begingl
\glpreamble \textit{Le myongya mingyam sidejam badanang ajam.} \\
	/le ˈmjongja ˈmiŋjam siˈdedʒam badaˈnaŋ ˈadʒam/ //

\gla Le myongya mingyam sidejam badanang ajam. //
\glb le mya-ong-ya ming-yam sideg-yam badan-ang ajam //
\glc \PatTI{} be.supposed.to-\Irr{}-\TsgM{} can-\Ptcp{} repair-\Ptcp{} father-\Aarg{} toy //
\endgl\xe

\ex\begingl
\glpreamble \textit{Ang mya ming sidegongya nay la-lataya adaley.} \\
	/aŋ mja miŋ sideˈgoŋja naɪ lalaˈtaja adaˈleɪ/ //

\gla Ang mya ming sidegongya nay la-lataya adaley //
\glb ang mya ming sideg-ong-ya nay la\til{}lata-ya ada-ley //
\glc \AgtT{} be.supposed.to can repair-\Irr{}-\TsgM{} and \Iter{}\til{}sell-\TsgM{} that-\PargI{} //
\glft `He should be able to repair and resell it.' //
\endgl\xe

\pex
\a\begingl
\glpreamble \textit{Ang ming sideja adaley, nay da-myongyāng.} \\
	/aŋ miŋ siˈdedʒa adaˈleɪ |naɪ da-mjoŋˈjaːŋ/ //

\gla Ang ming sideja adaley, nay da-myongyāng. //
\glb ang ming sideg-ya ada-ley nay da-mya-ong-yāng //
\glc \AgtT{} can repair-\TsgM{} that-\PargI{} and so-be.supposed.to-\Irr{}-\TsgM{}.\Aarg{} //
\glft `He can, and should, repair it.' //
\endgl

\a\ljudge*\begingl
\gla Ang ming nay mya sidegongya adaley. //
\glb ang ming nay mya sideg-ong-ya adaley //
\glc \AgtT{} can and be.supposed.to reapir-\Irr{}-\TsgM{} that-\PargI{} //
\endgl
\xe

\ex\begingl
\glpreamble \textit{Ang da-pinyaya Yan sa Pila.} \\
	/aŋ dapinˈjaja ˈjan sa ˈpila/ //

	\gla Ang da-pinyaya Yan sa Pila. //
	\glb Ang da-pinya-ya Yan sa Pila //
	\glc \AgtT{} such-ask-\TsgM{} Yan \Parg{} Pila //
	\glft `Yan asks Pila to (do so).' //
\endgl\xe

\ex\begingl
\glpreamble \textit{Ang da-pinyaya nay hisaya Yan sa Pila.} \\
	/aŋ dapinˈjaja naɪ hiˈsaja ˈjan sa ˈpila/ //

	\gla Ang da-pinyaya nay hisaya Yan sa Pila. //
	\glb Ang da-pinya-ya nay hisa-ya Yan sa Pila //
	\glc \AgtT{} such-ask-\TsgM{} and beg-\TsgM{} Yan \Parg{} Pila //
	\glft `Yan asks and begs Pila to (do so).' //
\endgl\xe

\ex\begingl
\glpreamble \textit{Ang manga da-pi-pinyaya nay hi-hisaya Yan sa Pila.} \\
	/aŋ ˈmaŋa dapipinˈjaja naɪ hihiˈsaja ˈjan sa ˈpila/ //

	\gla Ang manga da-pi-pinyaya nay hi-hisaya Yan sa Pila. //
	\glb Ang manga da-\Iter{}\til{}pinya-ya nay \Iter{}\til{}hisa-ya Yan sa Pila //
	\glc \AgtT{} \Prog{} such-ask-\TsgM{} and beg-\TsgM{} Yan \Parg{} Pila //
	\glft `Yan keeps asking and begging Pila to (do so).' //
\endgl\xe

\pex
\a\begingl
\glpreamble \textit{Ang ming da-pinyongya yes.} \\
	/aŋ miŋ dapinˈjoŋja jes/ //

	\gla Ang ming da-pinyongya yes. //
	\glb Ang ming da-pinya-ong-ya yes //
	\glc \AgtT{} can such-ask-\Irr{}-\TsgM{} \TsgF{}.\Parg{} //
	\glft `He could ask her to (do so).' //
\endgl

\a\ljudge*\begingl
	\gla Ang da-ming pinyongya yes. //
	\glb Ang da-ming pinya-ong-ya yes //
	\glc \AgtT{} such-can ask-\Irr{}-\TsgM{} \TsgF{}.\Parg{} //
\endgl
\xe

\pex
\a\begingl
\glpreamble \textit{Ang ming manga da-pinyongya yes} \\
	/aŋ miŋ ˌmaŋa dapinˈjoŋja jes/ //

	\gla Ang ming manga da-pinyongya yes. //
	\glb Ang ming manga da-pinya-ong-ya yes //
	\glc \AgtT{} can \Prog{} such-ask-\Irr{}-\TsgM{} \TsgF{}.\Parg{} //
	\glft `He could be asking her to (do so).' //
\endgl

\a\ljudge*\begingl
	\gla Ang ming da-manga pinyongya yes. //
	\glb Ang ming da-manga pinya-ong-ya yes //
	\glc \AgtT{} can such-\Prog{} ask-\Irr{}-\TsgM{} \TsgF{}.\Parg{} //
\endgl
\xe

\ex\begingl
\glpreamble \textit{Sitang-keca ang Yan.}\\
	/sitaŋˈketʃa aŋ ˈjan/ //

\gla Sitang-keca ang Yan. //
\glb sitang-ket-ya ang Yan //
\glc self-wash-\TsgM{} \Aarg{} Yan //
\glft `Yan washes himself.' //
\endgl\xe

\ex\begingl
\glpreamble \textit{Sitang-kecan ang Yan nay netuang yana.}\\
	/sitaŋˈketʃan aŋ ˈjan naɪ netuˈaŋ ˈjana/ //

\gla Sitang-kecan ang Yan nay netuang yana. //
\glb sitang-ket-yan ang Yan nay netu-ang yana //
\glc self-wash-\TplM{} \Aarg{} Yan and brother-\Aarg{} \TsgM{}.\Gen{} //
\glft `Yan and his brother wash themselves.' //
\endgl\xe

\ex\begingl
\glpreamble \textit{Sitang-keca nay (sitang-)dayungisaya ang Yan.}\\
	/sitaŋˈketʃa naɪ (sitaŋ)dajuŋiˈsaja aŋ ˈjan/ //

\gla Sitang-keca nay (sitang-)dayungisaya ang Yan. //
\glb sitang-ket-ya nay (sitang-)dayungisa-ya ang Yan //
\glc self-wash-\TsgM{} and (self-)dress-\TsgM{} \Aarg{} Yan //
\glft `Yan washes and dresses himself.' //
\endgl\xe

\ex\begingl
\glpreamble \textit{Ming sitang-keca nay (sitang-)dayungisaya iri ang Yan-Yan.}\\
	/miŋ sitaŋˈketʃa naɪ (sitaŋ)dajuŋiˈsaja ˈiri aŋ ˈjanjan/ //

\gla Ming sitang-keca nay (sitang-)dayungisaya iri ang Yan-Yan. //
\glb ming sitang-ket-ya nay (sitang-)dayungisa-ya iri ang Yan-Yan //
\glc can self-wash-\TsgM{} and (self-)dress-\TsgM{} already \Aarg{} Yan-Yan //
\glft `Yan-Yan can already wash and dress himself.' //
\endgl\xe
\xe

\pex
\a\ljudge*\begingl
\glpreamble \textit{Sitang-keca ming ang Yan.}\\
	/sitaŋˈketʃa miŋ aŋ ˈjan/ //

\gla Sitang-keca ming ang Yan. //
\glb sitang-ket-ya ming ang Yan //
\glc self-wash-\TsgM{} can \Aarg{} Yan //
\glft \textit{Intended:} `Yan can wash himself.' //
\endgl

\a\ljudge*\begingl
\glpreamble \textit{Sitang-ming keca ang Yan.}\\
	/sitaŋ miŋ ˈketʃa aŋ ˈjan/ //

\gla Sitang-ming keca ang Yan. //
\glb sitang-ming ket-ya ang Yan //
\glc self-can wash-\TsgM{} \Aarg{} Yan //
\glft \textit{Intended:} `Yan can wash himself.' //
\endgl

\xe

\pex
\a\begingl
\glpreamble \textit{Manga sitang-keca ang Yan.}\\
	/ˌmaŋa sitaŋˈketʃa aŋ ˈjan/ //

\gla Manga sitang-keca ang Yan. //
\glb manga sitang-ket-ya ang Yan //
\glc \Prog{} self-wash-\TsgM{} \Aarg{} Yan //
\glft `Yan is washing himself.' //
\endgl

\a\ljudge*\begingl
\gla Sitang-keca manga ang Yan. //
\glb sitang-ket-ya manga ang Yan //
\glc self-wash-\TsgM{} \Prog{} \Aarg{} Yan //
\endgl

\a\ljudge*\begingl
\gla Sitang-manga keca ang Yan. //
\glb sitang-manga ket-ya ang Yan //
\glc self-\Prog{} wash-\TsgM{} \Aarg{} Yan //
\endgl

\xe

\pex
\a\begingl
	\glpreamble \textit{Rua apaya-kay kaluy ang Latun.} \\
		/rua aˈpajaˌkaɪ kaˈlʊɪ aŋ ˈlatun/ //

	\gla Rua apaya-kay kaluy ang Latun. //
	\glb Rua apa-ya-kay kaluy ang Latun //
	\glc must laugh-\TsgM{}-a.little silently \Aarg{} Latun //
	\glft `Latun had to silently laugh a little.' //
\endgl

\a\ljudge*\begingl
	\gla Rua-kay apaya kaluy ang Latun. //
	\glb Rua-kay apa-ya kaluy ang Latun //
	\glc must-a.little laugh-\TsgM{} silently \Aarg{} Latun //
\endgl

\a\ljudge!\begingl
	\gla Rua apaya kaluy-kay ang Latun. //
	\glb Rua apa-ya kaluy-kay ang Latun //
	\glc must laugh-\TsgM{} silently-a.little \Aarg{} Latun //
	\glft `Latun had to laugh a little silently.' \\
		\textit{Intended:} `Latun had to silently laugh a little.' //
\endgl

\xe

\ex\begingl
\glpreamble \textit{Lampya nay naraya para ang Latun.} \\
	/ˈlampja naɪ naˈraja ˈpara aŋ ˈlatun/ //

\gla Lampya nay naraya para ang Latun. //
\glb lamp-ya nay nara-ya para ang Latun //
\glc walk-\TsgM{} and talk-\TsgM{} quickly \Aarg{} Latun //
\glft `Latun quickly walked and talked. //
\endgl\xe

\ex\begingl
\glpreamble \textit{Lampya nay naraya-kay para ang Latun.} \\
	/ˈlampja naɪ naˈrajaˌkaɪ ˈpara aŋ ˈlatun/ //

\gla Lampya nay naraya-kay para ang Latun. //
\glb lamp-ya nay nara-ya-kay para ang Latun //
\glc walk-\TsgM{} and talk-\TsgM{}-a.little quickly \Aarg{} Latun //
\glft `Latun quickly walked and talked a little. //
\endgl\xe

\ex\begingl
\glpreamble \textit{Ang kondasaya-ikan adaley.} \\
	/aŋ kondaˈsajaˌikan adaˈleɪ/ //

\gla Ang kondasaya-ikan adaley. //
\glb ang kond-asa-ya-ikan adaley //
\glc \AgtT{} eat-\Hab{}-\TsgM{}-much that-\PargI{} //
\glft `He eats it much.' //
\endgl\xe

\ex\begingl
\glpreamble \textit{Ang kondasaya ikan adaley.}\footnotemark \\
	/aŋ kondaˈsaja ˈikan adaˈleɪ/ //

\gla Ang kondasaya ikan adaley. //
\glb ang kond-asa-ya ikan adaley //
\glc \AgtT{} eat-\Hab{}-\TsgM{} wholly that-\PargI{} //
\glft `He usually eats it wholly/whole.' //
\endgl\xe

\footnotetext{\textit{Ikan} is unique in this regard. Other degree suffixes don't lead double lives as (prosodically) bound and free morphemes.}

\ex\begingl
\glpreamble \textit{Lampya para nay naraya-kay ang Latun.} \\
	/ˈlampja ˈpara naɪ naˈrajaˌkaɪ aŋ ˈlatun/ //

\gla Lampya para nay naraya-kay ang Latun. //
\glb lamp-ya para nay nara-ya-kay ang Latun //
\glc walk-\TsgM{} quickly and talk-\TsgM{}-a.little \Aarg{} Latun //
\glft `Latun walked quickly and talked a little. //
\endgl\xe

\ex\begingl
\glpreamble \textit{Lampya para nay naraya-kay tikimarya ang Latun.} \\
	/ˈlampja ˈpara naɪ naˈrajaˌkaɪ tikiˈmarja aŋ ˈlatun/ //

\gla Lampya para nay naraya-kay tikimarya ang Latun. //
\glb lamp-ya para nay nara-ya-kay incoherent ang Latun //
\glc walk-\TsgM{} quickly and talk-\TsgM{}-a.little incoherently \Aarg{} Latun //
\glft `Latun walked quickly and talked incoherently a little. //
\endgl\xe

\ex\begingl
\glpreamble \textit{Lampya-kay para nay naraya tikimarya ang Latun.} \\
	/ˈlampjaˌkaɪ ˈpara naɪ naˈraja tikiˈmarja aŋ ˈlatun/ //

\gla Lampya-kay para nay naraya tikimarya ang Latun. //
\glb lamp-ya-kay para nay nara-ya incoherent ang Latun //
\glc walk-\TsgM{}-a.little quickly and talk-\TsgM{} incoherently \Aarg{} Latun //
\glft `Latun walked quickly a little and talked incoherently. //
\endgl\xe

\ex\begingl
\glpreamble \textit{Lampya-kay para nay naraya-kay tikimarya ang Latun.} \\
	/ˈlampjaˌkaɪ ˈpara naɪ naˈrajaˌkaɪ tikiˈmarja aŋ ˈlatun/ //

\gla Lampya-kay para nay naraya-kay tikimarya ang Latun. //
\glb lamp-ya para nay nara-ya-kay incoherent ang Latun //
\glc walk-\TsgM{}-a.little quickly and talk-\TsgM{}-a.little incoherently \Aarg{} Latun //
\glft `Latun walked quickly a little and talked incoherently a little. //
\endgl\xe

\ex\begingl
\glpreamble \textit{Toryang-ikan.} \\
	/torˈjaŋˌikan/ //

\gla Toryang-ikan. //
\glb tor-yang-ikan //
\glc sleep-\Fsg{}-a.lot //
\glft `I slept a lot.' //
\endgl\xe

\ex\begingl
\glpreamble \textit{Toryang-ikaneng.} \\
	/torˈjaŋikaˌneŋ/ //

\gla Toryang-ikaneng. //
\glb tor-yang-ikan-eng //
\glc sleep-\Fsg{}-a.lot-rather //
\glft `I slept rather a lot.' //
\endgl\xe

\ex\ljudge*\begingl
\glpreamble \textit{Toryang-ikan nay eng.} \\
	/torˈjaŋˌikan nay eng/ //

\gla Toryang-ikan nay eng //
\glb tor-yang-ikan nay eng //
\glc sleep-\Fsg{}-a.lot and rather //
\glft `I slept rather and a lot.' //
\endgl\xe

\ex\ljudge*\begingl
\glpreamble \textit{Toryang-ikan nay ma.} \\
	/torˈjaŋˌikan nay ma/ //

\gla Toryang-ikan nay ma //
\glb tor-yang-ikan nay ma //
\glc sleep-\Fsg{}-a.lot and enough //
\glft `I slept a lot and enough.' //
\endgl\xe

\ex\begingl
\glpreamble \textit{Surpye bihanyam ang Pada} \\
	/ˈsurpje biˈhanjam aŋ ˈpada/ //

\gla Surpye bihanyam ang Pada //
\glb surp-ye bihan-yam ang Pada //
\glc seem-\TsgF{} understand-\Ptcp{} \Aarg{} Pada //
\glft `Pada seems to understand.' //
\endgl\xe

\pex
\a\begingl
\glpreamble \textit{Surpyeng bihanyam.} \\
	/surˈpjeŋ biˈhanjam/ //

\gla Surpyeng bihanyam //
\glb surp-yeng bihan-yam //
\glc seem-\TsgF{}.\Aarg{} understand-\Ptcp{} //
\glft `She seems to understand.' //
\endgl

\a\ljudge*\begingl
\gla Surpye bihanyeng //
\glb surp-ye bihan-yeng //
\glc seem-\TsgF{} understand-\TsgF{}.\Aarg{} //
\endgl

\a\ljudge*\begingl
\gla Surpa bihanyeng //
\glb surpa bihan-yeng //
\glc seem understand-\TsgF{}.\Aarg{} //
\endgl

\a\ljudge*\begingl
\gla Surpyam bihanyeng //
\glb surp-yam bihan-yeng //
\glc seem-\Ptcp{} understand-\TsgF{}.\Aarg{} //
\endgl
\xe

\pex
\a\begingl
\glpreamble \textit{Ang surpye bihanyam Pada narānas yana.} \\
	/aŋ ˈsurpje biˈhanjam ˈpada naˈraːnas ˈjana/ //

\gla Ang surpye bihanyam Pada narānas yana //
\glb ang surp-ye bihan-yam Pada narān-as yana //
\glc \AgtT{} seem-\TsgF{} understand-\Ptcp{} Pada language-\Parg{} \TsgM{}.\Gen{} //
\glft `Pada seems to understand his language.' //
\endgl

\a\ljudge*\begingl
\gla Surpye ang bihanyam Pada narānas yana //
\glb surp-ye ang bihan-yam Pada narān-as yana //
\glc seem-\TsgF{} \AgtT{} understand-\Ptcp{} Pada language-\Parg{} \TsgM{}.\Gen{} //
\endgl

\a\ljudge*\begingl
\gla Ang surpye Pada bihanyam narānas yana //
\glb ang surp-ye Pada bihan-yam narān-as yana //
\glc \AgtT{} seem-\TsgF{} Pada understand-\Ptcp{} language-\Parg{} \TsgM{}.\Gen{} //
\endgl
\xe

\pex
\a\begingl
\glpreamble \textit{Sa surpye bihanyam ang Pada narān yana.} \\
	/sa ˈsurpje biˈhanjam ˈpada naˈraːnas ˈjana/ //

\gla Sa surpye bihanyam ang Pada narān yana //
\glb sa surp-ye bihan-yam ang Pada narān yana //
\glc \PatT{} seem-\TsgF{} understand-\Ptcp{} \Aarg{} Pada language \TsgM{}.\Gen{} //
\glft `His language, Pada seems to understand (it).' //
\endgl

\a\ljudge*\begingl
\gla Sa surpye narān yana bihanyam ang Pada //
\glb sa surp-ye narān yana bihan-yam ang Pada //
\glc \PatT{} seem-\TsgF{} language \TsgM{}.\Gen{} understand-\Ptcp{} \Aarg{} Pada //
\endgl

\a\ljudge*\begingl
\gla Surpye sa bihanyam ang Pada narān yana //
\glb surp-ye sa bihan-yam ang Pada narān yana //
\glc seem-\TsgF{} \PatT{} understand-\Ptcp{} \Aarg{} Pada language \TsgM{}.\Gen{} //
\endgl
\xe

\pex
\a\begingl
\glpreamble \textit{Surpya sitang-birenyayam ang Niyas} \\
	/ˈsurpja siˌtaŋbirenˈjajam aŋ ˈnijas/ //

\gla Surpya sitang-birenyayam ang Niyas. //
\glb surp-ya sitang-birenya-yam ang Niyas. //
\glc seem-\TsgM{} self-doubt-\Ptcp{} \Aarg{} Niyas //
\glft `Niyas seems to doubt himself.' //
\endgl

\a\ljudge*\begingl
\gla Sitang-surpya birenyayam ang Niyas. //
\glb sitang-surp-ya birenya-yam ang Niyas. //
\glc self-seem-\TsgM{} doubt-\Ptcp{} \Aarg{} Niyas //
\endgl
\xe

\ex\begingl
\glpreamble \textit{Ang surpyan soyang mekacan-nama narayam sa Turayi?} \\
	/aŋ ˈsurpjan soˈjaŋ meˈkacaˌnama naˈrajam sa tuˈraji/ //

\gla Ang surpyan soyang mekacan-nama narayam sa Turayi? //
\glb ang surp-yan soyang mekat-yan-nama nara-yam sa Turayi //
\glc \AgtT{} seem-\TplM{} or pretend-\TplM{}-only speak-\Ptcp{} \Parg{} Turayi //
\glft `Do they seem to or only pretend to speak Turayi?' //
\endgl\xe

\pex
\a\begingl
\glpreamble \textit{Linkaye lingyam ang Banvā.} \\
	/liŋˈkaje ˈliŋjam aŋ banˈvaː/ //

\gla Linkaye lingyam ang Banvā. //
\glb linka-ye ling-yam ang Banvā //
\glc try-\TsgF{} climb.up-\Ptcp{} \Aarg{} Banvā //
\glft `Banvā tries to climb up (on it).' //
\endgl

\a\ljudge*\begingl
\gla Linkaye ang Banvā lingyam. //
\glb linka-ye ang Banvā ling-yam //
\glc try-\TsgF{} \Aarg{} Banvā climb.up-\Ptcp{} //
\endgl

\xe

\makefootnotehacks{A}
\pex
\a\begingl
\glpreamble \textit{Linkayeng lingyam.} \\
	/liŋkaˈjeŋ ˈliŋjam/ //

\gla Linkayeng lingyam. //
\glb linka-yeng ling-yam //
\glc try-\TsgF{}.\Aarg{} climb.up-\Ptcp{} //
\glft `She tries to climb up (on it).' //
\endgl

\a\ljudge?\begingl
\gla Linka {lingyeng.\fnhackA} //
\glb linka ling-yeng //
\glc try climb.up-\TsgF{}.\Aarg{} //
\endgl

\a\ljudge*\begingl
\gla Linkaye lingyeng. //
\glb linka-ye ling-yeng //
\glc try-\TsgF{} climb.up-\TsgF{}.\Aarg{} //
\endgl

\a\ljudge*\begingl
\gla Linkayam lingyeng //
\glb linka-yam ling-yeng //
\glc try-\Ptcp{} climb.up-\TsgF{}.\Aarg{} //
\endgl

\xe

\footnotehacktext{\emph{Linka-} `try' is a reasonably common verb, so there is a certain likeliness for it to go the way of \emph{div-} `stay' and become a modal particle.}

\pex
\a\begingl
\glpreamble \textit{Linkaye ang Banvā lingyam merengley.} \\
	/liŋˈkaje aŋ banˈvaː ˈliŋjam mereŋˈleɪ/ //

\gla Linkaye ang Banvā lingyam merengley. //
\glb linka-ye ang Banvā ling-yam mereng-ley //
\glc try-\TsgF{} \Aarg{} Banvā climb.up-\Ptcp{} wall-\PargI{} //
\endgl

\a\begingl
\gla Ang linkaye lingyam Banvā merengley. //
\glb ang linka-ye ling-yam Banvā mereng-ley //
\glc \AgtT{} try-\TsgF{} climb.up-\Ptcp{} Banvā wall-\PargI{} //
\glft `Banvā tries to climb up a wall.' //
\endgl

\a\ljudge*\begingl
\gla Ang linkaye Banvā lingyam merengley. //
\glb ang linka-ye Banvā ling-yam mereng-ley //
\glc \AgtT{} try-\TsgF{} Banvā climb.up-\Ptcp{} wall-\PargI{} //
\endgl

\a\ljudge*\begingl
\gla Linkaye ang lingyam Banvā merengley. //
\glb linka-ye ang ling-yam Banvā mereng-ley //
\glc try-\TsgF{} \AgtT{} climb.up-\Ptcp{} Banvā wall-\PargI{} //
\endgl
\xe

\pex
\a\begingl
\glpreamble \textit{Le linkaye lingyam ang Banvā mereng.} \\
	/le liŋˈkaje ˈliŋjam aŋ banˈvaː meˈreŋ/ //

\gla Le linkaye lingyam ang Banvā mereng. //
\glb le linka-ye ling-yam ang Banvā mereng //
\glc \PatTI{} try-\TsgF{} climb.up-\Ptcp{} \Aarg{} Banvā wall //
\glft `The wall, Banvā tries to climb (it) up.' //
\endgl

\a\begingl
\gla Le linkaye lingyam mereng ang Banvā. //
\glb le linka-ye ling-yam mereng ang Banvā //
\glc \PatTI{} try-\TsgF{} climb.up-\Ptcp{} wall \Aarg{} Banvā //
\endgl

\a\label{ex:rightsideptcp}\ljudge*\begingl
\gla Le linkaye ang Banvā lingyam mereng. //
\glb le linka-ye ang Banvā ling-yam mereng //
\glc \PatTI{} try-\TsgF{} \Aarg{} Banvā climb.up-\Ptcp{} wall //
\endgl

\a\ljudge*\begingl
\gla Linkaye ang Banvā le lingyam mereng. //
\glb linka-ye ang Banvā le ling-yam mereng //
\glc try-\TsgF{} \Aarg{} Banvā \PatTI{} climb.up-\Ptcp{} wall //
\endgl

\a\ljudge*\begingl
\gla Linkaye le lingyam ang Banvā mereng. //
\glb linka-ye le ling-yam ang Banvā mereng //
\glc try-\TsgF{} \PatTI{} climb.up-\Ptcp{} \Aarg{} Banvā wall //
\endgl
\xe

\pex
\a\begingl
\glpreamble \textit{Linkaye ang Pitu tapyyam vengānas ya Tipal.} \\
	/liŋˈkaje aŋ ˈpitu ˈtapjam veŋˈaːnas ja ˈtipal/ //

\gla Linkaye ang Pitu tapyyam vengānas ya Tipal. //
\glb linka-ye ang Pitu tapy-yam vengān-as ya Tipal //
\glc try-\TsgF{} \AgtT{} Pitu put-\Ptcp{} kiss-\Parg{} \Loc{} Tipal //
\glft `Pitu tries to give Tipal a kiss.' (literally: `P. tries to put a kiss on T.') //
\endgl

\a\begingl
\gla Ang linkaye tapyyam Pitu vengānas ya Tipal. //
\glb ang linka-ye tapy-yam Pitu vengān-as ya Tipal //
\glc \AgtT{} try-\TsgF{} put-\Ptcp{} Pitu kiss-\Parg{} \Loc{} Tipal //
\endgl
\xe

\pex
\a\begingl
\glpreamble \textit{Linkayeng tapyyam vengānas ya Tipal.} \\
	/liŋkaˈjeŋ ˈtapjam veŋˈaːnas ja ˈtipal/ //

\gla Linkayeng tapyyam vengānas ya Tipal. //
\glb linka-yeng tapy-yam vengān-as ya Tipal //
\glc try-\TsgF{}.\Aarg{} put-\Ptcp{} kiss-\Parg{} \Loc{} Tipal //
\glft `She tries to give Tipal a kiss.' //
\endgl

\a\ljudge?\begingl
\gla Ang linkaye tapyyam vengānas ya Tipal. //
\glb ang linka-ye tapy-yam vengān-as ya Tipal //
\glc \AgtT{} try-\TsgF{} put-\Ptcp{} kiss-\Parg{} \Loc{} Tipal //
\endgl
\xe

\pex
\a\begingl
\glpreamble \textit{Sa linkaye tapyyam ang Pitu vengān ya Tipal.} \\
	/sa liŋˈkaje ˈtapjam aŋ ˈpitu ˈveŋan ja ˈtipal/ //

\gla Sa linkaye tapyyam ang Pitu vengān ya Tipal. //
\glb sa linka-ye tapy-yam ang Pitu vengān ya Tipal //
\glc \PatT{} try-\TsgF{} put-\Ptcp{} \Aarg{} Pitu kiss \Loc{} Tipal //
\glft `A kiss Pitu tries to give Tipal.' //
\endgl

\a\label{ex:pattopofembeddedditrans}\ljudge?\begingl
\gla Sa linkayeng tapyyam vengān ya Tipal. //
\glb sa linka-yeng tapy-yam vengān ya Tipal //
\glc \PatT{} try-\TsgF{}.\Aarg{} put-\Ptcp{} kiss \Loc{} Tipal //
\endgl

\a\begingl
\gla (Da-)linkaye ang Pitu, sa tapyyeng vengān ya Tipal. //
\glb (da-)linka-ye ang Pitu sa tapy-yeng vengān ya Tipal //
\glc (so-)try-\TsgF{} \Aarg{} Pitu \PatT{} put-\TsgF{}.\Aarg{} kiss \Loc{} Tipal //
\endgl
\xe

\pex
\a\begingl
\glpreamble \textit{Linkaye lakukay ang Pitu tapyyam vengānas ya Tipal.}\footnotemark \\
	/liŋˈkaje lakuˈkaɪ aŋ ˈpitu ˈtapjam veˈŋanas ja ˈtipal/ //

\gla Linkaye lakukay ang Pitu tapyyam vengānas ya Tipal. //
\glb linka-ye lakukay ang Pitu tapy-yam vengān-as ya Tipal //
\glc try-\TsgF{} in.vain \Aarg{} Pitu put-\Ptcp{} kiss-\Parg{} \Loc{} Tipal //
\glft `Pitu tries in vain to kiss Tipal.' //
\endgl

\a\begingl
\gla Ang linkaye lakukay tapyyam Pitu vengānas ya Tipal. //
\glb ang linka-ye lakukay tapy-yam Pitu vengān-as ya Tipal //
\glc \AgtT{} try-\TsgF{} in.vain put-\Ptcp{} Pitu kiss-\Parg{} \Loc{} Tipal //
\endgl

\a\begingl
\gla Ang linkaye tapyyam lakukay Pitu vengānas ya Tipal. //
\glb ang linka-ye tapy-yam lakukay Pitu vengān-as ya Tipal //
\glc \AgtT{} try-\TsgF{} put-\Ptcp{} in.vain Pitu kiss-\Parg{} \Loc{} Tipal //
\endgl
\xe

\footnotetext{Both (b) and (c) sound slightly awkward to me for some reason, although they should be permissible.}

\pex
\a\begingl
\glpreamble \textit{Linkaya ang Ajān da-telbisayam ledo yayam.} \\
	/liŋˈkaja aŋ‿aˈdʒaːn datelbiˈsajam ˈledo ˈjajam/ //

\gla Linkaya ang Ajān da-telbisayam ledo yayam. //
\glb linka-ya ang Ajān da-telbisa-yam ledo yayam //
\glc try-\TsgM{} \Aarg{} Ajān such-point.out-\Ptcp{} friendly \TsgM{}.\Dat{} //
\glft `Ajān tries to point it out to him in a friendly way.' //
\endgl

\a\begingl
\gla Ang linkaya da-telbisayam ledo Ajān yayam. //
\glb ang linka-ya da-telbisa-yam ledo Ajān yayam //
\glc \AgtT{} try-\TsgM{} such-point.out-\Ptcp{} friendly Ajān \TsgM{}.\Dat{} //
\endgl
\xe

\ex\begingl
\glpreamble \textit{Ang linkaya etecam karomaya sitang-yās.} \\
	/aŋ linˈkaja eˈtetʃam karoˈmaja sitaŋˈjaːs/ //

\gla Ang linkaya etecam karomaya sitang-yās. //
\glb ang linka-ya etek-yam karomaya sitang-yās //
\glc \AgtT{} try-\TsgM{} heal-\Ptcp{} doctor self-\TsgM{}.\Parg{} //
\glft `The doctor tries to heal himself.' //
\endgl\xe

\pex
\a\begingl
\glpreamble \textit{Linkaya sitang-etecam karomayāng.} \\
	/linˈkaja siˌtaŋeˈtetʃam karomaˈjaːŋ/ //

\gla Linkaya sitang-etecam karomayāng. //
\glb linka-ya sitang-etek-yam karomaya-ang //
\glc try-\TsgM{} self-heal-\Ptcp{} doctor-\Aarg{} //
\glft `The doctor tries to heal himself.' //
\endgl

\a\ljudge*\begingl
\gla Sitang-linkaya etecam karomayāng. //
\glb sitang-linka-ya etek-yam karomaya-ang //
\glc self-try-\TsgM{} heal-\Ptcp{} doctor-\Aarg{} //
\endgl
\xe

\ex\begingl
\glpreamble \textit{Ang linkaya etecam sitang-yās.} \\
	/aŋ linˈkaja eˈtetʃam sitaŋˈjaːs/ //

\gla Ang linkaya etecam sitang-yās. //
\glb ang linka-ya etek-yam sitang-yās //
\glc \AgtT{} try-\TsgM{} heal-\Ptcp{} self-\TsgM{}.\Parg{} //
\glft `He tries to heal himself.' //
\endgl\xe

\pex
\a\begingl
\glpreamble \textit{Linkayāng sitang-etecam.} \\
	/linkaˈjaːŋ siˌtaŋeˈtetʃam/ //

\gla Linkayāng sitang-etecam. //
\glb linka-yāng sitang-etek-yam //
\glc try-\TsgM{}.\Aarg{} self-heal-\Ptcp{} //
\glft `He tries to heal himself.' //
\endgl

\a\ljudge*\begingl
\gla Sitang-linkayāng etecam. //
\glb sitang-linka-yāng etek-yam //
\glc self-try-\TsgM{}.\Aarg{} heal-\Ptcp{} //
\endgl
\xe

\pex
\a\begingl
\glpreamble \textit{Ang pinyaya Tipal sa Anang, nedrayeng.} \\
	/aŋ pinˈjaja ˈtipal sa‿aˈnaŋ | nedraˈjeŋ/ //

\gla Ang pinyaya Tipal sa Anang, nedrayeng. //
\glb ang pinya-ya Tipal sa Anang, nedrayeng //
\glc \AgtT{} ask-\TsgM{} Tipal \Parg{} Anang sit-\TsgM{}.\Aarg{} //
\glft `Tipal asks Anang to sit down.' //
\endgl

\a\begingl
\gla Ang pinyaya nedrayam Tipal sa Anang. //
\glb ang pinya-ya nedra-yam Tipal sa Anang //
\glc \AgtT{} ask-\TsgM{} sit-\Ptcp{} Tipal \Parg{} Anang //
\endgl

\a\begingl
\gla (Da-)pinyaya ang Tipal, nedraye ang Anang. //
\glb (da-)pinya-ya ang Tipal nedra-ye ang Anang //
\glc (so-)ask-\TsgM{} \Aarg{} Tipal sit-\TsgF{} \Aarg{} Anang //
\endgl

\a\ljudge?\begingl
\gla Ang pinyaya Tipal sa Anang nedrayam. //
\glb ang pinya-ya Tipal sa Anang nedra-yam //
\glc \AgtT{} ask-\TsgM{} Tipal \Parg{} Anang sit-\Ptcp{} //
\endgl

\a\ljudge*\begingl
\gla Pinyaya ang Tipal nedrayam sa Anang. //
\glb Pinya-ya ang Tipal nedra-yam sa Anang //
\glc ask-\TsgM{} \Aarg{} Tipal sit-\Ptcp{} \Parg{} Anang //
\glft \textit{Literally:} `Tipal asks to sit down Anang.' //
\endgl

\xe

\pex
\a\begingl
\glpreamble \textit{Ang pinyaya yes, nedrayeng.} \\
	/aŋ pinˈjaja yes | nedraˈjeŋ/ //

\gla Ang pinyaya yes, nedrayeng. //
\glb ang pinya-ya yes, nedra-yeng //
\glc \AgtT{} ask-\TsgM{} sit-\TsgF{}.\Aarg{} //
\endgl

\a\begingl
\gla (Da-)pinyayāng, nedrayeng. //
\glb (da-)pinya-yāng, nedra-yeng //
\glc (so-)ask-\TsgM{}.\Aarg{} sit-\TsgF{}.\Aarg{} //
\endgl

\a\begingl
\gla Ang pinyaya yes nedrayam. //
\glb Ang pinya-ya yes nedra-yam //
\glc \AgtT{} ask-\TsgM{} \TsgF{}.\Parg{} sit-\Ptcp{} //
\endgl

\a\ljudge?\begingl
\gla Pinyayāng nedrayam yes. //
\glb Pinya-yāng nedra-yam yes //
\glc ask-\TsgM{}.\Aarg{} sit-\Ptcp{} \TsgF{}.\Parg{} //
\endgl
\xe

\pex
\a\begingl
\glpreamble \textit{Ang (da-)pinyaye Misan sa Prano, le rimayāng kunang.} \\
	/aŋ (da)pinˈjaje ˈmisan sa ˈprano | le rimaˈjaːŋ kuˈnaŋ/ //

\gla Ang (da-)pinyaye Misan sa Prano, le rimayāng kunang. //
\glb ang (da-)pinya-ye Misan sa Prano le rima-yāng kunang //
\glc \AgtT{} (so-)ask-\TsgF{} Misan \Parg{} Prano \PatTI{} close-\TsgM{}.\Aarg{} door //
\glft `Misan asks Prano to close the door.' //
\endgl

\a\begingl
\gla Ang pinyaye Misan sa Prano rimayam kunangley. //
\glb ang pinya-ye Misan sa Prano rima-yam kunang //
\glc \AgtT{} ask-\TsgF{} Misan \Parg{} Prano close-\Ptcp{} door //
\endgl

\a\ljudge?\begingl
\gla Ang pinyaye rimayam kunangley Misan sa Prano. //
\glb ang pinya-ye rima-yam kunangley Misan sa Prano //
\glc \AgtT{} ask-\TsgF{} close-\Ptcp{} door-\PargI{} Misan \Parg{} Prano //
\endgl

\a\ljudge*\begingl
\gla Le pinyaye rimayam kunang ang Misan sa Prano. //
\glb le pinya-ye rima-yam kunang ang Misan sa Prano //
\glc \PargI{} ask-\TsgF{} close-\Ptcp{} door \Aarg{} Misan \Parg{} Prano //
\endgl
\xe

\pex
\a\begingl
\glpreamble \textit{Ang (da-)pinyaye yās, le rimayāng kunang.} \\
	/aŋ (da)pinˈjaje ˈjaːs | le rimaˈjaːŋ kuˈnaŋ/ //

\gla Ang (da-)pinyaye yās, le rimayāng kunang. //
\glb ang (da-)pinya-ye yās le rima-yāng kunang //
\glc \AgtT{} (so-)ask-\TsgF{} \TsgM{}.\Parg{} \PatTI{} close-\TsgM{}.\Aarg{} door //
\glft `She asks him to close the door.' //
\endgl

\a\begingl
\gla Ang pinyaye yās rimayam kunangley. //
\glb ang pinya-ye yās rima-yam kunang //
\glc \AgtT{} ask-\TsgF{} \TsgM{}.\Parg{} close-\Ptcp{} door //
\endgl

\a\ljudge??\begingl
\gla Ang pinyaye rimayam kunangley yās. //
\glb ang pinya-ye rima-yam kunang-ley yās //
\glc \AgtT{} ask-\TsgF{} close-\Ptcp{} door-\PargI{} \TsgM{}.\Parg{} //
\endgl

\a\ljudge*\begingl
\gla Le pinyayeng rimayam kunang yās. //
\glb le pinya-yeng rima-yam kunang yās //
\glc \PatTI{} ask-\TsgF{}.\Aarg{} close-\Ptcp{} door \TsgM{}.\Parg{} //
\endgl
\xe

\pex
\a\begingl
\glpreamble \textit{Ang (da-)pinyaya Akan sa Sedan, sa ilyāng koyaye yam Pila.} \\
	/aŋ (da)pinˈjaja ˈakan sa ˈsedan | sa ilˈjaːŋ koˈjaje jam ˈpila/ //

\gla Ang (da-)pinyaya Akan sa Sedan, sa ilyāng koyaye yam Pila. //
\glb ang (da-)pinya-ya Akan sa Sedan sa il-yāng koya-ye yam Pila //
\glc \AgtT{} (so-)ask-\TsgM{} Akan \Parg{} Sedan \PatTI{} give-\TsgM{}.\Aarg{} book-\Pl{} \Dat{} Pila //
\glft `Akan asks Sedan to give the books to Pila.' //
\endgl

\a\begingl
\gla Ang pinyaya Akan sa Sedan ilyam koyaye yam Pila. //
\glb ang pinya-ya Akan sa Sedan il-yam koya-ye yam Pila //
\glc \AgtT{} ask-\TsgM{} Akan \Parg{} Sedan give-\Ptcp{} book-\Pl{} \Dat{} Pila //
\endgl

\a\ljudge??\begingl
\gla Ang pinyaya ilyam koyajas yam Pila Akan sa Sedan. //
\glb ang pinya-ya il-yam koya-ye-as yam Pila Akan sa Sedan //
\glc \AgtT{} ask-\TsgM{} give-\Ptcp{} book-\Pl{}-\Parg{} \Dat{} Pila Akan \Parg{} Sedan //
\endgl

\a\ljudge*\begingl
\gla Sa pinyaya ilyam koyaye yam Pila ang Akan sa Sedan. //
\glb sa pinya-ya il-yam koyaye yam Pila ang Akan sa Sedan //
\glc \PatT{} ask-\TsgM{} give-\Ptcp{} book-\Pl{} \Dat{} Pila \Aarg{} Akan \Parg{} Sedan //
\endgl
\xe

\pex
\a\begingl
\glpreamble \textit{Ang (da-)pinyaya yās, sa ilyāng koyaye yeyam.} \\
	/aŋ (da)pinˈjaja ˈjaːs | sa ilˈjaːŋ koˈjaje ˈjejam/ //

\gla Ang (da-)pinyaya yās, sa ilyāng koyaye yeyam. //
\glb ang (da-)pinya-ya yās sa il-yāng koya-ye yeyam //
\glc \AgtT{} (so-)ask-\TsgM{} \TsgM{}.\Parg{} \PatTI{} give-\TsgM{}.\Aarg{} book-\Pl{} \TsgF{}.\Dat{} //
\glft `He asks him to give the books to her.' //
\endgl

\a\begingl
\gla Ang pinyaya yās ilyam koyaye yeyam. //
\glb ang pinya-ya yās il-yam koya-ye yeyam //
\glc \AgtT{} ask-\TsgM{} \TsgM{}.\Parg{} give-\Ptcp{} book-\Pl{} \TsgF{}.\Dat{} //
\endgl

\a\ljudge*\begingl
\gla Ang pinyaya ilyam koyajas yeyam yās. //
\glb ang pinya-ya il-yam koya-ye-as yeyam yās //
\glc \AgtT{} ask-\TsgM{} give-\Ptcp{} book-\Pl{}-\Parg{} \TsgF{}.\Dat{} \TsgM{}.\Parg{} //
\endgl

\a\ljudge*\begingl
\gla Sa pinyaya ilyam koyaye yeyam yās. //
\glb sa pinya-ya il-yam koyaye yeyam yās //
\glc \PatT{} ask-\TsgM{} give-\Ptcp{} book-\Pl{} \TsgF{}.\Dat{} \TsgM{}.\Parg{} //
\endgl
\xe

\ex\begingl
\glpreamble \textit{Ang pinyaye Misan sa Prano rimayam para kunangley.} \\
	/aŋ pinˈjaje ˈmisan sa ˈprano riˈmajam ˈpara kunaŋˈleɪ/ //

\gla Ang pinyaye Misan sa Prano rimayam para kunangley. //
\glb ang pinya-ye Misan sa Prano rima-yam para kunang //
\glc \AgtT{} ask-\TsgF{} Misan \Parg{} Prano close-\Ptcp{} quickly door //
\glft `Misan asks Prano to quickly close the door.' //
\endgl\xe

\pex
\a\begingl
\glpreamble \textit{Ang pinyaye Misan sa Prano ranicam sitang-yas.} \\
	/aŋ pinˈjaje ˈmisan sa ˈprano raˈnitʃam sitaŋˈjas/ //

\gla Ang pinyaye Misan sa Prano ranicam sitang-yas. //
\glb ang pinya-ye Misan sa Prano ranit-yam sitang-yas //
\glc \AgtT{} ask-\TsgF{} Misan \Parg{} Prano hide-\Ptcp{} self-\TsgM{}.\Parg{} //
\glft `Misan asks Prano to hide himself.' //
\endgl

\a\begingl
\gla Ang pinyaye Misan sa Prano sitang-ranicam. //
\glb ang pinya-ye Misan sa Prano sitang-ranit-yam //
\glc \AgtT{} ask-\TsgF{} Misan \Parg{} Prano self-hide-\Ptcp{} //
\endgl

\a\begingl
\gla Ang pinyaye sitang-ranicam Misan sa Prano. //
\glb ang pinya-ye sitang-ranit-yam Misan sa Prano //
\glc \AgtT{} ask-\TsgF{} self-hide-\Ptcp{} Misan \Parg{} Prano //
\endgl
\xe

\pex
\a\begingl
\glpreamble \textit{Galamye ang Caysu, apaya ang Niyas.} \\
	/gaˈlamje aŋ ˈtʃaɪsu | aˈpaja aŋ ˈnijas/ //

\gla Galamye ang Caysu, apaya ang Niyas. //
\glb galam-ye ang Caysu apa-ya ang Niyas //
\glc expect-\TsgF{} \Aarg{} Caysu laugh-\TsgM{} \Aarg{} Niyas //
\glft `Caysu expects Niyas to laugh.' //
\endgl

\a\ljudge*\begingl
\gla Galamye ang Caysu apayam ang/sa/yam/na/ri Niyas. //
\glb galam-ye ang Caysu apa-yam ang/sa/yam/na/ri Niyas //
\glc expect-\TsgF{} \Aarg{} Caysu laugh-\Ptcp{} \Aarg{}/\Parg{}/\Dat{}/\Gen{}/\Ins{} Niyas //
\endgl

\a\ljudge*\begingl
\gla Galamye ang Caysu ang/sa/yam/na/ri Niyas apayam. //
\glb galam-ye ang Caysu ang/sa/yam/na/ri Niyas apa-yam //
\glc expect-\TsgF{} \Aarg{} Caysu \Aarg{}/\Parg{}/\Dat{}/\Gen{}/\Ins{} Niyas laugh-\Ptcp{} //
\endgl

\a\ljudge*\begingl
\gla Galamye apayam ang Caysu ang/sa/yam/na/ri Niyas. //
\glb galam-ye apa-yam ang Caysu ang/sa/yam/na/ri Niyas //
\glc expect-\TsgF{} laugh-\Ptcp{} \Aarg{} Caysu \Aarg{}/\Parg{}/\Dat{}/\Gen{}/\Ins{} Niyas //
\endgl

\a\ljudge*\begingl
\gla Galamye apayam ang/sa/yam/na/ri Niyas ang Caysu. //
\glb galam-ye apa-yam ang/sa/yam/na/ri Niyas ang Caysu //
\glc expect-\TsgF{} laugh-\Ptcp{} \Aarg{}/\Parg{}/\Dat{}/\Gen{}/\Ins{} Niyas \Aarg{} Caysu //
\endgl

\a\ljudge*\begingl
\gla Galamye apaya ang Niyas ang Caysu. //
\glb galam-ye apa-ya ang Niyas ang Caysu //
\glc expect-\TsgF{} laugh-\TsgM{} \Aarg{} Niyas \Aarg{} Caysu //
\endgl

\xe

\pex
\a\begingl
\glpreamble \textit{Galamye ang Caysu, apayāng.} \\
	/gaˈlamje aŋ ˈtʃaɪsu | aˈpajaːŋ/ //

\gla Galamye ang Caysu, apayāng. //
\glb galam-ye ang Caysu apa-yāng //
\glc expect-\TsgF{} \Aarg{} Caysu laugh-\TsgM{}.\Aarg{} //
\glft `Caysu expects him to laugh.' //
\endgl

\a\ljudge*\begingl
\gla Galamye ang Caysu apayam yāng/yās/yayam/yana/yari. //
\glb galam-ye ang Caysu apa-yam yāng/yās/yayam/yana/yari //
\glc expect-\TsgF{} \Aarg{} Caysu laugh-\Ptcp{} \TsgM{}.\Aarg{}/\Parg{}/\Dat{}/\Gen{}/\Ins{} //
\endgl

\a\ljudge*\begingl
\gla Galamye ang Caysu yāng/yās/yayam/yana/yari apayam. //
\glb galam-ye ang Caysu yāng/yās/yayam/yana/yari apa-yam //
\glc expect-\TsgF{} \Aarg{} Caysu \TsgM{}.\Aarg{}/\Parg{}/\Dat{}/\Gen{}/\Ins{} laugh-\Ptcp{} //
\endgl

\a\ljudge*\begingl
\gla Galamye apayam ang Caysu yāng/yās/yayam/yana/yari. //
\glb galam-ye apa-yam ang Caysu yāng/yās/yayam/yana/yari //
\glc expect-\TsgF{} laugh-\Ptcp{} \Aarg{} Caysu \TsgM{}.\Aarg{}/\Parg{}/\Dat{}/\Gen{}/\Ins{} //
\endgl

\a\ljudge*\begingl
\gla Galamye apayam yāng/yās/yayam/yana/yari ang Caysu. //
\glb galam-ye apa-yam yāng/yās/yayam/yana/yari ang Caysu //
\glc expect-\TsgF{} laugh-\Ptcp{} \TsgM{}.\Aarg{}/\Parg{}/\Dat{}/\Gen{}/\Ins{} \Aarg{} Caysu //
\endgl

\a\ljudge*\begingl
\gla Galamye apayāng ang Caysu. //
\glb galam-ye apa-yāng ang Caysu //
\glc expect-\TsgF{} laugh-\TsgM{}.\Aarg{} \Aarg{} Caysu //
\endgl

\xe

\pex
\a\begingl
\glpreamble \textit{Galamye ang Diya, ang koronya Mican guratanley.} \\
	/gaˈlamje aŋ ˈdija | aŋ koˈronja ˈmitʃan guratanˈleɪ/ //

\gla Galamye ang Diya, ang koronya Mican guratanley. //
\glb galam-ye ang Diya ang koron-ya Mican guratan-ley //
\glc expect-\TsgF{} \Aarg{} Diya \AgtT{} know-\TsgM{} Mican answer-\PargI{} //
\glft `Diya expected Mican to know the answer.' //
\endgl

\a\ljudge*\begingl
\gla Galamye ang Diya koronyam ang/sa/yam/na/ri Mican guratanley. //
\glb galam-ye ang Diya koron-yam ang/sa/yam/na/ri Mican guratan-ley //
\glc expect-\TsgF{} \Aarg{} Diya know-\Ptcp{} \Aarg{}/\Parg{}/\Dat{}/\Gen{}/\Ins{} Mican answer-\PargI{} //
\endgl

\a\ljudge*\begingl
\gla Galamye ang Diya ang/sa/yam/na/ri Mican koronyam guratanley. //
\glb galam-ye ang Diya ang/sa/yam/na/ri Mican koron-yam guratan-ley //
\glc expect-\TsgF{} \Aarg{} Diya \Aarg{}/\Parg{}/\Dat{}/\Gen{}/\Ins{} Mican know-\Ptcp{} answer-\PargI{} //
\endgl

\a\ljudge*\begingl
\gla Galamye koronyam ang Diya ang/sa/yam/na/ri Mican guratanley. //
\glb galam-ye koron-yam ang Diya ang/sa/yam/na/ri Mican guratan-ley //
\glc expect-\TsgF{} know-\Ptcp{} \Aarg{} Diya \Aarg{}/\Parg{}/\Dat{}/\Gen{}/\Ins{} Mican answer-\PargI{} //
\endgl

\a\ljudge*\begingl
\gla Galamye koronyam ang/sa/yam/na/ri Mican guratanley ang Diya. //
\glb galam-ye koron-yam ang/sa/yam/na/ri Mican guratan-ley ang Diya //
\glc expect-\TsgF{} know-\Ptcp{} \Aarg{}/\Parg{}/\Dat{}/\Gen{}/\Ins{} Mican answer-\PargI{} \Aarg{} Diya //
\endgl

\a\ljudge*\begingl
\gla Galamye koronya ang Mican guratanley ang Diya. //
\glb galam-ye koron-ya ang Mican guratan-ley ang Diya //
\glc expect-\TsgF{} know-\TsgM{} \Aarg{} Mican answer-\PargI{} \Aarg{} Diya //
\endgl

\xe

\pex
\a\begingl
\glpreamble \textit{Galamye ang Diya, ang koronya guratanley.} \\
	/gaˈlamje aŋ ˈdija | aŋ koˈronja guratanˈleɪ/ //

\gla Galamye ang Diya, ang koronya guratanley. //
\glb galam-ye ang Diya ang koron-ya guratan-ley //
\glc expect-\TsgF{} \Aarg{} Diya \AgtT{} know-\TsgM{} answer-\PargI{} //
\glft `Diya expected him to know the answer.' //
\endgl

\a\ljudge*\begingl
\gla Galamye ang Diya koronyam yāng/yās/yayaym/yana/yari guratanley. //
\glb galam-ye ang Diya koron-yam yāng/yās/yayaym/yana/yari guratan-ley //
\glc expect-\TsgF{} \Aarg{} Diya know-\Ptcp{} \TsgM{}.\Aarg{}/\Parg{}/\Dat{}/\Gen{}/\Ins{} Mican answer-\PargI{} //
\endgl

\a\ljudge*\begingl
\gla Galamye ang Diya yāng/yās/yayaym/yana/yari koronyam guratanley. //
\glb galam-ye ang Diya yāng/yās/yayaym/yana/yari koron-yam guratan-ley //
\glc expect-\TsgF{} \Aarg{} Diya \TsgM{}.\Aarg{}/\Parg{}/\Dat{}/\Gen{}/\Ins{} know-\Ptcp{} answer-\PargI{} //
\endgl

\a\ljudge*\begingl
\gla Galamye koronyam ang Diya yāng/yās/yayaym/yana/yari guratanley. //
\glb galam-ye koron-yam ang Diya yāng/yās/yayaym/yana/yari guratan-ley //
\glc expect-\TsgF{} know-\Ptcp{} \Aarg{} Diya \TsgM{}.\Aarg{}/\Parg{}/\Dat{}/\Gen{}/\Ins{} answer-\PargI{} //
\endgl

\a\ljudge*\begingl
\gla Galamye koronyam yāng/yās/yayaym/yana/yari guratanley ang Diya. //
\glb galam-ye koron-yam yāng/yās/yayaym/yana/yari guratan-ley ang Diya //
\glc expect-\TsgF{} know-\Ptcp{} \TsgM{}.\Aarg{}/\Parg{}/\Dat{}/\Gen{}/\Ins{} answer-\PargI{} \Aarg{} Diya //
\endgl

\a\ljudge*\begingl
\gla Galamye koronyāng guratanley ang Diya. //
\glb galam-ye koron-yāng guratan-ley ang Diya //
\glc expect-\TsgF{} know-\TsgM{}.\Aarg{} answer-\PargI{} \Aarg{} Diya //
\endgl

\xe

\ex\begingl
\glpreamble \textit{Ang koronya Kaman apyanas palay-eng.} \\
	/aŋ koˈronja ˈkaman aˈpjanas palaɪˈeŋ/ //

\gla Ang koronya Kaman apyanas palay-eng. //
\glb ang koron-ya Kaman apyan-as palay-eng //
\glc \AgtT{} know-\TsgM{} Kaman joke-\Parg{} funny-\Comp{} //
\glft `Kaman knows a funnier joke.' //
\endgl\xe

\ex\ljudge*\begingl
\glpreamble \textit{Palay-eng ekeng apyanang.} \\
	/palaɪˈeŋ eˈkeŋ apjaˈnaŋ/ //

\gla Palay-eng ekeng apyanang. //
\glb Palay-eng ekeng apyan-ang. //
\glc funny-\Comp{} too joke-\Aarg{} //
\glft `The joke is too funnier.' //
\endgl\xe

\ex\begingl
\glpreamble \textit{Ang ningya apyanas palay-engkay.} \\
	/aŋ ˈniŋja aˈpjanas palaɪeŋˈkaɪ/ //

\gla Ang ningya apyanas palay-engkay. //
\glb ang ning-ya apyan-as palay-eng-kay. //
\glc \AgtT{} tell-\TsgM{} joke-\Parg{} funny-\Comp{}-a.little //
\glft `He tells a slightly funnier joke.' //
\endgl\xe

\makefootnotehacks{B}
\pex
\a\begingl
\glpreamble \textit{Ang koronya Kaman apyanas nay ninganas palay-eng.} \\
	/aŋ koˈronja aˈpjanas naɪ niŋˈanas palaɪˈeŋ/ //

\gla Ang koronya Kaman apyanas nay ninganas palay-eng. //
\glb ang koron-ya Kaman apyan-as nay ningan-as palay-eng //
\glc \AgtT{} know-\TsgM{} Kaman joke-\Parg{} and story-\Parg{} funny-\Comp{} //
\glft `Kaman knows funnier a joke and story.' //
\endgl

\a\ljudge?\begingl
\gla Ang koronya Kaman apyanas palay-eng nay ninganas {palay-eng.\fnhackB} //
\glb ang koron-ya Kaman apyan-as palay-eng nay ningan-as palay-eng //
\glc \AgtT{} know-\TsgM{} Kaman joke-\Parg{} funny-\Comp{} and story-\Parg{} funny-\Comp{} //
\endgl
\xe

\footnotehacktext{This is not wrong per se, but merely sounds unnecessarily redundant.}

\pex
\a\begingl
\glpreamble \textit{Ang koronya Kaman apyanas palay nay ban-eng.} \\
	/aŋ koˈronja ˈkaman aˈpjanas palaɪ naɪ banˈeŋ/ //

\gla Ang koronya Kaman apyanas palay nay ban-eng. //
\glb ang koron-ya Kaman apyan-as palay nay ban-eng //
\glc \AgtT{} know-\TsgM{} Kaman joke-\Parg{} funny and good-\Comp{} //
\glft `Kaman knows a funnier and better joke.' //
\endgl

\a\begingl
\gla Ang koronya Kaman apyanas palay-eng nay ban-eng. //
\glb ang koron-ya Kaman apyan-as palay-eng nay ban-eng //
\glc \AgtT{} know-\TsgM{} Kaman joke-\Parg{} funny-\Comp{} and good-\Comp{} //
\endgl

\a\begingl
\gla Ang koronya Kaman apyanas palay-eng nay(nay) da-ban-eng. //
\glb ang koron-ya Kaman apyan-as palay-eng nay(nay) da-ban-eng //
\glc \AgtT{} know-\TsgM{} Kaman joke-\Parg{} funny-\Comp{} and(\til{}also) one-good-\Comp{} //
\glft `Kaman knows a funnier joke and (also a) better one.' //
\endgl

\a\ljudge!\begingl
\gla Ang koronya Kaman apyanas palay-eng nay ban. //
\glb ang koron-ya Kaman apyan-as palay-eng nay ban //
\glc \AgtT{} know-\TsgM{} Kaman joke-\Parg{} funny-\Comp{} and good //
\glft `Kaman knows a funnier, and good, joke.' \\
	\textit{Intended:} `Kaman knows a funnier and better joke.' //
\endgl

\xe

\pex
\a\begingl
\glpreamble \textit{Ang koronya Kaman apyanas palay-eng nay ninganas ban-eng.} \\
	/aŋ koˈronja aˈpjanas palaɪˈeŋ naɪ niŋˈanas banˈeŋ/ //

\gla Ang koronya Kaman apyanas palay-eng nay ninganas ban-eng. //
\glb ang koron-ya Kaman apyan-as palay-eng nay ningan-as ban-eng //
\glc \AgtT{} know-\TsgM{} Kaman joke-\Parg{} funny-\Comp{} and story-\Parg{} good-\Comp{} //
\glft `Kaman knows a funnier joke and a better story.' //
\endgl

\a\ljudge!\begingl
\gla Ang koronya Kaman apyanas palay nay ninganas ban-eng. //
\glb ang koron-ya Kaman apyan-as palay nay ningan-as ban-eng //
\glc \AgtT{} know-\TsgM{} Kaman joke-\Parg{} funny and story-\Parg{} good-\Comp{} //
\glft `Kaman knows a funny joke and a better story.' \\
	\textit{Intended:} `Kaman knows a funnier joke and a better story.' //
\endgl
\xe

\pex
\a\begingl
\glpreamble \textit{Ang koronya Kaman apyanas palay-eng nay ninganas ban-vā.} \\
	/aŋ koˈronja aˈpjanas palaɪˈeŋ naɪ niŋˈanas banˈvaː/ //

\gla Ang koronya Kaman apyanas palay-eng nay ninganas ban-vā. //
\glb ang koron-ya Kaman apyan-as palay-eng nay ningan-as ban-vā //
\glc \AgtT{} know-\TsgM{} Kaman joke-\Parg{} funny-\Comp{} and story-\Parg{} good-\Supl{} //
\glft `Kaman knows a funnier joke and the best story.' //
\endgl

\a\ljudge*\begingl
\gla Ang koronya Kaman apyanas palay nay ninganas ban-eng nay -vā. //
\glb ang koron-ya Kaman apyan-as palay nay ningan-as ban-eng nay -vā //
\glc \AgtT{} know-\TsgM{} Kaman joke-\Parg{} funny and story-\Parg{} good-\Comp{} and -\Supl{} //
\endgl
\xe

\ex\begingl
\glpreamble \textit{Ang tahaya Linko nangās kivo-ikan.} \\
	/aŋ taˈhaja ˈliŋko naŋˈaːs ˈkivoˌikan/ //

\gla Ang tahaya Linko nangās kivo-ikan. //
\glb ang taha-ya Linko nanga-as kivo-ikan //
\glc \AgtT{} have-\TsgM{} Linko house-\Parg{} small-very //
\glft `Linko has a very small house.' //
\endgl\xe

\ex\ljudge*\begingl
\glpreamble \textit{Kivo-ikan ekeng nangāng.} \\
	/ˈkivoˌikan eˈkeŋ naŋˈaːŋ/ //

\gla Kivo-ikan ekeng nangāng. //
\glb kivo-ikan ekeng nanga-ang //
\glc small-very too house-\Aarg{} //
\glft `The house is too very small.' //
\endgl\xe

\ex\begingl
\glpreamble \textit{Kivo ekeng-ngas nangāng.} \\
	/ˈkivo eˈkeŋas naŋˈaːŋ/ //

\gla Kivo ekeng-ngas nangāng. //
\glb kivo ekeng-ngas nanga-ang //
\glc small too-almost house-\Aarg{} //
\glft `The house is almost too small.' //
\endgl\xe

\pex
\a\begingl
\glpreamble \textit{Kivo-ikannama nangāng.} \\
	/ˈkivoikaˌnama naŋˈaːŋ/ //

\gla Kivo-ikannama nangāng. //
\glb kivo-ikan-nama nanga-ang //
\glc small-very-just house-\Aarg{} //
\glft `The house is just very small.' //
\endgl

\a\ljudge*\begingl
\gla Kivo-ikan nay nama nangāng. //
\glb kivo-ikan nay nama nanga-ang //
\glc small-very and just house-\Aarg{} //
\endgl

\xe

\makefootnotehacks{C}
\pex
\a\begingl
\glpreamble \textit{Ang tahaya Linko nangās nay mondoas kivo-ikan.} \\
	/aŋ taˈhaja ˈliŋko naŋˈaːs naɪ monˈdoas ˈkivoˌikan/ //

\gla Ang tahaya Linko nangās nay mondoas kivo-ikan. //
\glb ang taha-ya Linko nanga-as and mondo-as kivo-ikan //
\glc \AgtT{} have-\TsgM{} Linko house-\Parg{} and garden-\Parg{} small-very //
\glft `Linko has a very small house and garden' //
\endgl

\a\ljudge?\begingl
\gla Ang tahaya Linko nangās kivo-ikan nay mondoas {kivo-ikan.\fnhackC} //
\glb ang taha-ya Linko nanga-as kivo-ikan nay mondo-as kivo-ikan //
\glc \AgtT{} have-\TsgM{} Linko house-\Parg{} small-very and garden small-very //
\endgl

\xe

\footnotehacktext{Again, not wrong as such, but unnecessarily wordy.}

\pex
\a\begingl
\glpreamble \textit{Ang tahaya Linko nangās kivo nay giro-ikan.} \\
	/aŋ taˈhaja ˈliŋko naŋˈaːs ˈkivo naɪ ˈgiroˌikan/ //

\gla Ang tahaya Linko nangās kivo nay giro-ikan. //
\glb ang taha-ya Linko nanga-as kivo nay giro-ikan //
\glc \AgtT{} have-\TsgM{} Linko house-\Parg{} small and cute-very //
\glft `Linko has a very small and cute house.' //
\endgl

\a\begingl
\gla Ang tahaya Linko nangās kivo-ikan nay giro-ikan. //
\glb ang taha-ya Linko nanga-as kivo-ikan nay giro-ikan //
\glc \AgtT{} have-\TsgM{} Linko house-\Parg{} small-very and cute-very //
\endgl

\a\begingl
\gla Ang tahaya Linko nangās kivo-ikan nay(nay) da-giro-ikan. //
\glb ang taha-ya Linko nanga-as kivo-ikan nay(\til{}nay) da-giro-ikan //
\glc \AgtT{} have-\TsgM{} Linko house-\Parg{} small-very and(\til{}also) one-cute-very //
\endgl

\a\ljudge!\begingl
\gla Ang tahaya Linko nangās kivo-ikan nay giro. //
\glb ang taha-ya Linko nanga-as kivo-ikan nay giro //
\glc \AgtT{} have-\TsgM{} Linko house-\Parg{} small-very and cute //
\glft `Linko has a very small, and cute, house.' \\
	\textit{Intended:} `Linko has a very small and cute house.' //
\endgl

\xe

\pex
\a\begingl
\glpreamble \textit{Ang tahaya Linko nangās kivo-ikan nay mondoas bino-ikan.} \\
	/aŋ taˈhaja ˈliŋko naŋˈaːs naŋˈaːs ˈkivoˌikan naɪ monˈdoas ˈbinoˌikan/ //

\gla Ang tahaya Linko nangās kivo-ikan nay mondoas bino-ikan. //
\glb ang taha-ya Linko nanga-as kivo-ikan nay mondo-as bino-ikan //
\glc \AgtT{} have-\TsgM{} Linko house-\Parg{} small-very and garden-\Parg{} colorful-very //
\glft `Linko has a very small house and a very colorful garden.' //
\endgl

\a\ljudge!\begingl
\gla Ang tahaya Linko nangās kivo nay mondoas bino-ikan. //
\glb ang taha-ya Linko nanga-as kivo nay mondo-as bino-ikan //
\glc \AgtT{} have-\TsgM{} Linko house-\Parg{} small and garden-\Parg{} colorful-very //
\glft `Linko has a small house and a very colorful garden.' \\
	\textit{Intended:} `Linko has a very small house and a very colorful garden.' //
\endgl

\xe

\pex
\a\begingl
\glpreamble \textit{Ang tahaya Linko nangās kivo-ikan nay mondoas bino-ven.} \\
	/aŋ taˈhaja ˈliŋko naŋˈaːs ˈkivoˌikan naɪ monˈdoas ˈbinoˌven/ //

\gla Ang tahaya Linko nangās kivo-ikan nay mondoas bino-ven. //
\glb ang taha-ya Linko nanga-as kivo-ikan nay mondo-as bino-ven //
\glc \AgtT{} have-\TsgM{} Linko house-\Parg{} small-very and garden-\Parg{} colorful-pretty //
\glft `Linko has a very small house and a pretty colorful garden.' //
\endgl

\a\ljudge*\begingl
\gla Ang tahaya Linko nangās kivo nay mondoas bino-ikan nay -ven. //
\glb ang taha-ya Linko nanga-as kivo nay mondo-as bino-ikan nay -ven //
\glc \AgtT{} have-\TsgM{} Linko house-\Parg{} small and garden-\Parg{} colorful-very and -pretty //
\endgl

\xe

\pex
\a\begingl
\glpreamble \textit{Ang noay da-tuvoley.} \\
	/aŋ ˈnwaɪ daˌtuvoˈleɪ/ //

\gla Ang noay da-tuvoley. //
\glb ang no-ay da-tuvo-ley //
\glc \AgtT{} want-\Fsg{} one-red-\PargI{} //
\glft `I want the red one.' //
\endgl

\a\begingl
\gla Ang noay danyaley tuvo. //
\glb ang no-ay danya-ley tuvo //
\glc \AgtT{} want-\Fsg{} such.one-\PargI{} red //
\endgl

\xe

\pex
\a\begingl
\glpreamble \textit{Le noyang da-tuvo.} \\
	/le noˈjaŋ daˈtuvo/ //

\gla Le noyang da-tuvo. //
\glb le no-yang da-tuvo //
\glc \PatTI{} want-\Fsg{}.\Aarg{} one-red //
\glft `The red one I want.' //
\endgl

\a\begingl
\gla Le noyang danya tuvo. //
\glb le no-yang danya tuvo //
\glc \PatTI{} want-\Fsg{}.\Aarg{} such.one red //
\endgl

\xe

\ex\begingl
\glpreamble \textit{Ang noay da-tuvoley kivo.} \\
	/aŋ ˈnwaɪ daˌtuvoˈleɪ ˈkivo/ //

\gla Ang noay da-tuvoley kivo. //
\glb ang no-ay da-tuvo-ley kivo //
\glc \AgtT{} want-\Fsg{} one-red-\PargI{} little //
\glft `I want the little red one.' //
\endgl
\xe

\pex
\a\ljudge?\begingl
\glpreamble \textit{Ang noay da-tuvoyeley.} \\
	/aŋ ˈnwaɪ datuˌvojeˈleɪ/ //

\gla Ang noay da-tuvoyeley. //
\glb ang no-ay da-tuvo-ye-ley //
\glc \AgtT{} want-\Fsg{} one-red-\Pl{}-\PargI{} //
\glft \textit{Intended:} `I want the red ones.' //
\endgl

\a\ljudge*\begingl
\gla Ang noay danyayeley tuvo. //
\glb ang no-ay danya-ye-ley tuvo //
\glc \AgtT{} want-\Fsg{} such.one-\Pl{}-\PargI{} red //
\endgl
\xe

\pex
\a\ljudge?\begingl
\glpreamble \textit{Ang noay da-tuvo-tuvoley.} \\
	/aŋ ˈnwaɪ daˈtuvotuˌvojeˈleɪ/ //

\gla Ang noay da-tuvo-tuvoley. //
\glb ang no-ay da-tuvo\til{}tuvo-ley //
\glc \AgtT{} want-\Fsg{} one-\Dim{}\til{}red-\PargI{} //
\glft \textit{Intended:} `I want the (cute) little red one.' //
\endgl

\a\ljudge*\begingl
\gla Ang noay danya-danyaley tuvo. //
\glb ang no-ay danya\til{}danya-ye-ley tuvo //
\glc \AgtT{} want-\Fsg{} \Dim{}\til{}such.one-\Pl{}-\PargI{} red //
\endgl

\xe

\pex
\a\begingl
\glpreamble \textit{Ang noay da-tuvoley nay da-lenoley.} \\
	/aŋ ˈnwaɪ daˌtuvoˈleɪ naɪ daˌlenoˈleɪ/ //

\gla Ang noay da-tuvoley nay da-lenoley. //
\glb ang no-ay da-tuvo-ley nay da-lenoley //
\glc \AgtT{} want-\Fsg{} one-red-\PargI{} and one-blue-\PargI{} //
\glft `I want the red one and the blue one.' //
\endgl

\a\ljudge*\begingl
\gla Ang noay da-tuvo nay da-lenoley. //
\glb ang no-ay da-tuvo nay da-lenoley //
\glc \AgtT{} want-\Fsg{} one-red and one-blue-\PargI{} //
\endgl

\a\ljudge*\begingl
\gla Ang noay da-tuvoley nay da-leno. //
\glb ang no-ay da-tuvo-ley nay da-leno //
\glc \AgtT{} want-\Fsg{} one-red-\PargI{} and one-blue //
\endgl

\a\ljudge*\begingl
\gla Ang noay da-tuvoley nay lenoley. //
\glb ang no-ay da-tuvo-ley nay leno-ley //
\glc \AgtT{} want-\Fsg{} one-red-\PargI{} and blue-\Parg{} //
\endgl

\a\ljudge!\begingl
\gla Ang noay da-tuvo nay lenoley. //
\glb ang no-ay da-tuvo nay leno-ley //
\glc \AgtT{} want-\Fsg{} one-red and blue-\Parg{} //
\glft `I want the red and blue one.' \\
	\textit{Intended:} `I want the red one and the blue one.' //
\endgl

\xe

\pex
\a\begingl
\glpreamble \textit{Ang noay da-yanaley.} \\
	/aŋ ˈnwaɪ daˌjanaˈleɪ/ //

\gla Ang noay da-yanaley. //
\glb ang no-ay da-yana-ley //
\glc \AgtT{} want-\Fsg{} one-\TsgM{}.\Gen{}-\PargI{} //
\glft `I want his one.' //
\endgl

\a\begingl
\gla Ang noay danyaley yana. //
\glb ang no-ay danya-ley yana //
\glc \AgtT{} want-\Fsg{} such.one-\PargI{} \TsgM{}.\Gen{} //
\endgl

\xe

\pex
\a\begingl
\glpreamble \textit{Le noyang da-yana.} \\
	/le noˈjaŋ daˈjana/ //

\gla Le noyang da-yana. //
\glb le no-yang da-yana //
\glc \PatTI{} want-\Fsg{}.\Aarg{} one-\TsgM{}.\Gen{} //
\glft `His one I want.' //
\endgl

\a\begingl
\gla Le noyang danya yana. //
\glb le no-yang danya yana //
\glc \PatTI{} want-\Fsg{}.\Aarg{} such.one \TsgM{}.\Gen{} //
\endgl

\xe

\pex
\a\ljudge?\begingl
\glpreamble \textit{Ang noay da-yanaley kivo.} \\
	/aŋ ˈnwaɪ daˌjanaˈleɪ ˈkivo/\footnotemark //

\gla Ang noay da-yanaley kivo. //
\glb ang no-ay da-yana-ley kivo //
\glc \AgtT{} want-\Fsg{} one-\TsgM{}.\Gen{}-\PargI{} little //
\glft `I want his little one.' //
\endgl

\a\ljudge*\begingl
\gla Ang noay da-yanaley vana. //
\glb ang no-ay da-yana-ley vana //
\glc \AgtT{} want-\Fsg{} one-\TsgM{}.\Gen{}-\PargI{} \Second{}.\Gen{} //
\endgl

\xe

\footnotetext{Strange-sounding, since it's otherwise customary to place the 
possessive pronoun after an attributive adjective: \textit{veney kivo yana} 
(dog little \TsgM{}.\Gen{}) `his little dog', not ?\textit{veney yana kivo}.
Hence, \textit{da-kivoley yana} would be more licit.}

\pex
\a\ljudge?\begingl
\glpreamble \textit{Ang noay da-yanayeley.} \\
	/aŋ ˈnwaɪ datuˌvojeˈleɪ/ //

\gla Ang noay da-yanayeley. //
\glb ang no-ay da-yana-ye-ley //
\glc \AgtT{} want-\Fsg{} one-\TsgM{}.\Gen{}-\Pl{}-\PargI{} //
\glft \textit{Intended:} `I want his ones.' //
\endgl

\a\ljudge*\begingl
\gla Ang noay danyayeley yana. //
\glb ang no-ay danya-ye-ley yana //
\glc \AgtT{} want-\Fsg{} such.one-\Pl{}-\PargI{} \TsgM{}.\Gen{} //
\endgl
\xe

\pex
\a\ljudge?\begingl
\glpreamble \textit{Ang noay da-yana-yanaley.} \\
	/aŋ ˈnwaɪ daˈjanajaˌnajeˈleɪ/ //

\gla Ang noay da-yana-yanaley. //
\glb ang no-ay da-yana\til{}yana-ley //
\glc \AgtT{} want-\Fsg{} one-\Dim{}\til{}\TsgM{}.\Gen{}-\PargI{} //
\glft \textit{Intended:} `I want the (cute) little his-one.' //
\endgl

\a\ljudge*\begingl
\gla Ang noay danya-danyaley yana. //
\glb ang no-ay danya\til{}danya-ye-ley yana //
\glc \AgtT{} want-\Fsg{} \Dim{}\til{}such.one-\Pl{}-\PargI{} \TsgM{}.\Gen{} //
\endgl

\xe

\pex
\a\begingl
\glpreamble \textit{Ang noay da-yanaley nay da-vanaley.} \\
	/aŋ ˈnwaɪ daˌjanaˈleɪ naɪ daˌvanaˈleɪ/ //

\gla Ang noay da-yanaley nay da-vanaley. //
\glb ang no-ay da-yana-ley nay da-vanaley //
\glc \AgtT{} want-\Fsg{} one-\TsgM{}.\Gen{}-\PargI{} and one-\Second{}.\Gen{}-\PargI{} //
\glft `I want (both) his and yours.' //
\endgl

\a\ljudge*\begingl
\gla Ang noay da-yana nay da-vanaley. //
\glb ang no-ay da-yana nay da-vanaley //
\glc \AgtT{} want-\Fsg{} one-\TsgM{}.\Gen{} and one-\Second{}.\Gen{}-\PargI{} //
\endgl

\a\ljudge*\begingl
\gla Ang noay da-yanaley nay da-vana. //
\glb ang no-ay da-yana-ley nay da-vana //
\glc \AgtT{} want-\Fsg{} one-\TsgM{}.\Gen{}-\PargI{} and one-\Second{}.\Gen{} //
\endgl

\a\ljudge*\begingl
\gla Ang noay da-yanaley nay vanaley. //
\glb ang no-ay da-yana-ley nay vana-ley //
\glc \AgtT{} want-\Fsg{} one-\TsgM{}.\Gen{}-\PargI{} and \Second{}.\Gen{}-\Parg{} //
\endgl

\a\ljudge!\begingl
\gla Ang noay da-yana nay vanaley. //
\glb ang no-ay da-yana nay vana-ley //
\glc \AgtT{} want-\Fsg{} one-\TsgM{}.\Gen{} and \Second{}.\Gen{}-\Parg{} //
\glft `I want him and yours.' \\
	\textit{Intended:} `I want (both) his and yours.' //
\endgl

\xe

\ex\begingl
\glpreamble \textit{Ang caya Akan veneyas yena.} \\
	/aŋ ˈtʃaja ˈakan veˈnejas ˈjena/ //

\gla Ang caya Akan veneyas yena. //
\glb ang ca-ya Akan veney-as yena //
\glc \AgtT{} love-\TsgM{} Akan dog-\Loc{} \TsgF{}.\Gen{} //
\glft `Akan loves her dog.' //
\endgl\xe

\pex
\a\begingl
\glpreamble \textit{Ang caya Akan veneyas kivo yena.} \\
	/aŋ ˈtʃaja ˈakan veˈnejas ˈkivo ˈjena/ //

\gla Ang caya Akan veneyas kivo yena. //
\glb ang ca-ya Akan veney-as kivo yena //
\glc \AgtT{} love-\TsgM{} Akan dog-\Loc{} little \TsgF{}.\Gen{} //
\glft `Akan loves her little dog.' //
\endgl

\a\ljudge?\begingl
\gla Ang caya Akan veneyas yena kivo. //
\glb ang ca-ya Akan veney-as yena kivo //
\glc \AgtT{} love-\TsgM{} Akan dog-\Loc{} \TsgF{}.\Gen{} little //
\endgl

\xe

\ex\begingl
\glpreamble \textit{Ang caya Akan veneyas yana nay yena.} \\
	/aŋ ˈtʃaja ˈakan veˈnejas ˈjana naɪ ˈjena/ //

\gla Ang caya Akan veneyas yana nay yena. //
\glb ang ca-ya Akan veney-as yana nay yena //
\glc \AgtT{} love-\TsgM{} Akan dog-\Loc{} \TsgM{}.\Gen{} and \TsgF{}.\Gen{} //
\glft `Akan loves his and her dog.' //
\endgl\xe

\pex
\a\begingl
\glpreamble \textit{Ang caya Akan veneyas kivo yana nay yena.} \\
	/aŋ ˈtʃaja ˈakan veˈnejas ˈkivo ˈjana naɪ ˈjena/ //

\gla Ang caya Akan veneyas kivo yana nay yena. //
\glb ang ca-ya Akan veney-as kivo yana nay yena //
\glc \AgtT{} love-\TsgM{} Akan dog-\Loc{} little \TsgM{}.\Gen{} and \TsgF{}.\Gen{} //
\glft `Akan loves his and her little dog.' //
\endgl

\a\ljudge?\begingl
\gla Ang caya Akan veneyas yana nay yena kivo. //
\glb ang ca-ya Akan veney-as yana nay yena kivo //
\glc \AgtT{} love-\TsgM{} Akan dog-\Loc{} \TsgM{}.\Gen{} and \TsgF{}.\Gen{} little //
\endgl

\xe

\pex
\a\begingl
\glpreamble \textit{Ang Misan lajāyas puti.} \\
	/aŋ ˈmisan laˈdʒaːjas ˈputi/ //

\gla Ang Misan lajāyas puti. //
\glb ang Misan lajāy-as puti //
\glc \Aarg{} Misan student-\Parg{} zealous //
\glft `Misan is a zealous student.' //
\endgl

\a\begingl
\gla Lajāyas puti ang Misan. //
\glb lajāy-as puti ang Misan //
\glc student-\Parg{} zealous \Aarg{} Misan //
\glft `A zealous student is what Misan is.' //
\endgl
\xe

\pex
\a\begingl
\glpreamble \textit{Ang Misan puti.} \\
	/aŋ ˈmisan ˈputi/ //

\gla Ang Misan puti. //
\glb ang Misan puti //
\glc \Aarg{} Misan zealous //
\glft `Misan is zealous.' //
\endgl

\a\begingl
\gla Puti ang Misan. //
\glb Puti ang Misan //
\glc zealous \Aarg{} Misan //
\endgl
\xe

\pex
\a\begingl
\glpreamble \textit{Puti lajāyang.} \\
	/ˈputi laˈdʒaːjang/ //

\gla Puti lajāyang. //
\glb Puti lajāy-ang //
\glc zealous student-\Aarg{} //
\glft `The student is zealous.' //
\endgl

\a\ljudge!\begingl
\gla Lajāyang puti. //
\glb lajāy-ang puti //
\glc student-\Aarg{} zealous //
\glft `The zealous student.' \\
	\textit{Intended: The student is zealous.} //
\endgl
\xe

\pex
\a\begingl
\glpreamble \textit{Yeng puti.} \\
	/ˈjeŋ ˈputi/ //

\gla Yeng puti. //
\glb yeng puti //
\glc \TsgF{}.\Aarg{} zealous //
\glft `She is zealous.' //
\endgl

\a\begingl
\gla Puti yeng. //
\glb Puti yeng //
\glc zealous \TsgF{}.\Aarg{} //
\endgl
\xe

\pex
\a\begingl
\glpreamble \textit{Ang Misan/lajāyang/yeng voy puti.} //
	/aŋ ˈmisan//laˈdʒaːjaŋ/ˈjeŋ voɪ ˈputi/ //

\gla Ang Misan @ / @ lajāyang @ / @ yeng voy puti //
\glb ang Misan / lajāy-ang / yeng voy puti //
\glc \Aarg{} Misan {} student {} \TsgF{}.\Aarg{} \Neg{} zealous //
\glft `Misan/the student/she is not zealous.' //
\endgl

\a\begingl
\gla Voy puti ang Misan @ / @ lajāyang @ / @ yeng //
\glb voy puti ang Misan / lajāy-ang / yeng //
\glc \Neg{} zealous \Aarg{} Misan {} student {} \TsgF{}.\Aarg{} //
\endgl

\a\ljudge*\begingl
\gla Puti voy ang Misan @ / @ lajāyang @ / @ yeng //
\glb puti voy ang Misan / lajāy-ang / yeng //
\glc zealous \Neg{} \Aarg{} Misan {} student {} \TsgF{}.\Aarg{} //
\endgl
\xe

\pex
\a\begingl
\glpreamble \textit{Ang Misan/lajāyang/yeng voy takomayās.} //
	/aŋ ˈmisan//laˈdʒaːjaŋ/ˈjeŋ voɪ takomaˈjaːs/ //

\gla Ang Misan @ / @ lajāyang @ / @ yeng voy takomayās //
\glb ang Misan / lajāy-ang / yeng voy takomaya-as //
\glc \Aarg{} Misan {} student {} \TsgF{}.\Aarg{} \Neg{} slob-\Parg{} //
\glft `Misan/the student/she is not a slob.' //
\endgl

\a\begingl
\gla Voy takomayās ang Misan @ / @ lajāyang @ / @ yeng //
\glb voy takomaya-as ang Misan / lajāy-ang / yeng //
\glc \Neg{} slob-\Parg{} \Aarg{} Misan {} student {} \TsgF{}.\Aarg{} //
\endgl

\a\ljudge*\begingl
\gla Takomayās voy ang Misan @ / @ lajāyang @ / @ yeng //
\glb takomaya-as voy ang Misan / lajāy-ang / yeng //
\glc slob-\Parg{} \Neg{} \Aarg{} Misan {} student {} \TsgF{}.\Aarg{} //
\endgl
\xe

\ex\begingl
\glpreamble \textit{Padu nay mico yarangang.} \\
	/ˈpadu naɪ ˈmitʃo jaˈraŋaŋ/ //

\gla Padu nay mico yarangang. //
\glb padu nay mico yarang-ang //
\glc reliable and strong ox-\Aarg{} //
\glft `The ox is reliable and strong.' //
\endgl\xe

\ex\begingl
\glpreamble \textit{Voy padarya nay sapu yarangang.} \\
	/voɪ paˈdarja naɪ ˈsapu jaˈraŋaŋ/ //

\gla Voy padarya nay sapu yarangang. //
\glb voy padarya nay sapu yarang-ang //
\glc \Neg{} unreliable and weak ox-\Aarg{} //
\glft `The ox is not unreliable and weak.' //
\endgl\xe

\ex\begingl
\glpreamble \textit{Padu nay voy sapu yarangang.} \\
	/ˈpadu naɪ voɪ ˈsapu jaˈraŋaŋ/ //

\gla Padu nay voy sapu yarangang. //
\glb padu nay voy sapu yarang-ang //
\glc reliable and \Neg{} weak ox-\Aarg{} //
\glft `The ox is reliable and not weak.' //
\endgl\xe

\ex\begingl
\glpreamble \textit{Yarangang voy depangas nay sapumayās.} \\
	/jaˈraŋaŋ voɪ deˈpaŋas naɪ sapumaˈjaːs/ //

\gla Yarangang voy depangas nay sapumayās. //
\glb yarang-ang voy depang-as nay sapumaya-as //
\glc ox-\Aarg{} \Neg{} fool-\Parg{} and weakling-\Parg{} //
\glft `The ox is no fool and weakling.' //
\endgl\xe

\ex\begingl
\glpreamble \textit{Ang mingyon palay ninganye na Baba.} \\
	/aŋ ˈmingjon paˈlaɪ niŋˈanje na ˈbaba/ //

\gla Ang mingyon palay ninganye na Baba. //
\glb ang ming-yon palay ningan-ye na Baba //
\glc \AgtT{} can-\TsgN{}.\Pl{} funny story-\Pl{} \Gen{} Grampa //
\glft `Grampa's stories can be funny.' //
\endgl\xe

\ex\begingl
\glpreamble \textit{Ang mingyo gumo babekalya kotaneri.} \\
	/aŋ ˈmiŋjo ˈgumo babeˈkalja kotaˈneri/ //

\gla Ang mingyo gumo babekalya kotaneri. //
\glb ang ming-yo gumo babekal-ja kotan-eri //
\glc \Aarg{} can-\TsgN{} work holiday-\Loc{} toil-\Ins{} //
\glft `Work on a holiday can be toil.' //
\endgl\xe

\pex
\a\begingl
\glpreamble \textit{Surpya kayto ang Denan.} \\
	/ˈsurpja ˈkaɪto aŋ ˈdenan/ //

\gla Surpya kayto ang Denan. //
\glb surp-ya kayto ang Denan //
\glc seem-\TsgM{} right \Aarg{} Denan //
\glft `Denan seems right.' //
\endgl

\a\begingl
\gla Surpya kayto ayonang. //
\glb surp-ya kayto ayonang //
\glc seem-\TsgM{} right man-\Aarg{} //
\glft `Them man seems right.' //
\endgl

\a\begingl
\gla Surpyāng kayto. //
\glb surp-yāng kayto //
\glc seem-\TsgM{}.\Aarg{} right //
\glft `He seems right.' //
\endgl

\xe

\pex
\a\begingl
\glpreamble \textit{Ang surpya Denan sobayās.} \\
	/aŋ ˈsurpja ˈdenan sobaˈjaːs/ //

\gla Ang surpya Denan sobayās. //
\glb ang surp-ya Denan sobaya-as //
\glc \AgtT{} seem-\TsgM{} Denan teacher-\Parg{} //
\glft `Denan seems to be a teacher.' //
\endgl

\a\begingl
\gla Surpreng, ang Denan sobayās. //
\glb surp-reng ang Denan sobaya-as //
\glc seem-\TsgI{} \Aarg{} Denan teacher-\Parg{} //
\glft `It seems that Denan is a teacher.' //
\endgl

\a\ljudge*\begingl
\gla Surpya ang Denan, yāng sobayās. //
\glb surp-ya ang Denan, yāng sobaya-as //
\glc seem-\TsgM{} \Aarg{} Denan, \TsgM{}.\Aarg{} teacher-\Parg{} //
\endgl

\xe

\ex object predicatives \xe

\vfill

\begin{multicols}{2}
\printglossary[style=mysuper,type=\leipzigtype]
\end{multicols}

\end{document}
