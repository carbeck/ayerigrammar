\documentclass[12pt,paper=a4]{scrartcl}

% Extended formatting of lists
\usepackage{enumitem}
\newlist{glossdefs}{itemize}{1}
\setlist[glossdefs]{nosep, leftmargin=3em, labelwidth=2.5em, align=left}

% Make multiple columns available in single-column document
\usepackage{multicol}

% Load font stuff for XeTeX
\usepackage{xltxtra}
\usepackage{fontspec}

% Fonts
\setmainfont{FreeSerif}[
 Ligatures=TeX,
]

% Clickable links in footnotes, TOC, etc.
\usepackage[
    xetex,
    hidelinks,
]{hyperref}

% Formatting of table of glossing abbreviations from Leipzig package manual
\usepackage[acronym,nomain,nonumberlist,nopostdot]{glossaries}
\usepackage{glossary-inline}%

\newglossarystyle{mysuper}{%
 \glossarystyle{super}% based on super
 \renewenvironment{theglossary}{%
 \begin{glossdefs}%
 }{%
 \end{glossdefs}%
 }%
 \renewcommand*{\glossaryheader}{}%
 \renewcommand*{\glsgroupheading}[1]{}%
 \renewcommand*{\glossaryentryfield}[5]{%
 \item[\glsentryitem{##1}\glstarget{##1}{##2}]
 \makefirstuc{##3}\glspostdescription{}
 }%
 \renewcommand*{\glsgroupskip}{}%
}%

% Formatting of glosses
\usepackage{expex}
\defineglwlevels{d}
\lingset{everygld=\scriptsize}

\usepackage{leipzig}

\newleipzig{AgtT}{at}{agent topic}
\newleipzig{PatT}{pt}{patient topic}
\newleipzig{DatT}{datt}{dative topic}
\newleipzig{GenT}{gent}{genitive topic}
\newleipzig{LocT}{loct}{locative topic}
\newleipzig{InsT}{inst}{instrumental topic}
\newleipzig{CauT}{caut}{causative topic}
\newleipzig{An}{an}{animate}
\newleipzig{Inan}{inan}{inanimate}
\newleipzig{Hab}{hab}{habitative}
\newleipzig{Ayr}{ayr}{Ayeri}
\newleipzig{NPst}{npst}{near past}
\newleipzig{RPst}{rpst}{remote past}
\newleipzig{NFut}{nfut}{near future}
\newleipzig{RFut}{rfut}{remote future}
\newleipzig{Agtz}{agtz}{agentizer}
\newleipzig{Dim}{dim}{diminutive}
\newleipzig{Comp}{comp}{comparative}
\newleipzig{Supl}{supl}{superlative}

\newleipzig{Dyn}{dyn}{dynamic}
\newleipzig{Hort}{hort}{hortative}
\newleipzig{Iter}{iter}{iterative}
\newleipzig{Aff}{aff}{affirmative}
\newleipzig{Int}{int}{intensifier}

\newleipzig{Adj}{adj}{adjective}
\newleipzig{Adp}{adp}{adposition}
\newleipzig{Lnk}{lnk}{linker}
\newleipzig{Nn}{nn}{noun}
\newleipzig{Post}{post}{postposition}
\newleipzig{Prep}{prep}{preposition}
\newleipzig{Sat}{sat}{satellite}
\newleipzig{Vb}{vb}{verb}

\newleipzig{AF}{af}{argument function}
\newleipzig{DF}{df}{discourse function}
\newleipzig{Adjc}{adj}{adjunct}
\newleipzig{Anim}{anim}{animacy}
\newleipzig{Case}{case}{case}
\newleipzig{Compar}{compar}{comparison}
\newleipzig{Compl}{comp}{complement}
\newleipzig{Fin}{fin}{finiteness}
\newleipzig{Gend}{gend}{gender}
\newleipzig{Index}{index}{index}
\newleipzig{Num}{num}{number}
\newleipzig{Obj}{obj}{object}
\newleipzig{SObj}{obj\fakesubscript{\normalfont\itshape θ}}{secondary object}
\newleipzig{Oblique}{obl}{oblique}
\newleipzig{Pers}{pers}{person}
\newleipzig{Pred}{pred}{predicator}
\newleipzig{Spec}{spec}{specificity}
\newleipzig{Sbj}{subj}{subject}
\newleipzig{Tense}{tense}{tense}
\newleipzig{XCompl}{xcomp}{open complement}

\makeglossaries

\newcommand{\AargI}{{\Aarg}.{\Inan}}
\newcommand{\PargI}{{\Parg}.{\Inan}}
\newcommand{\AgtTI}{{\AgtT}.{\Inan}}
\newcommand{\PatTI}{{\PatT}.{\Inan}}

\newcommand{\TsgM}{{\Tsg}.{\M}}
\newcommand{\TsgF}{{\Tsg}.{\F}}
\newcommand{\TsgN}{{\Tsg}.{\N}}
\newcommand{\TsgI}{{\Tsg}.{\Inan}}
\newcommand{\TplM}{{\Tpl}.{\M}}
\newcommand{\TplF}{{\Tpl}.{\F}}
\newcommand{\TplN}{{\Tpl}.{\N}}
\newcommand{\TplI}{{\Tpl}.{\Inan}}

\newcommand{\Oblq}[1]{\Oblique\fakesubscript{\normalfont\itshape #1}}
\newcommand{\OblqT}{\Oblique\fakesubscript{\normalfont\itshape θ}}

% Ability to draw AVMs
% (Requires avm.sty from http://nlp.stanford.edu/manning/tex/)
\usepackage{avm}
\avmfont{\scshape}
\avmvalfont{\normalfont}

% Other macros
\newcommand{\til}{$\sim$} %{\textasciitilde{}} % Tilde shortcut

\begin{document}

\ex\begingl
\glpreamble \textit{Ang ilya Ajān koyās yam Pila.}\\
	/aŋ‿ˈilja | aˈdʒaːn | koˈjaːs | jam ˈpila/ //

\gla Ang ilya Ajān koyās yam Pila. //
\glb ang il-ya Ajān koya-as yam Pila //
\glc \AgtT{} give-\TsgM{} Ajān book-\Parg{} \Dat{} Pila //
\glft `Ajān gives Pila a book.' //
\endgl\xe

\ex\begingl
\glpreamble \textit{Sa ilya Ajān ada-koya yam Pila.}\\
	/sa ˈilja | aˈdʒaːn | adaˈkoja | jam ˈpila/ //

\gla Sa ilya Ajān ada-koya yam Pila. //
\glb sa il-ya Ajān ada-koya yam Pila //
\glc \PatT{} give-\TsgM{} Ajān that-book \Dat{} Pila //
\glft `That book, Ajān gives (it) to Pila.' //
\endgl\xe

\ex\begingl
\glpreamble \textit{Ang ilya Ajān koyās yam ada-Pila.}\\
	/aŋ‿ˈilja | aˈdʒaːn | koˈjaːs | jam‿adaˈpila/ //

\gla Ang ilya Ajān koyās yam ada-Pila. //
\glb ang il-ya Ajān koya-as yam ada-Pila //
\glc \AgtT{} give-\TsgM{} Ajān book-\Parg{} \Dat{} that-Pila //
\glft `Ajān gives that Pila a book.' //
\endgl\xe

\ex\begingl
\glpreamble \textit{Ang ilya Ajān sa `Netuye Karamazov' yam Pila.}\\
	/aŋ‿ˈilja | aˈdʒaːn | sa neˈtuje karaˈmazov | jam ˈpila/ //

\gla Ang ilya Ajān sa `Netuye Karamazov' yam Pila. //
\glb ang il-ya Ajān sa `Netu-ye Karamazov' yam Pila //
\glc \AgtT{} give-\TsgM{} Ajān \Parg{} `Brother-\Pl{} Karamazov' \Dat{} Pila //
\glft `Ajān gives Pila the “Brothers Karamazov”.' //
\endgl\xe

\ex\begingl
\glpreamble \textit{Ang ilya Ajān koyās yam Pila nay Latun.}\\
	/aŋ‿ˈilja | aˈdʒaːn | koˈjaːs | jam ˈpila naɪ ˈlatun/ //

\gla Ang ilya Ajān koyās yam Pila nay Latun. //
\glb ang il-ya Ajān koya-as yam Pila nay Latun //
\glc \AgtT{} give-\TsgM{} Ajān book-\Parg{} \Dat{} Pila and Latun //
\glft `Ajān gives Pila and Latun a book.' //
\endgl\xe

\ex\begingl
\glpreamble \textit{Ang ilya Ajān koyās yam Pila nay netuyam yena.}\\
	/aŋ‿ˈilja | aˈdʒaːn | koˈjaːs | jam ˈpila naɪ neˈtujam ˈjena/ //

\gla Ang ilya Ajān koyās yam Pila nay netuyam yena. //
\glb ang il-ya Ajān koya-as yam Pila nay netu-yam yena //
\glc \AgtT{} give-\TsgM{} Ajān book-\Parg{} \Dat{} Pila and brother-\Dat{} her.\Gen{} //
\glft `Ajān gives Pila and her brother a book.' //
\endgl\xe

\ex\begingl
\glpreamble \textit{Ang ilyan Ajān nay Diyan koyās yam Pila.}\\
	/aŋ‿ˈiljan | aˈdʒaːn naɪ ˈdijan | koˈjaːs | jam ˈpila/ //

\gla Ang ilyan Ajān nay Diyan koyās yam Pila. //
\glb ang il-yan Ajān nay Diyan koya-as yam Pila //
\glc \AgtT{} give-\TplM{} Ajān and Diyan book-\Parg{} \Dat{} Pila //
\glft `Ajān and Diyan give Pila a book.' //
\endgl\xe

\ex\begingl
\glpreamble \textit{Ang ilya edāyon nay ledan koyās yam Pila.}\\
	/aŋ‿ˈilja | eˈdaːjon naɪ ˈledan | koˈjaːs | jam ˈpila/ //

\gla Ang ilya edāyon nay ledan koyās yam Pila. //
\glb ang il-ya eda-ayon nay ledan koya-as yam Pila //
\glc \AgtT{} give-\TsgM{} this-man and friend book-\Parg{} \Dat{} Pila //
\glft `This man and friend gives Pila a book.' //
\endgl\xe

\ex\begingl
\glpreamble \textit{Ang ilyan edāyon nay eda-ledan koyās yam Pila.}\\
	/aŋ‿ˈiljan | eˈdaːjon naɪ edaˈledan | koˈjaːs | jam ˈpila/ //

\gla Ang ilyan edāyon nay eda-ledan koyās yam Pila. //
\glb ang il-yan eda-ayon nay eda-ledan koya-as yam Pila //
\glc \AgtT{} give-\TplM{} this-man and this-friend book-\Parg{} \Dat{} Pila //
\glft `This man and this friend give Pila a book.' //
\endgl\xe

\ex\begingl
\glpreamble \textit{Ang ilya Ajān koyās nay migrayas yam Pila.}\\
	/aŋ‿ˈilja | aˈdʒaːn | koˈjaːs naɪ miˈgrajas | jam ˈpila/ //

\gla Ang ilya Ajān koyās nay migrayas yam Pila. //
\glb ang il-ya Ajān koya-as nay migray-as yam Pila //
\glc \AgtT{} give-\TsgM{} Ajān book-\Parg{} and flower-\Parg{} \Dat{} Pila //
\glft `Ajān gives Pila a book and a pen.' //
\endgl\xe

\ex\begingl
\glpreamble \textit{Ang ilya Ajān koyās nay dadangley yam Pila.}\\
	/aŋ‿ˈilja | aˈdʒaːn | koˈjaːs naɪ dadaŋˈleɪ | jam ˈpila/ //

\gla Ang ilya Ajān koyās nay dadangley yam Pila. //
\glb ang il-ya Ajān koya-as nay dadang-ley yam Pila //
\glc \AgtT{} give-\TsgM{} Ajān book-\Parg{} and pen-\PargI{} \Dat{} Pila //
\glft `Ajān gives Pila a book and a pen.' //
\endgl\xe

\ex\begingl
\glpreamble \textit{Ang ilya Ajān koyās-ikan yam Pila.}\\
	/aŋ‿ˈilja | aˈdʒaːn | koˌjaːsˈikan | jam ˈpila/ //

\gla Ang ilya Ajān koyās-ikan yam Pila. //
\glb ang il-ya Ajān koya-as-ikan yam Pila //
\glc \AgtT{} give-\TsgM{} Ajān book-\Parg{}-many \Dat{} Pila //
\glft `Ajān gives Pila many books.' //
\endgl\xe

\ex\begingl
\glpreamble \textit{Ang ilya Ajān koyajas nay dadangley-ikan merambay yam Pila.}\\
	/aŋ‿ˈilja | aˈdʒaːn | koˈjadʒas naɪ dadaŋˌleɪ‿ˈikan meramˈbaɪ | jam ˈpila/ //

\gla Ang ilya Ajān koyajas nay dadangley-ikan merambay yam Pila. //
\glb ang il-ya Ajān koya-ye-as nay dadang-ley-ikan merambay yam Pila //
\glc \AgtT{} give-\TsgM{} Ajān book-\Pl{}-\Parg{} and pen-\PargI{}-many useful \Dat{} Pila //
\glft `Ajān gives Pila many useful books and pens.' //
\endgl\xe

\ex\begingl
\glpreamble \textit{Ang ilya Ajān koyajas, nay(nay) dadangley-ikan merambay yam Pila.}\\
	/aŋ‿ˈilja | aˈdʒaːn | koˈjadʒas | naɪ(ˈnaɪ) dadaŋˌleɪ‿ˈikan meramˈbaɪ | jam ˈpila/ //

\gla Ang ilya Ajān koyajas, nay(nay) dadangley-ikan merambay yam Pila. //
\glb ang il-ya Ajān koya-ye-as nay(nay) dadang-ley-ikan merambay yam Pila //
\glc \AgtT{} give-\TsgM{} Ajān book-\Pl{}-\Parg{} and(.also) pen-\PargI{}-many useful \Dat{} Pila //
\glft `Ajān gives Pila books, and (also) many useful pens.' //
\endgl\xe

\ex\begingl
\glpreamble \textit{Ang ilya Ajān koyās-ikan nay dadangley merambay yam Pila.}\\
	/aŋ‿ˈilja | aˈdʒaːn | koˈjaːs‿ˈikan naɪ dadaŋˌleɪ meramˈbaɪ | jam ˈpila/ //

\gla Ang ilya Ajān koyās-ikan nay dadangley merambay yam Pila. //
\glb ang il-ya Ajān koya-as-ikan nay dadang-ley merambay yam Pila //
\glc \AgtT{} give-\TsgM{} Ajān book-\Parg{}-many and pen-\PargI{} useful \Dat{} Pila //
\glft `Ajān gives Pila many books and a useful pen.' //
\endgl\xe

\ex\begingl
\glpreamble \textit{Ang ilya Ajān koyās-ikan nay dadangyeley merambay yam Pila.}\\
	/aŋ‿ˈilja | aˈdʒaːn | koˈjaːs‿ˈikan naɪ dadaŋˌjeleɪ meramˈbaɪ | jam ˈpila/ //

\gla Ang ilya Ajān koyās-ikan nay dadangyeley merambay yam Pila. //
\glb ang il-ya Ajān koya-as-ikan nay dadang-ye-ley merambay yam Pila //
\glc \AgtT{} give-\TsgM{} Ajān book-\Parg{}-many and pen-\Pl{}-\PargI{} useful \Dat{} Pila //
\glft `Ajān gives Pila many books and useful pens.' //
\endgl\xe

\ex\begingl
\glpreamble \textit{Ang ilya Ajān koyajas nay dadangyeley merambay-ikan yam Pila.}\\
	/aŋ‿ˈilja | aˈdʒaːn | koˈjadʒas naɪ daˌdaŋjeˈleɪ meramˌbaɪ‿ˈikan | jam ˈpila/ //

\gla Ang ilya Ajān koyajas nay dadangyeley merambay-ikan yam Pila. //
\glb ang il-ya Ajān koya-ye-as nay dadang-ye-ley merambay-ikan yam Pila //
\glc \AgtT{} give-\TsgM{} Ajān book-\Pl{}-\Parg{} and pen-\Pl{}-\PargI{} useful-very \Dat{} Pila //
\glft `Ajān gives Pila very useful books and pens.' //
\endgl\xe

\ex\begingl
\glpreamble \textit{Merambay sikan koyajang nay dadangyeley? — Da-ikan. (*Ikan.)}\\
	/meramˈbaɪ ˈsikan | kojaˈdʒaŋ naɪ daˌdaŋjeˈleɪ/ — /daˈikan/ //

\gla Merambay sikan koyajang nay dadangyeley? — Da-ikan. //
\glb merambay sikan koya-ye-ang nay dadang-ye-ley {} da-ikan //
\glc useful how.much book-\Pl{}-\Aarg{} and pen-\Pl{}-\PargI{} {} so-very //
\glft `How useful are the books and pens?' — `Very.' //
\endgl\xe

\ex\begingl
\glpreamble \textit{Ang misya ku-depangas.}\\
	/aŋ ˈmisja | kudeˈpaŋas/ //

\gla Ang misya ku-depangas. //
\glb ang mis-ya ku-depang-as //
\glc \AgtT{} act-\TsgM{} like-fool-\Parg{} //
\glft `He acts like a fool.' //
\endgl\xe

\ex\begingl
\glpreamble \textit{Sa misyāng ku-depang.}\\
	/sa misˈjaːŋ | kudeˈpaŋ/ //

\gla Sa misya ku-depang. //
\glb sa mis-ya ku-depang //
\glc \PatT{} act-\TsgM{} like-fool //
\glft `Like a fool he acts.' //
\endgl\xe

\ex\begingl
\glpreamble \textit{Ang misya ku-da-depangas.}\\
	/aŋ ˈmisja | kudadeˈpaŋas/ //

\gla Ang misya ku-da-depangas. //
\glb ang mis-ya ku-da-depang-as //
\glc \AgtT{} act-\TsgM{}.\Aarg{} like-such-fool-\Parg{} //
\glft `He acts like such a fool.' //
\endgl\xe

\ex\begingl
\glpreamble \textit{Ang misya nay lentaya ku-depangas.}\\
	/aŋ ˈmisja naɪ lenˈtaja | kudeˈpaŋas/ //

\gla Ang misya nay lentaya ku-depangas. //
\glb ang mis-ya nay lenta-ya ku-depang-as //
\glc \AgtT{} act-\TsgM{} and sound-\TsgM{} like-fool-\Parg{} //
\glft `He acts and sounds like a fool.' //
\endgl\xe

\ex\begingl
\glpreamble \textit{Sa misyāng nay lentayāng ku-depang.}\\
	/sa misˈjaːng naɪ lentaˈjaːŋ | kudeˈpaŋ/ //

\gla Sa misyāng nay lentayāng ku-depangas. //
\glb sa mis-yāng nay lenta-yāng ku-depang-as //
\glc \PatT{} act-\TsgM{}.\Aarg{} and sound-\TsgM{}.\Aarg{} like-fool-\Parg{} //
\glft `Like a fool he acts and sounds.' //
\endgl\xe

\ex\begingl
\glpreamble \textit{Ang misya ku-depangas nay karayās.}\\
	/aŋ ˈmisja | kudeˈpaŋas naɪ karaˈjaːs/ //

\gla Ang misya ku-depangas nay karayās //
\glb ang mis-ya ku-depang-as nay karaya-as //
\glc \AgtT{} act-\TsgM{}.\Aarg{} like-fool-\Parg{} and coward-\Parg{} //
\glft `He acts like a fool and coward.' //
\endgl\xe

\ex\begingl
\glpreamble \textit{Ang misya ku-da-depangas nay karayās.}\\
	/aŋ ˈmisja | kudadeˈpaŋas naɪ karaˈjaːs/ //

\gla Ang  misyāng ku-da-depangas nay karayās //
\glb ang mis-yāng ku-da-depang-as nay karaya-as //
\glc \AgtT{} act-\TsgM{}.\Aarg{} like-such-fool-\Parg{} and coward-\Parg{} //
\glft `He acts like such a fool and coward.' //
\endgl\xe

\ex\begingl
\glpreamble \textit{Ang silvye ku-māvās yena.}\\
	/aŋ ˈsilvje | kuˌmaːˈvaːs ˈjena/ //

\gla Ang silvye ku-māvās yena. //
\glb ang silv-ye ku-māva-as yena //
\glc \AgtT{} look-\TsgF{} like-mother-\Parg{} \TsgF{}.\Gen{} //
\glft `She looks like her mother.' //
\endgl\xe

\ex\begingl
\glpreamble \textit{Ang silvye Pada ku-māvās yena.}\\
	/aŋ ˈsilvje | ˈpada | kuˌmaːˈvaːs ˈjena/ //

\gla Ang silvye Pada ku-māvās yena. //
\glb ang silv-ye Pada ku-māva-as yena //
\glc \AgtT{} look-\TsgF{} Pada like-mother-\Parg{} \TsgF{}.\Gen{} //
\glft `Pada looks like her mother.' //
\endgl\xe

\pex
\a\begingl
\glpreamble \textit{Ang haruya Pulan sa Linko.} \\
	/aŋ haˈruja | ˈpulan | sa ˈliŋko/ //

\gla Ang haruya Pulan sa Linko. //
\glb ang haru-ya Pulan sa Linko //
\glc \AgtT{} beat-\TsgM{} Pulan \Parg{} Linko //
\glft `Pulan beats Linko.' //
\endgl

\a\ljudge*\begingl
\gla Ang haruya ang Pulan sa Linko. //
\glb ang haru-ya ang Pulan sa Linko //
\glc \AgtT{} beat-\TsgM{} \Aarg{} Pulan \Parg{} Linko //
\endgl
\xe

\ex\begingl
\glpreamble \textit{Ang haruya Pulan yās.}\\
	/aŋ haˈruja | ˈpulan | yās/ //

\gla Ang haru-ya Pulan yās. //
\glb ang haru-ya Pulan yās //
\glc \AgtT{} beat-\TsgM{} Pulan \TsgM{}.\Parg{} //
\glft `Pulan beats him.' //
\endgl\xe

\pex
\a\begingl
\glpreamble \textit{Ang haruya sa Linko.} \\
	/aŋ haˈruja | sa ˈliŋko/ //

\gla Ang haruya sa Linko. //
\glb ang haru-ya sa Linko //
\glc \AgtT{} beat-\TsgM{} \Parg{} Linko //
\glft `He beats Linko.' //
\endgl

\a\ljudge*\begingl
\gla Ang haruya yāng sa Linko. //
\glb ang haru-ya yāng sa Linko //
\glc \AgtT{} beat-\TsgM{} \TsgM{}.\Aarg{} \Parg{} Linko //
\endgl
\xe

\ex\begingl
\glpreamble \textit{Ang haruya yās.}\\
	/aŋ haˈruja | ˈjās/ //

\gla Ang haruya yās. //
\glb ang haru-ya yās //
\glc \AgtT{} beat-\TsgM{} \TsgM{}.\Parg{} //
\glft `He beats him.' //
\endgl\xe

\pex
\a\begingl
\glpreamble \textit{Sa haruya ang Pulan Linko.} \\
	/sa haˈruja | aŋ ˈpulan | ˈliŋko/ //

\gla Sa haruya ang Pulan Linko. //
\glb sa haru-ya ang Pulan Linko //
\glc \PatT{} beat-\TsgM{} \Aarg{} Pulan Linko //
\glft `Linko, Pulan beats (him).' //
\endgl

\a\ljudge*\begingl
\gla Sa haruya ang Pulan sa Linko. //
\glb sa haru-ya ang Pulan sa Linko //
\glc \PatT{} beat-\TsgM{} \Aarg{} Pulan \Parg{} Linko //
\endgl\xe

\pex
\a\begingl
\glpreamble \textit{Sa haruya ang Pulan Sipra.} \\
	/sa haˈruja | aŋ ˈpulan | ˈsipra/ //

\gla Sa haruya ang Pulan Sipra. //
\glb sa haru-ya ang Pulan Sipra //
\glc \PatT{} beat-\TsgM{} \Aarg{} Pulan Sipra //
\glft `Sipra, Pulan beats (her).' //
\endgl

\a\ljudge*\begingl
\gla Sa haruye ang Pulan Sipra. //
\glb sa haru-ya ang Pulan Sipra //
\glc \PatT{} beat-\TsgF{} \Aarg{} Pulan Sipra //
\endgl\xe

\pex
\a\begingl
\glpreamble \textit{Sa haruya para ang Pulan ya.} \\
	/sa haˈruja | ˈpara | aŋ ˈpulan | ja/ //

\gla Sa haruya para ang Pulan ya. //
\glb sa haru-ya para ang Pulan ya //
\glc \PatT{} beat-\TsgM{} quickly \Aarg{} Pulan \TsgM{} //
\glft `It is him Pulan quickly beats.' //
\endgl

\a\ljudge*\begingl
\gla Sa haruya para ang Pulan yās. //
\glb sa haru-ya para ang Pulan yās //
\glc \PatT{} beat-\TsgM{} quickly \Aarg{} Pulan \TsgM{}.\Parg{} //
\endgl\xe

\ex\begingl
\glpreamble \textit{Sa haruya para ya ang Pulan.}\\
	/sa haˈruja | ˈpara ja | aŋ ˈpulan/ //

\gla Sa haruya para ya ang Pulan. //
\glb sa haru-ya para ya ang Pulan //
\glc \PatT{} beat-\TsgM{} quickly \TsgM{} \Aarg{} Pulan //
\glft `It is him Pulan quickly beats.' //
\endgl\xe

\ex\begingl
\glpreamble \textit{Sa haruyāng ya.}\\
	/sa haruˈjaːŋ ja/ //

\gla Sa haruyāng ya. //
\glb sa haru-yāng ya //
\glc \PatT{} beat-\TsgM{}.\Aarg{} \TsgM{} //
\glft `It is him that he beats.' //
\endgl\xe

\ex\begingl
\glpreamble \textit{Haruyās.}\\
	/haruˈjaːs/ //

\gla Haruyās. //
\glb haru-yās //
\glc beat-\TsgM{}.\Parg{} //
\glft `He is (being) beaten.' //
\endgl\xe

\ex\begingl
\glpreamble \textit{Ang haruya Pulan sa Linko lundari.}\\
	/aŋ haˈruja | ˈpulan | sa ˈliŋko | lunˈdari/ //

\gla Ang haruya Pulan sa Linko lundari. //
\glb ang haru-ya Pulan sa Linko lunda-ri //
\glc \AgtT{} beat-\TsgM{} Pulan \Parg{} Linko shoe-\Ins{} //
\glft `Pulan beats Linko with a shoe.' //
\endgl\xe

\ex\begingl
\glpreamble \textit{Ri haruya ang Pulan sa Linko lunda.}\\
	/ri haˈruja | aŋ ˈpulan | sa ˈliŋko | ˈlunda/ //

\gla Ri haruya ang Pulan sa Linko lundari. //
\glb ri haru-ya ang Pulan sa Linko lunda-ri //
\glc \InsT{} beat-\TsgM{} \Aarg{} Pulan \Parg{} Linko shoe-\Ins{} //
\glft `It is with a/the shoe that Pulan beats Linko.' //
\endgl\xe


\ex\begingl
\glpreamble \textit{Ri haruya para ang Pulan sa Linko adanya.}\\
	/ri haˈruja | para | aŋ ˈpulan | sa ˈliŋko | aˈdanja/ //

\gla Ri haruya para ang Pulan sa Linko adanya. //
\glb ri haru-ya para ang Pulan sa Linko adanya //
\glc \InsT{} beat-\TsgM{} quickly \Aarg{} Pulan \Parg{} Linko that //
\glft `It is with that that Pulan quickly beats Linko.' //
\endgl\xe

\ex\begingl
\glpreamble \textit{Ri haruya para adanya ang Pulan sa Linko.}\\
	/ri haˈruja | para | aˈdanja | aŋ ˈpulan | sa ˈliŋko | ˈara/ //

\gla Ri haruya para adanya ang Pulan sa Linko. //
\glb ri haru-ya para adanya ang Pulan sa Linko //
\glc \InsT{} beat-\TsgM{} quickly that \Aarg{} Pulan \Parg{} Linko //
\glft `It is with that that Pulan quickly beats Linko.' //
\endgl\xe

\ex\begingl
\glpreamble \judge?\textit{Ang sarāyn ay nay va kodanya.} \\
	\textit{Ang sarāyn kodanya, yang nay vāng.} \\
	/ja saraˈnaŋ | koˈdanja || ˈjaŋ naɪ ˈvaːŋ/ //

\gla Ang sarāyn kodanya,  yang nay vāng. //
\glb Ang sara-ayn kodan-ya yang nay vāng //
\glc \AgtT{} go-\Fpl{} lake-\Loc{} \Second{}.\Aarg{} and \Fsg{}.\Aarg{} //
\glft `You and I are going to the lake.' / `We are going to the lake, you and I.' //
\endgl

\ex\begingl
\glpreamble \judge?\textit{Ya sarāyn yang nay vāng kodan.} \\
	\textit{Ya saranang kodan, yang nay vāng.} \\
	/ja saraˈnaŋ | koˈdanja || ˈjaŋ naɪ ˈvaːŋ/ //

\gla Ya saranang kodan, yang nay vāng. //
\glb ya sara-nang kodan yang nay vāng //
\glc \LocT{} go-\Fpl{}.\Aarg{} lake \Second{}.\Aarg{} and \Fsg{}.\Aarg{} //
\glft `You and I are going to the lake.' / `We are going to the lake, you and I.' //
\endgl

\ex\begingl
\glpreamble \textit{Ang koronay sitang-nyama guratanley.}\\
	/aŋ koroˈnaɪ siˈtaŋˌnjama | guratanˈleɪ/ //

\gla Ang koronay sitang-nyama guratanley. //
\glb ang koron-ay sitang-nyama guratan-ley //
\glc \AgtT{} know-\Fsg{} self-even answer-\Parg{} //
\glft `Even I know the answer.' //
\endgl\xe

\ex\begingl
\glpreamble \textit{Le koronyang sitang-nyama guratan.}\\
	/le koronˈjaŋ siˈtaŋˌnjama | guˈratan/ //

\gla Le koronyang sitang-nyama guratan. //
\glb le koron-yang sitang-nyama guratan //
\glc \PatT{} know-\Fsg{}.\Aarg{} self-even answer //
\glft `The answer, even I know (it).' //
\endgl\xe

\ex\begingl
\glpreamble \textit{Ang kece nay dayungisaye māva yanjas yena.}\\
	/aŋ ˈketʃe naɪ dajuŋiˈsaje | ˈmaːva | ˈjandʒas ˈjena/ //

\gla Ang kece nay dayungisaye māva yanjas yena. //
\glb ang ket-ye nay dayungisa-ye māva yan-ye-as yena //
\glc \AgtT{} wash-\TsgF{} and dress-\TsgF{} mother boy-\Pl{}-\Parg{} \TsgF{}.\Gen{} //
\glft `The mother washes and dresses her boys.' //
\endgl\xe

\ex\begingl
\glpreamble \textit{Ang kece nay dayungisaye tas.}\\
	/aŋ ˈketʃe naɪ dajuŋiˈsaje tas/ //

\gla Ang kece nay dayungisaye tas. //
\glb ang ket-ye nay dayungisa-ye tas //
\glc \AgtT{} wash-\TsgF{} and dress-\TsgF{} \TplM{}.\Parg{} //
\glft `She washes and dresses them.' //
\endgl\xe

\ex\begingl
\glpreamble \textit{Sa keceng nay dayungisayeng yan.}\\
	/sa keˈtʃeŋ naɪ dajuŋisaˈjeŋ jan/ //

\gla Sa keceng nay dayungisayeng yan. //
\glb sa ket-yeng nay dayungisa-yeng yan //
\glc \PatT{} wash-\TsgF{} and dress-\TsgF{} \TplM{} //
\glft `Them it is whom she washes and dresses.' //
\endgl\xe

\ex\begingl
\glpreamble \textit{Sa ilyāng koyaye nay dadangye yeyam.}\footnotemark{} \\
	/sa ilˈjaːŋ | koˈjaje naɪ daˈdaŋje | jeˈjam/ //

\gla Sa ilyāng koyaye nay dadangye yeyam. //
\glb sa il-yāng koya-ye nay dadang-ye yeyam //
\glc \PatT{} give-\TsgM{} book-\Pl{} and pen-\Pl{} \TsgF{}.\Dat{} //
\gld %
	\mbox{\begin{avm}\[ \Anim{} & + \]\end{avm}}
	{}
	\mbox{\begin{avm}\[ \Anim{} & + \]\end{avm}}
	{}
	\mbox{\begin{avm}\[ \Anim{} & – \]\end{avm}}
	//
\glft `Books and pens he gives her.' //
\endgl

\footnotetext{I think that due to the way Ayeri deals with verb agreement, it makes more sense for it overall to have nearest-conjunct agreement with regards to gender/animacy than gender resolution, though see below. Number always resolves, though.}

% IDEA:
%
% M + F = M (N attested)
% F + M = M (N attested)
%
% → people have gender-resolution to M
% → resolution to N does occur
%
% M + N = M
% N + M = N
% F + N = M (F possible)
% N + F = N
%
% → people mixed with animate things have nearest-conjunct resolution
% → if F first, resolution to M (though nearest-conjunct resolution possible)
%
% M    + INAN = M
% INAN + M    = INAN
% F    + INAN = M (F possible)
% INAN + F    = INAN
% N    + INAN = N
% INAN + N    = INAN
%
% → people mixed with inanimate things have nearest-conjunct resolution
% → if F first, resolution to M (though nearest-conjunct resolution possible)

\ex\begingl
\glpreamble \textit{Ang sobisayan Mico nay Niva sungkoranas narān.} \\
	/aŋ sobiˈsajan | ˈmitʃo naɪ ˈniva | suŋkoˈranas naˈra:n/ //

\gla Ang sobisayan Mico nay Niva sungkoranas narān. //
\glb ang sobisa-yan Mico nay Niva sungkoran-as narān //
\glc \AgtT{} study-\TplM{} Mico and Niva science-\Parg{} language //
\gld %
	\mbox{\begin{avm}\[ \Anim{} & + \]\end{avm}}
	\mbox{\begin{avm}\[
		\Anim{} & + \\
		\Gend{} & \M{} \\
		\Num{}  & \Pl{} \\
	\]\end{avm}}
	\mbox{\begin{avm}\[
		\Anim{} & + \\
		\Gend{} & \M{} \\
		\Num{}  & \Sg{} \\
	\]\end{avm}}
	{}
	\mbox{\begin{avm}\[
		\Anim{} & + \\
		\Gend{} & \F{} \\
		\Num{}  & \Sg{} \\
	\]\end{avm}}
	//
\glft `Mico and Niva study linguistics.' //
\endgl\xe

\ex\begingl
\glpreamble \textit{Ang sobisayan Niva nay Mico sungkoranas narān.} \\
	/aŋ sobiˈsajan | ˈniva naɪ ˈmitʃo | suŋkoˈranas naˈra:n/ //

\gla Ang sobisayan Niva nay Mico sungkoranas narān. //
\glb ang sobisa-yan Niva nay Mico sungkoran-as narān //
\glc \AgtT{} study-\TplM{} Niva and Mico science-\Parg{} language //
\gld %
	\mbox{\begin{avm}\[ \Anim{} & + \]\end{avm}}
	\mbox{\begin{avm}\[
		\Anim{} & + \\
		\Gend{} & \M{} \\
		\Num{}  & \Pl{} \\
	\]\end{avm}}
	\mbox{\begin{avm}\[
		\Anim{} & + \\
		\Gend{} & \F{} \\
		\Num{}  & \Sg{} \\
	\]\end{avm}}
	{}
	\mbox{\begin{avm}\[
		\Anim{} & + \\
		\Gend{} & \M{} \\
		\Num{}  & \Sg{} \\
	\]\end{avm}}
	//
\glft `Niva and Mico study linguistics.' //
\endgl\xe

\ex\begingl
\glpreamble \textit{Toryan ang Yan nay veneyang yana.} \\
	/ˈtorjan | aŋ ˈjan naɪ veneˈjang ˈjana/ //

\gla Toryan ang Yan nay veneyang yana. //
\glb tor-yan ang Yan nay veney-ang yana //
\glc sleep-\TplM{} \Aarg{} Yan and dog-\Aarg{} \TsgM{}.\Aarg{} //
\gld %
	\mbox{\begin{avm}\[
		\Anim{} & + \\
		\Gend{} & \M{} \\
		\Num{}  & \Pl{} \\
	\]\end{avm}}
	{}
	\mbox{\begin{avm}\[
		\Anim{} & + \\
		\Gend{} & \M{} \\
		\Num{}  & \Sg{} \\
	\]\end{avm}}
	{}
	\mbox{\begin{avm}\[
		\Anim{} & + \\
		\Gend{} & \N{} \\
		\Num{}  & \Sg{} \\
	\]\end{avm}}
	//
\glft `Yan and his dog are sleeping.' //
\endgl\xe

\ex\begingl
\glpreamble \textit{Toryon veneyang nay badanang.} \\
	/ˈtorjon | veneˈjang naɪ badaˈnaŋ/ //

\gla Toryon veneyang nay badanang. //
\glb tor-yon veney-ang nay badan-ang //
\glc sleep-\TplN{} dog-\Aarg{} and father-\Aarg{} //
\gld %
	\mbox{\begin{avm}\[
		\Anim{} & + \\
		\Gend{} & \N{} \\
		\Num{}  & \Pl{} \\
	\]\end{avm}}
	\mbox{\begin{avm}\[
		\Anim{} & + \\
		\Gend{} & \N{} \\
		\Num{}  & \Sg{} \\
	\]\end{avm}}
	{}
	\mbox{\begin{avm}\[
		\Anim{} & + \\
		\Gend{} & \M{} \\
		\Num{}  & \Sg{} \\
	\]\end{avm}}
	//
\glft `The dog and father are sleeping.' //
\endgl\xe

\ex\begingl
\glpreamble \textit{Toryan mavāng nay veneyang.}\footnotemark{} \\
	/ˈtorjan | maˈvaːŋ naɪ veneˈjang/ //

\gla Toryan mavāng nay veneyang. //
\glb tor-yan mava-ang nay veney-ang //
\glc sleep-\TplM{} mother-\Aarg{} and dog-\Aarg{} //
\gld %
	\mbox{\begin{avm}\[
		\Anim{} & + \\
		\Gend{} & \M{} \\
		\Num{}  & \Pl{} \\
	\]\end{avm}}
	\mbox{\begin{avm}\[
		\Anim{} & + \\
		\Gend{} & \F{} \\
		\Num{}  & \Sg{} \\
	\]\end{avm}}
	{}
	\mbox{\begin{avm}\[
		\Anim{} & + \\
		\Gend{} & \N{} \\
		\Num{}  & \Sg{} \\
	\]\end{avm}}
	//
\glft `Mother and the dog are sleeping.' //
\endgl\xe

\footnotetext{Lack of gender resolution with regular nearest-conjunct agreement ought to be possible as well.}

\ex\begingl
\glpreamble \textit{Toryon veneyang nay mavāng.} \\
	/ˈtorjon | veneˈjang naɪ maˈvaːŋ/ //

\gla Toryon veneyang nay mavāng. //
\glb tor-yon veney-ang nay mava-ang //
\glc sleep-\TplN{} dog-\Aarg{} and mother-\Aarg{} //
\gld %
	\mbox{\begin{avm}\[
		\Anim{} & + \\
		\Gend{} & \N{} \\
		\Num{}  & \Pl{} \\
	\]\end{avm}}
	\mbox{\begin{avm}\[
		\Anim{} & + \\
		\Gend{} & \N{} \\
		\Num{}  & \Sg{} \\
	\]\end{avm}}
	{}
	\mbox{\begin{avm}\[
		\Anim{} & + \\
		\Gend{} & \F{} \\
		\Num{}  & \Sg{} \\
	\]\end{avm}}
	//
\glft `The dog and mother are sleeping.' //
\endgl\xe

\ex\begingl
	\glpreamble \textit{Ang sarayan kadanya Api-Api nay ajam yana tadayen.} \\
		/aŋ saˈrajan kaˈdanja | ˈapiˈapi naɪ ˈadʒam ˈjana | taˈdajen/ //

	\gla Ang sarayan kadanya Api-Api nay ajam yana tadayen. //
	\glb ang sara-yan kadanya Api-Api nay ajam yana tadayen //
	\glc \AgtT{} go-\TplM{} together Api-Api and toy \TsgM{}.\Gen{} everywhere //
	\gld %
		\mbox{\begin{avm}\[
			\Anim{} & + \\
		\]\end{avm}}
		\mbox{\begin{avm}\[
			\Anim{} & + \\
			\Gend{} & \M{} \\
			\Num{}  & \Pl{} \\
		\]\end{avm}}
		{}
		\mbox{\begin{avm}\[
			\Anim{} & + \\
			\Gend{} & \M{} \\
			\Num{}  & \Sg{} \\
		\]\end{avm}}
		{}
		\mbox{\begin{avm}\[
			\Anim{} & – \\
			\\
			\Num{}  & \Sg{} \\
		\]\end{avm}}
		//
	\glft `Api-Api and his toy are going everywhere together.' //
\endgl\xe

\ex\begingl
	\glpreamble \textit{Eng sarāran kadanya ajam nay Api-Api tadayen.} \\
		/eŋ saˈraːran kaˈdanja | ˈadʒam naɪ ˈapiˈapi | taˈdajen/ //

	\gla Eng sarāran kadanya ajam nay Api-Api tadayen. //
	\glb Eng sara-aran kadanya ajam nay Api-Api tadayen //
	\glc \AgtTI{} go-\TplI{} together toy and Api-Api everywhere //
	\gld %
		\mbox{\begin{avm}\[
			\Anim{} & – \\
		\]\end{avm}}
		\mbox{\begin{avm}\[
			\Anim{} & – \\
			\\
			\Num{}  & \Pl{} \\
		\]\end{avm}}
		{}
		\mbox{\begin{avm}\[
			\Anim{} & – \\
			\\
			\Num{}  & \Sg{} \\
		\]\end{avm}}
		{}
		\mbox{\begin{avm}\[
			\Anim{} & + \\
			\Gend{} & \M{} \\
			\Num{}  & \Sg{} \\
		\]\end{avm}}
		//
	\glft `The toy and Api-Api are going everywhere together.' //
\endgl\xe

\ex\begingl
	\glpreamble \textit{Ang sarayan kadanya Tavi-Tavi nay ajam yena tadayen.}\footnotemark{} \\
		/aŋ saˈrajan kaˈdanja | ˈtaviˈtavi naɪ ˈadʒam ˈjena | taˈdajen/ //

	\gla Ang sarayan kadanya Tavi-Tavi nay ajam yena tadayen. //
	\glb ang sara-yan kadanya Tavi-Tavi nay ajam yena tadayen //
	\glc \AgtT{} go-\TplM{} together Tavi-Tavi and toy \TsgF{}.\Gen{} everywhere //
	\gld %
		\mbox{\begin{avm}\[
			\Anim{} & + \\
		\]\end{avm}}
		\mbox{\begin{avm}\[
			\Anim{} & + \\
			\Gend{} & \M{} \\
			\Num{}  & \Pl{} \\
		\]\end{avm}}
		{}
		\mbox{\begin{avm}\[
			\Anim{} & + \\
			\Gend{} & \F{} \\
			\Num{}  & \Sg{} \\
		\]\end{avm}}
		{}
		\mbox{\begin{avm}\[
			\Anim{} & – \\
			\\
			\Num{}  & \Sg{} \\
		\]\end{avm}}
		//
	\glft `Tavi-Tavi and her toy are going everywhere together.' //
\endgl\xe

\footnotetext{Lack of gender resolution with regular nearest-conjunct agreement ought to be possible as well.}

\ex\begingl
	\glpreamble \textit{Eng sarāran kadanya ajam nay Tavi-Tavi tadayen.} \\
		/eŋ saˈraːran kaˈdanja | ˈadʒam naɪ ˈtaviˈtavi | taˈdajen/ //

	\gla Eng sarāran kadanya ajam nay Tavi-Tavi tadayen. //
	\glb Eng sara-aran kadanya ajam nay Tavi-Tavi tadayen //
	\glc \AgtTI{} go-\TplI{} together toy and Tavi-Tavi everywhere //
	\gld %
		\mbox{\begin{avm}\[
			\Anim{} & – \\
		\]\end{avm}}
		\mbox{\begin{avm}\[
			\Anim{} & – \\
			\\
			\Num{}  & \Pl{} \\
		\]\end{avm}}
		{}
		\mbox{\begin{avm}\[
			\Anim{} & – \\
			\\
			\Num{}  & \Sg{} \\
		\]\end{avm}}
		{}
		\mbox{\begin{avm}\[
			\Anim{} & + \\
			\Gend{} & \F{} \\
			\Num{}  & \Sg{} \\
		\]\end{avm}}
		//
	\glft `The toy and Tavi-Tavi are going everywhere together.' //
\endgl\xe

\ex\begingl
\glpreamble \textit{Ang bitojon sungkoran-narān nay payutān atasas.} \\
	/aŋ biˈtodʒon | suŋˌkoranaˈraːn naɪ pajuˈtaːn | aˈtasas/ //

\gla Ang bitojon sungkoran-narān nay payutān atasas. //
\glb ang bitog-yon sungkoran-narān nay payutān atas-as //
\glc \AgtT{} tear.apart-\TplN{} science.language and mathematics brain-\Parg{} //
\gld %
	\mbox{\begin{avm}\[ \Anim{} & + \]\end{avm}}
	\mbox{\begin{avm}\[
		\Anim{} & + \\
		\Gend{} & \N{} \\
		\Num{}  & \Pl{} \\
	\]\end{avm}}
	\mbox{\begin{avm}\[
		\Anim{} & + \\
		\Gend{} & \N{} \\
		\Num{}  & \Sg{} \\
	\]\end{avm}}
	{}
	\mbox{\begin{avm}\[
		\Anim{} & – \\
		\\
		\Num{} & \Sg{} \\
	\]\end{avm}}
	//
\glft `Linguistics and math tear the brain apart.' //
\endgl\xe

\ex\begingl
\glpreamble \textit{Eng bitogaran payutān nay sungkoran-narān atasas.} \\
	/eŋ bitoˈgaran | pajuˈtaːn naɪ suŋˌkoranaˈraːn | aˈtasas/ //

\gla Eng bitogaran payutān nay sungkoran-narān atasas. //
\glb eng bitog-aran payutān nay sungkoran-narān atas-as //
\glc \AgtT{} tear.apart-\TplI{} mathematics and science.language brain-\Parg{} //
\gld %
	\mbox{\begin{avm}\[ \Anim{} & – \]\end{avm}}
	\mbox{\begin{avm}\[
		\Anim{} & – \\
		\\
		\Num{}  & \Pl{} \\
	\]\end{avm}}
	\mbox{\begin{avm}\[
		\Anim{} & – \\
		\\
		\Num{}  & \Sg{} \\
	\]\end{avm}}
	{}
	\mbox{\begin{avm}\[
		\Anim{} & + \\
		\Gend{} & \N{} \\
		\Num{}  & \Sg{} \\
	\]\end{avm}}
	//
\glft `Math and linguistics tear the brain apart.' //
\endgl\xe

\pex
\a\begingl
\glpreamble \textit{Manga nimpye ang Misan.} \\
	/ˌmaŋa ˈnimpje | aŋ ˈmisan/ //

\gla Manga nimpye ang Misan. //
\glb manga nimp-ye ang Misan //
\glc \Prog{} run-\TsgF{} \Aarg{} Misan //
\glft `Misan is running.' //
\endgl

\a\ljudge*\begingl
\gla Nimpye manga ang Misan. //
\glb Nimp-ye manga ang Misan //
\glc run-\TsgF{} \Prog{} \Aarg{} Misan //
\endgl
\xe

\ex\begingl
\glpreamble \textit{Ang manga sahaya Tikim rangya.} \\
	/aŋ ˌmaŋa saˈhaja | ˈtikim | ˈraŋja/ //

\gla Ang manga sahaya Tikim rangya. //
\glb ang manga saha-ya Tikim rang-ya //
\glc \AgtT{} \Prog{} come-\TsgM{} Tikim home-\Loc{} //
\glft `Tikim is coming home.' //
\endgl\xe

\pex
\a\begingl
\glpreamble \textit{Ang manga sahaya rangya nay (manga) nedraya Tikim.} \\
	/aŋ ˌmaŋa saˈhaja ˈraŋja naɪ (ˌmaŋa) neˈdraja | ˈtikim / //

\gla Ang manga sahaya rangya nay (manga) nedraya Tikim. //
\glb ang manga saha-ya rang-ya nay (manga) nedra-ya Tikim //
\glc \AgtT{} \Prog{} come-\TsgM{} home-\Loc{} and (\Prog{}) sit-\TsgM{} Tikim //
\glft `Tikim is coming home and sitting down.' //
\endgl

\a\ljudge?\begingl
\glpreamble \textit{Ang manga sahaya nay nedraya Tikim rangya.} \\
	/aŋ ˌmaŋa saˈhaja naɪ neˈdraja | ˈtikim | ˈraŋja/ //

\gla Ang manga sahaya nay nedraya Tikim rangya. //
\glb ang manga saha-ya nay nedra-ya Tikim rang-ya //
\glc \AgtT{} \Prog{} come-\TsgM{} and sit-\TsgM{} Tikim home-\Loc{} //
\glft `\judge?Tikim is coming and sitting at home.' //
\endgl
\xe

\ex\begingl
\glpreamble \textit{Ang manga sahaya Tikim rangya, nay nedrayāng.} \\
	/aŋ ˌmaŋa saˈhaja | ˈtikim | ˈraŋja || naɪ nedraˈjaːŋ/ //

\gla Ang manga sahaya Tikim rangya, nay nedrayāng. //
\glb ang manga saha-ya Tikim rang-ya nay nedra-yāng //
\glc \AgtT{} \Prog{} come-\TsgM{} Tikim home-\Loc{} and sit-\TsgM{}.\Aarg{} //
\glft `Tikim is coming home and sits down.' //
\endgl\xe

\ex\begingl
\glpreamble \textit{Sa ming layaye ang Misan koya.} \\
	/sa ˌmiŋ laˈjaje | aŋ ˈmisan | ˈkoja/ //

\gla Sa ming layaye ang Misan koya. //
\glb Sa ming laya-ye ang Misan koya //
\glc \PatT{} can read-\TsgF{} \Aarg{} Misan book //
\glft `The book, Misan can read (it).' //
\endgl\xe

\pex
\a\begingl
\glpreamble \textit{Ming malyya ang Tikim.} \\
	/miŋ ˈmalja | aŋ ˈtikim/ //

\gla Ming malyya ang Tikim. //
\glb ming maly-ya ang Tikim //
\glc can sing-\TsgM{} \Aarg{} Tikim //
\glft `Tikim can sing.' //
\endgl

\a\ljudge*\begingl
\gla Malyya ming ang Tikim. //
\glb maly-ya ming ang Tikim //
\glc sing-\TsgM{} can \Aarg{} Tikim //
\endgl
\xe

\pex
\a\begingl
\glpreamble \textit{Ming malyya veno ang Tikim.} \\
	/miŋ ˈmalja ˈveno | aŋ ˈtikim/ //

\gla Ming malyya veno ang Tikim. //
\glb ming maly-ya veno ang Tikim //
\glc can sing-\TsgM{} beautifully \Aarg{} Tikim //
\glft `Tikim can sing beautifully.' //
\endgl

\a\ljudge*\begingl
\gla Ming veno malyya ang Tikim. //
\glb ming veno maly-ya ang Tikim //
\glc can beautifully sing-\TsgM{} \Aarg{} Tikim //
\endgl
\xe

\ex\begingl
\glpreamble \textit{Ming malyya nay tuyaya ang Tikim.} \\
	/miŋ ˈmalja naɪ tuˈjaja | aŋ ˈtikim / //

\gla Ming malyya nay tuyaya ang Tikim. //
\glb ming maly-ya nay tuya-ya ang Tikim //
\glc can sing-\TsgM{} and dance-\TsgM{} \Aarg{} Tikim //
\glft `Tikim can sing and dance.' //
\endgl\xe

\ex\begingl
\glpreamble \textit{Ming malyya ang Tikim, naynay tuyayāng.} \\
	/miŋ ˈmalja | aŋ ˈtikim || naɪˈnaɪ tujaˈjaːŋ/ //

\gla Ming malyya ang Tikim, naynay tuyayāng. //
\glb ming maly-ya ang Tikim naynay tuya-yāng //
\glc can sing-\TsgM{} \Aarg{} Tikim and.also dance-\TsgM{}.\Aarg{} //
\glft `Tikim can sing and also dances.' //
\endgl\xe

\ex\begingl
\glpreamble \textit{Mya manga nimpongye Sipra edauyi.} \\
	/mja ˌmaŋa nimˈpoŋje | ˈsipra | eˈdaui/ //

\gla Mya manga nimpongye Sipra edauyi. //
\glb mya manga nimp-ong-ye Sipra edauyi //
\glc be.supposed.to \Prog{} run-\Irr{}-\TsgF{} Sipra now //
\glft `Sipra ought to be running now.' //
\endgl\xe

\pex
\a\begingl
\glpreamble Le mya ming sidegongya badanang ajam. \\
	/le mja miŋ sideˈgoŋja | badaˈnaŋ | ˈadʒam/ //

\gla Le mya ming sidegongya badanang ajam. //
\glb le mya ming sideg-ong-ya badan-ang ajam //
\glc \PatTI{} be.supposed.to can repair-\Irr{}-\TsgM{} father-\Aarg{} toy //
\glft `The toy, father should be able to repair (it).' //
\endgl

\a\ljudge*\begingl
\glpreamble Le myongya mingyam sidejam badanang ajam. \\
	/le ˈmjongja | ˈmiŋjam | siˈdedʒam | badaˈnaŋ | ˈadʒam/ //

\gla Le myongya mingyam sidejam badanang ajam. //
\glb le mya-ong-ya ming-yam sideg-yam badan-ang ajam //
\glc \PatTI{} be.supposed.to-\Irr{}-\TsgM{} can-\Ptcp{} repair-\Ptcp{} father-\Aarg{} toy //
\endgl\xe

\ex\begingl
\glpreamble Ang mya ming sidegongya nay la-lataya adaley. \\
	/aŋ mja miŋ sideˈgoŋja naɪ lalaˈtaja | adaˈleɪ/ //

\gla Ang mya ming sidegongya nay la-lataya adaley //
\glb ang mya ming sideg-ong-ya nay la\til{}lata-ya ada-ley //
\glc \AgtT{} be.supposed.to can repair-\Irr{}-\TsgM{} and \Iter{}\til{}sell-\TsgM{} that-\PargI{} //
\glft `He should be able to repair and resell it.' //
\endgl\xe

\pex
\a\begingl
\glpreamble Ang ming sideja adaley, nay da-myongyāng. \\
	/aŋ miŋ siˈdedʒa | adaˈleɪ || naɪ da-mjoŋˈjaːŋ/ //

\gla Ang ming sideja adaley, nay da-myongyāng. //
\glb ang ming sideg-ya ada-ley nay da-mya-ong-yāng //
\glc \AgtT{} can repair-\TsgM{} that-\PargI{} and so-be.supposed.to-\Irr{}-\TsgM{}.\Aarg{} //
\glft `He can, and should, repair it.' //
\endgl

\a\ljudge?\begingl
\gla Ang ming nay mya sidegongya adaley. //
\glb ang ming nay mya sideg-ong-ya adaley //
\glc \AgtT{} can and be.supposed.to reapir-\Irr{}-\TsgM{} that-\PargI{} //
\endgl
\xe

\ex\begingl
	\gla Ang da-pinyaya Yan sa Pila. //
	\glb Ang da-pinya-ya Yan sa Pila //
	\glc \AgtT{} such-ask-\TsgM{} Yan \Parg{} Pila //
	\glft `Yan asks Pila to (do so).' //
\endgl\xe

\ex\begingl
	\gla Ang da-pinyaya nay hisaya Yan sa Pila. //
	\glb Ang da-pinya-ya nay hisa-ya Yan sa Pila //
	\glc \AgtT{} such-ask-\TsgM{} and beg-\TsgM{} Yan \Parg{} Pila //
	\glft `Yan asks and begs Pila to (do so).' //
\endgl\xe

\ex\begingl
	\gla Ang manga da-pi-pinyaya nay hi-hisaya Yan sa Pila. //
	\glb Ang manga da-\Iter{}\til{}pinya-ya nay \Iter{}\til{}hisa-ya Yan sa Pila //
	\glc \AgtT{} \Prog{} such-ask-\TsgM{} and beg-\TsgM{} Yan \Parg{} Pila //
	\glft `Yan keeps asking and begging Pila to (do so).' //
\endgl\xe

\ex\begingl
	\gla Ang ming da-pinyongya yes. //
	\glb Ang ming da-pinya-ong-ya yes //
	\glc \AgtT{} can such-ask-\Irr{}-\TsgM{} \TsgF{}.\Parg{} //
	\glft `He could ask her to (do so).' //
\endgl\xe

\ex\begingl
	\gla Ang ming manga da-pinyongya yes. //
	\glb Ang ming manga da-pinya-ong-ya yes //
	\glc \AgtT{} can \Prog{} such-ask-\Irr{}-\TsgM{} \TsgF{}.\Parg{} //
	\glft `He could be asking her to (do so).' //
\endgl\xe

...

\ex\begingl
\glpreamble \textit{Sitang-keca ang Yan.}\\
	/sitaŋˈketʃa | aŋ ˈjan/ //

\gla Sitang-keca ang Yan. //
\glb sitang-ket-ya ang Yan //
\glc self-wash-\TsgM{} \Aarg{} Yan //
\glft `Yan washes himself.' //
\endgl\xe

\ex\begingl
\glpreamble \textit{Sitang-kecan ang Yan nay netuang yana.}\\
	/sitaŋˈketʃan | aŋ ˈjan naɪ netuˈaŋ ˈjana/ //

\gla Sitang-kecan ang Yan nay netuang yana. //
\glb sitang-ket-yan ang Yan nay netu-ang yana //
\glc self-wash-\TplM{} \Aarg{} Yan and brother-\Aarg{} \TsgM{}.\Gen{} //
\glft `Yan and his brother wash themselves.' //
\endgl\xe

\ex\begingl
\glpreamble \textit{Sitang-keca nay (sitang-?)dayungisaya ang Yan.}\\
	/siˈtaŋ ˈketʃa naɪ dajuŋiˈsaja | aŋ ˈjan/ //

\gla Sitang-keca nay dayungisaya ang Yan. //
\glb sitang-ket-ya nay dayungisa-ya ang Yan //
\glc self-wash-\TsgM{} and dress-\TsgM{} \Aarg{} Yan //
\glft `Yan washes and dresses himself.' //
\endgl\xe

\pagebreak

\begin{multicols}{2}
\printglossary[style=mysuper,type=\leipzigtype]
\end{multicols}

\end{document}
