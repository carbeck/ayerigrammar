% arara: xelatex
% arara: biber
% arara: makeglossaries
% arara: xelatex

\documentclass[
	a4paper, % 9"×12" (folio) = oldpaper
	12pt,
	openany,
	twoside,
	final,
	oldfontcommands, % Compatibility with expex, which still uses \rm
]{memoir}

\author{Carsten Becker}
\title{A Grammar of Ayeri: Documenting a Fictional Language}
\date{\today}

% Layout for title page, cf. titlepages package
\newlength\drop
\makeatletter
\newcommand*{\titleS}{\begingroup% Scripts, T&H p 151
\drop = 0.1\textheight
\centering
\vspace*{\drop}
{\Huge A Grammar of Ayeri}\\[\baselineskip]
{\Large\scshape Documenting a Fictional Language}\\[\baselineskip]
{\large\itshape by Carsten Becker}\\[\baselineskip]
\vfill
\rule{0.4\textwidth}{0.4pt}\\[\baselineskip]
{\large\itshape Benung. The Ayeri Language Resource}\par
\vspace*{\drop}
\endgroup}
\makeatother

% Layout for running page titles
\nouppercaseheads

% Layout for chapter headers
\chapterstyle{southall}
\renewcommand*{\chaptitlefont}{\huge\sffamily\raggedright}

% Disemulate setspace
% \DisemulatePackage{setspace}
% \usepackage{setspace}
% \onehalfspacing

% Creative Commons icons
\usepackage{ccicons}

% Use Python
% \usepackage[gobble=auto]{pythontex}

% Load fonts for XeTeX, including support for Unicode etc.
\usepackage{fontspec}
\usepackage{xunicode}
\usepackage{xltxtra}

% Handle language and quotation marks
\usepackage{polyglossia}
\setdefaultlanguage[variant=american]{english}
\usepackage{csquotes} % Put quotations in \enquote{}!
\SetBlockEnvironment{quotation}
\renewcommand*{\mkccitation}[1]{ (#1)}

% Date and time
\usepackage[
	useregional,
	%style=mdyyyy,
]{datetime2}

% Extended formatting of lists
\usepackage{enumitem}
\newlist{glossdefs}{itemize}{1}
\setlist[glossdefs]{nosep, leftmargin=3em, labelwidth=2.5em, align=left}
\setlist[itemize]{noitemsep}

% Make multiple columns available in single-column document
\usepackage{multicol}

% Make text colors and color names available
\usepackage[xetex]{xcolor}

% Set main fonts
%\usepackage[config=mt-Junicode]{microtype}

\newfontfamily{\Tagati}[
	Renderer=Graphite,
	Scale=0.9,
	BoldFont={* Italic},
	HyphenChar=·,
]{Tagati Book G}

% \setmainfont{EB Garamond}[
% 	Ligatures={TeX,NoContextual},
% 	Numbers=Lowercase,
% 	BoldFont={Garamond Premier Pro Semibold},
% ]

\setmainfont{Junicode}[
	Ligatures={TeX,NoContextual},
	Numbers=Lowercase,
]

\setsansfont{Fira Sans}[
	Ligatures=TeX,
	Numbers=Lowercase,
	Scale=MatchLowercase,
	BoldFont={* SemiBold},
]

\setmonofont{Fira Mono}[
	Ligatures=TeX,
	Scale=MatchLowercase,
]

\newfontfamily{\OpenSansCond}[
	Ligatures=TeX,
	Numbers=Lowercase,
	Scale=MatchUppercase,
	BoldFont={Open Sans Condensed Bold},
]{Open Sans Condensed Light}

% Set fonts for section headers
\setsecheadstyle{\sffamily\Large\raggedright}
\setsubsecheadstyle{\sffamily\bfseries\normalsize\raggedright}

% Nicer footnotes
\usepackage[bottom,hang,norule]{footmisc}
\setlength{\footnotesep}{0.75\baselineskip}

% Load BibLaTeX (using Biber), configure citation styles
\usepackage[
	authordate-trad,
	backend=biber,
	safeinputenc,
	natbib,
]{biblatex-chicago}
\addbibresource{bibliography.bib}

% To make \textcite look like "Doe (2014: 213)"
\renewcommand*{\postnotedelim}{\addcolon\addspace}
\DeclareFieldFormat{postnote}{#1}
\DeclareFieldFormat{multipostnote}{#1}

% Clickable links in footnotes, TOC, etc.
\usepackage[
	xetex,
	bookmarks=true,
	colorlinks=false,
	linktoc=section,
	hidelinks,
	pdfusetitle,
]{hyperref}

% Formatting of URLs
\usepackage{url}
\def\UrlFont{\normalfont\itshape}

% Ability to include graphics and dealing with footnotes in descriptions
\usepackage{graphicx}
\usepackage[font={small,sf},labelfont={small,sf},format=plain]{caption}
\usepackage{subcaption}
\usepackage{wrapfig}
\setlength{\columnsep}{2\baselineskip}
\usepackage{tikz}

% Things for tables
\usepackage{longtable}
\usepackage{tabu}
\usepackage{booktabs}
\usepackage{rotating}

% Custom column types
\newcolumntype{B}{>{\bfseries}X}
\newcolumntype{I}{>{\itshape}X}
\newcolumntype{H}{>{\bfseries\footnotesize}X}

% Table header font
\newcommand{\tableheaderfont}{\rowfont{\bfseries\footnotesize}}

% Formatting of table of glossing abbreviations from Leipzig package manual
\usepackage[acronym,nomain,nonumberlist,nopostdot]{glossaries}
\usepackage{glossary-inline}%

\newglossarystyle{mysuper}{%
	\glossarystyle{super}% based on super
	\renewenvironment{theglossary}{%
		\begin{glossdefs}%
	}{%
		\end{glossdefs}%
	}%
	\renewcommand*{\glossaryheader}{}%
	\renewcommand*{\glsgroupheading}[1]{}%
	\renewcommand*{\glossaryentryfield}[5]{%
		\item[\glsentryitem{##1}\glstarget{##1}{##2}]
		\makefirstuc{##3}\glspostdescription{}
	}%
	\renewcommand*{\glsgroupskip}{}%
}%

% Formatting of glosses
\usepackage{expex}
\usepackage{leipzig}

\newleipzig{AgtT}{at}{agent topic}
\newleipzig{PatT}{pt}{patient topic}
\newleipzig{DatT}{datt}{dative topic}
\newleipzig{GenT}{gent}{genitive topic}
\newleipzig{LocT}{loct}{locative topic}
\newleipzig{InsT}{inst}{instrumental topic}
\newleipzig{CauT}{caut}{causative topic}
\newleipzig{An}{an}{animate}
\newleipzig{Inan}{inan}{inanimate}
\newleipzig{Hab}{hab}{habitative}
\newleipzig{Ayr}{ayr}{Ayeri}
\newleipzig{NPst}{npst}{near past}
\newleipzig{RPst}{rpst}{remote past}
\newleipzig{NFut}{nfut}{near future}
\newleipzig{RFut}{rfut}{remote future}
\newleipzig{Agtz}{agtz}{agentizer}

\newleipzig{Nn}{nn}{Noun}
\newleipzig{Vb}{vb}{Verb}
\newleipzig{Adj}{adj}{Adjective}

\makeglossaries

\newcommand{\AargI}{{\Aarg}.{\Inan}}
\newcommand{\PargI}{{\Parg}.{\Inan}}
\newcommand{\AgtTI}{{\AgtT}.{\Inan}}
\newcommand{\PatTI}{{\PatT}.{\Inan}}

\newcommand{\TsgM}{{\Tsg}.{\M}}
\newcommand{\TsgF}{{\Tsg}.{\F}}
\newcommand{\TsgN}{{\Tsg}.{\N}}
\newcommand{\TsgI}{{\Tsg}.{\Inan}}
\newcommand{\TplM}{{\Tpl}.{\M}}
\newcommand{\TplF}{{\Tpl}.{\F}}
\newcommand{\TplN}{{\Tpl}.{\N}}
\newcommand{\TplI}{{\Tpl}.{\Inan}}

% Smaller font in block quotes
\usepackage{relsize}
\AtBeginEnvironment{quote}{\noindent\smaller}
\AtBeginEnvironment{quotation}{\smaller}

% Macros
\newcommand{\fw}[1]{\textit{#1}} % Foreign Word
\newcommand{\tit}[1]{\textit{#1}} % Title of a work
\newcommand{\q}[1]{\enquote{#1}} % Context-aware quotation
\newcommand{\qq}[1]{\enquote*{#1}} % Explicit sublevel quotation
\newcommand{\tsup}[1]{\textsuperscript{#1}} % Superscript
\newcommand{\tsub}[1]{\textsubscript{#1}} % Subscript
\newcommand{\markyellow}[1]{\colorbox{yellow}{#1}} % Yellow highlighter
\newcommand{\ques}{\fakesuperscript{?}} % raised question mark
\newcommand{\orth}[1]{⟨#1⟩}  % Orthography brackets
\newcommand{\tc}[1]{\trailingcitation{#1}} % \trailingcitation from expex in examples
\newcommand{\rc}[1]{\rightcomment{#1}} % \rightcomment from expex in examples
\newcommand{\pct}{{\footnotesize \%}} % Percent sign with half space before it

\newcommand{\ayr}[1]{\smash{{\Tagati #1}}} % Plain Ayeri orthography
\newcommand{\rayr}[2]{\smash{{\Tagati #1}} \emph{#2}} % Ayeri orthography + *r*omanization
\newcommand{\tayr}[2]{#1 `#2'} % Romanization + *t*ranslation
\newcommand{\xayr}[3]{\smash{\Tagati #1} \emph{#2} `#3'} % Ayeri orthography + romanization + translation

% blah
\usepackage{lipsum}

%%%%%%%%%%%%%%%%%%%%%%%%%%%%%%%%%%%%%%%%%%%%%%%%%%%%%%%%%%%%%%%%%%%%%%%%%%%%%%%%

\begin{document}

%% HALF TITLE %%%%%%%%%%%%%%%%%%%%%%%%%%%%%%%%%%%%%%%%%%%%%%%%%%%%%%%%%%%%%%%%%%

\begin{titlingpage}
\begin{center}
{\Huge A Grammar of Ayeri}
\end{center}
\end{titlingpage}

%% TITLE PAGE %%%%%%%%%%%%%%%%%%%%%%%%%%%%%%%%%%%%%%%%%%%%%%%%%%%%%%%%%%%%%%%%%%

\begin{titlingpage}
\titleS
\clearpage
~\vfill
{\setlength\parindent{0pt}
© Carsten Becker, 2018. Some rights reserved.\\
Published under \textsc{cc-by-sa} 4.0 license.\\
Last edited: \today{}.\\

Set in Junicode and {\sffamily Fira Sans} with \XeTeX{}.\\

Ayeri is a fictional language. As such, it is not related to any naturally
existing languages, living or dead. No real-world languages have been used
specifically as sources of vocabulary. Ayeri is also not derived from any
specific real-world language family by means of sound changes. Due to Ayeri's
sound and spelling aesthetic being inspired by Austronesian languages such as
Malay, Indonesian, or Tagalog, however, occasional overlaps with words existing
in these languages may happen, but only accidentally so.\\

Links given in references are provided in good faith. Even though care has been
taken to use persistent \textsc{url}s wherever possible, it cannot be
guaranteed that any of these links will work indefinitely or show the exact
same content as on the date of access.\\

\begin{tabular}{@{} c @{\enspace} l}
\faicon{globe}
& \href{https://ayeri.de}{https://ayeri.de}\\
\faicon{cogs}
& \href{https://github.com/carbeck/ayerigrammar}
	{https://github.com/carbeck/ayerigrammar/}\\
\faicon{balance-scale}
& \href{https://creativecommons.org/licenses/by-sa/4.0/}%
	{https://creativecommons.org/licenses/by-sa/4.0/}%\\
\end{tabular}
}

\end{titlingpage}

%% FRONT MATTER %%%%%%%%%%%%%%%%%%%%%%%%%%%%%%%%%%%%%%%%%%%%%%%%%%%%%%%%%%%%%%%%

\frontmatter
\tableofcontents
\clearpage
\listoffigures
\clearpage
\listoftables
\chapter{Glossing Abbreviations}
\begin{multicols}{2}
\printglossary[style=mysuper,type=\leipzigtype]
\end{multicols}
\cleartorecto

%% MAIN MATTER %%%%%%%%%%%%%%%%%%%%%%%%%%%%%%%%%%%%%%%%%%%%%%%%%%%%%%%%%%%%%%%%%

\mainmatter

% kate: word-wrap true;

\chapter{Preface}

This is my latest attempt to write a grammar of Ayeri, a fictional language 
which I have been developing since December 2003. Getting to work on grammar 
writing again was triggered by a growing dissatisfaction with not having a 
central place of documentation, when the first thing people look for on my 
website is often the grammar, incomplete as well as partially inaccurate and 
outdated as it may be. In addition to that, there was a seminar on fictional 
languages at the University of Tübingen, Germany, in the summer semester of 
2016 \autocite{buch2016ss}. Ayeri was one of the languages that was chosen for 
students to explore and evaluate.

The student group who worked on Ayeri came to the conclusion that its
documentation is severely lacking in the description of basic elements and
assumptions, since whole chapters of the grammar had been missing to date
(\cite[12]{boga2016}).\footnote{\xayr{kuːtnsF/IknF}{Kutānas-ikan}{thanks a lot}
to Bella Boga, Madita Breuninger, Thora Daneyko, and Martina Stama-Kirr for
their hard work on making sense of my published materials in spite of
information being scattered all over the place, as well as their providing me
with the presentation concluding their group work.} This is to say that
previous attempts of writing a full-fledged grammar of Ayeri have been
incomplete due to loss of enthusiasm and creeping neglect.

Although the \tit{Ayeri Grammar} has so far been lying dormant for five years,
I have written a whole number of blog articles detailing various grammatical
issues \autocite[Blog]{benung}. These articles have been taken into
consideration here. This grammar writing attempt is thus not only a transferral
to a different typesetting system, but constitutes an extension to previous
formal documentation as well.

\begin{flushright}\itshape\footnotesize
Carsten Becker\\
\DTMsavenow{currentdate}
Marburg, \DTMenglishmonthname{\DTMfetchmonth{currentdate}} \the\year
\end{flushright}
\cleartorecto

\setcounter{chapter}{-1}
\chapter{Introduction}

In December 2003, the idea for a new fictional language was born, an idea that 
turned out to stick with me for over 10 years now.\footnote{A lot of the text 
here is taken from the blog article, ``\citetitle{benung:happybirthday}'' 
\parencite{benung:happybirthday}.} At that time, my seventeen years old self 
was still fairly new to this whole making-up languages business, read things 
about linguistics here and there, and was not shy to ask questions about 
terminology (and, looking at old mails, a little impertinently teenager-like 
so), for example on \tit{Conlang-L} and the \tit{Zompist Bulletin Board}. One 
thing seemed to catch my interest especially: syntactic alignments other than 
the \Nom{}/\Acc{} of the few languages I was familiar with, that is, German, 
English, and French. Apparently this curiosity was big enough for me to grow 
bored with my second fictional language, Daléian (declared `quite complete' 
after maybe half a year of work or so), and to start something new from scratch 
in order to put newly acquired knowledge to test.

I had read about `trigger languages' on \tit{Conlang-L} and wanted to try my 
hands on making my own. I cannot remember how long it took me to come up with a 
first draft of an Ayeri grammar, however, I do remember having been told that a 
good language cannot be made in a summer. Of course, I still did not really 
know what I was doing then, even though I thought I had understood things and 
authoritatively declared ``this is how it works'' in my first grammar draft 
when things sometimes really do not work that way. But at least an interest had 
been whetted.

In order to illustrate the various stages from the beginnings to current Ayeri,
I went through some old backups contemporary with the very early days. 
Here is a sentence from the oldest existing document related to it, titled 
``Draft of \& Ideas for my 3rd Conlang''---the file's last-changed date is 
\DTMdate{2003-12-14}, though I remember having started work on Ayeri in early 
December. I added glossing for convenience and according to what I could 
reconstruct from the notes. This uses vocabulary and grammatical markers just 
made up on the spot and for illustrative purposes; little of it actually 
managed to make it into actual work on Ayeri:

\ex\begingl
	\gla Ayevhoi agiaemaesim coyaielieðamavir vhaieloyaŋaiye. //
	\glb Ay-evhoi agia-ema-esim coyai-el-i-eðam-avir vhai-el-o-yaŋa-iye //
	\glc \Tsg{}.\An{}-\Sbj{} read-\Vb{}-\Sbj{}.\An{} book-\Nn{}-\An{}-\Indf{}-\Parg{} bed-\Nn{}-\Inan{}-on-\Loc{} //
	\glft `He reads a book on the bed.' //
\endgl\xe

According to the grammar draft of \DTMdate{2004-09-05}, this would have already 
changed to:

\ex\begingl
	\gla Ang layaiyạin mecoyalei ling *pinamea. //
	\glb Ang laya-iy-a-in me-coya-lei ling *pinam-ea //
	\glc \Aarg{}.\Sbj{} read-\Tsg{}.\An{}₁-a₁-\Sbj{} \Indf{}.\Inan{}-book-\PargI{} top.of bed-\Loc{} //
	\glft `He reads a book on the bed.' //
\endgl\xe

A word for `bed'---\rayr{pinmF}{pinam}---was only (re-)introduced on 
\DTMdate{2008-10-24}. In the current state of Ayeri, I would translate the 
sentence as follows:

\ex\begingl
	\gla Ang layaya koyaley ling pinamya. //
	\glb Ang laya=ya.Ø koya-ley ling pinam-ya //
	\glc \AgtT{} read=\Tsg{}.\M{}.\Top{} book-\PargI{} top.of bed-\Loc{} //
	\glft `He reads a book on a/the bed.' //
\endgl\xe

As you can see, quite a bit of morphology got lost already early on, especially 
the overt part-of-speech marking (!) and animacy marking on nouns. Also, 
prepositions were just incorporated into a noun complex as suffixes apparently. 
Gender was originally only divided into animate and inanimate, but I changed 
that at some point because only being familiar really with European languages, 
it felt awkward to me not to be able to explicitly distinguish `he', `she', and 
`it'.

A feature that also got lost is the assignment of thematic vowels in personal 
pronouns to 3rd-person referents: originally, every 3rd-person referent newly 
introduced into discourse would be assigned one of /a e i o u/ to disambiguate, 
and there was even a morpheme to mark that the speaker wanted to dissolve the 
association. Constituent order was theoretically variable at first, but I 
preferred \textsc{svo/avp} due to familiarity with that. Later on, however, I 
settled on \textsc{vso/vap}. Also, I had no idea about what was called 
``trigger morphology'' on \tit{Conlang-L} for the longest time---essentially, 
this referred to the Austronesian, or Philippine, aligment. I am not claiming 
that I know all about it now, just that due to reading up on the topic, I have 
a slightly more informed understanding now. Orthography changed as well over 
the years, so \orth{c} in the early examples encodes the /k/ sound, not /tʃ/ as 
it does today; diphthongs are spelled as \orth{Vi} instead of modern \orth{Vy}.

What was definitely beneficial for the development of Ayeri was the ever 
increasing amount of linguistics materials available online and my entering 
university (to study literature) in 2009, where I learned how to do research 
and also had a lot of interesting books available at the library.

One of the things people regularly compliment me on is Ayeri's script---note, 
however, that Tahano Hikamu was not the first one I came up with for Ayeri. 
Apparently, I had already been fascinated with the look of Javanese/Balinese 
writing early on; \autoref{fig:ayeriscript2004} shows a draft dated 
\DTMdate{2004-02-09}. However, since the letter shapes in this draft looked so 
confusingly alike that I could never memorize them. About a year later, I came 
up with the draft in \autoref{fig:th2005}. What is titled ``Another 
Experimental Script'' here is what would later turn into Tahano Hikamu, Ayeri's 
`native' script. According to the notes in my fictional language ring binder, 
the script looked much the same as today about a year from then, but things 
have only been mostly stable since about 2008.

\begin{figure}[tph]
	\centering
	\caption[First design for an Ayeri script]{First design for an Ayeri script (\DTMdate{2004-02-09})}
	\includegraphics[width=\textwidth, keepaspectratio]{images/ayeriscript2004-300dpi-bw.png}
	\label{fig:ayeriscript2004}
\end{figure}

\begin{figure}[tph]
	\centering
	\caption[First draft for Tahano Hikamu]{First draft for Tahano Hikamu (\DTMdate{2005-03-23})}
	\includegraphics[width=\textwidth, keepaspectratio]{images/th2005-300dpi-bw.png}
	\label{fig:th2005}
\end{figure}

Another important date in the history of Ayeri is when I decided to set up an 
improved website for Ayeri that would include a blog. The idea was that this 
way, I could more freely write on whatever detail I currently worked on in 
Ayeri, outside of the constraints of the grammar. Thus, \tit{Benung.~The Ayeri 
Language Resource} launched on \DTMdate{2011-03-01}. Being able to write short 
articles, however, probably also led to neglecting work on the actual formal 
reference grammar, which had been lying dormant from January 2011 on. This was 
always on the premise that I would eventually include the information form 
blog articles in the grammar. However, juggling such a big document had always 
felt daunting, so I let laziness take the better part of me eventually.\footnote{
Let me add to my defense, however, that I also worked on my B.A. thesis in 2013 
and my M.A. thesis in 2016, which required several months of preparation each 
and thus left me largely unable to work much on Ayeri.} This renewed attempt at 
documentation has been started with the intention to right those wrongs.

\cleartorecto

\chapter{Phoneme Inventory and Phonotactics}

This chapter will present charts depicting the phoneme inventory of Ayeri, 
give an analysis of the phonotactics of Ayeri's dictionary entries and also 
describe stress patterns.

\section{Phoneme Inventory}

\subsection{Consonants}

At 17 consonants -- /ʔ/ and /w/ only occur marginally --, Ayeri has a fairly 
mid-sized inventory. \autoref{tab:consonants} shows the full chart. The 
affricates /tʃ/ and /dʒ/ are allophones of /tj, kj/ and /dj, gj/, respectively.

\begin{sidewaystable}
\caption{Consonant inventory}
\begin{tabu} to \textwidth {H[2l] X[c] X[c] X[c] X[c] X[c] X[c] X[c] X[c] X[c] X[c] X[c] X[c]}
\toprule\tableheaderfont
	%
	& \multicolumn2{c}{Bilabials}
	& \multicolumn2{c}{Labiodentals}
	& \multicolumn2{c}{Alveolars}
	& \multicolumn2{c}{Palatals}
	& \multicolumn2{c}{Velars}
	& \multicolumn2{c}{Glottals}
	\\

\midrule

Plosives
	& p    & b         	% Bilabials
	&      &           	% Labiodentals
	& t    & d         	% Alveolars
	&      &           	% Palatals
	& k    & ɡ \orth{g}	% Velars
	& (ʔ)  &           	% Glottals
	\\

\midrule

Affricates
	&             &            	% Bilabials
	&             &            	% Labiodentals
	& tʃ \orth{c} & dʒ \orth{j}	% Alveolars
	&             &            	% Palatals
	&             &            	% Glottals
	\\

\midrule

Nasals
	&   & m          	% Bilabials
	&   &            	% Labiodentals
	&   & n          	% Alveolars
	&   &            	% Palatals
	&   & ŋ \orth{ng}	% Velars
	&   &            	% Glottals
	\\

\midrule

Fricatives
	&   &  	% Bilabials
	&   & v	% Labiodentals
	& s &  	% Alveolars
	&   &  	% Palatals
	&   &  	% Velars
	& h &  	% Glottals
	\\

\midrule

Taps/Flaps
	&   &  	% Bilabials
	&   &  	% Labiodentals
	&   & r	% Alveolars
	&   &  	% Palatals
	&   &  	% Velars
	&   &  	% Glottals
	\\

\midrule

Approximants
	&   & (w)       	% Bilabials
	&   &           	% Labiodentals
	&   & l         	% Alveolars
	&   & j \orth{y}	% Palatals
	&   &           	% Velars
	&   &           	% Glottals
	\\

\bottomrule
\end{tabu}
\label{tab:consonants}
\end{sidewaystable}

\subsection{Vowels}

Ayeri has a very basic five-vowel system:

\begin{table}\centering
\label{tab:vowels}
\caption{Vowel inventory}
\begin{tabu} to .5\textwidth{H X[c] X[c] X[c]}
\toprule\tableheaderfont

	& Front
	& Center
	& Back
	\\

\toprule

High
	& i
	&
	& u
	\\

Mid
	& e
	& (ə)
	& o
	\\

Back
	&
	& a
	&
	\\

\bottomrule
\end{tabu}
\end{table}

The lax vowels [ɪ ɛ ɔ ʊ] occur as allophones of their tense counterparts 
[i e o u] in closed syllables, for example:

\pex
	\a \rayr{miNF}{m\textbf{ing}} [mɪŋ] `can, be able',
	\a \rayr{EnFy}{\textbf{en}ya} [ˈɛnja] `everyone',
	\a \rayr{AgonF}{ag\textbf{on}} [ˈaɡɔn] `outer, foreign', and
	\a \rayr{pkurF}{pak\textbf{ur}} [ˈpakʊr] `ill, sick'.
\xe

/ə/ is a marginal phoneme and only occurs in the tense prefixes 
\xayr{k/}{kə-}{\NPst{}}, \xayr{m/}{mə-}{\Pst{}}, \xayr{v/}{və-}{\RPst{}}, as 
well as in the prefix \xayr{me/}{mə-}{some, whichever}. Otherwise, [ə] occurs 
as an allophone of /e/ in final unstressed position, e.g. in the word 
\rayr{mine}{min\textbf{e}} [ˈminə] `affair, matter, issue'.

Ayeri also possesses a number of diphthongs, these are: /aɪ aːɪ eɪ ɔɪ ʊɪ aʊ/.
Furthermore, the vowels [i e o u] may be long: [aː eː iː oː]. This is 
lexicalized in a few words, for example: \xayr{niis}{nīsa}{wanted}, 
\xayr{psiis}{pasīsa}{interesting}, \xayr{AreenF}{arēn}{anyway, however}, 
\xayr{leer}{lēra}{whore}, and \xayr{noonF}{nōn}{wish}. Otherwise, long vowels 
result from two same vowels next to each other, for instance:

\ex \xayr{AgY/}{aja-}{play} + \xayr{/AnF}{-an}{\Nmlz{}} → \xayr{AgYaanF}{ajān}{game, play}. \xe

Morphophonologically, long vowels also occur in double-marked relative pronouns 
where the agreement marker for the relative clause's head has been omitted,
for instance, \xayr{sinaa}{sinā}{of which, about which}, as in the following 
example:

\ex\begingl
	\gla Le turayāng taman sinā ang ningay tamala vās. //
	\glb Le tura-yāng taman-Ø si-Ø-na ang ning=ay.Ø tamala vās //
	\glc \PatTI{} send=\Tsg{}.\M{}.\Aarg{} letter-\Top{} \Rel{}-\PatTI{}-\Gen{} \AgtT{} tell=\Fsg{}.\Top{} yesterday \Ssg{}.\Parg{} //
	\glft `The letter which I told you about yesterday, he sent it.' //
\endgl\xe

This is to disambiguate it from the plain genitive-marked relative pronoun 
\xayr{sin}{sina}{which.\Gen{}}:

\ex\begingl
	\gla tamanang ledanena nā sina koronvāng //
	\glb taman-ang ledan-ena nā si-na koron-vāng //
	\glc letter-\Aarg{} friend-\Gen{} \Fsg.\Gen{} \Rel{}-\Gen{} know=\Ssg{}.\Aarg{} //
	\glft `the letter of my friend which you know' //
\endgl\xe

\section{Phonotactics}

For the purpose of statistical analyis, only the following parts of speech have 
been considered: nouns, adjectives, adverbs, pronouns, adpositions, 
conjunctions, and numerals. Verbs have been notably ignored as verb stems alone 
do not constitute independent words -- they are always inflected in some way, so 
that they may end in consonants or consonant clusters that independent words 
cannot end in. This also has repercussions for syllabification and stress, 
which depend on the inflection of the verb stem:

\begin{table}[h]
\label{ex:verbsyll}
\caption{Syllabification of inflected verbs}
\begin{tabu} to \linewidth {X[2l] X[3c] X[3c] X[3c]}
\toprule\tableheaderfont
Suffix
	& \emph{ca-} `love'
	& \emph{gum-} `work'
	& \emph{babr-} `mumble'
	\\

\toprule

\emph{-ay} (\Fsg{})
	& cā́y
	& gu.máy
	& ba.bráy
	\\

\emph{-va} (\Ssg{})
	& cá.va
	& gúm.va
	& ba.brá.va
	\\

\emph{-yam} (\Ptcp{})
	& cá.yam
	& gúm.yam
	& bá.bryam
	\\

\bottomrule
\end{tabu}
\end{table}

Since pure prefixes and suffixes like \xayr{sitNF/}{sitang-}{self-} or 
\xayr{/IknF}{-ikan}{much, many} do not constitute independent words either, 
they also have been omitted from statistical analysis.

\autocite{strasser:freq}

\cleartorecto

% \chapter{Blah}
% \lipsum[1-2]
% 
% \section{Lorem}
% \lipsum[3-4]
% 
% \subsection{Ipsum}
% \lipsum[5-6]
% 
% \subsection{Dolor}
% \lipsum[7-8]
% 
% \subsection{Sit Amet}
% \lipsum[8-9]
% 
% \cleartorecto

%% BIBLIOGRAPHY %%%%%%%%%%%%%%%%%%%%%%%%%%%%%%%%%%%%%%%%%%%%%%%%%%%%%%%%%%%%%%%%

\printbibliography

%% APPENDIX %%%%%%%%%%%%%%%%%%%%%%%%%%%%%%%%%%%%%%%%%%%%%%%%%%%%%%%%%%%%%%%%%%%%

\appendix

%%%%%%%%%%%%%%%%%%%%%%%%%%%%%%%%%%%%%%%%%%%%%%%%%%%%%%%%%%%%%%%%%%%%%%%%%%%%%%%%

\end{document}
