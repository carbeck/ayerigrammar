% arara: xelatex
% arara: biber
% arara: makeglossaries
% arara: pythontex
% arara: xelatex

\documentclass[
	a4paper, % 9"×12" (folio) = oldpaper
	12pt,
	openany,
	twoside,
	final,
	oldfontcommands, % Compatibility with expex, which still uses \rm
]{memoir}

\author{Carsten Becker}
\title{A Grammar of Ayeri: Documenting a Fictional Language}
\date{\today}

% Layout for title page, cf. titlepages package
\newlength\drop
\makeatletter
\newcommand*{\titleS}{\begingroup% Scripts, T&H p 151
\drop = 0.1\textheight
\centering
\vspace*{\drop}
{\Huge A Grammar of Ayeri}\\[\baselineskip]
{\Large\scshape Documenting a Fictional Language}\\[\baselineskip]
{\large\itshape by Carsten Becker}\\[\baselineskip]
\vfill
\rule{0.4\textwidth}{0.4pt}\\[\baselineskip]
{\large\itshape Benung. The Ayeri Language Resource}\par
\vspace*{\drop}
\endgroup}
\makeatother

% Layout for running page titles
\nouppercaseheads

% Layout for chapter headers
\chapterstyle{southall}
\renewcommand*{\chaptitlefont}{\huge\sffamily\raggedright}

% Disemulate setspace
% \DisemulatePackage{setspace}
% \usepackage{setspace}
% \onehalfspacing

% Creative Commons icons
\usepackage{ccicons}

% Use Python
\usepackage[gobble=auto]{pythontex}

% Load fonts for XeTeX, including support for Unicode etc.
\usepackage{fontspec}
\usepackage{xunicode}
\usepackage{xltxtra}

% Handle language and quotation marks
\usepackage{polyglossia}
\setdefaultlanguage[variant=american]{english}
\usepackage{csquotes} % Put quotations in \enquote{}!
\SetBlockEnvironment{quotation}
\renewcommand*{\mkccitation}[1]{ (#1)}

% Date and time
\usepackage[
	useregional,
	%style=mdyyyy,
]{datetime2}

% Extended formatting of lists
\usepackage{enumitem}
\newlist{glossdefs}{itemize}{1}
\setlist[glossdefs]{nosep, leftmargin=3em, labelwidth=2.5em, align=left}
\setlist[itemize]{noitemsep}

% Make multiple columns available in single-column document
\usepackage{multicol}

% Make text colors and color names available
\usepackage[xetex]{xcolor}

% Set main fonts
%\usepackage[config=mt-Junicode]{microtype}

\newfontfamily{\Tagati}[
	Renderer=Graphite,
	Scale=1.1,
	BoldFont={* Italic},
	HyphenChar=·,
]{Tagati Book G}

\setmainfont{EB Garamond}[
	Ligatures={TeX,NoContextual},
	Numbers=Lowercase,
	BoldFont={Garamond Premier Pro Semibold},
]

\setsansfont{Fira Sans}[
	Ligatures=TeX,
	Numbers=Lowercase,
	Scale=MatchLowercase,
	BoldFont={* SemiBold},
]

\setmonofont{Fira Mono}[
	Ligatures=TeX,
	Scale=MatchLowercase,
]

\newfontfamily{\OpenSansCond}[
	Ligatures=TeX,
	Numbers=Lowercase,
	Scale=MatchUppercase,
	BoldFont={Open Sans Condensed Bold},
]{Open Sans Condensed Light}

% Set fonts for section headers
\setsecheadstyle{\sffamily\Large\raggedright}
\setsubsecheadstyle{\sffamily\bfseries\normalsize\raggedright}

% Nicer footnotes
\usepackage[bottom,hang,norule]{footmisc}
\setlength{\footnotesep}{0.75\baselineskip}

% Load BibLaTeX (using Biber), configure citation styles
\usepackage[
	authordate-trad,
	backend=biber,
	safeinputenc,
	natbib,
]{biblatex-chicago}
\addbibresource{bibliography.bib}

% To make \textcite look like "Doe (2014: 213)"
\renewcommand*{\postnotedelim}{\addcolon\addspace}
\DeclareFieldFormat{postnote}{#1}
\DeclareFieldFormat{multipostnote}{#1}

% Clickable links in footnotes, TOC, etc.
\usepackage[
	xetex,
	bookmarks=true,
	colorlinks=false,
	linktoc=section,
	hidelinks,
	pdfusetitle,
]{hyperref}

% Formatting of URLs
\usepackage{url}
\def\UrlFont{\normalfont\itshape}

% Ability to include graphics and dealing with footnotes in descriptions
\usepackage{graphicx}
\usepackage[font={small,sf},labelfont={small,sf},format=plain]{caption}
\usepackage{subcaption}
\usepackage{wrapfig}
\setlength{\columnsep}{2\baselineskip}

% Things for tables
\usepackage{longtable}
\usepackage{tabu}
\usepackage{booktabs}
\usepackage{rotating}

% Custom column types
\newcolumntype{B}{>{\bfseries}X}
\newcolumntype{I}{>{\itshape}X}

% Formatting of table of glossing abbreviations from Leipzig package manual
\usepackage[acronym,nomain,nonumberlist,nopostdot]{glossaries}
\usepackage{glossary-inline}%

\newglossarystyle{mysuper}{%
	\glossarystyle{super}% based on super
	\renewenvironment{theglossary}{%
		\begin{glossdefs}%
	}{%
		\end{glossdefs}%
	}%
	\renewcommand*{\glossaryheader}{}%
	\renewcommand*{\glsgroupheading}[1]{}%
	\renewcommand*{\glossaryentryfield}[5]{%
		\item[\glsentryitem{##1}\glstarget{##1}{##2}]
		\makefirstuc{##3}\glspostdescription{}
	}%
	\renewcommand*{\glsgroupskip}{}%
}%

% Formatting of glosses
\usepackage{expex}
\usepackage{leipzig}

\newleipzig{AgtT}{at}{agent topic}
\newleipzig{PatT}{pt}{patient topic}
\newleipzig{DatT}{datt}{dative topic}
\newleipzig{GenT}{gent}{genitive topic}
\newleipzig{LocT}{loct}{locative topic}
\newleipzig{InsT}{inst}{instrumental topic}
\newleipzig{CauT}{caut}{causative topic}
\newleipzig{An}{an}{animate}
\newleipzig{Inan}{inan}{inanimate}
\newleipzig{Hab}{hab}{habitative}
\newleipzig{Ayr}{ayr}{Ayeri}

\newleipzig{Nn}{nn}{Noun}
\newleipzig{Vb}{vb}{Verb}
\newleipzig{Adj}{adj}{Adjective}

\makeglossaries

\newcommand{\AargI}{{\Aarg}.{\Inan}}
\newcommand{\PargI}{{\Parg}.{\Inan}}
\newcommand{\AgtTI}{{\AgtT}.{\Inan}}
\newcommand{\PatTI}{{\AgtT}.{\Inan}}
\newcommand{\TsgI}{{\Tsg}.{\Inan}}
\newcommand{\TplI}{{\Tpl}.{\Inan}}

% Smaller font in block quotes
\usepackage{relsize}
\AtBeginEnvironment{quote}{\noindent\smaller}
\AtBeginEnvironment{quotation}{\smaller}

% Macros
\newcommand{\fw}[1]{\textit{#1}} % Foreign Word
\newcommand{\tit}[1]{\textit{#1}} % Title of a work
\newcommand{\q}[1]{\enquote{#1}} % Context-aware quotation
\newcommand{\qq}[1]{\enquote*{#1}} % Explicit sublevel quotation
\newcommand{\tsup}[1]{\textsuperscript{#1}} % Superscript
\newcommand{\tsub}[1]{\textsubscript{#1}} % Subscript
\newcommand{\markyellow}[1]{\colorbox{yellow}{#1}} % Yellow highlighter
\newcommand{\ques}{\fakesuperscript{?}} % raised question mark
\newcommand{\orth}[1]{⟨#1⟩}  % Orthography brackets
\newcommand{\tc}[1]{\trailingcitation{#1}} % \trailingcitation aus expex in Beispielen
\newcommand{\rc}[1]{\rightcomment{#1}} % \rightcomment aus expex in Beispielen

\newcommand{\ayr}[1]{\smash{{\Tagati #1}}} % Plain Ayeri orthography
\newcommand{\rayr}[2]{\smash{{\Tagati #1}} \emph{#2}} % Ayeri orthography + *r*omanization
\newcommand{\tayr}[2]{#1 `#2'} % Romanization + *t*ranslation
\newcommand{\xayr}[3]{\smash{\Tagati #1} \emph{#2} `#3'} % Ayeri orthography + romanization + translation

% blah
\usepackage{lipsum}

%%%%%%%%%%%%%%%%%%%%%%%%%%%%%%%%%%%%%%%%%%%%%%%%%%%%%%%%%%%%%%%%%%%%%%%%%%%%%%%%

\begin{document}

%% HALF TITLE %%%%%%%%%%%%%%%%%%%%%%%%%%%%%%%%%%%%%%%%%%%%%%%%%%%%%%%%%%%%%%%%%%

\begin{titlingpage}
\begin{center}
{\Huge A Grammar of Ayeri}
\end{center}
\end{titlingpage}

%% TITLE PAGE %%%%%%%%%%%%%%%%%%%%%%%%%%%%%%%%%%%%%%%%%%%%%%%%%%%%%%%%%%%%%%%%%%

\begin{titlingpage}
\titleS
\clearpage
% \begin{pycode}[env1]
% import time, hashlib
% m = hashlib.md5()
% m.update(str(time.time()).encode('utf-8'))
% c = m.hexdigest()[1:8]
% \end{pycode}

\begin{minipage}[b][\textheight][b]{0.67\textwidth}\small
\ccbysa~Carsten Becker, \the\year.\\
Published under \textsc{cc-by-sa} 4.0 license.\\
Last edited: \today{}.\\[.5\baselineskip]

Set in Junicode and {\sffamily Fira Sans} with \XeTeX{}.\\[.5\baselineskip]

Ayeri is a fictional language spoken by fictional people in a fictional setting, and as such is not related to any naturally existing languages. It is thus not to be confused with \emph{Azeri}, a Turkic language spoken in Azerbaijan and its surrounding countries. Ayeri’s vocabulary is entirely a priori, this means, no real-world languages have been used specifically as sources of vocabulary. Due to the language’s sound and spelling aesthetic being inspired by Austronesian languages, it is not surprising if overlaps with existing words in those languages happen accidentally.\\[.5\baselineskip]

\href{http://benung.nfshost.com}{http://benung.nfshost.com}\\
\href{https://github.com/carbeck/ayerigrammar}{https://github.com/carbeck/ayerigrammar/}\\
\href{https://creativecommons.org/licenses/by-sa/4.0/}{https://creativecommons.org/licenses/by-sa/4.0/}%\\

% \begin{center}\tiny
% \texttt{\pyc[env1]{print(c.strip())}}
% \end{center}
\end{minipage}

\end{titlingpage}

%% FRONT MATTER %%%%%%%%%%%%%%%%%%%%%%%%%%%%%%%%%%%%%%%%%%%%%%%%%%%%%%%%%%%%%%%%

\frontmatter
\tableofcontents
\clearpage
\listoffigures
\clearpage
\listoftables
\chapter{Glossing Abbreviations}
\begin{multicols}{2}
\printglossary[style=mysuper,type=\leipzigtype]
\end{multicols}
\cleartorecto

%% MAIN MATTER %%%%%%%%%%%%%%%%%%%%%%%%%%%%%%%%%%%%%%%%%%%%%%%%%%%%%%%%%%%%%%%%%

\mainmatter

% kate: word-wrap true;

\chapter{Preface}

This is my latest attempt to write a grammar of Ayeri, a fictional language 
which I have been developing since December 2003. Getting to work on grammar 
writing again was triggered by a growing dissatisfaction with not having a 
central place of documentation, when the first thing people look for on my 
website is often the grammar, incomplete as well as partially inaccurate and 
outdated as it may be. In addition to that, there was a seminar on fictional 
languages at the University of Tübingen, Germany, in the summer semester of 
2016 \autocite{buch2016ss}. Ayeri was one of the languages that was chosen for 
students to explore and evaluate.

The student group who worked on Ayeri came to the conclusion that its 
documentation is severely lacking in the description of basic elements and 
assumptions, since whole chapters of the grammar had been missing to date 
(\cite[12]{boga2016}).\footnote{\xayr{kuːtnsF/IknF}{Kutānas-ikan}{thanks a 
lot} to Bella Boga, Madita Breuninger, Thora Daneyko, and Martina Stama-Kirr for 
their hard work on making sense of my published materials in spite of 
information being scattered all over the place, as well as their providing me 
with the presentation concluding their group work.} This is to say that previous 
attempts of writing a full-fledged grammar of Ayeri have been incomplete due to 
loss of enthusiasm and creeping neglect.

Although the \tit{Ayeri Grammar} has so far been lying dormant for five years, 
I have written a whole number of blog articles detailing various grammatical 
issues \autocite[Blog]{benung}. These articles have been taken into 
consideration here. This grammar writing attempt is thus not only a transferral 
to a different typesetting system, but constitutes an extension to previous 
formal documentation as well.

I hope that by transferring my previous grammar writing from LibreOffice to  
\LaTeX{}, combined with using GitHub as a version control system, maintaining 
and editing will become faster, more transparent, and more elegant, since 
\LaTeX{} operates on plain text files, and version control helps in keeping 
track of changes over time.

\begin{flushright}\itshape\footnotesize
Carsten Becker\\
Marburg, \DTMdate{2016-07-18}
\end{flushright}
\cleartorecto

% kate: word-wrap true;

\setcounter{chapter}{-1}
\chapter{Introduction}

\begin{minipage}{\linewidth}\raggedleft\smaller
\ayr{\larger prony AdnYaaNF si miNF thnojyaaNF, EdreNF voj kotnsF.}\\
\fw{Paronaya adanyāng si ming tahanoyyāng, edareng voy kotanas.}\\
`He who cannot write believes it not to be toil.'\\
--- Anonymous
\end{minipage}\bigskip\bigskip

\noindent In December 2003, the idea for a new fictional language was born, an
idea that turned out to stick with me for over 10 years now.\footnote{Most of
the text here is taken from the blog article,
``\citetitle{benung:happybirthday}'' \parencite{benung:happybirthday} with some
slight rephrasings and extensions.} At that time, my seventeen years old self
was still fairly new to this whole making-up languages business, read things
about linguistics here and there, and was not shy to ask questions about
terminology (and, looking at old mails, a little impertinently teenager-like
so), for example on \tit{Conlang-L} and the \tit{Zompist Bulletin Board}. One
thing seemed to catch my interest especially: syntactic alignments other than
the \Nom{}/\Acc{} of the few languages I was familiar with, that is, German,
English, and French. Apparently this curiosity was big enough for me to grow
bored with my second fictional language, Daléian (declared `quite complete'
after maybe half a year of work or so), and to start something new from scratch
in order to put newly acquired knowledge to test.

I had read about ``trigger languages'' on \tit{Conlang-L} and wanted to try my
hands on making my own. I cannot remember how long it took me to come up with a
first draft of an Ayeri grammar, however, I do remember having been told that a
good language cannot be made in a summer. Of course, I still did not really
know what I was doing then, even though I thought I had understood things and
authoritatively declared ``this is how it works'' in my first grammar draft
when things sometimes really do not work that way. But at least an interest had
been whetted.

In order to illustrate the various stages from the beginnings to current Ayeri,
I went through some old backups contemporary with the very early days. 
Here is a sentence from the oldest existing document related to it, titled 
``Draft of \& Ideas for my 3rd Conlang''---the file's last-changed date is 
\DTMdate{2003-12-14}, though I remember having started work on Ayeri in early 
December. I added glossing for convenience and according to what I could 
reconstruct from the notes. This uses vocabulary and grammatical markers just 
made up on the spot and for illustrative purposes; little of it actually 
managed to make it into actual work on Ayeri:

\ex\begingl
	\gla Ayevhoi agiaemaesim coyaielieðamavir vhaieloyaŋaiye. //
	\glb ay-evhoi agia-ema-esim coyai-el-i-eðam-avir vhai-el-o-yaŋa-iye //
	\glc \Tsg{}.\An{}-\Sbj{} read-\textsc{verb}-\Sbj{}.\An{} 
		book-\textsc{noun}-\An{}-\Indf{}-\Parg{} 
		bed-\textsc{noun}-\Inan{}-on-\Loc{} //
	\glft `He reads a book on the bed.' //
\endgl\xe

According to the grammar draft of \DTMdate{2004-09-05}, this would have already
changed to:

\ex\begingl
	\gla Ang layaiyạin mecoyalei ling *pinamea. //
	\glb ang laya-iy-a-in me-coya-lei ling *pinam-ea //
	\glc \Aarg{}.\Sbj{} read-\Tsg{}.\An{}₁-a₁-\Sbj{} 
		\Indf{}.\Inan{}-book-\PargI{} top.of bed-\Loc{} //
	\glft `He reads a book on the bed.' //
\endgl\xe

A word for `bed'---\rayr{pinmF}{pinam}---was only (re-)introduced on
\DTMdate{2008-10-24}, however. In the current state of Ayeri, I would 
translate the sentence as follows:

\ex\begingl
	\gla Ang layaya koyaley ling pinamya. //
	\glb ang laya=ya.Ø koya-ley ling pinam-ya //
	\glc \AgtT{} read=\Tsg{}.\M{}.\Top{} book-\PargI{} top.of bed-\Loc{} //
	\glft `He reads a book on a/the bed.' //
\endgl\xe

As you can see, quite a bit of morphology got lost already early on, especially
the overt part-of-speech marking (!) and animacy marking on nouns. Also,
prepositions were just incorporated into a noun complex as suffixes apparently.
Gender was originally only divided into animate and inanimate, but I changed
that at some point because only being really familiar with European languages,
it felt awkward to me not to be able to explicitly distinguish `he', `she', and
`it'.

A feature that also got lost is the assignment of thematic vowels in personal
pronouns to third-person referents: originally, every third-person referent
newly introduced into discourse would be assigned one of /a e i o u/ to
disambiguate, and there was even a morpheme to mark that the speaker wanted to
dissolve the association. Constituent order was theoretically variable at
first, but I preferred SVO/AVP due to familiarity with that. Later on, however,
I settled on VSO/VAP. Also, I had no idea about what was called ``trigger
morphology'' on \tit{Conlang-L} for the longest time---essentially, this
referred to the Austronesian, or Philippine, aligment. I am not claiming that I
know all about it now, just that due to reading up on the topic, I have a
slightly more informed understanding now. Orthography changed as well over the
years, so \orth{c} in the early examples encodes the /k/ sound, not /tʃ/ as it
does today; diphthongs were spelled as \orth{Vi} instead of modern \orth{Vy}.

What was definitely beneficial for the development of Ayeri was the ever 
increasing amount of linguistics materials available online and my entering 
university (to study literature) in 2009, where I learned how to do research 
and also had a lot of interesting books available at the library.

One of the things people regularly compliment me on is Ayeri's script---note,
however, that Tahano Hikamu\index{Tahano Hikamu} was not the first one I came
up with for Ayeri. Apparently, I had already been fascinated with the look of
Javanese/Balinese writing early on;\footnote{Compare, for instance, the charts
in \citet{kuipersmcdermott1996}. The Wikipedia articles on either script
contain a number of images depicting the scripts in use, both current and
historic.} \autoref{fig:ayeriscript2004} shows a draft dated
\DTMdate{2004-02-09}. However, the letter shapes in this draft looked so
confusingly alike that I could never memorize them. About a year later, I came
up with the draft in \autoref{fig:th2005}. What is titled ``Another
Experimental Script'' there is what would later turn into Tahano Hikamu,
Ayeri's `native' script. According to the notes in my fictional language ring
binder, the script looked much the same as today about a year from then, but
things have only been mostly stable since about 2008.

\begin{figure}[tp]
	\centering
	\includegraphics[width=.8\textwidth,keepaspectratio]%
		{images/ayeriscript2004-300dpi-bw.png}
	\caption[First design for an Ayeri script]{First design for an Ayeri 
		script (\DTMdate{2004-02-09})}
	\label{fig:ayeriscript2004}
\end{figure}

\begin{figure}[tp]
	\centering
	\includegraphics[width=\textwidth, keepaspectratio]%
		{images/th2005-300dpi-bw.png}
	\caption[First draft for Tahano Hikamu]{First draft for Tahano Hikamu 
		(\DTMdate{2005-03-23})}
	\label{fig:th2005}
\end{figure}

An important date in the history of Ayeri was when I decided to set up an
improved website for Ayeri that would include a blog. The idea was that this
way, I could more freely write on whatever detail I currently interested me in
Ayeri, outside of the constraints of the Grammar. Thus, \tit{Benung.~The Ayeri
Language Resource} launched on \DTMdate{2011-03-01}. Being able to write short
articles, however, probably also led to neglecting work on the actual formal
reference grammar, which had been lying dormant from January 2011 on. This was
always on the premise that I would eventually include the information from blog
articles in the grammar. However, juggling such a big document had always felt
daunting, so I let laziness take the better part of me eventually as enthusiasm
gradually subsided.\footnote{Let me add to my defense, however, that I also
worked on my B.\,A. thesis in 2013 and my M.\,A. thesis in 2016, which required
several months of preparation each and thus left me largely unable to work much
on Ayeri.} This renewed attempt at documentation has been started with the
intention to right those wrongs.

I hope that by now it should be clear which kind of a fictional language Ayeri
is: a personal, artistic language---or \emph{artlang} in community parlance.
Thus, my goal\index{goals} in creating Ayeri is not to propose yet another
international auxiliary language, like Esperanto. It is also not my goal to
make it as logical as possible, like Lojban. Neither is it my goal to engineer
it towards certain underlying premises, for example, to reach a maximal amount
of information density, like Ithkuil, or to get by on as few different words as
possible, like Toki Pona. It is also not a `what-if' language in the sense of
\textquote{What could the modern language of Old Irish speakers transplanted to
Australia look like?} or \textquote[][.]{Latin piped through Athabascan sound
changes}

Ayeri is rather an attempt to create an artificial language for personal
enjoyment and intellectual stimulation by creating a feedback loop between
reading up on linguistics and actively devising rules for a fictional language
accordingly, to see how things work within the frame I created, or to try and
see whether certain ideas work together at all when combined, and to better
understand why they do or do not. Ayeri will only ever be as perfect as
miniature models of things can be, since it has not grown organically from
millenia of human interaction, and I cannot and will never know about each and
every aspect of language myself, in spite of continued curiosity about these
matters. Nor will it be possible for me to replicate all the fascinating twists
and irregularities that natural languages normally entail. The ultimate
goal\index{goals} in my work on Ayeri is, I suppose, to make it emulate natural
languages to at least some degree of depth and complexity.

In writing this grammar, I hope that I will find a good balance between 
applying linguistic theory to already existing materials and ideas, and going 
forth to create rules for aspects of the language that have so far been 
neglected, often due to my not being aware of them. In my opinion, the split 
between being able to apply methods of linguistics to what has grown over the 
course of more than a decade on the one hand, and discovering and developing 
new aspects of the language on the other is what makes Ayeri an interesting 
piece of \textquote[][,]{informed nonsense} as a colleague of mine once put it.

If in the following text my (non-native) English is not always fully idiomatic,
you find that I got facts, theories or analyses wrong, or not all aspects of
the language or its description are equally thoroughly worked-out---which are
all very likely events---I ask you to remember that this work is a one-person
effort, so mistakes and errors are unavoidable. You are kindly invited to share
any constructive criticism you have with me, however, to correct or improve any
issues that might need correction or elaboration.

This book is structured in a way to go from the building blocks of the language
to increasingly larger structures. Thus, \autoref{ch:phonology} deals with the
phonology of Ayeri and \autoref{ch:writing} with its alphabet. Chapter
\ref{ch:morphtyp} contains a discussion of the various morphological means in a
general, typological way while the subsequent \autoref{ch:gramcat} discusses
the morphology of the individual parts of speech. Chapter \ref{ch:phrasestruct}
finally discusses how syntactic phrases are built up from words, eventually
leading to the formation of complete sentences.

\cleartorecto

% \chapter{Blah}
% \lipsum[1-2]
% 
% \section{Lorem}
% \lipsum[3-4]
% 
% \subsection{Ipsum}
% \lipsum[5-6]
% 
% \subsection{Dolor}
% \lipsum[7-8]
% 
% \subsection{Sit Amet}
% \lipsum[8-9]

\cleartorecto

%% BIBLIOGRAPHY %%%%%%%%%%%%%%%%%%%%%%%%%%%%%%%%%%%%%%%%%%%%%%%%%%%%%%%%%%%%%%%%

\printbibliography

%% APPENDIX %%%%%%%%%%%%%%%%%%%%%%%%%%%%%%%%%%%%%%%%%%%%%%%%%%%%%%%%%%%%%%%%%%%%

\appendix

%%%%%%%%%%%%%%%%%%%%%%%%%%%%%%%%%%%%%%%%%%%%%%%%%%%%%%%%%%%%%%%%%%%%%%%%%%%%%%%%

\end{document}
