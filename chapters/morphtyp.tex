% kate: word-wrap true;

\chapter{Morphological typology}
\index{typology!of morphemes|(}

The first chapter dealt with the smallest constituent parts of words---speech 
sounds, which ones there are, and how they assemble into valid words. 
Consequently, the following two chapters will be about the next step up from 
this: morphemes, the atoms of meaning. First we will have a more general look 
at which kinds of morphemes there are, and then look at them more closely by 
part of speech: what is their distribution, and how are morphemes put together 
to form inflected words? This chapter on morphological typology will first 
deal with general questions about Ayeri's degree of synthesis, and then will
try to answer questions about the kinds of functions the various morpheme
classes carry out in the language.

\section{Typology}
\label{sec:typology}

For the largest part, Ayeri is an \emph{agglutinative}\index{agglutination} 
language. \citet{comrie1989} says of agglutinating languages that in these, 
typically,

\blockcquote[43--44]{comrie1989}{a word may consist of more than one morpheme,
but the boundaries between morphemes in the word are always clear-cut;
moreover, a given morpheme has at least a reasonably invariant shape, so that
the identification of morphemes in terms of their phonetic shape is also
straightforward. […] As is suggested by the term agglutinating (cf. Latin
\fw{gluten} `glue'), it is as if the various affixes were just glued on one 
after the other (or one before the other, with prefixes).}

In Ayeri, root morphemes are modified by affixes for the purposes of inflection
and derivation, and these affixes, in the form of suffixes\index{suffixes} more
specifically, can be stacked, especially on verbs. Indeed, they vary little, so
that they are always easily recognizable. Suffixation in Ayeri is especially
prominent on verbs:

\ex\begingl
	\gla Le kondasayāng hemaye pruyya nay napayya kayvay. //
	\glb Le kond-asa=yāng hema-ye-Ø pruy-ya nay napay-ya kayvay //
	\glc \PatTI{} eat-\Hab{}=\TsgM{}.\Aarg{} egg-\Pl{}-\Top{} salt-\Loc{} 
		and pepper-\Loc{} without //
	\glft `He always eats his eggs without salt and pepper.' //
\endgl\xe

The verb root \xayr{koMd/}{kond-}{eat} is inflected here for a habitual action
with the suffix \rayr{/As}{-asa}, and also carries a person-inflection
clitic,\index{clitics} \rayr{/yaaNF}{-yāng}, marking a third person singular
masculine agent. With the notable exception of pronouns and related person-
inflection clitics, affixes tend to encode a single grammatical function. Verbs
are not the only part of speech that can inflect; nouns and the relativizing
conjunction can as well:

\pex
\a\label{ex:letters}\begingl
	\gla Ang mətahanay tamanyeley yeyam. //
	\glb Ang mə-tahan=ay.Ø taman-ye-ley yeyam. //
	\glc \AgtT{} \Pst{}-write=\Fsg{}.\Top{} letter-\Pl{}-\PargI{} 
		\TsgF{}.\Dat{} //
	\glft `I wrote letters to her.' //
\endgl

\a\label{ex:relative}\begingl
	\gla Le turayāng taman sinā ang ningay tamala vās. //
	\glb Le tura=yāng taman-Ø si-Ø-na ang ning=ay.Ø tamala vās //
	\glc \PatTI{} send=\Tsg{}.\M{}.\Aarg{} letter-\Top{} 
		\Rel{}-\PatTI{}-\Gen{} \AgtT{} tell=\Fsg{}.\Top{} yesterday 
		\Ssg{}.\Parg{} //
	\glft `The letter which I told you about yesterday, he sent it.' //
\endgl
\xe

The principle of not conflating several grammatical functions into a single
suffix can be observed in (\ref{ex:letters}) regarding the word
\xayr{tmnFyelej}{tamanyeley}{letters}, in which the plural marker 
\rayr{/ye}{-ye} is distinct from the inanimate-patient case marker 
\rayr{/lej}{-ley} (the latter, however, conflates animacy and case). Strictly 
speaking, the pronoun \xayr{yeymF}{yeyam}{to her} is also composed, namely of
the third person feminine base form \rayr{ye}{ye} and the dative case marker
\rayr{/ymF}{yam}. Example (\ref{ex:relative}) is one we have already 
encountered before (p.~\pageref{doublerel}). Here, the relative pronoun,
\xayr{sinaa}{sinā}{of/about which} is inflected for genitive case, and 
stress on the usually unstressed last syllable
suprasegmentally\index{suprasegmental} marks that this form is contracted from
\rayr{sileyen}{sileyena} (\textit{si-ley-ena}, \Rel{}-\PargI{}-\Gen{}).

So far, we have concentrated on suffixes, but there are a number of 
prefixes\index{prefixes} as well; (\ref{ex:letters}) exhibits the past prefix 
\rayr{m/}{mə-} (which is actually redundant in this case). There are also 
demonstrative prefixes on nouns, however. In the following example, the prefix 
\xayr{Ed/}{eda-}{this-} joins the noun \xayr{pehmF}{peham}{carpet} to indicate 
a specific carpet.

\ex\begingl
	\gla Le no intoyyang eda-peham. //
	\glb Le no int-oy=yang eda=peham-Ø //
	\glc \PatTI{} want buy-\Neg{}=\Fsg{}.\Aarg{} this=carpet-\Top{} //
	\glft `I do not want to buy this carpet.' //
\endgl\xe

% FIXME: 'Bound word' very problematic, see Spencer & Luís 2012: 42ff. Manga 
% might be best analyzed as inflection (or a non-projecting word?), actually!
Besides prefixes and suffixes, Ayeri also possesses at least one element in the
verb cluster whose status as a function word or a clitic is not fully clear.
This is the case with the marker \rayr{mN}{manga}, which is treated as an
independent word, but can modify verbs and prepositions---heads of verb phrases
(VPs) and prepositional phrases (PPs), respectively---is unstressed and appears
at the margin of its modification target:

\pex
\a\label{ex:prog}\begingl
	\gla Ang manga yavaya ayon bariley. //
	\glb Ang manga yava-ya ayon-Ø bari-ley //
	\glc \AgtT{} \Prog{} roast-\TsgM{} man-\Top{} meat-\PargI{} //
	\glft `The man is roasting meat.' //
\endgl

\a\label{ex:dyn}\begingl
	\gla Ya mətapyyāng maritay misley manga luga bari. //
	\glb Ya mə-tapy=yāng maritay mis-ley manga luga bari-Ø //
	\glc \LocT{} \Pst{}-put=\TsgM{}.\Aarg{} before spit-\PargI{} \Dyn{} 
		between meat-\Top{} //
	\glft `The meat, he had put a spit through it before.' //
\endgl

\xe

In (\ref{ex:prog}), \rayr{mN}{manga} modifies the verb \xayr{yv/}{yava-}{roast}
and indicates that this is a temporarily ongoing action, like the English
progressive, except not as strongly grammaticalized.\footnote{I suppose, a
better parallel is the so-called \fw{rheinische Verlaufsform} `Ripuarian
progressive' (\fw{sein} `be' + \fw{am/beim} `at the' + infinitive) in German, a
construction common in the colloquial language which parallels the English
progressive construction and is not yet fully grammaticalized
\citep[435]{dudengram2016}. Speakers will thus accept both \fw{Er lernt 
gerade}, literally `He studies right now', and \fw{Er ist am Lernen} `He is
studying'.
% 
% \ex[lingstyle=fnex,belowexskip=-1em]\label{ex:ripprog}\begingl
% 	\gla Der Mann ist Fleisch am Braten. //
% 	\glc The man is meat at.the roasting //
% 	\glft `The man is roasting meat.' //
% \endgl\xe
}
%
In (\ref{ex:dyn}), \rayr{mN}{manga} modifies the preposition, on the other 
hand, to indicate that it is dynamic: \rayr{lug}{luga} by itself means `among, 
between', while its dynamic form \rayr{mN lug}{manga luga} means `through; 
during, for'.

As we have seen in the examples above, person suffixes on verbs are single 
morphemes that encode more than one property, for example \rayr{/yeNF}{-yeng} 
encodes the person features third person, feminine, singular, and agent. 
Personal pronouns,\index{pronouns!personal} of which the person clitics on 
verbs are an instance, are the main case of fusion among agglutination in 
Ayeri, although some of the forms, like \xayr{yeymF}{yeyam}{to her} above, can 
be decomposed into root and suffix without problem.\footnote{Originally, 
Ayeri's personal pronouns were indeed agglutinative as well, so 
\xayr{yeNF}{yeng}{she} used to be \rayr{Iye\_aNF}{iyeang} (\fw{iy-e-ang}, 
\Tsg{}-\F{}-\Aarg{}). This also gives an explanation to \citet{boga2016}'s 
observation that Ayeri's plural pronouns are formed \textcquote[{[}15{]}; 
`possibly in an even too regular way']{boga2016}{[v]ielleicht sogar zu 
regelmäßig}.}

Perpendicular to the axis isolation–agglutination runs the axis 
analytic–syn\-thetic. On the latter axis, Ayeri scores mostly as 
\emph{synthetic}, since it prefers compactness over spreading a construction 
over several words, though it does not incorporate object noun phrases (NPs) 
and it is not possible to form `sentence-words' either, so it is not going so 
far as to be poly\-syn\-thetic \citep[45--46]{comrie1989}. It is nonetheless 
theoretically possible, due to suffixation being a prominent pattern, to form 
foot-long words like

\ex\label{ex:footlong}\begingl
	\gla da-mətahasongoyyang-ikan //
	\glb da=mə-taha-asa-ong-oy=yang=ikan //
	\glc such=\Pst{}-have-\Hab{}-\Irr{}-\Neg{}=\Fsg{}.\Aarg{}=much //
	\glft `I would not much used to have had such' //
\endgl\xe

Cases of analytic morphology are compound prepositions as we have seen 
with \xayr{mN lug}{manga luga}{through} in (\ref{ex:dyn}), but verbs as well 
show analytic structures not only with the progressive marker, but also with 
modals:

\ex\begingl
	\gla Ming sahoyyang dabas. //
	\glb Ming saha-oy=yang dabas //
	\glc can come-\Neg{}=\Fsg{}.\Aarg{} today //
	\glft `I can't come today.' //
\endgl\xe

Most of the information the VP contains in this example is marked on the
content verb, \xayr{sh/}{saha-}{come}, except for ability, which is expressed
by the particle \xayr{miNF}{ming}{can}. \rayr{miNF}{ming} is an uninflected
form of the verb expressing ability and may be counted as an auxiliary verb in
that the full semantic content of the VP is spread out over two verb forms, one
major, one minor---this probably should not be understood as a serial verb
construction, however \citep{aikhenvald2006}.%
\footnote{\rayr{mN}{manga} has, in fact, a verbal counterpart 
\xayr{mN/}{manga-}{move; remove} as well, which presumably served as the 
origin of both the progressive and the dynamic marker.\label{fn:mangaverb}}
Consider also the following example in which \rayr{miNF}{ming} is inflected
like a regular verb:

\ex\begingl
	\gla Da-mingya ang Diyan. //
	\glb Da=ming-ya ang Diyan. //
	\glc so=can-\TsgM{} \Aarg{} Diyan //
	\glft `Diyan can (do it).' //
\endgl\xe

\section{Morphological processes}

\subsection{Prefixation}
\index{prefixes|(}

Prefixes in Ayeri apply mainly to verbs, but nouns, pronouns, adjectives and
conjunctions as well can appear with them, some of which may be clitics;
reasons for their being clitics will be given at the appropriate cases in the
sections on the various parts of speech. With verbs, prefixes that are most
certainly `true' prefixes---that is, morphemes that have been semantically
bleached by grammaticalization to the point where they only express grammatical
functions \citep[157ff.]{lehmann2015} and which subcategorize words rather than
phrases \citep[117]{klavans1985}---are the tense prefixes marking both three
degrees of past and future tense, for example:

\ex\begingl
	\gla Ang səsarāyn ya Makapetang. //
	\glb Ang sə-sara=ayn.Ø ya Makapetang //
	\glc \AgtT{} \Fut{}-go=\Fpl{}.\Top{} \Loc{} Makapetang //
	\glft `We will go to Makapetang.' //
\endgl\xe

Here, the prefix \rayr{se/}{sə-} marks future tense on the verb, 
\xayr{sr/}{sara-}{go}. The other tense\index{tense} prefixes are \rayr{k/}{kə-} 
(\NPst{}), \rayr{m/}{mə-} (\Pst{}), \rayr{v/}{və-} (\RPst{}), and 
\rayr{p/}{pa-} (\NFut{}) and \rayr{ni/}{ni-} (\RFut{}). Besides this set of 
prefixes, there are also a number of proclitics that can appear with verbs, 
though not exclusively. These are the anaphora \xayr{d/}{da-}{thus, so, such} 
and the reflexive\index{pronouns!reflexive} marker \xayr{sitNF/}{sitang-}{self}:
 
% \pex
% \a\begingl
\ex\begingl
	\gla Da-mingya ang Diyan. //
	\glb Da=ming-ya ang Diyan. //
	\glc so=can-\TsgM{} \Aarg{} Diyan //
	\glft `Diyan can (do it).' //
\endgl
% 
% \a\begingl
% 	\gla Da-sahāra seyaraneng. //
% 	\glb Da=saha-ara seyaran-eng //
% 	\glc thus=come-\TsgI{} rain-\AargI{} //
% 	\glft `Here/Thus comes the rain.' //
% \endgl
\xe

\ex~\begingl
	\gla Sitang-kecāng. //
	\glb Sitang-ket=yāng //
	\glc \Refl{}-wash=\TsgM{}.\Aarg{} //
	\glft `He washes \emph{himself}.' //
\endgl\xe

\rayr{sitNF/}{sitang-} can also be used as a preverb in situations where the 
agent is also the instrument, so both of the following two sentences are 
equivalent in meaning:

\pex
\a\label{ex:sitang+pronoun}\begingl
	\gla Sa apicāng nanga ikan sitang-yari. //
	\glb Sa apit=yāng nanga ikan sitang-yari //
	\glc \PatT{} clean=\Tsg{}.\Aarg{} house complete 
		\Refl{}-\TsgM{}.\Ins{} //
	\glft `He cleaned the whole house by himself.' //
\endgl

\a\begingl
	\gla Sa sitang-apicāng nanga ikan. //
	\glb Sa sitang-apit=yāng nanga ikan //
	\glc \PatT{} \Refl{}-clean=\Tsg{}.\Aarg{} house complete //
	\glft (idem) //
\endgl
\xe

\phantomsection\label{nounprefixes}
Example (\ref{ex:sitang+pronoun}) shows the more common application of 
\rayr{sitNF/}{sitang-}, that is, as a reflexive modifier of pronouns. The 
prefix \rayr{d/}{da-} can as well be used with noun phrases and is part of the 
demonstrative set of prefixes, \xayr{d/}{da-}{such}, \xayr{Ed/}{eda-}{this}, 
and \xayr{Ad/}{ada-}{that}:

\ex\begingl
	\gla eda-ganang //
	\glb eda=gan-ang //
	\glc this=child-\Aarg{} //
	\glft `this child' //
\endgl\xe

The demonstrative prefixes are also used to form the demonstrative 
pronouns\index{pronouns!demonstrative} \xayr{EdnY}{edanya}{this one}, 
\xayr{AdnY}{adanya}{that one} and \xayr{dnY}{danya}{such one}. A special case 
in this regard is the postposition \xayr{d/naarY}{da-nārya}{in spite of, 
despite} where \rayr{d/}{da-} combines with the conjunction 
\xayr{naarY}{nārya}{but, although, except}. Originally, 
\xayr{dikpis}{dikapisa}{respective} is derived from \rayr{d/}{da-} + 
\xayr{Ikpis}{ikapisa}{bound, dependent}, which is an example of a combination 
with an adjective. There is also a fixed adverbial expression using one of 
these prefixes, \xayr{Ed/tdjymF}{eda-tadayyam}{for the time being, for now} 
(this-time-\Dat{}).

Last but not least, the prefix \xayr{ku/}{ku-}{like, as though} can be used 
with both adjectives and nouns (or, more precisely, phrases containing 
nominals):

\pex
\a\begingl
	\gla ku-koyaya //
	\glb ku=koya-ya //
	\glc like=book-\Loc{} //
	\glft `like in a book' //
\endgl

\a\begingl
	\gla ku-prasi //
	\glb ku=prasi //
	\glc like=sour //
	\glft `as though (it were) sour' //
\endgl
\xe

An example of a set-phrase adverbial consisting of \rayr{ku/}{ku-} and a verb 
is \xayr{ku/nsY}{ku-nasya}{as follows}, \rayr{nsY/}{nasy-} meaning `follow'. 
What is curious here is that this fossilized form is lacking person marking 
and is just extended with an epenthetic \textit{-a} since \textit{-sy} is not 
a permissible coda. The expected form would be 
*\rayr{ku/nsYreNF}{*ku-nasyareng} (like-follow=\TsgI{}.\Aarg{}).

Following \citet{klavans1985}, who suggests that clitics best be defined as
\textcquote[117]{klavans1985}{affixation at the phrasal level}, a very common 
kind of prefix to the verb \emph{phrase} are the topic markers. They are
counted as parts of the VP but do not interact with it regarding stress
assignment (they are always unstressed) while always being in an initial
position, preceding any other preverbal elements:

\pex
	\a\begingl
		\gla Ang tahanya tamanley. //
		\glb Ang tahan-ya taman-ley //
		\glc \AgtT{} write-\TsgM{} letter-\PargI{} //
		\glft `He writes a letter.' //
	\endgl
	\a \textit{Ang mətahanya tamanley.} `He wrote a letter.'
	\a \textit{Ang manga mətahanya tamanley.} `He was writing a letter.'
	\a \textit{Ang manga no mətahanya tamanley.} `He was wanting to write a 
		letter.'
\xe

% kudapalung 'other than that, apart from that' < ku-da-palung 'like-such-other'
% => combination of prefixes with adjective!

\index{prefixes|)}

\subsection{Suffixation}
\index{suffixes|(}

As a largely agglutinative language, most grammatical marking in Ayeri is done
by means of suffixes. These occur mainly with nouns and verbs, however,
quantifiers take the shape of suffixes as well. Quantifiers, then, may modify
content words almost regardless of their part of speech---noun, verb, adjective
or adverb. The most pervasive examples of suffixation are certainly those of
case marking on nouns and of person marking on verbs, for example:

\ex\label{ex:conjdecl}\begingl
	\gla Sa pəharuyang va manga miday tangya vana suyareri, vimyon! //
	\glb Sa pə-haru=yang va.Ø manga miday tang-ya vana suyar-eri, vimyon //
	\glc \PatT{} \NFut{}-beat=\Fsg{}.\Aarg{} \Ssg{}.\Top{} \Dyn{} around 
		ears-\Loc{} \Ssg{}.\Gen{} ladle-\Ins{}, monkey! //
	\glft `I'll beat you around your ears with a ladle, you monkey!' //
\endgl\xe

This example shows marking of \xayr{tNF}{tang}{ears} with the locative case
suffix \rayr{/y}{-ya} and the marking of \xayr{suyrF}{suyar}{ladle} with the
instrumental case suffix \rayr{/Eri}{-eri}; the previous examples already
provide instances of the exceedingly common markers for agent and patient case,
\rayr{/ANF}{-ang} and \rayr{/AsF}{-as}, respectively. Besides case, nouns can
also be marked for plural with the suffix \rayr{/ye}{-ye}, and verb roots may
be extended by the mood and aspect markers \rayr{/ONF}{-ong} (\Irr{}),
\rayr{/As}{-asa} (\Hab{}) and \rayr{/Oj}{-oy} (\Neg{}), the last of which is 
the most frequently occurring one. The mood suffixes can also be stacked,
leading to the long word in (\ref{ex:footlong}) above. Person marking on verbs
is realized as agreement suffixes or of clitic personal pronouns depending on
whether an agent NP proper is present or not for the verb to agree with. In
(\ref{ex:conjdecl}), a cliticized agent pronoun \xayr{/yaaNF}{-yāng}{he}
(\TsgM{}.\Aarg{}) appears.

As mentioned above, quantifiers appear as enclitics on almost any type of 
content word, like on the adverb \xayr{pr}{para}{fast} in the following example:

\ex
%\pex
% \a\begingl
% 	%\glpreamble With a verb: //
% 	\gla No sarayang-ikan //
% 	\glb No sara=yang=ikan //
% 	\glc want go=\Fsg{}.\Aarg{}=much //
% 	\glft `I really want to go.' //
% \endgl
%\a
\begingl
	%\glpreamble With an adverb: //
	\gla Tigalyeng para-ma. //
	\glb Tigal=yeng para=ma //
	\glc swim=\TsgF{}.\Aarg{} fast=enough //
	\glft `She swims fast enough.' //
\endgl

% \a\begingl
% 	%\glpreamble With a predicative adjective: //
% 	\gla Yang valuy-eng, sahavāng. //
% 	\glb Yang {valuy eng}, saha=vāng. //
% 	\glc \Fsg{}.\Aarg{} {glad rather}, come=\Ssg{}.\Aarg{} //
% 	\glft `I am rather glad that you come.' //
% \endgl
% 
% \a\begingl
% 	%\glpreamble With an attributive adjective: //
% 	\gla Adareng bahisley mino-ing //
% 	\glb Ada-reng bahis-ley mino=ing //
% 	\glc that=\TsgI{}.\Aarg{} day-\PargI{} happy=so //
% 	\glft `It was such a happy day.' //
% \endgl
% 
% \a\begingl
% 	%\glpreamble With a noun: //
% 	\gla Ang konjan prikanley-ani //
% 	\glb Ang kond=yan.Ø prikan-ley=ani //
% 	\glc \Aarg{} eat=\TsgM{}.\Top{} soup=not.at.all //
% 	\glft `They did not eat any soup at all.' //
% \endgl

\xe

\index{suffixes|)}

\subsection{Reduplication}
\label{subsec:reduplication}
\index{reduplication|(}

There are two patterns of reduplication for verbs, one with complete
reduplication of the imperative form to create a hortative statement
(\ref{ex:hort}), and one with partial reduplication as a way to express that an
action takes place again, that is, partial reduplication expresses a iterative,
as it were (\ref{ex:iter}). The imperative iterative, then, has a hortative
function as well (\ref{ex:hort+iter}):

\pex
\a\label{ex:hort}\begingl%
	\gla naru-naru //
	\glb naru\til{}nara-u //
	\glc \Hort{}\til{}speak-\Imp{} //
	\glft `let us speak' //
\endgl

\a\label{ex:iter}\begingl
	\gla na-narayeng //
	\glb na\til{}nara=yeng //
	\glc \Iter{}\til{}speak=\TsgF{}.\Aarg{} //
	\glft `she speaks again' //
\endgl

\a\label{ex:hort+iter}\begingl
	\gla na-naru //
	\glb na\til{}nara-u //
	\glc \Iter{}\til{}speak-\Imp{} //
	\glft `let us speak again' //
\endgl

\xe

With nouns, full reduplication is used to create a diminutive\index{diminutive}
form (\ref{ex:regdim}), though some reduplications are also lexicalized and may
use roots from other parts of speech as well to form nouns, for instance, the
words in (\ref{ex:otherredupnn}--\hyperref[ex:otherredupvb]{d}). There are also
a number of adjectives for which there exists a lexical reduplication with an
intensifying meaning; (\ref{ex:adjredup}) lists a few examples. This, however,
is not a productive derivation strategy.

\pex
	\a \makebox[10.5em][l]{\xayr{\larger venej}{veney}{dog}}
		→ \xayr{\larger venej/venej}{veney-veney}{little dog, 
			doggie}\label{ex:regdim}
	\a \makebox[10.5em][l]{\xayr{\larger gnF}{gan}{child}}
		→ \xayr{\larger ganF/ganF}{gan-gan}{grandchild}%
			\label{ex:otherredupnn}
	\a \makebox[10.5em][l]{\xayr{\larger kusNF}{kusang}{double (adj.)}}
		→ \xayr{\larger kusNF/kusNF}{kusang-kusang}{model} 
			\label{ex:otherredupadj}
	\a \makebox[10.5em][l]{\xayr{\larger vehF/}{veh-}{build}} → 
		\xayr{\larger veh/veh}{veha-veha}{tinkering}%
			\label{ex:otherredupvb}
\xe

\pex~\label{ex:adjredup}
	\a \makebox[10.5em][l]{\xayr{\larger ApnF}{apan}{wide}}
		→ \xayr{\larger ApnF/ApnF}{apan-apan}{extensive}
	%\a \makebox[10.5em][l]{\xayr{\larger IknF}{ikan}{complete}}
	%	→ \xayr{\larger IknF/IknF}{ikan-ikan}{entire, total}
	\a \makebox[10.5em][l]{\xayr{\larger kebj}{kebay}{alone}}
		→ \xayr{\larger kebj/kebj}{kebay-kebay}{all alone}
	%\a \makebox[10.5em][l]{\xayr{\larger pksF}{pakas}{special}}
	%	→ \xayr{\larger pksF/pksF}{pakas-pakas}{gay}
	\a \makebox[10.5em][l]{\xayr{\larger pisu}{pisu}{tired}}
		→ \xayr{\larger pisu/pisu}{pisu-pisu}{exhausting}
\xe

\index{reduplication|)}

\subsection{Suprasegmental modification}
\index{suprasegmental|(}

\index{morphophonology!of relative pronouns}
As written above (\autoref{doublerel}), case agreement on a complex-marked 
relative pronoun\index{pronouns!relative} can drop out under certain 
circumstances and is replaced by compensatory stress on the secondary case 
marker, which lengthens the syllable's nucleus vowel:

\ex\begingl
	\gla … tamanley sinā (*sina) ang ningay tamala vās //
	\glb … [taman-ley]₁ si-Ø₁-na (*si-na₁) ang ning=ay.Ø tamala vās //
	\glc … letter-\PargI{} \Rel{}-\PatTI{}-\Gen{} (*\Rel{}-\Gen{}) \AgtT{} 
		tell=\Fsg{}.\Top{} yesterday \Ssg{}.\Parg{} //
	\glft `… the letter which (*whose) I told you about yesterday' //
\endgl\xe

This can be reinterpreted so that vowel length/stress itself is what signifies 
the agreement of the relativizer with the preceding NP. Which grammatical role 
the relativizer's head instantiates as an agreement controller is essentially 
underspecified, hence I will gloss it as -\Agr{} in the following example 
instead of as full -\PargI{}:

\ex[everygla=\upshape]\begingl
	\gla /ˌsi.leɪ.ˈena/ → /si.ˈna(ː)/ //
	\glb /si-leɪ-ena/ → /si-ˈ-na(ː)/ //
	\glc \Rel{}-\PargI{}-\Gen{} {} \Rel{}-\Agr{}-\Gen{} //
\endgl\xe

Since \rayr{n}{na} as a light syllable cannot be stressed in word-final 
position under normal circumstances, it has to lengthen to \rayr{naa}{nā}.

\index{suprasegmental|)}

\subsection{Clitics}

I have been using the term `clitic' above and claimed that the one or the 
other morpheme in Ayeri is a clitic. Clitics, however, appear to be very 
elusive things which cannot be easily defined in a formal way 
\citep[126]{spencerluis2012}. Based on \citet{spencerluis2012}, with recourse to \citet{zwickypullum1983}, some typical characteristic traits appear to be:

\begin{itemize}
	\item Clitics behave in part like function words and in part like affixes, 
		but in any case they are not free morphemes 
		\citep[38, 42]{spencerluis2012}.
	\item Clitics tend to be phonologically weak items 
		\citep[39]{spencerluis2012}.
	\item Clitics prominently---and importantly---tend to attach 
		`promiscuously' to surrounding words. That is, unlike inflection, they 
		are not limited to connect to a certain part of speech or to align 
		with their host in semantics \citep[40, 108–109]{spencerluis2012}.
	\item Clitics tend to appear in a second position, whether that is after a 
		word or an intonational or syntactic phrase 
		\citep[41]{spencerluis2012}.
	\item Clitics tend to be templatic and to cluster, especially if they 
		encode inflection-like information \citep[41, 47--48]{spencerluis2012}.
	\item Clitics have none of the freedom of ordering found in true words and 
		phrases \citep[43]{spencerluis2012}.
	\item Positions of `special' clitics tend to not be available to free 
		words \citep[44]{spencerluis2012}.
	\item Clitics tend to be functional morphemes, and to realize a single 
		morphosyntactic property \citep[67, 179]{spencerluis2012}.
	\item There are no paradigmatic gaps \citep[108--109]{spencerluis2012}.
	\item There tends to be no morphophonemic alteration like vowel harmony, 
		stress shift or sandhi between a clitic and its host 
		\citep[108--109]{spencerluis2012}.
	\item There tends to be no idiosyncratic change in meaning when a clitic 
		and a clitic host come together, unlike there may be with inflection 
		\citep[108, 110]{spencerluis2012}.
	\item Similar to affixes, clitics and their host tend to be treated as a 
		syntactic unit, that is, there is lexical integrity, so it is not possible to put word material in between a clitic and its host 
		\citep[108, 110]{spencerluis2012}.
	\item Clitics usually get joined to a host word after inflection 
		\citep[108, 110]{spencerluis2012}.
	\item Affixes tend to go on every word in a conjunct (narrow scope), while 
		clitics have a tendency to treat a conjunct as a unit to attach to 
		(wide scope; \cite[139, 196\psqq]{spencerluis2012}).
\end{itemize}

However, \citet{spencerluis2012} point out many counterexamples in order to 
drive their point home that the border between clitics and affixes is often 
fuzzy. Given this definitional fuzziness, it comes as no surprise that 
according to their assessment, there is a lot of miscategorization in 
individual grammars as a result \citep[107]{spencerluis2012}. Another 
consequence of definitional fuzziness is that, since not all of the traits 
described above are always present, making a checklist and summing up the 
tally is only of limited value. The traits enumerated above are sufficient, 
but not necessary, conditions. In the following, I want to elaborate on the 
classification of various prefixes, suffixes, and particles as 
clitics.\footnote{The following discussion incorporates most of the content of 
a blog article I previously wrote on this topic, \citet{benung:clitics}. Since
clitics sit at the junction of morphology and syntax, it will be necessary at
times to deal with topics roughly which will be elaborated on later in more
detail.}

\subsubsection{Preposed particles and prefixes}

I think what should be rather unproblematic with regards to analysis as clitics
in Ayeri are the preverbal particles, that is, the topic marker, one or several
modal particles, the progressive marker, and also the emphatic affirmative and
negative discourse particles. All of these particles essentially have
functional rather than lexical content, and are usually unstressed. They come
in a cluster with a fixed order, and they appear in a position where no
ordinary word material could go, since Ayeri is strictly verb-initial. In
conjuncts it's also unnecessary to mark every verb with one or several of them:

\pex
\a\label{ex:clitics_1a}\begingl
	\gla Ang kece nay dayungisaye māva yanjas yena. //
	\glb ang ket-ye nay dayungisa-ye māva-Ø yan-ye-as yena //
	\glc \AgtT{} wash-\TsgF{} and dress-\TsgF{} mother-\Top{} boy-\Pl{}-\Parg{}
		\TsgF{}.\Gen{} //
	\glft `The mother washes and dresses her boys.' //
\endgl

\a\label{ex:clitics_1b}\begingl
	\gla Manga sahaya rangya nay nedraya ang Tikim. //
	\glb manga saha-ya rang-ya nay nedra-ya ang Tikim //
	\glc \Prog{} come-\TsgM{} home-\Loc{} and sit-\TsgM{} \Aarg{} Tikim //
	\glft `Tikim is coming home and sitting down.' //
\endgl

\a\label{ex:clitics_1c}\begingl
	\gla Ang mya ming sidegongya nay la-lataya ajamyeley. //
	\glb ang mya ming sideg-ong=ya.Ø nay la\til{}lata=ya.Ø ajam-ye-ley //
	\glc \AgtT{} be.supposed.to can repair-\Irr{}=\TsgM{}.\Top{} and
		\Iter{}\til{}sell=\TsgM{}.\Top{} toy-\Pl{}-\PargI{} //
	\glft `He should be able to repair and resell the toys.' //
\endgl
\xe

In (\ref{ex:clitics_1a}), therefore, the agent-topic marker \rayr{ANF}{ang}
only occurs before \xayr{ketYe}{kece}{(she) washes}, and the conjoined verb
\xayr{dyuNisye}{dayungisaye}{(she) dresses} is also within its scope.
Repeating the marker as well before the latter verb could either be considered
ungrammatical because there is only one topic there---\xayr{maav}{māva}
{mother}---or the sentence could be interpreted as having two conjoined clauses
with different subjects: `She\tsub{1} washes and mother\tsub{2} dresses her
boys.' The latter outcome has \rayr{maav}{māva} as the topic only of
\rayr{dyuNisye}{dayungisaye}, while \rayr{ketYe}{kece}'s topic is the person
marking on the verb---a pro-drop subject, essentially.\footnote{This claim is
further investigated below, \autoref{REFERENCE_GOES_HERE}; also compare
\autoref{subsec:persnumagr}.}

In (\ref{ex:clitics_1b}), then, the progressive marker \rayr{mN}{manga}
equally has scope over both verb conjuncts, \xayr{shy}{sahaya}{(he) comes}
and \xayr{nedFry}{nedraya}{(he) sits} in what is presumably a case of
extended/distributed exponence. This is to say that functionally contiguous
information can sometimes be split over several words, so that the functional
annotation of each verb in (\ref{ex:clitics_1b}) can be represented in the
fashion of the (incomplete) f-structure matrix \parencites[see][]
{bresnan2016}{buttking2015} shown in (\ref{ex:clitics_2}), which is an
attempt to exemplarily describe the phrase \xayr{ANF mN sh/}{ang manga
saha-}{is coming}. \rayr{mN}{manga} is treated there as being part of things
the verb inflects for, that is, progressive aspect, in spite of appearing
superficially as a function word. The topic marker \rayr{ANF}{ang} does not
reflect a morphological property of the verb in the way the progressive marker
does, but announces the case and---for agents and patients---the animacy value
of the topicalized noun phrases (NPs), so the f-structure in
(\ref{ex:clitics_2}) lists this information under the \Top{} relation.%
\footnote{In the chart, angular brackets group grammatical functions together.
Since the verb is basically the head of the clause, the first \Pred{} 
(predicator) lists the verb with its argument structure (a-structure).
In the case of (\ref{ex:clitics_2}), \ups{\Sbj{}} and \ups{\Oblq{loc}} indicate
that the verb, `come', governs two arguments: one syntactic subject and one
oblique argument in the form of a locative adverbial.}

\ex\label{ex:clitics_2}
\begin{avm}
\[
	\Pred{}	&	\astruct{come}{\ups{\Sbj{}}, \ups{\Oblq{loc}}} \\

	\Asp{}	&	\Prog{} \\

	\Top{}	&	\[
					\Case{}	&	\Aarg{} \\
					\Anim{}	&	$+$ \\
					...		&	... \\
				\] \\

	...		&	... \\
\]
\end{avm}
\xe

Modal particles, exemplified in (\ref{ex:clitics_1c}), are probably slightly
less typical as clitics since it seems feasible for them to be stressed for
contrast. What is not possible, however, is to front either \xayr{mY}{mya}{be
supposed to} or \xayr{miNF}{ming}{can}, and the verb itself also cannot precede
the particles, which is demonstrated in (\ref{ex:clitics_3}). It is also not
possible to coordinate any of the elements in the preverbal particle cluster
with \xayr{nj}{nay}{and}, as shown in (\ref{ex:clitics_4}).

\pex\label{ex:clitics_3}
\a \ljudge{*} \textit{mya ang ming sidegongya}
\a \ljudge{*} \textit{ming ang mya sidegongya}
\a \ljudge{*} \textit{sidegongya ang mya ming}
\xe

\pex~\label{ex:clitics_4}
\a \ljudge{*} \textit{ang \textbf{nay} mya ming sidegongya}
\a \ljudge{*} \textit{ang mya \textbf{nay} ming sidegongya}
\a \ljudge{*} \textit{ang mya ming \textbf{nay} sidegongya}
\xe

It needs to be pointed out that unlike verbs, modal particles in Ayeri resist
inflection, so in (\ref{ex:clitics_1c}) the irrealis suffix \rayr{/ONF}
{-ong} is realized on the verb \xayr{sidegoNY}{sidegongya}{(he) would
repair} instead of on one or both of the modal particles as
*\rayr{miNONF}{*mingong} and *\rayr{mYONF}{myong}, respectively. The
combination of \xayr{mY}{mya}{be supposed to} with an irrealis-marked verb
together indicates that the speaker thinks the action denoted by the verb
\emph{should} be carried out. Instead, the marking on the verb may be
interpreted as distributing to the constituent parts of whole verb complex.
The same goes for negation: only the verb can be negated, but not the modal
particle. Possibly, it would be useful in this case to abstract the modal
particles as a feature [\Mod{}\textsc{ality}] as listed by \citet[Feature
Table]{pargram} for purposes of functional representation. At least
superficially, it looks as though Ayeri acts differently from English here in
that the verb is possibly not a complement of the modal. This is probably
mostly apparent from the fact that in Ayeri, the verb inflects, not the modal
particle. Furthermore, modal particles cannot be modified by adverbs in the way
regular verbs can:

\pex\label{ex:clitics_5}
\a\label{ex:clitics_5a}\begingl
	\gla Ming tigalye ban nilay ang Diya. //
	\glb ming tigal-ye ban nilay ang Diya //
	\glc can swim-\TsgF{} good probably \Aarg{} Diya //
	\glft `Diya can probably swim well.' //
\endgl

\a\label{ex:clitics_5b}\ljudge{*}\begingl
	\gla Ming nilay tigalye ban ang Diya. //
	\glb ming nilay tigal-ye ban ang Diya //
	\glc can probably swim-\TsgF{} well \Aarg{} Diya //
\endgl
\xe

Combinations of topic particle and modal particle, as well as modal particle
and verb, can likewise not be interrupted by parenthetical material like
\xayr{nrtNF}{naratang}{they say}, so that:

\pex\label{ex:clitics_6}
\a\label{ex:clitics_6a}\begingl
	\gla \textbf{Naratang,} ang ming tigalye ban {} Diya kodanya. //
	\glb nara=tang ang ming tigal-ye ban Ø Diya kodan-ya //
	\glc say=\TplM{}.\Aarg{} \AgtT{} can swim-\TsgF{} well \Top{} Diya 
		lake-\Loc{}	//
	\glft `They say Diya can swim well in a lake.' //
\endgl

\a \ljudge{*} Ang, \textbf{naratang,} ming tigalye ban Diya kodanya.
\a \ljudge{*} Ang ming, \textbf{naratang,} tigalye ban Diya kodanya.
\a \ljudge{?} Ang ming tigalye, \textbf{naratang,} ban Diya kodanya.
\a Ang ming tigalye ban, \textbf{naratang,} Diya kodanya.
\a Ang ming tigalye ban Diya, \textbf{naratang,} kodanya.
\a Ang ming tigalye ban Diya kodanya, \textbf{naratang.}

\xe

Besides verbs, nouns as well have preposed modifiers. This is the case with
proper nouns specifically, where the name is preceded by a case marker instead
of receiving a case-marking suffix like generic nouns do. This case marker is
phonologically weak in that it is no longer than other affixes, and unstressed,
with the exception of the causative case marker \rayr{saa}{sā}, which bears
at least secondary stress since it contains a long vowel. We already saw case
particles preceding names in (\ref{ex:clitics_1b}) and (\ref{ex:clitics_5})
above: \rayr{ANF tikimF}{ang Tikim} and \rayr{ANF diy}{ang Diya}; 
\rayr{ANF}{ang} marks the proper-noun NPs as agents in both cases. The case
marker is missing when the NP is topicalized, as indicated in 
(\ref{ex:clitics_6}), where the agent NP appears as just \rayr{diy}{Diya},
not \rayr{ANF diy}{ang Diya}. While case suffixes have narrow scope as in 
(\ref{ex:clitics_7a}) and thus need to be repeated on every NP in a conjunct,
preposed case markers may be used with wide scope if both conjuncts are proper
nouns as in (\ref{ex:clitics_7c}). Narrow scope with proper nouns may add an
individuating connotation, exemplified by (\ref{ex:clitics_7d}).

\pex\label{ex:clitics_7}
\a\label{ex:clitics_7a}\begingl
	\gla Toryon veneyang nay badanang. //
	\glb tor-yon veney-ang nay badan-ang //
	\glc sleep-\TplN{} dog-\Aarg{} and father-\Aarg{} //
	\glft `The dog and father are (both) sleeping.' //
\endgl

\a\label{ex:clitics_7b}\ljudge{*}\begingl
	\gla Toryon veney nay badanang. //
	\glb tor-yon veney\_ nay badan-ang //
	\glc sleep-\TplN{} dog\_ and father-\Aarg{} //
\endgl

\a\label{ex:clitics_7c}\begingl
	\gla Sa sobisayan ang Niva nay {} Mico narānye. //
	\glb sa sobisa-yan ang Niva nay \_ Mico narān-ye-Ø //
	\glc \PatT{} study-\TplM{} \Aarg{} Niva and \_ Mico 
		language-\Pl{}-\Top{} //
	\glft `Languages is what Niva and Mico study.' //
\endgl

\a\label{ex:clitics_7d}\begingl
	\gla Sa sobisayan ang Niva nay ang Mico narānye. //
	\glb sa sobisa-yan ang Niva nay ang Mico narān-ye-Ø //
	\glc \PatT{} study-\TplM{} \Aarg{} Niva and \Aarg{} Mico 
		language-\Pl{}-\Top{}//
	\glft `Languages is what Niva and Mico (each) study.' //
\endgl
\xe

% [The blog article here first postulates that the case markers might attach
% phonologically to whatever precedes them, which would make them very
% clitic-like in not being `choosy' about attachment only to find that this
% does not make sense. We will thus skip this unnecessary information as well
% as a bunch of examples.]

Taking the above characteristics into account---inability to insert word
material, special positioning, and wide scope---one may argue that the preposed
case markers are clitics. It should be noted furthermore that a single NP
cannot be marked for two grammatical functions at the same time, so that case
markers cannot be coordinated, as is tried in (\ref{ex:precasecoord}) below 
with *\rayr{s nj ːs sopnF}{*sa nay sā Sopan}:

\pex
\a\label{ex:precasecoord}\ljudge*\begingl
	\gla Ang delacan sa nay sā Sopan. //
	\glb ang delak=yan.Ø sa nay sā Sopan //
	\glc \AgtT{} suffer.from=\TplM{}.\Top{} \Parg{} and \Caus{} Sopan //
	\glft \textit{Intended:} `They suffer from and due to Sopan.' //
\endgl

\a\begingl
	\gla Ang delacan sa Sopan, nay yasa. //
	\glb ang delak=yan.Ø sa Sopan nay yasa //
	\glc \AgtT{} suffer.from=\TplM{}.Top{} \Parg{} Sopan and \TsgM{}.\Caus{} //
	\glft `They suffer from Sopan, and due to him.' //
	\endgl
\xe

The case markers of proper nouns are necessarily proclitics rather than
enclitics to preceding word material, since it is possible for them to begin
utterances, where it is not possible to lean to the left, but only to the 
right. This is the case in equative sentences such as the one in 
(\ref{ex:clitics_11a}). In these cases as well, it is not possible for
parenthetical material to be placed between the case marker and its target of
modification, as in (\ref{ex:clitics_11b}); the particle and its head cohere
closely and behave essentially like a unit.

\pex\label{ex:clitics_11}
\a\label{ex:clitics_11a}\begingl
	\gla Ang Misan lajāyas puti. //
	\glb ang Misan lajāy-as puti //
	\glc \Aarg{} Misan student-\Parg{} zealous //
	\glft `Misan is a zealous student.' //
\endgl

\a\label{ex:clitics_11b}\ljudge{*}\begingl
	\gla Ang, paronyang, Misan lajāyas puti. //
	\glb ang paron=yang Misan lajāy-as puti //
	\glc \Aarg{} believe=\Fsg{}.\Aarg{} Misan student-\Parg{} zealous //
\endgl
\xe

The fact that case particles attach always to a proper noun very specifically
makes them unlike `typical' clitics, however, since according to
\citet{spencerluis2012}, as indicated initially, a typical and important
feature of clitics is their `promiscuous' attachment. This puts case particles
closer to affixes—just like the suffixed case markers. On the other hand, as
written above, clitics do not have to exhibit all traits often associated with
them in order to be counted as such. More typical of function words, on the
other hand, is the fact that there is no morphophonemic interaction between the
particle and the word it inflects, for instance, there is no form /saːdʒaːn/
resulting from \rayr{s}{sa} (\Parg{}) combined with \rayr{AgYaanF}{Ajān}. This
overlap in form between affix and function word is typical of clitics,
according to the traits excerpted from \citet{spencerluis2012} above.

From this discussion of prenominal particles, let us return to verbs again for
a moment. Besides the preverbal particles discussed above, there is also what
is spelled as a prefix on the verb which appears to be a little odd as such in
that it can have wide scope over conjoined verbs. This is the prefix \rayr{d/}
{da-}often meaning `so, thus', displayed in (\ref{ex:clitics_13}).

\ex\label{ex:clitics_13}\begingl
	\gla Ang da-pinyaya nay hisaya {} Yan sa Pila. //
	\glb ang da=pinya-ya nay hisa-ya Ø Yan sa Pila //
	\glc \AgtT{} so=ask-\TsgM{} and beg-\TsgM{} \Top{} Yan \Parg{} Pila //
	\glft `Yan asks and begs Pila to (do so).' //
\endgl\xe

\rayr{d/}{da-}, where it is not used for presentative purposes,
\footnote{Although this use is probably related to the anaphoric use.} is a
functional morpheme in that it basically acts as an anaphora for a
complementizer phrase (CP) the speaker chooses to drop. Thus, it does not mark
any of the intrinsic morphological categories of the verb (tense, aspect, mood,
modality, finiteness), just like the topic marker refers to a syntactic
relation the verb subcategorizes but none of its inflectional categories. As an
anaphora, \rayr{d/}{da-} cannot stand alone, though it is possible to use a
full demonstrative form \xayr{dnY}{danya}{such one} in its place:

\ex\label{ex:clitics_14}\begingl
	\gla Ang pinyaya nay hisaya {} Yan sa Pila danyaley. //
	\glb ang pinya-ya nay hisa-ya Ø Yan sa Pila danya-ley //
	\glc \AgtT{} ask-\TsgM{} and beg-\TsgM{} \Top{} Yan \Parg{} Pila
		such.one-\PargI{} //
	\glft 'Yan asks and begs Pila such.' //
\endgl\xe

Unlike the preverbal particles, \rayr{d/}{da-} can be associated with a full
form, though it still displays special syntax in that unlike English
\fw{-n't} or \fw{'ll}, for instance, it does not occur in the same place as
the full form. Note also how \rayr{d/}{da-} is appended to the right of tense
prefixes, which do express a property of the verb, as shown in (
\ref{ex:clitics_15}).

\pex\label{ex:clitics_15}
\a\label{ex:clitics_15a}\begingl
	\gla Ang da-məpinyaya sa Pila. //
	\glb ang da=mə-pinya=ya.Ø sa Pila //
	\glc \AgtT{} so=\Pst{}=ask=\TsgM{}.\Top{} \Parg{} Pila[/gloss] //
	\glft `He asked Pila to.' //
\endgl

\a\label{ex:clitics_15b}\begingl
	\gla Ang da-məpinyaya nay məhisaya {} Yan sa Pila. //
	\glb ang da=mə-pinya-ya nay mə-hisa-ya Ø Yan sa Pila //
	\glc \AgtT{} so=\Pst{}-ask-\TsgM{} and \Pst{}-beg-\TsgM{} \Top{} Yan 
		\Parg{} Pila //
	\glft `Yan asked and begged Pila to.' //
\endgl
\xe

The verb form in (\ref{ex:clitics_15}) becomes ungrammatical with the order of
its prefixes reversed, so it is not acceptable to say: 
\rayr{md/pinYy}{məda-pinyaya}, although note that pre- and suffixes proper
also have a fixed order in Ayeri, so this alone is probably not enough evidence
to claim that \rayr{d/}{da-} is not possibly a prefix. Furthermore, while the
tense prefixes undergo crasis, this is not the case with \rayr{d/}{da-}:

\pex\label{ex:clitics_16}
\a\label{ex:clitics_16a}\begingl
	\gla Da-amangreng. //
	\glb da=amang=reng //
	\glc thus=happen=\TsgI{}.\Aarg{} //
	\glft `It happens thus.' //
\endgl

\a\label{ex:clitics_16b}\ljudge{*} \fw{Dāmangreng.}
\xe

\pex~\label{ex:clitics_17}
\a\label{ex:clitics_17a}\begingl
	\gla Māmangreng. //
	\glb mə-amang=reng //
	\glc \Pst{}-happen=\TsgI{}.\Aarg{} //
	\glft `It happened.' //
\endgl

\a\label{ex:clitics_17b}\ljudge{*} \fw{Məamangreng.}
\xe

Besides the characteristic of not seeking out certain parts of speech, the 
\rayr{d/}{da-} prefix at least satisfies the criteria of being a phonologically
reduced form of an otherwise free functional morpheme, and it occurs in a place where normal syntax would not put its corresponding full form. It has wide scope over conjuncts, is attached outside of inflection for proper categories of the verb, and doesn't interact with its host with regards to morphophonemics. Besides these more typical traits of clitics, there is also no way to place words between \rayr{d/}{da-} and the verb stem:

\ex\label{ex:clitics_18}\begingl
	\gla Da, naratang, amangreng. //
	\glb da nara=tang amang=reng //
	\glc thus say=\TplM{}.\Aarg{} happen=\TsgI{}.\Aarg{} //
\endgl\xe

The prefix \xayr{sitNF/}{sitang-}{self} behaves in the same way as \rayr{d/}
{da-}, since it also abbreviates a reflexive NP, for instance,
\xayr{sitNF/yesF}{sitang-yes}{herself} where `herself' as a patient is
coreferential with the agent of the clause. Reflexivity, however, is a category
a verb in Ayeri could be said to inflect for that way, though on the other
hand, Ayeri also does not have any verbs which are obligatorily reflexive to
indicate anticausativity like in Romance languages:

\pex\label{ex:clitics_19}
\a\label{ex:clitics_19a}\begingl
	\gla Adruara biratayreng. //
	\glb adru-ara biratay-reng //
	\glc break-\TsgI{} pot-\AargI{} //
	\glft `The pot broke.' //
\endgl

\a\label{ex:clitics_19b}\ljudge{*}\begingl
	\gla Sitang-adruara biratayreng. //
	\glb sitang=adru-ara biratay-reng //
	\glc self=break-\TsgI{} pot-\AargI{} //
	\glft \textit{Intended:} `The pot broke.' (an unspecified force broke it) //
\endgl
\xe

\pex~\label{ex:clitics_20}
\a\label{ex:clitics_20a}\begingl\rc{French}%
	\gla Le pot s'est cassé. //
	\glb le pot se=est cassé //
	\glc the pot self=be.\Tsg{}.\Prs{} broken //
	\glft `The pot broke.' (an unspecified force broke it) //
\endgl

\a\label{ex:clitics_20b}\begingl
	\gla Le pot est cassé. //
	\glb le pot est cassé //
	\glc the pot be.\Tsg{}.\Prs{} broken //
	\glft `The pot is broken.' //
\endgl
\xe

Ayeri has a tendency to reuse prefixes with different parts of speech, and thus
\rayr{d/}{da-} is also used with nouns, forming part of the series of deictic
prefixes, \xayr{d/}{da-}{such (a)}, \xayr{Ed/}{eda-}{this},
\xayr{Ad/}{ada-}{that}. The prefix in all these cases represents a grammatical
function, is unstressed, and may have wide scope over conjoined NPs, unless an
individuating interpretation is intended, as in (\ref{ex:clitics_21b}). These
traits are typical of clitics, as we have seen, though (\ref{ex:clitics_22})
shows that unlike with verbs, the deictic prefixes do undergo crasis here, 
which is a trait more typically associated with affixes.

\pex\label{ex:clitics_21}
\a\label{ex:clitics_21a}\begingl
	\gla Sinyāng eda-ledanas nay viretāyās tondayena-hen? //
	\glb sinya-ang eda=ledan-as nay viretāya-as tonday-ena=hen //
	\glc who-\Aarg{} this=friend-\Parg{} and supporter-\Parg{} 
		art-\Gen{}=all //
	\glft `Who is this friend and supporter of all arts?' //
\endgl

\a\label{ex:clitics_21b}\begingl
	\gla Sinyāng eda-ledanas nay eda-viretāyās tondayena-hen? //
	\glb sinya-ang eda=ledan-as nay eda=viretāya-as tonday-ena=hen //
	\glc sinya-\Aarg{} eda=ledan-\Parg{} nay eda=viretāya-\Parg{} 
		tonday-\Gen{}=hen //
	\glft 'Who is/are this friend and this supporter of all arts?' //
\endgl
\xe

\ex~\label{ex:clitics_22}\begingl
	\gla Sa ming nelnang edāyon. //
	\glb sa ming nel=nang eda=ayon-Ø //
	\glc \Parg{} can help=\Fpl{}.\Aarg{} this=man-\Top{} //
	\glft `This man, we can help him.' //
\endgl\xe

The deictic prefixes also cannot be used with all types of NPs, only with those
headed by generic and proper nouns; the picky nature of the deictic prefixes
also makes them more typical of affixes than of clitics. The preverbal
particles, on the other hand, also only occur with verbs, and it was 
nonetheless argued for them to be classified as clitics above.

As mentioned initially, \citet{spencerluis2012} give numerous counterexamples
to the catalog of traits typically associated with clitics. One of this
counterexample is what they term `suspended affixation'. This occurs in
Turkish, for instance, where the plural suffix \fw{-lEr} and subsequent
suffixes can be left out in coordination (\ref{ex:clitics_23a}), as well as
case markers (\ref{ex:clitics_23b}), and adverbials with case-like functions
(\ref{ex:clitics_23c}):

\pex\label{ex:clitics_23}
\a\label{ex:clitics_23a}\begingl\rc{Turkish}
	\gla bütün kitap(…) ve defter-ler-imiz //
	\glb all book and notebook-\Pl{}-\Fpl{}.\Poss{} //
	\glft `all our books and notebooks'\tc{\citep[199]{spencerluis2012}} //
\endgl

\a\label{ex:clitics_23b}\begingl
	\gla Vapur hem Napoli(…) hem Venedik'-e uğruyormuş //
	\glb boat and Naples and Venice-\Loc{} stops.\Evid{} //
	\glft `Apparently the boat stops at both Naples and Venice'\tc{\citep[199]
	{spencerluis2012}} //
\endgl

\a\label{ex:clitics_23c}\begingl
	\gla öğretmen-ler(…) ve öğrenci-ler-le //
	\glb teacher-\Pl{} and student-\Pl{}-\textsc{with} //
	\glft `with (the) students and (the) teachers'\tc{\citep[199]{spencerluis2012}}
	//
\endgl
\xe

\citet{spencerluis2012} note that, in \textquote{the nominal domain especially,
wide scope inflection is widespread in the languages of Eurasia, becoming more
prominent from west to east,} and that wide scope affixation \textcquote[200]
{spencerluis2012}{can be found with inflectional and derivational morphology in
a number of languages, and it is often a symptom of recent and not quite
complete morphologization}. They report \citet{wälchli2005} to find that this
is especially the case with `natural coordination', that is, the combination of
items very frequently occurring in pairs like \fw{knife and fork} or
\fw{mother and father}, as opposed to cases of occasional coordination 
\citep[200]{spencerluis2012}. Whether this is also true for Ayeri as of now
would require a separate survey.\footnote{Or rather, devising supplemental
rules.}

Given the evidence from Turkish, the categorization of deictic prefixes as
\emph{either} affixes \emph{or} clitics is unclear, especially since the
diagnostic of scope is devalued by the Turkish examples. On the other hand,
suffixes on nouns do not behave this way in Ayeri, as demonstrated in
(\ref{ex:clitics_24})---they rather behave like typical affixes in that they
mandatorily occur on each conjunct. The question is, thus, whether an exception
should be made for prefixes on nouns. We may as well assume that they are
clitics.

\pex\label{ex:clitics_24}
\a\label{ex:clitics_24a}\begingl
	\gla sobayajang nay lajāyjang //
	\glb sobaya-ye-ang nay lajāy-ye-ang //
	\glc teacher-\Pl{}-\Aarg{} and student-\Pl{}-\Aarg{} //
	\glft '(the) teachers and (the) students' //
\endgl

\a\label{ex:clitics_24b}\ljudge{*}\begingl
	\gla sobayaye nay lajāyjang //
	\glb sobaya-ye nay lajāy-ye-ang //
	\glc teacher-\Pl{} and student-\Pl{}-\Aarg{} //
\endgl

\a\label{ex:clitics_24c}\ljudge{*}\begingl
	\gla sobaya nay lajāyjang //
	\glb sobaya nay lajāy-ye-ang //
	\glc teacher and student-\Pl{}-\Aarg{} //
\endgl
\xe

From a functional point of view, the exact nature of the deictic prefixes
should not matter either way---\citet[Feature Table]{pargram} also cites a
[\Deix{}\textsc{is}] feature with \Prox{}\textsc{imal} and
\Dist{}\textsc{al} as its values, which fits \xayr{Ed/}{eda-}{this} and
\xayr{Ad/}{ada-}{that} just fine. At present it is unclear to me, however,
how to represent `such (a)' in this respect, since it is clearly deictic, but
neither proximal nor distal. In this case, I suppose, that it should be possible to use [\Deix{} this/that/such] as well, hence:

...

\section{Marking strategies}
\label{sec:markstrat}
\index{marking strategies}
\index{dependent marking}

With regards to the dichotomy head--dependent marking, Ayeri is rather  
thoroughly dependent marking, albeit with the exception of agreement 
morphology on the verb. Dependent marking is exhibited, for instance, in the 
expression of possessive relationships, where the dependent is marked for 
genitive case\index{cases!genitive}:

\begin{multicols}{2}
\pex
\a\label{ex:gennoun}\begingl
	\glpreamble \begin{forest}
	where n children=0{tier=word,edge=dotted,font=\itshape}{}
	[{dema}, for tree={calign=first}
		[{dema}, name=head]
		[{na Tuvo}, for tree={calign=first}
			[{\textbf{na} Tuvo}, name=dependent]
		]
	]
	\node at (current bounding box.south) [below=0pt of head]
		{\textsc{\tiny head}};
	\node at (current bounding box.south) [below=0pt of dependent] 
		{\textsc{\tiny dependent}};
	\end{forest} //
	\gla dema \textbf{na} Tuvo //
	\glb dema \textbf{na} Tuvo //
	\glc aunt \textbf{\Gen{}} Tuvo //
	\glft `Tuvo's aunt' //
\endgl

\a\label{ex:genprn}\begingl
	\glpreamble \begin{forest}
	where n children=0{tier=word,edge=dotted,font=\itshape}{}
	[{kasu}, for tree={calign=first}
		[{kasu}, name=head]
		[{bariri}, for tree={calign=first}
			[{bariri}]
		]
		[{nā}, for tree={calign=first}
			[{\textbf{nā}}, name=dependent]
		]
	]
	\node at (current bounding box.south) [below=0pt of head]
		{\textsc{\tiny head}};
	\node at (current bounding box.south) [below=0pt of dependent] 
		{\textsc{\tiny dependent}};
	\end{forest} //
	\gla kasu bariri \textbf{nā} //
	\glb kasu bari-ri \textbf{nā} //
	\glc basket meat-\Ins{} \Fsg{}.\textbf{\Gen{}} //
	\glft `my basket of meat' //
\endgl

\xe
\end{multicols}

In (\ref{ex:gennoun}), \rayr{tuvo}{Tuvo} is grammatically in possession of her 
\xayr{dem}{dema}{aunt}; the possessee forms the head of the phrase while it is 
modified by the possessor, which receives the marking. In (\ref{ex:genprn}), 
\xayr{ksu}{kasu}{basket} forms the head and thus also the possessee while 
\xayr{naa}{nā}{my} serves as the dependent possessor; the genitive case is, 
then again, marked on the dependent. A further example of dependent marking is 
the locative case, which is marked on the prepositional object while the 
preposition itself, as the head of the PP, does not receive marking:

\begin{multicols}{2}
\ex\label{ex:loc}
\begingl
	\gla agonan minkay\textbf{ya} //
	\glb agonan minkay\textbf{-ya} //
	\glc outside village\textbf{-\Loc{}} //
	\glft `outside of the village' //
\endgl
\xe

\smaller\begin{forest}
where n children=0{tier=word,edge=dotted,font=\itshape}{}
[{agonan}, for tree={calign=first}
	[{agonan}, name=head]
	[{minkayya}, for tree={calign=first}
		[{minkay\textbf{ya}}, name=dependent]
	]
]
\node at (current bounding box.south) [below=0pt of head]
	{\textsc{\tiny head}};
\node at (current bounding box.south) [below=0pt of dependent] 
	{\textsc{\tiny dependent}};
\end{forest}
\end{multicols}

The relativizer, likewise, may agree in case with the NP in the matrix clause
to which it links the relative clause. This typically happens mainly in formal
language and---in terms of linear succession of words at the surface level of
the clause---if the relativizer cannot be immediately adjacent to the NP which
the relative clause modifies, for example, when an adjective or a possessive
pronoun is following the noun:

%\begin{minipage}[t][11em][t]{\linewidth}
\begin{multicols}{2}
\ex
\begingl
	\gla sangalas kivo s\textbf{as} … //
	\glb sangal-as kivo s\textbf{-as} … //
	\glc room-\Parg{} small \Rel{}\textbf{-\Parg{}} … //
	\glft `the small room which …' //
\endgl
\xe

\smaller\begin{forest}
where n children=0{tier=word,edge=dotted,font=\itshape}{}
[{sangalas}, for tree={calign=first}
	[{sangalas}, name=head]
	[{kivo}, for tree={calign=first}
		[{kivo}]
	]
	[{sas}, for tree={calign=first}
		[{s\textbf{as}}, name=dependent]
		[{...}, for tree={calign=first}
			[{...}]
		]
	]
]
\node at (current bounding box.south) [below=0pt of head]
	{\textsc{\tiny head}};
\node at (current bounding box.south) [below=0pt of dependent] 
	{\textsc{\tiny dependent}};
%
\coordinate [below=1em of head] (A);
\coordinate [below=1.75em of head] (B);
\coordinate [below=1.75em of dependent] (C);
\coordinate [below=1em of dependent] (D);
\draw [-latex] (A) -- (B) -- (C) -- (D);
\node (label) at ($(B)!0.5!(C)$) [below] {\tiny\itshape case agreement};
\end{forest}
\end{multicols}
%\end{minipage}

The only instance of head-marking there is in Ayeri is person-marking on the
verb, which manifests when the NP following the verb (agent or patient) is not
pronominal and thus there is no pronoun to cliticize to the verb stem, but the
verb still receives a suffix that indicates a relation with, usually, the agent
NP:

\begin{multicols}{2}
\ex\begingl
	\gla Mal\textbf{ya} ang Amān. //
	\glb Mal\textbf{-ya} ang Amān //
	\glc sing\textbf{-\TsgM{}} \Aarg{} Amān //
	\glft `Amān sings.' //
\endgl\xe

\smaller\begin{forest}
where n children=0{tier=word,edge=dotted,font=\itshape}{}
[{Malya}, for tree={calign=first}
	[{Mal\textbf{ya}}, name=head]
	[{ang Amān}, for tree={calign=first}
		[{ang Amān}, name=dependent]
	]
]
\node at (current bounding box.south) [below=0pt of head]
	{\textsc{\tiny head}};
\node at (current bounding box.south) [below=0pt of dependent] 
	{\textsc{\tiny dependent}};
%
\coordinate [below=1em of dependent] (A);
\coordinate [below=1.75em of dependent] (B);
\coordinate [below=1.75em of head] (C);
\coordinate [below=1em of head] (D);
\draw [-latex] (A) -- (B) -- (C) -- (D);
\node (label) at ($(B)!0.5!(C)$) [below] {\tiny\itshape person agreement};
\end{forest}

\end{multicols}

Sentences containing more than one NP also have topic marking on the verb, so 
that the verb agrees with one of the NPs in topicality. This may be the NP it 
has person agreement with or any other NP. The topicalized NP as a dependent 
of the verb is, in turn, zero-marked, so that the marking relationship for 
topics is bilateral and thus mixed:

\begin{multicols}{2}
\ex[glspace=0.4em]\begingl
	\gla \textbf{Sa} manya ang Ajān {} \textbf{Pila}. //
	\glb \textbf{Sa} man-ya ang Amān \textbf{Ø}=\textbf{Pila} //
	\glc \textbf{\PatT{}} greet-\TsgM{} \Aarg{} Ajān \textbf{\Top{}}=%
		\textbf{Pila} //
	\glft `Pila, Ajān greets her.' //
\endgl\xe

\smaller\begin{forest}
where n children=0{tier=word,edge=dotted,font=\itshape}{}
[{Sa manya}, for tree={calign=first}
	[{\textbf{Sa} manya}, name=head]
	[{ang Ajān}, for tree={calign=first}
		[{ang Ajān}]
	]
	[{Pila}, for tree={calign=first}
		[{\textbf{Ø Pila}}, name=dependent]
	]
]
\node at (current bounding box.south) [below=0em of head]
	{\textsc{\tiny head}};
\node at (current bounding box.south) [below=0em of dependent] 
	{\textsc{\tiny dependent}};
%
\coordinate [below=1em of dependent] (A);
\coordinate [below=1.75em of dependent] (B);
\coordinate [below=1.75em of head] (C);
\coordinate [below=1em of head] (D);
\draw [-latex] (A) -- (B) -- (C) -- (D);
\node (label) at ($(B)!0.5!(C)$) [below] {\tiny\itshape topic agreement};
\end{forest}

\end{multicols}

In this example, the verb exhibits canonic agreement with the agent, 
\rayr{AgYaanF}{Ajān}, in person, gender, and number. It is additionally marked 
for a patient topic, \rayr{pil}{Pila}, and thus serves as an agreement target 
for two different controller NPs. The agreement relation is of a different kind
for each of the two NPs, however.

\index{typology!of morphemes|)}
