% kate: word-wrap true;

\chapter{Morphological typology}
\index{typology!of morphemes|(}

The first chapter dealt with the smallest constituent parts of words---speech 
sounds, which ones there are, and how they assemble into valid words. 
Consequentially, the following chapters will be about the next step up from 
this: morphemes, the atoms of meaning. First we will have a more general look 
at which kinds of morphemes there are, and then look at them more closely by 
part of speech. This chapter on morphological typology will first deal with 
general questions about Ayeri's degree of synthesis, and then will try 
to answer questions about the kinds of processes the various morpheme classes 
carry out in the language.

\section{Typology}

For the largest part, Ayeri is an \emph{agglutinative}\index{agglutination} 
language since it modifies word roots with affixes for the purposes of 
inflection and derivation, and these affixes, in the form of 
suffixes\index{suffixes} more specifically, can be stacked, especially on 
verbs; the average number of morphemes per word is thus greater than 1:

\ex\begingl
	\gla Le kondasayāng hemaye pruyya nay napayya kayvay. //
	\glb Le kond-asa-yāng hema-ye-Ø pruy-ya nay napay-ya kayvay //
	\glc \PatTI{} eat-\Hab{}=\TsgM{}.\Aarg{} egg-\Pl{}-\Top{} salt-\Loc{} 
		and pepper-\Loc{} without //
	\glft `He always eats his eggs without salt and pepper.' //
\endgl\xe

The verb root \xayr{koMd/}{kond-}{eat} is inflected here for a habitual action 
with the suffix \rayr{/As}{-asa}, and also carries a person-inflection 
clitic,\index{clitics} \rayr{/yaaNF}{-yāng}, marking a third person singular 
masculine agent. With the exception of person-inflection clitics, affixes tend 
to encode a single grammatical function. Verbs are not the only part of speech 
that can inflect; nouns and the relativizing conjunction can as well:

\pex
\a\label{ex:letters}\begingl
	\gla Ang mətahanay tamanyeley yeyam. //
	\glb Ang mə-tahan-ay.Ø taman-ye-ley yeyam. //
	\glc \AgtT{} \Pst{}-write=\Fsg{}.\Top{} letter-\Pl{}-\PargI{} 
		\TsgF{}.\Dat{} //
	\glft `I wrote letters to her.' //
\endgl

\a\label{ex:relative}\begingl
	\gla Le turayāng taman sinā ang ningay tamala vās. //
	\glb Le tura-yāng taman-Ø si-Ø-na ang ning=ay.Ø tamala vās //
	\glc \PatTI{} send=\Tsg{}.\M{}.\Aarg{} letter-\Top{} 
		\Rel{}-\PatTI{}-\Gen{} \AgtT{} tell=\Fsg{}.\Top{} yesterday 
		\Ssg{}.\Parg{} //
	\glft `The letter which I told you about yesterday, he sent it.' //
\endgl
\xe

The principle of not conflating several grammatical functions into a single 
suffix can be observed in (\ref{ex:letters}) regarding the word 
\xayr{tmnFyelej}{tamanyeley}{letters}, in which the plural marker 
\rayr{/ye}{-ye} is distinct from the inanimate-patient case marker 
\rayr{/lej}{-ley}. Strictly speaking, the pronoun \xayr{yeymF}{yeyam}{to her} 
is also composed, namely of the third person feminine base form \rayr{ye}{ye} 
and the dative case marker \rayr{/ymF}{yam}. Example (\ref{ex:relative}) is one 
we have already encountered before (p.~\pageref{doublerel}). Here, the relative 
pronoun, \xayr{sinaa}{sinā}{of/about which} is inflected for genitive case, and 
stress on the usually unstressed last syllable 
suprasegmentally\index{suprasegmental} marks that this form is contracted from 
\rayr{sileyen}{sileyena} (\textit{si-ley-ena}, \Rel{}-\PargI{}-\Gen{}).

So far, we have concentrated on suffixes, but there are a number of 
prefixes\index{prefixes} as well; (\ref{ex:letters}) exhibits the past prefix 
\rayr{m/}{mə-} (which is actually redundant in this case). There are also 
deictic prefixes on nouns, however. In the following example, the prefix 
\xayr{Ed/}{eda-}{this-} joins the noun \xayr{pehmF}{peham}{carpet} to indicate a 
specific carpet.

\ex\begingl
	\gla Le no intoyyang eda-peham. //
	\glb Le no int-oy-yang eda-peham-Ø //
	\glc \PatTI{} want buy-\Neg{}=\Fsg{}.\Aarg{} this=carpet-\Top{} //
	\glft `I do not want to buy this carpet.' //
\endgl\xe

Besides prefixes and suffixes, Ayeri also possesses at least one grammatical 
morpheme of the kind \citet{zwicky1977} calls a `bound word'.\index{bound 
words} These are cases where morphemes which are

\blockcquote[6]{zwicky1977}{always bound and always unaccented show 
considerable syntactic freedom, in the sense that they can be associated with 
words of a variety of morphosyntactic categories. Frequently, such a \emph{bound 
word} is semantically associated with an entire consituent while being 
phonologically attached to one word of this constituent, and ordinarily the 
bound word is located at the very margins of the word, standing outside even 
inflectional affixes.}

This is the case with the marker \rayr{mN}{manga}, which is treated as an 
independent word, but can modify verbs and prepositions---heads of verb 
phrases (VPs) and prepositional phrases (PPs), respectively---is unstressed and 
appears at the margin of its modification target:

\pex
\a\label{ex:prog}\begingl
	\gla Ang manga yavaya ayon bariley. //
	\glb Ang manga yava-ya ayon-Ø bari-ley //
	\glc \AgtT{} \Prog{} roast-\TsgM{} man-\Top{} meat-\PargI{} //
	\glft `The man is roasting meat.' //
\endgl

\a\label{ex:dyn}\begingl
	\gla Ya mətapyyāng maritay misley manga luga bari. //
	\glb Ya mə-tapy-yāng maritay mis-ley manga luga bari-Ø //
	\glc \LocT{} \Pst{}-put=\TsgM{}.\Aarg{} before spit-\PargI{} \Dyn{} 
		between meat-\Top{} //
	\glft `The meat, he had put a spit through it before.' //
\endgl

\xe

In (\ref{ex:prog}), \rayr{mN}{manga} modifies the verb \xayr{yv/}{yava-}{roast} 
and indicates that this is a temporarily ongoing action, like the English 
progressive, except not as strongly grammaticalized.\footnote{I suppose, a 
better parallel is the so-called \fw{rheinische Verlaufsform} `Ripuarian 
progressive' (\fw{sein} `be' + \fw{am/beim} `at the' + infinitive) in German, a 
construction common in the colloquial language which parallels the English 
progressive construction and is not yet fully grammaticalized 
\citep[435]{dudengram2016}. Speakers will thus accept both \fw{Er lernt 
gerade}, literally `He studies right now', and \fw{Er ist am Lernen} `He is 
studying'.
% 
% \ex[lingstyle=fnex,belowexskip=-1em]\label{ex:ripprog}\begingl
% 	\gla Der Mann ist Fleisch am Braten. //
% 	\glc The man is meat at.the roasting //
% 	\glft `The man is roasting meat.' //
% \endgl\xe
}
%
In (\ref{ex:dyn}), \rayr{mN}{manga} modifies the preposition, on the other 
hand, to indicate that it is dynamic: \rayr{lug}{luga} by itself means `among, 
between', while its dynamic form \rayr{mN lug}{manga luga} means `through; 
during, while, for'.

As we have seen in the examples above, person suffixes on verbs are single 
morphemes that encode more than one property, for example \rayr{/yeNF}{-yeng} 
encodes the person features third person, feminine, singular, and agent. 
Personal pronouns,\index{pronouns!personal} of which the person clitics on 
verbs are an instance, are the main case of fusion among agglutination in 
Ayeri, although some of the forms, like \xayr{yeymF}{yeyam}{to her} above, can 
be decomposed into root and suffix without problem.\footnote{Originally, 
Ayeri's personal pronouns were indeed agglutinative as well, so 
\xayr{yeNF}{yeng}{she} used to be \rayr{Iye\_aNF}{iyeang} (\fw{iy-e-ang}, 
\Tsg{}-\F{}-\Aarg{}). This also gives an explanation to \citet{boga2016}'s 
observation that Ayeri's plural pronouns are formed \textcquote[{[}15{]}; 
`possibly in an even too regular way']{boga2016}{[v]ielleicht sogar zu 
regelmäßig}.}

Perpendicular to the axis isolation–agglutination runs the axis 
analytic–syn\-thetic. On the latter axis, Ayeri scores mostly as 
\emph{synthetic}, since it prefers compactness over spreading a construction 
over several words, though it does not incorporate object noun phrases (NPs), 
so it is not going as far as being poly\-syn\-thetic. It is nonetheless 
theoretically possible, due to suffixation being a prominent pattern, to form 
foot-long words like

\ex\label{ex:footlong}\begingl
	\gla da-mətahasongoyyang-ikan //
	\glb da=mə-taha-asa-ong-oy=yang=ikan //
	\glc such=\Pst{}-have-\Hab{}-\Irr{}-\Neg{}=\Fsg{}.\Aarg{}=much //
	\glft `I would not much used to have had such' //
\endgl\xe

Cases of analytic morphology are compound prepositions as we have seen 
one in (\ref{ex:dyn}), but verbs as well show analytic structures not only with 
the progressive marker, but also with modals:

\ex\begingl
	\gla Ming sahoyyang dabas. //
	\glb Ming saha-oy-yang dabas //
	\glc can come-\Neg{}=\Fsg{}.\Aarg{} today //
	\glft `I can't come today.' //
\endgl\xe

Most of the information the VP contains in this example is marked on the 
content verb, \xayr{sh/}{saha-}{come}, except for ability, which is expressed 
by the particle \xayr{miNF}{ming}{can}. \rayr{miNF}{ming} is an uninflected 
form 
of the verb expressing ability and may be counted as an auxiliary verb in 
that the full semantic content of the VP is spread out over two verb forms, 
one major, one minor.\footnote{\rayr{mN}{manga} has, in fact, a verbal 
counterpart \xayr{mN/}{manga-}{move; remove} as well, which served as the 
origin 
of both the progressive and the dynamic marker.} Consider also the following 
example in which \rayr{miNF}{ming} is inflected like a regular verb:

\ex\begingl
	\gla Da-mingya ang Diyan. //
	\glb Da-ming-ya ang Diyan. //
	\glc so-can-\TsgM{} \Aarg{} Diyan //
	\glft `Diyan can (do it).' //
\endgl\xe

\section{Morphological processes}

\subsection{Prefixation}
\index{prefixes|(}

Prefixes in Ayeri apply mainly to verbs, but nouns, pronouns, adjectives and 
conjunctions can also appear with them, some of which may be clitics; further 
tests need to be performed to determine their exact status.
% FIXME
With verbs, prefixes that are most certainly `true' prefixes---that is, 
morphemes that have been semantically bleached by grammaticalization to the 
point where they only express grammatical functions \citep[157ff.]{lehmann2015} 
and which subcategorize words rather than phrases 
\citep[117]{klavans1985}---are the tense prefixes marking both three degrees of 
past tense and three degrees of future tense, for example:

\ex\begingl
	\gla Ang səsarāyn ya Makapetang. //
	\glb Ang sə-sara-ayn.Ø ya Makapetang //
	\glc \AgtT{} \Fut{}-go=\Fpl{}.\Top{} \Loc{} Makapetang //
	\glft `We will go to Makapetang.' //
\endgl\xe

Here, the prefix \rayr{se/}{sə-} marks future tense on the verb, 
\xayr{sr/}{sara-}{go}. The other tense\index{tense} prefixes are \rayr{k/}{kə-} 
(\NPst{}), \rayr{m/}{mə-} (\Pst{}), \rayr{v/}{və-} (\RPst{}), and 
\rayr{p/}{pa-} 
(\NFut{}) and \rayr{ni/}{ni-} (\RFut{}). Besides this set of prefixes, there 
are 
also a number of proclitics that can appear with verbs, though not exclusively. 
These are the anaphora \xayr{d/}{da-}{thus, so, such} and the 
reflexive\index{pronouns!reflexive} marker \xayr{sitNF/}{sitang-}{self}:

% \pex
% \a\begingl
\ex\begingl
	\gla Da-mingya ang Diyan. //
	\glb Da-ming-ya ang Diyan. //
	\glc so-can-\TsgM{} \Aarg{} Diyan //
	\glft `Diyan can (do it).' //
\endgl
% 
% \a\begingl
% 	\gla Da-sahāra seyaraneng. //
% 	\glb Da-saha-ara seyaran-eng //
% 	\glc thus-come-\TsgI{} rain-\AargI{} //
% 	\glft `Here/Thus comes the rain.' //
% \endgl
\xe

\ex~\begingl
	\gla Sitang-kecāng. //
	\glb Sitang-ket-yāng //
	\glc self-wash=\TsgM{}.\Aarg{} //
	\glft `He washes \emph{himself}.' //
\endgl\xe

\rayr{sitNF/}{sitang-} can also be used as a preverb in situations where the 
agent is also the instrument, so both of the following two sentences are 
equivalent in meaning:

\pex
\a\label{ex:sitang+pronoun}\begingl
	\gla Sa apicāng nanga ikan sitang-yari. //
	\glb Sa apit-yāng nanga ikan sitang-yari //
	\glc \PatT{} clean=\Tsg{}.\Aarg{} house complete self-\TsgM{}.\Ins{} //
	\glft `He cleaned the whole house by himself.' //
\endgl

\a\begingl
	\gla Sa sitang-apicāng nanga ikan. //
	\glb Sa sitang-apit-yāng nanga ikan //
	\glc \PatT{} self-clean=\Tsg{}.\Aarg{} house complete //
	\glft (idem) //
\endgl
\xe

\phantomsection\label{nounprefixes}
Example (\ref{ex:sitang+pronoun}) shows the more common application of 
\rayr{sitNF/}{sitang-}, that is, as a reflexive modifier of pronouns. The 
prefix \rayr{d/}{da-} can as well be used with noun phrases and is part of the 
demonstrative set of prefixes, \xayr{d/}{da-}{such}, \xayr{Ed/}{eda-}{this}, 
and \xayr{Ad/}{ada-}{that}:

\ex\begingl
	\gla eda-ganang //
	\glb eda-gan-ang //
	\glc this=child-\Aarg{} //
	\glft `this child' //
\endgl\xe

The demonstrative prefixes are also used to form the demonstrative 
pronouns\index{pronouns!demonstrative} \xayr{EdnY}{edanya}{this one}, 
\xayr{AdnY}{adanya}{that one} and \xayr{dnY}{danya}{such one}. A special case 
is this regard is the postposition \xayr{d/naarY}{da-nārya}{in spite of, 
despite} where \rayr{d/}{da-} combines with the conjunction 
\xayr{naarY}{nārya}{but, although, except}. There is also a fixed adverbial 
expression using one of thse prefixes, \xayr{Ed/tdjymF}{eda-tadayyam}{for the 
time being, for now} (this=time-\Dat{}).

Last but not least, the prefix \xayr{ku/}{ku-}{like, as though} can be used 
with both adjectives and nouns (or, more precisely, phrases containing 
nominals):

\pex
\a\begingl
	\gla ku-koyaya //
	\glb ku-koya-ya //
	\glc like=book-\Loc{} //
	\glft `like in a book' //
\endgl

\a\begingl
	\gla ku-prasi //
	%\glb ku-prasi //
	\glb like=sour //
	\glft `as though (it were) sour' //
\endgl
\xe

An example of a set-phrase adverbial consisting of \rayr{ku/}{ku-} and a verb 
is \xayr{ku/nsY}{ku-nasya}{as follows}, \rayr{nsY/}{nasy-} meaning `follow'. 
What is curious here is that this fossilized form is lacking person marking 
and is just extended with an epenthetic \textit{-a} since \textit{-sy} is not 
a permissible coda. The expected form would be 
*\,\rayr{ku/nsYreNF}{*ku-nasyareng} (like=follow=\TsgI{}.\Aarg{}).

Following \citet{klavans1985}, who suggests that clitics best be defined as 
\textcquote[117]{klavans1985}[,]{affixation at the phrasal level} a very common 
kind of clitical prefix to the verb \emph{phrase} are the topic markers. They 
are counted as parts of the VP but do not interact with it regarding stress 
assignment (they are always unstressed) while always being in an initial 
position, preceding any other preverbal elements:

\pex
	\a\begingl
		\gla Ang tahanya tamanley. //
		\glb Ang tahan-ya taman-ley //
		\glc \AgtT{} write-\TsgM{} letter-\PargI{} //
		\glft `He writes a letter.' //
	\endgl
	\a \textit{Ang mətahanya tamanley.} `He wrote a letter.'
	\a \textit{Ang manga mətahanya tamanley.} `He was writing a letter.'
	\a \textit{Ang manga no mətahanya tamanley.} `He was wanting to write a 
		letter.'
\xe

\index{prefixes|)}

\subsection{Suffixation}
\index{suffixes|(}

As a largely agglutinative language, most grammatical marking in Ayeri is done 
by means of suffixes. These occur mainly with nouns and verbs, however, 
quantifiers take the shape of suffixes as well. Quantifiers, then, may modify 
content words almost regardless of their part of speech---noun, verb, adjective 
or adverb. The most pervasive examples of suffixation are certainly those of 
case marking on nouns and of person marking on verbs, for example:

\ex\label{ex:conjdecl}\begingl
	\gla Sa pəharuyang va manga miday tangya vana suyareri, vimyon! //
	\glb Sa pə-haru-yang va.Ø manga miday tang-ya vana suyar-eri, vimyon //
	\glc \PatT{} \NFut{}-beat=\Fsg{}.\Aarg{} \Ssg{}.\Top{} \Dyn{} around 
		ears-\Loc{} \Ssg{}.\Gen{} ladle-\Ins{}, monkey! //
	\glft `I'll beat you around your ears with a ladle, you monkey!' //
\endgl\xe

This example shows marking of \xayr{tNF}{tang}{ears} with the locative case 
suffix \rayr{/y}{-ya} and the marking of \xayr{suyrF}{suyar}{ladle} with the 
instrumental case suffix \rayr{/Eri}{-eri}; the previous examples already 
provide instances of the exceedingly common markers for agent and patient 
case, \rayr{/ANF}{-ang} and \rayr{/AsF}{-as}, respectively. Besides case, nouns 
can also be marked for plural with the suffix \rayr{/ye}{-ye}, and verb roots 
may be extended by the mood markers \rayr{/ONF}{-ong} (\Irr{}), 
\rayr{/As}{-asa} (\Hab{}) and \rayr{/Oj}{-oy} (\Neg{}), the last of which should 
be the most frequently occurring. The mood suffixes can also be stacked, leading 
to the long word in (\ref{ex:footlong}) above. Person marking on verbs comes as 
agreement suffix or as a clitic personal pronoun depending on whether an agent 
NP proper is present or not for the verb to agree with; in (\ref{ex:conjdecl}), 
a cliticized agent pronoun \xayr{/yaaNF}{-yāng}{he} (\TsgM{}.\Aarg{}) appears.

As mentioned above, quantifiers appear as enclitics on almost any type of 
content word, like on the adverb \xayr{pr}{para}{fast} in the following example:

\ex
%\pex
% \a\begingl
% 	%\glpreamble With a verb: //
% 	\gla No sarayang-ikan //
% 	\glb No sara-yang-ikan //
% 	\glc want go=\Fsg{}.\Aarg{}=much //
% 	\glft `I really want to go.' //
% \endgl
%\a
\begingl
	%\glpreamble With an adverb: //
	\gla Tigalyeng para-ma. //
	\glb Tigal-yeng para-ma //
	\glc swim=\TsgF{}.\Aarg{} fast=enough //
	\glft `She swims fast enough.' //
\endgl

% \a\begingl
% 	%\glpreamble With a predicative adjective: //
% 	\gla Yang valuy-eng, sahavāng. //
% 	\glb Yang valuy-eng, saha-vāng. //
% 	\glc \Fsg{}.\Aarg{} glad=rather, come=\Ssg{}.\Aarg{} //
% 	\glft `I am rather glad that you come.' //
% \endgl
% 
% \a\begingl
% 	%\glpreamble With an attributive adjective: //
% 	\gla Adareng bahisley mino-ing //
% 	\glb Ada-reng bahis-ley mino-ing //
% 	\glc that=\TsgI{}.\Aarg{} day-\PargI{} happy=so //
% 	\glft `It was such a happy day.' //
% \endgl
% 
% \a\begingl
% 	%\glpreamble With a noun: //
% 	\gla Ang konjan prikanley-ani //
% 	\glb Ang kond-yan.Ø prikan-ley-ani //
% 	\glc \Aarg{} eat=\TsgM{}.\Top{} soup=not.at.all //
% 	\glft `They did not eat any soup at all.' //
% \endgl

\xe

\index{suffixes|)}

\subsection{Reduplication}
\index{reduplication|(}

There are two patterns of reduplication for verbs, one with complete 
reduplication of the imperative form to create a hortative statement 
(\ref{ex:hort}), and one with partial reduplication as a way to express that an 
action takes place again, that is, partial reduplication expresses a 
frequentative, as it were (\ref{ex:iter}). The imperative iterative, then, has a 
hortative function as well (\ref{ex:hort+iter}):

\pex
\a\label{ex:hort}\begingl%
	\gla naru-naru //
	\glb nara-u-nara-u //
	\glc speak-\Imp{}\til\Hort{} //
	\glft `let us speak' //
\endgl

\a\label{ex:iter}\begingl
	\gla na-narayeng //
	\glb na-nara-yeng //
	\glc \Iter{}\til{}speak=\TsgF{}.\Aarg{} //
	\glft `she speaks again' //
\endgl

\a\label{ex:hort+iter}\begingl
	\gla na-naru //
	\glb na-nara-u //
	\glc \Iter{}\til{}speak-\Imp{} //
	\glft `let us speak again' //
\endgl

\xe

With nouns, full reduplication is used to create a diminutive\index{diminutive} 
form (\ref{ex:regdim}), though some reduplications are also lexicalized and 
may use roots from other parts of speech as well to form nouns, for instance, 
the words in (\ref{ex:otherredupnn}--\hyperref[ex:otherredupvb]{d}). There are 
also a number of adjectives for which there exists a lexical reduplication with 
an intensifying meaning; (\ref{ex:adjredup}) lists a few examples. This, 
however, is not a productive derivation strategy.

\pex
	\a \makebox[8em][l]{\tayr{veney}{dog}} → \tayr{veney-veney}{little dog, 
		doggie} \label{ex:regdim}
	\a \makebox[8em][l]{\tayr{gan}{child}} → \tayr{gan-gan}{grandchild} 
		\label{ex:otherredupnn}
	\a \makebox[8em][l]{\tayr{kusang}{double (adj.)}} → 
		\tayr{kusang-kusang}{model} \label{ex:otherredupadj}
	\a \makebox[8em][l]{\tayr{veh-}{build}} → \tayr{veha-veha}{tinkering} 
		\label{ex:otherredupvb}
\xe

\pex~\label{ex:adjredup}
	\a \makebox[5em][l]{\tayr{apan}{wide}} → \tayr{apan-apan}{extensive}
	%\a \makebox[5em][l]{\tayr{ikan}{complete}} → \tayr{ikan-ikan}{entire, 
	%	total}
	\a \makebox[5em][l]{\tayr{kebay}{alone}} → \tayr{kebay-kebay}{all alone}
	%\a \makebox[5em][l]{\tayr{pakas}{special}} → \tayr{pakas-pakas}{gay}
	\a \makebox[5em][l]{\tayr{pisu}{tired}} → \tayr{pisu-pisu}{exhausting}
\xe

\index{reduplication|)}

\subsection{Suprasegmental modification}
\index{suprasegmental|(}

\index{morphophonology!of relative pronouns}
As written above (\autoref{doublerel}), the case agreement on a complex-marked 
relative pronoun\index{pronouns!relative} can drop out under certain 
circumstances and is replaced by compensatory stress on the secondary case 
marker, which lengthens the syllable's nucleus vowel:

\ex\begingl
	\gla … tamanley sinā (*sina) ang ningay tamala vās //
	\glb … [taman-ley]₁ si-Ø₁-na (*si-na₁) ang ning=ay.Ø tamala vās //
	\glc … letter-\PargI{} \Rel{}-\PatTI{}-\Gen{} (*\Rel{}-\Gen{}) \AgtT{} 
		tell=\Fsg{}.\Top{} yesterday \Ssg{}.\Parg{} //
	\glft `… the letter which (*whose) I told you about yesterday' //
\endgl\xe

This can be reinterpreted so that the vowel length/stress itself is what 
signifies the agreement of the relativizer with the preceding NP. Which 
grammatical role the relativizer's head instantiates as an agreement controller
is essentially underspecified, hence I will gloss it as -\Agr{} in the 
following example instead of as full -\PargI{}:

\ex[everygla=\upshape]\begingl
	\gla /ˌsi.leɪ.ˈena/ → /si.ˈna(ː)/ //
	\glb /si-leɪ-ena/ → /si-ˈ-na(ː)/ //
	\glc \Rel{}-\PargI{}-\Gen{} {} \Rel{}-\Agr{}-\Gen{} //
\endgl\xe

Since \rayr{n}{na} as a light syllable cannot be stressed in word-final 
position under normal circumstances, it has to lengthen to \rayr{naa}{nā}.

\index{suprasegmental|)}

% \subsection{Clitics}
% 
% I have used the term `clitic' above and claimed that various bound morphemes 
% belong to this category. However, this term is notoriously vague and thus needs 
% some specification and formalization as to why certain morphemes in Ayeri are 
% probably best categorized as clitics rather than words or affixes. After the 
% publication of \citeauthor{zwicky1977}'s preliminary thoughts on the topic 
% \citep{zwicky1977}, a number of articles were published that tried to provide a 
% formal definition of what constitutes a clitic. \citet{zwicky1985} and 
% \citet{klavans1985} provide guidelines for determining whether a given element 
% which is suspected to be a clitic should better be categorized as a `particle' 
% word or as a suffix, or whether it is indeed an in-between thing, and thus a 
% clitic. \autoref{tab:clitichood} lists a catalog of 
% \textcquote[285]{zwicky1985}{symptoms} that is supposed to serve as a 
% heuristic in determining the status of the various bound morphemes listed at 
% the top and tries to categorize them by keeping score:
% 
% \begin{longtabu} to .75\linewidth {X[1r] X[6l]}
% 0 & `does not apply',\\
% 0.5 & `applies under certain circumstances', and\\
% 1 & `applies'.\\
% \end{longtabu}\addtocounter{table}{-1}
% 
% The table mostly contains `1' in the various fields, insofar most of the given 
% criteria for clitichood are met. Thus, the few places in which `0' or `0.5' 
% appear demand further discussion, as do the cases where the difference 
% quotient indicates a narrow decision. A `0' answer is also interesting as it 
% means that the morpheme is behaving uncharacteristically of clitics for the 
% respective criterion.
% 
% ...
% 
% \begin{table}[p]
% \caption{Clitichood tests according to \citet{klavans1985} and 
% 	\citet{zwicky1985}}
% \smaller
% \begin{tabu} to \linewidth {H[10l] X[c] X[c] X[c] X[c] X[c] X[c] X[c] X[c] X[c]}
% \toprule\tableheaderfont
% 
% Symptom
% 	& \rotatebox{90}{topic markers + VP}
% 	& \rotatebox{90}{dynamic marker + V'}
% 	& \rotatebox{90}{V' + pronoun}
% 	& \rotatebox{90}{da-}
% 	& \rotatebox{90}{eda-/ada-}
% 	& \rotatebox{90}{quantifiers}
% 	& \rotatebox{90}{case marker + N'}
% 	& \rotatebox{90}{ku- + NP}
% 	& \rotatebox{90}{mə- + N'}
% 	\\
% 
% \toprule
% \tableheaderfont\multicolumn{9}{c}{Clitic vs. word 
% 	\citep[286-289]{zwicky1985}} \\
% \toprule
% 
% 1.1 Word-internal sandhi applies
% 	& 0	% topic markers + VP
% 	& 0	% dynamic marker + V’
% 	& 1	% person marker
% 	& 1	% da-
% 	& 1	% eda-/ada-
% 	& 0.5	% quantifiers
% 	& 0	% case marker + N’
% 	& 1	% ku- + NP
% 	& 1	% mə- + N’
% 	\\ \midrule
% 
% 1.2 Word accent rules apply
% 	& 0	% topic markers + VP
% 	& 0	% dynamic marker + V’
% 	& 1	% person marker
% 	& 0.5	% da-
% 	& 0.5	% eda-/ada-
% 	& 0.5	% quantifiers
% 	& 0	% case marker + N’
% 	& 0.5	% ku- + NP
% 	& 0.5	% mə- + N’
% 	\\ \midrule
% 
% 1.3 Segmental phonology applies
% 	& --	% topic markers + VP
% 	& --	% dynamic marker + V’
% 	& --	% person marker
% 	& --	% da-
% 	& --	% eda-/ada-
% 	& --	% quantifiers
% 	& --	% case marker + N’
% 	& --	% ku- + NP
% 	& --	% mə- + N’
% 	\\ \midrule
% 
% 2 Accent dependent
% 	& 1	% topic markers + VP
% 	& 1	% dynamic marker + V’
% 	& 1	% person marker
% 	& 1	% da-
% 	& 1	% eda-/ada-
% 	& 0.5	% quantifiers
% 	& 1	% case marker + N’
% 	& 1	% ku- + NP
% 	& 1	% mə- + N’
% 	\\ \midrule
% 
% 3.1 Bound morpheme
% 	& 1	% topic markers + VP
% 	& 1	% dynamic marker + V’
% 	& 1	% person marker
% 	& 1	% da-
% 	& 1	% eda-/ada-
% 	& 1	% quantifiers
% 	& 1	% case marker + N’
% 	& 1	% ku- + NP
% 	& 1	% mə- + N’
% 	\\ \midrule
% 
% 3.2 Closes off word
% 	& 0	% topic markers + VP
% 	& 0	% dynamic marker + V’
% 	& 0	% person marker
% 	& 0	% da-
% 	& 0	% eda-/ada-
% 	& 1	% quantifiers
% 	& 0	% case marker + N’
% 	& 0	% ku- + NP
% 	& 0	% mə- + N’
% 	\\ \midrule
% 
% 3.3 Modifies a single constituent
% 	& 1	% topic markers + VP
% 	& 1	% dynamic marker + V’
% 	& 1	% person marker
% 	& 1	% da-
% 	& 1	% eda-/ada-
% 	& 1	% quantifiers
% 	& 1	% case marker + N’
% 	& 1	% ku- + NP
% 	& 1	% mə- + N’
% 	\\ \midrule
% 
% 3.4 Strict ordering
% 	& 1	% topic markers + VP
% 	& 1	% dynamic marker + V’
% 	& 1	% person marker
% 	& 1	% da-
% 	& 1	% eda-/ada-
% 	& 1	% quantifiers
% 	& 1	% case marker + N’
% 	& 1	% ku- + NP
% 	& 1	% mə- + N’
% 	\\ \midrule
% 
% 3.5 Only with a certain constituent
% 	& 1	% topic markers + VP
% 	& 0	% dynamic marker + V’
% 	& 1	% person marker
% 	& 0	% da-
% 	& 0.5	% eda-/ada-
% 	& 0	% quantifiers
% 	& 1	% case marker + N’
% 	& 1	% ku- + NP
% 	& 1	% mə- + N’
% 	\\ \midrule
% 
% 3.6 No internal construction
% 	& 1	% topic markers + VP
% 	& 1	% dynamic marker + V’
% 	& 0	% person marker
% 	& 1	% da-
% 	& 0	% eda-/ada-
% 	& 1	% quantifiers
% 	& 1	% case marker + N’
% 	& 1	% ku- + NP
% 	& 1	% mə- + N’
% 	\\ \midrule
% 
% 4.1 Immunity to deletion
% 	& 1	% topic markers + VP
% 	& 1	% dynamic marker + V’
% 	& 1	% person marker
% 	& 1	% da-
% 	& 1	% eda-/ada-
% 	& 1	% quantifiers
% 	& 1	% case marker + N’
% 	& 1	% ku- + NP
% 	& 1	% mə- + N’
% 	\\ \midrule
% 
% 4.2 Immunity to replacement
% 	& 1	% topic markers + VP
% 	& 1	% dynamic marker + V’
% 	& 1	% person marker
% 	& 1	% da-
% 	& 1	% eda-/ada-
% 	& 1	% quantifiers
% 	& 1	% case marker + N’
% 	& 1	% ku- + NP
% 	& 1	% mə- + N’
% 	\\ \midrule
% 
% 4.3 Unavailable to movement
% 	& 1	% topic markers + VP
% 	& 1	% dynamic marker + V’
% 	& 1	% person marker
% 	& 1	% da-
% 	& 1	% eda-/ada-
% 	& 1	% quantifiers
% 	& 1	% case marker + N’
% 	& 1	% ku- + NP
% 	& 1	% mə- + N’
% 	\\ \midrule
% 
% 5 For $z = x + y$, $y$ not part of syntactic rule on $z$
% 	& 1	% topic markers + VP
% 	& 1	% dynamic marker + V’
% 	& 1	% person marker
% 	& 1	% da-
% 	& 1	% eda-/ada-
% 	& 1	% quantifiers
% 	& 1	% case marker + N’
% 	& 1	% ku- + NP
% 	& 1	% mə- + N’
% 	\\
% 
% \toprule
% \tableheaderfont\multicolumn{9}{c}{Clitic vs. affix \citep{klavans1985}} \\
% \toprule
% 
% Semantic content
% 	& 0	% topic markers + VP
% 	& 0	% dynamic marker + V’
% 	& 1	% person marker
% 	& 1	% da-
% 	& 1	% eda-/ada-
% 	& 1	% quantifiers
% 	& 0	% case marker + N’
% 	& 1	% ku- + NP
% 	& 1	% mə- + N’
% 	\\ \midrule
% 
% No semantic relationship to host
% 	& 1	% topic markers + VP
% 	& 0	% dynamic marker + V’
% 	& 1	% person marker
% 	& 1	% da-
% 	& 1	% eda-/ada-
% 	& 1	% quantifiers
% 	& 0	% case marker + N’
% 	& 1	% ku- + NP
% 	& 1	% mə- + N’
% 	\\ \midrule
% 
% Subcategorization of phrasal node
% 	& 1	% topic markers + VP
% 	& 1	% dynamic marker + V’
% 	& 1	% person marker
% 	& 1	% da-
% 	& 1	% eda-/ada-
% 	& 1	% quantifiers
% 	& 1	% case marker + N’
% 	& 1	% ku- + NP
% 	& 1	% mə- + N’
% 	\\ \midrule
% 
% Not very selective of host
% 	& 1	% topic markers + VP
% 	& 0	% dynamic marker + V’
% 	& 0	% person marker
% 	& 1	% da-
% 	& 1	% eda-/ada-
% 	& 1	% quantifiers
% 	& 0	% case marker + N’
% 	& 1	% ku- + NP
% 	& 0	% mə- + N’
% 	\\
% 
% \bottomrule
% 
% Yes (clitic)
% 	& 13	% topic markers + VP
% 	& 10	% dynamic marker + V’
% 	& 14	% person marker
% 	& 14.5	% da-
% 	& 14	% eda-/ada-
% 	& 14.5	% quantifiers
% 	& 11	% case marker + N’
% 	& 15.5	% ku- + NP
% 	& 14.5	% mə- + N’
% 	\\
% 
% No (word or affix)
% 	& 4	% topic markers + VP
% 	& 7	% dynamic marker + V’
% 	& 3	% person marker
% 	& 2.5	% da-
% 	& 3	% eda-/ada-
% 	& 2.5	% quantifiers
% 	& 6	% case marker + N’
% 	& 1.5	% ku- + NP
% 	& 2.5	% mə- + N’
% 	\\
% 
% Difference quotient
% 	& 0.53	% topic markers + VP
% 	& 0.18	% dynamic marker + V’
% 	& 0.65	% person marker
% 	& 0.71	% da-
% 	& 0.65	% eda-/ada-
% 	& 0.71	% quantifiers
% 	& 0.29	% case marker + N’
% 	& 0.82	% ku- + NP
% 	& 0.71	% mə- + N’
% 	\\
% 
% \bottomrule
% \end{tabu}
% \label{tab:clitichood}
% \end{table}


\section{Marking strategies}
\index{marking strategies}
\index{dependent marking}

With regards to the dichotomy head--dependent marking, Ayeri is very 
thoroughly dependent marking. This is exhibited, for instance, in the expression 
of possessive relationships, where the dependent is marked for genitive 
case\index{cases!genitive}:

\begin{multicols}{2}
\pex
\a\label{ex:gennoun}\begingl
	\glpreamble \begin{forest}
	where n children=0{tier=word,edge=dotted,font=\itshape}{}
	[{dema}, for tree={calign=first}
		[{dema}, name=head]
		[{na Tuvo}, for tree={calign=first}
			[{\textbf{na} Tuvo}, name=dependent]
		]
	]
	\node at (current bounding box.south) [below=0pt of head]
		{\textsc{\tiny head}};
	\node at (current bounding box.south) [below=0pt of dependent] 
		{\textsc{\tiny dependent}};
	\end{forest} //
	\gla dema \textbf{na} Tuvo //
	\glb dema \textbf{na} Tuvo //
	\glc aunt \textbf{\Gen{}} Tuvo //
	\glft `Tuvo's aunt' //
\endgl

\a\label{ex:genprn}\begingl
	\glpreamble \begin{forest}
	where n children=0{tier=word,edge=dotted,font=\itshape}{}
	[{kasu}, for tree={calign=first}
		[{kasu}, name=head]
		[{barieri}, for tree={calign=first}
			[{barieri}]
		]
		[{nā}, for tree={calign=first}
			[{\textbf{nā}}, name=dependent]
		]
	]
	\node at (current bounding box.south) [below=0pt of head]
		{\textsc{\tiny head}};
	\node at (current bounding box.south) [below=0pt of dependent] 
		{\textsc{\tiny dependent}};
	\end{forest} //
	\gla kasu barieri \textbf{nā} //
	\glb kasu bari-eri \textbf{nā} //
	\glc basket meat-\Ins{} \Fsg{}.\textbf{\Gen{}} //
	\glft `my basket of meat' //
\endgl

\xe
\end{multicols}

In (\ref{ex:gennoun}), \rayr{tuvo}{Tuvo} is grammatically in possession of her 
\xayr{dem}{dema}{aunt}; the possessee forms the head of the phrase while it is 
modified by the possessor, which receives the marking. In (\ref{ex:genprn}), 
\xayr{ksu}{kasu}{basket} forms the head and thus also the possessee while 
\xayr{naa}{nā}{my} serves as the possessor and thus as the dependent possessor; 
the genitive case is, again, marked on the dependent. A further example of 
dependent marking is that the locative case is marked on the prepositional 
object while the preposition itself does not receive marking:

\begin{multicols}{2}
\ex\label{ex:loc}
\begingl
	\gla agonan minkay\textbf{ya} //
	\glb agonan minkay\textbf{-ya} //
	\glc outside village\textbf{-\Loc{}} //
	\glft `outside of the village' //
\endgl
\xe

\smaller\begin{forest}
where n children=0{tier=word,edge=dotted,font=\itshape}{}
[{agonan}, for tree={calign=first}
	[{agonan}, name=head]
	[{minkayya}, for tree={calign=first}
		[{minkay\textbf{ya}}, name=dependent]
	]
]
\node at (current bounding box.south) [below=0pt of head]
	{\textsc{\tiny head}};
\node at (current bounding box.south) [below=0pt of dependent] 
	{\textsc{\tiny dependent}};
\end{forest}
\end{multicols}

The relativizer, likewise, may agree in case with the noun phrase in the matrix 
clause which it modifies:

\begin{multicols}{2}
\ex
\begingl
	\gla sangalas kivo s\textbf{as} … //
	\glb sangal-as kivo s\textbf{-as} … //
	\glc room-\Parg{} small \Rel{}\textbf{-\Parg{}} … //
	\glft `the small room which …' //
\endgl
\xe

\smaller\begin{forest}
where n children=0{tier=word,edge=dotted,font=\itshape}{}
[{sangalas}, for tree={calign=first}
	[{sangalas}, name=head]
	[{kivo}, for tree={calign=first}
		[{kivo}]
	]
	[{sas}, for tree={calign=first}
		[{s\textbf{as}}, name=dependent]
		[{...}, for tree={calign=first}
			[{...}]
		]
	]
]
\node at (current bounding box.south) [below=0pt of head]
	{\textsc{\tiny head}};
\node at (current bounding box.south) [below=0pt of dependent] 
	{\textsc{\tiny dependent}};
\end{forest}
\end{multicols}

\index{typology!of morphemes|)}
