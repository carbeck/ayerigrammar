\chapter{Morphological typology}
\label{ch:morphtyp}
\index{typology!of morphemes|(}

The first chapter dealt with the smallest constituent parts of words---speech
sounds, which ones there are, and how they assemble into valid words.
Consequently, the following two chapters will be about the next step up from
this: morphemes, the atoms of meaning. First, we will have a more general look
at which kinds of morphemes there are, and then look at them more closely by
part of speech: what is their distribution, and how are morphemes put together
to form inflected words? This chapter on morphological typology will first deal
with general questions about Ayeri's degree of synthesis, and then will try to
answer questions about the functions various kinds of inflection carry out in
the language. In a prelude to both the morphology and syntax chapters, special
attention is given to discussing why certain affixes and `small words' should
be treated as clitics.

\section{Typology}
\label{sec:typology}

For the largest part, Ayeri is an \emph{agglutinative}\index{agglutination} 
language. \citet{comrie1989} says of agglutinating languages that in these, 
typically,

\blockcquote[43--44]{comrie1989}{a word may consist of more than one morpheme,
but the boundaries between morphemes in the word are always clear-cut;
moreover, a given morpheme has at least a reasonably invariant shape, so that
the identification of morphemes in terms of their phonetic shape is also
straightforward. […] As is suggested by the term agglutinating (cf. Latin
\fw{gluten} `glue'), it is as if the various affixes were just glued on one 
after the other (or one before the other, with prefixes).}

In Ayeri, root morphemes are modified by affixes for the purposes of inflection
and derivation, and these affixes, in the form of suffixes\index{suffixes} more
specifically, can be stacked, especially on verbs. Indeed, they vary little, so
that they are always easily recognizable. Suffixation in Ayeri is especially
prominent with verbs, as (\ref{ex:suffixation}) shows.

% \begin{figure}[h]
\ex\label{ex:suffixation}%
\begingl
	\gla Le @ kondasayāng hemaye pruyya nay napayya kayvay. //
	\glb le= kond-asa=yāng hema-ye-Ø pruy-ya nay napay-ya kayvay //
	\glc \PatTI{}= eat-\Hab{}=\TsgM{}.\Aarg{} egg-\Pl{}-\Top{} salt-\Loc{} 
		and pepper-\Loc{} without //
	\glft `He always eats his eggs without salt and pepper.' //
\endgl\xe
% \end{figure}

The verb root \xayr{koMd/}{kond-}{eat} is inflected here for a habitual action
with the suffix \rayr{/As}{-asa}, and also carries a pronominal
clitic,\index{clitics} \rayr{/yaaNF}{-yāng}, marking a third person singular
masculine agent. With the notable exception of pronouns and related pronominal
clitics, affixes tend to encode a single grammatical function. The examples in
(\ref{ex:inflectable}) illustrate that verbs are not the only part of speech
which can inflect; nouns, adjectives, and the relativizing conjunction can do
so as well.

\begin{figure}[h]
\pex\label{ex:inflectable}
\a\label{ex:letters}\begingl
	\gla Ang @ mətahanay tamanyeley yeyam. //
	\glb ang= mə-tahan=ay.Ø taman-ye-ley yeyam. //
	\glc \AgtT{}= \Pst{}-write=\Fsg{}.\Top{} letter-\Pl{}-\PargI{} 
		\TsgF{}.\Dat{} //
	\glft `I wrote letters to her.' //
\endgl

\a\label{ex:adjinfl}\begingl
	\gla Ang @ koronya Kaman apyanas palay-eng. //
	\glb ang= koron-ya Kaman apyan-as palay-eng //
	\glc \AgtT{}= know-\TsgM{} Kaman joke-\Parg{} funny-\Comp{} //
	\glft `Kaman knows a funnier joke.' //
\endgl

\a\label{ex:relative}\begingl
	\gla Le @ turayāng taman sinā ang @ ningay tamala vās. //
	\glb le= tura=yāng taman-Ø si-Ø-na ang= ning=ay.Ø tamala vās //
	\glc \PatTI{}= send=\Tsg{}.\M{}.\Aarg{} letter-\Top{} 
		\Rel{}-\PatTI{}-\Gen{} \AgtT{}= tell=\Fsg{}.\Top{} yesterday 
		\Ssg{}.\Parg{} //
	\glft `The letter which I told you about yesterday, he sent it.' //
\endgl
\xe
\end{figure}

The principle of not conflating several grammatical functions into a single
suffix can be observed in (\ref{ex:letters}) regarding the word
\xayr{tmnFyelej}{tamanyeley}{letters}, in which the plural marker 
\rayr{/ye}{-ye} is distinct from the inanimate-patient case marker 
\rayr{/lej}{-ley} (the latter, however, conflates animacy and case). Strictly 
speaking, the pronoun \xayr{yeymF}{yeyam}{to her} is also composed, namely of
the third person feminine base form \rayr{ye}{ye} and the dative case marker
\rayr{/ymF}{yam}. Example (\ref{ex:relative}) is one we have already 
encountered before (\autoref{doublerel}, p.~\pageref{doublerel}). Here, the
relative pronoun, \xayr{sinaa}{sinā}{of/about which} is inflected for genitive
case, and stress on the usually unstressed last syllable
suprasegmentally\index{suprasegmental} marks that this form is contracted from
\rayr{sileyen}{sileyena} (\textit{si-ley-ena}, \Rel{}-\PargI{}-\Gen{}).

So far, we have concentrated on suffixes, but there are a number of 
prefixes\index{prefixes} as well; (\ref{ex:letters}) exhibits the past prefix 
\rayr{m/}{mə-} (which is actually redundant in this case). There are also 
demonstrative prefixes on nouns, however. In the following example, the prefix 
\xayr{Ed/}{eda-}{this-} in (\ref{ex:demprefs}) joins the noun
\xayr{pehmF}{peham}{carpet} to indicate a specific carpet.

% \begin{figure}[h]
\ex\label{ex:demprefs}%
\begingl
	\gla Le @ no intoyyang eda-peham. //
	\glb le= no int-oy=yang eda=peham-Ø //
	\glc \PatTI{}= want buy-\Neg{}=\Fsg{}.\Aarg{} this=carpet-\Top{} //
	\glft `I do not want to buy this carpet.' //
\endgl\xe
% \end{figure}

Besides prefixes and suffixes, Ayeri also possesses at least one element in
both the verb cluster and cooccurring with adpositions which straddles the
border between inflection and a function word. This is the case with the clitic
marker \rayr{mN}{manga}, which is treated as an independent word in
orthography, but can modify verbs and adpositions---heads of verb phrases (VPs)
and prepositional phrases (PPs), re\-spec\-tive\-ly. It is unstressed and
appears at the margin of its modification target.

% \begin{figure}[h]
\pex
\a\label{ex:prog}\begingl
	\gla Ang @ manga @ yavaya ayon bariley. //
	\glb ang= manga= yava-ya ayon-Ø bari-ley //
	\glc \AgtT{}= \Prog{}= roast-\TsgM{} man-\Top{} meat-\PargI{} //
	\glft `The man is roasting meat.' //
\endgl

\a\label{ex:dyn}\begingl
	\gla Ya @ mətapyyāng maritay misley manga @ luga bari. //
	\glb ya= mə-tapy=yāng maritay mis-ley manga= luga bari-Ø //
	\glc \LocT{}= \Pst{}-put=\TsgM{}.\Aarg{} before spit-\PargI{} \Dir{}= 
		between meat-\Top{} //
	\glft `The meat, he had put a spit through it before.' //
\endgl
\xe
% \end{figure}

In (\ref{ex:prog}), \rayr{mN}{manga} modifies the verb \xayr{yv/}{yava-}{roast}
and indicates that this is a temporarily ongoing action, like the English
progressive, except not as strongly grammaticalized.\footnote{I suppose, a
better parallel is the so-called \fw{rheinische Verlaufsform} `Ripuarian
progressive' (\fw{sein} `be' + \fw{am/beim} `at the' + infinitive) in German, a
construction common in the colloquial language which parallels the English
progressive construction and is not yet fully grammaticalized
\citep[435]{dudengram2016}. Speakers will thus accept both \fw{Er lernt 
gerade}, literally `He studies right now', and \fw{Er ist am lernen} `He is
studying'.
% 
% \ex[lingstyle=fnex,belowexskip=-1em]\label{ex:ripprog}\begingl
% 	\gla Der Mann ist Fleisch am Braten. //
% 	\glc the man is meat at.the roasting //
% 	\glft `The man is roasting meat.' //
% \endgl\xe
}
%
In (\ref{ex:dyn}), \rayr{mN}{manga} modifies the preposition, on the other
hand, to indicate that it is directional: \rayr{lug}{luga} by itself means
`among, between', while its directional form \rayr{mN lug}{manga luga} means
`through; during, for'.

As we have seen in the examples above, person suffixes on verbs are single 
morphemes that encode more than one property, for example \rayr{/yeNF}{-yeng} 
encodes the person features third person, feminine, singular, and agent. 
Personal pronouns,\index{pronouns!personal} of which the person clitics on 
verbs are an instance, are the main case of fusion among agglutination in 
Ayeri, although some of the forms, like \xayr{yeymF}{yeyam}{to her} above, can 
be decomposed into root and suffix without problem.\footnote{Originally, 
Ayeri's personal pronouns were indeed agglutinative as well, so 
\xayr{yeNF}{yeng}{she} used to be \rayr{Iye\_aNF}{iyeang} (\fw{iy-e-ang}, 
\Tsg{}-\F{}-\Aarg{}). This also gives an explanation to \citet{boga2016}'s 
observation that Ayeri's plural pronouns are formed \textcquote[{[}15{]}; 
`possibly in an even too regular way']{boga2016}{[v]ielleicht sogar zu 
regelmäßig}.}

Perpendicular to the axis isolation–agglutination runs the axis 
analytic–syn\-thetic. On the latter axis, Ayeri scores mostly as 
\emph{synthetic}, since it prefers compactness over spreading a construction 
over several words, though it does not incorporate object noun phrases (NPs) 
and it is not possible to form `sentence-words' either, so it is not going so 
far as to be poly\-syn\-thetic \citep[45--46]{comrie1989}. It is nonetheless 
theoretically possible, due to suffixation being a prominent pattern, to form 
foot-long words like the one in (\ref{ex:footlong}).

% \begin{figure}
\ex\label{ex:footlong}\begingl
	\gla da-mətahasongoyyang-ikan //
	\glb da=mə-taha-asa-ong-oy=yang=ikan //
	\glc such=\Pst{}-have-\Hab{}-\Irr{}-\Neg{}=\Fsg{}.\Aarg{}=much //
	\glft `I would not much used to have had such' //
\endgl\xe
% \end{figure}

One case of analytic morphology is compound prepositions like \xayr{mN
lug}{manga luga}{through} in (\ref{ex:dyn}), but verbs as well show analytic
structures not only with the progressive marker, but also with modals, as
(\ref{ex:modanalytic}) shows.

% \begin{figure}[h]
\ex\label{ex:modanalytic}%
\begingl
	\gla Ming @ sahoyyang dabas. //
	\glb ming= saha-oy=yang dabas //
	\glc can= come-\Neg{}=\Fsg{}.\Aarg{} today //
	\glft `I can't come today.' //
\endgl\xe
% \end{figure}

Most of the information the inflectional phrase (IP) contains in this example
is marked on the content verb, \xayr{sh/}{saha-}{come}, except for ability,
which is expressed by the particle \xayr{miNF}{ming}{can}. \rayr{miNF}{ming} is
an uninflected form of the verb expressing ability and we might count it as an
auxiliary verb in that the full semantic content of the IP is spread out over
two verb forms, one major, one minor---this probably should not be understood
as a serial verb construction, however
\citep{aikhenvald2006}.\footnote{\rayr{mN}{manga} has, in fact, a verbal
counterpart \xayr{mN/}{manga-}{move; remove} as well, which presumably served
as the origin of both the progressive and the directional marker.
\label{fn:mangaverb}} As we will see later (\autoref{subsec:clitics}), though,
these modal particles behave more like clitics than function words. Consider,
on the other hand, example (\ref{ex:mingfull}), in which \rayr{miNF}{ming} is
inflected like a regular verb.

% \begin{figure}[h]
\ex\label{ex:mingfull}
\begingl
	\gla Da-mingya ang @ Diyan. //
	\glb da=ming-ya ang= Diyan. //
	\glc so=can-\TsgM{} \Aarg{}= Diyan //
	\glft `Diyan can (do it).' //
\endgl\xe
% \end{figure}

\section{Morphological processes}

\subsection{Prefixation}
\index{prefixes|(}

Prefixes in Ayeri apply mainly to verbs, but nouns, pronouns, adjectives and
conjunctions as well can appear with them. Some of these are likely clitics;
reasons for their being clitics will be discussed below in
\autoref{subsec:clitics}. With verbs, prefixes that are most certainly `true'
prefixes---that is, bound morphemes which have been semantically bleached by
grammaticalization to the point where they only express grammatical functions
\citep[157\psqq]{lehmann2015}, and which subcategorize for words rather than
phrases \citep[117]{klavans1985}, with a rather high obligation to be marked on
every conjunct in coordination \citep[139]{spencerluis2012}---are the tense
prefixes marking both three degrees of past and future tense, for example
\rayr{se/}{sə-} in (\ref{ex:tenseprefs}).

% \begin{figure}[h]
\ex\label{ex:tenseprefs}
\begingl
	\gla Ang @ səsarāyn ya @ Makapetang. //
	\glb ang= sə-sara=ayn.Ø ya= Makapetang //
	\glc \AgtT{}= \Fut{}-go=\Fpl{}.\Top{} \Loc{}= Makapetang //
	\glft `We will go to Makapetang.' //
\endgl\xe
% \end{figure}

Here, the prefix \rayr{se/}{sə-} marks future tense on the verb, 
\xayr{sr/}{sara-}{go}. The other tense\index{tense} prefixes are 
\rayr{k/}{kə-}  (\NPst{}), \rayr{m/}{mə-} (\Pst{}), \rayr{v/}{və-} (\RPst{}),
as well as \rayr{p/}{pa-} (\NFut{}) and \rayr{ni/}{ni-} (\RFut{}). Besides this
set of prefixes, there are also a number of proclitics that can appear with
verbs, though not exclusively. These are the anaphora \xayr{d/}{da-}{thus, so,
such} and the reflexive\index{pronouns!reflexive} marker
\xayr{sitNF/}{sitang-}{self}, compare (\ref{ex:davb}) and (\ref{ex:sitangvb}).
Furthermore, (\ref{ex:sitangpronvb}) shows that \rayr{sitNF/}{sitang-} can also
be used as a preverbal particle in situations where the agent is also the
instrument, so both of the following two sentences are equivalent in meaning.

\begin{figure}[h]
\begin{minipage}[t]{.5\linewidth}
\ex\label{ex:davb}
\begingl
	\gla Da-mingya ang @ Diyan. //
	\glb da=ming-ya ang= Diyan. //
	\glc so=can-\TsgM{} \Aarg{}= Diyan //
	\glft `Diyan can (\emph{do it}).' //
\endgl
\xe
\end{minipage}
~
\begin{minipage}[t]{.5\linewidth} 
\ex\label{ex:sitangvb}
\begingl
	\gla Sitang-kecāng. //
	\glb sitang=ket=yāng //
	\glc self=wash=\TsgM{}.\Aarg{} //
	\glft `He washes \emph{himself}.' //
\endgl\xe
\end{minipage}
\end{figure}

\begin{figure}[h]
\pex\label{ex:sitangpronvb}
\a\label{ex:sitang+pronoun}\begingl
	\gla Sa @ apicāng nanga ikan sitang-yari. //
	\glb sa= apit=yāng nanga ikan sitang=yari //
	\glc \PatT{}= clean=\Tsg{}.\Aarg{} house complete 
		self=\TsgM{}.\Ins{} //
	\glft `He cleaned the whole house by himself.' //
\endgl

\a\begingl
	\gla Sa @ sitang-apicāng nanga ikan. //
	\glb sa= sitang=apit=yāng nanga ikan //
	\glc \PatT{}= self-clean=\Tsg{}.\Aarg{} house complete //
	\glft (\fw{idem}) //
\endgl
\xe
\end{figure}

\phantomsection\label{nounprefixes}
Example (\ref{ex:sitang+pronoun}) shows the more common application of 
\rayr{sitNF/}{sitang-}, that is, as a reflexive modifier of pronouns. Moreover,
the prefix \rayr{d/}{da-} can as well be used with noun phrases and is part of
the demonstrative set of prefixes (which behave, in fact, like proclitics),
\xayr{d/}{da-}{such}, \xayr{Ed/}{eda-}{this}, and \xayr{Ad/}{ada-}{that} as shown in (\ref{ex:dempfxs}).

\begin{figure}[h]
\ex\label{ex:dempfxs}\begingl
	\gla eda- / ada- / da-ganang //
	\glb eda= / ada= / da=gan-ang //
	\glc this= / that= / such.a=child-\Aarg{} //
	\glft `this/that/such a child' //
\endgl\xe
\end{figure}

The demonstrative prefixes are also used to form the demonstrative 
pronouns\index{pronouns!demonstrative} \xayr{EdnY}{edanya}{this one}, 
\xayr{AdnY}{adanya}{that one} and \xayr{dnY}{danya}{(such) one}. A special case
in this regard is the postposition \xayr{d/naarY}{da-nārya}{in spite of,
despite} where \rayr{d/}{da-} combines with the conjunction
\xayr{naarY}{nārya}{but, although, except}. Originally, 
\xayr{dikpis}{dikapisa}{respective} is derived from \rayr{d/}{da-} + 
\xayr{Ikpis}{ikapisa}{bound, dependent}, which is an example of a combination 
with an adjective. There is also a fixed adverbial expression using one of
these prefixes, \xayr{Ed/tdjymF}{eda-tadayyam}{for the time being, for now}
(this=time-\Dat{}). Last but not least, the prefix \xayr{ku/}{ku-}{like, as
though} (also a proclitic) can be used with both adjectives and nouns, as well
as complement clauses, as shown in (\ref{ex:kucombin}).

\begin{figure}[h]
\pex\label{ex:kucombin}
\a\begingl
	\gla ku-koyaya //
	\glb ku=koya-ya //
	\glc like=book-\Loc{} //
	\glft `like in a book' //
\endgl

\a\begingl
	\gla ku-prasi //
	\glb ku=prasi //
	\glc like=sour //
	\glft `as though (it were) sour' //
\endgl

\a\begingl
	\gla ku-adareng turavangas //
	\glb ku=ada-reng turavang-as //
	\glc like=that-\AargI{} problem-\Parg{} //
	\glft `as though that were a problem' //
\endgl
\xe
\end{figure}

An example of a set-phrase adverbial consisting of \rayr{ku/}{ku-} and a verb 
is \xayr{ku/nsY}{ku-nasya}{as follows}, \rayr{nsY/}{nasy-} meaning `follow'. 
What is curious here is that this fossilized form is lacking person marking 
and is just extended with an epenthetic \textit{-a} since \textit{-sy} is not 
a permissible coda. The expected form would be 
*\rayr{ku/nsYreNF}{*ku-nasyareng} (like-follow=\TsgI{}.\Aarg{}).

Following \citet{klavans1985}, who suggests that clitics best be defined as
\textcquote[117]{klavans1985}{affixation at the phrasal level}, a very common
kind of prefix to the inflectional \emph{phrase} are the topic markers. They
are counted as parts of the IP but do not interact with it regarding stress
assignment (they are always unstressed) while always being in an initial
position, preceding any other preverbal elements:

\pex
	\a\begingl
		\gla Ang @ tahanya tamanley. //
		\glb ang= tahan-ya taman-ley //
		\glc \AgtT{}= write-\TsgM{} letter-\PargI{} //
		\glft `He writes a letter.' //
	\endgl
	\a \textit{Ang mətahanya tamanley.} `He wrote a letter.'
	\a \textit{Ang manga mətahanya tamanley.} `He was writing a letter.'
	\a \textit{Ang manga no mətahanya tamanley.} `He was wanting to write a 
		letter.'
\xe

The word \xayr{kudpluNF}{kudapalung}{other than that, apart from that} is an
interesting case in that it is a fossilized form of multiple proclitics being
stacked on an adjective. \rayr{kudpluNF}{kudapalung} is transparently made up
of the root \xayr{pluNF}{palung}{other, different} to which are added
\xayr{d/}{da-}{so, such} and \xayr{ku/}{ku-}{like, as though}.

\index{prefixes|)}

\subsection{Suffixation}
\index{suffixes|(}

As a largely agglutinative language, most grammatical marking in Ayeri happens
to be done by means of suffixes. These occur mainly with nouns and verbs,
however, some basic quantifiers take the shape of suffixes as well, but behave
more like enclitics. Quantifiers may modify content words almost regardless of
their part of speech---noun, verb, adjective or adverb. The most pervasive
examples of suffixation are certainly those of case marking on nouns and of
person marking on verbs, as exemplified in (\ref{ex:conjdecl}).

% \begin{figure}[h]
\ex\label{ex:conjdecl}\begingl
	\gla Sa @ pəharuyang va manga @ miday tangya vana suyareri, vimyon! //
	\glb sa= pə-haru=yang va.Ø manga= miday tang-ya vana suyar-eri vimyon //
	\glc \PatT{}= \NFut{}-beat=\Fsg{}.\Aarg{} \Ssg{}.\Top{} \Dir{}= around 
		ears-\Loc{} \Ssg{}.\Gen{} ladle-\Ins{} monkey //
	\glft `I'll beat you around your ears with a ladle, you monkey!' //
\endgl\xe
% \end{figure}

This example shows marking of \xayr{tNF}{tang}{ears} with the locative case
suffix \rayr{/y}{-ya} and the marking of \xayr{suyrF}{suyar}{ladle} with the
instrumental case suffix \rayr{/Eri}{-eri}; the previous examples already
provide instances of the exceedingly common markers for agent and patient case,
\rayr{/ANF}{-ang} and \rayr{/AsF}{-as}, respectively. Besides case, nouns can
also be marked for plural with the suffix \rayr{/ye}{-ye}, and verb roots may
be extended by the mood and aspect markers \rayr{/ONF}{-ong} (\Irr{}),
\rayr{/As}{-asa} (\Hab{}) and \rayr{/Oj}{-oy} (\Neg{}), the last of which is 
the most frequently occurring one. The mood suffixes can also be stacked,
leading to the long word in (\ref{ex:footlong}) above. Person marking on verbs
is realized as agreement suffixes or of clitic personal pronouns depending on
whether an agent NP proper is present or not for the verb to agree with. In
(\ref{ex:conjdecl}), a cliticized agent pronoun \xayr{/yaNF}{-yang}{I}
(\Fsg{}.\Aarg{}) appears.

As mentioned above, quantifiers appear as enclitics on almost any type of
content word, like on the adverb \xayr{pr}{para}{fast} in (\ref{ex:advclts}),
for instance.

% \begin{figure}[h]
\ex\label{ex:advclts}
%\pex
% \a\begingl
% 	%\glpreamble With a verb: //
% 	\gla No sarayang-ikan //
% 	\glb no sara=yang=ikan //
% 	\glc want go=\Fsg{}.\Aarg{}=much //
% 	\glft `I really want to go.' //
% \endgl
%\a
\begingl
	%\glpreamble With an adverb: //
	\gla Tigalyeng para-ma. //
	\glb tigal=yeng para=ma //
	\glc swim=\TsgF{}.\Aarg{} fast=enough //
	\glft `She swims fast enough.' //
\endgl

% \a\begingl
% 	%\glpreamble With a predicative adjective: //
% 	\gla Yang valuy-eng, sahavāng. //
% 	\glb yang {valuy=eng} saha=vāng. //
% 	\glc \Fsg{}.\Aarg{} {glad=rather} come=\Ssg{}.\Aarg{} //
% 	\glft `I am rather glad that you come.' //
% \endgl
% 
% \a\begingl
% 	%\glpreamble With an attributive adjective: //
% 	\gla Adareng bahisley mino-ing //
% 	\glb ada-reng bahis-ley mino=ing //
% 	\glc that=\TsgI{}.\Aarg{} day-\PargI{} happy=so //
% 	\glft `It was such a happy day.' //
% \endgl
% 
% \a\begingl
% 	%\glpreamble With a noun: //
% 	\gla Ang konjan prikanley-ani //
% 	\glb ang kond=yan.Ø prikan-ley=ani //
% 	\glc \Aarg{} eat=\TsgM{}.\Top{} soup=not.at.all //
% 	\glft `They did not eat any soup at all.' //
% \endgl
\xe
% \end{figure}

\index{suffixes|)}

\subsection{Reduplication}
\label{subsec:reduplication}
\index{reduplication|(}

There are two patterns of reduplication for verbs, one with complete
reduplication of the imperative form to create a hortative statement
(\ref{ex:hort}), and one with partial reduplication as a way to express that an
action takes place again, that is, partial reduplication expresses an
iterative, compare (\ref{ex:iter}). The imperative iterative, then, has a
hortative function as well in (\ref{ex:hort+iter}).

\begin{figure}[h]
\pex
\a\label{ex:hort}\begingl%
	\gla naru-naru //
	\glb naru\til{}nara-u //
	\glc \Hort{}\til{}speak-\Imp{} //
	\glft `let's speak' //
\endgl

\a\label{ex:iter}\begingl
	\gla na-narayeng //
	\glb na\til{}nara=yeng //
	\glc \Iter{}\til{}speak=\TsgF{}.\Aarg{} //
	\glft `she speaks again' //
\endgl

\a\label{ex:hort+iter}\begingl
	\gla na-naru //
	\glb na\til{}nara-u //
	\glc \Iter{}\til{}speak-\Imp{} //
	\glft `let's speak again' //
\endgl
\xe
\end{figure}

With nouns, full reduplication is used to create a diminutive\index{diminutive}
form in (\ref{ex:regdim}), though some reduplications are also lexicalized and
may use roots from other parts of speech as well to form nouns, for instance,
the words in (\ref{ex:otherredupnn}--\hyperref[ex:otherredupvb]{d}). There are
also a number of adjectives for which there exists a lexical reduplication with
an intensifying meaning; (\ref{ex:adjredup}) lists a few examples. This,
however, is not a productive derivation strategy.

\begin{figure}[h]
\pex
	\a \makebox[10.5em][l]{\xayr{\larger venej}{veney}{dog}}
		→ \xayr{\larger venej/venej}{veney-veney}{little dog, 
			doggie}\label{ex:regdim}
	\a \makebox[10.5em][l]{\xayr{\larger gnF}{gan}{child}}
		→ \xayr{\larger gnF/gnF}{gan-gan}{grandchild}%
			\label{ex:otherredupnn}
	\a \makebox[10.5em][l]{\xayr{\larger kusNF}{kusang}{double (adj.)}}
		→ \xayr{\larger kusNF/kusNF}{kusang-kusang}{model} 
			\label{ex:otherredupadj}
	\a \makebox[10.5em][l]{\xayr{\larger vehF/}{veh-}{build}} → 
		\xayr{\larger veh/veh}{veha-veha}{tinkering}%
			\label{ex:otherredupvb}
\xe
\end{figure}

\begin{figure}[h]
\pex\label{ex:adjredup}
	\a \makebox[10.5em][l]{\xayr{\larger ApnF}{apan}{wide}}
		→ \xayr{\larger ApnF/ApnF}{apan-apan}{extensive}
	%\a \makebox[10.5em][l]{\xayr{\larger IknF}{ikan}{complete}}
	%	→ \xayr{\larger IknF/IknF}{ikan-ikan}{entire, total}
	\a \makebox[10.5em][l]{\xayr{\larger kebj}{kebay}{alone}}
		→ \xayr{\larger kebj/kebj}{kebay-kebay}{all alone}
	%\a \makebox[10.5em][l]{\xayr{\larger pksF}{pakas}{special}}
	%	→ \xayr{\larger pksF/pksF}{pakas-pakas}{gay}
	\a \makebox[10.5em][l]{\xayr{\larger pisu}{pisu}{tired}}
		→ \xayr{\larger pisu/pisu}{pisu-pisu}{exhausting}
\xe
\end{figure}

\index{reduplication|)}

\subsection{Suprasegmental modification}
\index{suprasegmental|(}

\index{morphophonology!of relative pronouns}
As written above (\autoref{doublerel}), case agreement on a complex-marked
relative pronoun\index{pronouns!relative} can drop out under certain
circumstances and is replaced by compensatory stress on the secondary case
marker, which lengthens the syllable's nucleus vowel, compare
(\ref{ex:relpromorphphon}).

\begin{figure}[h]
\ex\label{ex:relpromorphphon}
\begingl
	\gla … tamanley sinā \textup{(}*sina\textup{)} ang @ ningay tamala
		vās //
	\glb … taman$_i$-ley si-Ø$_i$-na (*si-na$_i$) ang= ning=ay.Ø tamala vās //
	\glc … letter-\PargI{} \Rel{}-\PatTI{}-\Gen{} (*\Rel{}-\Gen{}) \AgtT{}= 
		tell=\Fsg{}.\Top{} yesterday \Ssg{}.\Parg{} //
	\glft `… the letter which (*whose) I told you about yesterday' //
\endgl\xe
\end{figure}

This can be reinterpreted so that vowel length/stress itself is what signifies
the agreement of the relativizer with the preceding NP. Which grammatical role
the relativizer's head represents an agreement controller is essentially
underspecified, hence I will gloss it as -\Agr{} in the following example
instead of as full -\PargI{}. This is illustrated in (\ref{ex:relprostress}).
Since \rayr{n}{na} as a light syllable cannot be stressed in word-final
position under normal circumstances, it has to lengthen to \rayr{naa}{nā}.

% \begin{figure}[h]
\ex[everygla=\upshape]\label{ex:relprostress}%
\begingl
	\gla /ˌsi.leɪ.ˈena/ → /si.ˈna(ː)/ //
	\glb /si-leɪ-ena/ → /si-ˈ-na-ː/ //
	\glc \Rel{}-\PargI{}-\Gen{} {} \Rel{}-\Agr{}-\Gen{}-\Agr{} //
\endgl\xe
% \end{figure}

\index{suprasegmental|)}

\subsection{Clitics}
\label{subsec:clitics}
\index{clitics|(}

I have been using the term `clitic' above and claimed that the one or the other
morpheme in Ayeri is a clitic. Clitics, however, cannot be easily defined in a
formal way, as it appears \citep[126]{spencerluis2012}. Based on
\citet{spencerluis2012}, with recourse to \citet{zwickypullum1983}, some
important, typical characteristics are:

\begin{itemize}
	\item Clitics behave in part like function words and in part like affixes, 
		but in any case they are not free morphemes 
		\citep[38, 42]{spencerluis2012}.
	\item Clitics tend to be phonologically weak items 
		\citep[39]{spencerluis2012}.
	\item Clitics prominently---and importantly---tend to attach 
		`promiscuously' to surrounding words. That is, unlike inflection, they 
		are not limited to connect to a certain part of speech or to align 
		with their host in semantics \citep[40, 108–109]{spencerluis2012}.
	% \item Clitics tend to appear in a second position, whether that is after
	% 	a word, or an intonational or syntactic phrase
	% 	\citep[41]{spencerluis2012}.
	\item Clitics tend to be templatic and to cluster, especially if they 
		encode inflection-like information \citep[41, 47--48]{spencerluis2012}.
	\item Clitics have none of the freedom of ordering found in independent 
		words and phrases \citep[43]{spencerluis2012}.
	\item Positions of `special' clitics tend to not be available to free 
		words \citep[44]{spencerluis2012}.
	% \item Clitics tend to be functional morphemes, and to realize a single 
	% 	morphosyntactic property \citep[67, 179]{spencerluis2012}.
	\item There are no paradigmatic gaps \citep[108--109]{spencerluis2012}.
	\item There tends to be no morphophonemic alteration like vowel harmony, 
		stress shift or sandhi between a clitic and its host 
		\citep[108--109]{spencerluis2012}.
	% \item There tends to be no idiosyncratic change in meaning when a clitic 
	% 	and a clitic host come together, unlike there may be with inflection 
	% 	\citep[108, 110]{spencerluis2012}.
	\item Similar to affixes, clitics and their host tend to be treated as a 
		syntactic unit, that is, lexical integrity prevents that word material
		can be put in between a clitic and its host 
		\citep[108, 110]{spencerluis2012}.
	% \item Clitics usually get joined to a host word after inflection 
	% 	\citep[108, 110]{spencerluis2012}.
	% \item Affixes tend to go on every word in a conjunct (narrow scope),
	% 	while clitics have a tendency to treat a conjunct as a unit to attach
	% 	to (wide scope; \cite[139, 196\psqq]{spencerluis2012}).
\end{itemize}

However, \citet{spencerluis2012} point out many counterexamples to the points
on this list in order to highlight that the border between clitics and affixes
is often fuzzy. Given this fuzziness, it comes as no surprise that according to
their assessment, there is a lot of miscategorization in individual grammars as
a result \citep[107]{spencerluis2012}. Another consequence of this lack of a
clear delineation between clitics and affixes is that, since not all of the
traits described above are always present, making a checklist and summing up
the tally is only of limited value. The traits listed above are thus sufficient
conditions only, not necessary ones. In the following, I want to elaborate on
the classification of various prefixes, suffixes, and particles as
clitics.\footnote{The following discussion incorporates most of the content of
a blog article I previously wrote on this topic, \citet{benung:clitics}, with
some additions and corrections. Since clitics sit at the junction of morphology
and syntax, it will be necessary at times to deal with topics roughly which
will be elaborated on in \autoref{ch:phrasestruct} in more detail.}

\subsubsection{Preposed particles and prefixes}

\label{clitics_preverb} What should be rather unproblematic with regards to
their classification as clitics in Ayeri are the preverbal particles, that is,
the topic \index{topic} marker, one or several modal particles\index{modal
verbs}, the progressive marker\index{aspect!progressive}, and also the emphatic
affirmative and negative discourse particles\index{discourse particles}. All of
these particles essentially have functional rather than lexical content, and
are usually unstressed. They come in a cluster with a fixed order, and they
appear in a position where no ordinary word material could go, since Ayeri is
strictly verb-initial.\footnote{The translation of
`\citetitle{shelley:ozymandias}' in \autoref{sec:ozymandias} deviates from this
rule in \ayr{nmaaNF }\ayr{smF }\ayr{kaarYo }\ayr{nj }\ayr{trYnFkj }
\ayr{beNYonF }\ayr{AdaahlY} \fw{namāng sam kāryo nay taryankay bengyon
adāhalya.} `Two big and torsoless legs stand in that desert' by having the
subject NP \xayr{nmaaNF smF —}{namāng sam ...}{two legs ...} precede the verb
\xayr{beNYonF}{bengyon}{(they) stand}. This is non-standard syntax in a
poetic text.} In conjuncts it is also unnecessary to mark every verb with one
or several preverbal particles, as we see in (\ref{ex:clitics_1}).

\begin{figure}[h]
\pex\label{ex:clitics_1}
\a\label{ex:clitics_1a}\begingl
	\gla Ang @ kece nay dayungisaye māva yanjas yena. //
	\glb ang= ket-ye nay dayungisa-ye māva-Ø yan-ye-as yena //
	\glc \AgtT{}= wash-\TsgF{} and dress-\TsgF{} mother-\Top{}
		boy-\Pl{}-\Parg{} \TsgF{}.\Gen{} //
	\glft `The mother washes and dresses her boys.' //
\endgl

\a\label{ex:clitics_1b}\begingl
	\gla Manga @ sahaya rangya nay nedraya ang @ Tikim. //
	\glb manga= saha-ya rang-ya nay nedra-ya ang= Tikim //
	\glc \Prog{}= come-\TsgM{} home-\Loc{} and sit-\TsgM{} \Aarg{}= Tikim //
	\glft `Tikim is coming home and sitting down.' //
\endgl

\a\label{ex:clitics_1c}\begingl
	\gla Ang @ mya @ ming @ sidegongya nay la-lataya ajamyeley. //
	\glb ang= mya= ming= sideg-ong=ya.Ø nay la\til{}lata=ya.Ø ajam-ye-ley //
	\glc \AgtT{}= shall= can= repair-\Irr{}=\TsgM{}.\Top{} and
		\Iter{}\til{}sell=\TsgM{}.\Top{} toy-\Pl{}-\PargI{} //
	\glft `He should be able to repair and resell the toys.' //
\endgl
\xe
\end{figure}

\label{clitics_preverb_top}
In (\ref{ex:clitics_1a}), therefore, the agent-topic \index{topic} marker
\rayr{ANF}{ang} only occurs before \xayr{ketYe}{kece}{(she) washes}, and the
conjoined verb \xayr{dyuNisye}{dayungisaye}{(she) dresses} is also within its
scope. Repeating the marker as well before the latter verb could either be
considered ungrammatical because there is only one topic
there---\xayr{maav}{māva} {mother}---or the sentence could be interpreted as
having two conjoined clauses with different subjects: `[She washes] and [mother
dresses] her boys.' The latter outcome has \rayr{maav}{māva} as the topic only
of \rayr{dyuNisye}{dayungisaye}, while \rayr{ketYe}{kece}'s topic is the person
marking on the verb---a pro-drop subject, essentially.\footnote{This claim is
further investigated below, p.~\pageref{subsubsec:suffixes}\,ff.; also compare
\autoref{subsec:persnumagr}.}

\label{clitics_preverb_prog}
In (\ref{ex:clitics_1b}), then, the progressive marker \rayr{mN}{manga} 
\index{aspect!progressive} equally has scope over both verb conjuncts,
\xayr{shy}{sahaya}{(he) comes} and \xayr{nedFry}{nedraya}{(he) sits} in what is
presumably a case of extended/distributed exponence. This is to say that
functionally contiguous information can sometimes be split over several words,
so that the functional annotation of each verb in (\ref{ex:clitics_1b}) can be
represented in the fashion of the (incomplete) f-structure matrix
\parencites[see][]{bresnan2016}{buttking2015} shown in (\ref{ex:clitics_2}),
which is an attempt to represent the phrase \xayr{ANF mN sh/}{ang manga
saha-}{is coming} formally. \rayr{mN}{manga} is treated there as being part of
things the verb inflects for, that is, progressive aspect, in spite of
appearing superficially as a function word. The topic marker \rayr{ANF}{ang}
does not reflect a morphological property of the verb in the way the
progressive marker does, but announces the case and---for agents and
patients---the animacy value of the topicalized noun phrase (NP), so the
f-structure in (\ref{ex:clitics_2}) lists this information under the \Top{}
function.\footnote{In the chart, angular brackets group grammatical functions.
Since the verb is the head of the clause, the first \Pred{} (predicator) lists
the verb with its argument structure (a-structure). In the case of
(\ref{ex:clitics_2}), \Subj{} and \Oblq{loc} indicate that `come', governs two
arguments: a subject and an oblique argument in the form of a locative
adverbial. These have been omitted for brevity.}

\ex\label{ex:clitics_2}
\begin{avm}
\[
	\Pred{}	&	\astruct{come}{\ups{\Subj{}}, \ups{\Oblq{loc}}} \\

	\Asp{}	&	\Prog{} \\

	\Top{}	&	\[
					\Case{}	&	\Aarg{} \\
					\Anim{}	&	$+$ \\
					...		&	... \\
				\] \\

	...		&	... \\
\]
\end{avm}
\xe

\label{clitics_preverb_modal}
Modal particles\index{modal verbs}, exemplified in (\ref{ex:clitics_1c}), are
probably slightly less typical as clitics since it seems feasible for them to
be stressed for contrast. What is not possible, however, is to front either
\xayr{mY}{mya}{be supposed to} or \xayr{miNF}{ming}{can}, and the verb itself
also cannot precede the particles, which is demonstrated in
(\ref{ex:clitics_3}). It is also not possible to coordinate any of the elements
in the preverbal particle cluster with \xayr{nj}{nay}{and}, as shown in
(\ref{ex:clitics_4}).

\begin{figure}[h]
\begin{minipage}[t]{.5\linewidth}%
\pex\label{ex:clitics_3}%
\a \ljudge{*} \textit{mya ang ming sidegongya}
\a \ljudge{*} \textit{ming ang mya sidegongya}
\a \ljudge{*} \textit{sidegongya ang mya ming}
\xe
\end{minipage}
~
\begin{minipage}[t]{.5\linewidth}%
\pex\label{ex:clitics_4}%
\a \ljudge{*} \textit{ang \textbf{nay} mya ming sidegongya}
\a \ljudge{*} \textit{ang mya \textbf{nay} ming sidegongya}
\a \ljudge{*} \textit{ang mya ming \textbf{nay} sidegongya}
\xe
\end{minipage}
\end{figure}

It needs to be pointed out that unlike verbs, modal particles in Ayeri resist
inflection, so in (\ref{ex:clitics_1c}) the irrealis suffix \rayr{/ONF} {-ong}
is realized on the verb \xayr{sidegoNY}{sidegongya}{(he) would repair} instead
of on one or both of the modal particles as *\rayr{miNONF}{*mingong} and
*\rayr{mYONF}{myong}, respectively. The combination of \xayr{mY}{mya}{be
supposed to} with an irrealis-marked verb together indicates that the speaker
thinks the action denoted by the verb \emph{should} be carried out. The marking
on the verb may then  be interpreted as being functionally shared by the
constituent parts of the whole verb complex. The same goes for negation: only
the verb can be negated, but not the modal particle. Possibly, it would be
useful in this case to abstract the modal particles as a feature
\Mod{}\textsc{ality} as listed by \citet[Feature Table]{pargram} for purposes
of functional representation. At least superficially, it looks as though Ayeri
acts differently from English here in that the content verb is not a complement
of the modal element. This assumption is supported by the fact that in Ayeri,
the verb inflects, but not the modal particle. Furthermore, modal particles
cannot be modified by adverbs in the way regular verbs can, see
(\ref{ex:clitics_5}).

\begin{figure}[h]
\pex\label{ex:clitics_5}
\a\label{ex:clitics_5a}\begingl
	\gla Ming @ tigalye ban nilay ang @ Diya. //
	\glb ming= tigal-ye ban nilay ang= Diya //
	\glc can= swim-\TsgF{} good probably \Aarg{}= Diya //
	\glft `Diya can probably swim well.' //
\endgl

\a\label{ex:clitics_5b}\ljudge{*}\begingl
	\gla Ming @ nilay tigalye ban ang @ Diya. //
	\glb ming= nilay tigal-ye ban ang= Diya //
	\glc can= probably swim-\TsgF{} well \Aarg{}= Diya //
\endgl
\xe
\end{figure}

Combinations of topic particle and modal particle, as well as modal particle
and verb, can likewise not be interrupted by parenthetical material like
\xayr{nrtNF}{naratang}{they say}, which we can see in the pattern emerging in
(\ref{ex:clitics_6}).

\begin{figure}[h]
\pex\label{ex:clitics_6}
\a\label{ex:clitics_6a}\begingl
	\gla \textbf{Naratang,} ang ming tigalye ban {} Diya kodanya. //
	\glb nara=tang ang ming tigal-ye ban Ø Diya kodan-ya //
	\glc say=\TplM{}.\Aarg{} \AgtT{} can swim-\TsgF{} well \Top{} Diya 
		lake-\Loc{}	//
	\glft `They say Diya can swim well in a lake.' //
\endgl

\a \ljudge{*} \textit{Ang, \textbf{naratang,} ming tigalye ban Diya kodanya.}
\a \ljudge{*} \textit{Ang ming, \textbf{naratang,} tigalye ban Diya kodanya.}
\a \ljudge{\ques} \textit{Ang ming tigalye, \textbf{naratang,} ban Diya
	kodanya.}
\a \textit{Ang ming tigalye ban, \textbf{naratang,} Diya kodanya.}
\a \textit{Ang ming tigalye ban Diya, \textbf{naratang,} kodanya.}
\a \textit{Ang ming tigalye ban Diya kodanya, \textbf{naratang.}}
\xe
\end{figure}

\label{clitics_prenoun_case}
Besides verbs, nouns as well have preposed modifiers\index{cases}. This is the
case with proper nouns specifically, where the name is preceded by a case
particle instead of receiving a case-marking suffix like generic nouns do. This
case marker is phonologically weak in that its phonological make-up is similar
to that of affixes, and unstressed, with the exception of the causative case
marker \rayr{saa}{sā}, which bears at least secondary stress since it contains
a long vowel. We already saw case particles preceding names in
(\ref{ex:clitics_1b}) and (\ref{ex:clitics_5}) above: \rayr{ANF tikimF}{ang
Tikim} and \rayr{ANF diy}{ang Diya};  \rayr{ANF}{ang} marks the proper-noun NPs
as agents in both cases. The case marker is missing when the NP is topicalized,
as indicated in (\ref{ex:clitics_6}), where the agent NP appears as just
\rayr{diy}{Diya}, not \rayr{ANF diy}{ang Diya}. While case suffixes have narrow
scope as shown in (\ref{ex:clitics_7a}) and thus need to be repeated on every
NP in a conjunct, preposed case markers as that in (\ref{ex:clitics_7c}) may be
used with wide scope if both conjuncts are proper nouns. Narrow scope with
proper nouns may add an individuating connotation, exemplified by
(\ref{ex:clitics_7d}).

\begin{figure}[h]
\pex\label{ex:clitics_7}
\a\label{ex:clitics_7a}\begingl
	\gla Toryon veneyang nay badanang. //
	\glb tor-yon veney-ang nay badan-ang //
	\glc sleep-\TplN{} dog-\Aarg{} and father-\Aarg{} //
	\glft `The dog and father are (both) sleeping.' //
\endgl

\a\label{ex:clitics_7b}\ljudge{*}\begingl
	\gla Toryon veney nay badanang. //
	\glb tor-yon veney\_ nay badan-ang //
	\glc sleep-\TplN{} dog\_ and father-\Aarg{} //
\endgl

\a\label{ex:clitics_7c}\begingl
	\gla Sa @ sobisayan ang @ Niva nay {} Mico narānye. //
	\glb sa= sobisa-yan ang= Niva nay \_ Mico narān-ye-Ø //
	\glc \PatT{}= study-\TplM{} \Aarg{} Niva and \_ Mico 
		language-\Pl{}-\Top{} //
	\glft `Languages is what Niva and Mico study.' //
\endgl

\a\label{ex:clitics_7d}\begingl
	\gla Sa @ sobisayan ang @ Niva nay ang @ Mico narānye. //
	\glb sa= sobisa-yan ang= Niva nay ang= Mico narān-ye-Ø //
	\glc \PatT{}= study-\TplM{} \Aarg{}= Niva and \Aarg{}= Mico 
		language-\Pl{}-\Top{}//
	\glft `Languages is what Niva and Mico (each) study.' //
\endgl
\xe
\end{figure}

% [The blog article here first postulates that the case markers might attach
% phonologically to whatever precedes them, which would make them very
% clitic-like in not being `choosy' about attachment only to find that this
% does not make sense. We will thus skip this unnecessary information as well
% as a bunch of examples.]

Taking the above characteristics into account---inability to insert word
material, special positioning, and wide scope---one may argue that the preposed
case markers are clitics. It should be noted furthermore that a single NP
cannot be marked for two grammatical functions at the same time, so that case
markers cannot be coordinated, as is attempted in (\ref{ex:precasecoord}) below
with *\rayr{s nj ːs sopnF}{*sa nay sā Sopan}.

\begin{figure}[h]
\pex
\a\label{ex:precasecoord}\ljudge*\begingl
	\gla Ang @ delacan sa nay sā Sopan. //
	\glb ang= delak=yan.Ø sa nay sā Sopan //
	\glc \AgtT{}= suffer.from=\TplM{}.\Top{} \Parg{} and \Caus{} Sopan //
	\glft \textit{Intended:} `They suffer from and due to Sopan.' //
\endgl

\a\begingl
	\gla Ang @ delacan sa @ Sopan, nay yasa. //
	\glb ang= delak=yan.Ø sa= Sopan nay yasa //
	\glc \AgtT{}= suffer.from=\TplM{}.\Top{} \Parg{}= Sopan and 
		\TsgM{}.\Caus{} //
	\glft `They suffer from Sopan, and due to him.' //
	\endgl
\xe
\end{figure}

The case markers of proper nouns are necessarily proclitics rather than
enclitics to preceding word material, since it is possible for them to begin
utterances, where it is not possible to lean to the left, but only to the 
right. This is the case in equative sentences such as the one in 
(\ref{ex:clitics_11a}). In these cases as well, it is not possible for
parenthetical material to be placed between the case marker and its target of
modification, as in (\ref{ex:clitics_11b}); the particle and its head cohere
closely and behave essentially like a unit.

\begin{figure}[h]
\pex\label{ex:clitics_11}
\a\label{ex:clitics_11a}\begingl
	\gla Ang @ Misan lajāyas puti. //
	\glb ang= Misan lajāy-as puti //
	\glc \Aarg{}= Misan student-\Parg{} zealous //
	\glft `Misan is a zealous student.' //
\endgl

\a\label{ex:clitics_11b}\ljudge{*}\begingl
	\gla Ang, paronyang, Misan lajāyas puti. //
	\glb ang paron=yang Misan lajāy-as puti //
	\glc \Aarg{} believe=\Fsg{}.\Aarg{} Misan student-\Parg{} zealous //
\endgl
\xe
\end{figure}

The fact that case particles attach always to a proper noun very specifically
makes them unlike `typical' clitics, since according to
\citet{spencerluis2012}, a typical and important feature of clitics is their
`promiscuous' attachment, as described initially. This puts case particles
closer to affixes---just like the suffixed case markers. On the other hand, as
previously pointed out, clitics do not have to exhibit all traits often
associated with them in order to be counted as such. Yet more typical of
function words, on the other hand, is the fact that there is no morphophonemic
interaction between a case particle and the word it modifies. Thus, for
instance, there is no form /saːdʒaːn/ resulting from the combination of
\rayr{s}{sa} (\Parg{}) with \rayr{AgYaanF}{Ajān}. This overlap in form between
affix and function word is typical of clitics, according to the traits
excerpted from \citet{spencerluis2012} above.

\label{clitics_prep_dyn}
As discussed previously, \rayr{mN}{manga} may not only modify verbs, but also
ad\-posi\-tions---which in the case of prepositions are often very
transparently derived from nouns. \rayr{mN}{manga} in combination with an
adposition \index{adpositions} indicates that there is motion into the
specified direction. The directional marker \rayr{mN}{manga} is thus a
functional morpheme and it always appears before the adposition itself.
Adpositions do not otherwise inflect, but \rayr{mN}{manga}, due to its
functional nature, could reasonably be construed as inflection, in spite of
appearing as a function word, just as its (related) verbal counterpart. This
double nature makes it a good candidate for a clitic. Applying a shuffling or
coordination test here to figure out whether \rayr{mN}{manga} is an adjuct is
moot, since there is nothing else which can appear in this position---the
position \rayr{mN}{manga} appears in is thus syntactically privileged;
\rayr{mN}{manga} can be said to exhibit special syntax, which is further
evidence for it being a clitic. With regards to the distinction between special
and simple clitics \citep{zwicky1977}, it ought to be classified as the former,
since even though it may be derived from the verb \xayr{mN/}{manga}{move}, this
verb does not constitute the particle's associated full form:

\begin{figure}[h]
\pex\label{ex:clitics_manga}
\a\label{ex:clitics_manga1}\begingl
	\gla Ang @ saraya Ajān manga @ kong nangaya. //
	\glb ang= sara-ya Ajān manga= kong nanga-ya //
	\glc \AgtT{} go-\TsgM{} Ajān \Dir{}= inside house-\Loc{} //
	\glft `Ajān goes into the house.' //
\endgl

\a\label{ex:clitics_manga2}\ljudge{\excl}\begingl
\gla Ang @ saraya mangayam Ajān kong nangaya. //
	\glb ang= sara-ya manga-yam Ajān kong nanga-ya //
	\glc \AgtT{}= go-\TsgM{} move-\Ptcp{} Ajān inside house-\Loc{} //
	\glft `Ajān goes to move inside the house.' //
\endgl
\xe
\end{figure}

Example (\ref{ex:clitics_manga2}) assumes that the hypothetical correct place
of the verb \xayr{mN/}{manga-}{move} to appear in is as a non-finite complement
to the main verb in the sentence, \xayr{sr/}{sara-}{go}. While not
ungrammatical \fw{per se}, the sentence would imply that \rayr{AgYaanF}{Ajān}
walks away in order to move around in the house, which is not what
(\ref{ex:clitics_manga1}) posits. There is thus no direct semantic relationship
between what we assumed to be the historical full form and the grammatical
marker, that is, the full verb and the directional particle cannot be used
interchangeably. When testing with parenthetical word material, it becomes
clear that \xayr{mN koNF}{manga kong}{into} forms a syntactic unit, which is
demonstrated in (\ref{ex:clitics_mangacohesion}). \rayr{mN}{manga} is a bound
morpheme, and thus not a function word proper.

\begin{figure}
\pex\label{ex:clitics_mangacohesion}
\a\label{ex:clitics_mangacohesion1}\begingl
	\gla Ang @ saraya Ajān, \textbf{narayāng}, manga @ kong nangaya. //
	\glb ang= sara-ya Ajān nara=yāng manga= kong nanga-ya //
	\glc \AgtT{}= go-\TsgM{} Ajān say=\TsgM{}.\Aarg{} \Dir{}= inside 
		house-\Loc{} //
	\glft `Ajān goes, he says, into the house.' //
\endgl

\a\label{ex:clitics_mangacohesion2}
	\ljudge{*} Ang saraya Ajān manga, \textbf{narayāng}, kong nangaya.
\a\label{ex:clitics_mangacohesion3}
	Ang saraya Ajān manga kong, \textbf{narayāng}, nangaya.
\xe
\end{figure}

Also, when testing \rayr{mN}{manga}'s behavior in terms of distribution over
coordinated NPs, we can see in (\ref{ex:clitics_mangacoord2}) that there is no
problem in condensing the sentence given in (\ref{ex:clitics_mangacoord1}) to
the extent that \rayr{mN}{manga} governs two adpositions in
coordination---\xayr{midj}{miday}{around} and
\xayr{koNF}{kong}{inside}---sharing the same adpositional object,
\xayr{nN}{nanga}{house}.

\begin{figure}
\pex\label{ex:clitics_mangacoord}
\a\label{ex:clitics_mangacoord1}\begingl
	\gla Ang @ saraya {} @ Ajān manga @ miday nangaya nay manga @ kong 
		nangaya. //
	\glb ang= sara-ya Ø= Ajān manga= miday nanga-ya nay manga= kong nanga-ya //
	\glc \AgtT{}= go-\TsgM{} \Top{}= Ajān \Dir{}= around house-\Loc{} and
		\Dir{}= inside house-\Loc{} //
	\glft `Ajān goes around the house and into the house.' //
\endgl

\a\label{ex:clitics_mangacoord2}\begingl
	\gla Ang @ saraya {} @ Ajān manga @ miday nay kong nangaya. //
	\glb ang= sara-ya Ø= Ajān manga= miday nay kong nanga-ya //
	\glc \AgtT{}= go-\TsgM{} \Top{}= Ajān \Dir{}= around and inside
		house-\Loc{} //
	\glft `Ajān goes around and into the house.' //
\endgl
\xe
\end{figure}

For all intents and purposes, thus, \rayr{mN}{manga} behaves syntactically like
a typical clitic in that it has wide scope over conjuncts, coheres tightly with
its target of modification, is located in a syntactically privileged position,
and unites properties of both function words and inflection.

\label{clitics_preverb_da}
From this discussion of prenominal (and one pre-adpositional) particles, let us
return to verbs again for a moment. Besides the preverbal particles discussed
above, there is also what is spelled as a prefix on the verb which appears to
be a little odd as such in that it can have wide scope over conjoined verbs.
This is the prefix \rayr{d/} {da-} often meaning `so, thus', displayed in
(\ref{ex:clitics_13}).

\begin{figure}
\ex\label{ex:clitics_13}\begingl
	\gla Ang @ da-pinyaya nay hisaya {} @ Yan sa @ Pila. //
	\glb ang= da=pinya-ya nay hisa-ya Ø= Yan sa= Pila //
	\glc \AgtT{}= so=ask-\TsgM{} and beg-\TsgM{} \Top{}= Yan \Parg{}= Pila //
	\glft `Yan asks and begs Pila to (do so).' //
\endgl\xe
\end{figure}

\rayr{d/}{da-}, where it is not used for presentative 
purposes,\footnote{Although this use is probably related to the anaphoric use.}
is a functional morpheme in that it basically acts as an anaphora for a
complementizer phrase (CP) the speaker chooses to drop. Thus, it does not mark
any of the intrinsic morphological categories of the verb (tense, aspect, mood,
modality, finiteness), just as the topic marker marks for none of the verb's
own categories of inflection, but instead refers to a syntactic relation the
verb subcategorizes for. As an anaphora, \rayr{d/}{da-} cannot stand alone,
though it is possible to use a full demonstrative form \xayr{dnY}{danya}{such
one} in its place, compare (\ref{ex:clitics_14}).

\begin{figure}[h]
\ex\label{ex:clitics_14}\begingl
	\gla Ang @ pinyaya nay hisaya {} @ Yan sa @ Pila danyaley. //
	\glb ang= pinya-ya nay hisa-ya Ø= Yan sa= Pila danya-ley //
	\glc \AgtT{}= ask-\TsgM{} and beg-\TsgM{} \Top{}= Yan \Parg{}= Pila
		such.one-\PargI{} //
	\glft `Yan asks and begs Pila such.' //
\endgl\xe
\end{figure}

Unlike the preverbal particles, \rayr{d/}{da-} can be associated with a full
form, though it still displays special syntax in that unlike English
\fw{-n't} or \fw{'ll}, for instance, it does not occur in the same place as
the full form. Note also how \rayr{d/}{da-} is appended to the right of tense
prefixes, which \emph{do} express a property of the verb, as shown in 
(\ref{ex:clitics_15}).

\begin{figure}[h]
\pex\label{ex:clitics_15}
\a\label{ex:clitics_15a}\begingl
	\gla Ang @ da-məpinyaya sa @ Pila. //
	\glb ang= da=mə-pinya=ya.Ø sa= Pila //
	\glc \AgtT{}= so=\Pst{}=ask=\TsgM{}.\Top{} \Parg{}= Pila //
	\glft `He asked Pila to.' //
\endgl

\a\label{ex:clitics_15b}\begingl
	\gla Ang @ da-məpinyaya nay məhisaya {} @ Yan sa @ Pila. //
	\glb ang= da=mə-pinya-ya nay mə-hisa-ya Ø= Yan sa= Pila //
	\glc \AgtT{}= so=\Pst{}-ask-\TsgM{} and \Pst{}-beg-\TsgM{} \Top{}= Yan 
		\Parg{}= Pila //
	\glft `Yan asked and begged Pila to.' //
\endgl
\xe
\end{figure}

The verb form in (\ref{ex:clitics_15}) becomes ungrammatical with the order of
its prefixes reversed, so it is not acceptable to say: 
\rayr{md/pinYy}{məda-pinyaya}, although note that pre- and suffixes proper
also have a fixed order in Ayeri, so this alone is probably not enough evidence
to claim that \rayr{d/}{da-} is not possibly a prefix. Furthermore, while the
tense prefixes undergo crasis, this is not the case with \rayr{d/}{da-}, as
(\ref{ex:clitics_16}) shows.

\begin{figure}[h]
\begin{minipage}[t]{.5\linewidth}
\pex\label{ex:clitics_17}
\a\label{ex:clitics_17a}\begingl
	\gla Māmangreng. //
	\glb mə-amang=reng //
	\glc \Pst{}-happen=\TsgI{}.\Aarg{} //
	\glft `It happened.' //
\endgl

\a\label{ex:clitics_17b}\ljudge{*} \fw{Məamangreng.}
\xe
\end{minipage}
~
\begin{minipage}[t]{.5\linewidth}
\pex\label{ex:clitics_16}
\a\label{ex:clitics_16a}\begingl
	\gla Da-amangreng. //
	\glb da=amang=reng //
	\glc thus=happen=\TsgI{}.\Aarg{} //
	\glft `It happens thus.' //
\endgl

\a\label{ex:clitics_16b}\ljudge{*} \fw{Dāmangreng.}
\xe
\end{minipage}
\end{figure}

Besides the characteristic of not seeking out certain parts of speech, the
\rayr{d/}{da-} prefix satisfies the criteria of being a phonologically reduced
form of an otherwise free functional morpheme, and it occurs in a place where
normal syntax would not put its corresponding full form. It has wide scope over
conjuncts, is attached outside of inflection for proper categories of the verb,
and doesn't interact with its host with regards to morphophonemics. In addition
to these more typical traits of clitics, there is also no way in
(\ref{ex:clitics_18}) to place words between \rayr{d/}{da-} and the verb stem.

% \begin{figure}[h]
\ex\label{ex:clitics_18}\begingl
	\gla Da, naratang, amangreng. //
	\glb da nara=tang amang=reng //
	\glc thus say=\TplM{}.\Aarg{} happen=\TsgI{}.\Aarg{} //
\endgl\xe
% \end{figure}

\label{clitics_preverb_refl}
The prefix \xayr{sitNF/}{sitang-}{self} behaves in the same way as \rayr{d/}
{da-}, since it also abbreviates a reflexive NP, for instance,
\xayr{sitNF/yesF}{sitang-yes}{herself} where `herself' as a patient is
coreferential with the agent of the clause. One might assume that reflexivity
is a verbal category of inflection in Ayeri, although, on the other hand, Ayeri
also does not have any verbs which appear as grammatically reflexive to
indicate unaccusativity like in Romance languages. The reflexive marking in
Ayeri is thus semantically motivated, not functionally.

\begin{figure}[h]
\pex\label{ex:clitics_19}
\a\label{ex:clitics_19a}\begingl
	\gla Adruara biratayreng. //
	\glb adru-ara biratay-reng //
	\glc break-\TsgI{} pot-\AargI{} //
	\glft `The pot broke.' //
\endgl

\a\label{ex:clitics_19b}\ljudge{*}\begingl
	\gla Sitang-adruara biratayreng. //
	\glb sitang=adru-ara biratay-reng //
	\glc self=break-\TsgI{} pot-\AargI{} //
	\glft \textit{Intended:} `The pot broke.' (an unspecified force broke 
		it) //
\endgl
\xe
\end{figure}

\begin{figure}[h]
\pex~\label{ex:clitics_20}%
French:
\a\label{ex:clitics_20a}\begingl
	\gla Le pot s'est cassé. //
	\glb le pot se=est cassé //
	\glc the pot self=be.\Tsg{}.\Prs{} broken //
	\glft `The pot broke.' (an unspecified force broke it) //
\endgl

\a\label{ex:clitics_20b}\begingl
	\gla Le pot est cassé. //
	\glb le pot est cassé //
	\glc the pot be.\Tsg{}.\Prs{} broken //
	\glft `The pot is broken.' //
\endgl
\xe
\end{figure}

\label{clitics_prenoun_dem}
Ayeri has a tendency to reuse prefixes with different parts of speech, and thus
\rayr{d/}{da-} is also used with nouns, forming part of the series of deictic
prefixes, \xayr{d/}{da-}{such (a)}, \xayr{Ed/}{eda-}{this},
\xayr{Ad/}{ada-}{that}. The prefix in all these cases represents a grammatical
function, is unstressed, and may have wide scope over conjoined NPs, unless an
individuating interpretation is intended, as in (\ref{ex:clitics_21b}). These
traits are typical of clitics, as we have seen, though (\ref{ex:clitics_22})
shows that unlike with verbs, the deictic prefixes do undergo crasis, which is
a trait more typically associated with affixes.

\begin{figure}[h]
\pex\label{ex:clitics_21}
\a\label{ex:clitics_21a}\begingl
	\gla Sinyāng eda-ledanas nay viretāyās tondayena-hen? //
	\glb sinya-ang eda=ledan-as nay viretāya-as tonday-ena=hen //
	\glc who-\Aarg{} this=friend-\Parg{} and supporter-\Parg{} 
		art-\Gen{}=all //
	\glft `Who is this friend and supporter of all arts?' //
\endgl

\a\label{ex:clitics_21b}\begingl
	\gla Sinyāng eda-ledanas nay eda-viretāyās tondayena-hen? //
	\glb sinya-ang eda=ledan-as nay eda=viretāya-as tonday-ena=hen //
	\glc sinya-\Aarg{} eda=ledan-\Parg{} nay eda=viretāya-\Parg{} 
		tonday-\Gen{}=hen //
	\glft `Who is/are this friend and this supporter of all arts?' //
\endgl
\xe
\end{figure}

\begin{figure}[h]
\ex\label{ex:clitics_22}\begingl
	\gla Sa @ ming @ nelnang edāyon. //
	\glb sa= ming= nel=nang eda=ayon-Ø //
	\glc \Parg{}= can= help=\Fpl{}.\Aarg{} this=man-\Top{} //
	\glft `This man, we can help him.' //
\endgl\xe
\end{figure}

The deictic prefixes also cannot be used with all types of NPs, only with those
headed by common and proper nouns; the picky nature of the deictic prefixes
also makes them more typical of affixes than of clitics. The preverbal
particles, on the other hand, also only occur with verbs, and it was
nonetheless argued for them to be classified as clitics above due to the
presence of other traits which make the particle under scrutiny clitic-like.

As mentioned initially, \citet{spencerluis2012} give numerous counterexamples
to the catalog of traits typically associated with clitics. One of these
counterexamples is what they call `suspended affixation'. This phenomenon
occurs in Turkish, for instance, where the plural suffix \fw{-lEr} and
subsequent suffixes can be left out in coordination (\ref{ex:clitics_23a}), as
well as case markers (\ref{ex:clitics_23b}), and adverbials with case-like
functions (\ref{ex:clitics_23c}).

\begin{figure}[h]
\pex\label{ex:clitics_23}
Turkish \citep[199]{spencerluis2012}:
\a\label{ex:clitics_23a}\begingl
	\gla bütün kitap\textup{(}…\textup{)} ve defter-ler-imiz //
	\glb all book and notebook-\Pl{}-\Fpl{}.\Poss{} //
	\glft `all our books and notebooks' //
\endgl

\a\label{ex:clitics_23b}\begingl
	\gla Vapur hem Napoli\textup{(}…\textup{)} hem Venedik'-e uğruyormuş //
	\glb boat and Naples and Venice-\Loc{} stops.\Evid{} //
	\glft `Apparently the boat stops at both Naples and Venice' //
\endgl

\a\label{ex:clitics_23c}\begingl
	\gla öğretmen-ler\textup{(}…\textup{)} ve öğrenci-ler-le //
	\glb teacher-\Pl{} and student-\Pl{}-\textsc{with} //
	\glft `with (the) students and (the) teachers' //
\endgl
\xe
\end{figure}

\citet{spencerluis2012} note that, in \textquote{the nominal domain especially,
wide scope inflection is widespread in the languages of Eurasia, becoming more
prominent from west to east}, and that wide scope affixation \textcquote[200]
{spencerluis2012}{can be found with inflectional and derivational morphology in
a number of languages, and it is often a symptom of recent and not quite
complete morphologization}. They report further that \citet{wälchli2005} finds
that suspended affixation is especially common with `natural coordination',
that is, the combination of items very frequently occurring in pairs like
\fw{knife and fork} or \fw{mother and father}, as opposed to cases of
occasional coordination \citep[200]{spencerluis2012}. Whether this is also true
for Ayeri as of now would require a separate survey.\footnote{Or rather,
devising supplemental rules.} Ayeri is not (intended to be) of Eurasian stock,
though since there is evidence for this phenomenon, it should at least be
considered.

Given the evidence from Turkish, the categorization of deictic prefixes as
\emph{either} affixes \emph{or} clitics is unclear, especially since the
diagnostic of scope is devalued by the Turkish examples. On the other hand,
suffixes on nouns do not behave this way in Ayeri, as demonstrated in
(\ref{ex:clitics_24})---they rather behave like typical affixes in that they
mandatorily occur on each conjunct. The question is, thus, whether an exception
should be made for prefixes on nouns. We may as well assume that they are
clitics.

\begin{figure}[h]
\pex\label{ex:clitics_24}
\a\label{ex:clitics_24a}\begingl
	\gla sobayajang nay lajāyjang //
	\glb sobaya-ye-ang nay lajāy-ye-ang //
	\glc teacher-\Pl{}-\Aarg{} and student-\Pl{}-\Aarg{} //
	\glft `(the) teachers and (the) students' //
\endgl

\a\label{ex:clitics_24b}\ljudge{*}\begingl
	\gla sobayaye nay lajāyjang //
	\glb sobaya-ye nay lajāy-ye-ang //
	\glc teacher-\Pl{} and student-\Pl{}-\Aarg{} //
\endgl

\a\label{ex:clitics_24c}\ljudge{*}\begingl
	\gla sobaya nay lajāyjang //
	\glb sobaya nay lajāy-ye-ang //
	\glc teacher and student-\Pl{}-\Aarg{} //
\endgl
\xe
\end{figure}

From a functional point of view, the exact nature of the deictic prefixes
should not matter either way---\citet[Feature Table]{pargram} also cites a
\Deix{}\textsc{is} feature with \Prox{}\textsc{imal} and \Dist{}\textsc{al}
as its values, which fits \xayr{Ed/}{eda-}{this} and \xayr{Ad/}{ada-}{that}
just fine. At present it is unclear, however, how to represent `such (a)' in
this respect, since it is clearly deictic, but neither  proximal nor distal. In
this case, it should be possible to use [\Deix{} $this$/$that$/$such$] as well,
hence:

\pex\label{ex:clitics_25}
\a\label{ex:clitics_25a}\begingl
	\gla edāyon //
	\glb eda=ayon //
	\glc this=man //
	\glft `this man' //
\endgl

\a\label{ex:clitics_25b}\begin{avm}
\[
	\Pred{}	&	`man' \\
	\Deix{}	&	$this$ \\
\]
\end{avm}
\xe

As described above, proper nouns are case marked by clitic case markers
preceding the noun. In fact, these markers must be located somewhere at the
left periphery of the NP, so the deictic prefixes stand in between the case
marker and the proper noun itself, which is unproblematic for lexical
integrity, since the deictic prefixes are not free morphemes. And even if they
were part of inflection, the case markers, as clitics, would be on the 
outside---the order \textsc{deictic prefix} -- \textsc{case marker} -- 
\textsc{noun} is ungrammatical. An example of this is given in 
(\ref{ex:clitics_26}).

\begin{figure}[h]
\pex\label{ex:clitics_26}
\a\label{ex:clitics_26a}\begingl
	\gla Ang @ koronay sa @ eda-Kagan. //
	\glb ang= koron=ay.Ø sa= eda=Kagan //
	\glc \AgtT{}= know=\Fsg{}.\Top{} \Parg{}= this=Kagan //
	\glft `I know this Kagan.' //
\endgl

\a\label{ex:clitics_26b}\ljudge{*}\begingl
	\gla Ang @ koronay eda-Kaganas. //
	\glb ang= koron=ay.Ø eda=Kagan-as //
	\glc \AgtT{}= know=\Fsg{}.\Top{} this=Kagan-\Parg{} //
\endgl

\a\label{ex:clitics_26c}\ljudge{*}\begingl
	\gla Ang @ koronay eda-sa @ Kagan. //
	\glb ang= koron=ay.Ø eda=sa= Kagan //
	\glc \AgtT{}= know=\Fsg{}.\Top{} this=\Parg{}= Kagan //
\endgl
\xe
\end{figure}

The question now is, what happens to coordinated proper nouns? Since the
suffixed case markers on common nouns have the distributional properties of
affixes, they occur on every conjunct, the deictic prefixes, however, only
occur on the first unless an individuating reading is intended, as shown in
(\ref{ex:clitics_20}). For proper nouns it ought to be possible for both a case
marker and a deictic prefix to have scope over coordinated proper nouns, as in
(\ref{ex:clitics_27a}). Yet, however, this seems slightly odd-sounding, so the
strategy in (\ref{ex:clitics_27b}) is preferrable, which avoids the problem
altogether by making the names an apposition to the demonstrative
\xayr{EdnY}{edanya}{this/these one(s)}.\footnote{Honestly, it is these cases
where you wish it were possible to just ask a speaker of your language for
their judgement instead of relying on your own intuition---which will most
certainly be tainted by interference from your native language---in order not
to carry over all too familiar patterns into your creation.} The example in
(\ref{ex:clitics_27c}) is unproblematic and here as well indicates that the two
persons are referred to individually and not as a group.

\begin{figure}[h]
\pex\label{ex:clitics_27}
\a\label{ex:clitics_27a}\begingl
	\gla Ang @ koronay sa @ eda-​Kagan nay Ijān. //
	\glb ang= koron=ay.Ø sa= eda=Kagan nay Ijān //
	\glc \AgtT{}= know=\Fsg{}.\Top{} \Parg{}= this=​Kagan and Ijān //
	\glft `I know these Kagan and Ijān.' //
\endgl

\a\label{ex:clitics_27b}\begingl
	\gla Ang @ koronay edanyās, Kagan nay Ijān. //
	\glb ang= koron=ay.Ø edanya-as Kagan nay Ijān //
	\glc \AgtT{}= know=\Fsg{}.\Top{} this.one-\Parg{} Kagan and Ijān //
	\glft `I know these, Kagan and Ijān.' //
\endgl

\a\label{ex:clitics_27c}\begingl
	\gla Ang @ koronay sa @ eda-​Kagan nay eda-​Ijān. //
	\glb ang= koron=ay.Ø sa= eda=​Kagan nay eda=Ijān //
	\glc \AgtT{}= know=\Fsg{}.\Top{} \Parg{}= this=​Kagan and this=​Ijān //
	\glft `I know this Kagan and this Ijān.' //
\endgl
\xe
\end{figure}

\label{clitics_preadj_da}
Of the deictic prefixes, \rayr{d/}{da-} is not only available to verbs and
nouns, but also to adjectives. Like with verbs, it is short for \xayr{dnY}
{danya}{(such) one} in this case, as demonstrated in (\ref{ex:clitics_28a}).
The resulting meaning is `the \textsc{adjective} one'; \rayr{d/}{da-}
essentially acts as a nominalizer, at least to the extent that the compound of
\rayr{d/}{da-} and an adjective inherits the distributional properties of
\rayr{dnY}{danya} as a demonstrative pronoun. Thus, it can be case- and 
topic-marked, as shown by (\ref{ex:clitics_28}bc). It can also be modified by
another adjective, as in (\ref{ex:clitics_28c}). On the other hand, it cannot
be reduplicated for diminution, and can also not be pluralized. Since
adjectives follow their heads, the original order of
\textsc{demonstrative}--\textsc{adjective} remains intact. \rayr{d/}{da-} is
thus similar in distribution to English simple clitics such as \fw{'ll}, which
occurs in the same place as its full form, the future tense auxiliary
\fw{will}.

\begin{figure}[h]
\pex\label{ex:clitics_28}
\a\label{ex:clitics_28a}\begingl
	\gla Le @ noyang danyaley tuvo. //
	\glb le= no=yang danya-Ø tuvo //
	\glc \PatTI{}= want=\Fsg{}.\Aarg{} such.one-\Top{} red //
	\glft `The red one I want.' //
\endgl

\a\label{ex:clitics_28b}\begingl
	\gla Ang @ noay da-tuvoley. //
	\glb ang= no=ay.Ø da-tuvo-ley //
	\glc \AgtT{}= want=\Fsg{}.\Top{} one-red-\PargI{} //
	\glft `I want the red one.' //
\endgl

\a\label{ex:clitics_28c}\begingl
	\gla Le @ noyang da-tuvo kivo. //
	\glb le= no=yang da-tuvo-Ø kivo //
	\glc \PatTI{}= want=\Fsg{}.\Aarg{} one=red-\Top{} small //
	\glft `The little red one I want.' //
\endgl
\xe
\end{figure}

The prefix, again, coheres tightly in that no additional material can be
inserted. Like with nouns above, inflecting each form in a group of coordinated
adjectives results in an individuating reading in (\ref{ex:clitics_29a}). It
should be possible for the prefix to take wide scope as in
(\ref{ex:clitics_29b}). However, it seems better to me to instead rephrase the
coordinated adjective as a relative clause like in (\ref{ex:clitics_28c}), for
instance, besides using the full form \rayr{dnY}{danya} + adjectives. Since
case is obligatorily marked on every conjunct, (\ref{ex:clitics_29d}) is not
grammatical.

\begin{figure}[h]
\pex\label{ex:clitics_29}
\a\label{ex:clitics_29a}\begingl
	\gla Ang @ noay da-tuvoley nay da-lenoley. //
	\glb ang= no=ay.Ø da=tuvo-ley nay da=leno-ley //
	\glc \AgtT{}= want=\Fsg{}.\Top{} one=red-\PargI{} and one=blue-\PargI{} //
	\glft `I want the blue one and the red one.' //
\endgl

\a\label{ex:clitics_29b}\ljudge{\ques}\begingl
	\gla Ang @ noay da-tuvoley nay lenoley. //
	\glb ang= no=ay.Ø da=tuvo-ley nay leno-ley //
	\glc \AgtT{}= want=\Fsg{}.\Top{} one=red-\PargI{} and blue-\PargI{} //
	\glft `I want the red and blue one.' //
\endgl

\a\label{ex:clitics_29c}\begingl
	\gla Ang @ noay adaley si tuvo nay leno. //
	\glb ang= no=ay.Ø ada-ley si tuvo nay leno //
	\glc \AgtT{}= want=\Fsg{}.\Top{} that-\PargI{} \Rel{} red and blue //
	\glft `I want that which is red and blue.' //
\endgl

\a\label{ex:clitics_29d}\ljudge{*}\begingl
	\gla Ang @ noay da-tuvo nay lenoley. //
	\glb ang= no=ay.Ø da=tuvo nay leno-ley //
	\glc \AgtT{}= want=\Fsg{}.\Top{} one=red and blue-\PargI{} //
\endgl
\xe
\end{figure}

Possessive pronouns like \xayr{naa}{nā}{my}, \xayr{vn}{vana}{your}, etc.\
behave the same way when derived from their usual role as modifiers to
free-standing anaphoras (\xayr{d/naa}{da-nā}{mine},
\xayr{d/vn}{da-vana}{yours}, etc.), except they cannot themselves be modified
by adjectives in the way \xayr{d/tuvo}{da- tuvo}{the red one} is in
(\ref{ex:clitics_28c}). Taking all of the examples above into account,
\rayr{d/}{da-} with adjectives and possessive pronouns seems to be most like a
simple clitic according to \citet{zwicky1977}'s definition, compared to the
other contexts it can appear in:

\blockcquote[5]{zwicky1977}{Cases where a free morpheme, when unaccented, may
be phonologically subordinated to a neighboring word. Cliticization of this
sort is usually associated with stylistic conditions, as in the casual speech
cliticization of object pronouns in English; there are both formal \emph{full}
pronouns and casual \emph{reduced} pronouns.}

Typical of a simple clitic as well, the distribution of \rayr{d/}{da-} is
restricted by grammatical context, as pointed out regarding example
(\ref{ex:clitics_27b}). Unlike in English, which \citet{zwicky1977} gives
examples of, the condition in Ayeri is likely not merely phonological in this
case. The nature of the condition, however, is not predetermined in
\citet{spencerluis2012}, when they elaborate on \citet{zwicky1977}'s definition
that

\blockcquote[44]{spencerluis2012}{we may therefore need to define simple
clitics along the lines of \citet{halpern1998}, namely, as clitics that may be
positioned in a subset of the positions within which the full forms are found,
rather than as clitics that have the same distribution as their full-form
counterparts as in \citet{zwicky1977}. Under this broader definition, we
capture the fact that simple clitics differ from special clitics in that they
can appear in some of the positions that are occupied by their corresponding
full forms, while special clitics never can.}

\label{clitics_prenoun_ku}
Besides deictic prefixes, nouns may also receive a prefix expressing likeness,
\rayr{ku/}{ku-}. This prefix is also applicable to adjectives, and is maybe
more adverbial in terms of semantics than purely functional morphemes like
\rayr{d/} {da-}. In contrast to \rayr{d/}{da-}, \rayr{ku/}{ku-} has no 
full-form equivalent. Some examples of it leaning on nouns are given in
(\ref{ex:clitics_30}). Like the deictic prefixes, \rayr{ku/}{ku-} appears in a
position which is restricted to functional morphemes. Any other modifiers
which appear as free words or phrases (adjectives, relative clauses, nominal
adjuncts) follow nouns and cannot appear in the position of \rayr{ku/}{ku-}.
Slightly untypical of a clitic, again, it is not fully `promiscuous' regarding
its  phonological host in that it requires a nominal, adjectival or phrasal
host.

\begin{figure}[h]
\pex\label{ex:clitics_30}
\a\label{ex:clitics_30a}\begingl
	\gla Ang @ misya {} @ Amān ku-depangas. //
	\glb ang= mis-ya Ø= Amān ku=depang-as //
	\glc \AgtT{}= act-\TsgM{} \Top{}= Amān like=fool-\Parg{} //
	\glft `Amān acts like a fool.' //
\endgl

\a\label{ex:clitics_30b}\begingl
	\gla Ang @ misya {} @ Amān ku-depangas nay karayās. //
	\glb ang= mis-ya Ø= Amān ku-depang-as nay karaya-as //
	\glc \AgtT{}= act-\TsgM{} \Top{}= Amān like=fool-\Parg{} and 
		coward-\Parg{} //
	\glft `Amān acts like a fool and a coward.' //
\endgl

\a\label{ex:clitics_30c}\begingl
	\gla Ang @ misya {} @ Amān ku-depangas nay ku-karayās. //
	\glb ang= mis-ya Ø= Amān ku-depang-as nay ku-karaya-as //
	\glc \AgtT{}= act-\TsgM{} \Top{}= Amān like=fool-\Parg{} and 
		like=coward-\Parg{}	//
	\glft `Amān acts like a fool and like a coward.' //
\endgl

\a\label{ex:clitics_30d}\begingl
	\gla Ang @ misya {} @ Amān ku-ada-depangas. //
	\glb ang= mis-ya Ø= Amān ku=ada=depang-as //
	\glc \AgtT{}= act-\TsgM{} \Top{}= Amān like=that=fool-\Parg{} //
	\glft `Amān acts like that fool.' //
\endgl

\a\label{ex:clitics_30e}\begingl
	\gla Ang @ misya {} @ Amān ku-ada-depangas nay ada-karayās. //
	\glb ang= mis-ya Ø= Amān ku=ada=depang-as nay ada=karayās //
	\glc \AgtT{}= act-\TsgM{} \Top{}= Amān like=that=fool-\Parg{} and 
		that=coward-\Parg{} //
	\glft `Amān acts like that fool and that coward.' //
\endgl

\a\label{ex:clitics_30f}\ljudge{*}\begingl
	\gla Ang @ misya {} @ Amān ada=ku=depangas. //
	\glb ang= mis-ya Ø= Amān ada=ku=depang-as //
	\glc \AgtT{}= act-\TsgM{} \Top{}= Amān that=like=fool-\Parg{} //
\endgl
\xe
\end{figure}

Generally, \rayr{ku/}{ku-} fulfills the function of the preposition \fw{like}
in English in (\ref{ex:clitics_30}). However, if it were a preposition in
Ayeri, it should trigger the locative on its dependent. In the examples above,
however, the NP which \rayr{ku/}{ku-} modifies takes the patient case, like
predicative NPs are otherwise wont to do. Moreover, while prepositions like
\xayr{koNF}{kong}{inside} in (\ref{ex:clitics_31}) are free morphemes in Ayeri,
\rayr{ku/}{ku-} is bound, which becomes apparent by introducing a parenthetical
remark in (\ref{ex:clitics_32}).

\begin{figure}[h]
\pex\label{ex:clitics_31}
\a\label{ex:clitics_31a}\begingl
	\gla Ang yomāy, \textbf{surpareng}, kong sayanya. //
	\glb ang yoma=ay.Ø surpa=reng kong sayan-ya //
	\glc \AgtT{} be=\Fsg{}.\Top{} seem=\TsgI{}.\Aarg{} inside cave-\Loc{} //
	\glft `I am, it seems, inside a cave.' //
\endgl

% \a\label{ex:clitics_31b}\begingl
% 	\gla Ang yomāy kong, \textbf{suprareng}, sayanya. //
% 	\glb ang yoma=ay.Ø kong surpa=reng sayan-ya //
% 	\glc \AgtT{} be=\Fsg{}.\Top{} inside seem=\TsgI{}.\Aarg{} cave-\Loc{} //
% 	\glft `I am inside, it seems, a cave.' //
% \endgl

\a\label{ex:clitics_31b}
	Ang yomāy kong, \textbf{suprareng}, sayanya.
\xe
\end{figure}

\begin{figure}[h]
\pex~\label{ex:clitics_32}
\a\label{ex:clitics_32a}\begingl
	\gla Ang @ misya {} @ Amān, \textbf{surpareng}, ku-depangas. //
	\glb ang= mis-ya Ø= Amān surpa=reng ku=depang-as //
	\glc \AgtT{}= acts=\TsgM{} \Top{}= Amān seem=\TsgI{}.\Aarg{} 
		like=fool-\Parg{} //
	\glft `Amān acts, it seems, like a fool.' //
\endgl

% \a\label{ex:clitics_32b}\ljudge{*}\begingl
% 	\gla Ang misya {} Amān ku, \textbf{surpareng}, depangas. //
% 	\glb ang mis-ya Ø Amān ku surpa=reng depang-as //
% 	\glc \AgtT{} acts=\TsgM{} \Top{} Amān like seem=\TsgI{}.\Aarg{} 
% 		fool-\Parg{} //
% \endgl

\a\label{ex:clitics_32b}\ljudge{*}
	Ang misya Amān ku, \textbf{surpareng}, depangas.
\xe
\end{figure}

Examples (\ref{ex:clitics_30}ab) show that similar to the deictic prefixes,
\rayr{ku/}{ku-} precedes its target of modification and can have wide scope
over coordinated NPs. As (\ref{ex:clitics_30c}) shows, narrow scope is possible
as well, and in this case, again, each conjunct is to be interpreted separately
instead of \rayr{ku/}{ku-} modifying both conjuncts collectively. As
illustrated in (\ref{ex:clitics_30d}), \rayr{ku/}{ku-} even precedes
\rayr{Ad/}{ada-} as a deictic prefix, for instance, if they appear together.
Reversing the order of the prefixes is not possible, as is shown in
(\ref{ex:clitics_30f}). As (\ref{ex:clitics_30e}) shows, \rayr{ku/}{ku-} may
also have scope over two individuated noun phrase conjuncts. Besides nouns,
\rayr{ku/}{ku-} is also applicable to pronouns, which makes
(\ref{ex:clitics_33}) possible, for example.

\begin{figure}[h]
\pex\label{ex:clitics_33}
\a\label{ex:clitics_33a}\begingl
	\gla Ang @ silvye {} @ Pada ku-yes. //
	\glb ang= silv-ye Ø= Pada ku=yes //
	\glc \AgtT{}= look-\TsgF{} \Top{}= Pada like=\TsgF{}.\Parg{} //
	\glft `Pada looks like her.' //
\endgl

\a\label{ex:clitics_33b}\begingl
	\gla Sa @ silvye ang @ Pada ku-ye. //
	\glb sa= silv-ye ang= Pada ku=ye //
	\glc \PatT{}= look-\TsgF{} \Aarg{}= Pada like=\TsgF{}.\Top{} //
	\glft `Like \emph{her} Pada looks.' //
\endgl
\xe
\end{figure}

With proper nouns, the same distributional properties as with generic nouns
apply, except that \rayr{ku/}{ku-} appears, rather idiosyncratically, as a
suffix at the right edge of an NP---or at the right edge of the first NP
conjunct---if the NP is preceded by a case marker, as shown in
(\ref{ex:clitics_34}). \label{clitics_preadj_ku} With adjectives, however,
there are no idiosyncrasies to this degree. \rayr{ku/} {ku-} appears only as a
prefix here, as with generic nouns, compare (\ref{ex:clitics_35}).

\begin{figure}[h]
\pex\label{ex:clitics_34}
\a\label{ex:clitics_34a}\begingl
	\gla Ang @ lentava sa @ Tagāti diyan-ku. //
	\glb ang= lenta=va.Ø sa= Tagāti diyan=ku //
	\glc \AgtT{}= sound=\Second{}.\Top{} \Parg{}= Tagāti worthy=like //
	\glft `You sound like Mr. Tagāti.' //
\endgl

\a\label{ex:clitics_34b}\begingl
	\gla Ang @ lentava sa @ Tagāti diyan-ku nay diranas yana. //
	\glb ang= lenta=va.Ø sa= Tagāti diyan=ku nay diran-as yana //
	\glc \AgtT{}= sound=\Second{}.\Top{} \Parg{}= Tagāti worthy=like and 
		uncle-\Parg{} \TsgM{}.\Gen{} //
	\glft `You sound like Mr. Tagāti and his uncle.' //
\endgl

\a\label{ex:clitics_34c}\begingl
	\gla Sa @ lentavāng ku-​Tagāti diyan. //
	\glb sa= lenta=vāng ku=​Tagāti diyan //
	\glc \PatT{}= sound=\Second{}.\Aarg{} like=​Tagāti worthy //
	\glft `Like Mr. Tagāti you sound.' //
\endgl
\xe
\end{figure}

\begin{figure}[h]
\pex\label{ex:clitics_35}
\a\label{ex:clitics_35a}\begingl
	\gla Surpya ku-suta ang Maran. //
	\glb surp-ya ku=suta ang Maran //
	\glc seem-\TsgM{} like=busy \Aarg{} Maran //
	\glft `Maran seems like he's busy.' //
\endgl

\a\label{ex:clitics_35b}\begingl
	\gla Surpya ku-suta nay baras ang Maran. //
	\glb surp-ya ku=suta nay baras ang Maran //
	\glc seem-\TsgM{} like=busy and gruff \Aarg{} Maran //
	\glft `Maran seems like he's busy and gruff.' //
\endgl
\xe
\end{figure}

As (\ref{ex:clitics_35b}) shows, \rayr{ku/}{ku-} again can have wide scope over
conjuncts. What further distinguishes \rayr{ku/}{ku-} from a prefix here is
that it does not undergo crasis if the adjective begins with an /u/, hence
\rayr{ku/Ubo}{ku-ubo} /kuˈubo/ `like bitter', not *\,\rayr{kuubo}{*kūbo}
/ˈkuːbo/. Again, the position \rayr{ku/}{ku-} appears in is special in that
whatever modifies adjectives usually trails after them.

\label{clitics_preverb_ku}
Besides attaching to words, \rayr{ku/}{ku-} is furthermore able to subordinate
non-finite CPs. Since \rayr{ku/}{ku-} leans on a whole phrase in 
(\ref{ex:clitics_35}), which affixes (at least in Ayeri) otherwise cannot do,
its status as a clitic should be unmistakable in this context.

\begin{figure}[h]
\ex\label{ex:clitics_36}\begingl
	\gla Silvyeng ku-tahayam misungas. //
	\glb silv=yeng ku=taha-yam misung-as //
	\glc look=\TsgF{}.\Aarg{} like=have-\Ptcp{} secret-\Parg{} //
	\glft `She looks as though having a secret.' //
\endgl\xe
\end{figure}

\subsubsection{Suffixes}
\label{subsubsec:suffixes}

\label{clitics_postverb_person}
Besides a number of prefixes and particles occuring before lexical heads which
are likely clitics, Ayeri also has a number of morphemes trailing lexical heads
as suffixes which do not seem quite like typical inflection. These are, for
one, part of the person suffixes on the verb. Especially tricky in this regard
is maybe that \textquote{a pronominal affix or incorporated pronominal is
effectively a clitic masquerading as an affix. Therefore, if there are
pronominal affixes then they should behave exactly like clitics with respect to
crucial aspects of morphosyntax} \parencites[144]{spencerluis2012}[also
compare][101]{corbett2006}. \citet{spencerluis2012} then proceed to give
examples from Breton and Irish where the person marking on the verb is in
complementary distribution with full NPs, as illustrated in
(\ref{ex:clitics_37}) and (\ref{ex:clitics_38}).

\begin{figure}
\pex\label{ex:clitics_37}
Breton \parencites[145]{spencerluis2012}[from][]{borsleyetal2007}:
\a\label{ex:clitics_37a}
\begingl
	\gla Bremañ e lennont al levrioù //
	\glb now \Prt{} read.\Prs{}.\Tpl{} the books //
	\glft `Now they are reading the books' //
\endgl

\a\label{ex:clitics_37b}\begingl
	\gla Bremañ e lenn ar vugale al levrioù //
	\glb now \Prt{} read.\Prs{}.\Tsg{} the children the books //
	\glft `Now the children are reading the books' //
\endgl

\a\label{ex:clitics_37c}\ljudge{*}\begingl
	\gla Bremañ e lennont ar vugale al levrioù //
	\glb now \Prt{} read.\Prs{}.\Tpl{} the children the books //
\endgl
\xe
\end{figure}

\begin{figure}
\pex\label{ex:clitics_38}
Irish \parencites[145]{spencerluis2012}[from][]{mccloskeyhale1984}:
\a\label{ex:clitics_38a}
\begingl
	\gla Chuirfinn \textup{(}*mé\textup{)} isteach ar an phost sin //
	\glb put.\Cond{}.\Fsg{} (​I​) in on the job that //
	\glft `I would apply for that job' //
\endgl

\a\label{ex:clitics_38b}\begingl
	\gla Chuirfeadh sibh isteach ar an phost sin //
	\glb put.\Cond{}.\Tsg{} you in on the job that //
	\glft `You would apply for that job' //
\endgl

\a\label{ex:clitics_38c}\begingl
	\gla Chuirfeadh Eoghan isteach ar an phost sin //
	\glb put.\Cond{}.\Tsg{} Owen in on the job that //
	\glft `Owen would apply for that job' //
\endgl
\xe
\end{figure}

What we can see in (\ref{ex:clitics_37}) is that, according to
\citet{spencerluis2012}, the verb shows no number marking, defaulting to the
singular form, in non-negative clauses if the subject of the verb is overt as
either a full noun or a pronoun: plural marking on the verb and a full subject
cannot coincide in this case, which is why (\ref{ex:clitics_37c}) is marked
ungrammatical. In (\ref{ex:clitics_38a}) we can see that there is no need for
an explicit first-person pronoun, since that function is already expressed by
person marking on the verb—person inflection on the verb seems to be in
complementary distribution with full subject pronouns at least for some parts
of the paradigm. In (\ref{ex:clitics_38b}) we have an overt second-person
subject pronoun, but in this case, the verb does not agree with it and instead
defaults to the third-person form, a clear case of which is given in
(\ref{ex:clitics_38c}).

While in Ayeri, there is no defaulting to a certain person in the presence of
an overt subject NP as such, there is still the effect of complementary
distribution between a pronominal suffix in the absence of an overt subject NP,
and a functionally impoverished as well as phonologically reduced form in its
presence, compare (\ref{ex:clitics_39}) to (\ref{ex:clitics_41}).

\begin{figure}
\ex\labels\label{ex:clitics_39}
\begin{minipage}[t]{.5\remaining}
\tl\quad\label{ex:clitics_39a}\begingl
	\gla Suta ang @ Niyas. //
	\glb suta ang= Niyas //
	\glc busy \Aarg{}= Niyas //
	\glft `Niyas is busy.' //
\endgl
\end{minipage}
~
\begin{minipage}[t]{.5\remaining}
\tl\quad\label{ex:clitics_39b}\begingl
	\gla Yāng suta. //
	\glb yāng suta //
	\glc \TsgM{}.\Aarg{} busy //
	\glft `He is busy.' //
\endgl
\end{minipage}
\xe\smallskip

\ex\labels\label{ex:clitics_40}
\begin{minipage}[t]{.5\remaining}
\tl\quad\label{ex:clitics_40a}\begingl
	\gla Lampya ang @ Niyas. //
	\glb lamp-ya ang= Niyas //
	\glc walk-\TsgM{} \Aarg{}= Niyas //
	\glft `Niyas walks.' //
\endgl
\end{minipage}
~
\begin{minipage}[t]{.5\remaining}
\tl\quad\label{ex:clitics_40b}\begingl
	\gla Lampyāng. //
	\glb lamp=yāng //
	\glc walk=\TsgM{}.\Aarg{} //
	\glft `He walks.' //
\endgl
\end{minipage}
\xe\smallskip

\ex\labels\label{ex:clitics_41}
\begin{minipage}[t]{.5\remaining}
\tl\quad\label{ex:clitics_41a}\ljudge{*}\begingl
	\gla Lapyāng ang @ Niyas. //
	\glb lamp=yāng ang= Niyas //
	\glc walk=\TsgM{}.\Aarg{} \Aarg{}= Niyas //
\endgl
\end{minipage}
~
\begin{minipage}[t]{.5\remaining}
\tl\quad\label{ex:clitics_41b}\ljudge{*}\begingl
	\gla Lampya yāng. //
	\glb lamp-ya yāng //
	\glc walk-\TsgM{} \TsgM{}.\Aarg{} //
\endgl
\end{minipage}
\xe
\end{figure}

Example (\ref{ex:clitics_39}b) shows the free form of the third singular
masculine agent pronoun, \xayr{yaaNF}{yāng}{he}. This is in complementary
distribution with a full NP, which in (\ref{ex:clitics_39}a) is \rayr{ANF
niysF}{ang Niyas}. In (\ref{ex:clitics_40}a) we can see that the verb agrees
with the subject NP in person, gender and number in that it exhibits the suffix
\rayr{/y}{-ya}. If, like in (\ref{ex:clitics_40}b), the overt subject NP is
missing, the verb is marked with the same form as the free pronoun,
\rayr{/yaaNF}{-yāng}, which feeds the verb as a syntactic argument. That is,
the person suffix itself realizes the \Subj{} function of the verb's argument
structure; no other exponent of person features is required, as
(\ref{ex:clitics_41}) illustrates. The definitions in (\ref{ex:clitics_42})
list the constituent parts of \xayr{lMpYaaNF}{lampyāng}{he walks} and their
associated grammatical features.\footnote{Normally, due to the lexical
integrity principle, \rayr{lMpYaaNF}{lampyāng} should be listed as one, but
Ayeri's very regular agglutinating nature makes splitting composed words very
convenient for illustration as in this case.}

\begin{figure}[h]
\begin{morphlex}
\ex\label{ex:clitics_42}
\adjustbox{valign=t}{%
\begin{tabu} to \remaining {@{} X[25l] X[5l] X[70l]}
\savetabu{morphlex}
\rayr{\larger lMpF/}{lamp-} (`walk')
	&	V\tsub{stem}
	&	\begin{tabular}[t]{l l l}
			\ups{\Pred}	& = & \astruct{walk}{\ups{\Subj{}}} \\
		\end{tabular}
\medskip\\

\rayr{\larger /yaaNF}{-yāng} (`he')
	&	V\tsub{infl}
	&	\begin{tabular}[t]{l l l}
			\ups{\Subj{}} & = & ↓ \\
				\quad\downs{\Pred{}}	& = & `$pro$' \\
				\quad\downs{\Pers{}}	& = & \Third{} \\
				\quad\downs{\Num{}}		& = & \Sg{} \\
				\quad\downs{\Gend{}}	& = & \M{} \\
				\quad\downs{\Anim{}}	& = & $+$ \\
				\quad\downs{\Case{}}	& = & \Aarg{} \\
		\end{tabular}
\end{tabu}%
}
\xe
\end{morphlex}
\end{figure}

\begin{figure}
\begin{morphlex}
\ex\label{ex:clitics_43}
\begin{avm}
\[
	\Pred{}	&	\astruct{walk}{\ups{\Subj{}}} \\

	\Subj{}	&	\[
					\Pred{}	&	`$pro$' \\
					\Pers{}	&	\Third{} \\
					\Num{}	&	\Sg{} \\
					\Gend{}	&	\M{} \\
					\Anim{}	&	$+$ \\
					\Case{}	&	\Aarg{} \\
				\]
\]
\end{avm}
\xe
\end{morphlex}
\end{figure}

Example (\ref{ex:clitics_43}) is an attempt to conceptualize in a formal way
that functionally, the inflection takes the role of the subject relation. Thus,
(\ref{ex:clitics_41}a) is ungrammatical in that the pronominal suffix
\rayr{/yaaNF}{-yāng} on the verb is redundant in the presence of a full NP
which expresses the same features except that the subject NP's \Pred{} value
is \rayr{niysF}{Niyas}, not `$pro$'. In effect, what is attempted in
(\ref{ex:clitics_41}), is to fill a grammatical relation with essentially the
same content in two places, which is redundant. Assuming a `$pro$' value for
the \Pred{} feature of \xayr{/yaaNF}{-yāng}{he} is \Lfg{}'s way to
model the fact that this pronominal suffix functions as a pro-form available
for predication, like a pronoun. Pronouns and full NPs necessarily exclude each
other, however.

Example (\ref{ex:clitics_44}) shows, then, the annotations for
\xayr{lMpY}{lampya}{walks} as agreeing with an overt NP. Here, the suffix does
not have a \Pred{} feature---it is not available for predication, so that a
full NP is permissible as a controller of agreement in the clause, with the
person suffix as its person-inflection agreement target. The agreement suffix
\rayr{/y}{-ya} thus reflects that the subject NP needs to have a certain set of
person features. The NP which controls verb agreement (in canonical cases the
agent NP) needs to match these features in order to establish an agreement
relationship. By constraining (\req{}) the subject's predicator to not be a
pro-form in (\ref{ex:clitics_44}), it should also be possible to rule out cases
like in (\ref{ex:clitics_41b}), where person agreement is triggered by a
pronominal NP. This is ungrammatical, since Ayeri basically supplants person
agreement with a pronominal suffix in those cases, as we have seen. If
\rayr{/yaaNF}{-yāng} were a simple inflectional affix, one of the two examples 
in (\ref{ex:clitics_41}) should be grammatical.

\begin{figure}[h]
\begin{morphlex}
\ex\label{ex:clitics_44}
\adjustbox{valign=t}{%
\begin{tabu} to \remaining {@{} X[10l] X[5l] X[85l]}
\savetabu{morphlexnarrow}
\rayr{\larger /y}{-ya₁}
	&	V\tsub{infl}
	&	\begin{tabular}[t]{l l l}
			\ups{\Subj{}} & = & ↓ \\
				\quad\downs{\Pred{}}	& \req{} & ¬\,`$pro$' \\
				\quad\downs{\Pers{}}	& = & \Third{} \\
				\quad\downs{\Num{}}		& = & \Sg{} \\
				\quad\downs{\Gend{}}	& = & \M{} \\
				\quad\downs{\Anim{}}	& = & $+$ \\
		\end{tabular}
\end{tabu}
}
\xe
\end{morphlex}
\end{figure}

The behavior of the pronominal person marking on the verb is thus rather
complex, and decidedly unlike inflection in that what looks like an affix on
the verb is also an argument of it, like a pronoun, as displayed in
(\ref{ex:clitics_42}). Another layer of complexity is added by the fact that
such an incorporated pronoun is also eligible for topicalization. As we have
seen above, topic marking on nouns is realized by suppressing the realization
of the overt case marker, whether it is a proclitic or a suffix. The
topic-marked forms of pronouns are also underspecified for case, and they
happen to be the same as those of the person-agreement suffixes, as exemplified
by \rayr{/y}{-ya} in (\ref{ex:clitics_40a}). Thus, a topic-marked pronominal
suffix on the verb will look exactly like ordinary agreement with a full NP,
except that there is no full NP to agree with---hence the subscript numbers in
(\ref{ex:clitics_44}) and (\ref{ex:clitics_45}) to distinguish between both
kinds of \rayr{/y}{-ya}.\footnote{Assuming that the person suffix on the verb
always co-indexes the topic and that it is therefore unnecessary to distinguish
a person-agreement suffix from a homophonous topicalized pronominal suffix is a
premature conclusion. In fact, the verb always agrees with its subject, which
is most often the agent NP. The topic may consist of any NP, also oblique ones.
The verb generally does not agree with non-agent or non-patient NPs, however.}

\begin{figure}[h]
\begin{morphlex}
\ex\label{ex:clitics_45}
\adjustbox{valign=t}{%
\begin{tabu} {\usetabu{morphlexnarrow}}
\rayr{\larger /y}{-ya₂}
	&	V\tsub{infl}
	&	\begin{tabular}[t]{l l l}
			\ups{\Subj{}} & = & ↓ \\
				\quad\downs{\Pred{}}	& = & `$pro$' \\
				\quad\downs{\Pers{}}	& = & \Third{} \\
				\quad\downs{\Num{}}		& = & \Sg{} \\
				\quad\downs{\Gend{}}	& = & \M{} \\
				\quad\downs{\Anim{}}	& = & $+$ \\
				\quad\downs{\Case{}}	& = & Ø $\implies$ \pass{\Top{}} \\
		\end{tabular}
\end{tabu}
}
\xe
\end{morphlex}
\end{figure}

Comparing the feature list in (\ref{ex:clitics_45}) with that in
(\ref{ex:clitics_42}) and (\ref{ex:clitics_44}), we see that
(\ref{ex:clitics_45}) is basically the same as (\ref{ex:clitics_42}), except
that either the \Case{} feature is absent, or that the suffix is
underspecified for case. In absence of an NP to agree with, it follows from
this definitional lack that the person marking on the verb itself is to be
identified as constituting the topic, and the correspondent of the preverbal
topic marker. In the following case, the preverbal topic marker defines that
the topic is an animate agent; this information is united with the functional
annotations in (\ref{ex:clitics_45}).

\begin{figure}[h]
\begin{morphlex}
\ex\label{ex:clitics_46}
\adjustbox{valign=t}{%
\begin{tabu} {\usetabu{morphlexnarrow}}
\rayr{\larger ANF}{ang}
	&	Cl
	&	\begin{tabular}[t]{l l l}
			\ups{\Top{}} & = & ↓ \\
				\quad\downs{\Case{}}	& = & \Aarg{} \\
				\quad\downs{\Anim{}}	& = & $+$ \\
		\end{tabular}
\end{tabu}
}
\xe
\end{morphlex}
\end{figure}

Instances of other case-unmarked nouns can be ruled out as being also part of
the topic relation on the grounds of cohesion and functional uniqueness: if the
topic is defined as an agent and it cannot be assumed from context that the
case-unmarked noun in question is also part of the agent NP, discard it as a
candidate.\footnote{As described in \autoref{subsec:relcs}, there may be
multiple topics under very limited circumstances.} Besides, every core thematic
role (agent, patient, recipient) can only be assigned once, so if the role
specified by the topic marker is already assigned, another NP in the same
clause cannot also be assigned the same role. This gets more difficult with
non-core roles, though it may be assumed that oblique arguments are less likely
to be topicalized.

Possibly confusing with regards to the status of the pronominal suffixes as
cli\-tics is that \textcquote[144]{spencerluis2012}{a pronominal affix or
incorporated pronominal is effectively a clitic masquerading as an affix}.
While the pronominal suffixes in Ayeri behave in a special way regarding
syntax, they lack wide scope, which is typical of affixes (apart from the
examples from Turkish in (\ref{ex:clitics_23})). Unlike Breton in
(\ref{ex:clitics_37}) or Irish in (\ref{ex:clitics_38}), Ayeri's pronominal
affixes do not default to some form, and verbs cannot be unmarked either, that
is, verbs always have to be inflected in some way, mostly for phonotactic
reasons. Thus, in coordination, every conjunct has to be inflected for person
features, as (\ref{ex:clitics_47}) shows.

\begin{figure}[h]
\pex\label{ex:clitics_47}
\a\label{ex:clitics_47a}\begingl
	\gla Nedrayāng nay layayāng. //
	\glb nedra=yāng nay laya-yāng //
	\glc sit=\TsgM{}.\Aarg{} and read=\TsgM{}.\Aarg{} //
	\glft `He is sits and reads.' //
\endgl

\a\label{ex:clitics_47b}\ljudge{*}\begingl
	\gla Nedrayāng nay laya. //
	\glb nedra=yāng nay laya //
	\glc sit=\TsgM{}.\Aarg{} and read //
\endgl

\a\label{ex:clitics_47c}\ljudge{*}\begingl
	\gla Nedra nay layayāng. //
	\glb nedra nay laya=yāng //
	\glc sit and read=\TsgM{}.\Aarg{} //
\endgl
\xe
\end{figure}

In the case of \xayr{nedFr/}{nedra-}{sit} and \xayr{ly/}{laya-}{read} in
(\ref{ex:clitics_47}), leaving off the person marking would theoretically
generate valid words, since *\rayr{nedFr}{*nedra} and *\rayr{ly}{*laya} satisfy
phonotactic constraints (see \autoref{sec:phonotactics}). However, Ayeri also
has a great number of verb stems which end in a consonant cluster, such as
\xayr{AnFlF/} {anl-}{bring} or \xayr{tpY/}{tapy-}{set}, which do not form valid
words as bare stems since words cannot end in CC. What would be possible
instead is that one conjunct might carry the full pronominal suffix as a
`strong' form and the other one might only partially co-index the required
features by using the less specific corresponding agreement marker as a `weak'
form. Differential marking of this kind, though, is simply not established.

\label{clitics_quant}
After briefly delving into the realm of syntax, let us return to morphology for
the second group of suffixes which need clarification. While Ayeri has
quantifiers which are independent words, there are also a number of very common
`little' quantifiers and degree adverbs which are customarily spelled as
suffixes, for instance, \xayr{/IknF}{-ikan}{much, many; very},
\xayr{/kj}{kay}{few; a little}, \xayr{/nm}{nama}{just, only}, and
\xayr{/nYm}{-nyama}{even}. All of these are adverbial in meaning and while they
are comparatively light in their semantics compared to regular content words,
they do not particularly resemble functional morphemes either.

A natural language which also contains suffixed quantifiers is West Greenlandic
\citep{bittner1995}. According to \citet{bittner1995}'s terminology,
\rayr{/IknF}{-ikan} in Ayeri as modifying a noun would be a D-quantifier, since
it forms \textcquote[59]{bittner1995}{a constituent [with] a projection of N}.
This is in contrast to A-quantifiers, which are defined as forming
\textcquote[59]{bittner1995}{a constituent with some projection of V}. That is,
A-quantifiers are adverbs like \fw{almost} (\rayr{/NsF}{-ngas} in Ayeri),
\fw{mostly}, or \fw{never}, which modify verbs, while D-quantifiers are words
like \fw{most}, \fw{some}, or \fw{every}, which modify nouns. Ayeri makes no
distinction between A- and D-quantifiers with regards to their being treated as
suffixes, however, so that one may find suffixed quantifiers in both groups,
sometimes even to the extent that the same quantifier may modify both nouns or
verbs. Example (\ref{ex:clitics_48}) gives two instances of suffixed
quantifiers from West Greenlandic for comparison with Ayeri in
(\ref{ex:clitics_49}).

%\needspace{3\baselineskip}
\begin{figure}[h]
\pex\label{ex:clitics_48}
West Greenlandic \citep[60, 63]{bittner1995}:
\a\label{ex:clitics_48a}\begingl
	\gla qaatuur-tuaanna-ngajap-p-a-a //
	\glb break-always-almost-\Ind{}-\Tr{}-\Tsg{}\tsub{1}.\Tsg{}\tsub{2} //
	\glft `he almost always breaks it' //
\endgl

\a\label{ex:clitics_48b}\begingl
	\gla qaqutigu-rujussuaq //
	\glb rarely-very //
	\glft `very rarely' //
\endgl
\xe
\end{figure}

\begin{figure}
\pex\label{ex:clitics_49}
\a\label{ex:clitics_49a}\begingl
	\gla Ang @ adruya tadayen-ngas adaley //
	\glb ang= adru=ya.Ø tadayen=ngas ada-ley //
	\glc \AgtT{}= break=\TsgM{}.\Top{} always=almost that-\PargI{} //
	\glft `he almost always breaks it' //
\endgl

\a\label{ex:clitics_49b}\begingl
	\gla kora-ikan //
	\glb kora=ikan //
	\glc rarely=very //
	\glft `very rarely' //
\endgl
\xe
\end{figure}

As we can see in (\ref{ex:clitics_48a}), West Greenlandic incorporates the
quantifier suffixes into the verb, while Ayeri---not a polysynthetic
language---proceeds more freely in (\ref{ex:clitics_49a}), in that
\xayr{tdyenF} {tadayen}{always, every time} is an adverb and as such a free
morpheme which is, however, in turn modified by a suffixed quantifier. Since
orthography may be treacherous, let us first try to establish whether
\xayr{/NsF}{-ngas}{almost} and \xayr{/IknF}{-ikan}{many, much, very} and their
like are free morphemes or not. As discussed initially regarding the preverbal
particles, it is possible to reorder free morphemes, while clitics---as bound
morphemes---cannot move around. Adverbs and adjectives are, if they optionally
add additional information to a lexical head, adjuncts, and according to
\citet{carnie2013} it is possible for adjuncts to switch places within the same
syntactic domain. Adjuncts can also be coordinated with other adjuncts in the
same syntactic domain. Furthermore, it is possible to replace \xbar{X} nodes
with pro-forms, like \fw{one} in English.

\begin{figure}[h]
\pex\label{ex:clitics_50}
\a\label{ex:clitics_50a}\begingl
	\gla kipisānye-ikan bino kāryo //
	\glb kipisān-ye=ikan bino kāryo //
	\glc painting-\Pl{}=many colorful big //
	\glft `many big colorful paintings' //
\endgl

\a\label{ex:clitics_50b} \fw{kipisānye-ikan kāryo bino} \\
	`many colorful big paintings'

\a\label{ex:clitics_50c} \ljudge{\excl} \fw{kipisānye bino-ikan kāryo} \\
	\hphantom{\excl}`very colorful big paintings'

\a\label{ex:clitics_50d} \ljudge{\excl} \fw{kipisānye bino kāryo-ikan} \\
	\hphantom{\excl}`very big colorful paintings'
\xe
\end{figure}

\begin{figure}[h]
\pex\label{ex:clitics_51}
\a\label{ex:clitics_51a}\begingl
	\gla kipisānye-ikan bino nay kāryo //
	\glb kipisān-ye=ikan bino nay kāryo //
	\glc painting-\Pl{}=many colorful and big //
	\glft `many big and colorful paintings' //
\endgl

\a\label{ex:clitics_51b} \ljudge{*} \fw{kipisānye-ikan nay bino kāryo} \\
	\hphantom{*}`many and colorful big paintings'

\a\label{ex:clitics_51c} \ljudge{\excl} \fw{kipisānye bino-ikan nay kāryo} \\
	\hphantom{\excl}`big and very colorful paintings'
\xe
\end{figure}

As (\ref{ex:clitics_50}cd) shows, moving \xayr{/IknF}{-ikan}{many, much, very}
into different positions results not necessarily in ungrammatical expressions,
but in expressions with meanings different from what was intended, since
\rayr{/IknF} {-ikan}'s scope changes from the noun to the adjective it is
appended to. On the other hand, comparing (\ref{ex:clitics_50a}) and (b), it is
possible for \xayr{kaarYo}{kāryo}{big} and \xayr{bino}{bino}{colorful} to
switch places with no ill effects. Example (\ref{ex:clitics_51b}) demonstrates
that placing a coordinating conjunction between \rayr{/IknF}{-ikan} and
\rayr{bino}{bino} does not work. The coordination in (\ref{ex:clitics_51c}), on
the other hand, is not a problem---not because it is possible to coordinate
\rayr{/IknF}{-ikan} and \rayr{kaarYo}{kāryo}, but because
\xayr{bino/IknF}{bino-ikan}{very colorful} is considered one syntactic unit
which is coordinated with \rayr{kaarYo}{kāryo}. Thus, in
(\ref{ex:clitics_50b}), we have actually been trying to coordinate
\xayr{kipisaanYe/IknF}{kipisānye-ikan}{many paintings} with \xayr{bino}{bino}
{colorful}, which does not work, since it is not possible to coordinate a
lexical head with an adjunct supposed to modify it, because they are of
different syntactic categories. In this regard it is worth mentioning that
Ayeri's quantifier suffixes are rather not complements either, since they are
not required to satisfy their head's argument structure.

One might argue that in (\ref{ex:clitics_50}) and (\ref{ex:clitics_51}) we
tried to compare apples to oranges in that \xayr{/IknF}{-ikan}{many, much,
very} and \xayr{bino}{bino}{colorful} are of different categories, since they
do not appear to operate on the same levels. So instead, let us look at
possibilities of word order change and coordination between different
quantifiers to ensure that we actually stay on the same level. With this comes
the problem, however, that it seems strange to modify the same lexical head
with multiple different quantifiers, so this test does not really seem suitable
to produce grammatical results. Also, with regards to coordination of
quantifiers, it is maybe more natural to oppose them with
s\xayr{soyNF}{soyang}{or} than to coordinate them; the grammatical structure of
two categorially identical elements connected by a grammatical conjunction
(even if the meaning is disjunctive) remains the same in either case.

\begin{figure}[h]
\pex\label{ex:clitics_52}
\a\label{ex:clitics_52a}\ljudge{*}\begingl
	\gla keynam-ikan-kay //
	\glb keynam=ikan-kay //
	\glc people=many-few //
	\glft `few many people' //
\endgl

\a\label{ex:clitics_52b}\ljudge{\ques}\begingl
	\gla keynam-ikan soyang kay //
	\glb keynam=ikan soyang kay //
	\glc people=many or few //
	\glft `few or many people' //
\endgl
\xe
\end{figure}

In example (\ref{ex:clitics_52a}) we see that it is indeed not possible to
combine multiple quantifiers to jointly modify a head in the way it is possible
for multiple adjectives to modify the same head as in (\ref{ex:clitics_50a}),
for instance. The example of quantifier disjunction in (\ref{ex:clitics_52b})
is also odd unless we permit a reading where
\xayr{kejnmF}{keynam}{people} has been suppressed in the second disjunct to
avoid repetition, although in the corresponding case of (\ref{ex:clitics_53b})
below, \xayr{d/kj}{da-kay}{few ones} would be preferable.

\begin{figure}[h]
\pex\label{ex:clitics_53}
\a\label{ex:clitics_53a}
	\ljudge{\ques\ques} \fw{keynam}[\fw{-ikan soyang -kay}] \\
		`[few or many] people'

\a\label{ex:clitics_53b}
	\ljudge{\ques} [\fw{keynam$_i$-ikan}]\fw{ soyang }[\fw{\_$_i$-kay}]	\\
		`[few \_$_i$] or [many people$_i$]'
\xe
\end{figure}

Both tests, moving \xayr{/IknF}{-ikan}{many, much, very} into other positions
and coordination, have failed so far, and we have evidence that
\rayr{/IknF}{-ikan} forms a syntactic unit with its head, which points to it
being a bound morpheme similar to an affix in spite of its adverb-like meaning.
As with free words, it is also possible to replace a quantifier's head with a
pro-form, as mentioned above in the comment on (\ref{ex:clitics_53b}), and
shown in more detail in (\ref{ex:clitics_54}). With quantifier suffixes there
seems to be an overlap between word-like and affix-like properties, which is
typical of clitics.

\begin{figure}[h]
\pex\label{ex:clitics_54}
\a\label{ex:clitics_54a}\begingl
	\gla Ang @ vacyan keynam-ikan seygoley. //
	\glb ang= vac-yan keynam-Ø=ikan seygo-ley. //
	\glc \AgtT{}= like-\TplM{} people-\Top{}=many apple-\PargI{} //
	\glft `Many people like apples.' //
\endgl

\a\label{ex:clitics_54b}\begingl
	\gla Ang @ vacyan danya-ikan seygoley. //
	\glb ang= vac-yan danya-Ø=ikan seygo-ley. //
	\glc \AgtT{}= like-\TplM{} such.one-\Top{}=many apple-\PargI{} //
	\glft `Many of them like apples.' //
\endgl

\a\label{ex:clitics_54c}\begingl
	\gla Ang @ vacyan da-ikan seygoley. //
	\glb ang= vac-yan da=ikan-Ø seygo-ley. //
	\glc \AgtT{}= like-\TplM{} one=many-\Top{} apple-\PargI{} //
	\glft `Many (of them) like apples.' //
\endgl
\xe
\end{figure}

Somewhat untypical of affixes, it seems to be possible to modify suffixed
quantifiers with adverbs like \xayr{EkenF}{ekeng}{too} and \xayr{kgnF}{kagan}
{far too}, as (\ref{ex:clitics_55}) shows. This suggests that at least in this
context, \xayr{/IknF}{-ikan}{many, much, very} may actually be the lexical head
of an adverbial phrase, which is at odds with its status as a bound morpheme,
at least as far as constituent structure is concerned.

\begin{figure}[h]
\ex\label{ex:clitics_55}\begingl
	\gla Ang @ vacyan keynam-ikan kagan disuley. //
	\glb ang= vac-yan keynam-Ø=ikan kagan disu-ley //
	\glc \AgtT{}= like-\TplM{} people-\Top{}=many far.too disu-\PargI{} //
	\glft `Far too many people like bananas.' //
\endgl\xe
\end{figure}

Inserting parenthetical word material in between morphemes as a test for
coherence may be especially interesting in the face of (\ref{ex:clitics_55}),
since here it is not entirely clear whether \xayr{kejnmF/IknF kgnF}{keynam-ikan
kagan}{too many people} forms a single unit, or whether \xayr{kgnF}{kagan}{far
too} is an adjunct to \xayr{kejnmF/IknF}{keynam-ikan}{many people}. Since signs
point to the status of suffixed quantifiers as clitics, there is the
possibility that \xayr{/IknF kgnF}{-ikan kagan}{far too many} constitutes a
clitic cluster similar to the preverbal one. Example (\ref{ex:clitics_56}),
therefore, lists examples which try to split up the expression at every
relevant point. According to this test, it looks indeed as though
\rayr{kejnmF/IknF kgnF}{keynam-ikan kagan} forms a syntactic unit, in that 
\rayr{/IknF kgnF}{-ikan kagan} cannot be split up internally and also cannot be
divided from \rayr{/IknF}{-ikan}'s head, \xayr{kejnmF}{keynam}{people}. On the
other hand, it is also possible to use other adverbs like
\xayr{ptu}{patu}{surprisingly} with quantifiers, as in (\ref{ex:patuquant_1}).

\begin{figure}[h]
\pex\label{ex:clitics_56}
\a\label{ex:clitics_56a}\begingl
	\gla Ang @ vacyan, \textbf{narayang}, keynam-ikan kagan disuley. //
	\glb ang= vac-yan nara=yang keynam-Ø=ikan kagan disu-ley //
	\glc \AgtT{}= like-\TplM{} say=\Fsg{}.\Aarg{} people-\Top{}=many far.too
		disu-\PargI{} //
	\glft `Far too many people, I say, like bananas.' //
\endgl

\a\label{ex:clitics_56b}
	\ljudge{*} Ang vacyan keynam, \textbf{narayang}, ikan kagan disuley.
\a\label{ex:clitics_56c}
	\ljudge{*} Ang vacyan keynam-ikan, \textbf{narayang}, kagan disuley.
\a\label{ex:clitics_56d}
	Ang vacyan keynam-ikan kagan, \textbf{narayang}, disuley.
\xe
\end{figure}

\begin{figure}[h]
\ex\label{ex:patuquant_1}\begingl
	\gla keynam-ikan patu //
	\glb keynam=ikan patu //
	\glc people=many surprisingly //
	\glft `surprisingly many people' //
\endgl\xe
\end{figure}

The question here as well is whether \rayr{ptu}{patu} refers to just 
\rayr{/IknF}{-ikan} or to \rayr{kejnmF/iknF}{keynam-ikan}. 
Replacing \rayr{kejnmF}{keynam} with a pronoun produces a grammatical outcome
(\ref{ex:patuquant_2_keynam}); doing so with \rayr{kejnmF/IknF}{keynam-ikan},
however, does not (\ref{ex:patuquant_2_keynamikan}). Replacing just 
\rayr{/IknF}{-ikan} at good last produces a very questionable expression as
well, however (\ref{ex:patuquant_2_ikan}).

\begin{figure}[h]
\pex\label{ex:patuquant_2}
\a\label{ex:patuquant_2_keynam}\begingl
	\gla keynam-ikan patu //
	\glb keynam=ikan patu //
	\glc people=many surprisingly //
	\glft `surprisingly many people' //
\endgl

\a\label{ex:patuquant_2_keynamikan}\ljudge*\begingl
	\gla danyāng patu //
	\glb danya-ang patu //
	\glc such.one-\Aarg{} surprisingly //
	\glft `surprisingly ones' //
\endgl

\a\label{ex:patuquant_2_ikan}\ljudge\ques\ques\begingl
	\gla keynam da-patu //
	\glb keynam da=patu //
	\glc people so=surprisingly //
	\glft `surprisingly so people' //
\endgl
\xe
\end{figure}

Another interesting distributional property of suffixed quantifiers in Ayeri is
that in spite of their being suffixed (for instance, to verbs) they can form
arguments of the verb, similar to pronominal suffixes. Thus, with verbs like
\xayr{koMd/}{kond-}{eat}, \xayr{/m}{-ma}{enough} appears suffixed to the verb
instead of as a predicative adverb. Incidentally, the examples in
(\ref{ex:clitics_57}) also show that a quantifier attaches after pronominal
suffixes, which we have already established as being clitics. An inflectional
affix would not normally appear in post-clitic position, which is further
evidence to the hypothesis that quantifier suffixes in Ayeri are clitics.

\begin{figure}
\begin{minipage}[t]{.55\remaining}
\pex\label{ex:clitics_57}
\a\label{ex:clitics_57a}\begingl
	\gla Kondanang=ma. //
	\glb kond=nang=ma //
	\glc eat=\Fpl{}.\Aarg{}=enough //
	\glft `We ate enough.' //
\endgl

\a\label{ex:clitics_57b}\begingl
	\gla Ang tangay-ikan vana. //
	\glb ang tang=ay.Ø=ikan vana //
	\glc \AgtT{} hear=\Fsg{}.\Top{}=much \Second{}.\Gen{} //
	\glft `I've heard much about you.' //
\endgl
\xe
\end{minipage}
~
\begin{minipage}[t]{.45\remaining}
\pex\label{ex:clitics_58}
\a\label{ex:clitics_58a}\begingl
	\gla Adareng edaya. //
	\glb ada-reng edaya //
	\glc that-\AargI{} here //
	\glft `It is here.' //
\endgl

\a\label{ex:clitics_58b}\begingl
	\gla Adareng-ma. //
	\glb ada-reng=ma //
	\glc that-\AargI{}=enough //
	\glft `That/It is enough.' //
\endgl
\xe
\end{minipage}
\end{figure}

Since Ayeri possesses a zero copula, equative phrases which treat quantifier
suffixes as predicative adverbs pose a difficulty in that quantifier suffixes
cannot stand alone like predicatives normally would. Thus, similar to the
behavior of \xayr{/m}{-ma}{enough} in (\ref{ex:clitics_57a}), the predicative
\rayr{/m}{-ma} in (\ref{ex:clitics_58b}) cliticizes to the only available word:
the subject, \xayr{AdreNF}{adareng}{that}.

If quantifier suffixes are clitics, they should also have wide scope over
conjuncts. Here as well, quantifier suffixes behave like typical clitics,
though, in that they can have scope over a conjunct as a whole, although not
totally unambiguously so.

\begin{figure}
\pex\label{ex:clitics_59}
\a\label{ex:clitics_59a}\begingl
	\gla Ang @ tahisayan koyās nay kihasley-ikan. //
	\glb ang= tahisa=yan.Ø koya-as nay kihas-ley=ikan //
	\glc \AgtT{}= own=\TplM{}.\Top{} book-\Parg{} and map-\PargI{}=many //
	\glft `They own many books and maps.' //
\endgl

\a\label{ex:clitics_59b}\begingl
	\gla Veneyang alingo nay para-ven. //
	\glb veney-ang alingo nay para=ven //
	\glc dog.\Aarg{} clever and quick=pretty //
	\glft `The dog is pretty clever and quick.' //
\endgl
\xe
\end{figure}

Thus, in (\ref{ex:clitics_59a}), while \rayr{koyaasF nj kihsFlej/IknF}{koyās
nay kihasley-ikan} is translated as `many books and maps' (nouns do not mark
plural if modified by a quantifier which indicates plurality), another possible
reading is `a book and many maps'. Ways to force the latter reading explicitly
are, for one, to use \xayr{koyaasF menF}{koyās men}{one/a single book}, or
alternatively, to reduplicate the coordinator \xayr{nj}{nay}{and} to
\xayr{njnj} {naynay}{and also}. Context should be sufficient to indicate the
correct reading of (\ref{ex:clitics_59a}) under normal circustances, however.
The same applies to (\ref{ex:clitics_59b}), where the non-distributive reading
can be made explicit by using \rayr{njnj}{naynay} instead of simple
\rayr{nj}{nay}. In both (\ref{ex:clitics_59a}) and (b), if the first conjunct
is modified by an adjective, the distribution of the quantifier over both
conjuncts is also blocked. Thus, in (\ref{ex:clitics_60a}), there is `a big
book and many maps', and in (\ref{ex:clitics_60b}) `the dog' is `surprisingly
clever and pretty quick'.

\begin{figure}[h]
\pex\label{ex:clitics_60}
\a\label{ex:clitics_60a}\begingl
	\gla Ang @ tahisayan koyās kāryo nay kihasley-ikan. //
	\glb ang= tahisa=yan.Ø koya-as kāryo nay kihas-ley=ikan //
	\glc \AgtT{}= own=\TplM{}.\Top{} book-P big and map-\PargI{}=many //
	\glft `They own a big book and many maps.' \\
		\textit{Not:} `They own many big books and maps.' //
\endgl

\a\label{ex:clitics_60b}\begingl
	\gla Veneyang alingo patu nay para-ven. //
	\glb veney-ang alingo patu nay para=ven //
	\glc dog-\Aarg{} clever surprisingly and quick=pretty //
	\glft `The dog is surprisingly clever and pretty quick.' \\
		\textit{Not:} `The dog is surprisingly pretty clever and quick.' //
\endgl
\xe
\end{figure}

The interpretations marked as errorneous in (\ref{ex:clitics_60}) can be
correctly achieved by ordinarily placing the adjective after the coordinated
constituent so that the adjective itself has scope over both conjuncts. This is
demonstrated in (\ref{ex:clitics_61}) and (\ref{ex:clitics_62}). Again, an
unambiguous and individuating interpretation can be achieved by placing the
quantifier suffix on each conjunct.

\begin{figure}
\pex\label{ex:clitics_61}
\a\label{ex:clitics_61a}\begingl
	\gla Ang @ tahisayan koyajas nay kihasyeley kāryo. //
	\glb ang= tahisa=yan.Ø koya-ye-as nay kihas-ye-ley kāryo //
	\glc \AgtT{}= own=\TplM{}.\Top{} book-\Pl{}-\Parg{} and map-\Pl{}-\PargI{}
		big //
	\glft `They own big books and maps.' //
\endgl

\a\label{ex:clitics_61b}\begingl
	\gla Ang @ tahisayan koyās nay kihasley-ikan kāryo. //
	\glb ang= tahisa=yan.Ø koya-as nay kihas-ley=ikan kāryo //
	\glc \AgtT{}= own=\TplM{}.\Top{} book-\Parg{} and map-\PargI{}=many 
		big //
	\glft `They own many big books and maps.' //
\endgl
\xe
\end{figure}

\begin{figure}[h]
\pex\label{ex:clitics_62}
\a\label{ex:clitics_62a}\begingl
	\gla Veneyang alingo nay para patu. //
	\glb veney-ang alingo nay para patu //
	\glc dog-\Aarg{} clever and quick surprisingly //
	\glft `The dog is surprisingly clever and quick.' //
\endgl

\a\label{ex:clitics_62b}\begingl
	\gla Veneyang alingo nay para-ven patu. //
	\glb veney-ang alingo nay para=ven patu //
	\glc dog-\Aarg{} clever and quick=pretty surprisingly //
	\glft `The dog is surprisingly pretty clever and quick.' //
\endgl
\xe
\end{figure}

The comparative suffixes on adjectives, \rayr{/ENF}{-eng} (\Comp{}) and
\rayr{/vaa}{-vā} (\Supl{}) are obviously derived from their quantifier
counterparts meaning `rather' and `most', which poses a slight problem. This
is, whether they act as clitics as well, or whether grammaticalization has
stripped them of some of the clitic-like properties of quantifier suffixes.
Consider, for instance example (\ref{ex:clitics_comp}).

\begin{figure}[h]
\pex\label{ex:clitics_comp}
\a\label{ex:clitics_comp1}\begingl
	\gla Ang @ koronya {} @ Kaman apyanas palay nay ban-eng. //
	\glb ang= koron-ya Ø= Kaman apyan-as palay nay ban-eng //
	\glc \AgtT{}= know-\TsgM{} \Top{}= Kaman joke-\Parg{} funny and
		good-\Comp{} //
	\glft `Kaman knows a rather funny and good joke.' \\
		\textit{or:} `Kaman knows a funnier and better joke.' //
\endgl

\a\label{ex:clitics_comp2}\begingl
	\gla Ang @ koronya {} @ Kaman apyanas palay-eng nay ban-eng. //
	\glb ang= koron-ya Ø= Kaman apyan-as palay-eng nay ban-eng //
	\glc \AgtT{}= know-\TsgM{} \Top{}= Kaman joke-\Parg{} funny-\Comp{} and
		good-\Comp{} //
	\glft `Kaman knows a funnier and better joke.' \\
		\textit{or:} `Kaman knows a rather funny and rather good joke.' //
\endgl

\a\label{ex:clitics_comp3}\begingl
	\gla Ang @ koronya {} @ Kaman apyanas palay-eng nay(nay) da-ban-eng. //
	\glb ang= koron-ya Ø= Kaman apyan-as palay-eng nay(nay) da-ban-eng //
	\glc \AgtT{}= know-\TsgM{} \Top{}= Kaman joke-\Parg{} funny-\Comp{}
		and(\til{}also) one-good-\Comp{} //
	\glft `Kaman knows a funnier joke and (also a) better one.' \\
		\textit{or:} `Kaman knows a rather funny joke and (also a) rather good 
		one.' //
\endgl
\xe
\end{figure}

What all of the above examples in (\ref{ex:clitics_comp}) show is that in
principle, both an interpretation of \rayr{/ENF}{-eng} as a quantifier suffix
and as a comparative suffix are possible, so that it is not easy to clearly
distinguish between an inflectional affix and a clitic here. A clear
distinction also cannot be made on phonological grounds in that even in the
reading as a quantifier clitic, \rayr{/ENF}{-eng} (\Comp{}) and
\rayr{/vaa}{-vā} (\Supl{}) are stressed 
\parencite[compare][90--92]{spencerluis2012}. With \rayr{/ENF}{-eng} it is
possible at least to test whether it undergoes crasis if appended to an
adjective stem ending in \rayr{/E}{-e}. As we have seen, however, this is not a
fully reliable metric either in that deictic prefixes show clitic-like behavior
but may still phonologically meld with the stem they attach to. The only
adjective ending in \rayr{/E}{-e} currently listed in the dictionary is 
\xayr{nke}{nake}{large, tall}. The combination of \rayr{nke}{nake} with both
kinds of \rayr{/ENF}{-eng} is tested in (\ref{ex:clitics_nake}).

\begin{figure}[h]
\pex\label{ex:clitics_nake}
\a\label{ex:clitics_nake1}\begingl
	\gla Ang @ tahayan enonley nake-eng. //
	\glb ang= taha=yan.Ø enon-ley nake=eng //
	\glc \Aarg{}= have=\TplM{}.\Top{} tower-\PargI{} tall=rather //
	\glft `They have a rather tall tower.' //
\endgl

\a\label{ex:clitics_nake2}\begingl
	\gla Ang @ tahayan enonley si nakēng da-nana. //
	\glb ang= taha=yan.Ø enon-ley si nake-eng da=nana-na //
	\glc \Aarg{}= have=\TplM{}.\Top{} tower-\PargI{} \Rel{} tall-\Comp{}
		one=\Fpl{}.\Gen{}-\Gen{} //
	\glft `They have a tower taller than ours.' //
\endgl
\xe
\end{figure}

As illustrated by (\ref{ex:clitics_nake2}), the purely comparative variant of
\rayr{/ENF}{-eng} should be able to be affected by crasis of two alike vowels.
Since adjectives in \rayr{/E}{-e} are exceedingly rare, though, this
observation should not matter much in effect.

\index{clitics|)}

\section{Marking strategies}
\label{sec:markstrat}
\index{marking strategies}
\index{dependent marking}

With regards to the dichotomy head--dependent marking, Ayeri is rather  
thoroughly dependent marking, albeit with the exception of agreement 
morphology on the verb. Dependent marking is exhibited, for instance, in the 
expression of possessive relationships, where the dependent is marked for 
genitive case\index{cases!genitive}:

\pex
\a\label{ex:gennoun}%
\begin{minipage}[t]{.5\remaining}%
\begingl
	\gla dema \textbf{na} @ Tuvo //
	\glb dema \textbf{na}= Tuvo //
	\glc aunt \textbf{\Gen{}}= Tuvo //
	\glft `Tuvo's aunt' //
\endgl
\end{minipage}
~
\begin{forest}
where n children=0{tier=word,edge=dotted,font=\itshape}{}
[{dema}, for tree={calign=first}
	[{dema}, name=head]
	[{na Tuvo}, for tree={calign=first}
		[{\textbf{na} Tuvo}, name=dependent]
	]
]
\node at (current bounding box.south) [below=0pt of head]
	{\textsc{\tiny head}};
\node at (current bounding box.south) [below=0pt of dependent] 
	{\textsc{\tiny dependent}};
\end{forest}

\a\label{ex:genprn}%
\begin{minipage}[t]{.5\remaining}%
\begingl
	\gla kasu bariri \textbf{nā} //
	\glb kasu bari-ri \textbf{nā} //
	\glc basket meat-\Ins{} \Fsg{}.\textbf{\Gen{}} //
	\glft `my basket of meat' //
\endgl
\end{minipage}
~
\begin{forest}
where n children=0{tier=word,edge=dotted,font=\itshape}{}
[{kasu}, for tree={calign=first}
	[{kasu}, name=head]
	[{bariri}, for tree={calign=first}
		[{bariri}]
	]
	[{nā}, for tree={calign=first}
		[{\textbf{nā}}, name=dependent]
	]
]
\node at (current bounding box.south) [below=0pt of head]
	{\textsc{\tiny head}};
\node at (current bounding box.south) [below=0pt of dependent] 
	{\textsc{\tiny dependent}};
\end{forest}
\xe

In (\ref{ex:gennoun}), \rayr{tuvo}{Tuvo} is grammatically in possession of her 
\xayr{dem}{dema}{aunt}; the possessee forms the head of the phrase while it is 
modified by the possessor, which receives the marking. In (\ref{ex:genprn}), 
\xayr{ksu}{kasu}{basket} forms the head and thus also the possessee while 
\xayr{naa}{nā}{my} serves as the dependent possessor; the genitive case is, 
then again, marked on the dependent. A further example of dependent marking is 
the locative case, which is marked on the prepositional object while the 
preposition itself, as the head of the PP, does not receive marking:

\ex\label{ex:loc}
\begin{minipage}[t]{.5\remaining}%
\begingl
	\gla agonan minkay\textbf{ya} //
	\glb agonan minkay\textbf{-ya} //
	\glc outside village\textbf{-\Loc{}} //
	\glft `outside of the village' //
\endgl
\end{minipage}
~
\begin{forest}
where n children=0{tier=word,edge=dotted,font=\itshape}{}
[{agonan}, for tree={calign=first}
	[{agonan}, name=head]
	[{minkayya}, for tree={calign=first}
		[{minkay\textbf{ya}}, name=dependent]
	]
]
\node at (current bounding box.south) [below=0pt of head]
	{\textsc{\tiny head}};
\node at (current bounding box.south) [below=0pt of dependent] 
	{\textsc{\tiny dependent}};
\end{forest}
\xe

The relativizer, likewise, may agree in case with the NP in the matrix clause
to which it links the relative clause. This typically happens mainly in formal
language and---in terms of linear succession of words at the surface level of
the clause---if the relativizer cannot be immediately adjacent to the NP which
the relative clause modifies, for example, when an adjective or a possessive
pronoun is following the noun:

\ex
\begin{minipage}[t]{.5\remaining}%
\begingl
	\gla sangalas kivo s\textbf{as} … //
	\glb sangal-as kivo s\textbf{-as} … //
	\glc room-\Parg{} small \Rel{}\textbf{-\Parg{}} … //
	\glft `the small room which …' //
\endgl
\end{minipage}
~
\begin{forest}
where n children=0{tier=word,edge=dotted,font=\itshape}{}
[{sangalas}, for tree={calign=first}
	[{sangalas}, name=head]
	[{kivo}, for tree={calign=first}
		[{kivo}]
	]
	[{sas}, for tree={calign=first}
		[{s\textbf{as}}, name=dependent]
		[{...}, for tree={calign=first}
			[{...}]
		]
	]
]
\node at (current bounding box.south) [below=0pt of head]
	{\textsc{\tiny head}};
\node at (current bounding box.south) [below=0pt of dependent] 
	{\textsc{\tiny dependent}};
%
\coordinate [below=1em of head] (A);
\coordinate [below=1.75em of head] (B);
\coordinate [below=1.75em of dependent] (C);
\coordinate [below=1em of dependent] (D);
\draw [-latex] (A) -- (B) -- (C) -- (D);
\node (label) at ($(B)!0.5!(C)$) [below] {\tiny\itshape case agreement};
\end{forest}
\xe

The only instance of head-marking there is in Ayeri is person-marking on the
verb, which manifests when the NP following the verb (agent or patient) is not
pronominal and thus there is no pronoun to cliticize to the verb stem, but the
verb still receives a suffix that indicates a relation with, usually, the agent
NP:

\ex
\begin{minipage}[t]{.5\remaining}%
\begingl
	\gla Mal\textbf{ya} ang @ Amān. //
	\glb mal\textbf{-ya} ang= Amān //
	\glc sing\textbf{-\TsgM{}} \Aarg{}= Amān //
	\glft `Amān sings.' //
\endgl
\end{minipage}
~
\begin{forest}
where n children=0{tier=word,edge=dotted,font=\itshape}{}
[{Malya}, for tree={calign=first}
	[{Mal\textbf{ya}}, name=head]
	[{ang Amān}, for tree={calign=first}
		[{ang Amān}, name=dependent]
	]
]
\node at (current bounding box.south) [below=0pt of head]
	{\textsc{\tiny head}};
\node at (current bounding box.south) [below=0pt of dependent] 
	{\textsc{\tiny dependent}};
%
\coordinate [below=1em of dependent] (A);
\coordinate [below=1.75em of dependent] (B);
\coordinate [below=1.75em of head] (C);
\coordinate [below=1em of head] (D);
\draw [-latex] (A) -- (B) -- (C) -- (D);
\node (label) at ($(B)!0.5!(C)$) [below] {\tiny\itshape person agreement};
\end{forest}
\xe

Sentences containing more than one NP also have topic marking on the verb, so
that morphologically the verb may be analyzed as agreeing with one of the NPs
in topicality, since topic and case are no categories the verb normally
inflects for. Syntactically, however, the topicalized NP depends on the verb,
so the relationship is mutual, though on different levels---morphology and
syntax.

\ex[glspace=0.4em]
\begin{minipage}[t]{.5\remaining}%
\begingl
	\gla \textbf{Sa} @ manya ang @ Ajān {} @ \textbf{Pila}. //
	\glb \textbf{Sa=} man-ya ang= Amān \textbf{Ø=} \textbf{Pila} //
	\glc \textbf{\PatT{}=} greet-\TsgM{} \Aarg{}= Ajān \textbf{\Top{}=} %
		\textbf{Pila} //
	\glft `Pila, Ajān greets her.' //
\endgl
\end{minipage}
~
\begin{forest}
where n children=0{tier=word,edge=dotted,font=\itshape}{}
[{Sa manya}, for tree={calign=first}
	[{\textbf{Sa} manya}, name=head]
	[{ang Ajān}, for tree={calign=first}
		[{ang Ajān}]
	]
	[{Pila}, for tree={calign=first}
		[{\textbf{Ø Pila}}, name=dependent]
	]
]
\node at (current bounding box.south) [below=0em of head]
	{\textsc{\tiny head}};
\node at (current bounding box.south) [below=0em of dependent] 
	{\textsc{\tiny dependent}};
%
\coordinate [below=1em of dependent] (A);
\coordinate [below=1.75em of dependent] (B);
\coordinate [below=1.75em of head] (C);
\coordinate [below=1em of head] (D);
\draw [-latex] (A) -- (B) -- (C) -- (D);
\node (label) at ($(B)!0.5!(C)$) [below] {\tiny\itshape topic agreement};
\end{forest}
\xe

In this example, the verb exhibits canonic agreement with the agent, 
\rayr{AgYaanF}{Ajān}, in person, gender, and number. It is additionally marked 
for a patient topic, \rayr{pil}{Pila}, and thus serves as an agreement target
for two different controller NPs. As far as morphology is concerned, topic
marking on the verb is an instance of head marking.

\index{typology!of morphemes|)}
