\setcounter{chapter}{-1}
\chapter{Introduction}
\label{ch:introduction}

\begin{flushright}\smaller
\ayr{\larger prony AdnYaaNF si miNF thnojyaaNF, EdreNF voj kotnsF.}\\
\fw{Paronaya adanyāng si ming tahanoyyāng, edareng voy kotanas.}\\
`He who cannot write believes it not to be toil.'\\
--- Anonymous\footnotemark
\end{flushright}\bigskip

\footnotetext{In the original Latin, \textit{Quia qui nescit scribere putat
hoc esse nullum laborem}. Scribe's note in Berlin, State Library, Cod.\ Lat.\
fol.\ 270 \parencite[see][589]{bluhme1858}.}

\noindent In December 2003, the idea for a new fictional language was born, an
idea that turned out to stick with me for over 10 years now.\footnote{Most of
the text here is taken from the blog article
``\citetitle{benung:happybirthday}'' \parencite{benung:happybirthday} with some
slight rephrasings and extensions. When I am calling Ayeri `fictional' here, I
am referring to the fact that Ayeri is the result of creative action and does
not lead an existence independent of the confines of my writing about it or
composing texts in it, as an author. Like a story, thus, it finds its
realization only in the act of telling or writing. Like a story as well, it
takes its premises from reality as a basis for imagining what could be.
Invented languages are often referred to as `constructed languages', or
`conlangs' for short, in the hobbyist communities revolving around them.} At
that time, my seventeen years old self was still fairly new to this whole
making-up languages business, read things about linguistics here and there, and
was not shy to ask questions about terminology (and, looking at old mails, a
little impertinently teenager-like so), for example on the
\citetitle*{conlangl} and the old \citetitle*{zbb}. One thing seemed to catch
my interest especially: syntactic alignments other than the
nominative--accusative scheme of the few languages I was familiar with, that
is, German, English, and French. Apparently, this curiosity was big enough for
me to grow bored with my second language invention, Daléian (declared `quite
complete' after maybe half a year of work or so), and to start something new
from scratch in order to put newly acquired knowledge to test.

I had read about ``trigger languages'' on \citetitle{conlangl} and wanted to
try my hands on making my own. I cannot remember how long it took me to come up
with a first draft of an Ayeri grammar, however, I do remember having been told
that a good language cannot be made in a summer. Of course, I still did not
really know what I was doing then, even though I thought I had understood
things and authoritatively declared ``this is how it works'' in my first
grammar draft when things sometimes really do not work that way. But at least
an interest had been whetted.

In order to illustrate the various stages from the beginnings to current Ayeri,
I went through some old backups contemporary with the very early days. 
Here is a sentence from the oldest existing document related to it, titled 
``Draft of \& Ideas for my 3rd Conlang''---the file's last-changed date is 
\DTMdate{2003-12-14}, though I remember having started work on Ayeri in early 
December. I added glossing for convenience and according to what I could 
reconstruct from the notes. This uses vocabulary and grammatical markers just 
made up on the spot and for illustrative purposes; little of it actually 
managed to make it into actual work on Ayeri:

\ex\begingl
	\gla Ayevhoi agiaemaesim coyaielieðamavir vhaieloyaŋaiye. //
	\glb ay-evhoi agia-ema-esim coyai-el-i-eðam-avir vhai-el-o-yaŋa-iye //
	\glc \Tsg{}.\An{}-\Subj{} read-\textsc{verb}-\Subj{}.\An{} 
		book-\textsc{noun}-\An{}-\Indf{}-\Parg{} 
		bed-\textsc{noun}-\Inan{}-on-\Loc{} //
	\glft `He reads a book on the bed.' //
\endgl\xe

According to the grammar draft of \DTMdate{2004-09-05}, this would have already
changed to:

\ex\begingl
	\gla Ang @ layaiyạin mecoyalei ling *pinamea. //
	\glb ang= laya-iy-a-in me-coya-lei ling *pinam-ea //
	\glc \Aarg{}.\Subj{}= read-\Tsg{}.\An{}₁-a₁-\Subj{} 
		\Indf{}.\Inan{}-book-\PargI{} top.of bed-\Loc{} //
	\glft `He reads a book on the bed.' //
\endgl\xe

A word for `bed'---\rayr{pinmF}{pinam}---was only (re-)introduced on
\DTMdate{2008-10-24}, however. In the current state of Ayeri, I would 
translate the sentence as follows:

\ex\begingl
	\gla Ang @ layaya koyaley ling pinamya. //
	\glb ang= laya=ya.Ø koya-ley ling pinam-ya //
	\glc \AgtT{}= read=\Tsg{}.\M{}.\Top{} book-\PargI{} top.of bed-\Loc{} //
	\glft `He reads a book on a/the bed.' //
\endgl\xe

As you can see, quite a bit of morphology got lost already early on, especially
the overt part-of-speech marking (!) and animacy marking on nouns. Also,
prepositions were just incorporated into a noun complex as suffixes apparently.
Gender was originally only divided into animate and inanimate, but I changed
that at some point because only being familiar with a few European languages,
it felt awkward to me at some point not to be able to explicitly distinguish
`he', `she', and `it'. These days, I would find it potentially more interesting
if I had not taken this step, but the double split in grammatical gender is
codified now.

Another feature that was lost is the assignment of thematic vowels in personal
pronouns to third-person referents: originally, every third-person referent
newly introduced would be assigned one of /a e i o u/ to disambiguate, and
there was even a morpheme to mark the dissolution of this association.
Constituent order was theoretically variable at first, but I preferred SVO due
to familiarity with that. Later on, however, I settled on VSO. Also, I had no
idea about what was called ``trigger morphology'' on \citetitle{conlangl} for
the longest time---essentially, this referred to the Austronesian, or
Philippine, alignment. Orthography changed as well over the years, so \orth{c}
in the early examples encodes the /k/ sound, not /ʧ/ as it does today;
diphthongs were spelled \orth{Vi} instead of modern \orth{Vy}.

What was definitely beneficial for the development of Ayeri was the ever 
increasing amount of linguistics materials available online and my entering 
university (to study literature) in 2009, where I learned how to do research 
and also had a lot of interesting books available at the library.

One of the things people regularly compliment me on is Ayeri's script---note,
however, that Tahano Hikamu\index{Tahano Hikamu} was not the first one I came
up with for Ayeri. Apparently, I had already been fascinated with the look of
Javanese/Balinese writing early on;\footnote{Compare, for instance, the charts
in \citet{kuipersmcdermott1996}. The Wikipedia articles on either script
contain a number of images depicting the scripts in use, both current and
historic.} \autoref{fig:ayeriscript2004} shows a draft dated
\DTMdate{2004-02-09}. However, the letter shapes in this draft looked so
confusingly alike that I could never memorize them. About a year later, I came
up with the draft in \autoref{fig:th2005}. What is titled ``Another
Experimental Script'' there is what would later turn into Tahano Hikamu,
Ayeri's `native' script. According to the notes in my binder, the script
looked much the same as today about a year from then, but things have only been
mostly stable since about 2008.

\begin{figure}[tp]
	\centering
	\includegraphics[width=.8\textwidth,keepaspectratio]%
		{images/ayeriscript2004-300dpi-bw.png}
	\caption[First design for an Ayeri script]{First design for an Ayeri 
		script (\DTMdate{2004-02-09})}
	\label{fig:ayeriscript2004}
\end{figure}

\begin{figure}[tp]
	\centering
	\includegraphics[width=\textwidth, keepaspectratio]%
		{images/th2005-300dpi-bw.png}
	\caption[First draft for Tahano Hikamu]{First draft for Tahano Hikamu 
		(\DTMdate{2005-03-23})}
	\label{fig:th2005}
\end{figure}

An important date in the history of Ayeri was when I decided to set up an
improved website for Ayeri that would include a blog. The idea was that this
way, I could more freely write on whatever detail currently interested me in
Ayeri, outside of the constraints of the \tit{Grammar}. Thus, \tit{Benung.~The
Ayeri Language Resource} launched on \DTMdate{2011-03-01}. Being able to write
short articles, however, probably also led to neglecting work on the actual
formal reference grammar, which had been lying dormant from January 2011 on.
This was always on the premise that I would eventually include the information
from blog articles in the grammar. However, juggling a complex, forty-page
document in a word processor had felt too daunting, so I let laziness take the
better part of me eventually as enthusiasm gradually subsided. 
% \footnote{Let me add to my defense, however, that I also worked on my B.\,A.
% thesis in 2013 and my M.\,A. thesis in 2016, which required several months of
% preparation each and thus left me largely unable to work much on Ayeri.}
The present, renewed attempt at documentation has been started with the
intention to right those wrongs.

\index{typology|(}
I hope that by now it should be clear which kind of language Ayeri is: a
personal, artistic language---or \emph{artlang} in community parlance. Thus, my
goal in creating Ayeri is not to propose yet another international auxiliary
language, like Esperanto. It is also not my goal to make it as logical as
possible, like Lojban. Neither is it my goal to engineer it towards certain
underlying premises, for example, maximal information density, like Ithkuil, or
to get by with as few different words as possible, like Toki Pona. Ayeri is
also not a `what-if' language in the sense of \textquote{What could the modern
language of Old Irish speakers transplanted to Australia look like?}\ or
\textquote[][.]{Latin piped through Algonquian sound changes}

Ayeri is rather an attempt to design an artificial language for personal
enjoyment and intellectual stimulation by creating a feedback loop between
reading up on linguistics and actively devising rules for an invented language
accordingly, to see how things work within the frame I created, or to try and
see whether certain ideas work together at all when combined, and to better
understand why they do or do not. Ayeri will only ever be as perfect as
miniature models of things can be, since it has not grown organically from
millenia of human interaction, and I can and will never know about each and
every aspect of language myself, in spite of continued curiosity about these
matters. Nor will it be possible for me to replicate all the fascinating twists
and irregularities that natural languages normally entail. The ultimate
goal in my work on Ayeri is, I suppose, to make it emulate natural
languages to at least some degree of depth and complexity.
\index{typology|)}

In writing this grammar, I hope that I found a good balance between applying
linguistic theory to already existing materials and ideas, and going forth to
create rules for aspects of the language that have so far been neglected, often
due to my not being aware of them. In my opinion, the split between being able
to apply methods of linguistics to what has grown over the course of more than
a decade on the one hand, and discovering and developing new aspects of the
language on the other, is what makes Ayeri an interesting piece of
\textquote[][,]{informed nonsense} as a dear colleague of mine (Servus, Oli!)\
once put it.

\index{desiderata|(}
I have tried to document here all things which I have already worked out for
Ayeri, and to fill in the most important gaps otherwise. However, there are
still some topics which have so far been left out of consideration. Most
notably, this is the in-world historical and social context of Ayeri: no real
language exists in a cultural vacuum, however, the ideas about Ayeri's cultural
embedding are too vague still to be gainfully documented. From this arises a
lack of existing work on historical and areal linguistics (dialects) as well as
on sociolinguistics (language contact, stratification). Moreover, since I am
more interested in morphosyntax than lexicography, there are no detailed
surveys so far on Ayeri's lexicon, for instance, regarding the structuring of
its vocabulary, stylistic levels, or onomatopoeia.
\index{desiderata|)}

If in the following text my (non-native) English is not always fully idiomatic,
you find that I got facts, theories or analyses wrong, or not all aspects of
the language or its description are equally thoroughly worked-out---which are
all very likely events---I ask you to remember that this work is a one-person
effort, so mistakes and errors are unavoidable. You are kindly invited to share
any constructive criticism you have with me, however, to correct or improve any
issues that might need correction or elaboration. Thanks in this regard to
Joseph Windsor and Greg Shuflin for valuable input on language and style.

This book is structured in a way to go from the building blocks of the language
to increasingly larger structures. Thus, \autoref{ch:phonology} deals with
aspects of Ayeri's phonology, and \autoref{ch:writing} with its alphabet.
Chapter \ref{ch:morphtyp} contains a discussion of the various morphological
means in a general, typological way, while the subsequent \autoref{ch:gramcat}
discusses the morphology of the individual parts of speech. Before dealing with
phrase types and their morphosyntactic makeup, \autoref{ch:syntyp} provides a
survey into Ayeri's syntactic alignment and tries to answer the question
whether Ayeri is a `configurational' language in spite of VSO word order, and
tries to find an answer to the `trigger-language' issue. Chapter
\ref{ch:phrasestruct} finally discusses how syntactic structures are built up
from words, eventually leading to the formation of complete sentences.
