\chapter{Writing system}
\label{ch:writing}
\index{Tahano Hikamu|(}

In the previous chapter, example words were given in Ayeri's script, \rayr{thno
hikmu}{Tahano Hikamu}, wherever possible. Thus, it seems advisable to include a
description of Ayeri's native writing system here as well. Literally,
\rayr{thno hikmu}{Tahano Hikamu} means `Round Script' (script round), which is
an old formation based on the word \xayr{thnF/}{tahan-}{write} that stuck. The
current word for `script' is \xayr{thnnF}{tahanan}{writing}.\footnote{Tahano
Hikamu was originally named thus because of an earlier draft for a
script\index{Box script} that never made it very far beyond the drawing board
and which was a lot more angular, see \autoref{fig:boxyhikamu}---Tahano
Hikamu was a lot more bubbly in comparison, especially early on
(\autoref{fig:th2005}). Unfortunately, there is no documentation of the Box
script\index{Box script} surviving that I know of.}

\begin{figure}[tp]\centering
\begin{minipage}{.475\linewidth}\centering
\includegraphics[width=\linewidth]{images/hinya-300dpi-clip.png}
\subcaption{Singularly attested: Box script}
\label{fig:boxy}
\end{minipage}
~
\begin{minipage}{.475\linewidth}\centering
\includegraphics[width=\linewidth]{images/tahano-300dpi-bw-clip.png}
\subcaption{Ayeri's native script: Tahano Hikamu}
\end{minipage}
\caption{Box script (undeciphered) and Tahano Hikamu}
\label{fig:boxyhikamu}
\end{figure}

As we have seen in the previous chapter, Ayeri's prosody strongly emphasizes
the syllable as a unit. Thus, it is not a surprise that Ayeri's native script,
Tahano Hikamu, is an alphasyllabary similar to the Brāhmī\index{Brāhmī scripts}
alphabets of India and Southeast Asia \parencites{salomon1996}{court1996}.
Scripts like these are

\blockcquote[376]{salomon1996}{based on the unit of the graphic 
\enquote{syllalbe} [\dots], which by definition always ends with a vowel (type V, 
CV, CCV, etc.). Syllables consisting of a vowel only (usually at the beginning 
of a word or sentence) are written with the \emph{full} or \emph{initial vowel 
signs} [\dots]. But when, as is much more frequently the case, the syllable 
consists of a consonant followed by a vowel, the vowel is indicated by a 
diacritic sign attached to the basic sign for the consonant}

For Tahano Hikamu the definition that a syllable consisting only of a vowel is
written with an initial vowel sign is only true under certain circumstances, as
we will see below. Moreover, Brāhmī\index{Brāhmī scripts} scripts are often characterized by
conjuncts of clustered consonants which may become quite large and sometimes
behave in an idiosyncratic way. Consonant conjuncts like Devanāgarī\index{Devanāgarī} {\FS
त्व}~\orth{tva} from {\FS त}~\orth{ta} + {\FS व}~\orth{va} or idiosyncratic
conjuncts like {\FS क्ष} \orth{kṣa} for {\FS क} \orth{ka} + {\FS ष} \orth{ṣa}
are not known in Tahano Hikamu, however, at least as far as Ayeri's spelling is
concerned. Subscript notation for consonant clusters and special diacritics
marking coda consonants like in Javanese\index{Javanese script} \citep[478--479]{kuipersmcdermott1996}
are likewise unknown to Tahano Hikamu. This does not mean, however, that final
consonants are simply omitted in writing, since closed syllalbes are reasonably
common enough in Ayeri to warrant indicating them. Thus, there is
\textcquote[476]{kuipersmcdermott1996}{a special mark to eliminate the vowel of
the previous syllable, thereby leaving a consonant in a syllable-final
position}. That is, a diacritic exists which marks the absence of an inherent
vowel, rendering the syllable consonant-only.

Another difference from Brāhmī-family\index{Brāhmī scripts} scripts is that vowel length and
diphthongs in [ɪ] are indicated by dedicated diacritics, so the long vowels are
not doubled versions of their short counterparts. Like in Kharoṣṭhī\index{Kharoṣṭhī}---another
historically important ancient script of India---initial vowels are not
represented by unique graphemes, but they are all written like post-consonantal
vowel diacritics \citep[377]{salomon1996}. In Tahano Hikamu, a character
without an inherent sound value serves as the base. For this reason, the
character is indicated in the table below as \ayr{ʔ} /Ø/; its native name is
\xayr{rnYn}{ranyan}{nothing}.\footnote{I will give the native names of
graphemes here, but will refer to them by their English names for clarity in
the running text.} Similar to a number of Brāhmī\index{Brāhmī scripts} scripts, Tahano Hikamu puts
diacritics not only below or above consonant bases, but also before them. This,
however, is not limited to vowel graphemes as in Devanāgarī\index{Devanāgarī} {\FS ि}~\orth{i} or
Javanese\index{Javanese script} \smash{\fontspec{Tuladha Jejeg}[Script=Javanese,
Scale=MatchLowercase]ꦺ}~\orth{e, é/è} \citep[478]{kuipersmcdermott1996}.
% \footnote{\citet{kuipersmcdermott1996} do not say, but it seems that both
% might be related, since they are both functionally the only prepended vowel
% diacritics and both represent a high front sound; this is just a guess,
% however.}

\section{Consonants}
\index{consonants|(}

Tahano Hikamu is mainly based on consonant bases that are modified by
diacritics. Since the vowel /a/ is so highly frequent in Ayeri, it is also the
vowel that is \fw{inherent} to every consonant grapheme if not further modified
by vowel diacritics. Consonant letters are simply referred to as \fw{pa},
\fw{ta}, \fw{ka}, etc. \autoref{tab:thcons} displays all the main consonants.
The customary collation is---similar to the IPA table---roughly grouping the
letters according to their sound value by anteriority (front → back) and
sonority (low → high). The script is monocameral, that is, there is no
distinction between capital letters and minuscule letters as in the Latin,
Greek, Cyrillic, Georgian, and Armenian alphabets. It is also written in lines
from left to right.

% \begin{center}\itshape
% 	pa, ta, ka;\\
% 	ba, da, ga;\\
% 	ma, na, nga;\\
% 	va, sa, ha;\\
% 	ra, la, ya;\\
% 	Ø.\\
% \end{center}

\begin{table}[t]
\caption{The consonant graphemes}

\begin{tabu} to \linewidth{X[c] X[c] X[c] X[c] X[c] X[c]}
\toprule
\tableheaderfont	/pa/ & /ta/ & /ka/ & /ba/ & /da/ & /ga/ \\
\rowfont{\Tagati\huge}	p & t & k & b & d & g \\

\midrule

\tableheaderfont	/ma/ & /na/ & /ŋa/ & /va/ & /sa/ & /ha/ \\
\rowfont{\Tagati\huge}	m & n & N & v & s & h \\

\midrule

\tableheaderfont	/ra/ & /la/ & /ja/ & /Ø/ \\
\rowfont{\Tagati\huge}	r & l & y & ʔ \\

\bottomrule
\end{tabu}
\label{tab:thcons}
\end{table}

\ayr{ʔ}, which in Ayeri has no sound value but is used as a base for initial
vowels, may also serve as the character for /ʔa/. What is, moreover,
interesting\index{syllabification} about \ayr{N}~\orth{nga} is that even though before, /ŋ/ was
treated strictly as a coda consonant in the previous chapter, it is in fact
treated as an onset consonant in writing if a vowel is following:

\ex[lingstyle=thex]\begingl
	\gla \ayr{p}	$+$	\ayr{NisF} //
	\glb /pa/	{}	/ŋis/ //
	\glft \rayr{\larger pNisF}{pangis} /paŋ.is/ `money' //
\endgl\xe

Tahano Hikamu contains a few ligatures. First of all, when two \ayr{n}
\orth{na} are in succession within a word, they will form a ligature \ayr{nn}
\orth{nana}:

\ex[lingstyle=thex]\begingl
	\gla \ayr{n}	$+$	\ayr{n}	→	\ayr{nn} //
	\glb /na/	{}	/na/	{}	/nana/ //
\endgl\xe

This is distinct from conjuncts like in Devanāgarī\index{Devanāgarī} et al., though, since the
unmodified sound value will still be /nana/, not */nna/, so the inherent vowel
of each \ayr{n} \orth{na} is not deleted, and each \ayr{n} \orth{na} retains
the ability to be modified by diacritics. Tahano Hikamu also has a few
ligatures of the kind you would find in Brāhmī\index{Brāhmī scripts} scripts. The difference is that
they are not productive, but fossilized.

\pex
	\a \ayr{\larger q}~\orth{kwa} ← \ayr{\larger k}~\orth{ka} + 
		\ayr{\larger v}~\orth{va}
	\a \ayr{\larger T}~\orth{tsa} ← \ayr{\larger t}~\orth{ta} + 
		\ayr{\larger s}~\orth{sa}
	\a \ayr{\larger x}~\orth{ksa} ← \ayr{\larger k}~\orth{ka} + 
		\ayr{\larger s}~\orth{sa}
\xe

These conjunct letters are, however, not normally employed by Ayeri.
\autoref{tab:thconsadd} shows all additional consonants, added to write other
languages. Individual languages may adapt the sound values slightly to fit
their own purposes.

\begin{table}[t]
\caption{Additional consonant graphemes}

\begin{tabu} to \linewidth{X[c] X[c] X[c] X[c] X[c] X[c]}
\toprule
\tableheaderfont	/fa/ & /wa/ & /tsa/ & /za/ & /ʃa/ & /ʒa/ \\
\rowfont{\Tagati\huge}	f & w & T & z & S & Z \\

\midrule

\tableheaderfont	/ça/ & /ksa/ & /kwa/ & /xa/ & /ɣa/ \\
\rowfont{\Tagati\huge}	C & x & q & X & G \\

\bottomrule
\end{tabu}
\label{tab:thconsadd}
\end{table}

\index{consonants|)}

\section{Vowels}
\index{vowels|(}

As mentioned above, vowels are written as diacritics that are added to 
consonants. In principle, every consonant has two slots for vowels, a primary 
one atop it, and a secondary one below it. Vowels added to consonants in 
the primary slot delete their inherent /a/:

\ex[lingstyle=thex]\begingl
	\gla \ayr{p}	→	\ayr{pe} //
	\glb /pa/	{}	/pe/ //
\endgl\xe

\begin{table}
\caption{Primary vowel graphemes}

\begin{tabu} to \linewidth{>{\bfseries}l X[c] X[c] X[c] X[c] X[c] X[c] X[c]}
\toprule
\tableheaderfont

	& /i/
	& /e/
	& /a/
	& /o/
	& /u/
	& /ə/
	& /aʊ/
	\\
	
\toprule
	
Diaritics
	& \Tagati\huge *i
	& \Tagati\huge *e
	& \huge ({\Tagati *a})
	& \Tagati\huge *o
	& \Tagati\huge *u
	& \Tagati\huge *ə
	& \Tagati\huge *au
	\\

\midrule

Independent
	& \Tagati\huge I
	& \Tagati\huge E
	& \Tagati\huge A
	& \Tagati\huge O
	& \Tagati\huge U
	& \Tagati\huge Ə
	& \Tagati\huge AU
	\\

\bottomrule
\end{tabu}
\label{tab:thvowstop}
\end{table}

\autoref{tab:thvowstop} gives the primary vowel signs. Of the vowel signs given
there, only \ayr{*ə}~\orth{ə} is not used in Ayeri. \ayr{*au}~\orth{au} is the
only diphthong for which a dedicated grapheme exists, even though its
occurrence is rather limited. The independent vowel graphemes are used at the
beginning of words or inside words when there is no other way to spell the
vowel, which is occasionally the case for secondary vowels. Secondary vowels
are vowels that are not parts of diphthongs (even though another language might
use them to spell diphthongs that are not covered by default), but follow the
vowel of a syllable directly. They are attached underneath a consonant base,
for example:

\ex[lingstyle=thex]\begingl
	\gla \ayr{y}	→	\ayr{ye}	→	\ayr{ye\_a} //
	\glb /ja/	{}	/je/		{}	/jea/ //
\endgl\xe

In fact, the principle that every consonant base with its diacritics represents
one syllable is slightly violated here, which is also the reason why secondary 
vowels very occasionally need to be spelled as independent vowels, for example 
when the secondary vowel is long, as in the word \xayr{ruAAnF}{ruān}{duty}:

\ex[lingstyle=thex]\label{ex:rwaa}\begingl
	\gla \ayr{ru}	→	\ayr{ruAA}	\quad	(\,\ayr{ruu\_a}) //
	\glb /ru/	{}	/rwaː/ 		\quad	\excl{}/ruːa/ //
\endgl\xe

Example (\ref{ex:rwaa}) uses a diacritic, \ayr{*aa}, to indicate length. If 
\ayr{*aa} is put directly under \rayr{ru}{ru} (the \ayr{*\_a} diacritic moves
down where it is not in the way), the syllable will incorrectly spell /ruːa/
instead of the intended /ruaː/. This is because diacritics modify consonants
and primary vowels, but there is no way to modify a secondary vowel directly.
\autoref{tab:thvowsbot} gives a list of secondary vowels corresponding to that
of primary vowels above. The vowels as well are just referred to by their sound
value; `primary' and `secondary', `superscript' and `subscript' or `upper' and
`lower' may be chosen to disambiguate\index{ambiguity} their positions; the native names may use
\xayr{Iraj}{iray}{high} and \xayr{Ejr}{eyra}{low} to disambiguate\index{ambiguity}, so \rayr{E 
Irj}{e iray} denotes the superscript \orth{e} diacritic while \rayr{E Ejr}{e 
eyra} denotes its subscript counterpart.

\begin{table}
\caption{Secondary vowel graphemes}

\begin{tabu} to \linewidth{X[c] X[c] X[c] X[c] X[c] X[c] X[c]}
\toprule
\tableheaderfont	/i/ & /e/ & /a/ & /o/ & /u/ & /ə/ & /aʊ/ \\
\rowfont{\Tagati\huge}	*\_i & *\_e & *\_a & *\_o & *\_u & *\_ə & *\_au \\

\bottomrule
\end{tabu}
\label{tab:thvowsbot}
\end{table}

As a further exception, those consonant bases with an ascender 
(\ayr{k}~\orth{ka}, \ayr{d}~\orth{da}, \ayr{C}~/ça/) move the primary vowel to 
the secondary slot below the consonant by default while indicating the vacancy 
of the primary slot at the top with a dot. This is done to avoid crossing the 
ascender of the consonant with a vowel diacritic:

\ex[lingstyle=thex]\begingl
	\gla \ayr{k}	→	\ayr{k\_i}	→	\ayr{ki} //
	\glb /ka/	{}	/ka.i/		{}	/ki/ //
\endgl\xe

If the primary vowel slot were not silenced by the \ayr{*\_F} diacritic, it 
could reasonably be assumed that the consonant is not losing its inherent /a/ 
and the vowel below the consonant indicates a secondary vowel, spelling /CaV/. 
If, however, a secondary vowel is \emph{actually} added, primary and secondary 
vowels will be assigned the regular primary and secondary slots, respectively, 
again (\ref{ex:kie}). This condition also holds true for subscript diacritics 
(\ref{ex:kii}).

\pex[lingstyle=thex]
\a\label{ex:kie}\begingl
	\gla \ayr{ki}	→	\ayr{ki\_e} //
	\glb /ki/	{}	/ki.e/ //
\endgl

\a\label{ex:kii}\begingl
	\gla \ayr{ki}	→	\ayr{kii} //
	\glb /ki/	{}	/kiː/ //
\endgl

\xe

The order of secondary vowels and subscript diacritics is iconic insofar as 
it follows the order of sounds in the syllable. Thus, secondary vowels appear 
below the consonant-doubling diacritic, \ayr{*F*}, while they appear above the 
syllable-final homorganic nasal diacritic, \ayr{*\_M}:

\pex[lingstyle=thex]\label{ex:subscrord}
\a\begingl
	\gla \ayr{pFp}	→	\ayr{pFpe\_a} //
	\glb /ppa/	→	/ppea/ //
\endgl

\a\begingl
	\gla \ayr{peM}	→	\ayr{pe\_aM} //
	\glb /peN/	→	/peaN/ //
\endgl
\xe

\index{vowels|)}

\section{Diacritics}
\index{diacritics|(}

We have already encountered a few diacritics, though Tahano Hikamu comes with a
lot more. Some of these diacrtics even undergo non-trivial positioning and
repositioning. As vowels are primarily expressed as superscripts, diacritics
are primarily realized as subscripts, so in the following, I will first
describe subscript diacritics; then prepended diacritics, which Ayeri also has
a number of, both as graphemes in their own right and as allographs of other
subscript diacritics; and lastly, superscript diacritics.

\subsection{Subscript diacritics}

%\begin{sidewaysfigure}[p]
\afterpage{%
\clearpage%
\begin{landscape}\centering
\begin{table}[p]
\caption{Subscript diacritics}
\begin{tabu} to \linewidth {>{\Tagati\huge}X[1] X[8l] X[18l] X[10l]}
\toprule
\tableheaderfont

	& Native name
	& Function
	& Example
	\\
	
\toprule

% \tablesubheaderfont\multicolumn{4}{c}{L~a~r~g~e~{ }~d~i~a~c~r~i~t~i~c~s}\\

% \midrule

*aa
	& \xayr{tupsti}{tupasati}{long-maker}
	& Lengthens the primary vowel of the syllable
	& \rayr{p}{pa} → \rayr{paa}{pā}
	\\

\midrule
	
*Y
	& \xayr{y Ejr}{ya eyra}{low ya}
	& \orth{ya} following another consonant, also across syllables. Marks 
		palatalization of \ayr{t}~\orth{ta}, \ayr{d}~\orth{da}, 
		\ayr{k}~\orth{ka}, \ayr{g}~\orth{ga} and \ayr{y}~\orth{ya} in 
		Ayeri.
	& \rayr{Ar}{ara} → \rayr{ArY}{arya}; \rayr{t}{ta} → \rayr{tY}{ca}
	\\
	
\midrule
	
*J
	& \xayr{riNy}{ringaya}{raiser}
	& Palatalizes a consonant (not used in Ayeri)
	& \rayr{t}{ta} → \ayr{tJ} /tʲa/, /ʧa/
	\\
	
\midrule
	
*H
	& \xayr{UlNy}{ulangaya}{breather}
	& Aspiration or frication of a consonant (not used in Ayeri)
	& \rayr{t}{ta} → \ayr{tH} /tʰa/, {\addfontfeature{RawFeature=+mgrk}/θa/}
	\\
	
\midrule
	
*\hspace{-.25em}ˀ
	& \xayr{rjpaay Ejr}{raypāya eyra}{low~stopper}
	& Glottal stop coda or glottalization of a consonant (consonant letters 
		with ascenders; not used in Ayeri)
	& \rayr{k}{ka} → \ayr{kQ} /kaʔ/; \rayr{d}{da} → \ayr{dQ} /d’a/
	\\

\midrule

% \tablesubheaderfont\multicolumn4{c}{S~m~a~l~l~{ }~d~i~a~c~r~i~t~i~c~s}\\

% \midrule

*F
	& \xayr{goMdy}{gondaya}{extinguisher}
	& Deletes the inherent /a/ of a consonant, e.g. in consonant clusters 
		or closed syllables
	& \rayr{pr}{para} → \rayr{pFr}{pra}, \rayr{prF}{par}
	\\
	
\midrule
	
*M
	& \xayr{vinaati}{vināti}{nasalizer}
	& Indicates a homorganic nasal or nasalizes the vowel, depending on the 
		language
	& \rayr{pd}{pada} → \rayr{pMd}{panda} /panda/ or /pãda/
	\\
	
\midrule
	
*F*
	& \xayr{kusNisaati}{kusangisāti}{duplicator}
	& Indicates a geminated or otherwise double consonant
	& \rayr{pl}{pala} → \rayr{plFl}{palla}
	\\

\bottomrule
\end{tabu}
\label{tab:thdiabot}
\end{table}
\end{landscape}
\clearpage%
}%
%\end{sidewaysfigure}

\autoref{tab:thdiabot} shows the bottom-attaching diacritics. The `large 
diacritics' (\ayr{*aa} through \ayr{*\hspace{-.25em}ˀ\hspace{.5em}}) cause the
secondary slot of consonants to move down below the diacritic. `Small
diacritics' (\ayr{*F} through \ayr{*F*}) can attach in this place as well as
secondary vowels, as does the homorganic nasal diacritic \ayr{*M} in this
diacritic-fraught example:

\ex[lingstyle=thex]\label{ex:caampuluy}\begingl
	\gla \ayr{tYaanF} $+$ \ayr{puluj} → \ayr{tYaaMpuluj} //
	\glb {/ˈʧaːn/} {} {/puˈlʊɪ/} {} {/ˌʧaːmpuˈlʊɪ/} //
% 	\glc \xayr{tYaanF}{cān}{love} {} \xayr{puluj}{puluy}{opposite} {}
% 		\xayr{tYaaMpuluj}{cāmpuluy}{heterosexual} //
	\glft \xayr{\larger tYaaMpuluj}{cāmpuluy}{heterosexual} //
\endgl\xe

It also needs to be noted that diacritics like \ayr{*Y} are applied 
progressively to words as a whole, not stopping at morpheme and syllable 
boundaries, so even though \tayr{toryeng}{she sleeps} may be composed of 
\xayr{torF/}{tor-}{sleep} + \rayr{/yeNF}{-yeng} (=\TsgF{}.\Aarg{}) and 
syllabifies\index{syllabification} as /tor.ˈjeŋ/, the spelling is not *\ayr{torF\zwsp{}yeNF} as one 
might expect, but \ayr{torYeNF}.

Even though the primary position for small diacritics is underneath consonants,
the diacritic deleting the inherent vowel, \ayr{*F}, very commonly also appears
after a consonant letter at the end of words:

\ex[everygla=\Tagati\Large,everyglb=\itshape]\begingl
	\gla y @ nimFreN\thafterdot{} pNn\thafterdot{} 
		nraanFyen. //
	\glb Ya nimreng pangan narānyena. //
	\glc ya= nim-reng pangan-Ø narān-ye-na //
	\glc \LocT{}= appear=\TsgI{}.\Aarg{} end-\Top{} word-\Pl{}-\Gen{} //
	\glft `It appears at the end of words.' //
\endgl\xe

This strategy is advantageous in that Tahano Hikamu leaves very little space
between individual words: \ayr{y nimFreN\thafterdot{} pNn\thafterdot{}
nraanFyen.} With the dot after the final consonant, word boundaries are more
visible.

\subsection{Prepended diacritics}

Example (\ref{ex:caampuluy}) leads us directly to the next class of 
diacritics---those that are prepended to the consonant letter, either because 
they are simply placed there or because of allography. Let us first list those 
diacritics that appear in front of consonants obligatorily 
(\autoref{tab:thdiapreobl}).

\begin{table}[tp]
\caption{Obligatorily prepended diacritics}
\begin{tabu} to \linewidth{>{\Tagati\huge}X[1] X[8l] X[16l] X[12l]}
\toprule
\tableheaderfont

	& Native name
	& Function
	& Example
	\\
	
\toprule

*j
	& \xayr{leMtMkusNF}{lentan\-kusang}{double-\allowbreak{}sound}
	& Makes a diphthong with /ɪ/
	& \rayr{pe}{pe} → \rayr{pej}{pey}
	\\
	
\midrule

*\_:
	& \xayr{tilmy}{tilamaya}{changer}
	& Marks raised vowels (i.e. umlaut; not used in Ayeri)
	& \rayr{po}{po} → \ayr{po\_:}~/pø/
	\\
	
\midrule

*R
	& \xayr{hiymy}{hiyamaya}{roller}
	& Marks retroflex consonants (not used in Ayeri)% \footnotemark
	& \rayr{t}{ta} → \ayr{tR}~/ʈa/
	\\

\bottomrule
\end{tabu}
\label{tab:thdiapreobl}
\end{table}

% \footnotetext{In a Tahano Hikamu orthography I devised for English once,
% \ayr{*R} was used for /ɚ/, as in the \textsc{nurse} vowel in American
% English: \rayr{nRsF}{nurse}.}

\begin{table}[tp]
\caption{Allographically prepended diacritics}
\begin{tabu} to \linewidth{>{\Tagati\huge}X[1] X[8l] X[16l] X[12l]}
\toprule
\tableheaderfont

	& Native name
	& Function
	& Example
	\\
	
\toprule

ː*
	& \xayr{tupsti mrinF}{tupasati marin}{anterior long-maker}
	& Lengthens the primary vowel of the syllable
	& \rayr{sY}{sya} → \rayr{sYaa}{syā},\newline
		\rayr{n}{na} → \rayr{naa}{nā}
	\\
	
\midrule

ʲ*
	& \xayr{y mrinF}{ya marin}{anterior ya}
	& \orth{ya} following another consonant, also across syllables.
	& \rayr{n}{na} → \rayr{nY}{nya}
	\\
	
	
	& \xayr{riNy mrinF}{ringaya marin}{anterior raiser}
	& Also used as an allograph for the palatalization proper diacritic.
	& \ayr{sH}~/sʰa/ → \ayr{sHY}~/sʰʲa/
	\\
	
\midrule

ʰ*
	& \xayr{UlNy mrinF}{ulangaya marin}{anterior breather}
	& (Pre-)Aspiration or frication of a consonant (not used in Ayeri)
	& \rayr{N}{nga} → \ayr{NH} /ŋʰa/;\newline
		\rayr{t}{ta} → \ayr{ʰt}~/ʰta/
	\\

\bottomrule
\end{tabu}
\label{tab:thdiapreallo}
\end{table}

As \autoref{tab:thdiapreobl} shows, the only obligatorily prepended diacritic
that Ayeri uses is the one that marks diphthongs, \ayr{*j}.\index{diphthongs}
However, \ayr{*j} changes into \ayr{y}~\orth{ya} proper when a vowel follows,
but stays \ayr{*j} when a \ayr{y}~\orth{ya} follows:

\pex
	\a \xayr{\larger hdj}{haday}{hero} → 
		\ayr{\larger hdyNF} (*\ayr{\larger hdjANF}) \fw{hadayang}
		`the hero' (hero-\Aarg{})
	\a \xayr{\larger tipuj}{tipuy}{grass} → 
		\ayr{\larger tipujy} (*\ayr{\larger tipuyY}) \fw{tipuyya} `in 
		the grass' (grass-\Loc{})
\xe

Besides \ayr{*j}, there are also a number of diacritics that are prepended to
consonants, but as context-sensitive allographs (\autoref{tab:thdiapreallo}).
The selection of the variant diacritics is not random or up to the aesthetic
eye of the writer (even though the device itself is certainly a matter of
aesthetics), but it is governed by rules. The prepended forms listed in
\autoref{tab:thdiapreallo} are thus triggered

\begin{enumerate}
\item when there is no stem or bowl for the regular subscript diacritic to 
	attach to, which is the case for \ayr{n}~\orth{na}, \ayr{N}~\orth{nga}, 
	\ayr{v}~\orth{va}, and \ayr{w}~\orth{wa}:
	
	\begin{multicols}{2}
	\pex[lingstyle=thex,]\label{ex:stemless}
	\a\begingl
		\gla \ayr{n} → \ayr{naa} //
		\glb /na/ {} /naː/ //
	\endgl
	
	\a\begingl
		\gla \ayr{N} → \ayr{Naa} //
		\glb /ŋa/ {} /ŋaː/ //
	\endgl
	
	\a\begingl
		\gla \ayr{v} → \ayr{vaa} //
		\glb /va/ {} /vaː/ //
	\endgl
	
	\a\begingl
		\gla \ayr{w} → \ayr{waa} //
		\glb /wa/ {} /waː/ //
	\endgl
	
	\xe
	\end{multicols}

\item when a large subscript diacritic would be added after another large 
	subscript diacritic---this position can only be occupied once, so 
	further large subscripts take their prepended form:
	
	\ex[lingstyle=thex,everygla=\normalsize,everyglb=\upshape\Large,
		aboveglcskip=0.5em,numoffset=\leftmargin]\label{ex:stacking}
	\begingl
		\gla {} {$+$ \ayr{*H}} {} {$+$ \ayr{*Y}} {} {$+$ \ayr{*i}} {}
			{$+$ \ayr{*aa}} {} //
		\glb \ayr{t} → \ayr{tH} → \ayr{tHY} → \ayr{tHYi} → 
			\ayr{tHYii} //
		\glc /ta/ {} /tʰa/ {} /tʰja/ {} /tʰji/ {} /tʰjiː/ //
	\endgl\xe
	
	The order of diacritics follows the logic of the respective 
	language's phoneme inventory, so if there are, for example, 
	retroflex consonants and both dental and retroflex consonants can be 
	aspirated, retroflexion would be marked first, then aspiration. If 
	there is a palatalization contrast on top of this, the diacritic would 
	be added after aspiration.
	
	When adding large diacritics to stemless consonants, they are prepended 
	from the beginning, as we saw in (\ref{ex:stemless}), and just like in 
	(\ref{ex:stacking}), this principle continues:
	
	\ex[lingstyle=thex,everygla=\normalsize,everyglb=\upshape\Large,
		aboveglcskip=0.5em,numoffset=\leftmargin]
	\begingl
		\gla {} {$+$ \ayr{*Y}} {} {$+$ \ayr{*aa}} {} {$+$ \ayr{*j}} 
			{} //
		\glb \ayr{n} → \ayr{nY} → \ayr{nYaa} → \ayr{nYaaj} //
		\glc /na/ {} /nja/ {} /njaː/ {} /njaːɪ/ //
	\endgl\xe

\item with consonants directly following \ayr{n}~\orth{na}, to avoid a clash 
	with its swash:
	
	\ex[lingstyle=thex,numoffset=\leftmargin]
	\begingl
		\gla \ayr{n} $+$ \ayr{paa} → \ayr{npaa} \quad
			(*\,\ayr{n\zwsp{}paa}) //
		\glb /na/ {} /paː/ {} /napaː/ {} {} //
	\endgl\xe
	
	An exception to this exception occurs, however, when the consonant is 
	not directly following. In this case, no reordering happens, only 
	\ayr{n}~\orth{na} \emph{may} reduce its swash in size to accommodate 
	the following prepended diacritic:
	% \footnote{The font used here is designed so that the reduced combination
	% looks nicer, but if unreduced, \ayr{n}~\orth{na}'s swash is not so long
	% as to cross the descender of \ayr{*j} either in this particular case.}
	
	\pex[lingstyle=thex,numoffset=\leftmargin]
	\begingl
		\gla \ayr{n} $+$ \ayr{pj} → \ayr{npj} \quad
			(\ques{}\ayr{n\zwsp{}pj}) //
		\glb /na/ {} /paɪ/ {} /napaɪ/ {} {} //
	\endgl\xe
	
\item in other cases where a clash of subscript diacritics needs to be avoided:

	\ex[lingstyle=thex,numoffset=\leftmargin]
	\begingl
		\gla \ayr{di} $+$ \ayr{paa} → \ayr{diːp} \quad 
			(*\,\ayr{dipaa}) //
		\glb /di/ {} /paː/ {} /dipaː/ {} {} //
	\endgl\xe
	
	Alternatively, the following solution is permissible:
	
	\ex[lingstyle=thex,numoffset=\leftmargin]%
	\begingl
		\gla \ayr{di} $+$ \ayr{paa} → 
		% Due to negligence when coding the Tahano Hikamu font, I did 
		% not build in a way to manually put a diacritic on top of ⟨ka⟩ 
		% and ⟨da⟩, thus I need to put it on the letter with LaTeX 
		% commands, which is very clumsy. Younger self: shame on you!
		\ayr{d\hspace{-.3em}\raisebox{1.5ex}{\zwsp i}\hspace{.3em}%
			paa} //
		\glb /di/ {} /paː/ {} /dipaː/ //
	\endgl\xe
	
	When two long syllables follow each other, as in 
	\tayr{bāmā}{mom-and-dad}, one of the length diacritics should 
	definitely be pulled to the front, as in (\ref{ex:bama}).
	
	\ex[lingstyle=thex,everyglb=\upshape\Large,aboveglcskip=0.5em,
	numoffset=\leftmargin]\label{ex:bama}
	\begingl
		\gla {} \ayr{baa} $+$ \ayr{maa} → \ayr{baaːm} \quad 
			(\ques{}\ayr{baamaa}) //
		\glb {\normalsize or:\quad} \ayr{baa} $+$ \ayr{maa} → 
			\ayr{ːbmaa} //
		\glc {} /baː/ {} /maː/ {} /baːmaː/ //
	\endgl
	\xe

\end{enumerate}

Generally, prepended diacritics apply only to a single consonant grapheme, not
a whole consonant cluster\index{consonants!clusters} as such. Thus, for instance, in words like 
\tayr{pray}{smooth} \ayr{*j} appears before \ayr{r}~\orth{ra}, not before
\ayr{p}~\orth{pa}, since \ayr{r}~\orth{ra} is the closest consonant before the
syllable nucleus which we are modifying by adding the \ayr{*j}. Since in the
case of \fw{pray} the inherent vowel of \ayr{p}~\orth{pa} is silent, it
receives a diacritic \ayr{*F} to mark this fact:

\ex[lingstyle=thex]\label{ex:clusterjsplit}
\begingl
\gla \ayr{pFrj} \quad (*\,\ayr{jpFr}) //
\glb /praɪ/ //
\endgl
\xe

What (\ref{ex:clusterjsplit}) shows is that essentially, /praɪ/ is split into
/p/ + /raɪ/ for purposes of spelling, rather than /pr/ + /aɪ/. If necessary, it
is also possible this way to distinguish, for instance, \ayr{tRs}~/ʈsa/ from
\ayr{tsR}~/tʂa/. It would be up to the respective language's orthography to
decide whether either combination spells /ʈʂa/ or whether the \ayr{*R}
diacritic is needed on both con\-so\-nants---that is, \ayr{tRsR}---to spell the
retroflex affricate.

\subsection{Superscript diacritics}

Ayeri's standard position for diacritics is below consonants, but sometimes it
is nicer to put them on top, especially for the letter \ayr{n}~\orth{na} due to
its swash, as well as for \ayr{v}~\orth{va} since the space below its flag is
empty otherwise, thus not providing much of a visual connection. The only
diacritic that is normally attaching to the top of consonants is that for the
glottal stop---we have already encountered its subscript allograph earlier.
Since Ayeri's phoneme inventory does not possess a phonemic glottal stop or
glottalization, this diacritic is not used in Ayeri. The list of superscript
diacritics is given in \autoref{tab:thdiatop}.

\begin{table}[tp]
\caption{Superscript diacritics}
\begin{tabu} to \linewidth{>{\Tagati\huge}X[1] X[8l] X[16l] X[12l]}
\toprule
\tableheaderfont

	& Native name
	& Function
	& Example
	\\
	
\toprule

*\_F
	& \xayr{goMdy liNF}{gondaya ling}{upper extinguisher}
	& Deletes inherent /a/ of consonant, e.g. in consonant clusters or 
		closed syllables
	& \rayr{vr}{vara} → \rayr{v\_Fr}{vra}
	\\
	
\midrule

*\_M
	& \xayr{vinaati liNF}{vināti ling}{upper nasalizer}
	& Indicates a homorganic nasal or nasalizes the vowel, depending on 
		language/context
	& \rayr{nd}{naka} → \rayr{nMk}{nanka} /naŋka/ or /nãka/
	\\
	
\midrule

*̔
	& \xayr{kusNisaati liNF}{kusangisāti ling}{upper duplicator}
	& Indicates a geminated or otherwise double consonant
	& \rayr{pn}{pana} → \rayr{pnFn}{panna}
	\\
	
\midrule

*Q
	& \xayr{rjpaay}{raypāya}{stopper}
	& Glottal stop coda or glottalization of a consonant (not used in Ayeri)
	& \rayr{t}{ta} → \ayr{tQ} /taʔ/;\newline
		\rayr{s}{sa} → \ayr{sQ} /s’a/
	\\

\bottomrule
\end{tabu}
\label{tab:thdiatop}
\end{table}

At times, it may be necessary to attach both a superscript diacritic and a
vowel sign above a consonant, compare (\ref{ex:subscrord2}). In this case, the
consonant-modifying diacritic is placed first and the vowel diacritic on top of
it---this is exactly equivalent to the rule exemplified for subscript
diacritics in (\ref{ex:subscrord}).

\pex[lingstyle=thex]\label{ex:subscrord2}
\a\begingl
	\gla \ayr{v̔}	→	\ayr{v̔e} //
	\glb /vva/	→	/vve/ //
\endgl

\a\begingl
	\gla \ayr{v̔}	→	\ayr{v̔\_M} //
	\glb /vva/	→	/vvaN/ //
\endgl
\xe

\index{diacritics|)}

\section{Numerals}
\index{numerals|(}

Ayeri uses a duodecimal number system, that is, a system based on the powers of
of 12, which is a typological rarity.\footnote{And one possibly
overrepresented by invented languages due to its rarity in natural languages.}
There is a digit for zero, so the system is positional, like the Hindu--Arabic
digits used by the Latin alphabet. The numerals for the numbers from 1 to 12
are shown in \autoref{tab:thnum}.

\begin{table}[t]
\caption{The numerals}

\begin{tabu} to \linewidth{X[c] X[c] X[c] X[c] X[c] X[c]}
\toprule
\tableheaderfont	1 & 2 & 3 & 4 & 5 & 6 \\
\rowfont{\Tagati\huge}	1 & 2 & 3 & 4 & 5 & 6 \\

\midrule

\tableheaderfont	7 & 8 & 9 & \ten & \elv & 10 \\
\rowfont{\Tagati\huge}	7 & 8 & 9 & ¹ & ² & 10 \\

\bottomrule
\end{tabu}
\label{tab:thnum}
\end{table}

% How are the various mathematical operations indicated, especially the basic 
% ones: addition, subtraction, multiplication, division, equality?

\index{numerals|)}

\section{Punctuation and abbreviations}
\index{punctuation|(}

Tahano Hikamu's system of manipulating the sound of syllables is very 
sophisticated, so it comes as no surprise that it is also host of a large 
number of punctuation marks. \autoref{tab:thpunctcom} lists the ones commonly 
encountered, \autoref{tab:thpunctuncom} the ones not so commonly encountered.

\begin{table}[tp]
\caption{Common punctuation marks}
\begin{tabu} to \linewidth{>{\Tagati\huge}X[1.5] X[8.5l] X[15.5l] X[11.5l]}
\toprule
\tableheaderfont

	& Native name
	& Function
	& Example
	\\
	
\toprule

.
	& \xayr{dnF}{dan}{dot}
	& Full stop
	& \xayr{sryaaNF.}{Sarayāng.}{He left.}
	\\
	
\midrule

/
	& \xayr{dnF/dnF}{dan-dan}{little dot}
	& A separator for small things, like clitics and abbreviations; 
		divides the constituents of reduplication
	& \xayr{Ad/nN}{ada-nanga}{this house}; %\newline
		\xayr{5/pd}{5:pd}{5~hrs}; %\newline
		\xayr{dnF/dnF}{dan-dan}{dot-dot, little dot}
	\\
	
\midrule

–
	& \xayr{puMtaanF}{puntān}{dash}
	& General sign for a longer pause, equivalent to a dash, 
		colon, semicolon, brackets 
	& \xayr{ynF – sru!}{Yan---saru!}{Yan---go!}
	\\

\midrule

?
	& \xayr{dMpFrMtnF}{dam\-pran\-tan}{question point}
	& Marks questions
	& \xayr{mnisu?}{Manisu?}{Hello?}
	\\

\midrule

!
	& \xayr{dMbhaanF}{dam\-ba\-hān}{shouting point}
	& Marks exclamations; strong exclamations may be marked by the \ayr{‼} 
		variant.
	& \xayr{mnisu!}{Manisu!}{Hello!}; %\newline
		\xayr{yi‼}{Yi!}{Urgh!}
	\\

\bottomrule
\end{tabu}
\label{tab:thpunctcom}
\end{table}

\ayr{.}~\orth{.} does not look very much like a dot or a point, but it is 
derived from a sign that looks like two circles stacked on top of each other, 
similar to \ayr{/}~\orth{-} (see \autoref{fig:th2005}). There is no mark for a 
comma as such, so \ayr{/}~\orth{-} or \ayr{–}~\orth{--} cannot be 
used in this way. Instead of a comma, a wide word space is used to separate 
syntactic units. A long dash \ayr{—}~\orth{---} is also sometimes found at the 
end of paragraphs or texts to mark their end. The strong 
exclamation mark \ayr{‼} may appear in its exclamatory function at the end 
of a line, but does not necessarily indicate strong emphatic force in this 
case, but just an emphatic statement.

\begin{table}[tp]
\caption{Less common punctuation marks}
\begin{tabu} to \linewidth{>{\Tagati\huge}X[1.5] X[8.5l] X[15.5l] X[11.5l]}
\toprule
\tableheaderfont

	& Native name
	& Function
	& Example
	\\
	
\toprule

“*”
	& \xayr{dnraanF}{danarān}{speaking point}
	& Quotation marks
	& \xayr{nryaaNF “mnisu!”}{Narayāng \enquote{Manisu!}}{He says, 
		\enquote{Hello!}}
	\\
	
\midrule

(*)
	& \xayr{dMkjvo}{dankayvo}{beside-point}
	& Bracketing of text
	& \xayr{bhisF (lrau)}{bahis (larau)}{a (nice) day}
	\\

\midrule

[*]
	& \xayr{dMgrnF}{dangaran}{name-point}
	& Explicitly marks a name as such. The closing bracket can be found as
		\ayr{*̕	} as well.
	& \rayr{[AgYaanF svti]}{Ajān Savati}; \rayr{[{\normalfont }pil lj 
		mrnF]}{Pila Lay Maran}
	\\
	
\midrule

·
	& \xayr{dnFsiMdj}{dansinday}{number-point}
	& Marks (duo-)decimal fractions
	& \xayr{17·45²82}{\textsc{17.45b82}}{19.37482}
	\\
	
\midrule

¶
	& \xayr{AdFrumy}{adrumaya}{breaker}
	& Marks line breaks within a phrase
	& 
	\\

\bottomrule
\end{tabu}
\label{tab:thpunctuncom}
\end{table}

Regarding the less common marks, some of these seem like all to bland copies of
modern punctuation in the Latin alphabet, especially the brackets and the
decimal point. Still, they may serve their purpose sometimes, and the brackets
\ayr{(*)} visually push off the text around the inclusion rather than
encapsulating it within them. The name brackets \ayr{[*]} are useful in that
many names in Ayeri are derived from common nouns. For example,
\rayr{AgYaanF}{Ajān} is literally `play, game', relating to a playful
character; \rayr{migorj}{Migoray} literally means `flower'. The name brackets,
make it unmistakably clear that a proper noun is intended rather than a common
noun. The line-breaker \ayr{¶} serves the purpose of marking the continuation
of a clause at the end of a line either generally or where there would be
ambiguity with the equivalent of a comma, which would otherwise be invisible at
the end of a line.

Two very common abbreviations are symbolic in nature, like the ampersand
\orth{\&} in the Latin alphabet. Incidentally, they correspond to it in
that the very common small word \xayr{nj}{nay}{and} may be abbreviated as
\ayr{\&}. Based on this, its reduplicated form \xayr{njnj}{naynay}{furthermore,
also} may be abbreviated as \ayr{+}.

\index{punctuation|)}

\section{Styles}

Just like the Latin alphabet's upright and cursive type, print and cursive
handwriting, roman and blackletter, Tahano Hikamu has different letter styles
associated with it.\footnote{Over the course of the years since Tahano Hikamu's
inception, I have liked to experiment by applying a number of different writing
styles to the script to change its look and feel while still staying true to
the overall character shapes and the system behind the script.} The example
text I will be using to illustrate the different styles in the following is an
Ayeri translation of the first article of the United Nations'
\tit{\citetitle{udhr}} \citep{benung:udhr}:

\blockcquote[Article 1]{udhr}{\textit{Sa vesayon keynam-ikan tiganeri nay
kaytanyeri sino nay kamo.\\
Ri toraytos tenuban nay iprang, nay ang mya rankyon sitanyās ku-netu.}

[All human beings are born free and equal in dignity and rights.\\
They are endowed with reason and conscience and should act towards each other 
in a spirit of brotherhood.]}

Previous examples all used a style I call `book' style since it comes close to
printed letters, or also what might be conceivable as being written with quills
or nibs on parchment or paper---of course, pen and paper is also what I used to
make up the letters in the first place, without a second thought about the
limitations of the supposed original writing utensils. The `book' style letters
are what I consider the canonical form. \autoref{fig:thbook} shows the above
article in this letter style.

\begin{figure}[ht]\centering
{\Tagati\Large s vesyonF kejnmF/IknF tigneri nj kjtnFyeri sino nj kmo.\\
ri torjtosF tenubnF nj IpFrNF, nj ANF mY rMkYonF sitnFyaasF 
ku/netu.}
\caption{Tahano Hikamu, `book style'}
\label{fig:thbook}
\end{figure}

As described above, I have long found the look of the Javanese
script\footnote{For examples, see \citet{everson2008}, or \tit{Wikipedia}.}
rather interesting and thus I tried applying the general aesthetics of what I
had seen of it to Tahano Hikamu at some point. As mentioned above as well,
there are no subscript letters in Ayeri, and the number of large swirling
diacritics is also rather low, so there is still definitely a difference in
appearance. The `angular' style is also the one that is comparable in function
to our bold face or italic style. This letter style
(\xayr{hinY}{hinya}{angular}) is displayed in \autoref{fig:thangular}.

\begin{figure}[ht]\centering
{\Tagati\itshape\Large s vesyonF kejnmF/IknF tigneri nj kjtnFyeri sino nj
kmo.\\
ri torjtosF tenubnF nj IpFrNF, nj ANF mY rMkYonF sitnFyaasF ku/netu.}
\caption{Tahano Hikamu, `angular style'}
\label{fig:thangular}
\end{figure}

The greatest difference to the `book' style is that many of the main strokes 
double to become a thick and a parallel thin line. The shape of 
\ayr{n}~\orth{na} also changes into a simple descending line. The vowel carrier
\ayr{ʔ} changes to a flattened \textit{O}-like circle, and the bottom curl in
\ayr{t}~\orth{ta} changes to a wedge. While the right side of the
\ayr{s}~\orth{sa} character in the `book style' consists of two strokes---a
flag and a downwards bow, both independently attached to the main stem---they
connect here to form an \emph{R}-like shape.

Reproducing the shapes of either the `book' style or the `angular' style by
hand accurately is slow, so I wondered what daily handwriting could look like.
This presupposes pen and paper again; \citet[377]{salomon1996} mentions that
inscriptions of Brāhmī and related scripts have been found on copper plates and
plates made of other metals, besides stone,
however.\footnote{\citet{salomon1996} also writes that
\textcquote[378]{salomon1996}{very few such documents survive in South Asia,
though we do have early non-epigraphic specimens on wood, leather, palm leaf,
and birch bark from Inner Asia}.} Metal plates can be inscribed with metal
styluses and should allow similar shapes as modern pens. Wax tablets should as
well allow for relative freedom of stroke direction, so the character shapes
are probably not implausible even without assuming that pen and paper are
(widely) available. \autoref{fig:thhand} shows what Tahano Hikamu might look
like quickly jotted down by hand.

\begin{figure}[ht]\centering
\includegraphics[width=0.75\linewidth]{images/tahanohand-300dpi-bw.png}
\caption{Tahano Hikamu, `hand style'}
\label{fig:thhand}
\end{figure}

Many letter shapes become simplified, specifically \ayr{b}~\orth{ba},
\ayr{g}~\orth{ga}, \ayr{k}~\orth{ka}, \ayr{n}~\orth{na}, \ayr{N}~\orth{nga},
the vowel carrier \ayr{ʔ}, and the vowel \ayr{*i}~\orth{i}. Not shown here is
the the vowel length diacritic, \ayr{*aa}, which is simplified to a reverse
\textit{C} shape. The abbreviation \xayr{\&}{nay}{and} is used throughout,
though in a shape that is more similar to its `angular' form \ayr{\itshape \&}.
\ayr{n}~\orth{na} is also taken from the `angular' style \ayr{\itshape n}, 
which means that it is possibly the acutal basic shape, rather than the `book'
style's \ayr{n}, or both are different developments from a shared ancestor.

Most recently, I also wondered what Tahano Hikamu might look like if it were
adapted to European blackletter style. This, of course, constitutes a sharp
contrast to Ayeri's usual look and feel, which made the experiment all the more
interesting, though decidedly non-`canonic'. \autoref{fig:thblack} shows what
our example passage might have looked like at a time when Gothic book hands
flourished.

\begin{figure}[ht]\centering
\includegraphics[width=0.75\linewidth]{images/tahanoblack-300dpi-bw.png}
\caption{Tahano Hikamu, `blackletter style'}
\label{fig:thblack}
\end{figure}

The letter shapes from the `book' style stay largely intact, though all curves
are broken up into at least two strokes, and strokes from the bottom right to
the top left are avoided completely. The characters that differ most are
\ayr{g}~\orth{ga}, \ayr{r}~\orth{ra}, \ayr{N}~\orth{nga}, and the vowel carrier
\ayr{ʔ}. \ayr{n}~\orth{na} again appears in the `angular' shape, though without
its descender word-internally and in the abbreviation \rayr{\&}{nay}.
\ayr{t}~\orth{ta} comes with a horizontal stroke instead of a curl at the
bottom; \ayr{s}~\orth{sa} gains a descender, as does \ayr{r}~\orth{ra}. Not
shown here either are changes to the `large' diacritics.

\index{Tahano Hikamu|)}
