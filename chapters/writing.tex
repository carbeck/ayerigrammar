% kate: word-wrap true;

\chapter{Writing System}
\index{Tahano Hikamu|(}

In the previous chapter, example words were given in Ayeri's script, \rayr{thno 
hikmu}{Tahano Hikamu}, wherever possible. Thus, it seems advisable to include a 
description of Ayeri's native writing system here as well. Literally, 
\rayr{thno hikmu}{Tahano Hikamu} means `Round Script' (script round), which is 
an old formation based on the word \xayr{thnF/}{tahan-}{write} that  stuck. The 
current word for `script' is \xayr{thnnF}{tahanan}{writing}. Tahano Hikamu was 
originally named thus because of an earlier draft for a script that never made 
it very far beyond the drawing board and which was a lot more angular, or 
\rayr{hinY}{hinya}, see \autoref{fig:hinya}.\footnote{Unfortunately, there is 
no documentation surviving that I know of.}\index{Tahano Hinya}

\begin{figure}[tp]
\caption{Tahano Hinya and Hikamu}

\begin{minipage}{.5\linewidth}
\includegraphics[width=\linewidth]{images/hinya-300dpi-clip.png}
\subcaption{Old and aborted draft: Tahano Hinya}
\label{fig:hinya}
\end{minipage}
~
\begin{minipage}{.5\linewidth}
\includegraphics[width=\linewidth]{images/tahano-300dpi-bw-clip.png}
\subcaption{Ayeri's native script: Tahano Hikamu}
\end{minipage}

\label{fig:hinyahikamu}
\end{figure}

As we have seen in the previous chapter, Ayeri's prosody strongly emphasizes 
the syllable as a unit. Thus, it is not a surprise that Ayeri's native script,
Tahano Hikamu, is an alphasyllabary like the Brāhmī-derived alphabets of India 
and Southeast Asia \parencites{salomon1996}{court1996}.\index{Brāhmī scripts} 
This means that its 

\blockcquote[376]{salomon1996}{system is based on the unit of the graphic 
\enquote{syllalbe} […], which by definition always ends with a vowel (type V, 
CV, CCV, etc.). Syllables consisting of a vowel only (usually at the beginning 
of a word or sentence) are written with the \emph{full} or \emph{initial vowel 
signs} […]. But when, as is much more frequently the case, the syllable 
consists 
of a consonant followed by a vowel, the vowel is indicated by a diacritic sign 
attached to the basic sign for the consonant […].}

For Tahano Hikamu, the definition that a syllable consisting only of a vowel is 
written with an initial vowel sign is only true under certain circumstances, as 
we will see below. Moreover, Brāmī scripts are often characterized by 
conjuncts of clustered consonants which may become quite large and sometimes 
behave in an idiosyncratic way. Consonant conjuncts like Devanāgarī {\FS 
त्व}~\orth{tva} ← {\FS त}~\orth{ta} + {\FS व}~\orth{va} or idiosyncratic 
conjuncts like {\FS क्ष} \orth{kṣa} ← {\FS क} \orth{ka} + {\FS ष} \orth{ṣa} are 
not known in Tahano Hikamu, however. Tahano Hikamu also does not know subscript 
notation for consonant clusters and special diacritics marking coda consonants 
like in Javanese \citep[478--479]{kuipersmcdermott1996}. This does not mean, 
however, that final consonants are simply omitted in writing, since closed 
syllalbes are reasonably common enough to warrant indicating them. Thus, like in 
the Sumatran scripts, there is \textcquote[476]{kuipersmcdermott1996}{a special 
mark to eliminate the vowel of the previous syllable, thereby leaving a 
consonant in a syllable-final position.} That is, there is a diacritic that 
marks the absence of an inherent vowel. With regards to Indian scripts, 
it is referred to as \fw{virāma}; the Ayeri name is 
\xayr{goMdy}{gondaya}{extinguisher}.

Another difference from Brāhmī-family scripts is that all basic vowels have 
unique graphemes; vowel length and diphthongs in [ɪ] are indicated by dedicated 
diacritics. Like in Kharoṣṭhī---another historically important 
ancient script of India---, initial vowels are not represented by unique 
graphemes but they are all written like post-consonantal diacritics with the 
grapheme for initial \orth{a} as a basis, \ayr{A} \citep[377]{salomon1996}. The 
\ayr{*a} on \ayr{ʔ} is understood as a diacritic as well, however, namely for 
/a/, which is why it is indicated in the table below as \ayr{ʔ} \orth{Ø} for no 
intrinsic sound value; its native name is 
\xayr{rnYn}{ranyan}{nothing}.\footnote{I will give the native names of graphemes 
here, but will refer to them by their English names for clarity.} Similar to the 
Javanese script, Tahano Hikamu puts diacritics not only below or above consonant 
bases, but also before them. This, however, is not limited to vowel graphemes as 
in Javanese \citep[478]{kuipersmcdermott1996}.

\section{Consonants}
\index{consonants|(}

Tahano Hikamu is mainly based on consonant bases that are modified by 
diacritics. Since the vowel /a/ is so highly frequent in Ayeri, it is also the 
vowel that is \fw{inherent} to every consonant grapheme if not further modified 
by vowel diacritics. Consonant letters are simply referred to as \textit{pa, ta, 
ka, ...} \autoref{fig:consonants} displays all the main consonants. The 
customary collation is---similar to the IPA table---roughly grouping the letters 
according to their sound value by anteriority (front → back) and sonority (low → 
high). The script is monocameral, that is, there is no distinction between 
capital letters and minuscule letters as in the Latin, Greek, Cyrillic, 
Georgian, and Armenian alphabet. It is also written in lines from left to right.

% \begin{center}\itshape
% 	pa, ta, ka;\\
% 	ba, da, ga;\\
% 	ma, na, nga;\\
% 	va, sa, ha;\\
% 	ra, la, ya;\\
% 	Ø.\\
% \end{center}

\begin{figure}[ht]
\caption{The consonant graphemes of Tahano Hikamu}

\begin{tabu} to \linewidth{X[c] X[c] X[c] X[c] X[c] X[c]}
\toprule
\tableheaderfont	pa & ta & ka & ba & da & ga \\
\rowfont{\Tagati\huge}	p & t & k & b & d & g \\

\midrule

\tableheaderfont	ma & na & ŋa & va & sa & ha \\
\rowfont{\Tagati\huge}	m & n & N & v & s & h \\

\midrule

\tableheaderfont	ra & la & ja & Ø \\
\rowfont{\Tagati\huge}	r & l & y & ʔ \\

\bottomrule
\end{tabu}
\label{fig:thcons}
\end{figure}

What is interesting about \ayr{N}~\orth{nga} is that even though before, /ŋ/ 
was treated strictly as a coda consonant, it is in fact treated as an onset 
consonant in Tahano Hikamu if a vowel is following:

\pex[lingstyle=thex]\begingl
	\gla \ayr{p}	+	\ayr{NisF} //
	\glb /pa/	{}	/ŋis/ //
	\glft \rayr{pNisF}{pangis} /paŋ.ɪs/ `money'
\endgl\xe

Tahano Hikamu knows a few ligatures. First of all, when two \ayr{n} \orth{na} 
are in succession, they will form a ligature \ayr{nn} \orth{nana}:

\pex[lingstyle=thex]\begingl
	\gla \ayr{n}	+	\ayr{n}	→	\ayr{nn} //
	\glb /na/	{}	/na/	{}	/nana/ //
\endgl\xe

\noindent This is distinct from conjuncts like in Devanāgarī et al., though, 
since the unmodified sound value will still be /nana/, not */nna/, so the 
inherent vowel of each \ayr{n} \orth{na} is not deleted, and each \ayr{n} 
\orth{na} retains the ability to be modified by diacritics. Tahano Hikamu also 
has a few ligatures of the kind you would find in Brāhmī scripts, however:

\pex
	\a \ayr{q}~\orth{kwa} ← \ayr{k}~\orth{ka} + \ayr{v}~\orth{va},
	\a \ayr{T}~\orth{tsa} ← \ayr{t}~\orth{ta} + \ayr{s}~\orth{sa}, and 
	\a \ayr{x}~\orth{ksa} ← \ayr{k}~\orth{ka} + \ayr{s}~\orth{sa}.
\xe

\noindent These conjunct letters are, however, not normally employed by Ayeri. 
\autoref{fig:thconsadd} shows all additional consonants, added to write other 
languages. Individual languages may adapt the sound values slightly to fit their 
own purposes.

\begin{figure}[ht]
\caption{Additional consonant graphemes of Tahano Hikamu}

\begin{tabu} to \linewidth{X[c] X[c] X[c] X[c] X[c] X[c]}
\toprule
\tableheaderfont	fa & wa & tsa & za & ʃa & ʒa \\
\rowfont{\Tagati\huge}	f & w & T & z & S & Z \\

\midrule

\tableheaderfont	ça & ksa & kwa & xa & ɣa \\
\rowfont{\Tagati\huge}	C & x & q & X & G \\

\bottomrule
\end{tabu}
\label{fig:thconsadd}
\end{figure}

\index{consonants|)}

\section{Vowels}
\index{vowels|(}

As mentioned above, vowels are written as diacritics that are added to 
consonants. In principle, every consonant has two slots for vowels, a primary 
one atop it, and a secondary one below it. Vowels added to consonants in 
the primary slot delete their inherent /a/:

\pex[lingstyle=thex]\begingl
	\gla \ayr{p}	→	\ayr{pe} //
	\glb /pa/	{}	/pe/ //
\endgl\xe

\begin{figure}[th]
\caption{Primary vowel graphemes of Tahano Hikamu}

\begin{tabu} to \linewidth{H[c] X[c] X[c] X[c] X[c] X[c] X[c] X[c]}
\toprule
\tableheaderfont

	& i
	& e
	& a
	& o
	& u
	& ə
	& au
	\\
	
\toprule
	
Diaritics
	& \Tagati\huge *i
	& \Tagati\huge *e
	& \huge ({\Tagati *a})
	& \Tagati\huge *o
	& \Tagati\huge *u
	& \Tagati\huge *ə
	& \Tagati\huge *au
	\\

\midrule

Independent
	& \Tagati\huge I
	& \Tagati\huge E
	& \Tagati\huge A
	& \Tagati\huge O
	& \Tagati\huge U
	& \Tagati\huge Ə
	& \Tagati\huge AU
	\\

\bottomrule
\end{tabu}
\label{fig:thvowstop}
\end{figure}

\autoref{fig:thvowstop} gives the primary vowel graphemes. Of the vowel 
graphemes given there, only \ayr{*ə}~\orth{ə} is not used in Ayeri. 
\ayr{*au}~\orth{au} is the only diphthong for which a dedicated grapheme 
exists, even though its occurrence is rather limited. The independent vowel 
graphemes are used at the beginning of words or inside words when there is no 
other way to spell the vowel, which is occasionally the case for secondary 
vowels. Secondary vowels are vowels that are not parts of diphthongs, but follow 
the vowel of a syllable directly. They are attached underneath a consonant base, 
for example:

\pex[lingstyle=thex]\begingl
	\gla \ayr{y}	→	\ayr{ye}	→	\ayr{ye\_a} //
	\glb /ja/	{}	/je/		{}	/je.a/ //
\endgl\xe

In fact, the principle that every consonant base with its diacritics represents 
one syllable is slightly violated here, which is also the reason why secondary 
vowels very occasionally need to be spelled as independent vowels, for example 
when the secondary vowel is long, as in the word \xayr{ruAAnF}{ruān}{duty}:

\pex[lingstyle=thex]\label{ex:rwaa}\begingl
	\gla \ayr{ru}	→	\ayr{ruAA}	\quad	(\,\ayr{ruu\_a}) //
	\glb /ru/	{}	/rwaː/ 		\quad	/ruːa/ //
\endgl\xe

Example (\ref{ex:rwaa}) uses a diacritic, \ayr{*aa}, to indicate length. If 
is put directly under \rayr{ru}{ru} (the \ayr{*\_a} diacritic moves down where 
it is not in the way), the syllable will incorrectly spell /ruːa/ instead of 
the intended /ruaː/. This is because diacritics modify consonants and primary 
vowels, but there is no direct way to modify a secondary vowel. 
\autoref{fig:thvowsbot} gives a list of secondary vowels corresponding to that 
of primary vowels above. The vowels as well are just referred to by their sound 
value; `primary' and `secondary', `superscript' and `subscript' or `upper' and 
`lower' may be chosen to disambiguate their positions; the native names may use 
\xayr{Iraj}{iray}{high} and \xayr{Ejr}{eyra}{low} to disambiguate, so \rayr{E 
Irj}{e iray} denotes the superscript \orth{e} diacritic while \rayr{E Ejr}{e 
eyra} denotes its subscript counterpart.

\begin{figure}[ht]
\caption{Secondary vowel graphemes of Tahano Hikamu}

\begin{tabu} to \linewidth{X[c] X[c] X[c] X[c] X[c] X[c] X[c]}
\toprule
\tableheaderfont	i & e & a & o & u & ə & au \\
\rowfont{\Tagati\huge}	*\_i & *\_e & *\_a & *\_o & *\_u & *\_ə & *\_au \\

\bottomrule
\end{tabu}
\label{fig:thvowsbot}
\end{figure}

As a further exception, those consonants bases with a descender (\ayr{k}, 
\ayr{d}, \ayr{C}) move the primary vowel to the secondary slot by default while 
indicating the vacancy of the primary slot with a dot; this is mainly done to 
avoid crossing the ascender of the consonant with a vowel diacritic:

\pex[lingstyle=thex]\begingl
	\gla \ayr{k}	→	\ayr{k\_i}	→	\ayr{ki} //
	\glb /ka/	{}	/ka.i/		{}	/ki/ //
\endgl\xe

If the primary vowel slot were not silenced by the \ayr{*\_F} diacritic, it 
could reasonably be assumed that the consonant is not losing its inherent /a/ 
and the vowel underneath the consonant indicates a secondary vowel, spelling 
/CaV/. If, however, a secondary vowel is \emph{actually} added, primary and 
secondary vowels will be assigned the regular primary and secondary slots, 
respectively, again:

\pex[lingstyle=thex]\begingl
	\gla \ayr{ki}	→	\ayr{ki\_e} //
	\glb /ki/	{}	/ki.e/ //
\endgl\xe

\index{vowels|)}

\section{Diacritics}

We have already encountered two diacritics, one for lengthening the 
primary vowel of a syllalbe and one for deleting the inherent vowel of a 
consonant. Tahano Hikamu comes with a lot more diacritics, however, which 
undergo non-trivial positioning and repositioning rules. As vowels are 
primarily superscripts, diacritics are primarily subscripts, so in the 
following I will first describe subscript diacritics; then preposed diacritics, 
which Ayeri also has a number of, both as graphemes in their own right and as 
allographs of other subscript diacritics; and lastly, superscript diacritics. 
\autoref{fig:thdiabot} shows the bottom-attaching diacritics.

\begin{sidewaysfigure}{p}
\caption{Bottom-attaching diacritics of Tahano Hikamu}
\begin{tabu} to \linewidth{{>\Tagati\huge}X[1] X[2] X[3] X[3]}
\toprule
\tableheaderfont
Grapheme
	& Native name
	& Function
	& Example
	\\
	
\toprule

\tablesubheaderfont\multicolumn4{c}{Large diacritics}\\

\midrule

*aa
	& \xayr{tupsti}{tupasati}{long-maker}
	& Lengthens the primary vowel of the syllable
	& \rayr{p}{pa} → \rayr{paa}{pā}
	\\
	
*Y
	& \xayr{y ejr}{ya eyra}{low ya}
	& Ya following another consonant, also across syllables. Marks 
		palatalization of \rayr{t}{ta}, \rayr{d}{da}, \rayr{k}{ka}, and 
		\rayr{g}{ga} in Ayeri.
	& \rayr{Ar}{ara} → \rayr{ArY}{arya};\newline
		\rayr{t}{ta} → \rayr{tY}{ca}
	
*J
	& \xayr{riNy}{ringaya}{raiser}
	& Palatalizes a consonant\footnotemark
	& \rayr{t}{ta} → \ayr{tJ} /tʲa/, /tʃa/
	\\
	
*H
	& \xayr{UlNy}{ulangaya}{breather}
	& Aspiration or frication of consonant\footnotemark
	& \rayr{t}{ta} → \ayr{tH} /tʰa/, /θa/
	\\
	
*Q
	& \xayr{rjpaay Ejr}{raypāya eyra}{low stopper}
	& Glottal stop coda or glottalization of a consonant, 
		depending on language/context\footnotemark
	& \rayr{t}{ta} → \ayr{tQ} /taʔ/;\newline
		\rayr{s}{sa} → \ayr{sQ} /s’a/

\midrule

\tablesubheaderfont\multicolumn4{c}{Small diacritics}\\

*F
	& \xayr{goMdy}{gondaya}{extinguisher}
	& Deletes inherent /a/ of consonant, e.g. in consonant clusters or 
		closed syllables
	& \rayr{pr}{para} → \rayr{pFr}{pra}, \rayr{prF}{par}
	\\
	
*M
	& \xayr{vinaati}{vināti}{noser}
	& Indicates a homorganic nasal or nasalizes the vowel, depending on 
		language/context
	& \rayr{pd}{pada} → \rayr{pMd}{panda} /panda/ or /pãda/
	\\
	
*F*
	& \xayr{kusNisaati}{kusangisāti}{duplicator}
	& Indicates a geminated or otherwise double consonant
	& \rayr{pl}{pala} → \rayr{plFl}{palla}
	\\

\midrule

\bottomrule
\end{tabu}
\label{fig:thvowstop}
\end{sidewaysfigure}

\footnotetext{Not used in Ayeri.}

The `large diacritics' in \autoref{fig:thvowstop} cause the secondary slot of 
consonants to appear below them. `Small diacritics' can attach in this place as 
well as secondary vowels, as does the homorganic nasal diacritic in this rather 
extreme example:

\pex[lingstyle=thex]\begingl
	\gla \ayr{tYaan}	+ \ayr{pluY}	→ \ayr{tYaaMpuluj} //
	\glb /tʃaːn/		{} /puˈlʊɪ/	→ /ˌtʃaːmpuˈlʊɪ/ //
	\glft \xayr{tYaaMpuluj}{cāmpuluy}{heterosexual}
\endgl\xe

\index{Tahano Hikamu|)}