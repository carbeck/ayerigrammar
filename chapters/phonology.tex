\chapter{Phoneme Inventory and Phonotactics}

This chapter will present charts depicting the phoneme inventory of Ayeri, 
give an analysis of the phonotactics of Ayeri's dictionary entries and also 
describe stress patterns.

\section{Phoneme Inventory}

\subsection{Consonants}

At 17 consonants, Ayeri has a fairly mid-sized inventory. 
\autoref{tab:consonants} shows the full chart. The sound /w/ only occurs 
marginally in \rayr{huAAky}{huākaya} [ˈwaːkaja] `frog'. Other instances of it 
are allophones of /u/ followed by a vowel, for instance in \rayr{ru\_a/}{rua-} 
/rwa/ `have to, must'. /w/ may also be an allophone of /uj/, as in 
\rayr{Adauyi}{adauyi} [aˈdawi] `then', \rayr{Edauyi}{edauyi} [eˈdawi] `now', or 
\rayr{nekuyi}{nekuyi} [ˈnekwi] `eyebrows'. The negative suffix \rayr{/Oj}{-oy} 
is also commonly contracted to [w] before a diphthong:

\ex
	\rayr{miNojAj}{ming\textbf{oy}ay → ming\textbf{u}ay} [mɪŋˈwaɪ] `I cannot' (can-\Neg{}-\Fsg{})
\xe

Moreover, the affricates /tʃ/ and /dʒ/ are usually allophones of /tj kj/ and 
/dj gj/, respectively. The plural marker \rayr{/ye}{-ye} is also commonly 
contracted to [dʒ] when a case suffix beginning with a vowel follows:

\pex
	\a \rayr{nYaanFye\_aNF}{nyān\textbf{ye}ang → nyān\textbf{j}ang} [ˈnjaːndʒaŋ] `persons' (person-\Pl{}-\Aarg{})
	\a \rayr{netuye\_asF}{netu\textbf{ye}as → netu\textbf{j}as} [neˈtudʒas] `brothers' (brother-\Pl{}-\Parg{})
\xe

The plural marker also may contract before the locative marker \rayr{/y}{-ya}, 
basically for dissimilation:\footnote{\rayr{/E\_a}{-ea} also occurs as an 
variant morpheme, so that \rayr{/ye}{-ye} + \rayr{/E\_a}{-ea} → 
\rayr{/yee\_a}{-yēa}.}

\ex
	\rayr{nivyey}{niva\textbf{ye}ya → niva\textbf{j}ya} [niˈvadʒja] `at the eyes' (eye-\Pl{}-\Loc{})
\xe

Dissimilation of the sequence \rayr{/yy}{-yaya} is attested in 
\citet[12]{becker:kafka:imperial}, where the relative pronoun 
\rayr{siyy}{siyaya} appears transcribed as \textit{sijya}:

\begin{quote}
As far as morphophonology is concerned, the relative pronoun complex 
\textit{sijya} `in/at/on which.\Loc{}' is interesting in so far as it is a 
contraction of \textit{*siyaya} `\Rel{}-\Loc{}-\Loc{}' that I introduced here
[...] Since this feature does not occur in previous texts, let's assume it's an 
acceptable variant.
\end{quote}

\noindent It is noted, however, that the contraction happens 
\textcquote[12]{becker:kafka:imperial}{only if both parts are grammatical 
suffixes}.

While vowels become long when two identical vowels come into succession,  
consonants do not geminate:

\pex
	\a \rayr{tvFvaanF}{ta\textbf{vv}āng} [taˈvaːŋ] `you get' (get=\Ssg{}.\Aarg{})
	\a \rayr{diʲsyNF}{dis\textbf{yy}ang} [dɪsˈjaŋ] `I fasten' (fasten=\Fsg{}.\Aarg{})
\xe

\begin{sidewaystable}[p]
\caption{Consonant inventory}
\begin{tabu} to \textwidth {H[2l] X[c] X[c] X[c] X[c] X[c] X[c] X[c] X[c] X[c] X[c] X[c] X[c]}
\toprule\tableheaderfont
	%
	& \multicolumn2{c}{Bilabials}
	& \multicolumn2{c}{Labiodentals}
	& \multicolumn2{c}{Alveolars}
	& \multicolumn2{c}{Palatals}
	& \multicolumn2{c}{Velars}
	& \multicolumn2{c}{Glottals}
	\\

\midrule

Plosives
	& p & b         	% Bilabials
	&   &           	% Labiodentals
	& t & d         	% Alveolars
	&   &           	% Palatals
	& k & ɡ \orth{g}	% Velars
	&   &           	% Glottals
	\\

\midrule

Affricates
	&             &            	% Bilabials
	&             &            	% Labiodentals
	& tʃ \orth{c} & dʒ \orth{j}	% Alveolars
	&             &            	% Palatals
	&             &            	% Glottals
	\\

\midrule

Nasals
	&   & m          	% Bilabials
	&   &            	% Labiodentals
	&   & n          	% Alveolars
	&   &            	% Palatals
	&   & ŋ \orth{ng}	% Velars
	&   &            	% Glottals
	\\

\midrule

Fricatives
	&   &  	% Bilabials
	&   & v	% Labiodentals
	& s &  	% Alveolars
	&   &  	% Palatals
	&   &  	% Velars
	& h &  	% Glottals
	\\

\midrule

Taps/Flaps
	&   &  	% Bilabials
	&   &  	% Labiodentals
	&   & r	% Alveolars
	&   &  	% Palatals
	&   &  	% Velars
	&   &  	% Glottals
	\\

\midrule

Approximants
	&   & (w)       	% Bilabials
	&   &           	% Labiodentals
	&   & l         	% Alveolars
	&   & j \orth{y}	% Palatals
	&   &           	% Velars
	&   &           	% Glottals
	\\

\bottomrule
\end{tabu}
\label{tab:consonants}
\end{sidewaystable}

\subsection{Vowels}

\begin{table}[h]\centering
\caption{Vowel inventory}
\begin{tabu} to .5\textwidth{H X[c] X[c] X[c]}
\toprule\tableheaderfont

	& Front
	& Center
	& Back
	\\

\toprule

High
	& i, iː
	&
	& u
	\\

Mid
	& e, eː
	& (ə)
	& o, oː
	\\

Back
	&
	& a, aː
	&
	\\

\bottomrule
\end{tabu}
\label{tab:vowels}
\end{table}

Ayeri has a very basic five-vowel system, shown in \autoref{tab:vowels}. The 
lax vowels [ɪ ɛ ɔ ʊ] occur as allophones of their tense counterparts 
[i e o u] in closed syllables, for example:

\pex
	\a \rayr{miNF}{m\textbf{ing}} [mɪŋ] `can, be able',
	\a \rayr{EnFy}{\textbf{en}ya} [ˈɛnja] `everyone',
	\a \rayr{AgonF}{ag\textbf{on}} [ˈaɡɔn] `outer, foreign', and
	\a \rayr{pkurF}{pak\textbf{ur}} [ˈpakʊr] `ill, sick'.
\xe

/ə/ is a marginal phoneme and only occurs in the tense prefixes 
\xayr{k/}{kə-}{\NPst{}}, \xayr{m/}{mə-}{\Pst{}}, \xayr{v/}{və-}{\RPst{}}, as 
well as in the prefix \xayr{me/}{mə-}{some, whichever}. Otherwise, [ə] occurs 
as an allophone of /e/ in final unstressed position, e.g. in the word 
\rayr{mine}{min\textbf{e}} [ˈminə] `affair, matter, issue'.

Ayeri also possesses a number of diphthongs, these are: /aɪ aːɪ eɪ ɔɪ ʊɪ aʊ/.
Furthermore, the vowels [i e o u] may be long: [aː eː iː oː]. Long vowels are  
lexicalized in a few words, for example: \xayr{yaaNF}{yāng}{he.\Aarg{}}, 
\xayr{laa}{lā}{tongue}, \xayr{niis}{nīsa}{wanted}, 
\xayr{psiis}{pasīsa}{interesting}, \xayr{AreenF}{arēn}{anyway, however}, 
\xayr{leer}{lēra}{whore}, and \xayr{noonF}{nōn}{wish}. Otherwise, long vowels 
result from two same vowels next to each other, for instance:

\ex \xayr{AgY/}{aja-}{play} + \xayr{/AnF}{-an}{\Nmlz{}} → \xayr{AgYaanF}{ajān}{game, play}. \xe

Morphophonologically, long vowels also occur in double-marked relative pronouns 
where the agreement marker for the relative clause's head has been omitted,
for instance, \xayr{sinaa}{sinā}{of which, about which}, as in the following 
example:

\ex\begingl
	\gla Le turayāng taman sinā ang ningay tamala vās. //
	\glb Le tura-yāng taman-Ø si-Ø-na ang ning=ay.Ø tamala vās //
	\glc \PatTI{} send=\Tsg{}.\M{}.\Aarg{} letter-\Top{} \Rel{}-\PatTI{}-\Gen{} \AgtT{} tell=\Fsg{}.\Top{} yesterday \Ssg{}.\Parg{} //
	\glft `The letter which I told you about yesterday, he sent it.' //
\endgl\xe

This is to disambiguate it from the plain genitive-marked relative pronoun 
\xayr{sin}{sina}{which.\Gen{}}:

\ex\begingl
	\gla tamanang ledanena nā sina koronvāng //
	\glb taman-ang ledan-ena nā si-na koron-vāng //
	\glc letter-\Aarg{} friend-\Gen{} \Fsg.\Gen{} \Rel{}-\Gen{} know=\Ssg{}.\Aarg{} //
	\glft `the letter of my friend which you know' //
\endgl\xe

As pointed out above, the word \xayr{laa}{lā}{tongue} ends in a long vowel, so 
the question is what happens when a case suffix beginning with a vowel is 
appended. To avoid a hiat, a glide /j/ may be inserted, so both of these are 
possible:

\pex
	\a\begingl
		\gla Aku lāas! //
		\glb Aka-u lā-as //
		\glc swallow-\Imp{} tongue-\Parg{} //
		\glft `Shut up!' //
	\endgl
	\a\begingl
		\gla {Aku lāyas!} //
		\glb (\textit{idem}) //
	\endgl
\xe

\section{Phonotactics}

For the purpose of this statistical analysis, all of the available translations 
into Ayeri since 2008 have been used as a text corpus; example sentences from 
various blog articles have also been added, as well as dictionary entries for 
all nouns, adjectives, adverbs, pronouns, adpositions, conjunctions, and 
numerals if they were not prefixes or suffixes. Borrowings have been deleted, 
if they could not reasonably be words in Ayeri. Altogether, the corpus 
comprises 5,502 words, of which 3,050 are unique.

Among the dictionary entries, verbs have notably been ignored, since verb stems 
alone do not constitute independent words -- they are always inflected in some 
way, so that they may end in consonants or consonant clusters that independent 
words cannot end in. This also has repercussions on syllabification and stress, 
which depend on the inflection of the verb stem:

\begin{table}[h]
\caption{Syllabification of inflected verbs}
\begin{tabu} to \linewidth {X[2l] X[3c] X[3c] X[3c]}
\toprule\tableheaderfont
Suffix
	& \emph{ca-} `love'
	& \emph{gum-} `work'
	& \emph{babr-} `mumble'
	\\

\toprule

\emph{-ay} (\Fsg{})
	& cā́y
	& gu.máy
	& ba.bráy
	\\

\emph{-va} (\Ssg{})
	& cá.va
	& gúm.va
	& ba.brá.va
	\\

\emph{-yam} (\Ptcp{})
	& cá.yam
	& gúm.yam
	& bá.bryam
	\\

\bottomrule
\end{tabu}
\label{ex:verbsyll}
\end{table}

The statistics have been generated by \tit{\citetitle{strasser:freq}} 
\autocite{strasser:freq}.
