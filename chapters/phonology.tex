\chapter{Phoneme Inventory and Phonotactics}

This chapter will present charts depicting the phoneme inventory of Ayeri, 
give an analysis of the phonotactics of Ayeri's dictionary entries and also 
describe stress patterns.

\section{Phoneme Inventory}

\subsection{Consonants}

At 17 consonants -- /ʔ/ and /w/ only occur marginally --, Ayeri has a fairly 
mid-sized inventory.

\begin{table}[h]
\label{tab:consonants}
\caption{Consonant inventory}
\begin{tabu} to \textwidth {H[2l] X[c] X[c] X[c] X[c] X[c] X[c] X[c] X[c] X[c] X[c] X[c] X[c]}
\toprule\tableheaderfont
	%
	& \multicolumn2{c}{Bilabials}
	& \multicolumn2{c}{Labiodentals}
	& \multicolumn2{c}{Alveolars}
	& \multicolumn2{c}{Palatals}
	& \multicolumn2{c}{Velars}
	& \multicolumn2{c}{Glottals}
	\\

\midrule

Plosives
	& p    & b	% Bilabials
	&      &  	% Labiodentals
	& t    & d	% Alveolars
	&      &  	% Palatals
	& k    & ɡ	% Velars
	& (ʔ)  &  	% Glottals
	\\

\midrule

Affricates
	&    &  	% Bilabials
	&    &  	% Labiodentals
	& tʃ & dʒ	% Alveolars
	&    &  	% Palatals
	&    &  	% Glottals
	\\

\midrule

Nasals
	&   & m % Bilabials
	&   &  	% Labiodentals
	&   & n	% Alveolars
	&   &  	% Palatals
	&   & ŋ	% Velars
	&   &  	% Glottals
	\\

\midrule

Fricatives
	&   &  	% Bilabials
	&   & v	% Labiodentals
	& s &  	% Alveolars
	&   &  	% Palatals
	&   &  	% Velars
	& h &  	% Glottals
	\\

\midrule

Taps/Flaps
	&   &  	% Bilabials
	&   &  	% Labiodentals
	&   & r	% Alveolars
	&   &  	% Palatals
	&   &  	% Velars
	&   &  	% Glottals
	\\

\midrule

Approx.
	&   & (w) 	% Bilabials
	&   &     	% Labiodentals
	&   & l   	% Alveolars
	&   & j   	% Palatals
	&   &     	% Velars
	&   &     	% Glottals
	\\

\bottomrule
\end{tabu}
\end{table}

The affricates /tʃ/ and /dʒ/ are allophones of /tj, kj/ and /dj, gj/, 
respectively.

\subsection{Vowels}

Ayeri has a very basic five-vowel system:

\begin{table}\centering
\label{tab:vowels}
\caption{Vowel inventory}
\begin{tabu} to .5\textwidth{H X[c] X[c] X[c]}
\toprule\tableheaderfont

	& Front
	& Center
	& Back
	\\

\toprule

High
	& i
	&
	& u
	\\

Mid
	& e
	& (ə)
	& o
	\\

Back
	&
	& a
	&
	\\

\bottomrule
\end{tabu}
\end{table}

The lax vowels [ɪ ɛ ɔ ʊ] occur as allophones of their tense counterparts 
[i e o u] in closed syllables, for example:

\pex
	\a \rayr{miNF}{m\textbf{ing}} [mɪŋ] `can, be able',
	\a \rayr{EnFy}{\textbf{en}ya} [ˈɛnja] `everyone',
	\a \rayr{AgonF}{ag\textbf{on}} [ˈaɡɔn] `outer, foreign', and
	\a \rayr{pkurF}{pak\textbf{ur}} [ˈpakʊr] `ill, sick'.
\xe

/ə/ is a marginal phoneme and only occurs in the 
tense prefixes \xayr{k/}{kə-}{\NPst{}}, \xayr{m/}{mə-}{\Pst{}}, \xayr{v/}{və-}
{\RPst{}}, as well as in the prefix \xayr{me/}{mə-}{some, whichever}. 
Otherwise, [ə] occurs as an allophone of /e/ in final unstressed position, e.g. 
in the word \rayr{mine}{mine} [ˈminə] `affair, matter, issue'.

\section{Phonotactics}

For the purpose of statistical analyis, only the following parts of speech have 
been considered: nouns, adjectives, adverbs, pronouns, adpositions, 
conjunctions, and numerals. Verbs have been notably ignored as verb stems do 
not constitute independent words -- they are always inflected in some way, so 
that they may end in consonants or consonant clusters that independent words 
cannot end in. This also has repercussions for syllabification and stress, 
which depends on the inflection of the verb stem:

\begin{table}[h]
\label{ex:verbsyll}
\caption{Syllabification of inflected verbs}
\begin{tabu} to \linewidth {X[2l] X[3c] X[3c] X[3c]}
\toprule\tableheaderfont
Suffix
	& \emph{ca-} `love'
	& \emph{gum-} `work'
	& \emph{babr-} `mumble'
	\\

\toprule

\emph{-ay} (\Fsg{})
	& cā́y
	& gu.máy
	& ba.bráy
	\\

\emph{-va} (\Ssg{})
	& cá.va
	& gúm.va
	& ba.brá.va
	\\

\emph{-yam} (\Ptcp{})
	& cá.yam
	& gúm.yam
	& bá.bryam
	\\

\bottomrule
\end{tabu}
\end{table}

Since pure prefixes and suffixes like \xayr{sitNF/}{sitang-}{self-} or 
\xayr{/IknF}{-ikan}{much, many} do not constitute independent words either, 
they also have been omitted from statistical analysis.
