\chapter{Phoneme Inventory and Phonotactics}

This chapter will present charts depicting the phoneme inventory of Ayeri, 
give an analysis of the phonotactics of Ayeri's dictionary entries and also 
describe stress patterns.

\section{Phoneme Inventory}

\subsection{Consonants}

\begin{sidewaystable}[p]
\caption[Consonant inventory]{Consonant inventory (divergent orthography in pointed brackets)}
\begin{tabu} to \textwidth {H[2l] X[c] X[c] X[c] X[c] X[c] X[c] X[c] X[c] X[c] X[c] X[c] X[c]}
\toprule\tableheaderfont
	%
	& \multicolumn2{c}{Bilabials}
	& \multicolumn2{c}{Labiodentals}
	& \multicolumn2{c}{Alveolars}
	& \multicolumn2{c}{Palatals}
	& \multicolumn2{c}{Velars}
	& \multicolumn2{c}{Glottals}
	\\

\midrule

Plosives
	& p & b	% Bilabials
	&   &  	% Labiodentals
	& t & d	% Alveolars
	&   &  	% Palatals
	& k & g	% Velars
	&   &  	% Glottals
	\\

\midrule

Affricates
	&             &            	% Bilabials
	&             &            	% Labiodentals
	& tʃ \orth{c} & dʒ \orth{j}	% Alveolars
	&             &            	% Palatals
	&             &            	% Glottals
	\\

\midrule

Nasals
	&   & m          	% Bilabials
	&   &            	% Labiodentals
	&   & n          	% Alveolars
	&   &            	% Palatals
	&   & ŋ \orth{ng}	% Velars
	&   &            	% Glottals
	\\

\midrule

Fricatives
	&   &  	% Bilabials
	&   & v	% Labiodentals
	& s &  	% Alveolars
	&   &  	% Palatals
	&   &  	% Velars
	& h &  	% Glottals
	\\

\midrule

Taps/Flaps
	&   &  	% Bilabials
	&   &  	% Labiodentals
	&   & r	% Alveolars
	&   &  	% Palatals
	&   &  	% Velars
	&   &  	% Glottals
	\\

\midrule

Approximants
	&   & (w)       	% Bilabials
	&   &           	% Labiodentals
	&   & l         	% Alveolars
	&   & j \orth{y}	% Palatals
	&   &           	% Velars
	&   &           	% Glottals
	\\

\bottomrule
\end{tabu}
\label{tab:consonants}
\end{sidewaystable}

At 17 consonants, Ayeri has a fairly mid-sized inventory. 
\autoref{tab:consonants} shows the full chart. The sound /w/ only occurs 
marginally in \rayr{huAAky}{huākaya} [ˈwaːkaja] `frog'. Other instances of it 
are allophones of /u/ followed by a vowel, for instance in \rayr{ru\_a/}{rua-} 
/rwa/ `have to, must'. /w/ may also be an allophone of /uj/, as in 
\rayr{Adauyi}{adauyi} [aˈdawi] `then', \rayr{Edauyi}{edauyi} [eˈdawi] `now', or 
\rayr{nekuyi}{nekuyi} [ˈnekwi] `eyebrows'. The negative suffix \rayr{/Oj}{-oy} 
is also commonly contracted to [w] before a diphthong:

\ex
	\rayr{miNojAj}{ming\textbf{oy}ay → ming\textbf{u}ay} [mɪŋˈwaɪ] `I cannot' (can-\Neg{}-\Fsg{})
\xe

Moreover, the affricates /tʃ/ and /dʒ/ are usually allophones of /tj kj/ and 
/dj gj/, respectively. The plural marker \rayr{/ye}{-ye} is also commonly 
contracted to [dʒ] when a case suffix beginning with a vowel follows:

\pex
	\a \rayr{nYaanFye\_aNF}{nyān\textbf{ye}ang → nyān\textbf{j}ang} [ˈnjaːndʒaŋ] `persons' (person-\Pl{}-\Aarg{})
	\a \rayr{netuye\_asF}{netu\textbf{ye}as → netu\textbf{j}as} [neˈtudʒas] `brothers' (brother-\Pl{}-\Parg{})
\xe

The plural marker also may contract before the locative marker \rayr{/y}{-ya}, 
basically for dissimilation:\footnote{\rayr{/E\_a}{-ea} also occurs as an 
variant morpheme, so that \rayr{/ye}{-ye} + \rayr{/E\_a}{-ea} → 
\rayr{/yee\_a}{-yēa}.}

\ex
	\rayr{nivyey}{niva\textbf{ye}ya → niva\textbf{j}ya} [niˈvadʒja] `at the eyes' (eye-\Pl{}-\Loc{})
\xe

Dissimilation of the sequence \rayr{/yy}{-yaya} is attested in 
\citet[12]{becker:kafka:imperial}, where the relative pronoun 
\rayr{siyy}{siyaya} appears transcribed as \textit{sijya}:

\begin{quote}
As far as morphophonology is concerned, the relative pronoun complex 
\textit{sijya} `in/at/on which.\Loc{}' is interesting in so far as it is a 
contraction of \textit{*siyaya} `\Rel{}-\Loc{}-\Loc{}' that I introduced here
[...] Since this feature does not occur in previous texts, let's assume it's an 
acceptable variant.
\end{quote}

\noindent It is also noted there, however, that the contraction happens 
\textcquote[12]{becker:kafka:imperial}{only if both parts are grammatical 
suffixes}.

While vowels become long when two identical vowels come into succession,  
consonants do not geminate but are treated like a single consonant:

\pex
	\a \rayr{tvFvaanF}{ta\textbf{vv}āng} [taˈvaːŋ] `you get' (get=\Ssg{}.\Aarg{})
	\a \rayr{diʲsyNF}{dis\textbf{yy}ang} [diˈsjaŋ] `I fasten' (fasten=\Fsg{}.\Aarg{})
\xe

\subsection{Vowels}

Ayeri has a very basic five-to-six-vowel system, shown in \autoref{tab:vowels}.

\begin{table}[ht]\centering
\caption[Vowel inventory]{Vowel inventory (divergent orthography in pointed brackets)}
\begin{tabu} to .5\textwidth{H[1] X[2c] X[2c] X[2c]}
\toprule\tableheaderfont

	& Front
	& Center
	& Back
	\\

\toprule

High
	& i, iː \orth{ī}
	&
	& u, uː \orth{ū}
	\\

Mid
	& e, eː \orth{ē}
	& (ə \orth{ə, e})
	& o, oː \orth{ō}
	\\

Back
	&
	& a, aː \orth{ā}
	&
	\\

\bottomrule
\end{tabu}
\label{tab:vowels}
\end{table}

The lax vowels [ɪ ɛ ɔ ʊ] occur as allophones of their tense counterparts 
[i e o u] in closed syllables, for example:

\pex
	\a \rayr{miNF}{m\textbf{ing}} [mɪŋ] `can, be able',
	\a \rayr{EnFy}{\textbf{en}ya} [ˈɛnja] `everyone',
	\a \rayr{AgonF}{ag\textbf{on}} [ˈagɔn] `outer, foreign', and
	\a \rayr{pkurF}{pak\textbf{ur}} [ˈpakʊr] `ill, sick'.
\xe

/ə/ is a marginal phoneme and only occurs in the tense prefixes 
\xayr{k/}{kə-}{\NPst{}}, \xayr{m/}{mə-}{\Pst{}}, \xayr{v/}{və-}{\RPst{}}, as 
well as in the prefix \xayr{me/}{mə-}{some, whichever}. Otherwise, [ə] occurs 
as an allophone of /e/ in final unstressed position, e.g. in the word 
\rayr{mine}{min\textbf{e}} [ˈminə] `affair, matter, issue'.

Ayeri also possesses a number of diphthongs, these are: /aɪ aːɪ eɪ ɔɪ ʊɪ aʊ/, 
spelled \orth{ay}, \orth{āy}, \orth{ey}, \orth{oy}, \orth{uy}, \orth{au}.
Furthermore, the vowels [i e a o u] may be long: [iː eː aː oː uː]. Long vowels 
are lexicalized in a few words, for example:

\pex
	\a \xayr{niis}{nīsa}{wanted}, \xayr{psiis}{pasīsa}{interesting};
	\a \xayr{AreenF}{arēn}{anyway, however}, \xayr{leer}{lēra}{whore};
	\a \xayr{laa}{lā}{tongue}, \xayr{yaaNF}{yāng}{he} (he.\Aarg{}); \label{ex:laa}
	\a \xayr{noonF}{nōn}{wish}; and 
	\a \xayr{bbuu\_anF}{babūan}{barbarian}.
\xe

\noindent Otherwise, long vowels result from two same vowels next to each other, 
for instance:

\ex
	\xayr{AgY/}{aja-}{play} + \xayr{/AnF}{-an}{\Nmlz{}} → 
	\xayr{AgYaanF}{ajān}{game, play}.\label{ex:longvwls}
\xe

Morphophonologically, long vowels also occur in double-marked relative pronouns 
where the agreement marker for the relative clause's head has been omitted,
for instance, \xayr{sinaa}{sinā}{of which, about which}, as in the following 
example:

\ex\begingl
	\gla Le turayāng taman sinā ang ningay tamala vās. //
	\glb Le tura-yāng taman-Ø si-Ø-na ang ning=ay.Ø tamala vās //
	\glc \PatTI{} send=\Tsg{}.\M{}.\Aarg{} letter-\Top{} \Rel{}-\PatTI{}-\Gen{} \AgtT{} tell=\Fsg{}.\Top{} yesterday \Ssg{}.\Parg{} //
	\glft `The letter which I told you about yesterday, he sent it.' //
\endgl\xe

This is to disambiguate it from the plain genitive-marked relative pronoun 
\xayr{sin}{sina}{which.\Gen{}}:\footnote{A variant which combines the 
allomorphs of the relativizer and the genitive case marker in the opposite way 
also exists: \rayr{s/}{s-} + \rayr{/En}{-ena} → \rayr{sen}{sena}.}

\ex\begingl
	\gla tamanang ledanena nā sina koronvāng //
	\glb taman-ang ledan-ena nā si-na koron-vāng //
	\glc letter-\Aarg{} friend-\Gen{} \Fsg.\Gen{} \Rel{}-\Gen{} know=\Ssg{}.\Aarg{} //
	\glft `the letter of my friend which you know' //
\endgl\xe

As pointed out in (\ref{ex:laa}), the word \xayr{laa}{lā}{tongue} ends in a 
long vowel, so the question is what happens when a case suffix beginning with a 
vowel is appended. To avoid a hiat, a glide /j/ may be inserted, so both of 
these are possible:

\pex
	\a\begingl
		\gla Aku lāas! //
		\glb Aka-u lā-as //
		\glc swallow-\Imp{} tongue-\Parg{} //
		\glft `Shut up!' //
	\endgl
	\a\begingl
		\gla {Aku lāyas!} //
		\glb (idem) //
	\endgl
\xe

\section{Phonotactics}

For the purpose of this statistical analysis, all of the available translations 
into Ayeri since 2008 have been used as a text corpus;\footnote{These texts are:
A Medieval Neighborhood Dispute (2015),
A Message from the Emperor (2012),
Article 1 of the Universal Declaration of Human Rights (2011),
The Beginning of Tolstoy's \tit{Anna Karenina} (2014),
Conlang Christmas Card Exchange 2008/09 (2009),
Conlang Holiday Card Exchange 2010/11 (2011),
Conlang Relay 15 (2008),
Conlang Relay 17 (2010),
Conlang Relay 18 (2011),
The First Two Chapters from Saint-Exupéry's \tit{Le Petit Prince} (2013),
The Four Candles (2010),
Honey Everlasting (2014),
LCC4 Relay (2011),
The Lord's Prayer (2015),
The North Wind and the Sun (2016),
The Origin of the Wind (2009),
Ozymandias (2011),
Please Call Stella … (2008),
Psalm 23 (2013),
The Scientific Method (2014),
The Sheep and the Horses (2012),
Sugar Fairies (2011),
The Upside-Down Ice Skater (2009).
The texts can be accessed from \citet[Examples]{benung}.
} example sentences from 
various blog articles have also been added, as well as dictionary entries for 
all nouns, adjectives, adverbs, pronouns, adpositions, conjunctions, and 
numerals if they were not prefixes or suffixes.\footnote{This section updates 
and extends a previous analysis of the phonological makeup of dictionary entries 
\autocite{becker:frequency}. The previous study had its focus on gathering 
frequency statistics for word generation, however, we want to know about words 
generally here.} Borrowings have been deleted, if they could not reasonably be 
words in Ayeri. Altogether, the corpus comprises 5,500 words, which is a very 
small figure for such a study, but there are only so many texts available 
unfortunately. Words may occur more than once.

Among the dictionary entries, verbs have notably been ignored, since verb stems 
alone do not constitute independent words---they are always inflected in some 
way, so that they may end in consonants or consonant clusters that independent 
words cannot end in. This also has repercussions on syllabification and stress, 
which depend on the inflection of the verb stem:

\begin{table}[h]
\caption{Syllabification of inflected verbs}
\begin{tabu} to \linewidth {X[2l] X[3c] X[3c] X[3c]}
\toprule\tableheaderfont
Suffix
	& \emph{ca-} `love'
	& \emph{gum-} `work'
	& \emph{babr-} `mumble'
	\\

\toprule

\emph{-ay} (\Fsg{})
	& cā́y
	& gu.máy
	& ba.bráy
	\\

\emph{-va} (\Ssg{})
	& cá.va
	& gúm.va
	& ba.brá.va
	\\

\emph{-yam} (\Ptcp{})
	& cá.yam
	& gúm.yam
	& bá.bryam
	\\

\bottomrule
\end{tabu}
\label{ex:verbsyll}
\end{table}

% The statistics have been aggregated by \tit{\citetitle{strasser:freq}} 
% \autocite{strasser:freq}. 
For the purpose of gathering statistics on phonemes, 
the words from translation texts were converted to IPA first. Fortunately, this 
is rather easy as Ayeri's romanization is very straightforward. Syllable breaks 
have also been inserted semi-automatically.

\subsection{Number of Syllables per Word}

First, let us see how many syllables words commonly have (see 
\autoref{tab:syllength}). The higher the syllable count, the more likely it is 
for them to be compounds or inflected words.

\begin{table}[hp]\centering
\caption[Relative frequency of words with different numbers of syllables]{Relative frequency of words with different numbers of syllables (n\,=\,5500)}
\begin{tabu} to .5\textwidth{X X[c] X[c]}
\tableheaderfont\toprule
Segment
	& Count
	& \multicolumn1{c}{Percentage}
	\\
\toprule

2 syllables
	& 2277
	& 41.40\pct
	\\
	
3 syllables
	& 1393
	& 25.33\pct
	\\
	
1 syllable
	& 1201
	& 21.84\pct
	\\
	
4 syllables
	& 547
	& 9.95\pct
	\\
	
5 syllables
	& 74
	& 1.35\pct
	\\
	
6 syllables
	& 8
	& 0.15\pct
	\\
	
\bottomrule
\end{tabu}
\label{tab:syllength}
\end{table}

Two-syllable words make up the bulk of the sample, which is not surprising since 
1,072 (55.43\pct) of the dictionary subsample are bisyllabic words. Most of 
Ayeri's roots are bisyllabic; unsurprisingly, most monosyllabic words are 
function words like the ones cited below. A few examples for each number of 
syllables per word:

\pex
	\a \xayr{yeNF}{yeng}{she} (she.\Aarg{}),\\
		\rayr{le}{le} (\PatT),\\
		\xayr{ru\_a}{rua}{must};
		
	\a \xayr{dtau}{datau}{normal},\\
		\xayr{mreNF}{mareng}{it suffices} (suffice=\TsgI{}.\Aarg{}),\\
		\xayr{nsj}{nasay}{near to};
		
	\a \xayr{AvnFyaaNF}{avanyāng}{he sinks} (sink=\TsgM{}.\Aarg{}),\\
		\xayr{nraanFye}{narānye}{words} (word-\Pl{}),\\
		\xayr{tovlej}{tovaley}{a cloak} (cloak-\PargI{});
		
	\a \rayr{hinYnFveno}{hinyanveno} (corner.beautiful, a place name),\\
		\xayr{mNstoNF}{mangasatong}{they used to move} (move-\Hab{}=\TplN{}.\Aarg{}),\\
		\xayr{mitnen}{mitanena}{of the palace} (palace-\Gen{});
		
	\a \xayr{hruymnsF}{haruyamanas}{beatings} (beat-\Ptcp{}-\Nmlz{}-\Parg{}),\\
		\xayr{sirutyen}{sirutayena}{of the night} (night-\Gen{}),\\
		\xayr{suMkornFkihsF}{sungkorankihas}{geography} (science.map);
		
	\a \xayr{kjtomynen}{kaytomayanena}{of righteousness} (righteous-\Nmlz{}-\Gen{}),\\
		\xayr{koronrYsynF}{koronaryasayan}{they used to forget} (forget-\Hab{}-\TplM{}),\\
		\xayr{nsimyye\_aNF/henF}{nasimayajang-hen}{all followers} (follow-\Agtz-\Pl{}-\Aarg{}=all).
\xe

\begin{sidewaystable}[pth]\centering
\caption[Relative frequency of syllable types per word]{Relative frequency of syllable types per word (n\,=\,5500)}
\begin{tabu} to \textwidth{H X[c] S[c] X[c] S[c] X[c] S[c] X[c] S[c] X[c] S[c]}
\tableheaderfont\toprule
Type
	& \multicolumn2{c}{Initial}
	& \multicolumn2{c}{Medial}
	& \multicolumn2{c}{Final}
	& \multicolumn2{c}{Single}
	& \multicolumn2{c}{Total}
	\\
	
\toprule
	
CV
	& 2904
	& 67.55\pct
	& 1975
	& 72.05\pct
	& 2108
	& 49.03\pct
	& 578
	& 48.13\pct
	& 7565
	& 60.33\pct
	\\
	
CCV
	& 55
	& 1.28\pct
	& 24
	& 0.88\pct
	& 47
	& 1.09\pct
	& 32
	& 2.66\pct
	& 158
	& 1.26\pct
	\\
	
CCCV
	& \multicolumn2{c}{—}
% 	& 0
% 	& 0.00\pct
	& \multicolumn2{c}{—}
% 	& 0
% 	& 0.00\pct
	& 2
	& 0.05\pct
	& \multicolumn2{c}{—}
% 	& 0
% 	& 0.00\pct
	& 2
	& 0.02\pct
	\\
	
CVC
	& 754
	& 17.54\pct
	& 613
	& 22.36\pct
	& 1903
	& 44.27\pct
	& 298
	& 24.81\pct
	& 3568
	& 28.45\pct
	\\
	
CCVC
	& 29
	& 0.67\pct
	& 10
	& 0.36\pct
	& 88
	& 2.05\pct
	& 9
	& 0.75\pct
	& 136
	& 1.08\pct
	\\
	
CVCC
	& 1
	& 0.02\pct
	& \multicolumn2{c}{—}
% 	& 0
% 	& 0.00\pct
	& \multicolumn2{c}{—}
% 	& 0
% 	& 0.00\pct
	& \multicolumn2{c}{—}
% 	& 0
% 	& 0.00\pct
	& 1
	& 0.01\pct
	\\

\midrule

V
	& 488
	& 11.35\pct
	& 95
	& 3.47\pct
	& 67
	& 1.56\pct
	& 2
	& 0.17\pct
	& 652
	& 5.20\pct
	\\
	
VC
	& 68
	& 1.58\pct
	& 24
	& 0.88\pct
	& 84
	& 1.95\pct
	& 282
	& 23.48\pct
	& 458
	& 3.65\pct
	\\
	
\bottomrule
	
Total
	& 4299
	& 100.00\pct
	& 2741
	& 100.00\pct
	& 4299
	& 100.00\pct
	& 1201
	& 100.00\pct
	& 12540
	& 100.00\pct
	\\

\bottomrule
\end{tabu}
\label{tab:syltype}
\end{sidewaystable}

\autoref{tab:syltype} shows the frequencies of syllable types by position in a 
word. It is important to note here that phonemes which consist of more than one 
segment---affricates, diphthongs, and long vowels---have been counted as only 
one of C (consonant) or V (vowel), respectively. The following subsections will 
elaborate on which sounds the Cs and Vs correspond to. Moreover, it is important 
to note that medial syllables have not been further distinguished by position in 
the word for the sake of this analysis, so anything between the second and the 
fifth medial syllable is treated the same. It would furthermore be possible to 
calculate the frequencies of one syllable type following the other, 
however, no such calculations have been performed here.

In all positions, CV is the most common syllable type, followed by CVC. With a 
very big margin, V is the next most common syllable type, which is also most 
common in initial syllables and least common in monosyllabic words. The cases 
with only a few attestations are the following:

\pex
	\a Initial CVCC:\\
		\rayr{silFvFnNF}{silvnang} /sɪlv.ˈnaŋ/ `I see' (see=\Fpl{}.\Aarg{});
		
	\a Final CCCV:\\
		\rayr{migFrFyo}{migryo} /ˈmi.grjo/ `flourishes' (flourish-\Tsg{}.\N{}),\\
		\rayr{subFrFyo}{subryo} /ˈsu.brjo/ `ceases' (cease-\Tsg{}.\N{});
	
	\a Single V:\\
		\rayr{Aj}{ay} /aɪ/ `I' (\Fsg{}.\Top{}).
\xe

The medial and final VC cases may seem like an oddity, but they are mostly due 
to the previous syllable ending in /ŋ/, with that syllable also containing a 
lax vowel, which means that this syllable must be closed. An alternative 
explanation would be to assume that /ŋ/ is ambisyllabic, or actually /n.g 
\textasciitilde{} ŋ.g/, but realized as [ŋ]. The high number of single-syllable 
VC is due to \xayr{ANF}{ang}{\AgtT}, which alone appears 255 times in the 
sample (4.63\pct{} of all words, 21.23\pct{} of monosyllabic words, 90.43\pct{} 
of monosyllabic VC words).

\subsection{Phonemic Makeup of Initial Syllables}

\begin{table}[pth]\centering
\caption[Relative frequency of onsets in initial syllables]{Relative frequency of onsets in initial syllables (n\,=\,4299)}
\begin{tabu} to 0.5\textwidth{X X[c] X[c]}
\tableheaderfont\toprule
Phoneme
	& Frequency
	& Percentage
	\\
	
\toprule

Ø
	& 556
	& 12.93\pct
	\\

\midrule

s
	& 488
	& 11.35\pct
	\\

t
	& 432
	& 10.05\pct
	\\

m
	& 418
	& 9.72\pct
	\\

k
	& 380
	& 8.84\pct
	\\

n
	& 375
	& 8.72\pct
	\\

p
	& 334
	& 7.77\pct
	\\

b
	& 231
	& 5.37\pct
	\\

d
	& 172
	& 4.00\pct
	\\

v
	& 164
	& 3.81\pct
	\\

l
	& 159
	& 3.70\pct
	\\

r
	& 134
	& 3.12\pct
	\\

j
	& 126
	& 2.93\pct
	\\

g
	& 111
	& 2.58\pct
	\\

h
	& 99
	& 2.30\pct
	\\

tʃ
	& 30
	& 0.70\pct
	\\

pr
	& 27
	& 0.63\pct
	\\

nj
	& 27
	& 0.63\pct
	\\

kr
	& 8
	& 0.19\pct
	\\

br
	& 8
	& 0.19\pct
	\\

tr
	& 6
	& 0.14\pct
	\\

dʒ
	& 4
	& 0.09\pct
	\\

gr
	& 3
	& 0.07\pct
	\\

w
	& 2
	& 0.05\pct
	\\

sw
	& 1
	& 0.02\pct
	\\

rw
	& 1
	& 0.02\pct
	\\

pj
	& 1
	& 0.02\pct
	\\

mj
	& 1
	& 0.02\pct
	\\

bw
	& 1
	& 0.02\pct
	\\

\bottomrule
\end{tabu}
\label{tab:initon}
\end{table}

The statistics in the following sections have been gathered from the IPA 
conversions of translated texts and dictionary entries mentioned above. The 
transcribed words have been split into syllables and then the collected contents 
of each position group were written into separate plain text files, one each for:

\begin{itemize}
	\item all initial syllables of polysyllabic words,
	\item all medial syllables of polysyllabic words,
	\item all final syllables of polysyllabic words, and 
	\item all monosyllabic words.
\end{itemize}

Monosyllabic words are both initial and final syllables at the same time; they 
have been counted separately for the purpose of this analysis. Onsets, nuclei 
and codas have been matched by regular expressions; the com\-mand line tools 
\texttt{grep}, \texttt{sort}, and \texttt{uniq} were used to aggregate all 
occurring variants for each syllable segment as well as their absolute 
frequencies:\footnote{However, \texttt{sort} was unable to handle all IPA 
characters, so \texttt{sed 'y/ɛɪɔʊəːʃʒŋ/EIOU@:SZN/'} had to be used to 
compensate by transcribing everything into X-SAMPA.}

\ex
	\texttt{C = (?:tʃ|dʒ|[ptkbdgmnŋvshrljw])\\
	V = (?:[aeɔʊ]ːɪ|[aeɔʊ]ɪ|aʊ|[ieaou]ː|[ieaouɪɛɔʊə])}
\xe

As we have seen above (\autoref{tab:syltype}), CCV syllables only make up 
1.28\pct{} of initial syllables, in so far it is no surprise that consonant 
clusters all appear at the bottom of \autoref{tab:initon}. There also seem to 
be combination patterns in that initial clusters exist for all plosives plus /r/, 
and almost all bilabials plus /j/, with the exception of /bj/, however, /nj/ is 
added to the group instead. Combinations with /w/ only occur for /b/, /r/, and 
/s/, which do not share an obvious connection. Syllables without a consonant 
filling the onset position are marked with \enquote*{Ø}; these numbers 
correspond to the VC and VCC rows in \autoref{tab:syltype}.

\begin{table}[pth]\centering
\caption[Relative frequency of nuclei in initial syllables]{Relative frequency of nuclei in initial syllables (n\,=\,4299)}
\begin{tabu} to 0.5\textwidth{X X[c] X[c]}
\tableheaderfont\toprule
Phoneme
	& Frequency
	& Percentage
	\\
	
\toprule

a
	& 1847
	& 42.96\pct
	\\

\midrule

i
	& 1011
	& 23.52\pct
	\\

\rowfont{\scriptsize\itshape}
\raggedleft
i
	& 802
	& 18.66\pct
	\\

\rowfont{\scriptsize\itshape}
\raggedleft
ɪ
	& 209
	& 4.86\pct
	\\

\midrule

e
	& 705
	& 16.40\pct
	\\

\rowfont{\scriptsize\itshape}
\raggedleft
e
	& 523
	& 12.17\pct
	\\

\rowfont{\scriptsize\itshape}
\raggedleft
ɛ
	& 164
	& 3.81\pct
	\\

\rowfont{\scriptsize\itshape}
\raggedleft
ə
	& 18
	& 0.42\pct
	\\

\midrule

u
	& 260
	& 6.05\pct
	\\

\rowfont{\scriptsize\itshape}
\raggedleft
u
	& 227
	& 5.28\pct
	\\

\rowfont{\scriptsize\itshape}
\raggedleft
ʊ
	& 33
	& 0.77\pct
	\\

\midrule

o
	& 227
	& 5.28\pct
	\\

\rowfont{\scriptsize\itshape}
\raggedleft
o
	& 188
	& 4.37\pct
	\\

\rowfont{\scriptsize\itshape}
\raggedleft
ɔ
	& 39
	& 0.91\pct
	\\

\midrule

aː
	& 109
	& 2.54\pct
	\\

aɪ
	& 88
	& 2.05\pct
	\\

eɪ
	& 40
	& 0.93\pct
	\\

eː
	& 4
	& 0.09\pct
	\\

ɔɪ
	& 3
	& 0.07\pct
	\\

ʊɪ
	& 1
	& 0.02\pct
	\\

oː
	& 1
	& 0.02\pct
	\\

iː
	& 1
	& 0.02\pct
	\\

eːɪ
	& 1
	& 0.02\pct
	\\

aʊ
	& 1
	& 0.02\pct
	\\

\bottomrule
\end{tabu}
\label{tab:initnuc}
\end{table}

Perhaps most striking about the nuclei of initial syllables presented in 
\autoref{tab:initnuc} is that it is plain vowels which occur most of the time. 
As mentioned above, lax vowels are counted here as allophones of tense ones as 
their distribution is complementary, which is why the plain vowels are 
presented as grouped. Long vowels and diphthongs find themselves below the 
5\pct{} threshold, and the words with single occurrences are:

\pex
	\a \xayr{kujsaanF}{kuysān}{comparison},
	\a \xayr{noonF}{nōn}{will, intention},
	\a \xayr{niis}{nīsa}{wanted},\footnotemark
	\a \xayr{seejry}{sēyraya}{will overcome} (\Fut{}-overcome-\Tsg{}.\M{}),
	\a \xayr{sautnF}{sautan}{cork}.
\xe
\footnotetext{\rayr{niis}{nīsa} and \rayr{noonF}{nōn} are both related to 
\xayr{no/}{no-}{want, plan}.}

As [eːɪ] only occurs due to allophony, it should not be counted as a phoneme for
the purposes of this analysis. On the other hand, the same could be said for a 
lot of cases of [aː] included here---this caveat applies to all nouns derived
from verbs ending in \textit{-a} with the very common nominalizing suffix 
\rayr{/AnF}{-an}, as exemplified in (\ref{ex:longvwls}) above. Similarly, the 
18 cases of /ə/ reported here are mostly from tense prefixes also mentioned 
above, for instance, \xayr{mkoronj}{məkoronay}{I knew} 
(\Pst{}-know=\Fsg{}.\Top{}).

\begin{table}[pth]\centering
\caption[Relative frequency of codas in initial syllables]{Relative frequency of codas in initial syllables (n\,=\,4299)}
\begin{tabu} to 0.5\textwidth{X X[c] X[c]}
\tableheaderfont\toprule
Phoneme
	& Frequency
	& Percentage
	\\
	
\toprule

Ø
	& 3447
	& 80.18\pct
	\\

\midrule

n
	& 299
	& 6.96\pct
	\\

ŋ
	& 236
	& 5.49\pct
	\\

r
	& 129
	& 3.00\pct
	\\

l
	& 89
	& 2.07\pct
	\\

m
	& 74
	& 1.72\pct
	\\

s
	& 20
	& 0.47\pct
	\\

h
	& 2
	& 0.05\pct
	\\

t
	& 1
	& 0.02\pct
	\\

lv
	& 1
	& 0.02\pct
	\\

k
	& 1
	& 0.02\pct
	\\

\bottomrule
\end{tabu}
\label{tab:initcod}
\end{table}

Initial-syllable codas (\autoref{tab:initcod}) are far less diverse than 
consonant onsets: there are only 10 attested segments in comparison to 28 for 
onsets (not counting empty codas of C(C)V syllables, which constitute the 
majority by a large margin), and the only cluster attested is /lv/ in the word 
\rayr{silFvFnNF}{silvnang} `I see' (see=\Fpl{}.\Aarg{}). There only being one 
such incidence of a CC cluster is very probably an effect of the small sample 
size. Furthermore, the only unvoiced single coda consonants attested are /s/, 
/h/, /t/, and /k/, the latter two only once, /h/ twice:

\pex
	\a \xayr{mehFvaaNF}{mehvāng}{you are supposed to} 
		(be.supposed.to=\Ssg{}.\Aarg{}),\footnotemark\\
		\xayr{rohFtaNF}{rohtang}{they bite} (bite=\TsgM{}.\Aarg{});
	\a \xayr{ptFlj}{patlay}{cousin};
	\a \xayr{sikF/sikF}{sik-sik}{tit}.
\xe
\footnotetext{The dictionary entry is \rayr{mY/}{mya-}, so this may be an 
instance of my changing a word in the dictionary with the old one staying in 
the text.}

\subsection{Phonemic Makeup of Medial Syllables}

\begin{table}[pth]\centering
\caption[Relative frequency of onsets in medial syllables]{Relative frequency of onsets in medial syllables (n\,=\,2741)}
\begin{tabu} to 0.5\textwidth{X X[c] X[c]}
\tableheaderfont\toprule
Phoneme
	& Frequency
	& Percentage
	\\
	
\toprule

Ø
	& 119
	& 4.34\pct
	\\

\midrule

r
	& 343
	& 12.51\pct
	\\

n
	& 260
	& 9.49\pct
	\\

j
	& 233
	& 8.50\pct
	\\

t
	& 222
	& 8.10\pct
	\\

d
	& 213
	& 7.77\pct
	\\

k
	& 189
	& 6.90\pct
	\\

s
	& 170
	& 6.20\pct
	\\

m
	& 169
	& 6.17\pct
	\\

l
	& 149
	& 5.44\pct
	\\

v
	& 148
	& 5.40\pct
	\\

h
	& 147
	& 5.36\pct
	\\

p
	& 119
	& 4.34\pct
	\\

g
	& 92
	& 3.36\pct
	\\

b
	& 89
	& 3.25\pct
	\\

tʃ
	& 20
	& 0.73\pct
	\\

dʒ
	& 15
	& 0.55\pct
	\\

tr
	& 11
	& 0.40\pct
	\\

dr
	& 8
	& 0.29\pct
	\\

pr
	& 7
	& 0.26\pct
	\\

w
	& 6
	& 0.22\pct
	\\

ŋ
	& 4
	& 0.15\pct
	\\

sj
	& 2
	& 0.07\pct
	\\

br
	& 2
	& 0.07\pct
	\\

sw
	& 1
	& 0.04\pct
	\\

kw
	& 1
	& 0.04\pct
	\\

kj
	& 1
	& 0.04\pct
	\\

bj
	& 1
	& 0.04\pct
	\\

\bottomrule
\end{tabu}
\label{tab:midon}
\end{table}

...

\begin{table}[pth]\centering
\caption[Relative frequency of nuclei in medial syllables]{Relative frequency of nuclei in medial syllables (n\,=\,2741)}
\begin{tabu} to 0.5\textwidth{X X[c] X[c]}
\tableheaderfont\toprule
Phoneme
	& Frequency
	& Percentage
	\\
	
\toprule

a
	& 1480
	& 53.99\pct
	\\

\midrule

i
	& 480
	& 17.51\pct
	\\

\rowfont{\scriptsize\itshape}
\raggedleft
i
	& 387
	& 14.12\pct
	\\

\rowfont{\scriptsize\itshape}
\raggedleft
ɪ
	& 93
	& 3.39\pct
	\\

\midrule

e
	& 254
	& 9.26\pct
	\\

\rowfont{\scriptsize\itshape}
\raggedleft
e
	& 206
	& 7.52\pct
	\\

\rowfont{\scriptsize\itshape}
\raggedleft
ɛ
	& 48
	& 1.75\pct
	\\

\midrule

o
	& 194
	& 7.08\pct
	\\

\rowfont{\scriptsize\itshape}
\raggedleft
o
	& 119
	& 4.34\pct
	\\

\rowfont{\scriptsize\itshape}
\raggedleft
ɔ
	& 75
	& 2.74\pct
	\\

\midrule

u
	& 120
	& 4.38\pct
	\\

\rowfont{\scriptsize\itshape}
\raggedleft
u
	& 102
	& 3.72\pct
	\\

\rowfont{\scriptsize\itshape}
\raggedleft
ʊ
	& 18
	& 0.66\pct
	\\

\midrule

aː
	& 110
	& 4.01\pct
	\\

aɪ
	& 51
	& 1.86\pct
	\\

ɔɪ
	& 33
	& 1.20\pct
	\\

eɪ
	& 5
	& 0.18\pct
	\\

eː
	& 5
	& 0.18\pct
	\\

aʊ
	& 5
	& 0.18\pct
	\\

ʊɪ
	& 2
	& 0.07\pct
	\\

uː
	& 1
	& 0.04\pct
	\\

iː
	& 1
	& 0.04\pct
	\\

\bottomrule
\end{tabu}
\label{tab:midnuc}
\end{table}

...

\begin{table}[pth]\centering
\caption[Relative frequency of codas in medial syllables]{Relative frequency of codas in medial syllables (n\,=\,2741)}
\begin{tabu} to 0.5\textwidth{X X[c] X[c]}
\tableheaderfont\toprule
Phoneme
	& Frequency
	& Percentage
	\\
	
\toprule

Ø
	& 2094
	& 76.40\pct
	\\

\midrule

n
	& 313
	& 11.42\pct
	\\

ŋ
	& 193
	& 7.04\pct
	\\

r
	& 48
	& 1.75\pct
	\\

m
	& 39
	& 1.42\pct
	\\

s
	& 32
	& 1.17\pct
	\\

l
	& 21
	& 0.77\pct
	\\

t
	& 1
	& 0.04\pct
	\\

\bottomrule
\end{tabu}
\label{tab:midcod}
\end{table}

...

\subsection{Phonemic Makeup of Final Syllables}

\begin{table}[pth]\centering
\caption[Relative frequency of onsets in final syllables]{Relative frequency of onsets in final syllables (n\,=\,4299)}
\begin{tabu} to \textwidth{X X[c] X[c] X X[c] X[c]}
\tableheaderfont\toprule
Phoneme
	& Frequency
	& Percentage
	& Phoneme
	& Frequency
	& Percentage
	\\
	
\toprule

Ø
	& 151
	& 3.51\pct
	& hj
	& 5
	& 0.12\pct
	\\

\cmidrule{1-3}

j
	& 1101
	& 25.61\pct
	& bj
	& 5
	& 0.12\pct
	\\

n
	& 528
	& 12.28\pct
	& tw
	& 4
	& 0.09\pct
	\\

r
	& 398
	& 9.26\pct
	& sw
	& 4
	& 0.09\pct
	\\

t
	& 266
	& 6.19\pct
	& sj
	& 4
	& 0.09\pct
	\\

s
	& 244
	& 5.68\pct
	& ŋ
	& 4
	& 0.09\pct
	\\

l
	& 238
	& 5.54\pct
	& kw
	& 3
	& 0.07\pct
	\\

k
	& 199
	& 4.63\pct
	& kr
	& 3
	& 0.07\pct
	\\

d
	& 184
	& 4.28\pct
	& br
	& 3
	& 0.07\pct
	\\

m
	& 154
	& 3.58\pct
	& vr
	& 2
	& 0.05\pct
	\\

v
	& 142
	& 3.30\pct
	& rw
	& 2
	& 0.05\pct
	\\

h
	& 128
	& 2.98\pct
	& nw
	& 2
	& 0.05\pct
	\\

p
	& 115
	& 2.68\pct
	& tv
	& 1
	& 0.02\pct
	\\

g
	& 103
	& 2.40\pct
	& tʃv
	& 1
	& 0.02\pct
	\\

dʒ
	& 73
	& 1.70\pct
	& tʃj
	& 1
	& 0.02\pct
	\\

b
	& 73
	& 1.70\pct
	& rj
	& 1
	& 0.02\pct
	\\

tʃ
	& 52
	& 1.21\pct
	& nj
	& 1
	& 0.02\pct
	\\

vj
	& 26
	& 0.60\pct
	& mw
	& 1
	& 0.02\pct
	\\

pj
	& 22
	& 0.51\pct
	& kt
	& 1
	& 0.02\pct
	\\

dʒj
	& 17
	& 0.40\pct
	& gt
	& 1
	& 0.02\pct
	\\

tr
	& 10
	& 0.23\pct
	& grj
	& 1
	& 0.02\pct
	\\

w
	& 9
	& 0.21\pct
	& dv
	& 1
	& 0.02\pct
	\\

pr
	& 7
	& 0.16\pct
	& dr
	& 1
	& 0.02\pct
	\\

kj
	& 6
	& 0.14\pct
	& brj
	& 1
	& 0.02\pct
	\\

\bottomrule
\end{tabu}
\label{tab:finon}
\end{table}

...

\begin{table}[pth]\centering
\caption[Relative frequency of nuclei in final syllables]{Relative frequency of nuclei in final syllables (n\,=\,4299)}
\begin{tabu} to 0.5\textwidth{X X[c] X[c]}
\tableheaderfont\toprule
Phoneme
	& Frequency
	& Percentage
	\\
	
\toprule

a
	& 2408
	& 56.01\pct
	\\

aː
	& 316
	& 7.35\pct
	\\

\midrule

o
	& 411
	& 9.56\pct
	\\

\rowfont{\scriptsize\itshape}
\raggedleft
o
	& 298
	& 6.93\pct
	\\

\rowfont{\scriptsize\itshape}
\raggedleft
ɔ
	& 113
	& 2.63\pct
	\\

\midrule

i
	& 289
	& 6.42\pct
	\\

\rowfont{\scriptsize\itshape}
\raggedleft
ɪ
	& 147
	& 3.42\pct
	\\

\rowfont{\scriptsize\itshape}
\raggedleft
i
	& 142
	& 3.30\pct
	\\

\midrule

aɪ
	& 254
	& 5.91\pct
	\\

\midrule

u
	& 207
	& 4.82\pct
	\\

\rowfont{\scriptsize\itshape}
\raggedleft
u
	& 155
	& 3.61\pct
	\\

\rowfont{\scriptsize\itshape}
\raggedleft
ʊ
	& 52
	& 1.21\pct
	\\

\midrule

e
	& 209
	& 4.85\pct
	\\

\rowfont{\scriptsize\itshape}
\raggedleft
ɛ
	& 127
	& 2.95\pct
	\\

\rowfont{\scriptsize\itshape}
\raggedleft
ə
	& 81
	& 1.88\pct
	\\

\rowfont{\scriptsize\itshape}
\raggedleft
e
	& 1
	& 0.02\pct
	\\

\midrule

eɪ
	& 103
	& 2.40\pct
	\\

ɔɪ
	& 42
	& 0.98\pct
	\\

aːɪ
	& 23
	& 0.54\pct
	\\

ʊɪ
	& 14
	& 0.33\pct
	\\

aʊ
	& 14
	& 0.33\pct
	\\

eː
	& 5
	& 0.12\pct
	\\

iː
	& 3
	& 0.07\pct
	\\

uː
	& 1
	& 0.02\pct
	\\

\bottomrule
\end{tabu}
\label{tab:finnuc}
\end{table}

...

\begin{table}[pth]\centering
\caption[Relative frequency of codas in final syllables]{Relative frequency of codas in final syllables (n\,=\,4299)}
\begin{tabu} to 0.5\textwidth{X X[c] X[c]}
\tableheaderfont\toprule
Phoneme
	& Frequency
	& Percentage
	\\
	
\toprule

Ø
	& 2224
	& 51.73\pct
	\\

\midrule

n
	& 899
	& 20.91\pct
	\\

ŋ
	& 651
	& 15.14\pct
	\\

s
	& 244
	& 5.68\pct
	\\

m
	& 225
	& 5.23\pct
	\\

l
	& 34
	& 0.79\pct
	\\

r
	& 21
	& 0.49\pct
	\\

k
	& 1
	& 0.02\pct
	\\

\bottomrule
\end{tabu}
\label{tab:fincod}
\end{table}

...

\subsection{Phonemic Makeup of Single Syllables}

\begin{table}[pth]\centering
\caption[Relative frequency of onsets in single syllables]{Relative frequency of onsets in single syllables (n\,=\,1201)}
\begin{tabu} to 0.5\textwidth{X X[c] X[c]}
\tableheaderfont\toprule
Phoneme
	& Frequency
	& Percentage
	\\
	
\toprule

Ø
	& 284
	& 23.65\pct
	\\

\midrule

n
	& 231
	& 19.23\pct
	\\

s
	& 147
	& 12.24\pct
	\\

j
	& 144
	& 11.99\pct
	\\

k
	& 51
	& 4.25\pct
	\\

v
	& 48
	& 4.00\pct
	\\

m
	& 46
	& 3.83\pct
	\\

l
	& 44
	& 3.66\pct
	\\

t
	& 41
	& 3.41\pct
	\\

d
	& 33
	& 2.75\pct
	\\

r
	& 26
	& 2.16\pct
	\\

h
	& 23
	& 1.92\pct
	\\

mj
	& 16
	& 1.33\pct
	\\

p
	& 13
	& 1.08\pct
	\\

tʃ
	& 9
	& 0.75\pct
	\\

g
	& 9
	& 0.75\pct
	\\

nj
	& 8
	& 0.67\pct
	\\

rw
	& 7
	& 0.58\pct
	\\

b
	& 7
	& 0.58\pct
	\\

pr
	& 5
	& 0.42\pct
	\\

dʒ
	& 3
	& 0.25\pct
	\\

tr
	& 2
	& 0.17\pct
	\\

nw
	& 1
	& 0.08\pct
	\\

ŋ
	& 1
	& 0.08\pct
	\\

kr
	& 1
	& 0.08\pct
	\\

br
	& 1
	& 0.08\pct
	\\

\bottomrule
\end{tabu}
\label{tab:singon}
\end{table}

...

\begin{table}[pth]\centering
\caption[Relative frequency of nuclei in single syllables]{Relative frequency of nuclei in single syllables (n\,=\,1201)}
\begin{tabu} to 0.5\textwidth{X X[c] X[c]}
\tableheaderfont\toprule
Phoneme
	& Frequency
	& Percentage
	\\
	
\toprule

a
	& 568
	& 47.29\pct
	\\

aɪ
	& 171
	& 14.24\pct
	\\

aː
	& 140
	& 11.66\pct
	\\

\midrule

i
	& 113
	& 9.41\pct
	\\

\rowfont{\scriptsize\itshape}
\raggedleft
i
	& 65
	& 5.41\pct
	\\

\rowfont{\scriptsize\itshape}
\raggedleft
ɪ
	& 48
	& 4.00\pct
	\\

\midrule

e
	& 104
	& 8.66\pct
	\\

\rowfont{\scriptsize\itshape}
\raggedleft
ɛ
	& 65
	& 5.41\pct
	\\

\rowfont{\scriptsize\itshape}
\raggedleft
e
	& 34
	& 2.83\pct
	\\

\rowfont{\scriptsize\itshape}
\raggedleft
ə
	& 5
	& 0.42\pct
	\\

\midrule

o
	& 45
	& 3.75\pct
	\\

\rowfont{\scriptsize\itshape}
\raggedleft
ɔ
	& 30
	& 2.50\pct
	\\

\rowfont{\scriptsize\itshape}
\raggedleft
o
	& 15
	& 1.25\pct
	\\

\midrule

u
	& 20
	& 1.67\pct
	\\

aːɪ
	& 14
	& 1.17\pct
	\\

ɔɪ
	& 10
	& 0.83\pct
	\\

iː
	& 6
	& 0.50\pct
	\\

eɪ
	& 5
	& 0.42\pct
	\\

ʊɪ
	& 3
	& 0.25\pct
	\\

oː
	& 2
	& 0.17\pct
	\\

\bottomrule
\end{tabu}
\label{tab:singnuc}
\end{table}

...

\begin{table}[pth]\centering
\caption[Relative frequency of codas in single syllables]{Relative frequency of codas in single syllables (n\,=\,1201)}
\begin{tabu} to 0.5\textwidth{X X[c] X[c]}
\tableheaderfont\toprule
Phoneme
	& Frequency
	& Percentage
	\\
	
\toprule

Ø
	& 612
	& 50.96\pct\\

\midrule

ŋ
	& 377
	& 31.39\pct\\
n
	& 105
	& 8.74\pct\\
s
	& 58
	& 4.83\pct\\
m
	& 36
	& 3.00\pct\\
l
	& 6
	& 0.50\pct\\
h
	& 4
	& 0.33\pct\\
r
	& 3
	& 0.25\pct\\

\bottomrule
\end{tabu}
\label{tab:singcod}
\end{table}

...
