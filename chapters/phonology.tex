\chapter{Phonology}

This chapter will present charts depicting the phoneme inventory of Ayeri and 
describe the various commonly encountered allophones of both consonants and 
vowels. Following this, a detailed statistical analysis of the words found in a 
number of translated texts from 2008 to 2016 as well as dictionary entries up 
to July 2016 will produce insights into Ayeri's phonotactics. Some notes on 
stress patterns and intonation will close the chapter.

\section{Phoneme Inventory}

\subsection{Consonants}

\index{consonants|(}%
\begin{sidewaysfigure}[p]
\caption[Consonant inventory]{Consonant inventory (divergent orthography in pointed brackets)}
\begin{tabu} to \linewidth {H[2l] X[c] X[c] X[c] X[c] X[c] X[c] X[c] X[c] X[c] X[c] X[c] X[c]}
\toprule\tableheaderfont
	%
	& \multicolumn2{c}{Bilabials}
	& \multicolumn2{c}{Labiodentals}
	& \multicolumn2{c}{Alveolars}
	& \multicolumn2{c}{Palatals}
	& \multicolumn2{c}{Velars}
	& \multicolumn2{c}{Glottals}
	\\

\midrule

Plosives
	& p & b	% Bilabials
	&   &  	% Labiodentals
	& t & d	% Alveolars
	&   &  	% Palatals
	& k & g	% Velars
	&   &  	% Glottals
	\\

\midrule

Affricates
	&             &            	% Bilabials
	&             &            	% Labiodentals
	& tʃ \orth{c} & dʒ \orth{j}	% Alveolars
	&             &            	% Palatals
	&             &            	% Glottals
	\\

\midrule

Nasals
	&   & m          	% Bilabials
	&   &            	% Labiodentals
	&   & n          	% Alveolars
	&   &            	% Palatals
	&   & ŋ \orth{ng}	% Velars
	&   &            	% Glottals
	\\

\midrule

Fricatives
	&   &  	% Bilabials
	&   & v	% Labiodentals
	& s &  	% Alveolars
	&   &  	% Palatals
	&   &  	% Velars
	& h &  	% Glottals
	\\

\midrule

Taps/Flaps
	&   &  	% Bilabials
	&   &  	% Labiodentals
	&   & r	% Alveolars
	&   &  	% Palatals
	&   &  	% Velars
	&   &  	% Glottals
	\\

\midrule

Approximants
	&   &           	% Bilabials
	&   &           	% Labiodentals
	&   & l         	% Alveolars
	&   & j \orth{y}	% Palatals
	&   &           	% Velars
	&   &           	% Glottals
	\\

\bottomrule
\end{tabu}
\label{fig:consonants}
\end{sidewaysfigure}

At 17 consonants, Ayeri has a \textquote{moderately small} inventory, according 
to \citet{wals1}. \autoref{fig:consonants} shows the full chart of consonant 
phonemes.

\index{allophony!consonants|(}%
Regarding allophony, /tj kj/ and /dj gj/ are usually realized as [tʃ] and [dʒ], 
respecitively, except if a homorganic nasal /n/ or /ŋ/ is preceding: for 
instance, \rayr{AMkYu}{ankyu} /ˈaŋkju/ `really' is realized as [ˈaŋkju], not as 
*[ˈaŋtʃu] or *[ˈantʃu]. It is important to note, however, that besides this 
synchronic palatalization process leading to [tʃ] and [dʒ] as \emph{allophones}, 
there is also a diachronic one in parallel here---or the diachronic process is 
still ongoing. For instance, there is no way to predict whether 
\xayr{kYun/}{cuna}{original, initial}, \xayr{pMtY}{panca}{finally, eventually}, 
and \xayr{vtY/}{vac-}{like}, or \xayr{dYrnF}{jaran}{pilgrimage}, 
\xayr{AgY/}{aja-}{play}, and \xayr{nudY/}{nuj-}{pour} have /tj/ or /kj/, /dj/ 
or /gj/, respectively, unless we consider the clues given by the conservative 
native spellings of the respective words.\footnote{Actual scribes would 
typically err in cases where the merger is complete, so this strategy would, in 
fact, be of limited use in the real world.} We can rather assume two sound 
changes, (1) tj, kj → tʃ, and (2) dj, gj → dʒ, leading to the \emph{phonemes} 
/tʃ/ and /dʒ/ in the present-day language.

The plural marker \rayr{/ye}{-ye} is commonly contracted to [dʒ] when a case 
suffix beginning with a vowel follows:\footnote{The customary romanization uses 
\orth{c} and \orth{j} for allophonic cases of [tʃ] and [dʒ] as well.}

\pex
	\a \rayr{nYaanFye\_aNF}{nyān\textbf{ye}ang → nyān\textbf{j}ang} [ˈnjaːndʒaŋ] `persons' (person-\Pl{}-\Aarg{});
	\a \rayr{netuye\_asF}{netu\textbf{ye}as → netu\textbf{j}as} [neˈtudʒas] `brothers' (brother-\Pl{}-\Parg{}).
\xe

The plural marker may also contract before the locative marker \rayr{/y}{-ya}
and the dative marker \rayr{/ymF}{-yam}, basically for dissimilation:%
\footnote{\rayr{/E\_a}{-ea} also occurs as an variant morpheme, so that 
\rayr{/ye}{-ye} + \rayr{/E\_a}{-ea} → \rayr{/yee\_a}{-yēa}.}

\pex
	\a \rayr{nivyey}{niva\textbf{ye}ya → niva\textbf{j}ya} [niˈvadʒja] `at the eyes' (eye-\Pl{}-\Loc{});
	\a \rayr{mviyeymF}{mavi\textbf{ye}yam → mavi\textbf{j}yam} [maˈvidʒjam] `to the sheep' (sheep-\Pl{}-\Dat{}).
\xe

Dissimilation of the sequence \rayr{/yy}{-yaya} is attested in my translation 
of Kafka's short story ``Eine kaiserliche Botschaft,'' where the relative 
pronoun \rayr{siyy}{siyaya} appears transcribed as \textit{sijya}:

\blockcquote[12]{becker:kafka:imperial}{As far as morphophonology\index{morphophonology} is concerned, 
the relative pronoun complex \textit{sijya} `in/at/on which.\Loc{}' is 
interesting in so far as it is a contraction of \textit{*siyaya} 
`\Rel{}-\Loc{}-\Loc{}' that I introduced here [...] Since this feature does not 
occur in previous texts, let's assume it's an acceptable variant.}

\noindent The contraction happens \textcquote[12]{becker:kafka:imperial}{only 
if both parts are grammatical suffixes}, however, so the environments this 
contraction may appear in are effectively limited to relative pronouns 
combining locative and locative, or locative and dative marking.

Lastly, /h/ may be realized as [ç] before front vowels, and as [x] before 
back vowels:

\pex
	\a \rayr{thi}{ta\textbf{hi}} [ˈtaçi] `favorable';
	\a \rayr{baho}{ba\textbf{ho}} [ˈbaxo] `loud'.
\xe

While vowels become long when two identical vowels come into succession,  
consonants do not geminate but are treated like a single consonant:

\pex
	\a \rayr{tvFvaaNF}{ta\textbf{vv}āng} [taˈvaːŋ] `you get' (get=\Ssg{}.\Aarg{}),
	\a \rayr{diʲsyNF}{dis\textbf{yy}ang} [diˈsjaŋ] `I fasten' (fasten=\Fsg{}.\Aarg{}).
\xe

With diphthongs\index{diphthongs}, the sequence /Vɪ.j/ is treated as though it were /Vj.j/, so 
the double /j/ simplifies to just a single /j/; however, the vowel remains lax 
in spite of being phonetically in an open position now:

\ex
	\rayr{tipujy}{tip\textbf{uyy}a} [tiˈpʊ.ja] `on the grass' (grass-\Loc{}).
\xe

\index{allophony!consonants|)}%
\index{consonants|)}%

\subsection{Vowels}

\index{vowels|(}%
\begin{figure}[ht]\centering
\caption[Vowel inventory]{Vowel inventory (divergent orthography in pointed brackets)}
\begin{tabu} to .5\linewidth{H[1] X[2c] X[2c] X[2c]}
\toprule\tableheaderfont

	& Front
	& Center
	& Back
	\\

\toprule

High
	& i, iː \orth{ī}
	&
	& u, uː \orth{ū}
	\\

Mid
	& e, eː \orth{ē}
	& (ə \orth{ə, e})
	& o, oː \orth{ō}
	\\

Back
	&
	& a, aː \orth{ā}
	&
	\\

\bottomrule
\end{tabu}
\label{fig:vowels}
\end{figure}

Ayeri's vowel system distinguishes five qualities, as shown in 
\autoref{fig:vowels}; \citet{wals2} classifies this as \textquote[][.]{average}
Length, however, is also a factor, and there are five diphthongs as well, as we 
will see below. The consonant--vowel ratio is 4.25, which \citet{wals3} also 
classifies as \textquote[][,]{average} although Ayeri finds itself at the upper 
end of the tier.

\index{allophony!vowels|(}%
The lax vowels [ɪ ɛ ɔ ʊ] occur as allophones of their tense counterparts 
/i e o u/ in closed syllables, for example:

\pex
	\a \rayr{miNF}{m\textbf{ing}} [mɪŋ] `can, be able',
	\a \rayr{EnFy}{\textbf{en}ya} [ˈɛnja] `everyone',
	\a \rayr{AgonF}{ag\textbf{on}} [ˈagɔn] `outer, foreign', and
	\a \rayr{pkurF}{pak\textbf{ur}} [ˈpakʊr] `ill, sick'.
\xe

/ə/ is a marginal phoneme and only occurs in the tense prefixes 
\xayr{k/}{kə-}{\NPst{}}, \xayr{m/}{mə-}{\Pst{}}, \xayr{v/}{və-}{\RPst{}}, as 
well as in the prefix \xayr{me/}{mə-}{some, whichever}. Otherwise, [ə] occurs 
as an allophone of /e/ in final unstressed position, for instance, in the word 
\rayr{mine}{min\textbf{e}} [ˈminə] `affair, matter, issue'.

Ayeri also possesses a number of diphthongs\index{diphthongs}, these are: /aɪ eɪ ɔɪ ʊɪ aʊ/, 
spelled \orth{ay}, \orth{ey}, \orth{oy}, \orth{uy}, and \orth{au}.
Furthermore, there are long equivalents of the short vowels: /iː eː aː oː uː/. 
Long vowels are lexicalized in a few words, for example:

\pex
	\a \xayr{niis}{nīsa}{wanted}, \xayr{psiis}{pasīsa}{interesting};
	\a \xayr{AreenF}{arēn}{anyway, however}, \xayr{leer}{lēra}{whore};
	\a \xayr{laa}{lā}{tongue}, \xayr{yaaNF}{yāng}{he} (he.\Aarg{}); \label{ex:laa}
	\a \xayr{noonF}{nōn}{will, intention}; and 
	\a \xayr{bbuu\_anF}{babūan}{barbarian}.\footnotemark
\xe

\footnotetext{I have gone years without /uː/, but it has always seemed slightly 
odd to me to lack a vowel in that position when all other vowels can be long. 
Therefore, \xayr{bbuu\_anF}{babūan}{barbarian} and its adjective 
\xayr{bbuu}{babū}{barbarian (adj.)} were coined as \rayr{pFrMkye}{prankaye}---%
things `that you put in specifically to make things fit', another new coining 
this decision resulted in.\label{fn:ū}}

\noindent Otherwise, long vowels result from two same vowels next to each other, 
for instance:

\ex
	\xayr{AgY/}{aja-}{play} + \xayr{/AnF}{-an}{\Nmlz{}} → 
	\xayr{AgYaanF}{ajān}{game, play}.\label{ex:longvwls}
\xe

\phantomsection
Morphophonologically\index{morphophonology}, long vowels also occur in double-marked relative pronouns 
where the agreement marker for the relative clause's head has been omitted,
for instance, \xayr{sinaa}{sinā}{of which, about which}, as in the following 
example:\label{doublerel}

\ex\begingl
	\gla Le turayāng taman sinā ang ningay tamala vās. //
	\glb Le tura-yāng taman-Ø si-Ø-na ang ning=ay.Ø tamala vās //
	\glc \PatTI{} send=\Tsg{}.\M{}.\Aarg{} letter-\Top{} \Rel{}-\PatTI{}-\Gen{} \AgtT{} tell=\Fsg{}.\Top{} yesterday \Ssg{}.\Parg{} //
	\glft `The letter which I told you about yesterday, he sent it.' //
\endgl\xe

\noindent This is to disambiguate it from the plain genitive-marked relative pronoun 
\xayr{sin}{sina}{which.\Gen{}}:\footnote{A variant which combines the 
allomorphs of the relativizer and the genitive case marker in the opposite way 
also exists: \rayr{s/}{s-} + \rayr{/En}{-ena} → \rayr{sen}{sena}.}

\ex\begingl
	\gla tamanreng ledanena nā sina koronvāng //
	\glb taman-reng ledan-ena nā si-na koron-vāng //
	\glc letter-\AargI{} friend-\Gen{} \Fsg.\Gen{} \Rel{}-\Gen{} know=\Ssg{}.\Aarg{} //
	\glft `the letter of my friend which you know' //
\endgl\xe

As pointed out in (\ref{ex:laa}), the word \xayr{laa}{lā}{tongue} ends in a 
long vowel, so the question is what happens when a case suffix beginning with a 
vowel is appended. To avoid a hiat, a glide /j/ may be inserted, so both of 
these are possible:

\pex
	\a\begingl
		\gla Aku lāas! //
		\glb Aka-u lā-as //
		\glc swallow-\Imp{} tongue-\Parg{} //
		\glft `Shut up!' //
	\endgl
	\a\begingl
		\gla {Aku lāyas!} //
		\glb (idem) //
	\endgl
\xe

\noindent With diphthongs, /ɪ/ coalesces with a following /j/ to /j/, but the 
initial vowel will not become tense, hence:

\ex
	\rayr{tipujy}{tip\textbf{uyy}a} [tiˈpʊ.ja] `on the grass' (grass-\Loc{}).
\xe

Moreover, /u/ is commonly realized as [w] when followed by a vowel, for example 
in \rayr{huAAky}{huākaya} [ˈwaːkaja] `frog' or \rayr{ru\_a/}{rua-} [rwa] `have 
to, must'. [w] may also be an allophone of /uj/, as in \rayr{Adauyi}{adauyi} 
[aˈdawi] `then', \rayr{Edauyi}{edauyi} [eˈdawi] `now', or \rayr{nekuyi}{nekuyi} 
[ˈnekwi] `eyebrows'. The negative suffix \rayr{/Oj}{-oy} is also commonly 
contracted to [w] before a diphthong:

\ex
	\rayr{miNojAj}{ming\textbf{oy}ay → ming\textbf{u}ay} [mɪŋˈwaɪ] `I cannot' (can-\Neg{}=\Fsg{}.\Top{}).
\xe

\index{allophony!vowels|)}%
\index{vowels|)}%

\section{Phonotactics}

For the purpose of this statistical analysis, all of the available translations 
into Ayeri from late 2008 to July 2016 have been used as a text 
corpus;\footnote{These texts are:
A Medieval Neighborhood Dispute (2015),
A Message from the Emperor (2012),
Article 1 of the Universal Declaration of Human Rights (2011),
The Beginning of Tolstoy's \tit{Anna Karenina} (2014),
Conlang Christmas Card Exchange 2008/09 (2009),
Conlang Holiday Card Exchange 2010/11 (2011),
Conlang Relay 15 (2008),
Conlang Relay 17 (2010),
Conlang Relay 18 (2011),
The First Two Chapters from Saint-Exupéry's \tit{Le Petit Prince} (2013),
The Four Candles (2010),
Honey Everlasting (2014),
LCC4 Relay (2011),
The Lord's Prayer (2015),
The North Wind and the Sun (2016),
The Origin of the Wind (2009),
Ozymandias (2011),
Please Call Stella … (2008),
Psalm 23 (2013),
The Scientific Method (2014),
The Sheep and the Horses (2012),
Sugar Fairies (2011),
The Upside-Down Ice Skater (2009).
The texts can be accessed from \citet[Examples]{benung}.
} example sentences from 
various blog articles have also been added, as well as dictionary entries for 
all nouns, adjectives, adverbs, pronouns, adpositions, conjunctions, and 
numerals if they were not prefixes or suffixes.\footnote{This section updates 
and extends a previous analysis of the phonological makeup of dictionary entries 
\autocite{becker:frequency}. The previous study had its focus on gathering 
frequency statistics for word generation, however, we want to know about words 
generally here.} Borrowings have been deleted, if they could not reasonably be 
words in Ayeri. Altogether, the corpus comprises 5,500 words, which is a very 
small figure for such a study, but there are only so many texts available 
unfortunately. Words may occur more than once.

Among the dictionary entries, verbs have notably been ignored, since verb stems 
alone do not constitute independent words---they are always inflected in some 
way, so that they may end in consonants or consonant clusters that independent 
words cannot end in. This also has repercussions on syllabification and stress, 
which depend on the inflection of the verb stem:

\begin{figure}[h]
\caption{Syllabification of inflected verbs}
\begin{tabu} to \linewidth {X[2l] X[3c] X[3c] X[3c]}
\toprule\tableheaderfont
Suffix
	& \emph{ca-} `love'
	& \emph{gum-} `work'
	& \emph{babr-} `mumble'
	\\

\toprule

\emph{-ay} (\Fsg{})
	& cā́y
	& gu.máy
	& ba.bráy
	\\

\emph{-va} (\Ssg{})
	& cá.va
	& gúm.va
	& ba.brá.va
	\\

\emph{-yam} (\Ptcp{})
	& cá.yam
	& gúm.yam
	& bá.bryam
	\\

\bottomrule
\end{tabu}
\label{fig:verbsyll}
\end{figure}

% The statistics have been aggregated by \tit{\citetitle{strasser:freq}} 
% \autocite{strasser:freq}. 
For the purpose of gathering statistics on phonemes, 
the words from translation texts were converted to IPA first. Fortunately, this 
is rather easy as Ayeri's romanization is very straightforward. Syllable breaks 
have also been inserted semi-automatically.

\subsection{Number of Syllables per Word}

First, let us see how many syllables words commonly have (see 
\autoref{tab:syllength}). The higher the syllable count, the more likely it is 
for them to be compounds or inflected words.

\begin{table}[htp]\centering
\caption[Frequency of words with different numbers of syllables]{Frequency of words with different numbers of syllables (n\,=\,5500)}
\begin{tabu} to .5\linewidth{X X[c] X[c]}
\tableheaderfont\toprule
Segment
	& Count
	& \multicolumn1{c}{Percentage}
	\\
\toprule

2 syllables
	& 2277
	& 41.40\pct
	\\
	
3 syllables
	& 1393
	& 25.33\pct
	\\
	
1 syllable
	& 1201
	& 21.84\pct
	\\
	
4 syllables
	& 547
	& 9.95\pct
	\\
	
5 syllables
	& 74
	& 1.35\pct
	\\
	
6 syllables
	& 8
	& 0.15\pct
	\\
	
\bottomrule
\end{tabu}
\label{tab:syllength}
\end{table}

Two-syllable words make up the bulk of the sample, which is not surprising since 
1,072 (55.43\pct) of the dictionary subsample are bisyllabic words. Most of 
Ayeri's roots are bisyllabic; unsurprisingly, most monosyllabic words are 
function words like the ones cited below. In the following, I will quote a few 
examples for each number of syllables per word:

\pex
	\a \xayr{yeNF}{yeng}{she} (she.\Aarg{}),\\
		%\rayr{le}{le} (\PatT),\\
		\xayr{ru\_a}{rua}{must};
		
	\a \xayr{dtau}{datau}{normal},\\
		%\xayr{mreNF}{mareng}{it suffices} (suffice=\TsgI{}.\Aarg{}),\\
		\xayr{nsj}{nasay}{near to};
		
	\a \xayr{AvnFyaaNF}{avanyāng}{he sinks} (sink=\TsgM{}.\Aarg{}),\\
		%\xayr{nraanFye}{narānye}{words} (word-\Pl{}),\\
		\xayr{tovlej}{tovaley}{a cloak} (cloak-\PargI{});
		
	\a \rayr{hinYnFveno}{hinyanveno} (corner.beautiful, a place name),\\
		%\xayr{mNstoNF}{mangasatong}{they used to move} (move-\Hab{}=\TplN{}.\Aarg{}),\\
		\xayr{mitnen}{mitanena}{of the palace} (palace-\Gen{});
		
	\a \xayr{hruymnsF}{haruyamanas}{a beating} (beat-\Ptcp{}-\Nmlz{}-\Parg{}),\\
		%\xayr{sirutyen}{sirutayena}{of the night} (night-\Gen{}),\\
		\xayr{suMkornFkihsF}{sungkorankihas}{geography} (science.map);
		
	\a \xayr{kjtomynen}{kaytomayanena}{of righteousness} (righteous-\Nmlz{}-\Gen{}),\\
		%\xayr{koronrYsynF}{koronaryasayan}{they used to forget} (forget-\Hab{}-\TplM{}),\\
		\xayr{nsimyye\_aNF/henF}{nasimayajang-hen}{all followers} (follow-\Agtz-\Pl{}-\Aarg{}=all).
\xe

\begin{sidewaystable}[pth]\centering
\caption[Frequency of syllable types per word]{Frequency of syllable types per word (n\,=\,5500)}
\begin{tabu} to \linewidth{H X[c] S[c] X[c] S[c] X[c] S[c] X[c] S[c] X[c] S[c]}
\tableheaderfont\toprule
Type
	& \multicolumn2{c}{Initial}
	& \multicolumn2{c}{Medial}
	& \multicolumn2{c}{Final}
	& \multicolumn2{c}{Single}
	& \multicolumn2{c}{Total}
	\\
	
\toprule
	
CV
	& 2896
	& 67.36\pct
	& 1974
	& 72.02\pct
	& 2109
	& 49.06\pct
	& 578
	& 48.13\pct
	& 7557
	& 60.26\pct
	\\
	
CCV
	& 55
	& 1.28\pct
	& 24
	& 0.88\pct
	& 46
	& 1.07\pct
	& 32
	& 2.66\pct
	& 157
	& 1.25\pct
	\\
	
CCCV
	& \multicolumn2{c}{—}
% 	& 0
% 	& 0.00\pct
	& \multicolumn2{c}{—}
% 	& 0
% 	& 0.00\pct
	& 2
	& 0.05\pct
	& \multicolumn2{c}{—}
% 	& 0
% 	& 0.00\pct
	& 2
	& 0.02\pct
	\\
	
CVC
	& 761
	& 17.70\pct
	& 610
	& 22.25\pct
	& 1902
	& 44.24\pct
	& 298
	& 24.81\pct
	& 3571
	& 28.48\pct
	\\
	
CCVC
	& 29
	& 0.67\pct
	& 10
	& 0.36\pct
	& 85
	& 1.98\pct
	& 9
	& 0.75\pct
	& 133
	& 1.06\pct
	\\
	
CVCC
	& 2
	& 0.05\pct
	& \multicolumn2{c}{—}
% 	& 0
% 	& 0.00\pct
	& \multicolumn2{c}{—}
% 	& 0
% 	& 0.00\pct
	& \multicolumn2{c}{—}
% 	& 0
% 	& 0.00\pct
	& 2
	& 0.02\pct
	\\

\midrule

V
	& 488
	& 11.35\pct
	& 95
	& 3.47\pct
	& 67
	& 1.56\pct
	& 2
	& 0.17\pct
	& 652
	& 5.20\pct
	\\
	
VC
	& 68
	& 1.58\pct
	& 28
	& 1.02\pct
	& 88
	& 2.05\pct
	& 282
	& 23.48\pct
	& 466
	& 3.72\pct
	\\
	
\bottomrule
	
Total
	& 4299
	& 100.00\pct
	& 2741
	& 100.00\pct
	& 4299
	& 100.00\pct
	& 1201
	& 100.00\pct
	& 12540
	& 100.00\pct
	\\

\bottomrule
\end{tabu}
\label{tab:syltype}
\end{sidewaystable}

\autoref{tab:syltype} shows the frequencies of syllable types\index{syllable!types} by position in a 
word. It is important to note here that phonemes which consist of more than one 
segment---affricates, diphthongs, and long vowels---have been counted as only 
one of C (consonant) or V (vowel), respectively. The following subsections will 
elaborate on which sounds the Cs and Vs correspond to. Moreover, it is important 
to note that medial syllables have not been further distinguished by position in 
the word for the sake of this analysis, so anything between the second and the 
fifth medial syllable is treated the same. It would furthermore be possible to 
calculate the frequencies of one syllable type following the other, 
however, no such calculations have been performed here.

In all positions, CV is the most common syllable type, followed by CVC. With a 
very big margin, V is the next most common syllable type, which is also most 
common in initial syllables and least common in monosyllabic words. The cases 
with only a few attestations are the following:

\pex
	\a Initial CVCC:\\
		\rayr{liMktNF}{linktang} /lɪŋk.ˈtaŋ/ `they try' (try=\TplM{}.\Aarg{}),\footnotemark\\
		\rayr{silFvFnNF}{silvnang} /sɪlv.ˈnaŋ/ `we see' (see=\Fpl{}.\Aarg{});
		
	\a Final CCCV:\\
		\rayr{migFrFyo}{migryo} /ˈmi.grjo/ `flourishes' (flourish-\Tsg{}.\N{}),\\
		\rayr{subFrFyo}{subryo} /ˈsu.brjo/ `ceases' (cease-\Tsg{}.\N{});
	
	\a Single V:\\
		\rayr{Aj}{ay} /aɪ/ `I' (\Fsg{}.\Top{}).
\xe

\footnotetext{The verb stem is found in the dictionary as \rayr{liMk/}{linka-}, 
with a final \textit{-a}, and thus is possibly an entry changed at a later point, 
or the example from the text (Sugar Fairies) chosen here contains an error.}

\phantomsection%
The medial and final VC cases may seem like an oddity, but they are mostly due 
to the previous syllable ending in /ŋ/, with that syllable also containing a 
lax vowel, which means that this syllable must be closed. An alternative 
explanation would be to assume that /ŋ/ is ambisyllabic, or actually /n.g 
\textasciitilde{} ŋ.g/, but realized as [ŋ].\label{ŋ} The high number of 
single-syllable VC is due to \xayr{ANF}{ang}{\AgtT}, which alone appears 255 
times in the sample (4.63\pct{} of all words, 21.23\pct{} of monosyllabic 
words, 90.43\pct{} of monosyllabic VC words).

\subsection{Phonemic Makeup of Initial Syllables}

\index{syllable!initial|(}%
The statistics in the following sections have been gathered from the IPA 
conversions of translated texts and dictionary entries mentioned above. The 
transcribed words have been split into syllables and then the collected contents 
of each position group were written into separate plain text files, one each for:

\begin{itemize}
	\item all initial syllables of polysyllabic words,
	\item all medial syllables of polysyllabic words,
	\item all final syllables of polysyllabic words, and 
	\item all monosyllabic words.
\end{itemize}

Monosyllabic words are both initial and final syllables at the same time; they 
have been counted separately for the purpose of this analysis. Onsets, nuclei 
and codas have been matched by regular expressions; the com\-mand line tools 
\texttt{grep}, \texttt{sort}, and \texttt{uniq} were used to aggregate all 
occurring variants for each syllable segment as well as their absolute 
frequencies:\footnote{However, \texttt{sort} was unable to handle all IPA 
characters, so \texttt{sed 'y/ɛɪɔʊəːʃʒŋ/EIOU@:SZN/'} had to be used to 
compensate by transcribing everything into X-SAMPA.}

\ex
	\texttt{C = (?:tʃ|dʒ|[ptkbdgmnŋvshrljw])\\
	V = (?:[ae]ː?ɪ|aʊ|[ieaou]ː?|[ɪɛɔʊə])}
\xe

\begin{table}[pth]\centering
\caption[Frequency of onsets in initial syllables]{Frequency of onsets in initial syllables (n\,=\,4299)}
\begin{tabu} to 0.5\linewidth{X X[c] X[c]}
\tableheaderfont\toprule
Phoneme
	& Frequency
	& Percentage
	\\
	
\toprule

Ø
	& 556
	& 12.93\pct
	\\

\midrule

s
	& 488
	& 11.35\pct
	\\

t
	& 432
	& 10.05\pct
	\\

m
	& 418
	& 9.72\pct
	\\

k
	& 380
	& 8.84\pct
	\\

n
	& 375
	& 8.72\pct
	\\

p
	& 334
	& 7.77\pct
	\\

b
	& 231
	& 5.37\pct
	\\

d
	& 172
	& 4.00\pct
	\\

v
	& 164
	& 3.81\pct
	\\

l
	& 159
	& 3.70\pct
	\\

r
	& 134
	& 3.12\pct
	\\

j
	& 126
	& 2.93\pct
	\\

g
	& 111
	& 2.58\pct
	\\

h
	& 99
	& 2.30\pct
	\\

tʃ
	& 30
	& 0.70\pct
	\\

pr
	& 27
	& 0.63\pct
	\\

nj
	& 27
	& 0.63\pct
	\\

kr
	& 8
	& 0.19\pct
	\\

br
	& 8
	& 0.19\pct
	\\

tr
	& 6
	& 0.14\pct
	\\

dʒ
	& 4
	& 0.09\pct
	\\

gr
	& 3
	& 0.07\pct
	\\

w
	& 2
	& 0.05\pct
	\\

sw
	& 1
	& 0.02\pct
	\\

rw
	& 1
	& 0.02\pct
	\\

pj
	& 1
	& 0.02\pct
	\\

mj
	& 1
	& 0.02\pct
	\\

bw
	& 1
	& 0.02\pct
	\\

\bottomrule
\end{tabu}
\label{tab:initon}
\end{table}

As we have seen above (\autoref{tab:syltype}), CCV syllables only make up 
1.28\pct{} of initial syllables, in so far it is no surprise that consonant 
clusters all appear at the bottom of \autoref{tab:initon}. There also seem to 
be combination patterns in that initial clusters exist for all plosives plus /r/, 
and almost all bilabials plus /j/, with the exception of /bj/, however, /nj/ is 
added to the group instead. Combinations with /w/ only occur for /b/, /r/, and 
/s/, which do not share an obvious connection. Syllables without a consonant 
filling the onset position are marked with \enquote*{Ø}; these numbers 
correspond to the VC and VCC rows in \autoref{tab:syltype}.

\begin{table}[pth]\centering
\caption[Frequency of nuclei in initial syllables]{Frequency of nuclei in initial syllables (n\,=\,4299)}
\begin{tabu} to 0.5\linewidth{X X[c] X[c]}
\tableheaderfont\toprule
Phoneme
	& Frequency
	& Percentage
	\\
	
\toprule

a
	& 1847
	& 42.96\pct
	\\

\midrule

i
	& 1011
	& 23.52\pct
	\\

\rowfont{\scriptsize\itshape}
\raggedleft
i
	& 802
	& 18.66\pct
	\\

\rowfont{\scriptsize\itshape}
\raggedleft
ɪ
	& 209
	& 4.86\pct
	\\

\midrule

e
	& 705
	& 16.40\pct
	\\

\rowfont{\scriptsize\itshape}
\raggedleft
e
	& 523
	& 12.17\pct
	\\

\rowfont{\scriptsize\itshape}
\raggedleft
ɛ
	& 164
	& 3.81\pct
	\\

\rowfont{\scriptsize\itshape}
\raggedleft
ə
	& 18
	& 0.42\pct
	\\

\midrule

u
	& 260
	& 6.05\pct
	\\

\rowfont{\scriptsize\itshape}
\raggedleft
u
	& 228
	& 5.30\pct
	\\

\rowfont{\scriptsize\itshape}
\raggedleft
ʊ
	& 32
	& 0.74\pct
	\\

\midrule

o
	& 227
	& 5.28\pct
	\\

\rowfont{\scriptsize\itshape}
\raggedleft
o
	& 188
	& 4.37\pct
	\\

\rowfont{\scriptsize\itshape}
\raggedleft
ɔ
	& 39
	& 0.91\pct
	\\

\midrule

aː
	& 109
	& 2.54\pct
	\\

aɪ
	& 88
	& 2.05\pct
	\\

eɪ
	& 40
	& 0.93\pct
	\\

eː
	& 4
	& 0.09\pct
	\\

ɔɪ
	& 3
	& 0.07\pct
	\\

ʊɪ
	& 1
	& 0.02\pct
	\\

oː
	& 1
	& 0.02\pct
	\\

iː
	& 1
	& 0.02\pct
	\\

eːɪ
	& 1
	& 0.02\pct
	\\

aʊ
	& 1
	& 0.02\pct
	\\

\bottomrule
\end{tabu}
\label{tab:initnuc}
\end{table}

Perhaps most striking about the nuclei of initial syllables presented in 
\autoref{tab:initnuc} is that it is plain vowels which occur most of the time. 
As mentioned above, lax vowels are counted here as allophones\index{allophony!vowels} of tense ones as 
their distribution is complementary, which is why the plain vowels are 
presented as grouped. Long vowels and diphthongs find themselves below the 
5\pct{} threshold, and the words with single occurrences are:

\pex
	\a \xayr{kujsaanF}{kuysān}{comparison},
	\a \xayr{noonF}{nōn}{will, intention},
	\a \xayr{niis}{nīsa}{wanted},\footnotemark
	\a \xayr{seejry}{sēyraya}{will overcome} (\Fut{}-overcome-\Tsg{}.\M{}),
	\a \xayr{sautnF}{sautan}{cork}.
\xe

\footnotetext{\rayr{niis}{nīsa} and \rayr{noonF}{nōn} are both related to 
\xayr{no/}{no-}{want, plan}.}

As [eːɪ] only occurs due to allophony\index{allophony!vowels}, it should not be counted as a phoneme for
the purposes of this analysis. On the other hand, the same could be said for a 
lot of cases of [aː] included here---this caveat applies to all nouns derived
from verbs ending in \textit{-a} with the very common nominalizing suffix 
\rayr{/AnF}{-an}, as exemplified in (\ref{ex:longvwls}) above. Similarly, the 
18 cases of /ə/ reported here are mostly from tense prefixes also mentioned 
above, for instance, \xayr{mkoronj}{məkoronay}{I knew} 
(\Pst{}-know=\Fsg{}.\Top{}).

\begin{table}[pth]\centering
\caption[Frequency of codas in initial syllables]{Frequency of codas in initial syllables (n\,=\,4299)}
\begin{tabu} to 0.5\linewidth{X X[c] X[c]}
\tableheaderfont\toprule
Phoneme
	& Frequency
	& Percentage
	\\
	
\toprule

Ø
	& 3441
	& 80.04\pct
	\\

\midrule

n
	& 298
	& 6.93\pct
	\\

ŋ
	& 243
	& 5.65\pct
	\\

r
	& 129
	& 3.00\pct
	\\

l
	& 88
	& 2.05\pct
	\\

m
	& 74
	& 1.72\pct
	\\

s
	& 20
	& 0.47\pct
	\\

t
	& 2
	& 0.05\pct
	\\

h
	& 2
	& 0.05\pct
	\\
	
tʃ
	& 1
	& 0.02\pct
	\\

ŋk
	& 1
	& 0.02\pct
	\\

lv
	& 1
	& 0.02\pct
	\\

k
	& 1
	& 0.02\pct
	\\

\bottomrule
\end{tabu}
\label{tab:initcod}
\end{table}

Initial-syllable codas (\autoref{tab:initcod}) are far less diverse than 
consonant onsets: there are only 10 attested segments in comparison to 28 for 
onsets (not counting empty codas of C(C)V syllables, which constitute the 
majority by a large margin), and the only two cluster attested are /ŋk/ in the 
word \rayr{liMktNF}{linktang} `they try' (try=\TplM{}.\Aarg{}), and /lv/ in the 
word \rayr{silFvFnNF}{silvnang} `I see' (see=\Fpl{}.\Aarg{}). There only being 
two incidences of a CC cluster is very probably an effect of the small 
sample size. Furthermore, the only unvoiced single coda consonants attested are 
/s/, /h/, /t/, /tʃ/ and /k/, the latter two only once, /h/ twice:

\pex
	\a \xayr{mehFvaaNF}{mehvāng}{you are supposed to} 
		(be.supposed.to=\Ssg{}.\Aarg{}),\footnotemark\\
		\xayr{rohFtNF}{rohtang}{they bite} (bite=\TsgM{}.\Aarg{});
 	\a \xayr{mutFv}{mutva}{you rub} (rub=\Ssg{}.\Top{}),\\
		\xayr{ptFlj}{patlay}{cousin};
	\a \xayr{sikF/sikF}{sik-sik}{tits};
	\a \xayr{vtYvaaNF}{vacvāng}{you like} (like=\Ssg{}.\Aarg{}).
\xe

\footnotetext{The dictionary entry for the verb is \rayr{mY/}{mya-}, so this 
may be an instance of my changing a word in the dictionary with the old one 
staying in the text (The Four Candles).}
\index{syllable!initial|)}%

\subsection{Phonemic Makeup of Medial Syllables}

\index{syllable!medial|(}%
\begin{table}[pth]\centering
\caption[Frequency of onsets in medial syllables]{Frequency of onsets in medial syllables (n\,=\,2741)}
\begin{tabu} to 0.5\linewidth{X X[c] X[c]}
\tableheaderfont\toprule
Phoneme
	& Frequency
	& Percentage
	\\
	
\toprule

Ø
	& 123
	& 4.49\pct
	\\

\midrule

r
	& 343
	& 12.51\pct
	\\

n
	& 260
	& 9.49\pct
	\\

j
	& 233
	& 8.50\pct
	\\

t
	& 222
	& 8.10\pct
	\\

d
	& 213
	& 7.77\pct
	\\

k
	& 189
	& 6.90\pct
	\\

s
	& 170
	& 6.20\pct
	\\

m
	& 169
	& 6.17\pct
	\\

l
	& 149
	& 5.44\pct
	\\

v
	& 148
	& 5.40\pct
	\\

h
	& 147
	& 5.36\pct
	\\

p
	& 119
	& 4.34\pct
	\\

g
	& 92
	& 3.36\pct
	\\

b
	& 89
	& 3.25\pct
	\\

tʃ
	& 20
	& 0.73\pct
	\\

dʒ
	& 15
	& 0.55\pct
	\\

tr
	& 11
	& 0.40\pct
	\\

dr
	& 8
	& 0.29\pct
	\\

pr
	& 7
	& 0.26\pct
	\\

w
	& 6
	& 0.22\pct
	\\

sj
	& 2
	& 0.07\pct
	\\

br
	& 2
	& 0.07\pct
	\\

sw
	& 1
	& 0.04\pct
	\\

kw
	& 1
	& 0.04\pct
	\\

kj
	& 1
	& 0.04\pct
	\\

bj
	& 1
	& 0.04\pct
	\\

\bottomrule
\end{tabu}
\label{tab:medon}
\end{table}

The onsets of medial syllables (\autoref{tab:medon}) show properties very 
similar to those of initial syllables. The order of most common consonants may 
different here---for example, the most common onset is /r/, not Ø or /s/---, 
but there are no restrictions for which consonants to appear in this position, 
with the exception of /ŋ/ for reasons stated above (see \autoref{ŋ}). Regarding 
initial clusters, there are further attestations for plosive plus /r/ (except 
for /kr/). Regarding clusters with /j/, the only one with a bilabial is /bj/, 
but the set is extended to /sj/ and /kj/. For clusters with /w/, only /sw/ and 
/kw/ occur here, while attestations for /bw/ and /rw/ as in initial-syllable 
onsets are lacking. This does not mean that those combinations are not 
principally possible in this position, however.

\begin{table}[pth]\centering
\caption[Frequency of nuclei in medial syllables]{Frequency of nuclei in medial syllables (n\,=\,2741)}
\begin{tabu} to 0.5\linewidth{X X[c] X[c]}
\tableheaderfont\toprule
Phoneme
	& Frequency
	& Percentage
	\\
	
\toprule

a
	& 1480
	& 53.99\pct
	\\

\midrule

i
	& 480
	& 17.51\pct
	\\

\rowfont{\scriptsize\itshape}
\raggedleft
i
	& 387
	& 14.12\pct
	\\

\rowfont{\scriptsize\itshape}
\raggedleft
ɪ
	& 93
	& 3.39\pct
	\\

\midrule

e
	& 254
	& 9.26\pct
	\\

\rowfont{\scriptsize\itshape}
\raggedleft
e
	& 206
	& 7.52\pct
	\\

\rowfont{\scriptsize\itshape}
\raggedleft
ɛ
	& 48
	& 1.75\pct
	\\

\midrule

o
	& 194
	& 7.08\pct
	\\

\rowfont{\scriptsize\itshape}
\raggedleft
o
	& 119
	& 4.34\pct
	\\

\rowfont{\scriptsize\itshape}
\raggedleft
ɔ
	& 75
	& 2.74\pct
	\\

\midrule

u
	& 120
	& 4.38\pct
	\\

\rowfont{\scriptsize\itshape}
\raggedleft
u
	& 101
	& 3.68\pct
	\\

\rowfont{\scriptsize\itshape}
\raggedleft
ʊ
	& 19
	& 0.69\pct
	\\

\midrule

aː
	& 110
	& 4.01\pct
	\\

aɪ
	& 51
	& 1.86\pct
	\\

ɔɪ
	& 33
	& 1.20\pct
	\\

eɪ
	& 5
	& 0.18\pct
	\\

eː
	& 5
	& 0.18\pct
	\\

aʊ
	& 5
	& 0.18\pct
	\\

ʊɪ
	& 2
	& 0.07\pct
	\\

uː
	& 1
	& 0.04\pct
	\\

iː
	& 1
	& 0.04\pct
	\\

\bottomrule
\end{tabu}
\label{tab:mednuc}
\end{table}

As with onset consonants, vowel nuclei of medial syllables 
(\autoref{tab:mednuc}) do not show significant differences compared to those of 
initial syllables either. /a/ is more common here, and /o/ and /u/ switch 
places. Instead of /eːɪ/, there is an attestation of /uː/ (see \autoref{fn:ū}), 
for which there is more reason to be counted as a phoneme than for /eːɪ/. The 
sequences /iː/ and /ʊɪ/ also only occur once and twice, respectively, namely in 
the following words:

\pex
	\a \xayr{psiis}{pasīsa}{interesting};\label{ex:pasīsa}
	\a \xayr{pulujlej}{puluyley}{a mirror} (mirror-\PargI{}),\\
		\xayr{tipujy}{tipuyya}{on the grass} (grass-\Loc{}).
\xe

The word in (\ref{ex:pasīsa}), \xayr{psiis}{pasīsa}{interesting}, should count 
as a lexeme in its own right, since it possesses idiomatic meaning. Nonetheless, 
it rather transparently constitutes a causative derivation of the verb 
\xayr{psY/}{pasy-}{wonder, be curious, be interested}, essentially meaning 
`making one wonder/curious'---the causative suffix \rayr{/Is}{-isa} can as well 
be used to derive adjectives with a causative or resultative meaning.

\begin{table}[pth]\centering
\caption[Frequency of codas in medial syllables]{Frequency of codas in medial syllables (n\,=\,2741)}
\begin{tabu} to 0.5\linewidth{X X[c] X[c]}
\tableheaderfont\toprule
Phoneme
	& Frequency
	& Percentage
	\\
	
\toprule

Ø
	& 2093
	& 76.36\pct
	\\

\midrule

n
	& 313
	& 11.42\pct
	\\

ŋ
	& 193
	& 7.04\pct
	\\

r
	& 48
	& 1.75\pct
	\\

m
	& 39
	& 1.42\pct
	\\

s
	& 32
	& 1.17\pct
	\\

l
	& 21
	& 0.77\pct
	\\

t
	& 1
	& 0.04\pct
	\\

g
	& 1
	& 0.04\pct
	\\

\bottomrule
\end{tabu}
\label{tab:medcod}
\end{table}

With medial-syllable codas (\autoref{tab:medcod}) again, sonorants and /s/ make 
up the largest number of consonants in this position; /t/ and /g/ only occur 
once each in

\pex
	\a \xayr{pNitFlnF}{pangitlan}{money change} and\label{ex:pangitlan}
	\a \xayr{telugFtoNF}{telugtong}{they survive} (survive=\TplN{}).\footnotemark
\xe

\footnotetext{The word for `money' is \rayr{pNisF}{pangis}, so 
(\ref{ex:pangitlan}) is probably a compound, albeit not a fully transparent one. 
The word for `change' is \rayr{til/}{tila-}, and there seems to be a 
nominalizing \rayr{/AnF}{-an}. Ayeri allows noun--verb compounds to have a 
nominalized verb in the second position in spite of it being the head---noun--%
noun compounds mostly come in a head-initial order---probably due to an 
avoidance of placing a derivative suffix in the middle of a word. Possibly, what 
happened after all is that \rayr{tilaanF}{tilān} underwent metathesis to 
*\rayr{ItFlaan}{*itlān} to match the rhyme of \rayr{pNisF}{pangis}. 
*\rayr{pNisitFlaanF}{*pangisitlān} then underwent irregular haplology (and 
shortening of the nominalizing suffix) to \rayr{pNitFlnF}{pangitlan}.}

As documented in \autoref{tab:syltype} above, Ayeri very strongly favors CV 
syllables in medial positions, hence the high count of zero segments here.
\index{syllable!medial|)}%

\subsection{Phonemic Makeup of Final Syllables}

\index{syllable!final|(}%
\begin{table}[pth]\centering
\caption[Frequency of onsets in final syllables]{Frequency of onsets in final syllables (n\,=\,4299)}
%\setlength{\columnsep}{1em}
\setlength{\multicolsep}{0em}
\begin{multicols}{2}
\begin{tabu} to \linewidth{X X[c] X[c]}
\tableheaderfont\toprule
Phoneme
	& Frequency
	& Percentage
	\\
	
\toprule

Ø
	& 155
	& 3.61\pct
	\\

\midrule

j
	& 1101
	& 25.61\pct
	\\

n
	& 528
	& 12.28\pct
	\\

r
	& 398
	& 9.26\pct
	\\

t
	& 268
	& 6.23\pct
	\\

s
	& 244
	& 5.68\pct
	\\

l
	& 238
	& 5.54\pct
	\\

k
	& 199
	& 4.63\pct
	\\

d
	& 184
	& 4.28\pct
	\\

m
	& 154
	& 3.58\pct
	\\

v
	& 144
	& 3.35\pct
	\\

h
	& 128
	& 2.98\pct
	\\

p
	& 115
	& 2.68\pct
	\\

g
	& 103
	& 2.40\pct
	\\

dʒ
	& 73
	& 1.70\pct
	\\

b
	& 73
	& 1.70\pct
	\\

tʃ
	& 52
	& 1.21\pct
	\\

vj
	& 26
	& 0.60\pct
	\\

pj
	& 22
	& 0.51\pct
	\\

dʒj
	& 17
	& 0.40\pct
	\\

tr
	& 10
	& 0.23\pct
	\\

w
	& 9
	& 0.21\pct\\

	
% 	& \dots
% 	&
% 	\\

\bottomrule
\end{tabu}

\begin{tabu} to \linewidth{X X[c] X[c]}
\tableheaderfont\toprule
Phoneme
	& Frequency
	& Percentage
	\\
	
\toprule

	
% 	& \dots
% 	&
% 	\\
% 
% \midrule[0pt]

pr
	& 7
	& 0.16\pct
	\\

kj
	& 6
	& 0.14\pct
	\\

hj
	& 5
	& 0.12\pct
	\\

bj
	& 5
	& 0.12\pct
	\\

tw
	& 4
	& 0.09\pct
	\\

sw
	& 4
	& 0.09\pct
	\\

sj
	& 4
	& 0.09\pct
	\\

kw
	& 3
	& 0.07\pct
	\\

kr
	& 3
	& 0.07\pct
	\\

br
	& 3
	& 0.07\pct
	\\

vr
	& 2
	& 0.05\pct
	\\

rw
	& 2
	& 0.05\pct
	\\

nw
	& 2
	& 0.05\pct
	\\

tʃj
	& 1
	& 0.02\pct
	\\

rj
	& 1
	& 0.02\pct
	\\

nj
	& 1
	& 0.02\pct
	\\

mw
	& 1
	& 0.02\pct
	\\

grj
	& 1
	& 0.02\pct
	\\

dv
	& 1
	& 0.02\pct
	\\

dr
	& 1
	& 0.02\pct
	\\

brj
	& 1
	& 0.02\pct\\

\bottomrule
\end{tabu}
\end{multicols}
\label{tab:finon}
\end{table}

The onsets of final syllables of polysyllabic words (\autoref{tab:finon}) show 
the greatest amount of variety, which is due to Ayeri mostly using suffixes for 
grammatical purposes. Hence it is no surprise that combinations with /j/ and, 
indeed, /j/ itself as an onset, are especially common, since /j/ is also what a 
number of very common suffixes start with, for example the plural marker 
\rayr{/ye}{-ye}, the locative marker \rayr{/y}{-ya}, the dative and participle 
marker \rayr{/ymF}{-yam}, as well as third-person animate pronoun agreement 
suffixes, and the various first-person and third-person animate pronominal 
clitics. \autoref{fig:verbsyll} above shows exemplarily how verbs resyllabify 
when suffixes are attached. Even though single-segment onsets are strongly 
preferred, Cr, Cw, and especially C(C)j seem to be generally 
permissible.\footnote{The sequence /sj/ poses difficulty here as there are 
examples for /Vs.jV/ as well as for /V.sjV/, and I cannot tell for sure if 
there is a strict rule in operation. It seems that /V.sjV/ is more likely to 
occur when the second syllable is stressed, whereas /Vs.jV/ is more likely to 
occur when the first syllable is stressed. Ayeri's own Tahano Hikamu 
orthography would not show the difference either, since /sja/ is spelled 
\ayr{sY} either way, and there is no heeding morpheme breaks either. /CsjV/ 
will be /C.sjV/ in any case, since Ayeri avoids final consonant clusters if 
possible, see \autoref{tab:syltype}.}

\begin{table}[pth]\centering
\caption[Frequency of nuclei in final syllables]{Frequency of nuclei in final syllables (n\,=\,4299)}
\begin{tabu} to 0.5\linewidth{X X[c] X[c]}
\tableheaderfont\toprule
Phoneme
	& Frequency
	& Percentage
	\\
	
\toprule

a
	& 2408
	& 56.01\pct
	\\

aː
	& 316
	& 7.35\pct
	\\

\midrule

o
	& 411
	& 9.56\pct
	\\

\rowfont{\scriptsize\itshape}
\raggedleft
o
	& 298
	& 6.93\pct
	\\

\rowfont{\scriptsize\itshape}
\raggedleft
ɔ
	& 113
	& 2.63\pct
	\\

\midrule

i
	& 289
	& 6.42\pct
	\\

\rowfont{\scriptsize\itshape}
\raggedleft
ɪ
	& 147
	& 3.42\pct
	\\

\rowfont{\scriptsize\itshape}
\raggedleft
i
	& 142
	& 3.30\pct
	\\

\midrule

aɪ
	& 254
	& 5.91\pct
	\\

\midrule

u
	& 207
	& 4.82\pct
	\\

\rowfont{\scriptsize\itshape}
\raggedleft
u
	& 155
	& 3.61\pct
	\\

\rowfont{\scriptsize\itshape}
\raggedleft
ʊ
	& 52
	& 1.21\pct
	\\

\midrule

e
	& 209
	& 4.85\pct
	\\

\rowfont{\scriptsize\itshape}
\raggedleft
ɛ
	& 127
	& 2.95\pct
	\\

\rowfont{\scriptsize\itshape}
\raggedleft
ə
	& 81
	& 1.88\pct
	\\

\rowfont{\scriptsize\itshape}
\raggedleft
e
	& 1
	& 0.02\pct
	\\

\midrule

eɪ
	& 103
	& 2.40\pct
	\\

ɔɪ
	& 42
	& 0.98\pct
	\\

aːɪ
	& 23
	& 0.54\pct
	\\

ʊɪ
	& 14
	& 0.33\pct
	\\

aʊ
	& 14
	& 0.33\pct
	\\

eː
	& 5
	& 0.12\pct
	\\

iː
	& 3
	& 0.07\pct
	\\

uː
	& 1
	& 0.02\pct
	\\

\bottomrule
\end{tabu}
\label{tab:finnuc}
\end{table}

Nuclei of final syllables (\autoref{tab:finnuc}) do not bear striking 
differences to nuclei in other positions. /aː/ comes out second here due to the 
common nominalizer \rayr{/AnF}{-an}, which lengthens the vowel of verb stems 
ending in /a/, as demonstrated in (\ref{ex:longvwls}). /aɪ/ is also fairly 
common here as it is the topic-marked first-person pronoun/pronominal clitic; 
for the same reason, /aːɪ/ occurs a number of times---the vowel-lengthening 
rule applies here as well, so its status as a phoneme is marginal. All 
instances of /eː/ in the sample are from the word \xayr{AreenF}{arēn}{anyway, 
however}; all evidence for /iː/ is from \xayr{sirii}{sirī}{due to which} (see 
\autoref{doublerel}). The only evidence for /uː/ in the sample is from 
\xayr{bbuu}{babū}{barbarian (adj.)}.

\begin{table}[pth]\centering
\caption[Frequency of codas in final syllables]{Frequency of codas in final syllables (n\,=\,4299)}
\begin{tabu} to 0.5\linewidth{X X[c] X[c]}
\tableheaderfont\toprule
Phoneme
	& Frequency
	& Percentage
	\\
	
\toprule

Ø
	& 2224
	& 51.73\pct
	\\

\midrule

n
	& 899
	& 20.91\pct
	\\

ŋ
	& 651
	& 15.14\pct
	\\

s
	& 244
	& 5.68\pct
	\\

m
	& 225
	& 5.23\pct
	\\

l
	& 34
	& 0.79\pct
	\\

r
	& 21
	& 0.49\pct
	\\

k
	& 1
	& 0.02\pct
	\\

\bottomrule
\end{tabu}
\label{tab:fincod}
\end{table}

The list of coda consonants in final syllables (\autoref{tab:fincod}) is very 
slightly more restrictive than even that of coda consonants in medial syllables
(see \autoref{tab:medcod}), since the only non-sonorant attested is /k/, which 
only occurs in \xayr{sikF/sikF}{sik-sik}{tits} again, which---besides being a 
vulgar term, thus maybe slightly more dispositioned to allow for deviating 
phonotactics---looks quite like onomatopoeia, possibly for the sound of 
sucking.\footnote{\citet[489--490]{kroonen2013} identifies PGmc \textit{*sūgan-,
*sūkan-} `to suck' as an iterative of PGmc \textit{*sukkōn-, *sugōn-} `to suck' 
and reconstructs PIE \textit{*souḱ-neh₂-}. However, he does not say anything 
about the Germanic word being onomatopoetic in origin.}
\index{syllable!final|)}%

\subsection{Phonemic Makeup of Single Syllables}

\index{syllable!monosyllabic words|(}%
\begin{table}[pth]\centering
\caption[Frequency of onsets in single syllables]{Frequency of onsets in single syllables (n\,=\,1201)}
\begin{tabu} to 0.5\linewidth{X X[c] X[c]}
\tableheaderfont\toprule
Phoneme
	& Frequency
	& Percentage
	\\
	
\toprule

Ø
	& 284
	& 23.65\pct
	\\

\midrule

n
	& 231
	& 19.23\pct
	\\

s
	& 147
	& 12.24\pct
	\\

j
	& 144
	& 11.99\pct
	\\

k
	& 51
	& 4.25\pct
	\\

v
	& 48
	& 4.00\pct
	\\

m
	& 46
	& 3.83\pct
	\\

l
	& 44
	& 3.66\pct
	\\

t
	& 41
	& 3.41\pct
	\\

d
	& 33
	& 2.75\pct
	\\

r
	& 26
	& 2.16\pct
	\\

h
	& 23
	& 1.92\pct
	\\

mj
	& 16
	& 1.33\pct
	\\

p
	& 13
	& 1.08\pct
	\\

tʃ
	& 9
	& 0.75\pct
	\\

g
	& 9
	& 0.75\pct
	\\

nj
	& 8
	& 0.67\pct
	\\

rw
	& 7
	& 0.58\pct
	\\

b
	& 7
	& 0.58\pct
	\\

pr
	& 5
	& 0.42\pct
	\\

dʒ
	& 3
	& 0.25\pct
	\\

tr
	& 2
	& 0.17\pct
	\\

nw
	& 1
	& 0.08\pct
	\\

ŋ
	& 1
	& 0.08\pct
	\\

kr
	& 1
	& 0.08\pct
	\\

br
	& 1
	& 0.08\pct
	\\

\bottomrule
\end{tabu}
\label{tab:singon}
\end{table}

Onsets of single syllables (\autoref{tab:singon}) appear to be the least varied 
category. Still, none of the basic set of consonant morphemes (see 
\autoref{fig:consonants}) is missing---the frequency order is just completely 
different from the other onsets surveyed, not merely a mix of initial and final 
syllables. Consonant clusters with /j/, /w/ and /r/ exist here as well. 
Combinations with /j/ are only present for /m/ and /n/, while /r/ again 
combines with plosives; /w/ combines with /n/ and /r/ at least, which we have 
already seen in final-syllable onsets (see \autoref{tab:finon}). Whereas /mj/ 
has only occurred once in initial-syllable onsets so far (see 
\autoref{tab:initon}), it occurs a few more times here, all in the word 
\xayr{mY}{mya}{be supposed to}, which is very commonly used as an indeclinable 
modal particle.

A consonant onset that can only be found in monosyllables is /ŋ/,\footnote{At 
least according to the analysis chosen here, see \autoref{ŋ} for an 
explanation.} in \xayr{/NsF}{-ngas}{almost}, a quantifier suffix that has 
managed to sneak in due to being marked as an adverb in the dictionary, since 
it can modify a verb:

\ex\begingl
	\gla Apayeng-ngas. //
	\glb Apa-yeng-ngas //
	\glc laugh=\TsgF{}.\Aarg{}=almost //
	\glft `She almost laughed.' //
\endgl\xe

\noindent Here, \rayr{/NsF}{-ngas} modifies the verb complex like any other 
adverb:

\ex\begingl
	\gla Apayeng baho. //
	\glb Apa-yeng baho //
	\glc laugh=\TsgF{}.\Aarg{} loudly //
	\glft `She laughs loudly.' //
\endgl\xe

\noindent However, whereas \xayr{bho}{baho}{loud} is treated as a separate unit 
in terms of intonation, \rayr{/NsF}{-ngas} is unstressed and binds to what it 
follows:

\pex
	\a \rayr{ApyeNF/NsF.}{Apayeng-ngas.} [ˌapaˈjɛŋas];
	\a \rayr{ApyeNF bho.}{Apayeng baho.} [ˌapaˈjɛŋ ˈbaxo].
\xe

\begin{table}[pth]\centering
\caption[Frequency of nuclei in single syllables]{Frequency of nuclei in single syllables (n\,=\,1201)}
\begin{tabu} to 0.5\linewidth{X X[c] X[c]}
\tableheaderfont\toprule
Phoneme
	& Frequency
	& Percentage
	\\
	
\toprule

a
	& 568
	& 47.29\pct
	\\

aɪ
	& 171
	& 14.24\pct
	\\

aː
	& 140
	& 11.66\pct
	\\

\midrule

i
	& 113
	& 9.41\pct
	\\

\rowfont{\scriptsize\itshape}
\raggedleft
i
	& 65
	& 5.41\pct
	\\

\rowfont{\scriptsize\itshape}
\raggedleft
ɪ
	& 48
	& 4.00\pct
	\\

\midrule

e
	& 104
	& 8.66\pct
	\\

\rowfont{\scriptsize\itshape}
\raggedleft
ɛ
	& 65
	& 5.41\pct
	\\

\rowfont{\scriptsize\itshape}
\raggedleft
e
	& 34
	& 2.83\pct
	\\

\rowfont{\scriptsize\itshape}
\raggedleft
ə
	& 5
	& 0.42\pct
	\\

\midrule

o
	& 45
	& 3.75\pct
	\\

\rowfont{\scriptsize\itshape}
\raggedleft
ɔ
	& 30
	& 2.50\pct
	\\

\rowfont{\scriptsize\itshape}
\raggedleft
o
	& 15
	& 1.25\pct
	\\

\midrule

u
	& 20
	& 1.67\pct
	\\

aːɪ
	& 14
	& 1.17\pct
	\\

ɔɪ
	& 10
	& 0.83\pct
	\\

iː
	& 6
	& 0.50\pct
	\\

eɪ
	& 5
	& 0.42\pct
	\\

ʊɪ
	& 3
	& 0.25\pct
	\\

oː
	& 2
	& 0.17\pct
	\\

\bottomrule
\end{tabu}
\label{tab:singnuc}
\end{table}

As with onset consonants of monosyllabic words, nuclei of this syllable type 
are the least diverse group again (\autoref{tab:singnuc}). One segment that is 
notably absent is /aʊ/, and the marginally phonemic /eː/ is not present either. 
By having /a/, /aɪ/, /aː/ at the top, monosyllabic words behave similar to 
final syllables of polysyllabic words (see \autoref{tab:finnuc}), however, the 
order of the most common vowels bears more similarities to that of initial and 
medial syllables (see Tables \ref{tab:initnuc} and \ref{tab:mednuc}). The very 
uncommon /oː/ features twice in this group, namely in two instances of the word 
\xayr{noonF}{nōn}{will, intention}.\footnote{Ayeri used to have \rayr{/On}{-on} 
as a nominalizer beside \rayr{/An}{-an}, however, it was not very productive and 
has long fallen out of use. \rayr{noonF}{nōn} is thus, in fact, originally a 
nominalization of \xayr{no/}{no-}{want, plan}.}

\begin{table}[pth]\centering
\caption[Frequency of codas in single syllables]{Frequency of codas in single syllables (n\,=\,1201)}
\begin{tabu} to 0.5\linewidth{X X[c] X[c]}
\tableheaderfont\toprule
Phoneme
	& Frequency
	& Percentage
	\\
	
\toprule

Ø
	& 612
	& 50.96\pct\\

\midrule

ŋ
	& 377
	& 31.39\pct\\
n
	& 105
	& 8.74\pct\\
s
	& 58
	& 4.83\pct\\
m
	& 36
	& 3.00\pct\\
l
	& 6
	& 0.50\pct\\
h
	& 4
	& 0.33\pct\\
r
	& 3
	& 0.25\pct\\

\bottomrule
\end{tabu}
\label{tab:singcod}
\end{table}

Like the other syllable segments of monosyllabic words, coda consonants 
(\autoref{tab:singcod}) as well show the lowest degree of variety among all the 
coda consonants of the various syllable classes discussed so far. The order is 
basically the same as that of final-syllable codas (see \autoref{tab:fincod}), 
though /ŋ/ supersedes /n/ and there is some attestation of final /h/. As noted 
above, the prevalence of /ŋ/ is due to the agent-topic marker \rayr{ANF}{ang} 
(see \autoref{ŋ}). /h/ only occurs in the interjections \rayr{AH!}{ah!} and 
\rayr{AAH!}{āh!}, so its status as an actual phoneme in this position is 
marginal at best.
\index{syllable!monosyllabic words|)}%

\subsection{Cross-Syllable Consonant Clusters}

\index{consonants!clusters|(}%
\begin{table}[thp]
\caption[Frequency of cross-syllable consonant clusters]{Frequency of cross-syllable consonant clusters (n\,=\,1270)}
\begin{tabu} to \linewidth {H[3c] X[17]}
\tableheaderfont\toprule
Interval [\pct{}]
	& \multicolumn1{c}{Consonant cluster}
	\\

\toprule

0.00 \dots{} 0.09
	& g.t, h.t, h.v, k.s, l.n, lv.n, m.bj, m.d, m.dʒ, m.l, m.n, m.pr, m.r, n.dv, n.g, n.h, n.w, ŋ.dʒj, ŋ.kw, ŋ.m, ŋ.n, ŋ.rj, ŋ.t, ŋk.t, r.b, r.dʒ, r.g, r.l, r.m, r.sj, r.tʃ, r.v, s.dʒ, s.h, s.l, s.n, s.p, s.v, t.v, tʃ.v (0.08\pct)
	\\

\midrule

0.10 \dots{} 0.24
	& l.bj, m.br, m.t, n.s, ŋ.b, ŋ.h, ŋ.p, ŋ.w, r.dʒj, r.pj, s.dʒj, s.m, t.l (0.16\pct); l.dʒ, l.p, m.k, n.sj, ŋ.dʒ, ŋ.g, ŋ.s, r.pr (0.24\pct)
	\\

0.25 \dots{} 0.49
	& m.v, r.s, s.r (0.31\pct); n.r, s.t (0.39\pct); m.pj, n.dʒj, r.d (0.47\pct)
	\\

\midrule

0.50 \dots{} 0.74
	& ŋ.kj, ŋ.v, r.k, r.n (0.55\pct); l.b, l.t, ŋ.r (0.71\pct)
	\\
	
0.75 \dots{} 1.00
	& r.p, r.t (0.87\pct); l.vj (0.94\pct)
	\\

\midrule

1.0 \dots{} 2.4
	& m.j (1.18\pct); ŋ.l (1.34\pct); n.tʃ (1.50\pct); n.dʒ (2.13\pct); n.v (2.28\pct); l.j (2.36\pct)
	\\

2.5 \dots{} 4.9
	& m.p (2.52\pct); s.j (2.60\pct); n.l (2.91\pct); l.v (3.15\pct); m.b (3.23\pct); ŋ.k (3.78\pct)
	\\

\midrule

5 \dots{} 9
	& n.t (5.28\pct); n.d (6.85\pct); ŋ.j (7.32\pct); r.j (8.98\pct)
	\\

10+
	& n.j (25.35\pct)
	\\

\bottomrule

\end{tabu}
\label{tab:xsylclusters}
\end{table}

Since a table detailing every combination with its absolute and relative 
frequency would be too large here, \autoref{tab:xsylclusters} gives the 
attested combinations ordered by brackets. As can be expected, bilabials cluster 
mostly with bilabials (74.11\pct), alveolars with alveolars (33.44\pct), and 
velars with velars (28.51\pct). However, at least for alveolars and velars, the 
score is even higher with /j/: 52.64\pct{} and 44.93\pct{}, respectively. /j/ 
is also the most common second consonant overall, at 47.8\pct{} of all 
consonant clusters; /n.j/ is the most common cluster at 25.35\pct{}. Alveolars 
provide the highest variety of both first and second consonants, with 6 
different phonemes making up 74.65\pct{} of C₁, and 8 different phonemes making 
up 28.74\pct{} of C₂.

Labiodentals and glottals occur least frequently, on the other hand: There is 
only one cluster with /v/ as a first consonant, namely, /lv.n/ (0.08\pct). 
For /h/, there are two, which are /h.v/ and /h.t/ (0.16\pct). Altogether, 
however, there are 97 combinations in /v/ (7.64\pct)---most commonly /l.v/ 
(3.15\pct) and /n.v/ (2.28\pct)---while there are only 4 in /h/ (0.31\pct): 
/n.h/, /s.h/, and twice /ŋ.h/.

At 924 first consonants (72.76\pct), the nasals /m/, /n/, and /ŋ/ make up the 
largest group going by manner of articulation, followed by the tap /r/, which 
appears 175 times as the first consonant (13.78\pct). For second consonants, 
approximants constitute the largest group at 669 combinations (52.68\pct), 
followed by 387 pairs with plosives second (30.47\pct).
\index{consonants!clusters|)}%

\section{Notes on Prosody}

\subsection{Stress}

Ayeri uses dynamic stress, that is, stress is primarily based on differences in 
the loudness of syllables. Which syllable is stressed depends on a mix of which 
position in a word a syllable occupies and the phonemic shape of syllables. 
English, on the other hand, possesses a system where a certain syllable in a 
word will stay stressed even if prefixes or suffixes are added to a word:

\ex
	\rc{English}%
	\textit{establish} /ɪˈstæblɪʃ/\\
	\textit{establishment} /ɪˈstæblɪʃmənt/\\
	\textit{disestablish} /dɪsɪˈstæblɪʃ/
\xe

In all cases, stress stays on the second syllable of \textit{establish}, 
whether a prefix or a suffix is added. Thus, it is not possible to predict the 
stressed syllable in a given word without knowing something about its 
morphology---one cannot simply count $n$ syllables from the front or the 
back and reach a valid conclusion. German may be an even more illustrative 
example than English here, as it is still richer in morphology than English:

\ex
	\rc{German}%
	\textit{reden} /ˈreːdən/ `talk' (talk-\Inf{})\\
	\textit{redete} /ˈreːdətə/ `talked' (talk-\Pst{}-1/3\Sg{})\\
	\textit{geredet} /gəˈreːdət/ `talked' (\Ptcp{}-talk-\Ptcp{})\\
	\textit{überredete} /ˌyːbərˈreːdətə/ `persuaded' (over.talk-\Pst{}-1/3\Sg{})
\xe

In all these words, stress remains on (the first syllable of) the word stem, as 
is typical of modern Germanic languages, no matter whether one or several 
affixes are added.\footnote{\citet[282--???]{wiese1996} elaborates that ...}
The position of this syllable in a given inflected word is 
effectively variable, though, so counting syllables is, again, no use. In Ayeri,
complications are slightly different. To demonstrate, the complete declension 
paradigm for \xayr{niv}{niva}{eye} is given in \autoref{fig:nivadecl}.

\begin{figure}[ht]
\caption{Declension paradigm for Ayeri \xayr{niv}{niva}{eye}}
\begin{tabu} to \linewidth {X[1] I[2] X[3] I[2] X[3]}
\tableheaderfont\toprule

	& \multicolumn2{c}{Singular}
	& \multicolumn2{c}{Plural}
	\\

\midrule
	
\Top{}
	& ní.va
	& `the eye'
	%
	& ni.vá.ye
	& `the eyes'
	\\

\midrule

\Aarg{}
	& ni.vā́ng
	& `eye'
	%
	& ni.va.jáng
	& `eyes'
	\\

\Parg{}
	& ni.vā́s
	& `eye'
	%
	& ni.vá.jas
	& `eyes'
	\\

\Dat{}
	& ni.vá.yam\footnotemark
	& `to the eye'
	%
	& ni.vá.jyam
	& `to the eyes'
	\\

\midrule

\Gen{}
	& ni.vá.na
	& `of the eye'
	%
	& ni.va.yé.na
	& `of the eyes'
	\\
	
\Loc{}
	& ni.vá.ya
	& `at the eye'
	%
	& ni.vá.jya
	& `at the eyes'
	\\

\Caus{}
	& ni.va.í.sa
	& `due to the eye'
	%
	& ni.va.jí.sa
	& `due to the eyes'
	\\

\Ins{}
	& ni.vá.ri
	& `with the eye'
	%
	& ni.va.yé.ri
	& `with the eyes'
	\\

\bottomrule
\end{tabu}
\label{fig:nivadecl}
\end{figure}

\footnotetext{Final-syllable stress is possible as well, also in the plural.}

It may appear that in the table above, stress is always on the penultimate 
syllable, which is indeed the case for most forms quoted there, but compare 
the superficially unmarked form \rayr{nis}{nisa}, which is bisyllabic with 
stress on the first (=\,penultimate) syllable, to the agent and patient singular 
forms, \rayr{nivaaNF}{nivāng} and \rayr{nivaasF}{nivās}, respectively. These are 
also bisyllabic, however, they are stressed on the second (=\,ultimate) syllable. 
Similarly, compare the agent and patient plural forms to each other: the agent 
plural form \rayr{nivye\_aNF}{nivajang} is trisyllabic and has its main stress 
on the third (=\,ultimate) syllable, while the equally trisyllabic patient 
plural form \rayr{nivye\_asF}{nivajas} is stressed on the second 
(=\,penultimate) syllable again.

It should have become clear that even though the basic form \rayr{nis}{nisa} 
has first-syllable stress, \textit{ni} will not necessarily carry stress across 
the whole paradigm, as it would be the case in English or German. It should also 
have become clear that the basic algorithm to determine stressed syllables in 
Ayeri is based on counting syllables from the right edge of a word, although 
some complications need to be factored in.

The basic foot in Ayeri is a trochee, and it does not matter whether the 
syllable is open or closed, or whether there are complex onsets or codas, or
no onsets or codas:

\pex
\a\begingl
	\gla ×́		{}	×	||	{} //
	\glb \fw{ha}	-	\fw{ri}	{}	{`pithy, striking'} //
\endgl

\a\begingl
	\gla ×́		{}	×		||	{} //
	\glb \fw{sa}	-	\fw{yan}	{}	{`hole, cave'} //
	\glb \fw{sem}	-	\fw{ba}		{}	{`comb'} //
\endgl

\a\begingl
	\gla ×́		{}	×		||	{} //
	\glb \fw{bri}	-	\fw{ha}		{}	{`grace'} //
	\glb \fw{ba}	-	\fw{brya}	{}	{`(he) mumbles'} //
	\glb \fw{a}	-	\fw{gu}		{}	{`chicken'} //
\endgl
\xe

That stress assignment is basically trochaic can be deduced from words with 
more than two syllables. Metricization furthermore runs from right to left:

\pex
\a\begingl
	\gla ×		|	×́		{}	×		|| 	{} //
	\glb \fw{ba}	-	\fw{ha}		-	\fw{lan}	{}	{`target, goal'} //
	\glb \fw{jar}	-	\fw{ma}		-	\fw{ya}		{}	{`pilgrim'} //
% 	\glb \fw{gin}	-	\fw{da}		-	\fw{ti}		{}	{`poet'} //
\endgl

\a\label{ex:4sylstress}\begingl
	\gla ×́		{}	×		|	×́		{}	×		||	{} //
	\glb \fw{ho}	-	\fw{ra}		-	\fw{ma}		-	\fw{ya}		{}	{`sinner'} //
%	\glb \fw{ke}	-	\fw{ba}		-	\fw{ye}		-	\fw{na}		{}	{`automatic'} //
	\glb \fw{ya}	-	\fw{ma}		-	\fw{na}		-	\fw{ti}		{}	{`causer'} //
\endgl
\xe

In the case of (\ref{ex:4sylstress}), the stressed syllables of the first foot 
bear secondary stress while those of the second foot bear primary stress. 
Complications, then, come in the form of syllables ending in /ŋ/, containing a 
long vowel, or containing a diphthong, or a combination of those features.
Ayeri does not have syllables that contain a diphthong and also end in /ŋ/, 
though, since consonant codas after a diphthong are largely avoided.%
\footnote{This would it possible to alternatively analyze diphthongs in /ɪ/ 
as /Vj/ sequences, essentially.} Thus, we can make a feature matrix:

\begin{figure}[h]\centering
\caption{Types of heavy syllables}
\scshape
\begin{tabu} to .75\linewidth {H X[c] X[c] X[c]}
\tableheaderfont\toprule

	& [+\,diph, –\,ŋ]
	& [–\,diph, +\,ŋ]
	& [–\,diph, –\,ŋ]
	\\
	
\toprule

[+\,long]
	& ++
	& ++
	& ++
	\\

[–\,long]
	& +
	& +
	& –
	\\

\bottomrule
\end{tabu}
\label{fig:heavysyl}
\end{figure}

As the name suggests, heavy syllables always carry stress, and they can be 
thought of as filling a whole foot by themselves:

\pex
\a\begingl
	\gla ×		|	–́		||	{} //
	\glb \fw{ma}	-	\fw{tay}	{}	{`summer, wet season'} //
	\glb \fw{pa}	-	\fw{dang}	{}	{`mind; heart, mood'} //
	\glb \fw{ka}	-	\fw{nāy}	{}	{`I marry' (marry=\Fsg{}.\Top{})} //
	\glb \fw{bras}	-	\fw{yāng}	{}	{`he bathes' (bathe=\TsgM{}.\Aarg{})} //
	\glb \fw{na}	-	\fw{rān}	{}	{`word; speech'} //
\endgl

\a\label{ex:firstheavy}\begingl
	\gla –́		|	×		˄	||	{} //
	\glb \fw{kār}	-	\fw{yo}		{}	{}	{`strong'} //
	\glb \fw{key}	-	\fw{nam}	{}	{}	{`humans, people'} //
	\glb \fw{kan}	-	\fw{ka}		{}	{}	{`mind; heart, mood'} //
\endgl
\xe

Unfortunately, there are no bisyllabic examples for the feature sets 
\textsc{[+\,long, \mbox{–\,diph,} +\,ŋ]} and \textsc{[+\,long, +\,diph, –\,ŋ]} 
in the first syllable. If there were, they would group with 
(\ref{ex:firstheavy}). Moreover, the analysis of stress in (\ref{ex:firstheavy}) 
as ×́ × would of course be equally possible. The difference may matter in verse, 
but it is not meaningful in ordinary speech.

So far, we have only looked at heavy syllables combined with regular ones. In 
this case, another property of heavy syllables will become apparent: long 
syllables outweigh those containing a diphthong or ending in /ŋ/. They are 
essentially superheavy, which is why the cells in \autoref{fig:heavysyl} are 
marked with two plus signs. The following example shows what happens when long 
syllables are combined with other heavy syllables:

\pex
\a\begingl
	\gla –		|	–́		||	{} //
	\glb \fw{bay}	-	\fw{hāy}	{}	{`I govern' (govern=\Fsg{}.\Top{})} //
	\glb \fw{say}	-	\fw{lyang}	{}	{`I sail' (sail=\Fsg{}.\Aarg{})} //
	\glb \fw{kay}	-	\fw{vān}	{}	{`container'} //
\endgl

\a\begingl
	\gla –		|	–́		||	{} //
	\glb \fw{kong}	-	\fw{āyn}	{}	{`we enter' (enter=\Fpl{}.\Top{})} //
	\glb \fw{keng}	-	\fw{vāng}	{}	{`you notice' (notice=\Ssg{}.\Aarg{})} //
	\glb \fw{lang}	-	\fw{-vā}	{}	{`in the most tiresome way' (tiresome=\Supl{})} //
\endgl
\xe

Again, initial \textsc{[+\,long, +\,diph, –\,ŋ]} and 
\textsc{[+\,long, –\,diph, +\,ŋ]}, as well as final 
\textsc{[–\,long, +\,diph, +\,ŋ]} don't exist, thus only:

\ex
\begingl
	% [+ long, + diph, - ŋ] + [- long, + diph, + ŋ]
	% [+ long, - diph, + ŋ] + [- long, + diph, + ŋ]
	% [+ long, - diph, - ŋ] + [- long, + diph, + ŋ]
	%
	% [+ long, + diph, - ŋ] + [- long, - diph, + ŋ]
	% [+ long, - diph, + ŋ] + [- long, - diph, + ŋ]
	%
	\gla –́		|	–		||	{} //
	\glb \fw{cā}	-	\fw{nang}	{}	{`love' (love-\Aarg{})} //
\endgl
\xe

\pex
\a\begingl
	\gla –		|	–́		||	{} //
	\glb \fw{bay}	-	\fw{tang}	{}	{`blood'} //
\endgl

\a\begingl
	\gla –		|	–́		||	{} //
	\glb \fw{pang}	-	\fw{lay}	{}	{`goddess'} //
\endgl
\xe

Again, initial \textsc{[+\,long, +\,diph, –\,ŋ]} and 
\textsc{[+\,long, –\,diph, +\,ŋ]} doesn't happen, thus only:

\pex
	% [+ long, + diph, - ŋ] + [+ long, - diph, + ŋ]
	% [+ long, + diph, - ŋ] + [+ long, - diph, - ŋ]
	%
	% [+ long, - diph, + ŋ] + [+ long, + diph, - ŋ]
	% [+ long, - diph, + ŋ] + [+ long, - diph, - ŋ]
	%
\a\begingl
	\gla –́		|	–́		||	{} //
	\glb \fw{mā}	-	\fw{sāy}	{}	{`I traveled' (\Pst{}-travel=\Fsg.\Top{})} //
\endgl

\a\begingl
	\gla –́		|	–		||	{} //
	\glb \fw{nā}	-	\fw{reng}	{}	{`rather'} //
\endgl

\xe

