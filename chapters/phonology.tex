\chapter{Phoneme Inventory and Phonotactics}

This chapter will present charts depicting the phoneme inventory of Ayeri, 
give an analysis of the phonotactics of Ayeri's dictionary entries and also 
describe stress patterns.

\section{Phoneme Inventory}

\subsection{Consonants}

\begin{sidewaystable}[p]
\caption{Consonant inventory}
\begin{tabu} to \textwidth {H[2l] X[c] X[c] X[c] X[c] X[c] X[c] X[c] X[c] X[c] X[c] X[c] X[c]}
\toprule\tableheaderfont
	%
	& \multicolumn2{c}{Bilabials}
	& \multicolumn2{c}{Labiodentals}
	& \multicolumn2{c}{Alveolars}
	& \multicolumn2{c}{Palatals}
	& \multicolumn2{c}{Velars}
	& \multicolumn2{c}{Glottals}
	\\

\midrule

Plosives
	& p & b         	% Bilabials
	&   &           	% Labiodentals
	& t & d         	% Alveolars
	&   &           	% Palatals
	& k & ɡ \orth{g}	% Velars
	&   &           	% Glottals
	\\

\midrule

Affricates
	&             &            	% Bilabials
	&             &            	% Labiodentals
	& tʃ \orth{c} & dʒ \orth{j}	% Alveolars
	&             &            	% Palatals
	&             &            	% Glottals
	\\

\midrule

Nasals
	&   & m          	% Bilabials
	&   &            	% Labiodentals
	&   & n          	% Alveolars
	&   &            	% Palatals
	&   & ŋ \orth{ng}	% Velars
	&   &            	% Glottals
	\\

\midrule

Fricatives
	&   &  	% Bilabials
	&   & v	% Labiodentals
	& s &  	% Alveolars
	&   &  	% Palatals
	&   &  	% Velars
	& h &  	% Glottals
	\\

\midrule

Taps/Flaps
	&   &  	% Bilabials
	&   &  	% Labiodentals
	&   & r	% Alveolars
	&   &  	% Palatals
	&   &  	% Velars
	&   &  	% Glottals
	\\

\midrule

Approximants
	&   & (w)       	% Bilabials
	&   &           	% Labiodentals
	&   & l         	% Alveolars
	&   & j \orth{y}	% Palatals
	&   &           	% Velars
	&   &           	% Glottals
	\\

\bottomrule
\end{tabu}
\label{tab:consonants}
\end{sidewaystable}

At 17 consonants, Ayeri has a fairly mid-sized inventory. 
\autoref{tab:consonants} shows the full chart. The sound /w/ only occurs 
marginally in \rayr{huAAky}{huākaya} [ˈwaːkaja] `frog'. Other instances of it 
are allophones of /u/ followed by a vowel, for instance in \rayr{ru\_a/}{rua-} 
/rwa/ `have to, must'. /w/ may also be an allophone of /uj/, as in 
\rayr{Adauyi}{adauyi} [aˈdawi] `then', \rayr{Edauyi}{edauyi} [eˈdawi] `now', or 
\rayr{nekuyi}{nekuyi} [ˈnekwi] `eyebrows'. The negative suffix \rayr{/Oj}{-oy} 
is also commonly contracted to [w] before a diphthong:

\ex
	\rayr{miNojAj}{ming\textbf{oy}ay → ming\textbf{u}ay} [mɪŋˈwaɪ] `I cannot' (can-\Neg{}-\Fsg{})
\xe

Moreover, the affricates /tʃ/ and /dʒ/ are usually allophones of /tj kj/ and 
/dj gj/, respectively. The plural marker \rayr{/ye}{-ye} is also commonly 
contracted to [dʒ] when a case suffix beginning with a vowel follows:

\pex
	\a \rayr{nYaanFye\_aNF}{nyān\textbf{ye}ang → nyān\textbf{j}ang} [ˈnjaːndʒaŋ] `persons' (person-\Pl{}-\Aarg{})
	\a \rayr{netuye\_asF}{netu\textbf{ye}as → netu\textbf{j}as} [neˈtudʒas] `brothers' (brother-\Pl{}-\Parg{})
\xe

The plural marker also may contract before the locative marker \rayr{/y}{-ya}, 
basically for dissimilation:\footnote{\rayr{/E\_a}{-ea} also occurs as an 
variant morpheme, so that \rayr{/ye}{-ye} + \rayr{/E\_a}{-ea} → 
\rayr{/yee\_a}{-yēa}.}

\ex
	\rayr{nivyey}{niva\textbf{ye}ya → niva\textbf{j}ya} [niˈvadʒja] `at the eyes' (eye-\Pl{}-\Loc{})
\xe

Dissimilation of the sequence \rayr{/yy}{-yaya} is attested in 
\citet[12]{becker:kafka:imperial}, where the relative pronoun 
\rayr{siyy}{siyaya} appears transcribed as \textit{sijya}:

\begin{quote}
As far as morphophonology is concerned, the relative pronoun complex 
\textit{sijya} `in/at/on which.\Loc{}' is interesting in so far as it is a 
contraction of \textit{*siyaya} `\Rel{}-\Loc{}-\Loc{}' that I introduced here
[...] Since this feature does not occur in previous texts, let's assume it's an 
acceptable variant.
\end{quote}

\noindent It is noted, however, that the contraction happens 
\textcquote[12]{becker:kafka:imperial}{only if both parts are grammatical 
suffixes}.

While vowels become long when two identical vowels come into succession,  
consonants do not geminate but are treated like a single consonant:

\pex
	\a \rayr{tvFvaanF}{ta\textbf{vv}āng} [taˈvaːŋ] `you get' (get=\Ssg{}.\Aarg{})
	\a \rayr{diʲsyNF}{dis\textbf{yy}ang} [diˈsjaŋ] `I fasten' (fasten=\Fsg{}.\Aarg{})
\xe

\subsection{Vowels}

Ayeri has a very basic five-vowel system, shown in \autoref{tab:vowels}.

\begin{table}[ht]\centering
\caption{Vowel inventory}
\begin{tabu} to .5\textwidth{H X[c] X[c] X[c]}
\toprule\tableheaderfont

	& Front
	& Center
	& Back
	\\

\toprule

High
	& i, iː
	&
	& u, uː
	\\

Mid
	& e, eː
	& (ə)
	& o, oː
	\\

Back
	&
	& a, aː
	&
	\\

\bottomrule
\end{tabu}
\label{tab:vowels}
\end{table}

The lax vowels [ɪ ɛ ɔ ʊ] occur as allophones of their tense counterparts 
[i e o u] in closed syllables, for example:

\pex
	\a \rayr{miNF}{m\textbf{ing}} [mɪŋ] `can, be able',
	\a \rayr{EnFy}{\textbf{en}ya} [ˈɛnja] `everyone',
	\a \rayr{AgonF}{ag\textbf{on}} [ˈaɡɔn] `outer, foreign', and
	\a \rayr{pkurF}{pak\textbf{ur}} [ˈpakʊr] `ill, sick'.
\xe

/ə/ is a marginal phoneme and only occurs in the tense prefixes 
\xayr{k/}{kə-}{\NPst{}}, \xayr{m/}{mə-}{\Pst{}}, \xayr{v/}{və-}{\RPst{}}, as 
well as in the prefix \xayr{me/}{mə-}{some, whichever}. Otherwise, [ə] occurs 
as an allophone of /e/ in final unstressed position, e.g. in the word 
\rayr{mine}{min\textbf{e}} [ˈminə] `affair, matter, issue'.

Ayeri also possesses a number of diphthongs, these are: /aɪ aːɪ eɪ ɔɪ ʊɪ aʊ/.
Furthermore, the vowels [i e a o u] may be long: [iː eː aː oː uː]. Long vowels 
are lexicalized in a few words, for example:

\pex
	\a \xayr{niis}{nīsa}{wanted}, \xayr{psiis}{pasīsa}{interesting};
	\a \xayr{AreenF}{arēn}{anyway, however}, \xayr{leer}{lēra}{whore};
	\a \xayr{laa}{lā}{tongue}, \xayr{yaaNF}{yāng}{he.\Aarg{}}; \label{ex:laa}
	\a \xayr{noonF}{nōn}{wish}; and 
	\a \xayr{bbuu}{babū}{barbarian}.
\xe

\noindent Otherwise, long vowels result from two same vowels next to each other, 
for instance:

\ex \xayr{AgY/}{aja-}{play} + \xayr{/AnF}{-an}{\Nmlz{}} → \xayr{AgYaanF}{ajān}{game, play}. \xe

Morphophonologically, long vowels also occur in double-marked relative pronouns 
where the agreement marker for the relative clause's head has been omitted,
for instance, \xayr{sinaa}{sinā}{of which, about which}, as in the following 
example:

\ex\begingl
	\gla Le turayāng taman sinā ang ningay tamala vās. //
	\glb Le tura-yāng taman-Ø si-Ø-na ang ning=ay.Ø tamala vās //
	\glc \PatTI{} send=\Tsg{}.\M{}.\Aarg{} letter-\Top{} \Rel{}-\PatTI{}-\Gen{} \AgtT{} tell=\Fsg{}.\Top{} yesterday \Ssg{}.\Parg{} //
	\glft `The letter which I told you about yesterday, he sent it.' //
\endgl\xe

This is to disambiguate it from the plain genitive-marked relative pronoun 
\xayr{sin}{sina}{which.\Gen{}}:\footnote{A variant which combines the 
allomorphs of the relativizer and the genitive case marker in the opposite way 
also exists: \rayr{s/}{s-} + \rayr{/En}{-ena} → \rayr{sen}{sena}.}

\ex\begingl
	\gla tamanang ledanena nā sina koronvāng //
	\glb taman-ang ledan-ena nā si-na koron-vāng //
	\glc letter-\Aarg{} friend-\Gen{} \Fsg.\Gen{} \Rel{}-\Gen{} know=\Ssg{}.\Aarg{} //
	\glft `the letter of my friend which you know' //
\endgl\xe

As pointed out in (\ref{ex:laa}), the word \xayr{laa}{lā}{tongue} ends in a 
long vowel, so the question is what happens when a case suffix beginning with a 
vowel is appended. To avoid a hiat, a glide /j/ may be inserted, so both of 
these are possible:

\pex
	\a\begingl
		\gla Aku lāas! //
		\glb Aka-u lā-as //
		\glc swallow-\Imp{} tongue-\Parg{} //
		\glft `Shut up!' //
	\endgl
	\a\begingl
		\gla {Aku lāyas!} //
		\glb (idem) //
	\endgl
\xe

\section{Phonotactics}

For the purpose of this statistical analysis, all of the available translations 
into Ayeri since 2008 have been used as a text corpus; example sentences from 
various blog articles have also been added, as well as dictionary entries for 
all nouns, adjectives, adverbs, pronouns, adpositions, conjunctions, and 
numerals if they were not prefixes or suffixes.\footnote{This section updates 
and extends a previous analysis of the phonological makeup of dictionary entries 
\autocite{becker:frequency}. The previous study had its focus on gathering 
frequency statistics for word generation, however, we want to know about words 
generally here.} Borrowings have been deleted, if they could not reasonably be 
words in Ayeri. Altogether, the corpus comprises 5,499 words; words may occur 
more than once.

Among the dictionary entries, verbs have notably been ignored, since verb stems 
alone do not constitute independent words -- they are always inflected in some 
way, so that they may end in consonants or consonant clusters that independent 
words cannot end in. This also has repercussions on syllabification and stress, 
which depend on the inflection of the verb stem:

\begin{table}[h]
\caption{Syllabification of inflected verbs}
\begin{tabu} to \linewidth {X[2l] X[3c] X[3c] X[3c]}
\toprule\tableheaderfont
Suffix
	& \emph{ca-} `love'
	& \emph{gum-} `work'
	& \emph{babr-} `mumble'
	\\

\toprule

\emph{-ay} (\Fsg{})
	& cā́y
	& gu.máy
	& ba.bráy
	\\

\emph{-va} (\Ssg{})
	& cá.va
	& gúm.va
	& ba.brá.va
	\\

\emph{-yam} (\Ptcp{})
	& cá.yam
	& gúm.yam
	& bá.bryam
	\\

\bottomrule
\end{tabu}
\label{ex:verbsyll}
\end{table}

% The statistics have been aggregated by \tit{\citetitle{strasser:freq}} 
% \autocite{strasser:freq}. 
For the purpose of gathering statistics on phonemes, 
the words from translation texts were converted to IPA first. Fortunately, this 
is rather easy as Ayeri's romanization is very straightforward. Syllable breaks 
have also been inserted semi-automatically.

\subsection{Number of Syllables per Word}

First, let us see how many syllables words commonly have (see 
\autoref{tab:syllength}). The higher the syllable count, the more likely it is for 
them to be compounds or inflected words.

\begin{table}[hp]\centering
\caption[Relative frequency of words with different numbers of syllables]{Relative frequency of words with different numbers of syllables (n\,=\,5499)}
\begin{tabu} to .5\textwidth{X X[c] X[c]}
\tableheaderfont\toprule
Segment
	& Count
	& \multicolumn1{c}{Percentage}
	\\
\toprule

2 syllables
	& 2277
	& 41.41\pct
	\\
	
3 syllables
	& 1392
	& 25.31\pct
	\\
	
1 syllable
	& 1201
	& 21.84\pct
	\\
	
4 syllables
	& 547
	& 9.95\pct
	\\
	
5 syllables
	& 74
	& 1.35\pct
	\\
	
6 syllables
	& 8
	& 0.15\pct
	\\
	
\bottomrule
\end{tabu}
\label{tab:syllength}
\end{table}

Two-syllable words make up the bulk of the sample, which is not surprising since 
1,072 (55.43\pct) of the dictionary subsample are bisyllabic words. Most of 
Ayeri's roots are bisyllabic; unsurprisingly, most monosyllabic words are 
function words like the ones cited below. A few examples for each number of 
syllables per word:

\pex
	\a \xayr{yeNF}{yeng}{she.\Aarg{}},\\
		\xayr{le}{le}{\PatT},\\
		\xayr{ru\_a}{rua}{must};
		
	\a \xayr{dtau}{datau}{normal},\\
		\xayr{mreNF}{mareng}{suffice=\TsgI{}.\Aarg{}},\\
		\xayr{nsj}{nasay}{near to};
		
	\a \xayr{AvnFyaaNF}{avanyāng}{sink=\TsgM{}.\Aarg{}},\\
		\xayr{nraanFye}{narānye}{word-\Pl{}},\\
		\xayr{tovlej}{tovaley}{cloak-\PargI{}};
		
	\a \xayr{hinYnFveno}{hinyanveno}{corner.beautiful} (likely a place name),\\
		\xayr{mNstoNF}{mangasatong}{move-\Hab{}=\TplN{}.\Aarg{}},\\
		\xayr{mitnen}{mitanena}{palace-\Gen{}};
		
	\a \xayr{hruymnsF}{haruyamanas}{beat-\Ptcp{}-\Nmlz{}-\Parg{}},\\
		\xayr{sirutyen}{sirutayena}{night-\Gen{}},\\
		\xayr{suMkornFkihsF}{sungkorankihas}{science.map};
		
	\a \xayr{kjtomynen}{kaytomayanena}{righteous-\Nmlz{}-\Gen{}},\\
		\xayr{koronrYsynF}{koronaryasayan}{forget-\Hab{}-\TplM{}},\\
		\xayr{nsimyye\_aNF/henF}{nasimayajang-hen}{follow-\Agtz-\Pl{}-\Aarg{}=all}.
\xe

\autoref{tab:syltype} shows the frequencies of syllable types by position in a 
word. It is important to note here that phonemes which consist of more than one 
segment -- affricates, diphthongs, and long vowels -- have been counted as only 
one of C (consonant) or V (vowel), respectively. The following subsections will 
elaborate on which sounds the Cs and Vs correspond to.

\begin{sidewaystable}[hp]\centering
\caption[Relative frequency of syllable types per word]{Relative frequency of syllable types per word (n\,=\,5499)}
\begin{tabu} to \textwidth{H X[c] S[c] X[c] S[c] X[c] S[c] X[c] S[c] X[c] S[c]}
\tableheaderfont\toprule
Type
	& \multicolumn2{c}{Initial}
	& \multicolumn2{c}{Medial}
	& \multicolumn2{c}{Final}
	& \multicolumn2{c}{Single}
	& \multicolumn2{c}{Total}
	\\
	
\toprule
	
CV
	& 2903
	& 67.54\pct
	& 1972
	& 71.97\pct
	& 2108
	& 49.05\pct
	& 578
	& 48.13\pct
	& 7561
	& 60.31\pct
	\\
	
CCV
	& 55
	& 1.28\pct
	& 25
	& 0.91\pct
	& 47
	& 1.09\pct
	& 32
	& 2.66\pct
	& 159
	& 1.27\pct
	\\
	
CCCV
	& \multicolumn2{c}{—}
% 	& 0
% 	& 0.00\pct
	& \multicolumn2{c}{—}
% 	& 0
% 	& 0.00\pct
	& 2
	& 0.05\pct
	& \multicolumn2{c}{—}
% 	& 0
% 	& 0.00\pct
	& 2
	& 0.02\pct
	\\
	
CVC
	& 754
	& 17.54\pct
	& 614
	& 22.41\pct
	& 1903
	& 44.28\pct
	& 298
	& 24.81\pct
	& 3569
	& 28.47\pct
	\\
	
CCVC
	& 29
	& 0.67\pct
	& 10
	& 0.36\pct
	& 88
	& 2.05\pct
	& 9
	& 0.75\pct
	& 136
	& 1.08\pct
	\\
	
CVCC
	& 1
	& 0.02\pct
	& \multicolumn2{c}{—}
% 	& 0
% 	& 0.00\pct
	& \multicolumn2{c}{—}
% 	& 0
% 	& 0.00\pct
	& \multicolumn2{c}{—}
% 	& 0
% 	& 0.00\pct
	& 1
	& 0.01\pct
	\\

\midrule

V
	& 488
	& 11.35\pct
	& 95
	& 3.47\pct
	& 67
	& 1.56\pct
	& 2
	& 0.17\pct
	& 652
	& 5.20\pct
	\\
	
VC
	& 68
	& 1.58\pct
	& 24
	& 0.88\pct
	& 83
	& 1.93\pct
	& 282
	& 23.48\pct
	& 457
	& 3.65\pct
	\\
	
\bottomrule
	
Total
	& 4298
	& 100.00\pct
	& 2740
	& 100.00\pct
	& 4298
	& 100.00\pct
	& 1201
	& 100.00\pct
	& 12537
	& 100.00\pct
	\\

\bottomrule
\end{tabu}
\label{tab:syltype}
\end{sidewaystable}

In all positions, CV is the most common syllable type, followed by CVC. With a 
very big margin, V is the next most common syllable type, which is also most 
common in initial syllables and least common in monosyllabic words. The cases 
with only a few attestations are the following:

\pex
	\a Initial CVCC:\\
		\rayr{silFvFnNF}{silvnang} /sɪlv.ˈnaŋ/ `see=\Fpl{}.\Aarg{}';
		
	\a Final CCCV:\\
		\rayr{migFrFyo}{migryo} /ˈmi.ɡrjo/ `flourish-\Tsg{}.\N{}',\\
		\rayr{subFrFyo}{subryo} /ˈsu.brjo/ `cease-\Tsg{}.\N{}';
	
	\a Single V:\\
		\rayr{Aj}{ay} /aɪ/ `\Fsg{}.\Top{}'.
\xe

The medial and final VC cases may seem like an oddity, but they are mostly due 
to the previous syllable ending in /ŋ/, with that syllable also containing a 
lax vowel, which means that this syllable must be closed. An alternative 
explanation would be to assume that /ŋ/ is ambisyllabic, or actually /n.ɡ 
\textasciitilde{} ŋ.ɡ/, but realized as [ŋ]. The high number of single-syllable 
VC is due to \xayr{ANF}{ang}{\AgtT}, which alone appears 255 times in the 
sample (4.63\pct{} of all words, 21.23\pct{} of monosyllabic words, 90.43\pct{} 
of monosyllabic VC words).

\subsection{Phonemic Makeup of Initial Syllables}

...

\subsection{Phonemic Makeup of Initial Syllables}

...

\subsection{Phonemic Makeup of Final Syllables}

...

\subsection{Phonemic Makeup of Single Syllables}

...
