\chapter{Grammatical categories}
\label{ch:gramcat}

While the previous chapter was about general mechanisms of morphological
marking in Ayeri, this chapter will dive into the various parts of speech in
order to describe their morphology with a closer look. I will begin with nouns
as the main carriers of meaning, thhn deal with other parts of speech that
regularly feature in the noun phrase or in combination with it---pronouns,
adjectives, and adpositions. Following this, there will be a discussion of
verbs and adverbs before moving on to numerals and conjunctions.

\section{Nouns}
\label{sec:nouns}
\index{nouns|(}

Nouns in Ayeri have gender\index{gender} and number as their inherent
grammatical properties. Besides common nouns, there are, of course, also proper
nouns (that is, names) and nominalizations. Nouns, as the heads of NPs\index{phrase types!noun phrase}, are
also assigned case by the verb\index{verbs}, which is a third grammatical property
they display. For an illustration of the declension paradigms, compare Tables
\ref{tab:anideclcons} to \ref{tab:inandeclvow}.

\begin{table}[t]
\caption[Declension paradigm for \xayr{bdnF}{badan}{father}]{Declension 
paradigm for \xayr{bdnF}{badan}{father} (animate; consonantal root)}
\begin{tabu} to \linewidth {X[1] I[2] X[4] I[2] X[4]}
\tableheaderfont\toprule

	& \multicolumn2{c}{Singular}
	& \multicolumn2{c}{Plural}
	\\

\midrule
	
\Top{}
	& badan
	& `the father'
	%
	& badanye
	& `the fathers'
	\\

\midrule

\Aarg{}
	& badanang
	& `father'
	%
	& badanjang
	& `fathers'
	\\

\Parg{}
	& badanas
	& `father' (obj.)
	%
	& badanjas
	& `fathers' (obj.)
	\\

\Dat{}
	& badanyam
	& `to the father'
	%
	& badanjyam
	& `to the fathers'
	\\

\midrule

\Gen{}
	& badanena
	& `of the father'
	%
	& badanyena
	& `of the fathers'
	\\
	
\Loc{}
	& badanya
	& `at/in the father'
	%
	& badanjya
	& `at/in the fathers'
	\\

\Caus{}
	& badanisa
	& `due to the father'
	%
	& badanjisa
	& `due to the fathers'
	\\

\Ins{}
	& badaneri
	& `with the father'
	%
	& badanyeri
	& `with the fathers'
	\\

\bottomrule
\end{tabu}
\label{tab:anideclcons}
\end{table}

\begin{table}[t]
\caption[Declension paradigm for \xayr{maav}{māva}{mother}]{Declension 
paradigm for \xayr{maav}{māva}{mother} (animate; vocalic root)}
\begin{tabu} to \linewidth {X[1] I[2] X[4] I[2] X[4]}
\tableheaderfont\toprule

	& \multicolumn2{c}{Singular}
	& \multicolumn2{c}{Plural}
	\\

\midrule
	
\Top{}
	& māva
	& `the mother'
	%
	& māvaye
	& `the mothers'
	\\

\midrule

\Aarg{}
	& māvāng
	& `mother'
	%
	& māvajang
	& `mothers'
	\\

\Parg{}
	& māvās
	& `mother' (obj.)
	%
	& māvajas
	& `mothers' (obj.)
	\\

\Dat{}
	& māvayam
	& `to the mother'
	%
	& māvajyam
	& `to the mothers'
	\\

\midrule

\Gen{}
	& māvana
	& `of the mother'
	%
	& māvayena
	& `of the mothers'
	\\
	
\Loc{}
	& māvaya
	& `at/in the mother'
	%
	& māvajya
	& `at/in the mothers'
	\\

\Caus{}
	& māvaisa
	& `due to the mother'
	%
	& māvajisa
	& `due to the mothers'
	\\

\Ins{}
	& māvari
	& `with the mother'
	%
	& māvayeri
	& `with the mothers'
	\\

\bottomrule
\end{tabu}
\label{tab:anideclvow}
\end{table}

\begin{table}[t]
\caption[Declension paradigm for \xayr{kirinF}{kirin}{street}]{Declension 
paradigm for \xayr{kirinF}{kirin}{street} (inanimate; consonantal root)}
\begin{tabu} to \linewidth {X[1] I[2] X[4] I[2] X[4]}
\tableheaderfont\toprule

	& \multicolumn2{c}{Singular}
	& \multicolumn2{c}{Plural}
	\\

\midrule
	
\Top{}
	& kirin
	& `the street'
	%
	& kirinye
	& `the streets'
	\\

\midrule

\Aarg{}
	& kirinreng
	& `street'
	%
	& kirinyereng
	& `streets'
	\\

\Parg{}
	& kirinley
	& `street' (obj.)
	%
	& kirinyeley
	& `streets' (obj.)
	\\

\Dat{}
	& kirinyam
	& `to the street'
	%
	& kirinjyam
	& `to the streets'
	\\

\midrule

\Gen{}
	& kirinena
	& `of the street'
	%
	& kirinyena
	& `of the streets'
	\\
	
\Loc{}
	& kirinya
	& `at/in the street'
	%
	& kirinjya
	& `at/in the streets'
	\\

\Caus{}
	& kirinisa
	& `due to the street'
	%
	& kirinjisa
	& `due to the streets'
	\\

\Ins{}
	& kirineri
	& `with the street'
	%
	& kirinyeri
	& `with the streets'
	\\

\bottomrule
\end{tabu}
\label{tab:inandeclcons}
\end{table}

\begin{table}[t]
\caption[Declension paradigm for \xayr{per}{pera}{measure}]{Declension 
paradigm for \xayr{per}{pera}{measure} (inanimate; vocalic root)}
\begin{tabu} to \linewidth {X[1] I[2] X[4] I[2] X[4]}
\tableheaderfont\toprule

	& \multicolumn2{c}{Singular}
	& \multicolumn2{c}{Plural}
	\\

\midrule
	
\Top{}
	& pera
	& `the measure'
	%
	& peraye
	& `the measures'
	\\

\midrule

\Aarg{}
	& perareng
	& `measure'
	%
	& perayereng
	& `measures'
	\\

\Parg{}
	& peraley
	& `measure' (obj.)
	%
	& perayeley
	& `measures' (obj.)
	\\

\Dat{}
	& perayam
	& `to the measure'
	%
	& perajyam
	& `to the measures'
	\\

\midrule

\Gen{}
	& perana
	& `of the measure'
	%
	& perayena
	& `of the measures'
	\\
	
\Loc{}
	& peraya
	& `at/in the measure'
	%
	& perajya
	& `at/in the measures'
	\\

\Caus{}
	& peraisa
	& `due to the measure'
	%
	& perajisa
	& `due to the measures'
	\\

\Ins{}
	& perari
	& `with the measure'
	%
	& perayeri
	& `with the measures'
	\\

\bottomrule
\end{tabu}
\label{tab:inandeclvow}
\end{table}

\subsection{Gender}
\label{subsec:gender}
\index{gender|(}

% \begin{figure}[tb]\centering
% \begin{forest}
% where n children=0{tier=word}{}, shorter edges,
% [grammatical gender
% 	[animate
% 		[masculine]
% 		[feminine]
% 		[neuter]
% 	]
% 	[inanimate]
% ]
% \end{forest}
% \caption{Grammatical genders in Ayeri}
% \label{fig:gramgend}
% \end{figure}

Grammatical gender in Ayeri consists of two tiers which are subdivided into
four classes based on a mixture of semantic and ontological properties, see 
(\ref{ex:gramgend}).

\begin{figure}
\ex\label{ex:gramgend}
\begin{forest}
where n children=0{tier=word}{}, shorter edges,
[grammatical gender
	[animate
		[masculine]
		[feminine]
		[neuter]
	]
	[inanimate]
]
\end{forest}
\xe
\end{figure}

The animate\index{animacy} gender refers, broadly speaking, to 
entities that are considered alive or are closely associated with living
things, such as events, concepts, or activities executed or connected to them.
The `masculine' and `feminine' subcategories are applied to humans, animals
whose sex is known (for example on behalf of breeding them or keeping them as
pets), and gods---basically anything that shows sexual dimorphism or is assumed
to be an exponent of it, as well as nouns referring to such entities in a
functional way, for instance, \xayr{bdnF}{badan}{father} and
\xayr{maav}{māva}{mother}. The remainder falls into the `neuter' 
category---plants, for instance, body parts, or animals whose sex is unknown. 
The `inanimate'\index{animacy} category typically contains materials and things, such as 
tools. Furthermore, animals and plants change their category to inanimate\index{animacy} as 
well if they serve as food. There are exceptions to either group, where 
elements appear in them for no obviously discernable reason. In order to 
illustrate, (\ref{ex:noungendex}) gives a few examples of each category.

\begin{figure}
\pex\label{ex:noungendex}
	\a Animate masculine:\medskip\\
		\xayr{\larger bdnF}{badan}{father}, 
		\xayr{\larger netu}{netu}{brother}, 
		\xayr{\larger AguynF}{aguyan}{rooster}, 
		\rayr{\larger AgYaanF}{Ajān}, 
		\rayr{\larger ltunF}{Latun}
		% FIXME: bull? stallion? dog?
	
	\a Animate feminine:\medskip\\
		\xayr{\larger maav}{māva}{mother}, 
		\xayr{\larger kin}{kina}{sister}, 
		\xayr{\larger Aguvj}{aguvay}{hen}, 
		\rayr{\larger mh}{Maha}, 
		\rayr{\larger tFraanj}{Trānay}
		% FIXME: cow? mare? bitch?
	
	\a Animate neuter:\medskip\\
		\xayr{\larger AdNF}{adang}{palm tree},
		\xayr{\larger bino}{bino}{color},
		\xayr{\larger IkmF}{ikam}{deer},
		\xayr{\larger kdaanF}{kadān}{harvest},
		\xayr{\larger tYaanF}{cān}{love},
		\xayr{\larger nN}{nanga}{house}, 
		\xayr{\larger nrymnF}{narayaman}{speaking},
		\xayr{\larger tMpu}{tampu}{luck},
		\xayr{\larger yil}{yila}{foot}
	
	\needspace{2\baselineskip}
	\a Inanimate:\medskip\\
		\xayr{\larger AhlF}{ahal}{sand},
		\xayr{\larger hem}{hema}{egg},
		\xayr{\larger khnF}{kahan}{spear},
		\xayr{\larger meluNF}{melung}{yogurt},
		\xayr{\larger nusaanF}{nusān}{damage},
		\xayr{\larger pyutaanF}{payutān}{mathematics}
\xe
\end{figure}

There are also a number of duplicates like French\index{French} \fw{le livre} `the book' and
\fw{la livre} `the pound', for instance, \ayr{bnnF} \fw{banan} (an.) 
`kindness, charity' or \ayr{bino} \fw{bino} (an.) `color' on the one hand, and
\ayr{bnnF} \fw{banan} (inan.)\ `quality' or \ayr{bino} \emph{bino} (inan.)\ 
`paint' on the other. Gender is reified by case marking\index{case} as well as verb\index{verbs}
agreement\index{agreement}; it is not possible to read the gender of a noun from its
phonological makeup. (\ref{ex:gender}) illustrates differences in case marking\index{case}
and agreement\index{agreement} (inherent information on grammatical features underneath the
NPs).

\begin{figure}
\pex\label{ex:gender}
\a\label{ex:gender1}\begingl
	\gla Ang @ konja badan hemaley. //
	\glb ang= kond-ya badan-Ø hema-ley //
	\glc {} {} {\tiny [\TsgM{}.\An{}}] {\tiny [\TsgI{}]} //
	\glc \AgtT{}.\An{}= eat-\TsgM{}.\An{} father-\Top{} egg-\PargI{} //
	\glft `Father eats an egg.' //
\endgl

\a\label{ex:gender2}\begingl
	\gla Sa @ tombara kahanreng burang. //
	\glb sa= tomb-ara kahan-reng burang-Ø //
	\glc {} {} {\tiny [\TsgI{}]} {\tiny [\TsgN{}.\An{}]} //
	\glc \PatT{}.\An{}= kill-\TsgI{} spear-\AargI{} animal-\Top{} //
	\glft `The animal, the spear kills it.' //
\endgl
\xe
\end{figure}

In example (\ref{ex:gender1}), the noun in the agent NP,
\xayr{bdnF}{badan}{father}, bears the features [\Gend{}~\M{}, \Anim{}~$+$],
which triggers the animate agent\index{semantic role!agent} topic agreement\index{agreement} marker \rayr{ANF}{ang} on the
verb\index{verbs}, since the agent\index{semantic role!agent} NP\index{phrase types!noun phrase} is also topicalized\index{grammatical function!topic}. The verb\index{verbs} also agrees\index{agreement} in person
and number with the agent\index{semantic role!agent} NP\index{phrase types!noun phrase} by way of the person marker \rayr{/y}{-ya} for
third person singular masculine. The object of the sentence,
\xayr{hem}{hema}{egg}, on the other hand, bears the feature
[\Gend{}~\Inan{}, \Anim{}~$-$], so it receives the inanimate\index{animacy} patient case
marker \rayr{/lej}{-ley} rather than its animate counterpart \rayr{/AsF}{-as}.

In (\ref{ex:gender2}), on the other hand, we see an inanimate\index{animacy} agent\index{semantic role!agent},
\xayr{khnF}{kahan}{spear}, so the verb\index{verbs} receives the marker \rayr{/Ar}{-ara} 
for third person singular inanimate\index{animacy} rather than its animate neuter counterpart
\rayr{/yo}{-yo}. The (non-topicalized) NP's\index{phrase types!noun phrase} case marking\index{case} shows that the agent\index{semantic role!agent} 
of the clause is inanimate\index{animacy}: \rayr{khnF}{kahan} carries the marker
\rayr{/reNF}{-reng}, which marks it as an inanimate\index{animacy} agent\index{semantic role!agent}. The object of the 
sentence, \xayr{burNF}{burang}{animal}, is also the topic, hence topic
agreement\index{agreement} on the verb\index{verbs} uses the marker \rayr{s}{sa} according to the NP\index{phrase types!noun phrase} being
animate, rather than its inanimate\index{animacy} counterpart \rayr{le}{le}.

\index{gender|)}

\subsection{Number}
\index{number|(}

Ayeri only distinguishes singular and plural in nouns, which receive plural 
marking; verbs\index{verbs}, then, agree\index{agreement} with agent NPs\index{phrase types!noun phrase} in number in the canonical case. 
Ordinarily, nouns in Ayeri are countable, however, there is also a group of 
uncountable nouns as well as a (small) group of nouns which are always plural. 
As above, I will list a few words from each group for illustration:

%\needspace{2\baselineskip}
\begin{figure}[h]
\pex[everyex={\tabcolsep=0em},]
	\a Countable nouns:\label{ex:plurals}\medskip\\
		\begin{tabular}[t]
		{l @{\enspace---\enspace} l}
		
		\xayr{\larger AgYmF}{ajam}{toy}
			& \xayr{\larger AgYmFye}{ajamye}{toys}, %
			\medskip \\
				
		\xayr{\larger devo}{devo}{head}
			& \xayr{\larger devoye}{devoye}{heads}, %
			\medskip \\
				
		\xayr{\larger InunF}{inun}{fish}
			& \xayr{\larger InunFye}{inunye}{fish} (pl.),%
			\medskip \\
				
		\xayr{\larger netu}{netu}{brother}
			& \xayr{\larger netuye}{netuye}{brothers};
			\\
		\end{tabular}
	
	\a Uncountable nouns:\medskip\\
		\xayr{\larger AhlF}{ahal}{sand}, 
		\xayr{\larger bkj}{bakay}{stuff}, 
		\xayr{\larger ghaanF}{gahān}{hope}, 
		\xayr{\larger miNnF}{mingan}{ability};
	
	\a Plurale tantum nouns:\medskip\\
		\xayr{\larger burNF}{burang}{lifestock, 
			cattle},\footnotemark~
		\xayr{\larger gneNnF}{ganengan}{siblings}, 
		\xayr{\larger kejnmF}{keynam}{people}, 
		\xayr{\larger tNF}{tang}{ears}.
\xe
\end{figure}

\footnotetext{Specifically in this meaning; \rayr{burNF}{burang} can also 
simply mean `animal', in which case there is a plural form 
\xayr{burNFye}{burangye}{animals}.}

Most concrete things that exist as discrete entities are countable, also, for
instance, animals and lifestock. Fish, deer, sheep etc.\ are thus countable,
unlike in English\index{English}; pants, pliers, scissors, glasses, etc.\ are by default\index{default}
singular\index{number!singular}, also unlike in English\index{English}. Uncountable, on the other hand, are materials
in general or abstract concepts. There are also a number of nouns which are
plural\index{number!plural} by default\index{default}, most notably entities which often occur in groups, but there
is as well the odd word for which there seems to be no reason to be included in
this group, for instance, \xayr{bino}{bino}{paint}, and
\xayr{giMbj}{gimbay}{sorrows}. A few body parts are also \fw{plurale tantum}
nouns, especially those which occur in pairs (\xayr{niv}{niva}{eye} is a
notable exception).

As demonstrated in (\ref{ex:plurals}), the noun plural\index{number!plural} marker is 
\rayr{/ye}{-ye}, which in native orthography also occurs in the variant 
\ayr{*Ye} or \ayr{ʲ*e} due to allography. As described above
(\autoref{pluralmorph}, p.~\pageref{pluralmorph}), the plural marker may also
be reduced to [dʒ] \orth{-j} before case\index{case} suffixes\index{suffixes} beginning with /j/ or with a
vowel other than /e/, like \rayr{/ANF}{-ang} (\Aarg{}) or \rayr{/ymF}{-yam}
(\Dat{}), as demonstrated in (\ref{ex:pluralaltern}). For \fw{pluralia tantum},
to express a singular\index{number!singular} entity, it is always possible to use a genitive phrase
like \xayr{—/En menF}{…-ena men}{one of …} (…-\Gen{} one), like in
(\ref{ex:dualplural}).

\begin{figure}[h]
\ex\labels\label{ex:pluralaltern}
	\begin{tabular}[t]{@{\tl\quad} l @{\enspace→\enspace} l @{\smallskip}}
	\rayr{\larger dirnNF}{diranang} (uncle-\Aarg{})
		+ \rayr{\larger /ye}{-ye} (\Pl{})
		& \rayr{\larger dirnFye\_aNF}{diranjang} (uncle-\Pl{}-\Aarg{}),
		\\
	\rayr{\larger dirnen}{diranena} (uncle-\Gen{})
		+ \rayr{\larger /ye}{-ye} (\Pl{})
		& \rayr{\larger dirnFyen}{diranyena} (uncle-\Pl{}-\Gen{}),
		\\
	\rayr{\larger dirnFymF}{diranyam} (uncle-\Dat{})
		+ \rayr{\larger /ye}{-ye} (\Pl{})
		& \rayr{\larger dirnFyeymF}{diranjyam} (uncle-\Pl{}-\Dat{}).
		\\
	\end{tabular}
\xe
\end{figure}

\begin{figure}[h]
\pex\label{ex:dualplural}
\a\begingl
	\gla Nupayon tangang nā. //
	\glb nupa-yon tang-ang nā //
	\glc hurt-\TplN{} ears-\Aarg{} \Fsg{}.\Gen{} //
	\glft `My ears hurt.' //
\endgl

\a\label{ex:gensubj}\begingl
	\gla Na @ nupareng tang men nā. //
	\glb na= nupa=reng tang-Ø men nā //
	\glc \GenT{}= hurt=\TsgI{}.\Aarg{} ears-\Top{} one \Fsg{}.\Gen{} //
	\glft `Of my ears, one is hurting.' //
\endgl
\xe
\end{figure}

Number in nouns can also be manipulated by quantifiers\index{quantifiers} which attach to declined
nouns as suffixes\index{suffixes}---or rather, enclitics. In this case, when plurality is
indicated by the quantifier\index{quantifiers}, the noun is not additionally marked for number;
the verb\index{verbs}, however, keeps agreeing\index{agreement} in number. This is illustrated in
(\ref{ex:quantplur}).

\begin{figure}[h]
\pex\label{ex:quantplur}
\a\begingl
	\gla Ajayon ganjang kivo. //
	\glb aja-yon gan-ye-ang kivo //
	\glc play-\TplN{} child-\Pl{}-\Aarg{} small //
	\glft `The small children are playing.' //
\endgl
	
\a\label{ex:nounquant}\begingl
	\gla Ajayon ganang-ikan kivo. //
	\glb aja-yon gan-ang=ikan kivo. //
	\glc play-\TplN{} child-\Aarg{}=many small //
	\glft `Many small children are playing.' //
\endgl
\xe
\end{figure}

Likewise, when nouns are modified by numerals,\index{numerals} plurality is not
normally marked again on the noun. In example (\ref{ex:plurnorm}), we see a
plural noun, \xayr{nN}{nanga}{house}, and in (\ref{ex:plurnum}) the same phrase
is repeated with plurality implied by the use of a numeral,
\xayr{smF}{sam}{two}. The plural noun itself appears unmarked in its singular
form in this case. An exception to this is the use of numeral powers, like
\xayr{lnF}{lan}{dozen}, \xayr{menNF}{menang}{gross}, etc.\ in an unspecified
way, like `dozens of people'. To convey that the numeral is not to be
understood as a precise value, the modified noun appears in the plural---even
if it is a \fw{plurale tantum} like \xayr{kejnmF}{keynam}{people} in
(\ref{ex:keynamplur}).

\begin{figure}[h]
\pex
\a\label{ex:plurnorm}\begingl
	\gla Ang @ no @ vehya sitang-yām nangajas veno nay hiro. //
	\glb ang= no= veh=ya.Ø sitang=yām nanga-ye-as veno nay hiro //
	\glc \AgtT{}= want= build=\TsgM.\Top{} self=\TsgM{}.\Dat{} 
		house-\Pl{}-\Parg{} pretty and new //
	\glft `He wants to build himself pretty new houses.' //
\endgl

\a\label{ex:plurnum}\begingl
	\gla Ang @ no @ vehya sitang-yām nangās sam veno nay hiro. //
	\glb ang= no= veh=ya.Ø sitang=yām nanga-as sam veno nay hiro //
	\glc \AgtT{}= want= build=\TsgM.\Top{} self=\TsgM{}.\Dat{} house-\Parg{} 
		two pretty and new //
	\glft `He wants to build himself two pretty new houses.' //
\endgl
\xe
\end{figure}

\begin{figure}
\ex\label{ex:keynamplur}%
\begingl
	\gla Bengyon keynamjang menang. //
	\glb beng-yon keynam-ye-ang menang //
	\glc attend-\TsgN{} people-\Pl{}-\Aarg{} gross //
	\glft `Hundreds of people attended.' //
\endgl\xe
\end{figure}

% This is a new rule; earlier, names were treated as countable but still
% carried special case marking. I found this slightly weird, however, so, let
% us simply assert this new rule, which should make things more consistent. The
% odd case of a pluralized name could still be explained as individual
% variation, though I can't think of an example where this was ever an issue.
%
As we have seen in various examples above, proper nouns\index{nouns!proper} in Ayeri do not receive
inflection for case\index{case} by suffixes\index{suffixes} as common nouns do, and for the purpose of
number they are treated as uncountable in Ayeri---they resist inflection by
suffixation\index{suffixes}, marking their special status.\footnote{Many common names\index{nouns!proper} in Ayeri
are derived from regular words in the language, so the language needs to have a
way to distinguish between regular use and use as a name\index{nouns!proper}. For instance, the
name\index{nouns!proper} \rayr{ynF}{Yan} also means `boy, son' as a common noun.} However, they can
still be modified by quantifiers\index{quantifiers} and quantifying\index{quantifiers} clitics\index{clitics}; verb\index{verbs} agreement\index{agreement} as
well can be used to indicate plurality, compare (\ref{ex:verbplur}).

\begin{figure}[h]
\pex\label{ex:verbplur}
\a\begingl
	\gla Sahayan cabo ekeng ang @ Yan. //
	\glb saha-yan cabo ekeng ang= Yan //
	\glc come-\TplM{} late too \Aarg{}= Yan //
	\glft `The Yans are coming too late.' //
\endgl

\a\begingl
	\gla Ang @ apateng sa @ Yan-ikan. //
	\glb ang= apa=teng sa= Yan=ikan //
	\glc \AgtT{}= laugh=\TplF{}.\Aarg{} \Parg{}= Yan=all //
	\glft `They laughed at (all) the Yans.' //
\endgl
\xe
\end{figure}

\index{number|)}

\subsection{Case}
\label{subsec:case}
\index{case|(}

As demonstrated in the declension tables at the beginning of this section
(Tables \ref{tab:anideclcons}--\ref{tab:inandeclvow}), Ayeri's NPs\index{phrase types!noun phrase} are marked
for case, which is governed by the verb\index{verbs} or assigned to adjuncts\index{grammatical function!adjunct} freely
depending on their purpose or meaning. Since in Ayeri, case marking is at least
partially based on semantics rather than purely on function or structure. This
causes a few exceptions, so it is better, in my opinion, not to use the classic
labels of nominative (S/A) and accusative (O), or of absolutive (S/P) and
ergative (O) for the first two core roles. Instead, I will be using the terms
`agent' and `patient', which I hope brings about some more clarity, especially
when discussing the mentioned exceptions later on. For a discussion of how
Ayeri deals with subjecthood, see \autoref{subsec:subjecthood}.

\subsubsection{Agent}
\label{subsubsec:agent}
\index{case!agent|(}
\index{semantic role!agent|(}
\index{semantic role!experiencer|(}

What I call `agent' here is, to quote \citet{fillmore1968},
\textcquote[46]{fillmore1968}{the case of the typically animate perceived
instigator of the action identified by the verb}. \citet{fillmore1968}
himself qualifies this definition, however, in that the \textcquote[46,
footnote 31]{fillmore1968}{escape qualification `typically' expresses my
awareness that contexts which I will say require agents are sometimes occupied
by `inanimate'\index{animacy} nouns like \fw{robot} or `human institution' nouns like
\fw{nation}}. \citet{payne1997} summarizes on prototypical agents with regards
to their topicality that a \textcquote[151]{payne1997}{less technical way of
expressing this fact is to say that people identify with and like to talk about
things that act, move, control events, and have power}.

Agents in Ayeri frequently embody the properties quoted by both
\citet{fillmore1968} and \citet{payne1997} in this regard, including
\citet{fillmore1968}'s caveat. However, importantly, `agent' in Ayeri is a
macrorole that may be applied to, for instance, instruments\index{semantic role!instrument}, experiencers, and
less typical actors as well, specifically, in absence of more prototypical
candidates for agenthood in a sentence. It thus comes very close to a
nominative, except that it does not need to be locus of the sentence's
topic\index{grammatical function!topic}---although agents very typically are
topics\index{grammatical function!topic}, as \citet[151]{payne1997} goes on to note.

The agent is marked by the suffix\index{suffixes} \rayr{/ANF}{-ang} for animate referents and
the suffix\index{suffixes} \rayr{/reNF}{-reng} for inanimate\index{animacy} referents; names\index{nouns!proper} and verbal\index{verbs} topic\index{grammatical function!topic}
agreement\index{agreement} are marked by the clitic\index{clitics} case markers \rayr{ANF}{ang} and
\rayr{ENF}{eng}, respectively. See (\ref{ex:agtmarking1}) and
(\ref{ex:agtmarking2}) for examples of each marker.

\begin{figure}
\pex\label{ex:agtmarking1}
\a\begingl
	\gla \textbf{Ang} @ tinkaya {} @ \textbf{Yan} kunangley. //
	\glb ang= tinka-ya Ø= Yan kunang-ley //
	\glc \AgtT{}= open-\TsgM{} \Top{}= Yan door-\PargI{} //
	\glft `Yan opens the door.' //
\endgl

\a\begingl
	\gla Le @ tinkaya \textbf{ayonang} kunang. //
	\glb le= tinka-ya ayon-ang kunang-Ø //
	\glc \PatT{}= open-\TsgM{} man-\Aarg{} door-\Top{} //
	\glft `The door is opened by a/the man',\\
		\textit{or:} `The door, a/the man opens it.' //
\endgl
\xe
\end{figure}

\begin{figure}
\pex\label{ex:agtmarking2}
\a\begingl
	\gla \textbf{Eng} @ tinkāra \textbf{tinkay} kunangley. //
	\glb eng= tinka-ara tinkay-Ø kunang-ley //
	\glc \AgtTI{}= open-\TsgI{} key-\Top{} door-\PargI{} //
	\glft `The key opens the door.' //
\endgl

\a\begingl
	\gla Tinkāra \textbf{kunangreng}. //
	\glb tinka-ara kunang-reng //
	\glc open-\TsgI{} door-\AargI{} //
	\glft `The door opens.' //
\endgl
\xe
\end{figure}

In predicative\index{grammatical function!predicative complement} constructions, the constituent which a quality is assigned to or
which a judgment is made about is also assigned the agent case, as
(\ref{ex:prednpagt}) shows. With regards to constituents' roles in ditransitive\index{verbs!ditransitive}
argument frames, donors are represented by agents in Ayeri as well, since they
are the origin of whatever is conceptually passed on to the recipient\index{semantic role!recipient} party,
compare (\ref{ex:ditragt}). Moreover, as (\ref{ex:causagt}) shows, the causee\index{causation}
is marked as an agent, not as a patient\index{semantic role!patient}, since that would be semantically
incongrouous.

\begin{figure}[h]
\pex\label{ex:prednpagt}
\a\begingl
	\gla Tado \textbf{tinkayreng}. //
	\glb tado tinkay-reng //
	\glc old key-\AargI{} //
	\glft `The key is old.' //
\endgl

\a\begingl
	\gla \textbf{Ang} @ \textbf{Yan} nimpayās ban. //
	\glb ang= Yan nimpaya-as ban //
	\glc \Aarg{}= Yan runner-\Parg{} good //
	\glft `Yan is a good runner.' //
\endgl
\xe
\end{figure}

\begin{figure}[h]
\ex\label{ex:ditragt}%
\begingl
	\gla Le @ ilya \textbf{ang} @ \textbf{Yan} tinkay yam @ Cānlay. //
	\glb le= il-ya ang= Yan tinkay-Ø yam= Cānlay //
	\glc \PatT{}= give-\TsgM{} \Aarg{}= Yan key-\Top{} \Dat{}= Cānlay //
	\glft `The key, Yan gives it to Cānlay.' //
\endgl\xe
\end{figure}

\begin{figure}[h]
\ex\label{ex:causagt}
\begingl
	\gla Sā @ tinkaya \textbf{ang} @ \textbf{Yan} kunangley yan. //
	\glb sā= tinka-ya ang= Yan kunang-ley yan.Ø //
	\glc \CauT{}= open-\TsgM{} \Aarg{}= Yan door-\PargI{} \TsgM{}.\Top{} //
	\glft `They make Yan open a/the door',\\
		\textit{or:} `Because of them, Yan opens the door.' //
\endgl
\xe
\end{figure}

\index{semantic role!experiencer|)}
\index{semantic role!agent|)}
\index{case!agent|)}

\subsubsection{Patient}
\index{case!patient|(}
\index{semantic role!patient|(}
\index{semantic role!theme|(}

Patients are less of a definitional problem than agents, since in transitive\index{verbs!transitive}
sentences, they are very typically undergoers, that is, the constituent which
is acted on, affected, or produced by the action expressed by the verb\index{verbs}. The
patient case is thus assigned to direct objects\index{grammatical function!primary object}---but also to predicative\index{grammatical function!predicative complement}
nominals. Animate patients are marked by \rayr{/AsF}{-as}, inanimate\index{animacy} ones by
\rayr{/lej}{-ley}; for names\index{nouns!proper} and verbal\index{verbs} topic\index{grammatical function!topic} agreement\index{agreement}, the markers are
\rayr{s}{sa} and \rayr{le}{le}, respectively, compare (\ref{ex:patmarking1})
and (\ref{ex:patmarking2}). In ditransitive\index{verbs!ditransitive} sentences like the one in
(\ref{ex:ditrpat}), the theme is represented by the patient.

\begin{figure}
\pex\label{ex:patmarking1}
\a\begingl
	\gla Ang @ silvye {} @ Briha \textbf{sa} @ \textbf{Taryan}. //
	\glb ang= silv-ye Ø= Briha sa= Taryan //
	\glc \AgtT{}= see-\TsgF{} \Top{}= Briha \Parg{}= Taryan//
	\glft `Briha sees Taryan.' //
\endgl

\a\begingl
	\gla \textbf{Sa} @ manye ang @ Briha {} @ \textbf{Taryan}. //
	\glb sa= man-ye ang= Briha Ø= Taryan //
	\glc \PatT{}= greet-\TsgF{} \Aarg{}= Briha \Top{}= Taryan //
	\glft `Taryan is greeted by Briha',\\
		\textit{or:} `Taryan, Briha greets him.' //
\endgl
\xe
\end{figure}

\begin{figure}
\pex\label{ex:patmarking2}
\a\begingl
	\gla Ang @ rimaye {} @ Briha \textbf{kunangley}. //
	\glb ang= rima-ye Ø= Briha kunang-ley //
	\glc \AgtT{}= close-\TsgF{} \Top{}= Briha door-\PargI{} //
	\glft `Briha closes a/the door.' //
\endgl

\a\begingl
	\gla \textbf{Le} @ rimaye ang @ Briha \textbf{kunang}. //
	\glb le= rima-ye ang= Briha kunang-Ø //
	\glc \PatTI{}= close-\TsgF{} \Aarg{}= Briha door-\Top{} //
	\glft `The door is closed by Briha',\\
		\textit{or:} `The door, Briha closes it.' //
\endgl
\xe
\end{figure}

\begin{figure}
\ex\label{ex:ditrpat}
\begingl
	\gla Ang @ ilya {} @ Taryan \textbf{koyaley} yam @ Kandan. //
	\glb ang= il-ya Ø Taryan= koya-ley yam= Kandan //
	\glc \AgtT{}= give-\TsgM{} \Top{}= Taryan book-\PargI{} \Dat{}= Kandan //
	\glft `Taryan gives Kandan a book.' //
\endgl
\xe
\end{figure}

As the translations of the examples above show, topicalizing\index{grammatical function!topic} the patient can be
used to create an effect similar to English's\index{English} passive voice, except that the
patient will not become marked by the agent case\index{case!agent} for logical reasons---this is
a notable difference from the nominative. Even if the agent NP\index{phrase types!noun phrase} is omitted to
form a passive\index{voice!passive} in (\ref{ex:agtnotnom}), the patient NP\index{phrase types!noun phrase} will not be changed to
the agent case\index{case!agent}, since that would reverse the direction of action.\index{semantic role!agent}

\begin{figure}
\ex\label{ex:agtnotnom}\begingl
	\gla Manya sa @ Taryan. {} Manya ang @ Taryan. //
	\glb man-ya sa= Taryan ≠ Man-ya ang= Taryan //
	\glc greet-\TsgM{} \Parg{}= Taryan {} greet-\TsgM{} \Aarg{}= Taryan //
	\glft `Taryan is greeted.' ≠ `Taryan greets.' //
\endgl\xe
\end{figure}

Example (\ref{ex:agtnotnom}) shows that the case of the NP\index{phrase types!noun phrase} will not change, however,
the verb\index{verbs} will: it now agrees\index{agreement} with the next argument in line, the patient NP\index{phrase types!noun phrase}. It
will not do so, however, if the order\index{word order} of arguments is simply scrambled, as in
(\ref{ex:verbscram}). This is to say that the verb\index{verbs} does not simply agree\index{agreement} with
whichever NP\index{phrase types!noun phrase} follows\index{word order} it, even if it can be assumed that verb\index{verbs} agreement\index{agreement} in Ayeri
developed along similar lines in-world, which will become especially apparent
in the discussion of pronouns\index{pronouns}.\footnote{Mismatches in agreement\index{agreement} in connection
to scrambling such as exemplified by (\ref{ex:scramfalse}) are to be expected,
however. \citet{corbett2006}, notes that with regards to agreement\index{agreement} in NP\index{phrase types!noun phrase}
conjuncts, \textcquote[62]{corbett2006}{distant agreement\index{agreement} is rare, and that
agreement\index{agreement} with the nearest noun phrase\index{phrase types!noun phrase} or agreement\index{agreement} with all (resolution\index{resolution}) is
much more common}. If there were an extensive corpus of texts written by Ayeri
speakers, it might be interesting to gather statistics on the number of words
between target and controller in relation to the prevalence of agreement\index{agreement}
mismatches.}

\begin{figure}[h]
\pex[aboveglftskip=2em]\label{ex:verbscram}
\a\label{ex:scramcorr}\begingl
	\gla Sa @ manye {} @ Taryan ang @ Briha. //
	\glb sa= man-ye Ø= Taryan ang= Briha //
	\glc \PatT{}= greet-\tikzmark{target}\TsgF{} \Top{}= Taryan 
		\Aarg{}= \tikzmark{controller}Briha //
	\glft `Taryan is greeted by Briha',\\
		\textit{or:} `Taryan, Briha greets him.' //
\endgl

\a\label{ex:scramfalse}\ljudge* \begingl
	\gla Sa @ manya {} @ Taryan ang= Briha. //
	\glb sa= man-ya Ø= Taryan ang= Briha //
	\glc \PatT{}= greet-\tikzmark{target2}\TsgM{} \Top{}= %
		\tikzmark{controller2}Taryan \Aarg{}= Briha //
\endgl\xe
\begin{tikzpicture}[remember picture, overlay]
	\coordinate [below right=.25em and 1em   of {pic cs:controller}] (A);
	\coordinate [below right=1em   and 1em   of {pic cs:controller}] (B);
	\coordinate [below right=1em   and .75em of {pic cs:target}    ] (C);
	\coordinate [below right=.25em and .75em of {pic cs:target}    ] (D);
	\draw [-latex] (A) -- (B) -- (C) -- (D);
	\node (label) at ($(B)!0.5!(C)$) [below] {\tiny\itshape person 
		agreement};
	
	\coordinate [below right=.25em and 1em   of {pic cs:controller2}] (A);
	\coordinate [below right=1em   and 1em   of {pic cs:controller2}] (B);
	\coordinate [below right=1em   and .75em of {pic cs:target2}    ] (C);
	\coordinate [below right=.25em and .75em of {pic cs:target2}    ] (D);
	\draw [-latex, dashed] (A) -- (B) -- (C) -- (D);
	\node (label) at ($(B)!0.5!(C)$) [below] {\tiny\itshape *person 
		agreement};
\end{tikzpicture}
\end{figure}

Besides being the default\index{default} case for direct objects\index{grammatical function!primary object}, the patient case is also 
assigned to predicative\index{grammatical function!predicative complement} nominals, by analogy with transitive\index{verbs!transitive} sentences and in 
spite of the likening nature of the construction, compare (\ref{ex:predpat}).

\begin{figure}[h]
\ex\label{ex:predpat}%
\begingl
	\gla Ang @ Yan \textbf{nimpayās} ban. //
	\glb ang= Yan nimpaya-as ban //
	\glc \Aarg{}= Yan runner-\Parg{} good //
	\glft `Yan is a good runner.' //
\endgl\xe
\end{figure}

\index{semantic role!theme|)}
\index{semantic role!patient|)}
\index{case!patient|)}

\subsubsection{Dative}
\label{subsubsec:dative}
\index{case!dative|(}
\index{semantic role!recipient|(}
\index{semantic role!beneficiary|(}
\index{semantic role!goal|(}

The most typical use of the dative is for the recipient NP\index{phrase types!noun phrase} in a ditransitive\index{verbs!ditransitive}
clause; as such, it may be a recipient proper or the entity to whose benefit (or detriment)
the action is carried out. The dative can furthermore be used to mark movement
toward a place. The case suffix\index{suffixes} for datives is \rayr{/ymF}{-yam} for both
animate and inanimate\index{animacy} entities. Names\index{nouns!proper} and verbal\index{verbs} topic\index{grammatical function!topic} agreement\index{agreement} are marked
equally by \rayr{ymF}{yam}. Verbs\index{verbs} do not exhibit person\index{person} agreement\index{agreement} with dative
NPs\index{phrase types!noun phrase}, since experiencers\index{semantic role!experiencer} are treated as agents\index{semantic role!agent}.

\begin{figure}[h]
\pex\label{ex:datregular}
\a\begingl
	\gla Ang @ ilya {} @ Taryan koyaley \textbf{ayonyam}. //
	\glb ang= il-ya Ø= Taryan koya-ley ayon-yam //
	\glc \AgtT{}= give-\TsgM{} \Top{}= Taryan book-\PargI{} 
		man-\Dat{} //
	\glft `Taryan gives a book to the man.' //
\endgl

\a\begingl
	\gla Ang @ ilya {} @ Taryan koyaley \textbf{yam} @ \textbf{Kandan}. //
	\glb ang= il-ya Ø= Taryan koya-ley yam= Kandan //
	\glc \AgtT{}= give-\TsgM{} \Top{}= Taryan book-\PargI{} \Dat{}= Kandan //
	\glft `Taryan gives Kandan a book.' //
\endgl

\a\begingl
	\gla \textbf{Yam} @ ilya ang @ Taryan koyaley \textbf{ayon}. //
	\glb yam= il-ya ang= Taryan koya-ley ayon-Ø //
	\glc \DatT{}= give-\TsgM{} \Aarg{}= Taryan book-\PargI{} man-\Top{} //
	\glft `The man is given a book by Taryan',\\
		\textit{or:} `The man, Taryan gives him a book.' //
\endgl
\xe
\end{figure}

The three examples in (\ref{ex:datregular}) show the regular use of the dative
as the case the recipient of the theme\index{semantic role!theme} appears in. It is also possible for
dative NPs\index{phrase types!noun phrase} to appear as topics\index{grammatical function!topic}---person\index{person} agreement\index{agreement} is unaffected by this,
though, since topicalization\index{grammatical function!topic} and subject\index{grammatical function!subject} marking are different processes in
Ayeri.

As mentioned above, the dative can also take on an allative meaning insofar as
it marks the target of a motion, as displayed in (\ref{ex:datloc}). As an
extension of this means, the adpositional object\index{grammatical function!adpositional object} may as well appear in the
dative, since Ayeri cannot distinguish, for instance, `up' from `to the top of'
with just the preposition\index{adpositions!prepositions}, in this case \xayr{liNF}{ling}{on top of}. With the
adpositional object\index{grammatical function!adpositional object} in the locative case\index{case!locative} (see below), the phrase in
(\ref{ex:datlocprep}) would imply that the man is literally going to the top
of the temple, that is, ending up on its roof.

\begin{figure}[h]
\pex
\a\label{ex:datloc}\begingl
	\gla Ang @ nimpye lay \textbf{māvayam} yena. //
	\glb ang= nimp-ye lay-Ø māva-yam yena //
	\glc \AgtT{}= run-\TsgF{} girl-\Top{} mother-\Dat{} \TsgF{}.\Gen{} //
	\glft `The girl runs to her mother.' //
\endgl

\a\label{ex:datlocprep}\begingl
	\gla Ang @ saraya ayon manga @ ling \textbf{natrangyam}. //
	\glb ang= sara-ya ayon-Ø manga= ling natrang-yam //
	\glc \AgtT{}= go-\TsgM{} man-\Top{} \Dir{}= top temple-\Dat{} //
	\glft `The man goes up to the temple.' //
\endgl
\xe
\end{figure}

Lastly, the dative case is also used to mark resultative NPs\index{phrase types!noun phrase}, that is, NPs\index{phrase types!noun phrase}
which express the result of an action performed on the semantic patient\index{semantic role!patient} of a
clause. This not only includes syntactic objects, but also patient\index{semantic role!patient}-subjects\index{grammatical function!subject} of
agentless\index{semantic role!agent} sentences and the subjects\index{grammatical function!subject} of unaccusative verbs\index{verbs!unaccusative}
\citep{perlmutter1978}, that is, verbs whose syntactic subject\index{grammatical function!subject} is not
performing the action expressed by the verb, but undergoing it. The resultative
dative NP\index{phrase types!noun phrase} is fronted to occur after\index{word order} the verb\index{verbs} in contrast to regular recipients,
beneficiaries, or goals\index{semantic role!goal}. A clause may thus contain two dative NPs\index{phrase types!noun phrase}. These,
however, are still required to be functionally unique. That is, one may not
have two recipients or two resultatives in the same clause.

\begin{figure}[h]
\ex\label{ex:resultdat}\begingl
	\gla Ang @ visya nernanjyam {} @ Niyas seygoley ganyam. //
	\glb ang= vis-ya nernan-ye-yam Ø= Niyas seygo-ley gan-yam //
	\glc \AgtT{}= cut-\TsgM{} piece-\Pl{}-\Dat{} \Top{}= Niyas apple-\PargI{}
		child-\Dat{} //
	\glft `Niyas cuts the apple into pieces for the child.' //
\endgl\xe
\end{figure}

Hence, the first dative NP\index{phrase types!noun phrase} in (\ref{ex:resultdat}),
\xayr{nerFnnFyeymF}{nernanjyam}{(in)to pieces}, expresses the result of cutting
the object of the clause, \xayr{sejgolej}{seygoley}{apple}. The second dative
NP\index{phrase types!noun phrase}, \xayr{gnFymF}{ganyam}{for the child}, expresses the (optional) beneficiary
of the action.

\index{semantic role!goal|)}
\index{semantic role!beneficiary|)}
\index{semantic role!recipient|)}
\index{case!dative|)}

\subsubsection{Genitive}
\label{subsubsec:genitive}
\index{case!genitive|(}
\index{semantic role!possessor|(}
\index{semantic role!source|(}

The genitive is used to mark possessors; attributive genitives follow\index{word order} the
possessee. It can also be used for ablative meanings, that is, to mark the
place from which a motion originates, in analogy to the dative's\index{case!dative} allative use.
The genitive is marked on common nouns\index{nouns!common} with the suffix\index{suffixes} \rayr{/n}{-na}. If a
noun stem ends in a consonant, the marker becomes \rayr{/En}{-ena}, compare
Tables \ref{tab:anideclcons}--\ref{tab:inandeclvow} above. Names\index{nouns!proper} and verbal\index{verbs}
topic\index{grammatical function!topic} agreement\index{agreement} are marked by \rayr{n}{na}. There is no animacy\index{animacy} distinction in
the genitive case. Examples of the genitive case markers are given in
(\ref{ex:genmarking}).

\begin{figure}[h]
\pex\label{ex:genmarking}
\a\begingl
	\gla Pakur ledanang \textbf{netuna} nā. //
	\glb pakur ledan-ang netu-na nā //
	\glc sick friend-\Aarg{} brother-\Gen{} \Fsg{}.\Gen{} //
	\glft `My brother's friend is sick.' //
\endgl

% \a\begingl
% 	\gla Kopo dilengyereng \textbf{ajānena}. //
% 	\glb kopo dileng-ye-reng ajān-ena //
% 	\glc difficult rule-\Pl{}-\AargI{} game-\Gen{} //
% 	\glft `The rules of the game are difficult.' //
% \endgl

\a\begingl
	\gla Ang nakasyo tamo ibangya \textbf{na} @ \textbf{Niyas}. //
	\glb ang nakas-yo tamo-Ø ibang-ya na= Niyas //
	\glc \AgtT{} grow-\TsgN{} wheat-\Top{} field-\Loc{} \Gen{}= Niyas //
	\glft `There is wheat growing on Niyas's field.' //
\endgl

\a\begingl
	\gla \textbf{Na} @ nakasyo tamoang ibangya {} @ \textbf{Niyas}. //
	\glb na= nakas-yo tamo-ang ibang-ya Ø= Niyas //
	\glc \GenT{}= grow-\TsgN{} wheat-\Aarg{} field-\Loc{} \Top{}= Niyas //
	\glft `Regarding Niyas, there is wheat growing on his field.' //
\endgl
\xe
\end{figure}

Futhermore, Ayeri does not make a distinction between alienable and inalienable
possession at least in the formal language, so that typically inalienable
things such as body parts, relatives and family members, or personal items and
tools are all treated as described in (\ref{ex:genmarking}). Consider
(\ref{ex:inalposs}) for an illustration of various inalienable things. However,
inalienably possessed NPs\index{phrase types!noun phrase} may still appear without a possessor in less
formal language. Besides body parts and family members, this also typically extends to \xayr{rNF}
{rang}{home}.

\begin{figure}
\ex\label{ex:inalposs}
\begingl
	\gla Ang @ puntaye māva \textbf{nā} mitrangas \textbf{yena} sembari 
		\textbf{yena}. //
	\glb ang= punta-ye māva-Ø nā mitrang-as yena semba-ri 
		yena //
	\glc \AgtT{}= brush-\TsgF{} mother-\Top{} \Fsg{}.\Gen{} hair-\Parg{} 
		\TsgF{}.\Gen{} comb-\Ins{} \TsgF{}.\Gen{} //
	\glft `My mother is brushing her hair with her comb.' //
\endgl\xe
\end{figure}

The above examples show the regular use of the genitive as a marker of
possession. Apart from possession, however, the genitive can also be used to
mark origin, that is, it has a secondary function as an ablative. This is shown
in (\ref{ex:genabl}).

\begin{figure}[h]
\ex\label{ex:genabl}%
\begingl
	\gla Ang @ sahaya {} @ Vetayan \textbf{rimanena}. //
	\glb ang= saha-ya Ø= Vetayan riman-ena //
	\glc \AgtT{}= come-\TsgM{} \Top{}= Vetayan city-\Gen{} //
	\glft `Vetayan comes from the city.' //
\endgl
\xe
\end{figure}

\index{semantic role!source|)}
\index{semantic role!possessor|)}
\index{case!genitive|)}

\subsubsection{Locative}
\index{case!locative|(}
\index{semantic role!location|(}

The locative marks basic locations, often the default\index{default} that is associated with a
verb\index{verbs}. It is also the case in which adpositional objects\index{grammatical function!adpositional object} normally appear,
besides the special cases using the dative\index{case!dative} as mentioned above.
Common nouns\index{nouns!common} are marked by \rayr{/y}{-ya};\footnote{Older texts still exhibit
an allomorph\index{allomorphy} \rayr{/E\_a}{-ea}, used especially in combination with the plural\index{number!plural}
suffix\index{suffixes} \rayr{/ye}{ye}, giving \rayr{/yee\_a}{-yēa}. The modern language uses
\rayr{/yey}{-jya}.} names\index{nouns!proper} and verbal\index{verbs} topic\index{grammatical function!topic} agreement\index{agreement} use the marker 
\rayr{y}{ya}. There is no difference made between animate and inanimate\index{animacy} 
referents in the locative.

\begin{figure}
\pex\label{ex:locplain}
\a\label{ex:locnedra}\begingl
	\gla Ang @ nedraya paray \textbf{hinya}. //
	\glb ang= nedra-ya paray-Ø hin-ya //
	\glc \AgtT{}= sit-\TsgM{} cat-\Top{} box-\Loc{} //
	\glft `The cat sits in the box.' //
\endgl

\a\label{ex:locnara}\begingl
	\gla Ang @ naraya {} @ Ajān \textbf{ya} @ \textbf{Kaman}. //
	\glb ang= nara-ya Ø= Ajān ya= Kaman //
	\glc \AgtT{}= speak-\TsgM{} \Top{}= Ajān \Loc{}= Kaman //
	\glft `Ajān speaks to Kaman.' //
\endgl

\a\label{ex:locmit}\begingl
	\gla \textbf{Ya} @ mica ang @ Kaman {} @ \textbf{Visamhinang}. //
	\glb ya= mit-ya ang= Kaman Ø= Visamhinang //
	\glc \LocT{}= live-\TsgM{} \Aarg{}= Kaman \Top{}= Visamhinang //
	\glft `Kaman lives in Visamhinang',\\
		\textit{or:} `Visamhinang is where Kaman lives.' //
\endgl
\xe
\end{figure}

The example sentences in (\ref{ex:locplain}) show locative NPs\index{phrase types!noun phrase} that are not
further specified by adpositions so that the correct interpretation may be
dependent on context and the experience of the addressee. Example
(\ref{ex:locnedra}) is an instance of this circumstance, in that experience
tells that cats like to sit inside boxes, so further specifying the position
with the preposition\index{adpositions!prepositions} \xayr{koNF}{kong}{inside} would be emphasizing that the
cat is not sitting just anywhere, but really \emph{inside} the box as opposed
to on top of it, for instance. The sentence in example (\ref{ex:expladp}) has
the cat sitting on top of the box.

\begin{figure}[h]
\ex\label{ex:expladp}
\begingl
	\gla Ang @ nedraya paray ling hinya. //
	\glb ang= nedra-ya paray-Ø ling hin-ya //
	\glc \AgtT{}= sit-\TsgM{} cat-\Top{} on.top box-\Loc{} //
	\glft `The cat sits on the box.' //
\endgl\xe
\end{figure}

Ayeri also has a number of postpositions\index{adpositions!postpositions}. This does not change the fact that
the adpositional object\index{grammatical function!adpositional object} is marked for locative case, however, as we see in
(\ref{ex:locpostpos}), where \xayr{tenYnF}{tenyan}{death} is marked for locative 
case governed by the postposition\index{adpositions!postpositions} \xayr{pesnF}{pesan}{until}.

\begin{figure}[h]
\ex\label{ex:locpostpos}%
\begingl
	\gla Ang @ mican edaya \textbf{tenyanya} tan pesan. //
	\glb ang= mit-yan edaya tenyan-ya tan pesan //
	\glc \AgtT{}= live-\TplM{} here death-\Loc{} \TplM{}.\Gen{} until //
	\glft `They lived here until their death.' //
\endgl\xe
\end{figure}

\index{semantic role!location|)}
\index{case!locative|)}

\subsubsection{Causative}
\label{subsubsec:causative}
\index{case!causative|(}
\index{semantic role!causer|(}

The causative marks the cause or causer of an action, the instigator or the
reason on behalf of which an agent\index{semantic role!agent} is acting. It is thus similar to the agent
case\index{case!agent}, though it does not replace it in Ayeri; verbs\index{verbs} do not exhibit person\index{person}
agreement\index{agreement} with causers even though their action logically supersedes or
precedes that of the agent\index{semantic role!agent} in the embedded event. \citet{dixon2000} writes that
a \textcquote[30]{dixon2000}{causer refers to someone or something (which can
be an event or state) that initiates or controls the activity. This is the
defining property of the syntactic--semantic function A (transitive subject)}.
According to \citet[176]{comrie1989}, the causee---the agent\index{semantic role!agent} of the event
controlled by the causer---normally takes the highest place in the hierarchy\index{hierarchy} of
syntactic constituents that is not already filled, in this case, by the causer.
This observation, however, is complicated by Ayeri's more or less
semantics-based case marking as well as topicalization\index{grammatical function!topic}. In the following, I
will give examples of nominal marking for cause as before; a discussion of the
morphosyntax of Ayeri's morphological causative constructions will be deferred
to the section on valency-increasing operations, compare
\autoref{subsubsec:valincr}.

Causers or causes are marked by \rayr{/Is}{-isa} for common nouns\index{nouns!common}; names\index{nouns!proper} and
verbal\index{verbs} topic\index{grammatical function!topic} agreement\index{agreement} use the marker \rayr{saa}{sā}. As stated above, verbs\index{verbs} do
not agree\index{agreement} with causers even though they have agent-like\index{semantic role!agent} semantics. There is no
animacy\index{animacy} distinction in the marking of causers. Examples of the case marker in
its various positions are provided by (\ref{ex:caumarking}).

\begin{figure}[h]
\pex\label{ex:caumarking}
\a\begingl
	\gla Ang @ rua @ sarāyn \textbf{seyaranisa}. //
	\glb ang= rua= sara=ayn.Ø seyaran-isa //
	\glc \AgtT{}= must= leave=\Fpl{}.\Top{} rain-\Caus{} //
	\glft `We had to leave due to the rain.' //
\endgl

\a\begingl
	\gla Ang @ yomāy edaya \textbf{sā} @ \textbf{Apican}. //
	\glb ang= yoma=ay.Ø edaya sā= Apican //
	\glc \AgtT{}= be=\Fsg{}.\Top{} here \Caus{}= Apican //
	\glft `I am here because of Apican.' //
\endgl

\a\label{ex:caustop}\begingl
	\gla \textbf{Sā} @ nimpvāng hakasley \textbf{yan}. //
	\glb sā= nimp=vāng hakas-ley yan.Ø //
	\glc \CauT{}= run=\Second{}.\Aarg{} mile-\PargI{} \TplM{}.\Top{} //
	\glft `You run a mile because of them',\\
		\textit{or:} `They make you run a mile.' //
\endgl
\xe
\end{figure}

Regarding the typological\index{typology} oddities mentioned above, example (\ref{ex:caustop}) 
shows what happens in Ayeri with regards to the marking of causers. 
Essentially, the causer topic\index{grammatical function!topic} was grammaticalized to express a causative 
relationship.

\index{semantic role!causer|)}
\index{case!causative|)}

\subsubsection{Instrumental}
\label{subsubsec:instrumental}
\index{case!instrumental|(}
\index{semantic role!instrument|(}

The instrumental marks the means by which an action is carried out by an agent\index{semantic role!agent}.
This can be a tool as well as an animate being by whose help the action is
brought about. The instrumental, thus, marks secondary agents\index{semantic role!agent} in effect. Verbs\index{verbs},
however, never show person\index{person} agreement\index{agreement} with instrumental NPs\index{phrase types!noun phrase}. Common nouns\index{nouns!common} are
marked by \rayr{/ri}{-ri} when ending in a vowel and by \rayr{/Eri}{-eri}
when ending in a consonant; names\index{nouns!proper} and verbal\index{verbs} topic\index{grammatical function!topic} agreement\index{agreement} are marked by
\rayr{ri}{ri}. With nouns ending in \fw{-e}, as well as the plural\index{number!plural} marker
\rayr{/ye}{-ye}, there is variation\index{allomorphy} regarding whether \rayr{/ri}{-ri} or 
\rayr{/Eri}{-eri} is used, so that both \rayr{/yeri}{-yeri} and \rayr{/yeeri}
{-yēri} may be found as plural\index{number!plural} forms. In passive-like constructions\index{voice!passive}, it is not
grammatical to reintroduce the agent\index{semantic role!agent} as an instrumental; the agent\index{semantic role!agent} simply
remains in the clause in this case, though as a non-topic\index{grammatical function!topic} constituent. Examples
for the case markers are given in (\ref{ex:insmarking}).

\begin{figure}
\pex\label{ex:insmarking}
% \a\begingl
% 	\gla Ang @ visye {} @ Pila seygoley \textbf{tihangeri} yena. //
% 	\glb ang= vis-ye Ø= Pila seygo-ley tihang-eri yena. //
% 	\glc \AgtT{}= cut-\TsgF{} \Top{}= Pila apple-\PargI{} 
% 		knife-\Ins{} \TsgF{}.\Gen{} //
% 	\glft `Pila cuts an apple with her knife.' //
% \endgl
%
\a\begingl
	\gla Ang @ lihoyya-ma badan \textbf{nihanyeri} 
		\textbf{\textup{(}nihanyēri\textup{)}}. //
	\glb ang= liha-oy-ya=ma badan-Ø nihan-ye-ri (nihan-ye-eri) //
	\glc \AgtT{}= earn-\Neg{}-\TsgM{}=enough father-\Top{} 
		nihan-\Pl{}-\Ins{} (nihan-\Pl{}-\Ins) //
	\glft `Father did not earn enough with his fruits.' //
\endgl

\a\begingl
	\gla Ang @ lingya {} @ Mindan mehiras \textbf{ri} @ \textbf{Kadijān}. //
	\glb ang= ling-ya Ø= Mindan mehir-as ri= Kadijān. //
	\glc \AgtT{}= climb.up-\TsgM{} \Top{}= Mindan tree-\Parg{} 
		\Ins{}= Kadijān //
	\glft `Mindan climbs a tree with Kadijān's help.' //
\endgl

\a\begingl
	\gla \textbf{Ri} @ tavya gino ang @ Kan \textbf{nimpur}. //
	\glb ri= tav-ya gino ang= Kan nimpur-Ø //
	\glc \InsT{}= become-\TsgM{} drunk \Aarg{}= Kan wine-\Top{} //
	\glft `The wine, Kan becomes drunk on it.' //
\endgl
\xe
\end{figure}

The instrumental may also be used for cases where the instrumental NP\index{phrase types!noun phrase} acts as a
nominal complement\index{grammatical function!closed complement} describing an attribute of its antecedent head, as in
(\ref{ex:nounadjc}).

\begin{figure}[h]
\ex\label{ex:nounadjc}
\begingl
	\gla Ang @ pegayo sinya kasuley \textbf{bariri} nā? //
	\glb ang= pega-yo sinya-Ø kasu-ley bari-ri nā //
	\glc \AgtT{}= steal-\TsgN{} who-\Top{} basket-\PargI{} 
		meat-\Ins{} \Fsg{}.\Gen{} //
	\glft `Who stole my basket of meat?' //
\endgl\xe
\end{figure}

Here, \xayr{bri}{bari}{meat} is marked as an instrumental since it serves as an
attribute of \xayr{ksu}{kasu}{basket}. The instrumental NP\index{phrase types!noun phrase} describes what its
antecedent contains or entails more specifically: it is a basket \fw{with} meat
in it. Note, however, that this use of the instrumental is different from
expressing accompaniment. Thus, it is not possible to use the sentence in
(\ref{ex:wrongcomit}) to express `Ajān comes (together) with Pila'.

\ex\label{ex:wrongcomit}\ljudge* \begingl
	\gla Ang @ sahaya {} @ Ajān \textbf{ri} @ \textbf{Pila}. //
	\glb ang= saha-ya Ø= Ajān ri= Pila //
	\glc \AgtT{}= come-\TsgM{} \Top{}= Ajān \Ins{}= Pila //
\endgl\xe

The sentence in (\ref{ex:wrongcomit}) would instead imply that \rayr{pil}{Pila} helps \rayr{AgYaanF}{Ajān}
to come, for example, because he has a sprained ankle and thus needs support to
get around. To express accompaniment, instead, the preposition\index{adpositions!prepositions}
\xayr{kjvo}{kayvo}{with, along, beside} has to be used; the prepositional
object\index{grammatical function!adpositional object} appears in the locative case, as usual, then, compare
(\ref{ex:comitwith}).

\begin{figure}[h]
\ex\label{ex:comitwith}%
\begingl
	\gla Ang @ sahaya {} @ Ajān \textbf{kayvo} \textbf{ya} @ \textbf{Pila}. //
	\glb ang= saha-ya Ø= Ajān kayvo ya= Pila //
	\glc \AgtT{}= come-\TsgM{} \Top{}= Ajān with \Loc{}= Pila //
	\glft `Ajān comes (together) with Pila.' //
\endgl\xe
\end{figure}

Theoretically, it should be possible as well to use the instrumental together
with prepositions\index{adpositions!prepositions} for some kind of prolative meaning. The adposition\index{adpositions} would
indicate the place \emph{by way of} a motion is happening, as in
(\ref{ex:viains}).

\begin{figure}[h]
\ex\label{ex:viains}
\begingl
	\gla Ang @ pukay manga @ luga \textbf{lahaneri}. //
	\glb ang= puk=ay.Ø manga= luga lahan-eri //
	\glc \AgtT{}= jump=\Fsg{}.\Top{} \Dir{}= top fence-\Ins{} //
	\glft `I jump over the fence.' //
\endgl\xe
\end{figure}

This use of the instrumental is unattested in previous translations into Ayeri,
however, but could be considered a stylistic alternative---in the case of the
example above, to the construction with the word for `over',
\rayr{EjrrY}{eyrarya} in (\ref{ex:vialoc}).

\begin{figure}[h]
\ex\label{ex:vialoc}
\begingl
	\gla Ang @ pukay manga @ eyrarya lahanya. //
	\glb ang= puk=ay.Ø manga= eyrarya lahan-ya //
	\glc \AgtT{}= jump=\Fsg{}.\Top{} \Dir{}= over fence-\Loc{} //
	\glft `I jump over the fence.' //
\endgl\xe
\end{figure}

A more literal translation of \rayr{mN lug lhneri}{manga luga lahaneri} is `by
way of the top of the fence', though without the verbosity of the English
translation, since both ways to express the circumstance are about equally long
in Ayeri.

\index{semantic role!instrument|)}
\index{case!instrumental|)}

\subsubsection{Case-unmarked nouns}
\label{subsec:uncased}

Case morphology is applied to nouns in Ayeri basically whenever nouns serve as
complements\index{grammatical function!closed complement} or as adjuncts\index{grammatical function!adjunct}, though there are a number of exceptions to this rule,
as we will see below. For one, the case-unmarked form is the citation form, not
the one declined for agent\index{case!agent}. As a first exception, the unmarked form can be
found when addressing people---one might speak of an unmarked vocative, as
illustrated in (\ref{ex:vocative}).

\begin{figure}[h]
\pex\label{ex:vocative}
\a\label{ex:vocnoun}\begingl
	\gla Raypu, \textbf{petāya}! //
	\glb raypa-u petāya //
	\glc stop-\Imp{} idiot //
	\glft `Stop it, you idiot!' //
\endgl

\a\label{ex:vocname}\begingl
	\gla Sahu edaya, \textbf{Diras}! //
	\glb saha-u edaya Diras //
	\glc come-\Imp{} here Diras //
	\glft `Come here, Diras!' //
\endgl
\xe
\end{figure}

Imperative\index{mood!imperative} forms have underlying second-person\index{person} agents\index{semantic role!agent}, so both the `idiot' in
(\ref{ex:vocnoun}) and Diras in (\ref{ex:vocname}) would be the implied agents\index{semantic role!agent}
of their sentences, yet neither the noun nor the name are marked by the agent\index{case!agent}
markers \rayr{/ANF}{-ang} and \rayr{ANF}{ang}, respectively, since the
addressees occur as appositions. Another case where nouns are not necessarily
marked for case is attested in translations for the prefix\index{prefixes}
\xayr{ku/}{ku-}{like, as though} when the phrase acts as a depictive secondary
predicate, and thus similar to an adverb (compare \autoref{subsubsec:depict},
p.~\pageref{subsubsec:depict}). This is exemplified by
(\ref{ex:depictapnp}).

\begin{figure}[h]
\pex\label{ex:depictapnp}
\a\label{ex:kuudhr}\begingl
	\gla … nay ang @ mya @ rankyon sitanyās \textbf{ku-netu}. //
	\glb … nay ang= mya= rank=yon.Ø sitanya-as ku=netu //
	\glc … and \AgtT{}= shall= treat=\TplN{}.\Top{} 
		each.other-\Parg{} like=brother //
	\glft `… and they shall treat each other like brothers.'\footnotemark%
	\tc{\citep{benung:udhr}}//
\endgl

\a\label{ex:kukafka}\begingl
	\gla … ang @ nunaya \textbf{ku-vipin} … //
	\glb … ang= nuna=ya.Ø ku=vipin … //
	\glc … \AgtT{}= fly=\TsgM{}.\Top{} like=bird … //
	\glft `… he (would) fly like a bird …'%
	\tc{\citep[14]{becker:kafka:imperial}}//
\endgl
\xe
\end{figure}

\footnotetext{The original English text this was translated from has 
\textcquote[Article 1]{udhr}{and should act towards one another in a spirit of 
brotherhood}.}

Strikingly, in example (\ref{ex:kuudhr}), \xayr{netu}{netu}{brother} in 
\xayr{ku/netu}{ku-netu}{like brothers} is not even inflected for plural;
likewise, \xayr{ku/vipinF}{ku-vipin}{like a bird} in (\ref{ex:kukafka}) is not
inflected for case. The depictive NP\index{phrase types!noun phrase} in (\ref{ex:kuudhr}) is also a little
unusual in that it does not occur after\index{word order} the verb in the position of an adverb
as depictives usually would.

Nouns may also be unmarked if they act as modifiers in a compound\index{compounds} and the head
is marked for the NP's\index{phrase types!noun phrase} case and number, for instance as in
(\ref{ex:compunmkd}). Here, \xayr{mpNF}{mapang}{finger}, the modifier in the
compound\index{compounds}, acts in the way of an adjective\index{adjectives} in that `fingernail' is not used as a
syntactic unit as far as case marking goes. Instead, the case marker appears on
the compound's\index{compounds} head, \xayr{rlnF}{ralan}{nail}. Compounds\index{compounds} will be described in
more detail in \autoref{subsec:compounds}.

\begin{figure}[h]
\ex\label{ex:compunmkd}
\begingl
	\gla ralanyeri mapang //
	\glb ralan-ye-ri mapang //
	\glc nail-\Pl{}-\Ins{} finger //
	\glft `with the fingernails' //
\endgl\xe
\end{figure}

Lastly, and probably most importantly, nouns appear superficially unmarked if
topicalized\index{grammatical function!topic}, since the topic\index{grammatical function!topic} marker is a null-morpheme (\fw{-Ø}) if viewed
systematically. We have already seen numerous examples of this above, but 
(\ref{ex:topzeromkd}) gives an example again explicitly.

\begin{figure}[h]
\ex\label{ex:topzeromkd}
\begingl
	\gla Saru-nama, ang nupoyya \textbf{veney} aruno vās. //
	\glb sar-u=nama ang nupa-oy-ya veney-Ø aruno vās //
	\glc go-\Imp{}=just \AgtT{} hurt-\Neg{}-\TsgM{} dog-\Top{} brown 
		\Second{}.\Parg{} //
	\glft `Just go, the brown dog won't hurt you.' //
\endgl\xe
\end{figure}

\index{case|)}

\subsection{Prefixes on nouns}
\label{subsec:nounpref}
\index{prefixes|(}

All of the nominal morphology we have so far dealt with in this section was
suffixing. As mentioned in the previous section already
(p.~\pageref{nounprefixes}), there are also a number of prefixes which can be
applied to nouns. I have just given two examples of the prefix
\xayr{ku/}{ku-}{like, as though} above, but \rayr{ku/}{ku-} applies not only
to nouns, but can be combined with other parts of speech as well. As discussed
in \autoref{subsec:clitics} (p.~\pageref{clitics_prenoun_ku}~ff.), it behaves
in the way of a special clitic in \citet{zwicky1977}'s terminology, since no
corresponding full form exists in its place. (\ref{ex:kucasemkd}) cites another
example from the Ayeri translation of Kafka's short story \enquote{Eine
kaiserliche Botschaft} to illustrate.

\begin{figure}[h]
\ex\label{ex:kucasemkd}
\label{ex:kukafka2}\begingl
	\gla … saylingyāng kovaro naynay, ku-ranyāng palung. //
	\glb … sayling=yāng kovaro naynay ku=ranya-ang palung //
	\glc … progress=\TsgM{}.\Aarg{} easy also like=nobody-\Aarg{} else //
	\glft `… he also got on easily, like nobody else.'%
	\tc{\citep[12]{becker:kafka:imperial}}//
\endgl\xe
\end{figure}

In this example, we can see \rayr{ku/}{ku-} attaching to a properly inflected
NP\index{phrase types!noun phrase}. The NP \xayr{rnYaaNF pluNF}{ranyāng palung}{nobody else} is case-marked for
agent since it can be understood to refer to the verb
\xayr{sjliNF/}{sayling-}{progress} in the main clause, so \xayr{rnYaaNF
pluNF}{ranyāng palung}{nobody else} can replace \xayr{/yaaNF}{-yāng}{he} in the
main clause. While this section deals mainly with prefixes on nouns, it should
be mentioned for completeness that \rayr{ku/}{ku-} may also appear as a suffix\index{suffixes}
under certain conditions. As discussed in \autoref{subsec:clitics}
(p.~\pageref{clitics_prenoun_ku}~ff.), \rayr{ku/}{ku-} moves to the end of the
noun phrase\index{phrase types!noun phrase} when a proper-noun is marked by a case-marker particle. Example
(\ref{ex:kuposvar}) repeats (\ref{ex:clitics_34}) from the previous chapter for
convenience.

\begin{figure}[h]
\pex\label{ex:kuposvar}
\a\begingl
	\gla Ang @ lentava sa @ Tagāti diyan-ku. //
	\glb ang= lenta=va.Ø sa= Tagāti diyan=ku //
	\glc \AgtT{}= sound=\Second{}.\Top{} \Parg{}= Tagāti worthy=like //
	\glft `You sound like Mr. Tagāti.' //
\endgl

\a\begingl
	\gla Ang @ lentava sa @ Tagāti diyan-ku nay diranas yana. //
	\glb ang= lenta=va.Ø sa= Tagāti diyan=ku nay diran-as yana //
	\glc \AgtT{}= sound=\Second{}.\Top{} \Parg{}= Tagāti worthy=like and 
		uncle-\Parg{} \TsgM{}.\Gen{} //
	\glft `You sound like Mr. Tagāti and his uncle.' //
\endgl

\a\begingl
	\gla Sa @ lentavāng ku-​Tagāti diyan. //
	\glb sa= lenta=vāng ku=​Tagāti diyan //
	\glc \PatT{}= sound=\Second{}.\Aarg{} like=​Tagāti worthy //
	\glft `Like Mr. Tagāti you sound.' //
\endgl
\xe
\end{figure}

Besides \rayr{ku/}{ku-}, there are also the demonstrative prefixes
\xayr{d/}{da-}{such}, \xayr{Ed/}{eda-}{this}, and \xayr{Ad/}{ada-}{that},
which have already been mentioned in the previous section as well (see
\autoref{subsec:clitics}, p.~\pageref{clitics_prenoun_dem}). The demonstrative
prefixes undergo crasis with nouns beginning with \fw{a-}, that is, they form
phonological words with their hosts for all means and purposes. An example of
this is given in (\ref{ex:noundemclit}), where \xayr{Ed/}{eda-}{this} merges
with \xayr{AyonF}{ayon}{man} to become \xayr{EdaayonF}{edāyon}{this man}. The
demonstrative prefixes are special clitics\index{clitics} since no contemporary free form
exists.

\begin{figure}[h]
\pex\label{ex:noundemclit}
\a\begingl
	\gla da-nanga kāryo //
	\glb da=nanga kāryo //
	\glc such=house big //
	\glft `such a big house' //
\endgl

\a\begingl
	\gla edāyon nake //
	\glb eda=ayon nake //
	\glc this=man tall //
	\glft `this tall man' //
\endgl

\a\begingl
	\gla ada-envan alingo //
	\glb ada=envan alingo //
	\glc that=woman clever //
	\glft `that clever woman' //
\endgl
\xe
\end{figure}

Moreover, there is a proclitic\index{clitics} \rayr{me/}{mə-} in complementary distribution\index{complementary distribution}
with the demonstrative prefixes. This particle adds a meaning along the lines
of `just any', `whatsoever', `some' to the noun. Note that this clitic\index{clitics} is
distinct from the morpheme indicating an inspecific quantity,
\xayr{/ArilF}{-aril}{some}. Uncharacteristically of a clitic\index{clitics}, but also like the
deictic clitics\index{clitics}, \rayr{me/}{mə-} forms a long vowel if the noun it leans on
begins with an /e/. An example of this is given in (\ref{ex:menoun}).

\begin{figure}[h]
\pex\label{ex:menoun}
\a\begingl
	\gla Ang @ lampyo mə-veney kayvo kirinya. //
	\glb ang= lamp-yo mə=veney-Ø kayvo kirin-ya //
	\glc \AgtT{}= walk-\TsgN{} some=dog-\Top{} along street-\Loc{} //
	\glft `Some dog is walking along the street.' //
\endgl

\a\begingl
	\gla Ang @ noyan mēntānley pegamayayam. //
	\glb ang= no=yan mə=entān-ley pegamaya-yam //
	\glc \AgtT{}= want=\TsgM{}.\Top{} some=punishment-\PargI{} 
		thief-\Dat{} //
	\glft `They demanded some kind of punishment for the thief.' //
\endgl
\xe
\end{figure}

\index{prefixes|)}

\subsection{Compounding}
\label{subsec:compounds}
\index{compounds|(}

With regards to the classification of compounds, \citet{bauer2001} gives some 
helpful typological\index{typology} guidelines. Besides the compound types recognized by 
Sanskrit\index{Sanskrit} grammarians---endocentric (\fw{tatpuruṣa}), coordinative 
(\fw{dvandva}), adjectival-endo\-cent\-ric (\fw{kar\-ma\-dhā\-ra\-ya}), and 
exocentric (\fw{bahuvrīhi})---he also adds synthetic compounds, which Sanskrit\index{Sanskrit} 
did not have \citep[697]{bauer2001}. Overall, he finds that determinative, or 
endocentric, compounds are the most common ones in the languages of the world 
\citep[697]{bauer2001}, especially if the head refers to a location\index{semantic role!location} or source\index{semantic role!source} 
of sorts \citep[702]{bauer2001}.

\citet{gaeta2008}, then, adds to \citet{bauer2001}'s research, based on a
larger sample of grammars surveyed, that compounds for the largest part
correlate with the constituent order\index{word order} of the language, both regarding the order\index{word order}
of verb\index{verbs} and object\index{grammatical function!primary object} and that of noun and genitive \citep[129--133]{gaeta2008}.
Mismatches in headedness occur, but appear to constitute the minority of cases
and may often be explained through historical changes in syntax; he discerns
for one that \textcquote[135]{gaeta2008}{morphology is not autonomous from
syntax}, and that secondly, \textcquote[135]{gaeta2008}{[s]yntax seems to be
the motor of change, which may be then reflected in compounds}. Thirdly, he
finds that lexical conservativism causes atavisms to linger on, reflecting the
syntax of earlier stages of the language \citep[138--139]{gaeta2008}.

For the purpose of gaining at least a little insight into which
types\index{typology} of compounds Ayeri allows---besides endocentric
compounds---a small, non-exhaustive\index{desiderata} survey was conducted based on 130 compounds
from the Ayeri dictionary \citep[Dictionary]{benung}; \autoref{tab:comptyp}
shows the various compound classes and the number of words for each. `Harmonic'
and `disharmonic', respectively, refer to the order\index{word order} of elements; the order\index{word order} is
`harmonic' if it is following the normal constituent order\index{word order} of the language and
`disharmonic' if it is at odds with it \citep{gaeta2008}.

\begin{table}[t]
\caption[Compounds in the Ayeri dictionary]{Compounds in the Ayeri dictionary 
\citep{benung} and their classification (n\,=\,130)}
\begin{tabu} to \linewidth {X[3.5l] X[c] X[c] X[c] X[c] X[c] X[c]}
\tableheaderfont\toprule
Type
	& \multicolumn2{c}{Harmonic}
	& \multicolumn2{c}{Disharmonic}
	& \multicolumn2{c}{Total}
	\\
\toprule

Endocentric (N\,+\,N)
	& 67
	& 51.54\pct
	& 2
	& 1.54\pct
	& 69
	& 53.08\pct
	\\
	
Endocentric (N\,+\,Adj)
	& 18
	& 13.85\pct
	& 4
	& 3.08\pct
	& 22
	& 16.92\pct
	\\

Synthetic (V\,+\,N)
	& 16
	& 12.31\pct
	& 4
	& 3.08\pct
	& 20
	& 15.38\pct
	\\

Coordinative (N\,+\,N)
	& 9
	& 6.92\pct
	& \multicolumn2{c}{---}
	& 9
	& 6.92\pct
	\\
	
Exocentric (N\,+\,N)
	& 1
	& 0.77\pct
	& 3
	& 2.31\pct
	& 4
	& 3.08\pct
	\\
	
\midrule

Unclear
	& 6
	& 4.62\pct
	& \multicolumn2{c}{---}
	& 6
	& 4.62\pct
	\\
	
\midrule

Total
	& 117
	& 90.00\pct
	& 13
	& 10.00\pct
	& 130
	& 100\pct
	\\
	
\bottomrule
\end{tabu}
\label{tab:comptyp}
\end{table}

Unsurprisingly, the largest number of compound nouns in the sample were
endocentric compounds of the regular kind, which means that, just as genitive\index{case!genitive}
attributes follow\index{word order} nouns, noun compounds are headed left. Especially compounds
with adjectives\index{adjectives} are interesting insofar as this is also the normal order\index{word order} for
free adjectives\index{adjectives}, so to illustrate, some tests will be necessary to show that
these adjectives\index{adjectives} form a unit with the head noun and are unable to undergo
comparison, for instance. Synthetic\index{synthesis} compounds exist in Ayeri and produce nouns.
These are compounds in which \textcquote[701]{bauer2001}{the modifying element
in the compound is (usually) interpreted as an argument of the verb from which
the head is derived}. There are also a number of coordinative compounds. This
group, however, is lexicalized and not productive. Exocentric compounds
constitute the minority of the sample. In the following, I will give examples
for each type. It needs to be noted as well that unlike Germanic languages,
Ayeri does not allow compounds of arbitrary length to be strung together, like
in the ridiculous but no less real example from (former) German\index{German} legislation in
(\ref{ex:REUeAUeG}) \parencite[see, for instance,][]{sz:rindfleisch}.

\begin{figure}[h]
\ex\label{ex:REUeAUeG}%
German:\medskip \\
\begingl%
	\gla %
Rind\-fleisch\-­eti\-ket\-tie\-rungs\-­über\-wa\-chungs\-­auf\-gaben\-über\-tra%
\-gungs\-gesetz//
	\glb rind-fleisch-etikettierung-s-überwachung-s­-aufgabe-n%
		-übertragung-s-gesetz //
	\glc cow-meat-labeling-\Lnk{}-supervision-\Lnk{}-duty-\Pl{}%
		-delegation-\Lnk{}-law//
	\glft `law on the delegation of duties in the supervision of beef 
		labeling' //
\endgl\xe
\end{figure}

In stark contrast, Ayeri allows only two elements in compounds. Furthermore,
this section on compounds is located within the section on nouns because Ayeri
almost only possesses compounds involving nouns, and the majority of these also
results in a noun.

\subsubsection{Endocentric compounds}
\label{subsubsec:endocomp}

To start with the largest group, endocentric/\fw{tatpuruṣa} compounds, the bulk
of these compounds combines two nouns, one of which is the head which is
modified by a dependent noun. Since Ayeri exhibits a rather strict head-initial
word order\index{word order}, it comes as no surprise, following \citet{gaeta2008}, that most of
these compounds follow this order\index{word order} strictly: the second noun modifies the first,
which is opposite of how English\index{English}, for instance, typically operates. Examples
from Ayeri are given in (\ref{ex:endonoun}).

\begin{figure}[h]
\ex\label{ex:endonoun}\labels
	\begin{tabular}[t]{@{\tl\quad} l @{\enspace←\enspace} l @{\smallskip}}
	\xayr{\larger betjniMpurF}{betaynimpur}{grape}
		& \xayr{\larger betj}{betay}{berry}
		+ \xayr{\larger niMpurF}{nimpur}{wine}
		\\
	\xayr{\larger krirynF}{karirayan}{vertigo}
		& \xayr{\larger krF}{kar}{fear}
		+ \xayr{\larger IrynF}{irayan}{height}\footnotemark
		\\
	\xayr{\larger pikunMdiNF}{pikunanding}{mustache}
		& \xayr{\larger piku}{piku}{beard}
		+ \xayr{\larger nMdiNF}{nanding}{lips}
		\\
	\xayr{\larger tpjperinF}{tapayperin}{sunblind}
		& \xayr{\larger tpj}{tapay}{screen}
		+ \xayr{\larger perinF}{perin}{sun}
		\\
	\end{tabular}
\xe
\end{figure}

\footnotetext{\rayr{IrynF}{irayan}, however, is a transparent nominalization 
of \xayr{Irj}{iray}{high}.}

The example words in (\ref{ex:endonoun}) show that the relationships between 
the modifier and the head are various: a grape is a berry \emph{used} to 
make wine from \parencite[compare][702]{bauer2001}; vertigo is the fear 
\emph{of} height; a mustache is a beard \fw{located} over the lips 
\parencite[702]{bauer2001}; and a sunblind is a screen \fw{against} the 
sun.
% \footnote{Further examples include:
% \xayr{AvnMdirunF}{avanandirun}{square root}, lit. `base-square'; 
% \xayr{bidmihnye}{bidamihanaye}{xylophone}, lit. `block-wood-\Pl{}';
% \xayr{bgmFtupoj}{bagamtupoy}{dragon}, lit. `lizard-fire'; 
% \xayr{binMpdNF}{binampadang}{memory}, lit. `picture-mind'; 
% \xayr{burNu\_in}{buranguina}{elephant}, lit. `animal-nose'; 
% \xayr{dgmiMdoj}{dagamindoy}{menu}, lit. `choose-card'; 
% \xayr{dlMpsiNF}{dalampasing}{giraffe}, lit. `cow-neck'; 
% \xayr{drMdevo}{darandevo}{skull}, lit. `bone-head'; 
% \xayr{deveMthaanF}{deventahān}{alphabet}, lit. `system-writing'; 
% \xayr{glimehirF}{galimehir}{resin, tar}, lit. `juice-tree'; 
% \xayr{koybhisF}{koyabahis}{diary}, lit. `book-day'; 
% \xayr{ltuMkem}{latunkema}{tiger}, lit. `lion-stripe'; 
% \xayr{lonupt}{lonupata}{poultice}, lit. `bandage-mash'; 
% \xayr{mliMkronF}{malinkaron}{coast, seashore}, lit. `shore-sea'; 
% \xayr{mehirFgtNF}{mehirgatang}{ovaries}, lit. `tree-womb'; 
% \xayr{mehisiNj}{mehisingay}{conifer}, lit. `tree-needle'; 
% \xayr{mikYnFsitemF}{micansitem}{electric}, lit. `power-lightning'; 
% \xayr{mirMthnF}{mirantahan}{typeface}, lit. `kind-writing'; 
% \xayr{mirMthaanF}{mirantahān}{spelling}, lit. `way-writing'; 
% \xayr{mitFrmtau}{mitramatau}{pubic hair}, lit. `hair-tangle'; 
% \xayr{mitFrnvsNF}{mitranavasang}{axillary hair}, lit. `hair-sweat'; 
% \xayr{nrMbesuhej}{narambesuhey}{dictionary}, lit. `word-list'; 
% \xayr{niMpurivnF}{nimpurivan}{vinyard}, lit. `wine-mountain'; 
% \xayr{ptyelNF}{patayelang}{concrete}, lit. `mash-stone'; 
% \xayr{pikulkj}{pikulakay}{goatee}, lit. `beard-chin'; 
% \xayr{prihiNumo}{prihingumo}{desk}, lit. `table-work'; 
% \xayr{rgMterFpeNF}{raganterpeng}{diameter}, lit. `line-middle'; 
% \xayr{rlmpNF}{ralamapang}{fingernail}, lit. `nail-finger'; 
% \xayr{ridspj}{ridasapay}{glove}, lit. `sock-hand'; 
% \xayr{sNumospoj}{sangumosapoy}{ticket office}, lit. `office-ticket'; 
% \xayr{svtkNF}{savatakang}{tank}, lit. `cart-armor'; 
% \xayr{sayMprl}{sayamparal}{urine hole}, lit. `hole-penis'; 
% \xayr{syNu\_in}{sayanguina}{nostril}, lit. `hole-nose'; 
% \xayr{syniv}{sayaniva}{eye socket}, lit. `hole-eye'; 
% \xayr{selNblN}{selangbalang}{search engine}, lit. `machine-search'; 
% \xayr{selMkurnF}{selangkuran}{computer}, lit. `machine-counting'; 
% \xayr{sepFrkronF}{seprakaron}{ditch}, lit. `cleft-water'; 
% \xayr{similitnF}{similitan}{borderland}, lit. `land-margin-\Nmlz{}'; 
% \xayr{similitj}{similitay}{republic}, lit. `land-democracy'; 
% \xayr{sirjyil}{sirayyila}{knee}, lit. `joint-foot'; 
% \xayr{sirjtinu}{siraytinu}{elbow}, lit. `joint-arm'; 
% \xayr{sirukronF}{sirukaron}{starfish}, lit. `star-water'; 
% \xayr{sirusitFrmF}{sirusitram}{comet}, lit. `star-tail'; 
% \xayr{sirutj}{sirutay}{night}, lit. `star-time'; 
% \xayr{sitNlugaanF}{sitanglugān}{incest}, lit. `self-entry'; 
% \xayr{trFtrihimF}{tartarihim}{tobacco}, lit. `pipe-weed'; 
% \xayr{tepilFpihaanF}{tepilpihān}{fester}, lit. `sore-pus'; 
% \xayr{tFreMdpNisF}{trendapangis}{bank}, lit. `hall-money'; 
% \xayr{tuptinu}{tupatinu}{fathom}, lit. `length-arm'; 
% \xayr{veb\_osnF}{vebaosan}{slug}, lit. `snail-slime'; 
% \xayr{veMkubesonF}{venkubeson}{navy}, lit. `army-ship'; 
% \xayr{vinimyonF}{vinimayon}{monkey}, lit. `forest-man'; 
% \xayr{yno\_avnF}{yanoavan}{area, region}, lit. `place-ground'; 
% \xayr{yelNFssaanF}{yelangsasān}{cobblestone}, lit. `stone-way'; 
% \xayr{yenukrFdNF}{yenukardang}{classmates}, lit. `group-school'; 
% \xayr{yutnjkonF}{yutanaykon}{foreskin}, lit. `skin-cover'.
% }
\citet{bauer2001} mentions that \textquote{there may be special morphophonemic
processes which apply between the elements of compounds}, such as
\textcquote[695]{bauer2001}{phonological merger[s] between the elements of the
compound}. This also occasionally happens in Ayeri, as the example words in
(\ref{ex:endonounmod}) show.

\begin{figure}[h]
\pex\label{ex:endonounmod}
	\a \xayr{\larger AvrrnF}{avararan}{wetland} \\
		← \xayr{\larger AvnF}{avan}{ground}
		+ \xayr{\larger rro}{raro}{wet}
		+ \rayr{\larger /AnF}{-an} (\Nmlz{})
	\a \xayr{\larger mehimitFrNF}{mehimitrang}{fiber tree} \\
		← \xayr{\larger mehirF}{mehir}{tree}
		+ \xayr{\larger mitFrNF}{mitrang}{hair, fiber}
	\a \xayr{\larger niNMpinmF}{ningampinam}{bedtime story} \\
		← \xayr{\larger niNnF}{ningan}{story}
		+ \xayr{\larger pinmF}{pinam}{bed}
	\a \xayr{\larger pdilmikYnF}{padilamican}{gravitational force} \\
		← \xayr{\larger pdilnF}{padilan}{attraction}
		+ \xayr{\larger mikYnF}{mican}{force, power}
\xe
\end{figure}

There is a modicum of alteration happening in all of the heads of the example 
words in (\ref{ex:endonounmod}), mostly nasals assimilating to the stop or 
nasal which the modifier begins with (/n/~+~/p/~→~/mp/, /n/~+~/m/~→~/m/), 
though \rayr{AvrrnF}{avararan} and \rayr{mehimitFrNF}{mehimitrang} even delete 
whole coda segments.
% Stuff may even be mashed together completely, but examples??
\citet[703]{bauer2001} notes that very commonly, genitive and plural markers 
may form linking elements, though he also gives examples of languages which
allow other case markers on the modifying element in languages with head-final
order\index{word order}; individual languages may allow even more case inflection. However, this
appears not to happen in Ayeri. The only element that comes up time and again
in between the two halves of compounds is the nominalizer \rayr{/AnF}{-an},
which signifies that the head is being formed by a nominalized root, such as in
\rayr{pdilmikYnF}{padilamican}, where \xayr{pdilnF}{padilan}{attraction} is a 
nominalization of \xayr{pdilF/}{padil-}{attract}, or in 
\rayr{niNMpinmF}{ningampinam}, where \xayr{niNnF}{ningan}{story} is derived 
from the verb \xayr{niNF/}{ning-}{tell}. However, since Ayeri is head-initial
and possessive phrases are dependent marking\index{marking strategies!dependent-marking}, genitive\index{case!genitive} or other case marking\index{case}
would be expected on the second element, not the first. Case marking\index{case} on a
compound, however, does not inflect just the modifier, but the whole NP\index{phrase types!noun phrase}, as 
(\ref{ex:compinfl}) shows.

\begin{figure}[h]
\ex\label{ex:compinfl}\begingl
	\gla Ang @ ningya sipikanena koyabahisena. //
	\glb ang= ning-ya sipik-an-ena koyabahis-ena //
	\glc \AgtT{}= talk-\TsgM{}.\Top{} keep-\Nmlz{}-\Gen{} book.day-\Gen{} //
	\glft `He talks about keeping a journal.' //
\endgl\xe
\end{figure}

\rayr{koybhisen}{koyabahisena} in this example is not to be interpreted as 
`book of day(s)' but as `of a day-book'. Inflection between the parts of a
compound can happen nonetheless, though. In compounds which are formed \fw{ad
hoc} or which are otherwise transparent in their composition (`loose'
compounds\label{loosecomp}), inflection often is deferred to the head noun
instead of the edge of the compound as a whole; the modifier is possibly treated
as an adjunct\index{grammatical function!adjunct} in this case, and stays uninflected. An example of this is given
in (\ref{ex:nouncompdiv}).

\begin{figure}[h]
\ex\label{ex:nouncompdiv}\begingl
	\gla Sa @ trayeng tipin ralanyeri mapang yena. //
	\glb sa= tra=yeng tipin-Ø ralan-ye-ri mapang yena //
	\glc \PatT{}= scratch=\TsgF{}.\Aarg{} itch-\Top{} nail-\Pl{}-\Ins{} 
		finger \TsgF{}.\Gen{} //
	\glft `The itch, she scratches it with her fingernails.' //
\endgl\xe
\end{figure}

Besides noun modifiers, there are also compounds where the modifier is an 
adjective\index{adjectives}. In classical Sanskrit\index{Sanskrit} terminology, this type is called 
\fw{karmadhāraya} \citep[698--699]{bauer2001}.\footnote{\citet{bauer2001} 
also mentions that appositional compounds like \fw{maid-servant}, \fw{woman
doctor} and \fw{fighter-bomber} are counted in this category
\citep[699]{bauer2001}. Ayeri, however, does not possess such formations in
particular.} Examples in Ayeri include those listed in
(\ref{ex:ayrnounadjcomp}). In all of these cases, the adjective\index{adjectives} forms a unified
lexeme with the head noun, hence it is not comparable, as the examples in
(\ref{ex:nounadjcompsupl}) show.
% \footnote{Further examples include: 
% \xayr{bhisino}{bahisino}{holiday, day off}, lit. `day-free'; 
% \xayr{dikuMtrinF}{dikuntaring}{bureaucracy}, lit. `passion-administrative'; 
% \xayr{leMtMkusNF}{lentankusang}{diphthong}, lit. `sound-double'; 
% \xayr{nNbnY}{nangabanya}{hospital}, lit. `house-sick'; 
% \xayr{naraaMtiynF}{narāntiyan}{conlang}, lit. `language-created-\Nmlz{}'; 
% \xayr{rohMpraanF}{rohamparān}{snack}, lit. `bite-quick-\Nmlz{}'; 
% \xayr{rohMkivo}{rohankivo}{snack}, lit. `bite-small'; 
% \xayr{sNumirj}{sangumiray}{ministry, authority}, lit. `office-high'; 
% \xayr{tbMpehu}{tabampehu}{lower jaw}, lit. `jaw-loose'
% \xayr{tabnikp}{tabanikapa}{upper jaw}, lit. `jaw-attached'.
% }

\begin{figure}[h]
\ex\labels\label{ex:ayrnounadjcomp}
	\begin{tabular}[t]{@{\tl\quad} l @{\enspace←\enspace} l @{\smallskip}}
	\xayr{\larger krFdNirj}{kardangiray}{university}
		& \xayr{\larger krFdNF}{kardang}{school}
		+ \xayr{\larger Irj}{iray}{high} \\
		
	\xayr{\larger mrsFhri}{marashari}{witticism}
		& \xayr{\larger mrsF}{maras}{phrase}
		+ \xayr{\larger hri}{hari}{pithy} \\
		
	\xayr{\larger silFvniknF}{silvanikan}{overview}
		& \xayr{\larger silFvnF}{silvan}{view}
		+ \xayr{\larger IknF}{ikan}{whole} \\
		
	\xayr{\larger vipimkaarY}{vipimakārya}{crow}
		& \xayr{\larger vipinF}{vipin}{bird}
		+ \xayr{\larger mkaarY}{makārya}{black} \\
	\end{tabular}
\xe
\end{figure}

\begin{figure}[h]
\ex\label{ex:nounadjcompsupl}\labels
\begin{tabular}[t]{@{} l @{\quad} l @{\hspace{2em}} l}

		& \textsc{comparative}			& \textsc{superlative} \medskip \\

	\tl	& \ljudge*\fw{kardangiray-eng}	& \fw{kardangiray-vā} \\
		& kardang-iray=eng				& kardang-iray=vā \\
		& school-[high=\Comp{}]			& school-*[high=\Supl{}] \\
		& `higher-school'				& `highest-school' \medskip \\

	\tl	& \ljudge*\fw{marashari-eng}	& \fw{marashari-vā} \\
		& maras-hari=eng				& maras-hari=vā \\
		& phrase-[pithy=\Comp{}]		& phrase-*[pithy=\Supl{}] \\
		& `pithier-phrase'				& `pithiest-phrase' \\
\end{tabular}
\xe
\end{figure}

In fact, it \emph{is} possible to form \rayr{krFdNirj/vaa}{kardangiray-vā} and
\rayr{mrsFhti/vaa}{marasari-vā}, but they mean `most universities' and `most
witticisms', that is, \rayr{/vaa}{-vā} here does not mark the adjectival part
as a superlative\index{comparison} form; the suffix\index{suffixes} modifies the noun--adjective\index{adjectives} compound as a
whole: \textit{$[$school-high$]$=most}, \textit{$[$phrase-pithy$]$=most}.
\xayr{/ENF}{-eng}{rather} as a quantifier\index{quantifiers} does not combine with nouns, which is
why the first examples in (\ref{ex:nounadjcompsupl}ab) are both ungrammatical
\fw{per se}.

Since the meaning of noun--adjective\index{adjectives} compounds is often idiomatic, they also
cannot be divided as shown above in (\ref{ex:nouncompdiv}), since a
\xayr{krFdNirj}{kardangiray}{university} is not a
\xayr{krFdNF}{kardang}{school} which is \xayr{Irj}{iray}{high} in the literal
sense, but a school of the highest tier. \rayr{krFdNen Irj}{kardangena iray}
(school-\Gen{} high), then, can only be interpreted in the literal sense, `of
the high school', but not as `of the university', which thus can only be
\rayr{krFdNiryen}{kardangirayena}.

In the sample, there were also a few compounds which were categorized as
noun--noun combinations and which look as though they violate head-initial
order\index{word order}. All of these involve \xayr{sitNF}{sitang}{self} as a modifier, for
instance, as in (\ref{ex:sitangcomp}).

\begin{figure}
\ex\labels\label{ex:sitangcomp}
	\begin{tabular}[t]{@{\tl\quad} l @{\enspace←\enspace} l @{\smallskip}}
	\xayr{\larger sitNFleMtnF}{sitanglentan}{vowel}
		& \xayr{\larger sitNF}{sitang}{self}
		+ \xayr{\larger leMtnF}{lentan}{sound}
		\\
	\xayr{\larger sitNFpronaanF}{sitangparonān}{self-confidence}
		& \xayr{\larger sitNF}{sitang}{self}
		+ \xayr{\larger pronaanF}{paronān}{faith}
		\\
	\xayr{\larger sitNFtenYnF}{sitangtenyan}{suicide}
		& \xayr{\larger sitNF}{sitang}{self}
		+ \xayr{\larger tenYnF}{tenyan}{death}
		\\
	\end{tabular}
\xe
\end{figure}

\rayr{sitNF}{sitang} does not exist as a noun by itself in Ayeri, the word for 
`self' is its nominalization, \rayr{sitNnF}{sitangan}. Nonetheless, it looks 
as though it could have plausibly been a noun once. However, this noun 
may have been grammaticalized into a reflexive morpheme of a more 
general kind, which in turn gave rise to the form \rayr{sitNnF}{sitangan} as a 
renovation.\footnote{A little bit of language history would certainly simplify 
things here and lend them credence. Let us simply assume that 
\rayr{sitNF}{sitang} used to be a noun meaning something like `self' at a 
previous stage of Ayeri and was repurposed as a reflexive prefix\index{prefixes}. 
\citet{lehmann2015} quotes a few examples of what he calls `autophoric' nouns 
that came to be used as reflexive pronouns in their respective language: 
\textcquote[45--46]{lehmann2015}{Typical examples are Sanskrit \fw{tan} 
`body, person' and \fw{ātmán} `breath, soul', Buginese \fw{elena} `body',
Okinawan \fw{dūna} `body', !Xu \fw{l’esi} `body', Basque \fw{burua} `head',
Abkhaz \fw{a-xə̀} `the head'. In their respective languages, all these nouns
are translation equivalents of English \fw{self}}. Thus, it would not be out of
line at all to assume such a grammaticalization\index{grammaticalization} path for Ayeri as well.} The
reflexive \rayr{sitNF}{sitang} is used---as we have seen in the previous
chapter---as a prefix\index{prefixes}, so there are two ways to intepret these formations:
first, \rayr{sitNF}{sitang} may be the reflexive prefix\index{prefixes} here and thus the
compound follows the normal syntactic order\index{word order}; or second, the order\index{word order} of elements
is reversed and thus may reflect an earlier stage of Ayeri where
\rayr{sitNF}{sitang} was still a noun and modifiers could still appear in front
of their heads, at least optionally so \citep[133--137]{gaeta2008}.

There are a number of genuinely reversed endocentric compounds as well,
however, in which the modifier comes first and the head last. There are only a
few of these in the sample; (\ref{ex:endocomp}) lists all of them.

\begin{figure}[h]
\ex\labels\label{ex:endocomp}
	\begin{tabular}[t]{@{\tl\quad} l @{\enspace←\enspace} l @{\smallskip}}
	\xayr{\larger bript}{baripata}{ground meat}
		& \xayr{\larger bri}{bari}{meat}
		+ \xayr{\larger pt}{pata}{mash}
		\\
	\xayr{\larger kjvoleMtnF}{kayvolentan}{consonant}
		& \xayr{\larger kjvo}{kayvo}{with}
		+ \xayr{\larger leMtnF}{lentan}{sound}
		\\
	\xayr{\larger maavgneNF}{māvaganeng}{mother's siblings}
		& \xayr{\larger maav}{māva}{mother}
		+ \xayr{\larger gneNF}{ganeng}{siblings}
		\\
	\xayr{\larger mtinMdiNF}{matinanding}{labia}
		& \xayr{\larger mtiknF}{matikan}{hot}
		+ \xayr{\larger nMdiNF}{nanding}{lips}
		\\
	\xayr{\larger muyvirNF}{muyavirang}{brass}
		& \xayr{\larger muy}{muya}{false}
		+ \xayr{\larger AvirNF}{avirang}{gold}
		\\
	\xayr{\larger tonisjtNF}{tonisaytang}{self-assured}
		& \xayr{\larger tonis}{tonisa}{assured}
		+ \ques{}\,\xayr{\larger sitNnF}{sitangan}{self}
		\\
	\end{tabular}
\xe
\end{figure}

% Given the discussion of \rayr{sitNF}{sitang} above, one word among the
% examples above whose origin is not quite clear is
% \rayr{tonisjtNF}{tonisaytang}, which appears to contain a deviant form of
% either \rayr{sitNF}{sitang} or \rayr{sitNnF}{sitangan}, which is preceded by
% the adjective \xayr{tonis}{tonisa}{assured, ascertained}.

All of the previously mentioned compounds involving nominal elements formed 
nouns, though there are also a few denominal compounds in the sample. This 
process is not productive, however, and interestingly, only noun--adjective\index{adjectives} 
combinations appear in this group. These are listed in (\ref{ex:denomcomp}).

\begin{figure}[h]
\ex\labels\label{ex:denomcomp}
	\begin{tabular}[t]{@{\tl\quad} l @{\enspace←\enspace} l @{\smallskip}}
	\xayr{\larger mirMpluj}{mirampaluy}{otherwise}
		& \xayr{\larger mirnF}{miran}{way}
		+ \ques{}\,\xayr{\larger pluNF}{palung}{different}
		\\
	\xayr{\larger pdbnY}{padabanya}{insane}
		& \xayr{\larger pdNF}{padang}{mind}
		+ \xayr{\larger bny}{banaya}{sick}
		\\
	\xayr{\larger teMkris/}{tenkarisa-}{be scared to death}
		& \xayr{\larger tenF}{ten}{life}
		+ \xayr{\larger kris}{karisa}{frightened}
		\\
	\end{tabular}
\xe
\end{figure}

As for the examples in (\ref{ex:denomcomp}), \rayr{mirMpluj}{mirampaluy} is an
adverb whose modifier is probably a mangling of \rayr{pluNF}{palung}.
\rayr{pdbnY}{padabanya} is an adjective\index{adjectives} meaning `insane' rather than the
expected `insanity' (instead: \rayr{pdbnYaanF}{padabanyān}). Lastly,
\rayr{teMkris/}{tenkarisa-} acts as a verb, possibly from conversion or
reinterpretation, since the suffix\index{suffixes} \rayr{/Is}{-isa} also forms morphological
causatives\index{causation} of a number of verbs\index{verbs}. Besides these irregularities, there is also at
least one noun compound which uses a postposition\index{adpositions!postpositions} as an adjectival modifier,
given in (\ref{ex:nounpostposcomp}). This compound must be derived from the
phrase \xayr{silFvnFy kjvj}{silvanya kayvay}{without sight} (see-\Nmlz{}-\Loc{}
without), though here as well, the word roots are simply juxtaposed, as is the
common way to form compounds in Ayeri.

\begin{figure}[h]
\ex\label{ex:nounpostposcomp}
	\xayr{\larger silFvMkjvj}{silvankayvay}{blindness} 
	← \xayr{\larger silFvnF}{silvan}{sight}
	+ \xayr{\larger kjvj}{kayvay}{without}
\xe
\end{figure}

\subsubsection{Synthetic compounds}
\index{synthesis|(}

According to \citet{bauer2001}, (semi-)synthetic compounds, or verbal(-nexus)
compounds, are compounds that have \textcquote[701]{bauer2001}{been variously
defined as being based on word-groups or syntactic constructions
\citep[2]{botha1984}, or as compounds whose head elements are derived from
verbs\index{verbs} \citep[3607]{lieber1994}}. Examples of this type in English would include
\fw{truck-driver}, \fw{peace-keeping}, and \fw{home-made}. He mentions also 
that synthetic compounds have been mainly discussed with regards to Germanic
languages, but that according to \citet[3608]{lieber1994}, the phenomenon is
much more widespread.

Ayeri possesses compounds like this as well, and the regular case again follows
the constituent order\index{word order}, here that of verbs\index{verbs} and nouns: Ayeri is a VO language,
and thus the verb\index{verbs} as the head of the compound is usually found on the left side
with its nominal modifier following\index{word order} it \citep[compare][129--133]{gaeta2008},
compare (\ref{ex:verbnouncomp}).

\begin{figure}[h]
\ex\labels\label{ex:verbnouncomp}
	\begin{tabular}[t]{@{\tl\quad} l @{\enspace←\enspace} l @{\smallskip}}
	\xayr{\larger AnFlgonnF}{anlagonan}{pronunciation}
		& \xayr{\larger AnFlF/}{anl-}{bring}
		+ \xayr{\larger AgonnF}{agonan}{outside}
		\\
	\xayr{\larger npkronF}{napakaron}{acid}
		& \xayr{\larger npF/}{nap-}{burn}
		+ \xayr{\larger kronF}{karon}{water}
		\\
	\xayr{\larger npperinF}{napaperin}{sunburn}
		& \xayr{\larger npF/}{nap-}{burn}
		+ \xayr{\larger perinF}{perin}{sun}
		\\
	\xayr{\larger telFbssaanF}{telbasasān}{waysign}
		& \xayr{\larger telFb/}{telba-}{show}
		+ \xayr{\larger ssaanF}{sasān}{way}
		\\
	\end{tabular}
\xe
\end{figure}

Here as well, the relations between verb\index{verbs} and noun are various, that is, the
nominal modifier is not simply the direct object of the verb: to pronounce
something means to bring it \emph{to} the outside; a sunburn is a burn
\emph{caused by} the sun;\index{causation} and a waysign \emph{shows} the way
(\rayr{ssaanF}{sasān} is the object here). Even though \rayr{kronF}{karon} may
serve as an agent\index{semantic role!agent} (or a causer\index{semantic role!causer}) of the burning effect of acid (similar for
\xayr{npperinF}{napaperin}{sunburn}), the verb-first order\index{word order} is justified here as
well, since verbs always come first\index{word order} in Ayeri sentences, and any other NPs\index{phrase types!noun phrase},
whether actor\index{semantic role!agent} or undergoer\index{semantic role!patient}, are following.
% \footnote{Further examples include:
% \xayr{bimkNnF}{bimakangan}{photo}, lit. `paint-light-\Nmlz{}'; 
% \xayr{IlgonnF}{ilagonan}{edition}, lit. `give-out-\Nmlz{}'; 
% \xayr{lMtmidj}{lantamiday}{diversion}, lit. `lead-around'; 
% \xayr{nbisFmaavy}{nabismāvaya}{motherfucker}, lit. `fuck-mother-\Agtz{}'; 
% \xayr{nrkhu}{narakahu}{phone}, lit. `speak-far'; 
% \xayr{srsjliNF}{sarasayling}{progress}, lit. `go-further'; 
% \xayr{silFvkhu}{silvakahu}{TV}, lit. `see-far'; 
% \xayr{silFvmrinnF}{silvamarinan}{preview}, lit. `see-before-\Nmlz{}'; 
% \xayr{telFbgonnF}{telbagonan}{advertisement}, lit. `show-out'; 
% \xayr{vliktu}{valikatu}{masochist}, lit. `enjoys pain'.
% }

Just as with endocentric compounds, there are a number of seeming exceptions to
the verb-first order\index{word order} of synthetic compounds. These are just as far and few
between, however, and whether they should all be counted as noun--verb
combinations is also questionable, since they appear to all be formed with
nominalized verbs. The verbal element may thus be only indirectly verbal for
the purposes of compounding. If interpreted as noun--noun combinations, the
nominal first element would reasonably form the head again for some of the
words in (\ref{ex:compvbrev}).

\begin{figure}[h]
\pex\label{ex:compvbrev}
	\a \xayr{\larger mripuMtymF}{maripuntayam}{spread} \\
		← \xayr{\larger mrinF}{marin}{surface}
		+ \xayr{\larger puMt/}{punta-}{stroke}
		+ \rayr{\larger /ymF}{-yam} (\Dat{})
	\a \xayr{\larger ssnFlekaanF}{sasanlekān}{labyrinth} \\
		← \xayr{\larger ssaanF}{sasān}{way}
		+ \xayr{\larger lek/}{leka-}{guess}
		+ \rayr{\larger /AnF}{-an} (\Nmlz{})
	\a \xayr{\larger selNnunaanF}{selangnunān}{plane} \\
		← \xayr{\larger selNF}{selang}{machine}
		+ \xayr{\larger nun/}{nuna-}{fly}
		+ \rayr{\larger /AnF}{-an} (\Nmlz{})
	\a \xayr{\larger siMturaanF}{sinturān}{radio} \\
		← \xayr{\larger siMto}{sinto}{wave}
		+ \xayr{\larger tur/}{tura-}{send}
		+ \rayr{\larger /AnF}{-an} (\Nmlz{})
\xe
\end{figure}

\rayr{mripuMtymF}{maripuntayam} is special in that it contains the dative
suffix\index{suffixes} \rayr{/ymF}{-yam} which is lexicalized as a part of the word: something
made or intended for spreading on a surface. A few more such verbal\index{verbs} derivations\index{derivation}
can be found, though not compounds, among others in those words listed in 
(\ref{ex:yamderiv}).

\begin{figure}[h]
\ex\label{ex:yamderiv}\labels
	\begin{tabular}[t]{@{\tl\quad} l @{\enspace←\enspace} l @{\smallskip}}
	\xayr{\larger gFrenYmF}{grenyam}{extremity}
		& \xayr{\larger gFren/}{gren-}{reach out}
		\\
	\xayr{\larger lugymF}{lugayam}{password}
		& \xayr{\larger lug/}{luga-}{go through} 
		\\
	\xayr{\larger shymF}{sahayam}{future}
		& \xayr{\larger sh/}{saha-}{come}
		\\
	\end{tabular}
\xe
\end{figure}

There is also \xayr{mripuMt/}{maripunta-}{spread over} as the verb corresonding
to \rayr{mripuMtymF}{maripuntayam}, though its meaning is less specific. Here
as well, however, the verbal part is last instead of first. For the other
example words (\ref{ex:compvbrev}b--d), an interpretation of the second part as
a deverbal noun is possible: a labyrinth as a way or path which requires
guessing, a plane as a machine for flight, and radio as a transmission of
waves. In the latter case, \rayr{siMturaanF}{sinturān}, however, the head is
still on the wrong side even if one interprets all of the above examples as
noun--noun compounds with a deverbal element.

\index{synthesis|)}

\subsubsection{Coordinative compounds}
\index{coordination|(}

Coordinative compounds are a very small group among the sample drawn from the
dictionary, and not a very productive one. \citet{bauer2001} defines this class
as having \textcquote[699]{bauer2001}{two or more words in a coordinate
relationship, such that the entity denoted is the totality of the entities
denoted by each of the elements}. He cautions that they are very easily
confused with appositional (also \fw{karmadhāraya}) compounds in that both
types of compound allow inserting an \fw{and} between both elements. The
nominal coordinative compounds included in the sample are listed in
(\ref{ex:ayrdvand}).

\begin{figure}[h]
\ex\labels\label{ex:ayrdvand}
	\begin{tabular}[t]{@{\tl\quad} l @{\enspace←\enspace} l @{\smallskip}}
	\xayr{\larger baaːm}{bāmā}{mom-and-dad}
		& \xayr{\larger baa(baa)}{bā(bā)}{dad}
		+ \xayr{\larger maa(maa)}{mā(mā)}{mom}
		\\
	\xayr{\larger pFrujnpj}{pruynapay}{seasoning}
		& \xayr{\larger pruj}{pruy}{salt}
		+ \xayr{\larger npj}{napay}{pepper}
		\\
	\xayr{\larger spjyil}{sapayyila}{hands-and-feet}
		& \xayr{\larger spj}{sapay}{hand}
		+ \xayr{\larger yil}{yila}{foot}
		\\
	\xayr{\larger simileno}{simileno}{horizon}
		& \xayr{\larger similF}{simil}{country}
		+ \xayr{\larger leno}{leno}{sky}
		\\
	\xayr{\larger sitemFrugonF}{sitemrugon}{thunderstorm}
		& \xayr{\larger sitemF}{sitem}{lightning}
		+ \xayr{\larger rugonF}{rugon}{thunder}
		\\
	\xayr{\larger vekmFdekej}{vekamdekey}{dishes}
		& \xayr{\larger vekmF}{vekam}{plate}
		+ \xayr{\larger dekej}{dekey}{fork}
		\\
	\end{tabular}
\xe
\end{figure}

None of the two elements recognizably forms the head in these examples, but
both elements are typical components of the thing the compound signifies.
\citet[699]{bauer2001} mentions that coordinative adjective\index{adjectives} compounds are rare,
or at least rarely documented in the grammars he surveyed. In our sample, only
the compound in (\ref{ex:adjadjcomp}) is included. This compound forms a noun
from the combination of two adjectives\index{adjectives}, insofar it is relevant to this section
even though the component parts are not nouns.

\begin{figure}[h]
\ex\label{ex:adjadjcomp}
	\xayr{\larger mkgisu}{makagisu}{twilight}
		← \xayr{\larger mk}{maka}{light}
		+ \xayr{\larger gisu}{gisu}{dark}
\xe
\end{figure}

The sample also includes the two words in (\ref{ex:verbverbcomp}), which are,
however, neither made up from nouns, nor do they form a noun in combination.
Instead, they are technically verbs\index{verbs} combining to form directional adverbs and
have been exceptionally included here for completeness.

\begin{figure}[h]
\ex\labels\label{ex:verbverbcomp}
	\begin{tabular}[t]{@{\tl\quad} l @{\enspace←\enspace} l @{\smallskip}}
	\xayr{\larger mNsh}{mangasaha}{towards}
		& \xayr{\larger mN/}{manga-}{move}
		+ \xayr{\larger sh/}{saha-}{come}
		\\
	\xayr{\larger mNsr}{mangasara}{away}
		& \xayr{\larger mN/}{manga-}{move}
		+ \xayr{\larger sr}{sara-}{go}
		\\
	\end{tabular}
\xe
\end{figure}

\index{coordination|)}

\subsubsection{Exocentric compounds}
\index{idioms|(}

In exocentric compounds, the modifier is not a hyponym of its head
\citep[700]{bauer2001}, which means that the modifier is not describing a
property that more closely determines its head. So while a \fw{dog house} is a
type of house made for dogs, the head of an \fw{egghead} is neither for eggs,
nor containing eggs, nor made of eggs; instead, it refers to an egg-shaped
skull metaphorically. And while a \fw{bluecollar} may wear a blue shirt
professionally, the referent it signifies is not a type of collar, but the
relationship is metonymical in that the blue collar is part of the guise of the
signified entity as a whole. The sample from the Ayeri dictionary contains a
few compounds of this kind as well, listed in (\ref{ex:exocentcomp}). Again,
however, it is not a very productive group.

\begin{figure}[h]
\ex\labels\label{ex:exocentcomp}
	\begin{tabular}[t]{@{\tl\quad} l @{\enspace←\enspace} l @{\smallskip}}
	\xayr{\larger AvnFyonNF}{avanyonang}{artery}
		& \xayr{\larger AvnF}{avan}{bottom, down}
		+ \xayr{\larger yonNF}{yonang}{stream}
		\\
	\xayr{\larger bjtMdevo}{baytandevo}{headache}
		& \xayr{\larger bjtNF}{baytang}{blood}
		+ \xayr{\larger devo}{devo}{head}
		\\
	\xayr{\larger linFyonNF}{linyonang}{vein}
		& \xayr{\larger liNF}{ling}{top, up}
		+ \xayr{\larger yonNF}{yonang}{steam}
		\\
	\xayr{\larger siMdjnN}{sindaynanga}{address}
		& \xayr{\larger sindj}{sinday}{number}
		+ \xayr{\larger nN}{nanga}{house}
		\\
	\end{tabular}
\xe
\end{figure}

What is striking here is that only one out of four examples shows the expected
head-initial order\index{word order}: \rayr{siMdjnN}{sindaynanga}. The other three examples all
have the head component on the right side, preceded\index{word order} by a modifier. However,
what all of these have in common, is that they are only metaphorically or
metonymically describing the thing they signify: veins and arteries are not
literally streams going up or down (they are a kind of stream flowing in
different directions, however, so these are probably on the borderline between
exocentric and endocentric); a headache is related to the head, but has not
directly to do with being made of or containing blood (the rationale behind
this being a superstition that you have too much blood in your head, which is
said to cause the pain); and a house number may be part of an address, but is
in a \fw{pars pro toto} relationship to it.

\index{idioms|)}

\subsubsection{A few mysterious cases}

The following words from our sample were either undeterminable as to their 
composition due to parts of the word not being clear regarding one of their 
constituent parts, either because I tweaked the constituent so much as to not 
be readily recognizable anymore, or because I forgot to make an entry in the 
dictionary, or even deleted or changed it. The words in question are listed in 
(\ref{ex:mystcomp}).

\begin{figure}[h]
\ex\labels\label{ex:mystcomp}
	\begin{tabular}[t]{@{\tl\quad} l @{\enspace←\enspace} l @{\smallskip}}
	\xayr{\larger btNimnF}{batangiman}{mosquito}
		& \xayr{\larger bjtNF}{baytang}{blood}
		+ ?
		\\
	\xayr{\larger kirinlNF}{kirinalang}{avenue}
		& \xayr{\larger kirinF}{kirin}{street}
		+ ?
		\\
	\xayr{\larger niNMbkrF}{ningambakar}{telltale}
		& \xayr{\larger niNnF}{ningan}{story}
		+ ?
		\\
	\xayr{\larger rgyesuj}{ragayesuy}{grid}
		& \xayr{\larger rgnF}{ragan}{line}
		+ ?
		\\
	\xayr{\larger terjmino}{teraymino}{melancholic}
		& ?
		+ \xayr{\larger mino}{mino}{happy}
		\\
	\xayr{\larger vetjsno}{vetaysano}{fare}
		& ?
		+ \rayr{\larger ssaanF}{sasān} (earlier \rayr{\larger 
			ssno}{sasano}) `way'
		\\
	\end{tabular}
\xe
\end{figure}

For all of the components represented by a question mark, there is no 
corresponding dictionary entry. At least in \rayr{bjtNimnF}{baytangiman}, the 
*\rayr{ImnF}{*iman} part looks as though it could be a noun due to the 
\rayr{/AnF}{-an} nominalizer suffix\index{suffixes}. *\rayr{terj}{*teray} in 
\rayr{terjmino}{teraymino} might also be an adjective\index{adjectives} supposed to mean `sad' 
(which would make it an adjectival coordinative compound), although the 
dictionary entry for that is \rayr{gidj}{giday}. Even though parts of all 
these words are unclear, they all seem to follow the correct syntactic order\index{word order}, 
judging by those parts that are identifiable. And even in the case of 
\rayr{vetjsno}{vetaysano}, which is missing the first part, it can be 
reasonably assumed that the identifiable part, *\rayr{sno}{*sano}, is the 
modifier, and *\rayr{vetj}{vetay} may have once been intended to mean `money' 
or `fee' or something along these lines.

With the exception of \rayr{niNMbkrF}{ningambakar}, all of the mystery words 
were entered into the dictionary in 2006. Digging through old archives and 
translations, I could determine at least that *\rayr{bkrF}{*bakar} was once 
intended to mean `lie', and *\rayr{terj}{*teray} was indeed intended to 
mean `sad'.

\index{compounds|)}

\subsection{Reduplication}
\index{reduplication|(}

\citet{wiltshiremarantz2000} write that it has been suggested that 
reduplication serves an iconic function, 
\textcquote[561]{wiltshiremarantz2000}{with the repetition of phonological 
material indicating a repetition or intensity in the semantics}, so with 
regards to nouns it mainly serves to indicate plurality of various kinds. 
However, they find that in fact, reduplication serves all kinds of functions, 
also ones without iconic meanings, and mention Agta\index{Agta}, an Austronesian language 
of the Philippines, which uses reduplication to form diminutives 
\citep[6--9]{healey1960}. As we have seen in \autoref{subsec:reduplication} 
above, so does Ayeri, and it is justified in doing so since there is 
real-world evidence for this use of reduplication. A few examples of diminutive\index{diminutive}
reduplication are given in (\ref{ex:dimredup}).

\begin{figure}[h]
\ex\labels\label{ex:dimredup}
	\begin{tabular}[t]{@{\tl\quad} l @{\enspace→\enspace} l @{\smallskip}}
	\xayr{\larger limu}{limu}{shirt}
		& \xayr{\larger limu/limu}{limu-limu}{little shirt}
		\\
	\xayr{\larger nN}{nanga}{house}
		& \xayr{\larger nN/nN}{nanga-nanga}{little house}
		\\
	\xayr{\larger spj}{sapay}{hand}
		& \xayr{\larger spj/spj}{sapay-sapay}{little hand}
		\\
	\xayr{\larger venej}{veney}{dog}
		& \xayr{\larger venej/venej}{veney-veney}{little dog}
		\\
	\end{tabular}
\xe
\end{figure}

Diminutive\index{diminutive} reduplication involves full-stem reduplication in Ayeri. Besides the
productive use of reduplication for diminutive\index{diminutive} marking, there are a number of
diminutive\index{diminutive} formations which have been lexicalized, such as in the examples
given in (\ref{ex:lexdimredup}). There are also at least two documented cases
where the reduplicated root is not a noun, but the reduplication results in a
noun; compare (\ref{ex:nomzredup}).

\begin{figure}[h]
\ex\labels\label{ex:lexdimredup}
	\begin{tabular}[t]{@{\tl\quad} l @{\enspace→\enspace} l @{\smallskip}}
	\xayr{\larger Agu}{agu}{chicken}
		& \xayr{\larger Agu/Agu}{agu-agu}{chick}
		\\
	\xayr{\larger gnF}{gan}{child}
		& \xayr{\larger gnF/gnF}{gan-gan}{grandchild}
		\\
	\xayr{\larger psiNF}{pasing}{tube}
		& \xayr{\larger psiNF/psiNF}{pasing-pasing}{straw}
		\\
	\xayr{\larger poyu}{poyu}{cheek; bacon}
		& \xayr{\larger poyu/poyu}{poyu-poyu}{butt}
		\\
	\end{tabular}
\xe
\end{figure}

\begin{figure}[h]
\ex\labels\label{ex:nomzredup}
	\begin{tabular}[t]{@{\tl\quad} l @{\enspace→\enspace} l @{\smallskip}}
	\xayr{\larger kusNF}{kusang}{double (adj.)}
		& \xayr{\larger kusNF/kusNF}{kusang-kusang}{model}
		\\
	\xayr{\larger veh/}{veh-}{build}
		& \xayr{\larger veh/veh}{veha-veha}{tinkering}
		\\
	\end{tabular}
\xe
\end{figure}

Reduplicated nouns behave like regular nouns with regards to inflection, that 
is, they receive prefixes\index{prefixes} and suffixes\index{suffixes} just like the simplexes from which they 
are derived. This is illustrated in (\ref{ex:diminfl}) for \xayr{venej/venej}
{veney-veney}{little dog}, from \xayr{venej}{veney}{dog}.

\begin{figure}[h]
\ex\label{ex:diminfl}\begingl
	\gla Puco mino \textbf{veney-veneyang}. //
	\glb puk-yo mino veney\til{}veney-ang //
	\glc jump-\TsgN{} happily \Dim{}\til{}dog-\Aarg{} //
	\glft `The little dog is jumping happily.' //
\endgl\xe
\end{figure}

In (\ref{ex:diminfl}), the reduplicated noun \rayr{venej/venej}{veney-veney} is
marked as an agent\index{semantic role!agent} in that the agent\index{case} suffix\index{suffixes} \rayr{/ANF}{-ang} is appended to
the noun as a unit \emph{after} reduplicating the noun stem. In other words,
the following formation in which the root is reduplicated along with its
declension suffix\index{suffixes} is ungrammatical for the purpose of forming a diminutive\index{diminutive}:
*\rayr{\larger veneyNF/veneyNF}{*veneyang-veneyang}. Likewise, the reduplicated
form is not treated in the way of an endocentric compound\index{compounds}, so case\index{case} and
plural\index{number!plural} marking cannot be appended to the first element: *\rayr{\larger veneyNF
venej}{*veneyang veney}.

While ordinary nouns undergo full reduplication to form a diminutive\index{diminutive}, in 
compounds\index{compounds}, only the head is reduplicated, unless the compound\index{compounds} is strongly 
lexicalized or has an idiomatic meaning going beyond that of its components. 
(\ref{ex:compredup}) shows the simple case of a transparent endocentric 
compound\index{compounds}.

\begin{figure}[h]
\ex\label{ex:compredup}\begingl
	\gla Ya @ yomayo mehir-mehirang seygo veno kay pang nanga nana. //
	\glb ya= yoma-yo mehir\til{}mehir-ang seygo veno kay pang nanga-Ø nana //
	\glc \LocT{}= be-\TsgN{} \Dim{}\til{}tree-\Aarg{} apple pretty three 
		back house-\Top{} \Fpl{}.\Gen{} //
	\glft `There are three pretty little apple trees behind our house.' //
\endgl\xe
\end{figure}

In this example, being endearing or otherwise small is treated as a property of
the head, \xayr{mehirF}{mehir}{tree}, not of the whole compound\index{compounds} 
\xayr{mehirFsejgo}{mehirseygo}{apple tree}, or the dependent, 
\xayr{sejgo}{seygo}{apple}---after all, an apple tree which is small is 
rather a small tree with apples on it than a tree with small apples. The 
avoidance of the fully reduplicated form 
\rayr{mehirFsejgo/mehirFsejgo}{mehirseygo-mehirseygo} is probably related to
the notion of economy of expression.

\index{reduplication|)}

\subsection{Nominalization}
\label{subsec:nominalization}
\index{nominalization|(}
\index{derivation|(}

Some accidental ways of deriving nouns have been mentioned above, for instance,
some reduplicated non-nominal roots like \xayr{kusNF}{kusang}{double} or
\xayr{veh/}{veha-}{build} may form nouns. However, Ayeri also has some 
dedicated morphology to derive nouns from other parts of speech. The most 
common and highly productive way to derive a noun, is the suffix\index{suffixes} 
\rayr{/AnF}{-an}. The examples in (\ref{ex:vb-nn}) illustrate some derivations 
from verbs\index{verbs}, and (\ref{ex:adj-nn}) shows derivations from adjectives\index{adjectives} to nouns. 
As \xayr{kuhnF}{kuhan}{oar} shows, the nominalization may have an idiomatic 
meaning.

\begin{figure}[h]
\ex\label{ex:vb-nn}\labels
	\begin{tabular}[t]{@{\tl\quad} l @{\enspace→\enspace} l @{\smallskip}}
	\xayr{\larger blNF/}{balang-}{search (v.)}
		& \xayr{\larger blNnF}{balangan}{search (n.)}
		\\
	\xayr{\larger kuhF/}{kuh-}{row}
		& \xayr{\larger kuhnF}{kuhan}{oar}
		\\
	\xayr{\larger rigF/}{rig-}{draw}
		& \xayr{\larger rignF}{rigan}{drawing}
		\\
	\xayr{\larger vehF/}{veh-}{build}
		& \xayr{\larger vehnF}{vehan}{building}
		\\
	\end{tabular}
\xe
\end{figure}

\begin{figure}[h]
\ex\label{ex:adj-nn}\labels
	\begin{tabular}[t]{@{\tl\quad} l @{\enspace→\enspace} l @{\smallskip}}
	\xayr{\larger Apitu}{apitu}{clean}
		& \xayr{\larger Apitu\_anF}{apituan}{cleanliness}
		\\
	\xayr{\larger gir}{gira}{urgent}
		& \xayr{\larger giraanF}{girān}{hurry}
		\\
	\xayr{\larger pkisF}{pakis}{serious}
		& \xayr{\larger pkisnF}{pakisan}{seriousness}
		\\
	\xayr{\larger vp}{vapa}{skillful}
		& \xayr{\larger vpnF}{vapan}{skill}
		\\
	\end{tabular}
\xe
\end{figure}

Occasionally, it may even happen that a noun is derived from a noun with a 
related but sometimes more basic meaning using the nominalizer 
\rayr{/AnF}{-an}. This process, however, is not productive, so compared to
deverbalization and deadjectivization, examples of this derivation strategy are
few. (\ref{ex:nn-nn}) gives examples of such renominalizations.

\begin{figure}[h]
\ex\label{ex:nn-nn}\labels
	\begin{tabular}[t]{@{\tl\quad} l @{\enspace→\enspace} l @{\smallskip}}
	\xayr{\larger AgYmF}{ajam}{toy}
		& \xayr{\larger AgYmnF}{ajaman}{game}
		\\
	\xayr{\larger kelNF}{kelang}{chain}
		& \xayr{\larger kelNnF}{kelangan}{connection}
		\\
	\xayr{\larger nN}{nanga}{house}
		& \xayr{\larger nNaanF}{nangān}{household}
		\\
	\xayr{\larger tenF}{ten}{life}
		& \xayr{\larger tennF}{tenan}{soul}
		\\
	\end{tabular}
\xe
\end{figure}

There are also some apparent nominalizations in \rayr{/AmF}{-am} and 
\rayr{/ANF}{-ang}, although these are irregular and non-productive; compare 
(\ref{ex:nounamderiv}) and (\ref{ex:nounangderiv}). At least the \rayr{/AmF}
{-am} derivations in (\ref{ex:nounamderiv}) seem to have a connotation of being
tools used for the action they derive from; the \rayr{/ANF}{-ang} derivations
listed seem to derive a more abstract related term. As mentioned, however,
these tendencies are not entirely regular.

\begin{figure}
\ex\labels\label{ex:nounamderiv}
	\begin{tabular}[t]{@{\tl\quad} l @{\enspace→\enspace} l @{\smallskip}}
	\xayr{\larger AgY/}{aja-}{play}
		& \xayr{\larger AgYmF}{ajam}{toy}
		\\
	\xayr{\larger ginF/}{gin-}{drink}
		& \xayr{\larger ginmF}{ginam}{glass}
		\\
	\xayr{\larger mikF/}{mik-}{poison (v.)}
		& \xayr{\larger mikmF}{mikam}{poison (n.), venom}
		\\
	\xayr{\larger nun/}{nuna-}{fly}
		& \xayr{\larger nunmF}{nunam}{feather}
		\\
	\end{tabular}
\xe
\end{figure}

\begin{figure}
\ex\labels\label{ex:nounangderiv}
	\begin{tabular}[t]{@{\tl\quad} l @{\enspace→\enspace} l @{\smallskip}}
	\xayr{\larger bjh/}{bayha-}{rule}
		& \xayr{\larger bjhNF}{bayhang}{government}
		\\
	\xayr{\larger hp}{hapa}{remaining}
		& \xayr{\larger hpNF}{hapang}{remainder}
		\\
	\xayr{\larger kd/}{kada-}{collect}
		& \xayr{\larger kdNF}{kadang}{committee; alliance}
		\\
	\xayr{\larger mim}{mima}{possible}
		& \xayr{\larger mimNF}{mimang}{access}
		\\
	\end{tabular}
\xe
\end{figure}

Agentive\index{semantic role!agent} nouns can be formed from regular nouns with the suffix\index{suffixes} 
\rayr{/my}{-maya}, compare the examples in (\ref{ex:mayaregular}). An 
epenthetic /a/ may be introduced to break up consonant clusters that would
otherwise be either difficult to pronounce or violating phonotactics. When the
stem of the word to which the agentive\index{semantic role!agent} suffix\index{suffixes} is attached ends in a consonant
or /Ca/, it is also often found fused with the root\index{allomorphy}, sometimes with the first
/a/ of \fw{-Caya} lengthened\index{morphophonology}, compare (\ref{ex:mayairregular}). Specifically
feminine\index{gender} agentive\index{semantic role!agent} nouns can be derived with the related suffix\index{suffixes}
\rayr{/vy}{-vaya}; two examples of this are given in (\ref{ex:vaya}).

\begin{figure}[h]
\ex\label{ex:mayaregular}\labels
	\begin{tabular}[t]{@{\tl\quad} l @{\enspace→\enspace} l @{\smallskip}}
	\xayr{\larger AnFlF/}{anl-}{bring}
		& \xayr{\larger AnFlmy}{anlamaya}{waiter}
		\\
	\xayr{\larger hor}{hora}{sin}
		& \xayr{\larger hormy}{horamaya}{sinner}
		\\
	\xayr{\larger nsY/}{nasy-}{follow}
		& \xayr{\larger nsYmy}{nasyamaya}{follower}
		\\
	\xayr{\larger teb/}{teba-}{bake}
		& \xayr{\larger tebmy}{tebamaya}{baker}
		\\
	\end{tabular}
\xe
\end{figure}

\begin{figure}[h]
\ex\label{ex:mayairregular}\labels
	\begin{tabular}[t]{@{\tl\quad} l @{\enspace→\enspace} l @{\smallskip}}
	\xayr{\larger As/}{asa-}{travel}
		& \xayr{\larger Asaay}{asāya}{traveler}
		\\
	\xayr{\larger IbutF/}{ibut-}{trade}
		& \xayr{\larger Ibuty}{ibutaya}{trader, merchant}
		\\
	\xayr{\larger lMtF/}{lant-}{lead}
		& \xayr{\larger lMty}{lantaya}{leader; driver}
		\\
	\xayr{\larger tNF/}{tang-}{listen}
		& \xayr{\larger tNy}{tangaya}{listener}
		\\
	\end{tabular}
\xe
\end{figure}

\begin{figure}[h]
\ex\label{ex:vaya}\labels
	\begin{tabular}[t]{@{\tl\quad} l @{\enspace→\enspace} l @{\smallskip}}
	\xayr{\larger gnF}{gan}{child}
		& \xayr{\larger gnFvy}{ganvaya}{governess}
		\\
	\xayr{\larger lnY}{lanya}{king}
		& \xayr{\larger lnFvy}{lanvaya}{queen}
		\\
	\end{tabular}
\xe
\end{figure}

Besides the agentive\index{semantic role!agent} suffixes\index{suffixes}, there is also a derivational suffix\index{suffixes} for makers of
things, \rayr{/Ati}{-ati} (contracting\index{morphophonology} to \rayr{/AtYF/}{-ac-} before a vowel),
though this is not too productive, and sometimes irregular, as
\xayr{sirFtNti}{sirtangati}{youth} in (\ref{ex:makerderiv}) shows. Moreover,
there are instances of nominalization where a tool of sorts is derived with a
suffix\index{suffixes} \rayr{/(E)rYnF}{-(e)ryan}, which is related to the instrumental\index{case!instrumental}\index{semantic role!instrument} suffix\index{suffixes}
\rayr{/Eri}{-eri} in combination with the nominalizer \rayr{/AnF}{-an}; compare
(\ref{ex:toolderiv}).

\begin{figure}[h]
\ex\labels\label{ex:makerderiv}
	\begin{tabular}[t]{@{\tl\quad} l @{\enspace→\enspace} l @{\smallskip}}
	\xayr{\larger giMdi}{gindi}{poem}
		& \xayr{\larger giMdti}{gindati}{poet}
		\\
	\xayr{\larger sirFtNF}{sirtang}{young}
		& \xayr{\larger sirFtNti}{sirtangati}{youth}
		\\
	\xayr{\larger thnF/}{tahan-}{write}
		& \xayr{\larger thnti}{tahanati}{scribe}
		\\
	\xayr{\larger vehimF}{vehim}{piece of clothing}
		& \xayr{\larger vehimti}{vehimati}{tailor}
		\\
	\end{tabular}
\xe
\end{figure}

\begin{figure}[h]
\ex\labels\label{ex:toolderiv}
	\begin{tabular}[t]{@{\tl\quad} l @{\enspace→\enspace} l @{\smallskip}}
	\xayr{\larger gurF/}{gur-}{turn}
		& \xayr{\larger gurFynF}{guryan}{coil, cylinder}
		\\
	\xayr{\larger misF/}{mis-}{behave}
		& \xayr{\larger miserYnF}{miseryan}{method, strategy}
		\\
	\xayr{\larger npF/}{nap-}{burn}
		& \xayr{\larger nperYnF}{naperyan}{tinder}
		\\
	\xayr{\larger pr/}{pra-}{glitter, gleam}
		& \xayr{\larger pFrrYnF}{praryan}{spark}
		\\
	\end{tabular}
\xe
\end{figure}

\index{gerund|(}

While \rayr{/AnF}{-an} derives nouns from verbs\index{verbs} to produce nouns that act as
such in every way, it may sometimes be preferable to refer to the action itself
by a noun, compare (\ref{ex:enggerund}) for an example from English\index{English}. In
(\ref{ex:devnouneng}), \fw{building} is simply a noun derived from the verb
\fw{build}. It acts as a noun in every way, for example, it can serve as a 
subject and object, it can be pluralized, it can take determiners, and can be 
modified by adjectives.

% \begin{figure}[h]
\pex\label{ex:enggerund}%
	English:
	\a\label{ex:devnouneng} \fw{Manhattan is famous for its tall
		\textbf{buildings}.}
	\a\label{ex:gerundeng} \fw{\textbf{Building} a house is an expensive
		endeavor.}
\xe
% \end{figure}

The form of \fw{building} in (\ref{ex:gerundeng}), however, is a gerund, and as
such underlies the restriction that it cannot be pluralized
\citep[35]{payne1997}. As we have seen at the beginning of this section on
nominalization, Ayeri can derive \xayr{vehnF}{vehan}{building, construction}
from the verb \xayr{vehF/}{veh-}{build}, which acts like every other common
noun\index{nouns!common}, much like in the English\index{English} example in (\ref{ex:devnouneng}).

\begin{figure}
\pex\label{ex:vbnomz}
\a\label{ex:nomz-sbj-adj}\begingl
	\gla Lesāra sirimang \textbf{vehānreng} \textbf{tado}. //
	\glb lesa-ara sirimang vehān-reng tado //
	\glc collapse-\TsgI{} about.to building-\AargI{} old//
	\glft `The old building is about to collapse.' //
\endgl

\a\label{ex:nomz-obj-det}\begingl
	\gla Le @ vacyang \textbf{eda-vehān}. //
	\glb le= vac=yang eda=vehān-Ø //
	\glc \PatTI{}= like-\Fsg{}.\Aarg{} this=building-\Top{} //
	\glft `This building, I like it.' //
\endgl

\a\label{ex:nomz-pl-poss}\begingl
	\gla Ang @ latayo bayhang \textbf{vehānyeley} \textbf{yona}. //
	\glb ang= lata-yo bayhang-Ø vehān-ye-ley yona //
	\glc \AgtT{}= sell-\TsgN{} government-\Top{} 
		building-\Pl{}-\PargI{} \TsgN{}.\Gen{} //
	\glft `The government is selling its buildings.' //
\endgl

\a\label{ex:nomz-qty}\begingl
	\gla Le @ ming @ kuysāran \textbf{vehān-kay} dirasyam ran. //
	\glb le= ming= kuysa-aran vehān-Ø=kay diras-yam ran //
	\glc \PatTI{}= can= compare-\TplI{} building-\Top=few 
		splendor-\Dat{} \TsgI{}.\Gen{} //
	\glft `Few buildings can compare to its splendor.' //
\endgl
\xe
\end{figure}

The examples in (\ref{ex:vbnomz}) condense several properties into one for
illustration. For instance, (\ref{ex:nomz-sbj-adj}) shows that \rayr{vehaanF}
{vehān} can serve as the subject of a clause, and that it can as well
be modified by an adjective\index{adjectives}---the choice of adjectives\index{adjectives} is not subject to any
distributional restrictions other than those imposed by the semantic frame of
\textsc{house}. In the next example, (\ref{ex:nomz-obj-det}),
\rayr{vehaanF}{vehān} serves as the object of the clause and is being
determined by the demonstrative prefix\index{prefixes} \xayr{Ed/}{eda-}{this}. The third
example, (\ref{ex:nomz-pl-poss}), shows \rayr{vehaanF}{vehān} both pluralized\index{number!plural}
and modified by a possessive pronoun\index{pronouns!possessive}, \xayr{yon}{yona}{of it}. And finally, in
(\ref{ex:nomz-qty}) we see \rayr{vehaanF}{vehān} quantified by the enclitic
\xayr{/kj}{-kay}{few}.

\begin{figure}[h]
\ex\label{ex:kafkagerund}\begingl
	\gla … nay ang @ pətangongva ankyu \textbf{haruyamanas} nanang … //
		% megayena yana kunangya vana. //
	\glb … nay ang= pə-tang-ong=va.Ø ankyu haru-yam-an-as nanang … //
		% mega-ye-na yana kunang-ya vana //
	\glc … and \AgtT{}= \NFut{}-hear-\Irr{}=\Second{}.\Top{} truly 
		beat-\Ptcp{}-\Nmlz{}-\Parg{} great … // % fist-\Pl{}-\Gen{} 
		% \TsgM{}.\Gen{} door-\Loc{} \Second{}.\Gen{} //
	\glft `… and you would indeed hear the magnificent beating …' //
		% at your door very soon.' //
\endgl\xe
\end{figure}

Similar to the English\index{English} example in (\ref{ex:gerundeng}), Ayeri can also derive
nouns from the participle\index{participle} of a verb\index{verbs} describing the action as such---a gerund.
(\ref{ex:kafkagerund}) again draws on the Ayeri translation of Kafka's short
story \enquote{Eine kaiserliche Botschaft} \citep[2, 14]{becker:kafka:imperial}
for an example. The annotations to this translation contain a comment on the
grammatical rules which operate in this passage, more specifically also on the
gerund derivation \xayr{hruymnF}{haruyaman}{beating}:

\blockcquote[14--15]{becker:kafka:imperial}{Furthermore, I wrote
\fw{haruyaman} `beating' instead of \fw{haruan} `beat(ing)' because I wanted to
emphasize the process of beating as an incomplete action. This is possible here
because the word is not topicalized and neither is it marked as a dative, which
would also require \fw{haruyamanyam} `beat-\Ptcp{}-\Nmlz{}-\Dat{}' to become
\fw{haruanyam} `beat-\Nmlz{}-\Dat{}' (the participle\index{participle} marker \fw{-yam} is 
derived from the dative case ending \fw{-yam}).}

We can read from this description that the participle\index{participle} marker in Ayeri has
possibly been grammaticalized from the dative case\index{case!dative} marker, or that it is at
least synchronically homonymous. In order for case\index{case} marking to operate, this
formation has to be nominalized, which is done in the usual way by appending
\rayr{/AnF}{-an}, thus yielding the suffix\index{suffixes} cluster \rayr{/ymnF}{-yaman} for the
derivation of verbs as gerunds. If the gerund is marked for dative case\index{case!dative}, the
suffix\index{suffixes} cluster *\rayr{/ymnFymF}{*-yamanyam} undergoes haplology to a
simple nominalized form with the suffix\index{suffixes} cluster \rayr{/AnFymF}{-anyam}. See
(\ref{ex:datnmlz}) for an example.

\begin{figure}[h]
\ex\label{ex:datnmlz}%
\begingl
	\gla haru- {} haruyam {} haruyaman {} *haruyamanyam {} haruanyam //
	\glb haru- → haru-yam → haru-yam-an → haru-yam-an-yam → 
		haru-an-yam //
	\glc beat {} beat-\Ptcp{} {} beat-\Ptcp{}-\Nmlz{} {} 
		beat-\Ptcp{}-\Nmlz{}-\Dat{} {} beat-\Nmlz{}-\Dat{} //
\endgl\xe
\end{figure}

The comment on the translation also makes a little note on the gerund being
possible because the word is not topicalized. This is based on an old rule that
gerunds cannot be topicalized unless nominalized first, however, usage has
since changed so that earlier, \rayr{hruymF}{haruyam} would have constituted
the gerund form, while even by the time of translating the short story, it had
changed to \rayr{hruymnF}{haruyaman}. This is encountered in
(\ref{ex:exuperygerund}), an example from the partial translation of
Saint-Exupéry's story \enquote{Le petit prince} \citep[3, 13]
{benung:petitprince}. A more literal translation of this sentence would be `The
distinguishing of China and Arizona, I knew it at first sight', so the whole
passage \rayr{pluNFymnF — n byokivo}{palungyaman … na Bayokivo} forms the topic
of the sentence here, headed by the gerund
\xayr{pluNFymnF}{palungyaman}{distinguishing}. According to the old rule
obliquely quoted in the comment to the passage in (\ref{ex:kafkagerund}), this
should not be possible. As mentioned before, though, use has changed.

\begin{figure}[h]
\ex\label{ex:exuperygerund}\begingl
	\gla Sa @ koronyang \textbf{palungyaman} na @ Baysānterpeng nay na @
		Bayokivo menaneri nivānyena. //
	\glb sa= koron=yang palung-yam-an-Ø na= Baysānterpeng nay na= 
		Bayokivo menan-eri nivān-ye-na //
	\glc \PatT{}= knew=\Fsg{}.\Aarg{} distinguish-\Ptcp{}-\Nmlz{}-\Top{} 
		\Gen{}= Realm.Middle and \Gen{}= Spring.Little first-\Ins{} 
		glimpse-\Pl{}-\Gen{} //
	\glft `I knew the difference between China and Arizona at first sight.' //
\endgl\xe
\end{figure}

A rule we can gather from (\ref{ex:exuperygerund}) is that gerunds are treated
as animate\index{animacy} nouns. Since they are impersonal, they trigger neuter\index{gender} agreement\index{agreement} on
verbs\index{verbs}. They can also be the objects of sentences. The passage in
(\ref{ex:kafkagerund}) furthermore illustrates that gerunds can be modified by
adjectives\index{adjectives}. The example in (\ref{ex:scimethgerund}) shows a gerund used as an
agent-subject as well \citep{benung:scientificmethod}.

\begin{figure}[h]
\ex\label{ex:scimethgerund}\begingl
	\gla \textbf{Dilayamanang} kalamena bahalanas ayonena … //
	\glb dila-yam-an-ang kalam-ena bahalan-as ayon-ena … //
	\glc find.out-\Ptcp{}-\Nmlz{}-\Aarg{} truth-\Gen{} goal-\Parg{} 
		man-\Gen{} … //
	\glft `(If) finding out the truth is the goal of the man …' //
\endgl\xe
\end{figure}

All the passages on gerunds quoted before show that gerunds in Ayeri do
not behave like transitive verbs\index{verbs!transitive} as in English\index{English}. Thus, what would be the object
of the former verb appears in the genitive case\index{case!genitive} in Ayeri. As in English\index{English},
however, gerunds in Ayeri cannot be pluralized\index{number!plural}; compare (\ref{ex:grndplur}). It
is possible, however, to quantify gerunds insofar as the quantifier\index{quantifiers} does not
imply countable quantities of the action. Moreover, it is possible for gerunds
to be modified by possessors\index{semantic role!possessor}. The two sentences in (\ref{ex:grndmod}) exemplify
this use.

\begin{figure}[h]
\ex\label{ex:grndplur}\ljudge*\begingl
	\gla Noyo \textbf{vehayamanjang} nangayena. //
	\glb noyo veha-yam-an-ye-ang nanga-ye-na //
	\glc expensive build-\Ptcp{}-\Nmlz{}-\Pl{}-\Aarg{} house-\Pl{}-\Gen{} //
	\glft `*The buildings of houses are expensive.' //
\endgl\xe
\end{figure}

\begin{figure}[h]
\pex\label{ex:grndmod}
\a\begingl
	\gla Ang @ lugayan \textbf{delacamanas-ikan} kayanya pang. //
	\glb ang= luga=yan.Ø delak-yam-an-as=ikan kayan-ya pang //
	\glc \AgtT{}= go.through=\TplM{}.\Top{} 
		suffer-\Ptcp{}-\Nmlz{}-\Parg{}=much war-\Loc{} after //
	\glft `They went through a lot of suffering after the war.' //
\endgl

\a\begingl
	\gla Krico \textbf{malyyamanang} muya \textbf{tan}. //
	\glb krit-yo maly-yam-an-ang muya tan //
	\glc annoy-\TsgN{} sing-\Ptcp{}-\Nmlz{}-\Aarg{} wrong \TplM{}.\Gen{} //
	\glft `Their off singing is annoying.' //
\endgl
\xe
\end{figure}

\index{gerund|)}
\index{derivation|)}
\index{nominalization|)}
\index{nouns|)}

\section{Pronouns}
\index{pronouns|(}

Ayeri possesses different kinds of pronouns in the sense that there is a closed
class of words which contains anaphora of various types---personal pronouns,
demonstrative pronouns, interrogative pronouns, relative pronouns, as well as
reflexive and reciprocal expressions. Each class of pronouns will be discussed
in the following.

\subsection{Personal pronouns}
\label{subsec:perspro}
\index{pronouns!personal|(}
\index{person|(}

\begin{table}[tp]\centering
\caption{Personal pronouns}

\begin{tabu} to \linewidth{l X[c] X[c] X[c] X[c] X[c] X[c] X[c] X[c]}
\tableheaderfont\toprule
Person
	& \Top{}
	& \Aarg{}
	& \Parg{}
	& \Dat{}
	& \Gen{}
	& \Loc{}
	& \Caus{}
	& \Ins{}
	\\
\toprule

\Fsg{}
	& ay	% \Top{}
	& yang	% \Aarg{}
	& yas	% \Parg{}
	& yām	% \Dat{}
	& nā	% \Gen{}
	& yā	% \Loc{}
	& sā	% \Caus{}
	& rī	% \Ins{}
	\\
	
\midrule

\Ssg{}
	& va	% \Top{}
	& vāng	% \Aarg{}
	& vās	% \Parg{}
	& vayam	% \Dat{}
	& vana	% \Gen{}
	& vaya	% \Loc{}
	& vasa	% \Caus{}
	& vari	% \Ins{}
	\\

\midrule

\TsgM{}
	& ya	% \Top{}
	& yāng	% \Aarg{}
	& yās	% \Parg{}
	& yayam	% \Dat{}
	& yana	% \Gen{}
	& yāy	% \Loc{}
	& yasa	% \Caus{}
	& yari	% \Ins{}
	\\

\TsgF{}
	& ye	% \Top{}
	& yeng	% \Aarg{}
	& yes	% \Parg{}
	& yeyam	% \Dat{}
	& yena	% \Gen{}
	& yea	% \Loc{}
	& yesa	% \Caus{}
	& yeri	% \Ins{}
	\\

\TsgN{}
	& yo	% \Top{}
	& yong	% \Aarg{}
	& yos	% \Parg{}
	& yoyam	% \Dat{}
	& yona	% \Gen{}
	& yoa	% \Loc{}
	& yosa	% \Caus{}
	& yori	% \Ins{}
	\\

\TsgI{}
	& ra	% \Top{}
	& reng	% \Aarg{}
	& rey	% \Parg{}
	& rayam	% \Dat{}
	& ran	% \Gen{}
	& raya	% \Loc{}
	& rasa	% \Caus{}
	& rari	% \Ins{}
	\\

\midrule

\Fpl{}
	& ayn	% \Top{}
	& nang	% \Aarg{}
	& nas	% \Parg{}
	& nyam	% \Dat{}
	& nana	% \Gen{}
	& nyā	% \Loc{}
	& nisa	% \Caus{}
	& ni	% \Ins{}
	\\
	
\midrule

\Spl{}
	& va	% \Top{}
	& vāng	% \Aarg{}
	& vās	% \Parg{}
	& vayam	% \Dat{}
	& vana	% \Gen{}
	& vaya	% \Loc{}
	& vasa	% \Caus{}
	& vari	% \Ins{}
	\\

\midrule

\TplM{}
	& yan	% \Top{}
	& tang	% \Aarg{}
	& tas	% \Parg{}
	& cam	% \Dat{}
	& tan	% \Gen{}
	& ca	% \Loc{}
	& tis	% \Caus{}
	& ti	% \Ins{}
	\\

\TplF{}
	& yen	% \Top{}
	& teng	% \Aarg{}
	& tes	% \Parg{}
	& teyam	% \Dat{}
	& ten	% \Gen{}
	& teya	% \Loc{}
	& tēs	% \Caus{}
	& teri	% \Ins{}
	\\

\TplN{}
	& yon	% \Top{}
	& tong	% \Aarg{}
	& tos	% \Parg{}
	& toyam	% \Dat{}
	& ton	% \Gen{}
	& toya	% \Loc{}
	& tōs	% \Caus{}
	& tori	% \Ins{}
	\\

\TplI{}
	& ran	% \Top{}
	& teng	% \Aarg{}
	& tey	% \Parg{}
	& racam	% \Dat{}
	& ten	% \Gen{}
	& raca	% \Loc{}
	& ratas	% \Caus{}
	& ray	% \Ins{}
	\\

\bottomrule
\end{tabu}
\label{tab:perspro}
\end{table}

As \autoref{tab:perspro} shows, Ayeri possesses quite a large number of
personal pronouns with (maybe unnaturally) little syncretism between the
different paradigm slots overall (the second person is a notable exception);
there are also no gaps in the paradigm. Ayeri's personal pronouns reflect the
grammatical features also found in nouns, that is, number\index{number}, gender\index{gender}, and case\index{case};
person is added to this. The individual forms range from completely fused
to fully transparent even within the same case\index{case} paradigm, for instance,
\xayr{yaamF}{yām}{(to/for) me} (\Fsg{}.\Dat{}) on the one hand, and 
\xayr{yymF}{yayam}{(to/for) him} (transparently \TsgM{}-\Dat{}) on the other. 
Originally, all pronouns have been regular formations based on the respective
unmarked pronominal element listed in the \Top{} column of
\autoref{tab:perspro} declined by adding a case\index{case} suffix\index{suffixes} (see
\autoref{subsec:case}). Use has caused many of these formations to contract and
erode as grammaticalization\index{grammaticalization} progressed, for instance the first person agent and
third person animate masculine plural pronouns; compare (\ref{ex:prongen}).

\begin{figure}[h]
\pex\label{ex:prongen}
\a\begingl
	\gla ayang → yang //
	\glb ay-ang {} yang //
	\glc \makebox[\widthof{\Tsg{}-\M{}-\Pl{}-\Gen{}}][l]{\Fsg{}-\Aarg{}} {} 
		\Fsg{}.\Aarg{} //
\endgl

\a\begingl
	\gla iyatena → tan //
	\glb iy-a-t-ena {} tan //
	\glc \Tsg{}-\M{}-\Pl{}-\Gen{} {} \TsgM{}.\Gen{}\footnotemark //
\endgl
\xe
\end{figure}

\footnotetext{Strictly speaking, this could as well be glossed as \fw{t<a>n} 
(\Tsg{}.\Gen{}<\M{}>). I chose to gloss the pronoun in the above way, however, 
in order to not overly complicate things.}

The plural\index{number!plural} series used to be derived by adding \rayr{/nF}{-n} or, in the third 
person, \rayr{/tF/}{\mbox{-t-}} to the pronoun stem, which can still be easily 
observed in the unmarked pronouns as well as in the alternation between 
\rayr{yF/}{y-} and \rayr{tF/}{t-} in the third person pronouns. The same goes 
for the gender-marking\index{gender} thematic vowel in the animate third person pronouns, 
which has been retained as a distinctive feature even in the non-core pronouns 
despite sometimes heavy modifications. A further interesting property of Ayeri 
is that synchronically, singular\index{number} and plural are distinguished, except for the 
second person, where the forms are the same, basically like in English\index{English}. 
\citet{lehmann2015} explains, however, that this is not an unusual route for 
languages to take:

\blockcquote[42]{lehmann2015}{New pronouns, especially for the second person 
singular, are often obtained by shifting pronouns around in the paradigm,
especially by substituting marked forms for unmarked ones. This explains, for
instance, the use of [...]\ English \fw{you} for the second person singular
[...]}

The second person singular subject pronoun of English\index{English} used to be \fw{thou}, 
cognate to German\index{German} \fw{du}, which can still be found in Shakespeare, for 
instance. Something along the lines of English\index{English} \fw{you} as a second 
person plural pronoun replacing second person singular \fw{thou} by way of a 
deferential singular use of a plural pronoun \citep[you, pron., adj., and 
n.]{oed} may have happened in Ayeri as well.

\begin{figure}[h]
\pex\label{ex:perspro}
\a\label{ex:pronfull}\begingl
	\gla Ang @ harya {} @ Paradan tandās kaleri. //
	\glb ang= har-ya Ø= Paradan tanda-as kal-eri //
	\glc \AgtT{}= beat-\TsgM{} \Top{}= Paradan fly-\Parg{} rag-\Ins{} //
	\glft `Paradan, he beats the fly with a rag.' //
\endgl

\a\label{ex:pronagt}\begingl
	\gla Sa @ haryāng tanda kaleri. //
	\glb sa= har=yāng tanda-Ø kal-eri //
	\glc \PatT{}= beat=\TsgM{}.\Aarg{} fly-\Top{} rag-\Ins{} //
	\glft `The fly, he beats it with a rag.' //
\endgl

\a\label{ex:pronpat}\begingl
	\gla Ang @ harya {} @ Paradan yos kaleri. //
	\glb ang= har-ya Ø= Paradan yos kal-eri //
	\glc \AgtT{}= beat-\TsgM{} \Top{}= Paradan \TsgN{}.\Parg{} rag-\Ins{} //
	\glft `Paradan, he beats it with a rag.' //
\endgl

\a\label{ex:pronins}\begingl
	\gla Ang @ harya {} @ Paradan tandās rari. //
	\glb ang= har-ya Ø= Paradan tanda-as rari //
	\glc \AgtT{}= beat-\TsgM{} \Top{}= Paradan fly-\Parg{} \TsgI{}.\Ins{} //
	\glft `Paradan, he beats the fly with it.' //
\endgl
\xe
\end{figure}

The personal pronouns are used in just the same way as their full-NP\index{phrase types!noun phrase}
counterparts would be, also in the non-core cases\index{case}. (\ref{ex:pronfull}) shows a
sentence with full subject and object NPs\index{phrase types!noun phrase}; (\ref{ex:pronagt}) shows a variation
of the sentence with the agent, \rayr{prdnF}{Paradan}, replaced by the third
person singular masculine agent pronoun \xayr{yaaNF}{yāng}{he}. In
(\ref{ex:pronpat}), then, the patient, \xayr{tMdaasF}{tandās}{fly}, is replaced
with the third person singular neuter patient pronoun \rayr{yosF}{yos}. In
(\ref{ex:pronins}), lastly, the instrument\index{semantic role!instrument}, \xayr{kleri}{kaleri}{with a rag} is
replaced with the third person singular inanimate instrumental pronoun
\xayr{rri}{rari}{with it}. Furthermore, complex NPs\index{phrase types!noun phrase} are in complementary
distribution\index{complementary distribution} with pronouns\index{phrase types!determiner phrase}, since pronouns are anaphora for NPs\index{phrase types!noun phrase}. Thus also, an
NP\index{phrase types!noun phrase} which contains an adjective is wholly replaced by a personal-pronoun DP 
(determiner phrase), as in (\ref{ex:procompldist}).

\begin{figure}[h]
\pex\label{ex:procompldist}
\a\begingl
	\gla Ang @ ninye vehimley veno. //
	\glb ang= nin=ye.Ø vehim-ley veno //
	\glc \AgtT{}= wear=\TsgF{}.\Top{} dress-\PargI{} beautiful //
	\glft `She wears a beautiful dress.' //
\endgl

\a\ljudge* \begingl
	\gla Ang @ ninye adaley veno. //
	\glb ang= nin=ye.Ø ada-ley veno //
	\glc \AgtT{}= wear=\TsgF{}.\Top{} that-\PargI{} beautiful //
	\glft `*She wears a beautiful it.' //
\endgl

\a\begingl
	\gla Ang @ ninye adaley. //
	\glb ang= nin=ye.Ø ada-ley //
	\glc \AgtT{}= wear=\TsgF{}.\Top{} that-\PargI{} //
	\glft `She wears it.' //
\endgl
\xe
\end{figure}

Comparing the example sentences in (\ref{ex:perspro}) with the \Top{} column
in \autoref{tab:perspro} an important property of personal pronouns becomes 
apparent. That is, the `unmarked' (or rather, zero-marked) pronoun forms are 
also the ones showing as verb\index{verbs} agreement\index{agreement}. An important difference in this 
respect, however, is that the third person singular inanimate\index{animacy} verb\index{verbs} agreement\index{agreement} 
marker is not \rayr{/r}{-ra}, but \rayr{/Ar}{-ara}. The following two examples 
illustrate the parallel more clearly---observe the person marking on the verb\index{verbs} 
in (\ref{ex:verbinfl1}) and the corresponding object pronouns in 
(\ref{ex:verbinfl2}).

\begin{figure}
\pex\label{ex:verbinfl1}
\a\begingl
	\gla Sa @ man\textbf{ya} ang @ Ajān {} @ Pila. //
	\glb sa= man-ya ang= ​Ajān Ø= ​Pila //
	\glc \PatT{}= greet-\TsgM{} \Aarg{}= ​Ajān \Top{}= ​Pila //
	\glft `Pila, Ajān greets her.' //
\endgl

\a\begingl
	\gla Sa @ man\textbf{ye} ang @ Pila {} @ Ajān. //
	\glb sa= man-ye ang= Pila Ø= ​Ajān //
	\glc \PatT{}= greet-\TsgF{} \Aarg{}= Pila \Top{}= ​Ajān //
	\glft `Ajān, she greets him.' //
\endgl

\xe
\end{figure}

\begin{figure}
\pex\label{ex:verbinfl2}
\a\begingl
	\gla Sa @ manye ang @ Pila \textbf{ya}. //
	\glb sa= man-ye ang= Pila ya.Ø //
	\glc \PatT{}= greet-\TsgF{} \Aarg{}= Pila \TsgM{}.\Top{} //
	\glft `Pila, she greets him.' //
\endgl

\a\begingl
	\gla Sa @ manya ang @ Ajān \textbf{ye}. //
	\glb sa= man-ya ang= ​Ajān ye.Ø //
	\glc \PatT{}= greet-\TsgM{} \Aarg{}= ​Ajān \TsgF{}.\Top{} //
	\glft `Ajān, he greets her.' //
\endgl
\xe
\end{figure}

Another important property of both pronouns and verbs\index{verbs} is that agent\index{case!agent} pronouns
(and patient\index{case!patient} pronouns under certain circumstances) replace person agreement\index{agreement} by
cliticizing\index{clitics} to the verb\index{verbs} stem. Since person agreement\index{agreement} morphology is a domain of
verbs\index{verbs}, it will be dealt with in more detail in the section on verbs\index{verbs} proper. 
Example (\ref{ex:agtproclt_1}) again has full subject and object NPs\index{phrase types!noun phrase}; the verb\index{verbs}
displays \rayr{/y}{-ya} as the agreement\index{agreement} suffix\index{suffixes} for the masculine agent\index{case!agent} NP\index{phrase types!noun phrase}.
Example (\ref{ex:agtproclt_2}), then, replaces the agent\index{case!agent} NP\index{phrase types!noun phrase} with a pronoun.
This is not expressed by a free form like \fw{he}, though, but as a pronominal
clitic, \xayr{/yaaNF}{-yāng}{he}.

\begin{figure}[h]
\pex\label{ex:agtproclt}
\a\label{ex:agtproclt_1}%
\begingl
	\gla Sa @ man\textbf{ya} \textbf{ang} @ \textbf{Ajān} {} @ Pila. //
	\glb sa= man-ya ang= ​Ajān Ø= ​Pila //
	\glc \PatT{}= greet-\TsgM{} \Aarg{}= ​Ajān \Top{}= ​Pila //
	\glft `Pila, Ajān greets her.' //
\endgl

\a\label{ex:agtproclt_2}%
\begingl
	\gla Sa @ man\textbf{yāng} {} @ Pila. //
	\glb sa= man=yāng Ø= ​Pila //
	\glc \PatT{}= greet=\TsgM{}.\Aarg{} \Top{}= Pila //
	\glft `Pila, he greets her.' //
\endgl
\xe
\end{figure}

\index{person|)}
\index{pronouns!personal|)}
\index{pronouns!possessive|(}

\phantomsection\label{phsec:possadj}
Possessive pronouns are special compared to regular personal pronouns in that,
like adjectives\index{adjectives}, they need \rayr{d/}{da-} as a supporting particle in order to
stand alone. The main use for the genitive\index{case!genitive} pronouns in \autoref{tab:perspro} is
to show possession. This means that unlike personal pronouns\index{pronouns!personal}, they are by
themselves not in complementary distribution\index{complementary distribution} with nominal NPs\index{phrase types!noun phrase}, compare
(\ref{ex:procompldist}). Instead, they may be used as modifiers, as
(\ref{ex:adjgen}) shows.

\begin{figure}[h]
\ex\label{ex:adjgen}%
\begingl
	\gla nangaya ledo nā //
	\glb nanga-ya ledo nā //
	\glc house-\Loc{} blue \Fsg{}.\Gen{} //
	\glft `in my blue house' //
\endgl
\xe
\end{figure}

However, possessives do not share typical morphological properties of
adjectives\index{adjectives}, namely, they cannot be compared (*\xayr{naa/ENF}{*nā-eng}{*myer},
*\xayr{naa/vaa}{*nā-vā}{*myest}). Fronting\index{word order} them in predicative\index{grammatical function!predicative complement} statements like
the one in (\ref{ex:genpred}) is possible even without the supporting particle,
though. Alternatively, a phrasal construction with
\xayr{vilFyNF/}{vilyang-}{belong}, as indicated in (\ref{ex:genphrase}), may be
used.

\begin{figure}[h]
\pex\label{ex:genpred}
\a\label{ex:genpred_1}\begingl
	\gla Ada-nangāng da-nā. //
	\glb ada=nanga-ang da-nā //
	\glc that=house-\Aarg{} one=\Fsg{}.\Gen{} //
	\glft `That house is mine.' //
\endgl

\a\label{ex:genpred_2}\begingl
	\gla Nā ada-nangāng. //
	\glb nā ada=nanga-ang //
	\glc \Fsg{}.\Gen{} that=house-\Aarg{} //
	\glft `Mine is that house.' //
\endgl
\xe
\end{figure}

\begin{figure}[h]
\ex\label{ex:genphrase}%
\begingl
	\gla Ang @ vilyangyo ada-nanga yas. //
	\glb ang= vilyang-yo ada=nanga-Ø yas //
	\glc \AgtT{}= belong-\TsgN{} that=house-\Top{} \Fsg{}.\Parg{} //
	\glft `That house belongs to me.' //
\endgl
\xe
\end{figure}

\index{pronouns!possessive|)}

\subsection{Demonstrative pronouns}
\label{subsec:dempro}
\index{pronouns!demonstrative|(}

\begin{table}[tp]\centering
\caption{Demonstrative pronouns}

\begin{tabu} to \linewidth{l X[c] X[c] X[c]}
\tableheaderfont\toprule

Case
	& Proximal
	& Distal
	& Indefinite
	\\
\toprule

\Top{}
	& edanya
	& adanya
	& danya
	\\
	
\midrule
	
\Aarg{}
	& edanyāng
	& adanyāng
	& \emph{danyāng}
	\\

\Aarg{}.\Inan{}
	& edareng, \emph{edanyareng}
	& adareng, adanyareng
	& \emph{danyareng}
	\\

\Parg{}
	& edanyās
	& adanyās
	& danyās
	\\

\Parg{}.\Inan{}
	& edaley
	& \emph{adaley}
	& danyaley
	\\

\Dat{}
	& \emph{edayam}
	& adayam
	& \emph{danyayam}
	\\

\midrule

\Gen{}
	& edanyana
	& adanyana
	& danyana
	\\
	
\Loc{}
	& \emph{edanyaya}
	& adanyaya
	& \emph{danyaya}
	\\
	
\Caus{}
	& \emph{edanyasa}
	& \emph{adanyasa}
	& \emph{danyasa}
	\\
	
\Ins{}
	& \emph{edanyari}
	& \emph{adanyari}
	& \emph{danyari}
	\\

\bottomrule
\end{tabu}
\label{tab:detpro}
\end{table}

Demonstrative pronouns in Ayeri are formed with the demonstrative 
prefixes\index{prefixes}: \xayr{Ed/}{eda-}{this} (proximal), \xayr{Ad/}{ada-}{that} 
(distal), and \xayr{d/}{da-}{such} (indefinite). These are combined with a 
morpheme \rayr{nY}{nya}, which is related to the word for `person', 
\rayr{nYaanF}{nyān}. \autoref{tab:detpro} gives the declined forms for all of 
them. Those forms attested in the corpus gathered from dictionary entries and 
example texts also used for the syllable structure analyses in 
\autoref{sec:phonotactics} appear in upright type, those that should be 
grammatical as well otherwise are given in italic type. The corpus is very 
small, but the prevalence of some forms is possibly reflecting varying degrees 
of grammaticalization\index{grammaticalization} at least to some extent. \autoref{tab:detprontokenfq} 
gives the token frequencies of the various attested forms.

\begin{table}[tp]\centering
\caption{Token frequencies of attested demonstrative pronouns}

\begin{tabu} to .75\linewidth {>{\itshape}X[2l] X[2l] X[1c] X[1c]}
\tableheaderfont\toprule

Pronoun
	& Gloss
	& \multicolumn{2}{c}{Frequency}
	\\

\toprule

edanya
	& this.\Top{}
	& 1
	& 1.69\pct
	\\

adanya
	& that.\Top{}
	& 9
	& 15.25\pct
	\\

danya
	& such.\Top{}
	& 1
	& 1.69\pct
	\\

\midrule

edanyāng
	& this.\Aarg{}
	& 4
	& 6.78\pct
	\\

adanyāng
	& that.\Aarg{}
	& 8
	& 13.56\pct
	\\

edareng
	& this.\AargI{}
	& 3
	& 5.08\pct
	\\

adareng
	& that.\AargI{}
	& 15
	& 25.42\pct
	\\

adanyareng
	& that.\AargI{}
	& 1
	& 1.69\pct
	\\

\midrule

edanyās
	& this.\Parg{}
	& 1
	& 1.69\pct
	\\

adanyās
	& that.\Parg{}
	& 2
	& 3.39\pct
	\\

danyās
	& such.\Parg{}
	& 2
	& 3.39\pct
	\\

edaley
	& this.\PargI{}
	& 2
	& 3.39\pct
	\\

danyaley
	& such.\PargI{}
	& 2
	& 3.39\pct
	\\

\midrule

adayam
	& that.\Dat{}
	& 3
	& 5.08\pct
	\\

\midrule

edanyana
	& this.\Gen{}
	& 1
	& 1.69\pct
	\\

adanyana
	& that.\Gen{}
	& 2
	& 3.39\pct
	\\

danyana
	& such.\Gen{}
	& 1
	& 1.69\pct
	\\

\midrule

adanyaya
	& that.\Loc{}
	& 1
	& 1.69\pct
	\\

\bottomrule

\textup{Total}
	& 
	& 59
	& 100\pct
	\\

\bottomrule
\end{tabu}
\label{tab:detprontokenfq}
\end{table}

Of all the cases, the agent\index{case!agent} demonstratives have the highest token frequency at
a combined 52.5\pct{}, especially the distal pronouns are very frequent in the
sample. Moreover, the distal inanimate\index{animacy} agent\index{case!agent} demonstative occurs twice as often
as its animate counterpart, the shortened form \xayr{AdreNF}{adareng}{that
(one)} being far more current than the full form \rayr{AdnYreNF}{adanyareng}.
Interestingly, the shortened form \xayr{EdreNF}{edareng}{this one} is also the
only one attested for the inanimate\index{animacy} proximate agent\index{case!agent}; similarly, the only dative\index{case!dative}
demonstrative attested once is shortened as well: \xayr{AdymF}{adayam}{(to/for)
that}. For non-core cases\index{case}, only `long' demonstratives are attested, albeit
sparingly so.

Regarding the variation between `long' and `short' forms, it is not surprising
that those demonstratives with a high frequency of use are eroded in some way:
it seems that Ayeri prefers them to stay trisyllabic, which is achieved by
dropping the \rayr{nY}{nya} part.\footnote{According to the so-called Zipf's
law, word length and token frequency correlate in that the most frequently used
words in a language also tend to be the shortest \citep[25--27]{zipf1935}.} A
further reason for dropping the \rayr{nY}{nya} part especially in the inanimate\index{animacy}
demonstratives may be that it is perceived as a marker of animacy\index{animacy}---it has been
noted above already that it is related to the word \xayr{nYaanF}{nyān}{person}.
Both factors, high frequency and semantic mismatch, may thus encourage
contraction. Still, the question of high frequency especially of 
\rayr{AdreNF}{adareng} remains. It may be explained by looking at a few 
typical examples of this word in context, however; see (\ref{ex:demexpl}).

\begin{figure}[h]
\pex[glspace=0.5em]\label{ex:demexpl}
\a\begingl
	\gla Nay ang @ nelyo-ikan sungkorankihas, adareng tono. //
	\glb nay ang= nel-yo=ikan sungkorankihas ada-reng tono //
	\glc and \AgtT{}= help-\TsgN{}=much geography that-\AargI{} certain //
	\glft `And geography, that's for sure, helped me a lot.'%
		\tc{\citep[13]{benung:petitprince}} //
\endgl

\a\begingl
	\gla Adareng merambay-ikan, le @ sundalvāng sasān {vana ...} //
	\glb ada-reng merambay=ikan le= sundal=vāng sasān-Ø {vana ...} //
	\glc that-\AargI{} useful=very \PatTI{}= lose=\Second{}.\Aarg{} way-\Top{} 
		{\Second{}.\Gen{} ...} //
	\glft `It’s very useful if you get lost [...]'%
		\tc{\citep[14]{benung:petitprince}} //
\endgl

\a\begingl
	\gla Adareng danyaley segasena boa tinka. //
	\glb ada-reng danya-ley segas-ena boa tinka //
	\glc that-\AargI{} such-\PargI{} snake-\Gen{} boa closed //
	\glft `The one of the closed boa snake.'\footnotemark%
		\tc{\citep[22]{benung:petitprince}} //
\endgl
\xe
\end{figure}

\footnotetext{More literal translations of this sentence are `That is the one 
of the closed boa snake' or `That is one of a closed boa snake'.}

In all of the example sentences in (\ref{ex:demexpl}),
\xayr{AdreNF}{adareng}{that (one)} serves as a dummy pronoun together with a
predicative adjective or NP\index{phrase types!noun phrase}, which is the main reason why it occurs so
frequently. This is to say, Ayeri prefers the demonstrative pronoun
\rayr{AdreNF}{adareng} as the dummy agent\index{case!agent} in predicative\index{grammatical function!predicative complement} contexts over the
personal pronoun \xayr{reNF}{reng}{it}. Otherwise, however, demonstrative
pronouns work regularly as deictic\index{deixis} anaphora: `this', `that', and `such (a)',
except that as nominal elements they are declined for case\index{case}---but not for number\index{number}, 
which is a notable difference between demonstrative pronouns and
personal pronouns\index{pronouns!personal}. Example (\ref{ex:demproanaph1}) illustrates the use of the
indefinite demonstrative pronoun, \xayr{dnY}{danya}{(such) one} in reference to
the singular NP \xayr{nNaasF}{nangās}{house}; (\ref{ex:demproanaph2}) gives an
example of a demonstrative pronoun in an oblique case\index{case}, \xayr{AdnYri}{adanyari}
{due to that}, with reference to the plural NP
\xayr{Ed/migorjye}{eda-migorayye}{these flowers}. In the latter example, the
pronoun does not inflect for its antecedent's \textsc{\Num{}ber} feature.

\begin{figure}[h]
\pex\label{ex:demproanaph1}
\a\begingl
	\gla Ang @ vehya {} @ Ajān nangās. //
	\glb ang= veh-ya Ø= Ajān nanga-as //
	\glc \AgtT{}= build-\TsgM{} \Top{}= Ajān house-\Parg{} //
	\glft `Ajān builds a house.' //
\endgl

\a\begingl
	\gla Nangās? Sa @ vehyāng may danya. //
	\glb nanga-as sa= veh=yāng may danya-Ø //
	\glc house-\Parg{} \PatT{}= build=\TsgM{}.\Aarg{} \Aff{} such-\Top{} //
	\glft `A house? He builds one indeed.' //
\endgl

\xe
\end{figure}

\begin{figure}[h]
\pex\label{ex:demproanaph2}
\a\begingl
	\gla Sā @ hasuyeng eda-migorayye. //
	\glb sā= hasu=yeng eda=migoray-ye-Ø //
	\glc \CauT{}= sneeze=\TsgF{}.\Aarg{} this=flower-\Pl{}-\Top{} //
	\glft `These flowers make her sneeze.' //
\endgl

\a\begingl
	\gla Ang @ tipinyon nivaye yena adanyari naynay. //
	\glb ang= tipin-yon niva-ye-Ø yena adanya-ri naynay //
	\glc \AgtT{}= itch-\TplN{} eye-\Pl{}-\Top{} \TsgF{}.\Gen{} that-\Caus{} 
		as.well //
	\glft `Her eyes are itching due to that/them/those [the flowers] as 
		well.' //
\endgl
\xe
\end{figure}

As mentioned in the previous chapter (\autoref{nounprefixes},
p.~\pageref{nounprefixes}), the prefix\index{prefixes} \xayr{d/}{da-}{such, so} can combine
with a range of syntactic phrase types, but most notably NPs\index{phrase types!noun phrase}, to serve as an
indefinite demonstrative meaning `such (a)', as in (\ref{ex:danoun}).

\begin{figure}[h]
\ex\label{ex:danoun}%
\begingl
	\gla Adareng da-dipakanas. //
	\glb adareng da=dipakan-as //
	\glc that-\AargI{} such=pity-\Parg{} //
	\glft `That is such a pity.' //
\endgl
\xe
\end{figure}

\rayr{d/}{da-} can be used to express English\index{English} `one' in the sense of a deictic\index{deixis}
anaphora as well. Thus, in order to express `the \textsc{adjective} one', it
may be necessary to use the full demonstrative pronoun, \rayr{dnY}{danya},
since adjectives\index{adjectives} themselves do not decline, and Ayeri largely avoids undeclined
NPs\index{phrase types!noun phrase}. An example is given in (\ref{ex:danyaadj}). Also see
\autoref{subsec:uncased} above for examples of situations where nouns regularly
do not exhibit case\index{case} marking. It is also possible, however, to abbreviate
\rayr{dnY}{danya} to the prefixed\index{prefixes} form \rayr{d/}{da-}, which may be
complemented by adjectives\index{adjectives} and possessive pronouns\index{pronouns!possessive} alike. The adjective\index{adjectives} or
pronoun basically forms a complex anaphora, then, which in most circumstances
can be marked for case\index{case} and topic\index{grammatical function!topic} like any other nominal element, as
demonstrated in (\ref{ex:redone}).

\begin{figure}
\pex\label{ex:danyaadj}
\a\begingl
	\gla Silvyo ku-mino-ing danyāng kivo. //
	\glb silv-yo ku=mino=ing danya-ang kivo //
	\glc look-\TsgN{} like=happy=so such-\Aarg{} little //
	\glft `The little one looks so happy.' //
\endgl

\a\label{ex:danyatop}\begingl
	\gla Sa @ noyang danya tuvo. //
	\glb sa= no=yang danya-Ø tuvo //
	\glc \PatT{}= want=\Fsg{}.\Aarg{} such-\Top{} red //
	\glft `I want the red one.' //
\endgl
\xe
\end{figure}

\begin{figure}[h]
\ex\label{ex:redone}\begingl
	\gla Sa @ noyang da-tuvo. //
	\glb sa= no=yang da=tuvo-Ø //
	\glc \PatT{}= want=\Fsg{}.\Aarg{} such=red-\Top{} //
	\glft `I want the red one.' //
\endgl\xe
\end{figure}

If incorporated in this way, the adjective\index{adjectives} cannot take comparison\index{comparison} morphology:
(\ref{ex:demadjsupl1}) is not possible since inflections cannot be appended to
clitics\index{clitics}.
% (if we analyze \rayr{/ENF}{-eng} and \rayr{/vaa}{-vā} as such in this
% context)
Moreover, the meaning of (\ref{ex:demadjsupl2}) differs from what was
intended, since the \rayr{/vaa}{-vā} clitic is appended not to the adjective\index{adjectives},
but to the composite nominal as such.

\begin{figure}[h]
\pex
\a\label{ex:demadjsupl1}\ljudge*\begingl
	\gla da-tuvo-vāley //
	\glb da=tuvo=vā-ley //
	\glc one=red=\Supl{}-\PargI{} //
	\glft \textit{Intended:} `the reddest one' //
\endgl

\a\label{ex:demadjsupl2}\ljudge\excl\begingl
	\gla da-tuvoley-vā //
	\glb da=tuvo-ley=vā //
	\glc one=red-\PargI{}=most/*\Supl{} //
	\glft `most red ones' \\
		\textit{Intended:} `the reddest one' //
\endgl
\xe
\end{figure}

\index{pronouns!demonstrative|)}

\subsection{Interrogative pronouns}
\label{subsec:interpro}
\index{pronouns!interrogative|(}

\begin{table}\centering
\caption{Interrogative pronouns}
\begin{tabu} to \linewidth {l l X}
\tableheaderfont\toprule
Pronoun
	& Literal meaning
	& Idiomatic meaning
	\\

\toprule

\fw{sinya} % \rayr{sinY}{sinya}
	& `which one' (\tayr{nyān}{person}) %\xayr{nYaanF}{nyān}{person})
	& `who', `what', `which'
	\\

\midrule

\fw{sikan} % \rayr{siknF}{sikan}
	& `how much' (\tayr{ikan}{much}) %\xayr{IknF}{ikan}{much})
	& `how much', `how many'
	\\

\fw{sikay} % \rayr{sikj}{sikay}
	& `with what' (\tayr{kayvo}{with}) %\xayr{kjvo}{kayvo}{with})
	& `how' (tool, circumstance)
	\\

\fw{simin} % \rayr{siminF}{simin}
	& `which way' (\tayr{miran}{way}) %\xayr{mirnF}{miran}{way})
	& `how' (way, procedure)
	\\

\fw{sitaday} % \rayr{sitdj}{sitaday}
	& `which time' (\tayr{taday}{time}) %\xayr{tdj}{taday}{time})
	& `when'
	\\

\fw{siyan} % \rayr{siynF}{siyan}
	& `which place' (\tayr{yano}{place}) %\xayr{yno}{yano}{place})
	& `where'
	\\

\bottomrule
\end{tabu}
\label{tab:interpro}
\end{table}

The interrogative pronouns are all formed with \rayr{si/}{si-}, combined with
a lexical element or a case\index{case} marker; \rayr{si/}{si-} is also related to the
relativizer\index{pronouns!relative} \rayr{si}{si}. The interrogative pronouns are listed in
\autoref{tab:interpro}. All interrogative pronouns share the property that they
are placed \fw{in situ}. That is, they appear in the same position as the
phrase they stand in for, so there will not be movement of the question word to
the front as in English\index{English}. Additionally, impersonal interrogative pronouns cannot
be topicalized\index{grammatical function!topic} since they also do not inflect for case\index{case}, which preempts the
difference between zero-marked topicalized\index{grammatical function!topic} and overtly case-marked
untopicalized\index{grammatical function!topic} forms. This is illustrated in (\ref{ex:qprondist}).

\begin{figure}[h]
\pex\label{ex:qprondist}
\a\begingl
	\gla Sa @ petigavāng inun sikan? //
	\glb sa= petiga=vāng inun-Ø sikan //
	\glc \PatT{}= catch=\Second{}.\Aarg{} fish-\Top{} how.much //
	\glft `How much fish did you catch?' //
\endgl

\a\begingl
	\gla Sa-sahavāng sitaday? //
	\glb sa\til{}saha=vāng sitaday //
	\glc \Iter{}\til{}come=\Second{}.\Aarg{} when //
	\glft `When will you return?' //
\endgl
\xe
\end{figure}

In the table on interrogative pronouns above, \xayr{sinY}{sinya}{who, what,
which} is seperated from the other pronouns because it behaves differently.
Namely, it can be declined for all cases\index{case} according to the syntactic or semantic
role of the NP\index{phrase types!noun phrase} it replaces, and it can also be topicalized\index{grammatical function!topic}, since the element
asked about is likely high in discourse salience; compare (\ref{ex:qprotop}).

\begin{figure}[h]
\pex\label{ex:qprotop}
\a\begingl
	\gla Ang @ yomayo sinya adaya?\footnotemark //
	\glb ang= yoma-yo sinya-Ø adaya //
	\glc \AgtT{}= exist-\TsgN{} who-\Top{} there //
	\glft `Who is there?' //
\endgl

\a\begingl
	\gla Sa @ narayeng sinya? //
	\glb sa= nara=yeng sinya-Ø //
	\glc \PatT{}= say=\TsgF{}.\Aarg{} what-\Top{} //
	\glft `What did she say?' //
\endgl
\xe
\end{figure}

\footnotetext{This may be shortened to just \xayr{sinYaaNF Ady?}{sinyāng 
adaya?}{who (is) there?} (who-\Aarg{} there).}

\begin{table}[tp]\centering
\caption{Declension paradigm for \xayr{sinY}{sinya}{who, what}}
\begin{tabu} to \linewidth {l l X}
\tableheaderfont\toprule
Case
	& Pronoun
	& Translation
	\\

\toprule

\Top{}
	& \fw{sinya} % \rayr{sinY}{sinya}
	& `who', `what'
	\\

\midrule

\Aarg{}
	& \fw{sinyāng} % \rayr{sinYaaNF}{sinyāng}
	& `who', `what'
	\\

\AargI{}
	& \fw{sinyareng} % \rayr{sinYreNF}{sinyareng}
	& `who', `what'
	\\
\Parg{}
	& \fw{sinyās} % \rayr{sinYaasF}{sinyās}
	& `whom', `what'
	\\
\PargI{}
	& \fw{sinyaley} % \rayr{sinYlej}{sinyaley}
	& `whom', `what'
	\\
\Dat{}
	& \fw{sinyayam} % \rayr{sinYymF}{sinyayam}
	& `for/to whom', `for/to what'
	\\

\midrule

\Gen{}
	& \fw{sinyana} % \rayr{sinYn}{sinyana}
	& `whose', `from whom', `from what'
	\\

\Loc{}
	& \fw{sinyaya} % \rayr{sinYy}{sinyaya}
	& `in/at/on whom', `in/at/on what'
	\\

\Caus{}
	& \fw{sinyisa} % \rayr{sinYis}{sinyisa}
	& `due to/because of whom', `due to/because of what'
	\\

\Ins{}
	& \fw{sinyari} % \rayr{sinYri}{sinyari}
	& `by whose help', `with what'
	\\

\bottomrule
\end{tabu}
\label{tab:sinya}
\end{table}

Ayeri does not strictly distinguish animate from inanimate\index{animacy} referents in its
interrogative pronouns, so there is no distinction between `who' and `what'.
\rayr{sinY}{sinya} and/or the verb\index{verbs} will instead inflect according to context
and to the speaker's expectations or knowledge (compare \autoref{tab:sinya}).
Thus, there is also no dedicated question word for `why', since in Ayeri one
can simply ask `due to what/whom' by inflecting \rayr{sinY}{sinya};
\rayr{sinYis}{sinyisa} is \rayr{sinY}{sinya} marked for causative case\index{case!causative} by the
suffix\index{suffixes} \rayr{/Is}{-isa}. Declension of \rayr{sinY}{sinya} for different
purposes is shown in (\ref{ex:sinyacase}).

\begin{figure}[h]
\pex\label{ex:sinyacase}
\a\begingl
	\gla Le @ kayāng adanya sinyayam? //
	\glb le= ka=yāng adanya-Ø sinya-yam //
	\glc \PatTI{}= throw.away=\TsgM{}.\Aarg{} that-\Top{} what-\Dat{} //
	\glft `Why (= what for) did he throw that away?' //
\endgl

\a\begingl
	\gla Ang @ prantoyva sinyisa? //
	\glb ang= prant-oy=va.Ø sinya-isa //
	\glc \AgtT{}= ask-\Neg{}=\Second{}.\Top{} what-\Caus{} //
	\glft `Why (= because of what) did you not ask?' //
\endgl
\xe
\end{figure}

While there is no single, dedicated word for `why', Ayeri distinguishes between
two kinds of `how': \rayr{siminF}{simin}, on the one hand, asks about the way
by which---or the circumstances under which---an action is carried out, see
(\ref{ex:simin}). \rayr{sikj}{sikay}, on the other hand, asks for the means or
tools used to carry out an action, see (\ref{ex:sikay}). Thus, the correct
answer to the question in (\ref{ex:simin}) needs to treat the process of making
bread, since \rayr{siminF}{simin} asks about the way of doing something; a
correct answer to the question in (\ref{ex:sikay}), on the other hand, will
likely mention grinding utensils, like a mill or a pestle.

\begin{figure}[h]
\pex
\a\label{ex:simin}\begingl
	\gla Le @ tiyavāng vadisān simin? //
	\glb le= tiya=vāng vadisān-Ø simin //
	\glc \PatTI{}= make=\Second{}.\Aarg{} bread-\Top{} how //
	\glft `How do you make bread?' //
\endgl

\a\label{ex:sikay}\begingl
	\gla Le @ peralvāng sagan sikay? //
	\glb le= peral=vāng sagan-Ø sikay //
	\glc \PatTI{}= grind=\Second{}.\Aarg{} flour-\Top{} how //
	\glft `How do you grind flour?' //
\endgl
\xe
\end{figure}

Comparing Tables \ref{tab:interpro} and \ref{tab:sinya}, strikingly, there are
two possbilities to express `where'---lexical \rayr{siynF}{siyan} and synthetic
\rayr{sinYy}{sinyaya}. These, however, are not strictly interchangable, even
though some variation is to be expected. While \rayr{siynF}{siyan} refers to
\emph{places} in general, the \rayr{sinY}{sinya} series refers to
\emph{discourse participants} both animate and inanimate\index{animacy} more specifically, as
shown in (\ref{ex:siyansinya}).

\begin{figure}[h]
\pex\label{ex:siyansinya}
\a\begingl
	\gla Saravāng siyan? --- Ya @ Sikatay. //
	\glb sara=vāng siyan --- ya= Sikatay //
	\glc go=\Second{}.\Aarg{} where --- \Loc{}= Sikatay //
	\glft `\,\enquote{Where are you going?}---\enquote{To Sikatay.}\,' //
\endgl

\a\begingl
	\gla Ya @ divvāng sinya? --- Ya @ Haki. //
	\glb ya= div=vāng sinya-Ø --- ya= Haki //
	\glc \LocT{}= stay=\Second{}.\Aarg{} who-\Top{} --- \Loc{}= Haki //
	\glft `\,\enquote{Who are you staying with?}---\enquote{At Haki's}\,' //
\endgl
\xe
\end{figure}

\index{pronouns!interrogative|)}

\subsection{Indefinite pronouns}
\label{subsec:indefpro}
\index{pronouns!indefinite|(}
\index{typology|(}

\citet[56]{haspelmath1997} notes how descriptions of languages often do not
document indefinite pronouns---whether they simply do not exist in this
language or whether they escaped the author's attention remains unknown in
these cases. It may thus be duly noted here that Ayeri does indeed possess
indefinite pronouns.\footnote{Since it is an invented language, the value of
this assertion to linguistic typology\index{typology} remains doubtful, however.} In order to
classify languages, \citet{haspelmath1997} generalizes the map displayed in
\autoref{fig:haspeltab} based on a sample of 100 languages from all continents,
although he notes that this sample has a European bias due to the availability
of data \citep[2]{haspelmath1997}. Languages typically form continguous areas
on the map, even though they may carve it up quite differently, and with
overlaps between the different semantic groupings 1--9.

An interesting question that \citet{haspelmath1997} poses towards the end of
his book is whether there are any correlations between word order\index{word order} typology\index{typology} and
the preference for generic nouns\index{nouns!generic} (`person', `thing', `place', `time', `manner')
or, for instance, interrogative-based systems \citep[239--241]{haspelmath1997}.
From \citet{haspelmath1997}'s concluding statistics it looks as though there is
a slight preference of languages with which Ayeri shares basic typological\index{typology}
traits---such as verb-initial, verb--object, and noun--genitive word order,
also having prepositions---for basing indefinite pronouns on generic nouns\index{nouns!generic}.
\citet{haspelmath1997} concedes that these seeming correlations are skewed by
areal effects, \textcquote[241]{haspelmath1997}{because indefinite pronouns
have a strongly areal distribution}.\footnote{The map in \citetitle{wals}
\citep{wals46A} suggests areal clusters at least for generic-noun\index{nouns!generic} based systems
in Africa and Southeast Asia. \citetitle{wals} classifies 60\pct{} of the
sampled languages as possessing interrogative-based indefinite pronouns, with
evidence for this type quoted for all continents except Africa. The next
smaller group, generic-noun\index{nouns!generic} based, falls behind at 26\pct. The lack of evidence
for the interrogative type in Africa despite being the most frequent one in the
set may be due to the unavailability of data. Crossreferencing
constituent-order\index{word order} and indefinite-pronoun systems did not yield a result which
obviously suggested a correlation.} He still presumes, however, that word-order\index{word order}
typology may have an effect on the formation of indefinites insofar as it
correlates with grammaticalization more generally \citep[239]{haspelmath1997}.

\begin{figure}[tp]\centering
\scalebox{.9}{%
\begin{tikzpicture}[x=5em]
\node (1) at (1,3) {(1)};
\node (2) at (2,3) {(2)};
\node (3) at (3,3) {(3)};
\node (4) at (4,4) {(4)};
\node (5) at (4,2) {(5)};
\node (6) at (5,4) {(6)};
\node (7) at (6,5) {(7)};
\node (8) at (5,2) {(8)};
\node (9) at (6,1) {(9)};
%
\draw (1) -- (2);
\draw (2) -- (3);
\draw (3) -- (4);
\draw (3) -- (5);
\draw (4) -- (5);
\draw (4) -- (6);
\draw (5) -- (8);
\draw (6) -- (7);
\draw (6) -- (8);
\draw (8) -- (9);
%
\node[haspanno]            (l1) at (1) {specific known};
\node[haspanno]            (l2) at (2) {specific unknown};
\node[haspanno]            (l3) at (3) {irrealis non-specific};
\node[haspanno, above=2ex] (l4) at (4) {question};
\node[haspanno]            (l5) at (5) {conditional};
\node[haspanno, above=2ex] (l6) at (6) {indirect negation};
\node[haspanno]            (l7) at (7) {direct negation};
\node[haspanno]            (l8) at (8) {comparative};
\node[haspanno]            (l9) at (9) {free choice};
\end{tikzpicture}
}
\caption[The implicational map for indefinite pronoun functions]{The 
implicational map for indefinite pronoun functions \citep[4]{haspelmath1997}}
\label{fig:haspeltab}
\end{figure}

\begin{table}\centering
\caption{Indefinite pronouns}

\begin{tabu} to \linewidth {C[2] X[3c] X[3c] X[3c]}
\toprule\tableheaderfont

Property
	& every
	& some
	& none
	\\
\toprule
	
person
	& enya % every/any
	& arilinya % some
	& ranya % none
	\\
	
thing
	& enya % every/any
	& arilinya, arilya % some
	& ranya % none
	\\
\midrule
	
place
	& yanen % every/any
	& yāril % some
	& yanoy % none
	\\
\midrule
	
time
	& tadayen % every/any
	& tajaril; metay % some
	& tadoy; jānyam % none
	\\
\midrule
	
manner
	& arēn % every/any
	& miranaril % some
	& aremoy % none
	\\
	
\midrule

reason
	& --- % every/any
	& yāril % some
	& --- % none
	\\

\bottomrule

\end{tabu}

\label{tab:indeftab}
\end{table}

\citet{haspelmath1997} mentions generic nouns\index{nouns!generic}, and these can be combined with
the quantifying\index{quantifiers} expressions `every', `any', `some', and `none' into an array
like the one presented in \autoref{tab:indeftab}. Ayeri does not distinguish
`every' from `any' as English\index{English} does; there is also no distinction in polarity
(affirmative versus negative\index{negation}) the way English\index{English} has it. See (\ref{ex:englpol})
for an example.

\begin{figure}[h]
\pex\label{ex:englpol}%
	English:
	\a\ljudge* \fw{I don't know something about this.}
	\a \fw{I don't know anything about this.}
\xe
\end{figure}

Likewise, Ayeri does not distinguish between animate and inanimate\index{animacy} indefinite 
referents. The same pronouns are used for either, although the shortening of 
\rayr{ArilinY}{arilinya}, \rayr{ArilY}{arilya}, can only be used for 
inanimates\index{animacy}, similar to the distinction in the demonstrative pronouns\index{pronouns!demonstrative} between 
\xayr{AdnYaaNF}{adanyāng}{that one} (that.one-\Aarg{}) and 
\xayr{AdnYreNF}{adareng}{that one} (that.one-\AargI{}; see 
\autoref{subsec:dempro}). Two further features stand out, however.

\phantomsection\label{indefprocomp}
Firstly, most of the pronouns in the chart have a lexical part---Ayeri's
indefinite pronouns are based on generic nouns\index{nouns!generic}. Thus, the pronouns referring to
people and things all have the \rayr{/nY}{-nya} element in common, which we
also find in the interrogative and demonstrative pronouns\index{pronouns!demonstrative}, and which also
appears in the word \xayr{nYaanF}{nyān}{person}. In the same way, the pronouns
related to the notion of place have a \rayr{y/}{ya-} or \rayr{ynF/}{yan-} part,
which we also find in \xayr{yno}{yano}{place}.\footnote{\rayr{yno}{yano} itself
is an old nominalization\index{nominalization} and very likely related as a morpheme to the locative\index{case!locative}
suffix\index{suffixes} \rayr{/y}{-ya}.} In a regular continuation of this pattern, the
indefinite pronouns of time all have an element related to
\xayr{tdj}{taday}{time} in common, which is obscured somewhat by palatalization
in \rayr{tdYrilF}{tajaril}. The exception to this series, then, is
\rayr{dYaanFymF}{jānyam}, which is the multiplicative numeral\index{numerals} formed from
\xayr{dY}{ja}{zero}, thus means `zero times' or `not once' rather than 
`never', although it can also be used emphatically for the latter. The series
of manner pronouns is an absolute exception in that it must be a residue from
an older layer of grammaticalization\index{grammaticalization} since \rayr{Are/}{are-} is not a
recognizable morpheme in the modern language.\footnote{I probably made this up
as I was going, many years ago, and without considering systematic
implications, since I was unaware of them at the time.} \rayr{mirnrilF}
{miranaril} is a regular formation of \xayr{mirnF}{miran}{way, manner} combined
with the quantifier\index{quantifiers} (!) for indefinite amounts, \xayr{/ArilF}{-aril}{some}.

This observation leads to the second regular feature, that is, affixes as 
modifiers to generic nouns\index{nouns!generic}. The `every' series regularly features the 
morpheme \rayr{EnF}{en}, either prefixed\index{prefixes} or suffixed\index{suffixes}, which is related to the 
quantifier\index{quantifiers} \xayr{/henF}{-hen}{every, all, each} and can presumably be found 
even on \rayr{AreenF}{arēn} in spite of its obscure lexical base. In the same 
manner, the series related to inspecific generic-noun\index{nouns!generic} referents is marked by 
the affix \rayr{ArilF}{aril} which, as we have just seen above, is otherwise 
used to refer to inspecific quantities, for instance, 
\xayr{vdiːsnF/ArilF}{vadisān-aril}{some bread} (bread=some). In the case of 
\rayr{mirnrilF}{miranaril}, the suffix\index{suffixes} seems somewhat of an 
odd choice, since manner is not a quantifiable variable in the same way people,
things, locations\index{semantic role!location}, or moments are. Possibly, it is chosen rather in analogy
with the other pronouns in this series than on semantic grounds. In any event,
\rayr{metj}{metay} has the semantically more `proper' \rayr{me/}{me-} prefix\index{prefixes},
relating it to absolute inspecificity.\footnote{Compare German\index{German}
\fw{irgendjemand} and French\index{French} \fw{n'importe qui} `no matter who'.} This 
alternation is employed to distinguish between the meaning of `sometime', that 
is, occurring once at an unspecified point in time, and 
\xayr{tdYrilF}{tajaril}{sometimes}, which refers to repeated occurrence at
inspecific times. The alternation between \rayr{mirnrilF}{miranaril} and
regularly derived \rayr{me/mirnF}{mə-miran} can be leveraged to express a
specificity difference as well. While the former suggests that an action is
carried out or an event is happening by means of a specific, though unknown
procedure, the latter suggests just any possible procedure. Lastly, the
negative\index{negation} series is reguarly marked by the negative\index{negation} suffix\index{suffixes} \rayr{/Oj}{-oy},
which also occurs with adjectives and verbs (see sections \ref{subsec:adjneg}
and \ref{subsubsec:verbneg}). An outlier in this series is the 
person/thing-related indefinite pronoun, \rayr{rnY}{ranya}. The etymological 
connections of the \rayr{r}{ra} part are not presently known, perhaps the 
postposition \xayr{rnF}{ran}{against} is related.

The chart in \autoref{tab:indeftab} only tells half the truth by not giving any
information on use contexts for the individual forms, so how do they fit in
with the chart from \citet{haspelmath1997} quoted at the beginning of this
section? Regarding the functions of indefinite pronouns annotated to the
numbers on the map, \citet{haspelmath1997} gives the example sentences in
(\ref{ex:indeftypo}, which, however, mostly only give one example for either
the `person' or `thing' category at a time. It is up to the reader to
generalize from this \citep[2--3]{haspelmath1997}.\footnote{These appear here
reordered according to numerical order. The book lists them according to their
logical order as tracing the map, the enumeration somewhat confusingly tied in
with the running enumeration of examples.}

\begin{figure}[h]
\pex[labeltype=numeric]\label{ex:indeftypo}
\a specific, known to the speaker: \smallskip\\ % 1
	\textit{\underline{Somebody} called while you were away: guess who!}
	
\a specific, unknown to the speaker: \smallskip\\ % 2
	\textit{I heard \underline{something}, but I couldn't tell what kind of 
	sound it was.}
	
\a non-specific, irrealis: \smallskip\\ % 3
	\textit{Please try \underline{somewhere} else.}
	
\a polar question: \smallskip\\ % 4
	\textit{Did \underline{anybody} tell you anything about it?}
	
\a conditional protasis: \smallskip\\ % 5
	\textit{If you see \underline{anything}, tell me immediately.}
	
\a indirect negation: \smallskip\\ % 6
	\textit{I don't think that \underline{anybody} knows the answer.}
	
\a direct negation: \smallskip\\ % 7
	\textit{\underline{Nobody} knows the answer.}
	
\a standard of comparison: \smallskip\\ % 8
	\textit{In Freiburg the weather is nicer than \underline{anywhere} in 
	Germany.}
	
\a free choice: \smallskip\\ % 9
	\textit{\underline{Anybody} can solve this simple problem.}
\xe
\end{figure}

\begin{figure}\centering
\scalebox{.9}{%
\begin{tikzpicture}[x=5em]
\node (1) at (1,3) {(1)};
\node (2) at (2,3) {(2)};
\node (3) at (3,3) {(3)};
\node (4) at (4,4) {(4)};
\node (5) at (4,2) {(5)};
\node (6) at (5,4) {(6)};
\node (7) at (6,5) {(7)};
\node (8) at (5,2) {(8)};
\node (9) at (6,1) {(9)};
%
\draw (1) -- (2);
\draw (2) -- (3);
\draw (3) -- (4);
\draw (3) -- (5);
\draw (4) -- (5);
\draw (4) -- (6);
\draw (5) -- (8);
\draw (6) -- (7);
\draw (6) -- (8);
\draw (8) -- (9);
%
\node[haspanno]            (l1) at (1) {specific known};
\node[haspanno]            (l2) at (2) {specific unknown};
\node[haspanno]            (l3) at (3) {irrealis non-specific};
\node[haspanno, above=2ex] (l4) at (4) {question};
\node[haspanno]            (l5) at (5) {conditional};
\node[haspanno, above=2ex] (l6) at (6) {indirect negation};
\node[haspanno]            (l7) at (7) {direct negation};
\node[haspanno]            (l8) at (8) {comparative};
\node[haspanno]            (l9) at (9) {free choice};
%
\draw[semithick] (0.5,5.0) -- (3.0,5.0) -- (4.0,5.0) -- (5.5,5.0) 
-- (5.5,3.0) -- (4.5,3.0) -- (4.5,1.0) -- (4.0,1.0) -- (3.0,1.0) -- (0.5,1.0) 
-- (0.5,5.0);
%
\draw[dotted, semithick] (2.75,3.5) -- (3.0,3.5) -- (4.0,4.5) -- (4.25,4.5) 
-- (4.25,1.5) -- (4.0,1.5) -- (3.0,2.5) -- (2.75,2.5) 
-- (2.75,3.5);
%
\node[draw, loosely dotted, semithick, fit=(1) (2)] {};
%
\node[draw, dashed, semithick, fit=(6) (7)] {};
%
\node[draw, dash dot, semithick, fit=(8) (9)] {};
%
\node[haspanno2, below=.25em, anchor=north west,] at (0.50,1.00)
	{↑ `some' series};
\node[haspanno2, above=.75em, anchor=south west,] at (0.75,3.25)
	{plain generic nouns ↓};
\node[haspanno2, above=.25em, anchor=south west,] at (2.75,3.50) {mə- ↓};
\node[haspanno2, below=.25em, anchor=north west,] at (5.50,3.50)
	{↑ `none' series};
\node[haspanno2, below=.25em, anchor=north west,] at (4.75,0.5)
	{↑ `every' series};
\end{tikzpicture}
}
\caption[Map of indefinite pronoun functions in Ayeri]{Map of indefinite 
pronoun functions in Ayeri}
\label{fig:haspeltabayr}
\end{figure}

As we have seen in \autoref{tab:indeftab} above, Ayeri does not make a
difference between `every' and `any', which is why the `some' series can be
applied to all of (1)--(5); it can also be used for indirect negation\index{negation} (6). The
pronouns from the `none' column, then, are used to express direct negation\index{negation} (7).
Since double negation\index{negation}---that is, agreement\index{agreement} in negation\index{negation} between verbs\index{verbs} and
indefinite pronouns for purposes of emphasis rather than double negation\index{negation} in the
strictly logical sense---is possible, the `none' series may also be employed
for indirect negation\index{negation} (6). Moreover, Ayeri uses the `every' series for both
standard of comparison (8) and free choice (9). Besides this, 
absolute-indefinite \rayr{me/}{me-} can be used for (3) to (6) in combination 
with a (generic) noun\index{nouns!generic} to attach to.

It needs to be noted that only the 
indefinite pronouns with person or thing reference (those including 
\rayr{nY}{nya}) decline; they can also be topicalized\index{grammatical function!topic}. The other indefinites, 
relating to place, time and manner, are indeclinable and also cannot be topics\index{grammatical function!topic} 
for this reason.
%
% Off the top of my head I'm not sure if I've done this before or if it is a 
% new rule I made up here for the purpose of spicing things up a little:
%
For the `specific' categories (1) and (2) it is furthermore possible to use the
plain generic nouns\index{nouns!generic}, \xayr{nYaanF}{nyān}{person}, \xayr{linY}{linya}{thing},
\xayr{yno}{yano}{place}, \xayr{tdj}{taday}{time}, \xayr{mirnF}{miran}{way}, 
however. \autoref{fig:haspeltabayr} shows the groupings for Ayeri; 
(\ref{ex:indefex}) gives examples of all types.

\needspace{3\baselineskip}
\pex[labeltype=numeric,interpartskip=1em]\label{ex:indefex}
\a specific, known to the speaker:\vspace{.5em} % 1
	\beginsubsub
	\b{a.} \begingl
		\gla Ang @ sahaya \textbf{arilinya}, leku, sinyāng adaley! //
		\glb ang= saha-ya arilinya-Ø lek-u sinya-ang ada-ley //
		\glc \AgtT{}= come-\TsgM{} someone-\Top{} guess-\Imp{} 
			who-\Aarg{} that-\PargI{} //
		\glft `Someone came, guess who it is!' //
		\endgl\vspace{.5em}
		
	\b{b.} \begingl
		\gla Le @ ilta ningyang \textbf{linya} vayam. //
		\glb le= ilta ning=yang linya-Ø vayam //
		\glc \PatTI{}= need tell=\Fsg{}.\Aarg{} thing-\Top{} 
			\Second{}.\Dat{} //
		\glft `I need to tell you something.' //
		\endgl
	\endsubsub

\needspace{3\baselineskip}
\a specific, unknown to the speaker:\vspace{.5em} % 2
	\beginsubsub
	\b{a.} \begingl
		\gla Ang @ pegaya \textbf{arilinya} pangisley nā. //
		\glb ang= pega-ya arilinya-Ø pangis-ley nā //
		\glc \AgtT{}= steal-\TsgM{} someone-\Top{} money-\PargI{} 
			\Fsg{}.\Gen{} //
		\glft `Someone stole my money.' //
		\endgl\vspace{.5em}
		
	\b{b.} \begingl
		\gla Ang @ sarayan \textbf{yanoya} agon. //
		\glb ang= sara=yan yano-ya agon //
		\glc \AgtT{}= go=\TplM{}.\Top{} place-\Loc{} foreign //
		\glft `They are going somewhere foreign.' //
		\endgl
	\endsubsub

\needspace{3\baselineskip}	
\a non-specific, irrealis:\vspace{.5em} % 3
	\beginsubsub
	\b{a.} \begingl
		\gla Pinyan, prantu \textbf{yāril} palung. //
		\glb pinyan prant-u yāril palung //
		\glc please ask-\Imp{} somewhere different //
		\glft `Please ask somewhere else.' //
		\endgl\vspace{.5em}
		
	\b{b.} \begingl
		\gla Le @ ilta @ miranang adanya \textbf{mə-}miraneri 
			palung. //
		\glb le= ilta= mira=nang adanya-Ø mə-miran-eri palung //
		\glc \PatTI{}= need= do=\Fsg{}.\Aarg{} that.one-\Top{} 
			some-way-\Ins{} different //
		\glft `We need to do that in some other way.' //
		\endgl
	\endsubsub

\needspace{3\baselineskip}	
\a polar question\index{questions}:\vspace{.5em} % 4
	\beginsubsub
	\b{a.} \begingl
		\gla Ang @ koronva \textbf{arilinyaley} edanyana? //
		\glb ang= koron=va.Ø arilinya-ley edanya-na //
		\glc \AgtT{}= know=\Second{}.\Top{} something-\PargI{} 
			this.one-\Gen{} //
		\glft `Do you know anything about this?' //
		\endgl\vspace{.5em}
		
	\b{b.} \begingl
		\gla Yomaya \textbf{mə-}nyānang si ang @ vaca mirongya 
			edanyaley? //
		\glb yoma-ya mə-nyān-ang si ang= vaca mira-ong=ya.Ø
			edanya-ley //
		\glc exist-\TsgM{} some-person-\Aarg{} \Rel{} \AgtT{}= 
			like do-\Irr{}=\TsgM{}.\Top{} this-\PargI{} //
		\glft `Is there \emph{anyone} who would like to do this?' //
		\endgl
	\endsubsub

\needspace{3\baselineskip}
\a conditional protasis:\vspace{.5em} % 5
	\beginsubsub
	\b{a.} \begingl
		\gla Ang @ ming pengalayn sitanyās \textbf{yāril}, adareng 
			pray-ven. //
		\glb ang= ming pengal=ayn.Ø sitanya-as yāril ada-reng 
			pray=ven //
		\glc \AgtT{}= can meet-\Fpl{}.\Top{} each.other-\Parg{} 
			somewhere that-\AargI{} great=pretty //
		\glft `If we can meet somewhere that would be pretty great.' //
		\endgl\vspace{.5em}
		
	\b{b.} \begingl
		\gla Sa @ na-naravāng \textbf{mə-}lentan, ang @ haray vās! //
		\glb sa= na\til{}nara=vāng mə-lentan-Ø ang= har=ay.Ø vās //
		\glc \PatT{}= \Iter{}\til{}say=\Second{}.\Aarg{} some-sound-\Top{} 
			\AgtT{}= punch-\Fsg{}.\Top{} \Second{}.\Parg{} //
		\glft `You make any more sound, I'm gonna punch you!' //
		\endgl
	\endsubsub

\needspace{3\baselineskip}	
\a indirect negation:\vspace{.5em} % 6
	\beginsubsub
	\b{a.} \begingl
		\gla Paronoyyang, ang @ no @ tahaya \textbf{arilinya} adaley. //
		\glb paron-oy=yang ang= no= taha-ya arilinya-Ø ada-ley //
		\glc believe-\Neg{}=\Fsg{}.\Aarg{} \AgtT{}= want= 
			have-\Tsg{}.\M{} anyone-\Top{} that-\PargI{} //
		\glft `I don't think anyone wants to have that.' //
		\endgl\vspace{.5em}
	
	\b{b.} \begingl
		\gla Paronoyyang, le @ ming @ sungvāng adanya \textbf{yanoy}. //
		\glb paron-oy=yang le= ming= sung=vāng adanya-Ø yanoy //
		\glc believe-\Neg{}=\Fsg{}.\Aarg{} \PatTI{}= can= 
			find=\Second{}.\Aarg{} that.one-\Top{} nowhere //
		\glft `I don't think you can find that \emph{anywhere}.' //
		\endgl
	\endsubsub

\needspace{3\baselineskip}	
\a direct negation:\vspace{.5em} % 7
	\beginsubsub
	\b{a.} \begingl
		\gla Ang @ koronya \textbf{ranya} guratanley. //
		\glb ang= koron-ya ranya-Ø guratan-ley //
		\glc \AgtT{}= know-\TsgM{} nobody-\Top{} answer-\PargI{} //
		\glft `Nobody knows the answer.' //
		\endgl\vspace{.5em}
		
	\b{b.} \begingl
		\gla Le @ ming @ sungvāng adanya \textbf{yanoy}. //
		\glb le= ming= sung=vāng adanya-Ø yanoy //
		\glc \PatTI{}= can= find=\Second{}.\Aarg{} that.one-\Top{} nowhere //
		\glft `You can't find that anywhere.' //
		\endgl
	\endsubsub

\needspace{3\baselineskip}	
\a standard of comparison:\vspace{.5em} % 8
	\beginsubsub
	\b{a.} \begingl
		\gla Sa @ engyeng larau \textbf{enya} palung. //
		\glb sa= eng=yeng larau enya-Ø palung //
		\glc \PatT{}= be.more=\TsgF{}.\Aarg{} nice anyone different //
		\glft `She is nicer than anyone else.' //
		\endgl\vspace{.5em}
		
	\b{b.} \begingl
		\gla Ang @ engyo ban eda-riman \textbf{yanen} palung. //
		\glb ang= eng-yo ban eda=riman-Ø yanen palung //
		\glc \AgtT{}= be.more-\TsgN{} good this=city-\Top{} anywhere 
			different //
		\glft `This city is better than anywhere else.' //
		\endgl
	\endsubsub

\needspace{3\baselineskip}
\a free choice:\vspace{.5em} % 9
	\beginsubsub
	\b{a.} \begingl
		\gla Ang @ ming @ guraca \textbf{enya} eda-prantanley. //
		\glb ang= ming= gurat-ya enya-Ø eda=prantan-ley //
		\glc \AgtT{}= can= answer-\TsgM{} anyone-\Top{} 
				this=question-\PargI{} //
		\glft `Anyone can answer this question.' //
		\endgl\vspace{.5em}
		
	\b{b.} \begingl
		\gla Epayeng \textbf{tadayen} si sa @ pinyaya ye ang @ Tapan. //
		\glb epa=yeng tadayen si sa= pinya-ya ye ang= Tapan //
		\glc refuse=\TsgF{}.\Aarg{} everytime \Rel{} \PatT{}=
				ask-\TsgM{} \TsgF{}.\Top{} \Aarg{}= Tapan //
		\glft `She refused everytime Tapan asked her.' //
		\endgl
	\endsubsub
	
\xe

\index{typology|)}
\index{pronouns!indefinite|)}

\subsection{Relative pronouns}
\label{subsec:relpro}
\index{pronouns!relative|(}
\index{relative clause|(}

\begin{table}[tp]\centering
\caption{Relative pronouns}

\begin{tabu} to \linewidth {l X[c] X[c] X[c] X[c] X[c] X[c]}
\tableheaderfont\toprule
Case
	& Pronoun
	& \multicolumn{5}{c}{Pronoun with secondary inflection}
	\\

\cmidrule{3-7}
	& 
	& \Dat{}
	& \Gen{}
	& \Loc{}
	& \Caus{}
	& \Ins{}
	\\
	
\toprule

Ø
	& si % Ø
	& siyām % \Dat{}
	& sinā % \Gen{}
	& siyā % \Loc{}
	& sisā % \Caus{}
	& sirī % \Ins{}
	\\

\midrule

\Aarg{}
	& sang % Ø
	& sangyam % \Dat{}
	& sangena % \Gen{}
	& sangya % \Loc{}
	& sangisa % \Caus{}
	& sangeri % \Ins{}
	\\

\Aarg{}.\Inan{}
	& sireng % Ø
	& sirengyam % \Dat{}
	& sirengena % \Gen{}
	& sirengya % \Loc{}
	& sirengisa % \Caus{}
	& sirengeri % \Ins{}
	\\
	
\Parg{}
	& sas % Ø
	& sasyam % \Dat{}
	& sasena % \Gen{}
	& sasya % \Loc{}
	& sasisa % \Caus{}
	& saseri % \Ins{}
	\\

\Parg{}.\Inan{}
	& siley % Ø
	& sileyyam % \Dat{}
	& sileyena % \Gen{}
	& sileyya % \Loc{}
	& sileyisa % \Caus{}
	& sileyeri % \Ins{}
	\\

\Dat{}
	& siyam % Ø
	& siyamyam % \Dat{}
	& siyamena % \Gen{}
	& siyamya % \Loc{}
	& siyamisa % \Caus{}
	& siyameri % \Ins{}
	\\

\midrule

\Gen{}
	& sina/sena % Ø
	& sinayam % \Dat{}
	& sinana % \Gen{}
	& sinaya % \Loc{}
	& sinaisa % \Caus{}
	& sinari % \Ins{}
	\\
	
\Loc{}\footnotemark
	& siya % Ø
	& siyayam % \Dat{}
	& siyana % \Gen{}
	& siyaya % \Loc{}
	& siyaisa % \Caus{}
	& siyari % \Ins{}
	\\
	
\Caus{}
	& sisa % Ø
	& sisayam % \Dat{}
	& sisana % \Gen{}
	& sisaya % \Loc{}
	& sisaisa % \Caus{}
	& sisari % \Ins{}
	\\
	
\Ins{}
	& seri % Ø
	& seriyam % \Dat{}
	& serina % \Gen{}
	& seriya % \Loc{}
	& serīsa % \Caus{}
	& seriri % \Ins{}
	\\

\bottomrule
\end{tabu}
\label{tab:relpro}
\end{table}

\footnotetext{The contracted form \fw{sijya} for \rayr{siyy}{siyaya} is
attested once, compare \citet[12]{becker:kafka:imperial}. Likewise, it should
be possible for \rayr{siyymF}{siyayam} to contract to \fw{sijyam}. The native
spelling of both the long and the contracted forms would not differ, though,
since contracted \rayr{/ye} {-ye} is also still spelled that way in spite of
the difference in pronunciation.}

As described before, Ayeri connects relative clauses to main clauses with the
relativizer \rayr{si}{si}. This relativizer can be declined for case\index{case} in
accordance with the relative clause's head in the matrix clause. The respective
forms can be gathered from \autoref{tab:relpro} (column `Pronoun').

\begin{figure}[h]
\pex
\a\label{ex:n-rel}\begingl
	\gla Eryyo tarela natrangās si tado. //
	\glb ery-yo tarela natranga-as si tado //
	\glc use-\TsgN{} still temple-\Parg{} \Rel{} old //
	\glft `The temple, which is old, is still being used.' //
\endgl

\a\label{ex:n-adj-rel}\begingl
	\gla Edanyāng ayonas sirtang sas ang @ sihabaya mondoas nana. //
	\glb edanya-ang ayon-as sirtang si-as ang= sihaba=ya mondo-as nana //
	\glc this-\Aarg{} man-\Parg{} young \Rel{}-\Parg{} 
		\AgtT{}= tend=\TsgM{}.\Top{} garden-\Parg{} \Fpl{}.\Gen{} //
	\glft `This is the young man who tends our garden.' //
\endgl
\xe
\end{figure}

As explained in \autoref{sec:markstrat}, if the relativizer is immediately 
following\index{word order} its lexical head, only the base form \rayr{si}{si} is used, which is 
illustrated in (\ref{ex:n-rel}). Here, the head of the relative clause is
\xayr{ntFrNaasF}{natrangās}{the temple}, which is immediately followed by the
relative clause. If word material is intervening, however, as in
(\ref{ex:n-adj-rel}), the relative pronoun may be inflected to agree\index{agreement} in case\index{case}
with its antecedent in more formal language for referential clarity:
\rayr{ssF}{sas} agrees\index{agreement} in case\index{case} with \rayr{AyonsF}{ayonas} two words over to the
left. Relative pronouns do not agree\index{agreement} in number\index{number} with their heads, though, and in
gender\index{gender} only insofar as it is relevant to nominal case\index{case} inflection, that is,
agents and patients are distinguished for animacy\index{animacy}.

A special property of the relative pronoun is that it can be declined for its 
role in the relative clause as well to express more complex relationships 
between the main clause and the relative clause. The respective forms can be 
found in the columns titled `pronoun with secondary inflection' in 
\autoref{tab:relpro}. The token frequency of the actually occurring complex 
relative pronouns in the very small corpus gathered from example texts and 
dictionary entries (see \autoref{sec:phonotactics}) is given in 
\autoref{tab:relprotokenfreq}.

\begin{table}[tp]\centering
\caption{Token frequencies of attested complex relative pronouns}

\begin{tabu} to .75\linewidth {>{\itshape}X[2l] X[2l] X[1c]}
\tableheaderfont\toprule

Pronoun & Gloss & Frequency \\

\toprule

siyā	& \Rel{}.Ø.\Loc{} & 7 \\
sirī	& \Rel{}.Ø.\Ins{} & 3 \\
sinā	& \Rel{}.Ø.\Gen{} & 1 \\
siyām	& \Rel{}.Ø.\Dat{} & 1 \\

\bottomrule

\textup{Total}	& & 12 \\

\bottomrule
\end{tabu}
\label{tab:relprotokenfreq}
\end{table}

Compared to the unmarked relativizer \rayr{si}{si}, which occurs 50 times in
the sample (out of 80), the complex relative pronouns have a very low
frequency. This is not surprising, since `for whom', `by which', etc.\ are
quite specialized expressions. It also seems that those forms unmarked for
their antecedent are preferred, since these are the only ones attested. The
sample is really much too small to make actually meaningful judgments here,
however. Complex relative pronouns are illustrated in (\ref{ex:relcompl}).
Importantly, a complex relative pronoun cannot form the topic\index{grammatical function!topic} of the relative
clause even though it is marked for case\index{case} according to the relative clause's
syntactic domain. Furthermore, the relative pronoun cannot receive inflection
for an agent\index{case!agent} or a patient\index{case!patient} of the embedded clause. Compare (\ref{ex:reltop}) to
(\ref{ex:relpat}) for examples.

\begin{figure}
\pex\label{ex:relcompl}
\a\begingl[glspace=.33em]
	\gla Le @ vacyang koya sileyya ang @ layāy adanyana. //
	\glb le= vac=yang koya-Ø si-ley-ya ang= laya=ay.Ø adanya-na //
	\glc \PatTI{}= like=\Fsg{}.\Aarg{} book-\Top{} \Rel{}-\PargI{}-\Loc{}
	\AgtT{}= read=\Fsg{}.\Top{} that-\Gen{} //
	\glft `I like the book in which I read about it.' //
\endgl

\a\label{ex:reldat}\begingl
	\gla Ya @ saratang yano siyām sarasatang. //
	\glb ya= sara=tang yano-Ø si-Ø-yām sara-asa=tang //
	\glc \LocT{}= go=\TplM{}.\Aarg{} place-\Top{} \Rel{}-\Loc{}-\Dat{} 
		go-\Hab{}=\TplM{}.\Aarg{} //
	\glft `They went to the place to which they always went.' //
\endgl
\xe
\end{figure}

\begin{figure}
\ex\label{ex:reltop}
% \a\begingl
\ljudge* \begingl
	\gla Mica edaya sobayāng {si \textup{(\ques{}\textit{sī})}} na @ ihayang 
		koyaley. //
	\glb mit-ya edaya sobaya-ang si-Ø-Ø na= iha=yang koya-ley //
	\glc live-\TsgM{} here teacher-\Aarg{} \Rel{}-\Aarg{}-\Top{} \GenT{}= 
		borrow=\Fsg{}.\Aarg{} book-\PargI{} //
% \endgl
% 
% \a\begingl
% 	\gla Mica edaya sobayāng sinā ang ihāy koyaley. //
% 	\glb mit-ya edaya sobaya-ang si-Ø-nā ang iha=ay.Ø koya-ley //
% 	\glc live-\TsgM{} here teacher-\Aarg{} \Rel{}-\Aarg{}-\Gen{} \AgtT{} 
% 		borrow=\Fsg{}.\Top{} book-\PargI{} //
	\glft `Here lives the teacher from whom I borrowed a book.' //
\endgl
\xe
\end{figure}

\begin{figure}
\ex\label{ex:relagt}
% \a\begingl
\ljudge* \begingl
	\gla Mica edaya sobayāng sāng le @ sobya payutān yām. //
	\glb mit-ya edaya sobaya-ang si-Ø-ang le= sob-ya payutān-Ø yām //
	\glc live-\TsgM{} here teacher-\Aarg{} \Rel{}-\Aarg{}-\Aarg{} \PatTI{}= 
		teach-\TsgM{} math-\Top{} \Fsg{}.\Dat{} //
% \endgl
% 
% \a\begingl
% 	\gla Mica edaya sobayāng si le sobyāng payutān yām. //
% 	\glb mit-ya edaya sobaya-ang si le sob=yāng payutān-Ø yām //
% 	\glc live-\TsgM{} here teacher-\Aarg{} \Rel{} \PatTI{} 
% 		teach=\TsgM{}.\Aarg{} math-\Top{} \Fsg{}.\Dat{} //
	\glft `Here lives the teacher who taught me math.' //
\endgl
\xe
\end{figure}

Example (\ref{ex:reltop}) shows a sentence in which the relative pronoun, 
ungrammatically, forms the controller of topic\index{grammatical function!topic} agreement\index{agreement} on the verb\index{verbs} in the 
relative clause: \rayr{n}{na} as a genitive topic\index{grammatical function!topic} is supposed to refer to 
\xayr{sobyaaNF}{sobayāng}{teacher} in the matrix clause by way of the
relativizer \rayr{si}{si}. This relativizer would then necessarily carry a
zero-morpheme topic\index{grammatical function!topic} marker. There is no resumptive pronoun\index{pronouns!resumptive} in the relative
clause, however, so the relativizer itself forms the anaphora in the relative
clause referring to the relativized argument in the matrix clause. This is not
possible.

In (\ref{ex:relagt}), the relative pronoun *\rayr{saaNF}{*sāng} carries no
overt case\index{case} agreement\index{agreement} since it follows\index{word order} its antecedent (*\rayr{sNNF}{*sangang}
otherwise)---the long vowel identifies it as the agent\index{case!agent} of the relative clause;
the verb\index{verbs} agrees\index{agreement} accordingly. There is no resumptive agent pronoun here either,
so the relative pronoun stands in for the agent NP\index{phrase types!noun phrase} that would be necessary if
the relative clause were an independent sentence. Using a relative pronoun as
an agent-NP\index{phrase types!noun phrase} replacement in this sentence is likewise ungrammatical, though, and
so is verb\index{verbs} agreement\index{agreement} with the declined relative pronoun. Similarly, in
(\ref{ex:relpat}), the relative pronoun carries case\index{case} marking for the patient\index{case!patient} of
the relative clause, since the agent\index{semantic role!agent} of the matrix clause serves as the patient\index{semantic role!patient}
NP\index{phrase types!noun phrase} of the embedded clause. This is not grammatical either.

\begin{figure}
\ex\label{ex:relpat}
% \a\begingl
\ljudge* \begingl
	\gla Mica edaya sobayāng sās ya @ kradasayang kardang. //
	\glb mit-ya edaya sobaya-ang si-Ø-as ya= krad-asa=yang kardang-Ø //
	\glc live-\TsgM{} here teacher-\Aarg{} \Rel{}-\Aarg{}-\Parg{} \LocT{}=
		hate-\Hab{}=\Fsg{}.\Aarg{} school-\Top{} //
% \endgl
% 
% \a\begingl
% 	\gla Mica edaya sobayāng si ya kradasayang (yas) kardang. //
% 	\glb mit-ya edaya sobaya-ang si ya krad-asa=yang (yas) kardang-Ø //
% 	\glc live-\TsgM{} here teacher-\Aarg{} \Rel{} \LocT{}
% 		hate-\Hab{}=\Fsg{}.\Aarg{} (\TsgM{}.\Parg{}) school-\Top{} //
	\glft `Here lives the teacher whom I used to hate in school.' //
\endgl
\xe
\end{figure}

Altogether, it seems that in Ayeri, core arguments of intransitive\index{verbs!intransitive} and
transitive\index{verbs!transitive} clauses---agents\index{semantic role!agent} and patients\index{semantic role!patient}---cannot precede\index{word order} the embedded verb\index{verbs} of
a relative clause; the verb\index{verbs} firmly forms the head of the embedded clause in
this regard. The relative pronoun also cannot receive secondary marking for
agents\index{case!agent} or patients\index{semantic role!patient}, and neither can it stand in directly as the agent\index{case!agent} and
patient\index{semantic role!patient}\index{case!patient} NP\index{phrase types!noun phrase} of the relative clause, respectively. It is interesting in this
regard that Ayeri \emph{does} allow this for recipients\index{semantic role!recipient}, however, maybe since
by their nature as goals\index{semantic role!goal} they carry something of a locative\index{case!locative} connotation
(compare (\ref{ex:reldat})) and are thus less tightly integrated with verbs\index{verbs},
occupying a middle ground between core arguments and adverbials like the
locative\index{case!locative} proper.\footnote{This would be interesting to explore in terms of
grammaticalization\index{grammaticalization}, since it is possible that this behavior reflects a stage of
the language before \rayr{/ymF}{-yam} had been grammaticalized as the dative\index{case!dative}
marker. In this respect, it would as well be necessary to explore whether the
similarity between the dative\index{case!dative} marker \rayr{/ymF}{-yam} and the locative\index{case!locative} marker
\rayr{/y}{-ya} is indeed etymological or merely incidental.}

\index{relative clause|)}
\index{pronouns!relative|)}

\subsection{Reflexives and reciprocals}
\label{subsec:reflrec}
\index{pronouns!reflexive|(}

As mentioned previously, Ayeri forms its reflexives with the prefix\index{prefixes} 
\rayr{sitNF/}{sitang-} in combination with a personal pronoun\index{pronouns!personal}, compare 
(\ref{ex:reflpat}). If the agent\index{semantic role!agent} of the action is the same as the reflexive 
patient\index{semantic role!patient}---that is, the agent\index{semantic role!agent} acts on itself---the reflexive prefix\index{prefixes} can also 
migrate onto the verb\index{verbs} instead, which is demonstrated in (\ref{ex:reflvb}).

\begin{figure}
\ex\label{ex:reflpat}\begingl
	\gla Ang @ silvye sitang-yes puluyya. //
	\glb ang= silv=ye.Ø sitang=yes puluy-ya //
	\glc \AgtT{}= see=\TsgF{}.\Top{} self=\TsgF{}.\Parg{} mirror-\Loc{} //
	\glft `She sees herself in the mirror.' //
\endgl\xe
\end{figure}

Doing the same with a non-patient\index{semantic role!patient} pronoun does not work, however. Thus, the
sentence in (\ref{ex:reflvb}) with the reflexive \rayr{sitNF/}{sitang} marked
on the verb\index{verbs} is not equivalent to the one in (\ref{ex:reflloc}). Here,
\rayr{sitNF/}{sitang-} appears together with a personal pronoun\index{pronouns!personal} in the locative
case\index{case!locative}, even though here as well, the agent\index{semantic role!agent} and the locative pronoun refer to the
same entity. It may be noted furthermore that the genitive\index{case!genitive}/possessive pronoun\index{pronouns!possessive}
series conveys the meaning of `one's own', which is completely regular in
meaning (`of X-self'), compare (\ref{ex:emphposs}).

\begin{figure}
\ex\label{ex:reflvb}\begingl
	\gla Ang @ sitang-silvye puluyya. //
	\glb ang= sitang=silv=ye.Ø puluy-ya //
	\glc \AgtT{}= self=see=\TsgF{}.\Top{} mirror-\Loc{} //
	\glft `She sees herself in the mirror.' //
\endgl\xe
\end{figure}

\begin{figure}
\ex\label{ex:reflloc}\begingl
	\gla Ang @ silvye sitang-yea puluyya. //
	\glb ang= silv=ye.Ø sitang=yea puluy-ya //
	\glc \AgtT{}= look=\TsgF{}.\Top{} self=\TsgF{}.\Loc{} mirror-\Loc{} //
	\glft `She looks at herself in the mirror.' //
\endgl\xe
\end{figure}

\begin{figure}
\ex\label{ex:emphposs}%
\begingl
	\gla Le @ no @ eryongyang pakay sitang-nā. //
	\glb le= no= ery-ong=yang pakay-Ø sitang=nā //
	\glc \PatTI{}= want= use-\Irr{}=\Fsg{}.\Aarg{} umbrella-\Top{} 
		self=\Fsg{}.\Gen{} //
	\glft `I'd like to use my own umbrella.' //
\endgl\xe
\end{figure}

\rayr{sitNF}{sitang} is also used to carry quantifiers\index{quantifiers} referring to a
pronominal suffix\index{suffixes} as in (\ref{ex:sitangquant}). Appending a quantifier\index{quantifiers} directly
to the conjugated verb\index{verbs} itself can be ambiguous; compare
(\ref{ex:nositangquant}). It appears that \rayr{sitNF}{sitang} does not act as
the controller of the verbal\index{verbs} topic\index{grammatical function!topic} marker, however. This is illustrated also by
the ability of \rayr{sitNF}{sitang} and a non-topic\index{grammatical function!topic} agent\index{semantic role!agent} pronominal suffix\index{suffixes} to
appear side by side, as in (\ref{ex:sitangnotop}). For an analysis from the
point of view of syntax, refer to \autoref{subsec:expsitang}. As described
previously, lexical NPs\index{phrase types!noun phrase} and pronominal suffixes\index{suffixes} on the verb\index{verbs} are mutually
exclusive; see \autoref{clitics_postverb_person}
(p.~\pageref{clitics_postverb_person}). The correct answer to the question,
\xayr{ANF koronFy sinY gurtnF?}{Ang koronya sinya guratan?}{Who knows the
answer?}, is \xayr{yNF/nYm}{Yang-nyama}{Even I}, with the quantifier\index{quantifiers} clitic\index{clitics}
leaning on the free pronoun directly, however, since there is no referential
ambiguity in this. Introducing an adverb shows that the reflexive--quantifier\index{quantifiers}
compound follows\index{word order} the conjugated verb\index{verbs} and its adjuncts\index{grammatical function!adjunct}, as in
(\ref{ex:sitangqtorder}).

\begin{figure}
\pex
\a\label{ex:sitangquant}\begingl
	\gla Ang @ koronay sitang-nyama guratanley. //
	\glb ang= koron=ay.Ø sitang=nyama guratan-ley //
	\glc \AgtT{}= know=\Fsg{}.\Top{} self=even answer-\PargI{} //
	\glft `Even I know the answer.' //
\endgl

\a\label{ex:nositangquant}\ljudge\excl\begingl
	\gla Ang @ koronay-nyama guratanley. //
	\glb ang= koron=ay.Ø=nyama guratan-ley //
	\glc \AgtT{}= know=\Fsg{}.\Top{}=even answer-\PargI{} //
	\glft `I even know the answer.' \\
		\textit{Intended:} `Even I know the answer.' //
\endgl
\xe
\end{figure}

\begin{figure}
\ex\label{ex:sitangnotop}%
\begingl
	\gla Le @ koronyang sitang-nyama guratan. //
	\glb le= koron=yang sitang=nyama guratan-Ø //
	\glc \PatTI{}= know=\Fsg{}.\Aarg{} self=even answer-\Top{} //
	\glft `The answer, even I know it.' //
\endgl\xe
\end{figure}

\begin{figure}
\ex\label{ex:sitangqtorder}%
\begingl
	\gla Nimpyāng para-ma sitang-nama. //
	\glb nimp=yāng para=ma sitang=nama //
	\glc run=\TsgM{}.\Aarg{} quick=enough self=only //
	\glft `Only he is running quickly enough.' //
\endgl\xe
\end{figure}

\index{pronouns!reflexive|)}
\index{pronouns!reciprocal|(}

Besides reflexive pronouns, Ayeri also has a reciprocal pronoun,
\xayr{sitnY}{sitanya}{each other}. This pronoun acts the same as other pronouns
and can be inflected according to its function in the clause, as
(\ref{ex:recippro}) shows.

\begin{figure}
\pex\label{ex:recippro}
\a\begingl
	\gla Ang @ narayan {} @ Ajān nay Pila sitanyaya. //
	\glb ang= nara-yan Ø= Ajān nay Pila sitanya-ya //
	\glc \AgtT{}= talk-\TplM{} \Top{}= Ajān and Pila each.other-\Loc{} //
	\glft `Ajān and Pila talk to each other.' //
\endgl

\a\begingl
	\gla Sa @ ming @ tangtang sitanya. //
	\glb sa= ming= tang=tang sitanya-Ø //
	\glc \PatT{}= can= hear=\TplM{}.\Aarg{} each.other-\Top{} //
	\glft `They can hear each other.' //
\endgl
\xe
\end{figure}

\index{pronouns!reciprocal|)}
\index{pronouns|)}

\section{Adjectives}
\label{sec:adjectives}
\index{adjectives|(}

Adjectives are one of the parts of speech in Ayeri which do not inflect for any
of the grammatical properties of their heads, that is, there is no agreement\index{agreement}
relation between adjectives and nominal heads. They do inflect for comparison\index{comparison}
under certain circumstances, however, and can also take various affixes that
modify the meaning of the adjective stem.

\subsection{Comparison}
\label{subsec:adjcomp}
\index{comparison|(}

In cases where a comparee is left unexpressed or the patient\index{case!patient} forms the standard
of comparison, Ayeri uses clitic\index{clitics} suffixes\index{suffixes} on adjectives. The suffixes\index{suffixes} involved
are \rayr{/ENF}{-eng} (\Comp{}) and \rayr{/vaa}{-vā} (\Supl{}). Adjective
comparison is exemplified in (\ref{ex:sfxcomp}). In (\ref{ex:sfxcomp2}) the
comparee is missing, while in (\ref{ex:sfxcomp1}), the quality under
comparison, \xayr{tiNrti\_asF bnF/ENF}{tingracas ban-eng}{a better musician},
is a patient\index{case!patient} NP\index{phrase types!noun phrase}; the standard, \rayr{mh}{Maha}, is expressed by an adverbial
genitive\index{case!genitive} NP\index{phrase types!noun phrase}. The example in (\ref{ex:sfxsupl}) similarly expresses a quality
without a group of referents to compare to. In all these cases, it is also
possible, however, to use a more complex analytic construction using verbs\index{verbs!comparative}
(compare \autoref{subsec:exs}).

\begin{figure}[h]
\pex\label{ex:sfxcomp}
\a\label{ex:sfxcomp2}\begingl
	\gla Yeng ganyena men si alingo-eng. //
	\glb yeng gan-ye-na men si alingo=eng //
	\glc \TsgF{}.\Aarg{} child-\Pl{}-\Gen{} one \Rel{} clever=\Comp{} //
	\glft `She is one of the more clever children.' //
\endgl

\a\label{ex:sfxcomp1}\begingl
	\gla Ang @ tavya {} @ Diyan tingracas ban-eng na @ Maha. //
	\glb ang= tav-ya Ø= Diyan tingrati-as ban=eng na= Maha //
	\glc \AgtT{}= become-\TsgM{} \Top{}= Diyan musician-\Parg{} good=\Comp{} 
		\Gen{}= Maha //
	\glft `Diyan became a better musician than Maha.' //
\endgl

%%% Try to express this with the analytic construction with eng-, and you'll 
%%% fail?! > Diyan became a better musician than Maha---only possible with a 
%%% relative clause: Diyan became a musician who is better than Maha. (Sa 
%%% tavya ang Diyan tingrati si ang engya ban sa Maha)

\a\label{ex:sfxsupl}\begingl
	\gla Naratang, yāng pokamayās para-vā. //
	\glb nara=tang yāng pokamaya-as para-vā //
	\glc say=\TplM{}.\Aarg{} \TsgM.\Aarg{} shooter-\Parg{} fast=\Supl{} //
	\glft `They said he is the fastest shooter.' //
\endgl
\xe
\end{figure}

\index{comparison|)}

\subsection{Negation}
\label{subsec:adjneg}
\index{negation|(}

Adjectives in Ayeri can be negated in two ways: categorially with 
\rayr{/ArY}{-arya}, and pragmatically with \rayr{/Oj}{-oy}. These correspond to
English\index{English} \fw{un-}, and \fw{in-}, \fw{il-}, \fw{ir-}, etc.\ for categorial
negation, and to \fw{not} for pragmatic negation. \rayr{/Oj}{-oy} absorbs the
vowel of the root it is attached to if said root ends in a vowel.

\begin{figure}[h]
\ex\label{ex:adjarya}\begingl
	\gla Telbaya miseryanang ku-ardārya. //
	\glb telba-ya miseryan-ang ku=arda-arya //
	\glc show-\TsgM{} method-\Aarg{} like=suitable-\Neg{} //
	\glft `The method proved unsuitable.' //
\endgl\xe
\end{figure}

\begin{figure}[h]
\ex\label{ex:adjoy}\begingl
	\gla Pakoy eda-yanoreng. //
	\glb paka-oy eda=yano-reng //
	\glc safe-\Neg{} this=place-\AargI{} //
	\glft `This place is not safe.' //
\endgl\xe
\end{figure}

Example (\ref{ex:adjarya}) displays an adjective which carries the categorial
negation marker \rayr{/ArY}{-arya}; the adjective in (\ref{ex:adjoy}) carries
the simple, pragmatic negation marker \rayr{/Oj}{\mbox{-oy}}. Which one to use
is up to the speaker, since both negate the described property. The categorial
marker puts an emphasis more on expressing a general opposite, while the
pragmatic marker simply negates, so that it is not necessarily implied that the
negative state persists. The place that is \xayr{pkoj}{pakoy}{not safe} now is
not necessarily \xayr{pkaarY}{pakārya}{unsafe} in general, but simply not safe
in the context of the here and now of the utterance.

Besides \fw{ad hoc} derivation\index{derivation} of categorial negatives with \rayr{/ArY}{-arya},
there are also a few lexicalized instances. These have an idiomatic meaning and
the negator or the word itself may be irregularly reduced. A few examples are
listed in (\ref{ex:adjnegstrats}).

\begin{figure}[h]
\ex\labels\label{ex:adjnegstrats}
	\begin{tabular}[t]{@{\tl\quad} l @{\enspace→\enspace} l @{\smallskip}}
	\xayr{\larger bnF}{ban}{good}
		& \xayr{\larger bny}{banaya}{ill, sick}
		\\
	\xayr{\larger kovro}{kovaro}{easy}
		& \xayr{\larger kovrY}{kovarya}{awkward}
		\\
	\xayr{\larger sirimNF}{sirimang}{straight}
		& \xayr{\larger sirimy}{sirimaya}{passive}
		\\
	\end{tabular}
\xe
\end{figure}

\index{negation|)}

\subsection{Adjectivization}
\index{derivation|(}

Adjectives in Ayeri are very commonly zero derivations, that is, there is
rather free conversion between nouns and adjectives,\footnote{Adjectives and
split-off modifiers in noun--noun compounds\index{compounds} are thus similar at least
superficially (compare \autoref{subsubsec:endocomp}).} compare
(\ref{ex:adjzeroderiv}).

\begin{figure}[h]
\ex\labels\label{ex:adjzeroderiv}
	\begin{tabular}[t]{@{\tl\quad} l @{\enspace\til\enspace} l 
		@{\smallskip}}
	\xayr{\larger Ayeri}{Ayeri}{Ayeri (n.)}
		& \xayr{\larger Ayeri}{Ayeri}{Ayeri (adj.)}
		\\
	\xayr{\larger dis}{disa}{soap, lye}
		& \xayr{\larger dis}{disa}{soapy, alkaline}
		\\
	\xayr{\larger gino}{gino}{drink}
		& \xayr{\larger gino}{gino}{drunk}
		\\
	\xayr{\larger phmj}{pahamay}{danger}
		& \xayr{\larger phmj}{pahamay}{dangerous}
		\\
	\xayr{\larger seMpj}{sempay}{peace}
		& \xayr{\larger seMpj}{sempay}{peaceful}
		\\
	\end{tabular}
\xe
\end{figure}

Adjectives can also be derived from verbs\index{verbs} with the causative\index{case!causative} suffix\index{suffixes} 
\rayr{/Is}{-isa}, which often corresponds to adjectives derived from the 
past participle form in English\index{English}---the meaning is often, but not necessarily,
relating to an achieved state. The suffix\index{suffixes} may change the last vowel to
\rayr{U}{u} or drop it; a specific pattern to these changes is not
recognizable. The derivations may be idiomatic occasionally, as some
derivations in (\ref{ex:adjderiv}) show.

\begin{figure}
\ex\labels\label{ex:adjderiv}
	\begin{tabular}[t]{@{\tl\quad} l @{\enspace→\enspace} l @{\smallskip}}
	\xayr{\larger kelNF/}{kelang-}{connect}
		& \xayr{\larger kelNisu}{kelangisu}{connected, related}
		\\
	\xayr{\larger pluNF/}{palung-}{distinguish}
		& \xayr{\larger pluNis}{palungisa}{various}
		\\
	\xayr{\larger suMdl/}{sundala-}{lose}
		& \xayr{\larger suMdlisu}{sundalisu}{lost}
		\\
	\xayr{\larger thnF/}{tahan-}{write}
		& \xayr{\larger thnisF}{tahanis}{literary}
		\\
	\xayr{\larger ves/}{vesa-}{give birth}
		& \xayr{\larger vesis}{vesisa}{native}
		\\
	\end{tabular}
\xe
\end{figure}

There are also at least two words where an \rayr{/Is}{-isa} adjective is
derived not from a verb, but from a word of a different part of speech---in
this case, a noun, and another adjective. These are given in
(\ref{ex:isaotherpos}).

\begin{figure}[h]
\ex\labels\label{ex:isaotherpos}
	\begin{tabular}[t]{@{\tl\quad} l @{\enspace→\enspace} l @{\smallskip}}
	\xayr{\larger ApinF}{apin}{luck}
		& \xayr{\larger Apinis}{apinisa}{lucky}
		\\
	\xayr{\larger Irj}{iray}{high}
		& \xayr{\larger Iryisu}{irayisu}{exalting}
		\\
	\end{tabular}
\xe
\end{figure}

\index{derivation|)}

\subsection{Other affixes}
\label{subsec:adjaffx}

As with nouns, other affixes which can be attached to adjectives as clitic\index{clitics}
hosts, are the prefix\index{prefixes} \rayr{ku/}{ku-}, expressing semblance as in
(\ref{ex:adjsembl}), as well as quantifying\index{quantifiers} and grading suffixes\index{suffixes}, of which the
suffixes\index{suffixes} used to express comparative and superlative\index{comparison} are, essentially, a
grammaticalized variety, since \rayr{/ENF}{-eng} can also be used like `rather'
as in (\ref{ex:adjquant}).

\begin{figure}[h]
\ex\label{ex:adjsembl}\begingl
	\gla Ku-pikisu paray-parayang. //
	\glb ku=pikisu paray\til{}paray-ang //
	\glc like=scared \Dim{}\til{}cat-\Aarg{} //
	\glft `The kitten is like scared.' //
\endgl\xe
\end{figure}

\begin{figure}[h]
\ex\label{ex:adjquant}\begingl
	\gla Napay-eng eda-prikanreng. //
	\glb napay=eng eda=prikan-reng //
	\glc spicy=rather this=soup-\AargI{} //
	\glft `This soup is rather spicy.' //
\endgl\xe
\end{figure}

\index{adjectives|)}


\section{Adpositions}
\label{sec:adpositions}
\index{adpositions|(}
\index{semantic role!location|(}

Adpositions are another part of speech in Ayeri whose stem itself does not 
inflect. Ayeri's most basic adpositions are derived from relational nouns, 
which is likely the reason why Ayeri mostly employs prepositions\index{adpositions!prepositions}, with 
postpositions and ambipositions being less important placement patterns 
\parencites[110--111]{hagege2010}[81\psqq]{lehmann2015}. Adpositions in their 
most basic use trigger locative marking on the governed NP\index{phrase types!noun phrase}, the adpositional
object\index{grammatical function!adpositional object}.\footnote{For allative and ablative meanings, an NP may also
appear in the dative and the genitive, respectively, though without being governed by an adposition, as described in
\autoref{subsubsec:dative}. Also compare \autoref{phsec:lingavandat}, p.~\pageref{phsec:lingavandat}~ff.}
The metaphor \textsc{time equals space} with the future\index{tense!future} conceptualized 
as lying ahead and the past\index{tense!past} behind also holds in Ayeri,
so that some of the words describing locations also double to describe temporal
relations.

\subsection{Prepositions}
\label{subsec:prepositions}
\index{adpositions!prepositions|(}

\begin{table}\centering
\caption{Prepositions (simple)}
\begin{tabu} to \linewidth {>{\itshape}l l X}
\tableheaderfont\toprule
\multicolumn{2}{c}{Preposition}
	& Etymology (or related to)
	\\

\toprule

agonan
	& `outside'
	& \tayr{agonan}{outside}
	\\
	
avan
	& `bottom, ground'
	& \tayr{avan}{ground, bottom; soil}
	\\

% dayrin
% 	& `side'
% 	& \tayr{dayrin}{waist}
% 	\\

eyran
	& `under, below'
	& \tayr{eyran}{sole}
	\\

eyrarya
	& `over'
	& \tayr{eyran}{sole} + \fw{-arya} (\Neg{})
	\\

kayvo
	& `with, beside'\footnotemark
	& \tayr{kayv-}{accompany}
	\\

kong
	& `inside, within'
	& \tayr{kong}{inside}
	\\
	
ling
	& `on'
	& \tayr{ling}{top}
	\\

luga
	& `among, between'
	& \tayr{luga-}{pass, penetrate}
	\\

mangasaha
	& `towards, in\,+\,\emph{time}'
	& \tayr{manga saha-}{coming}
	\\

mangasara
	& `away'
	& \tayr{manga sara-}{going}
	\\

marin
	& `front, on (walls etc.)'
	& \tayr{marin}{face, surface}
	\\

miday
	& `around'
	& \tayr{miday-}{surround}
	\\

nasay
	& `near, close'
	& \tayr{nasay}{proximity}
	\\

nuveng
	& `left'
	& \tayr{nuho}{liver}
	\\

pang
	& `behind, ago'
	& \tayr{pang}{back}
	\\

patameng
	& `right'
	& \tayr{patam}{heart}
	\\

\bottomrule
\end{tabu}

\label{tab:prepos}
\end{table}

\footnotetext{There is also a preposition \xayr{djrin}{dayrin}{side} listed in 
the dictionary, however, this has never seen much use. Instead, 
\rayr{kjvo}{kayvo} has come to cover `beside, to the side of' as well.}

\autoref{tab:prepos} gives all the words in Ayeri which may be used as
prepositions. As mentioned above, most of these are derived\index{derivation} transparently from
nouns, so they have probably been grammaticalized relatively recently---their
non-preposition meaning is still transparent, they are still phonologically
rather complex, and some of them are even polysyllabic in spite of not being
composed and covering rather basic meanings.\footnote{Unsurprisingly,
\citet[129]{hagege2010} references Zipf regarding speech economy and token
frequency. According to \citet[134--141]{lehmann2015}, the phonological
integrity of morphemic units reduces as grammaticalization\index{grammaticalization} is progressing (with
token frequency increasing due to increasing obligatoriness).
\citet{bybeehopper2001b} see the reason for phonological reduction of highly
frequent phonological material \textcquote[11]{bybeehopper2001b}{in the
automatization of neuro-motor sequences […]. Such reductions are systematic
across speakers; that is, they do not represent \enquote{sloppy} or
\enquote{lazy} speech}. Hence, for example, English's\index{English} most basic prepositions
are extremely short and simple words, for instance, \fw{of, at, in}, which
derive from the slightly more complex PIE forms \fw{*h₂ep-ó}, \fw{*h₂ed},
\fw{*h₁en(-i)}, respectively \citep[1, 39, 269]{kroonen2013}. Since adpositions
frequently grammaticalize into case\index{case} markers, it may be assumed that the
phonologically much more simple case\index{case} affixes of Ayeri constitute an older layer
of basic adpositions. Their non-suffixed\index{suffixes} forms may be remnants of this use.}
Since these nouns have ceased to function as common nouns\index{nouns!common} in this context due
to grammaticalization\index{grammaticalization}, however, it is not possible to inflect them in the way
described in \autoref{sec:nouns}. Thus, for example, while (\ref{ex:lingnn}) is
grammatical, (\ref{ex:lingpr}) is not. Instead, the grammatical way to express
(\ref{ex:lingpr}) is given in (\ref{ex:lingpp}), using \rayr{liNF}{ling} as a
preposition with the adpositional object\index{grammatical function!adpositional object} in the locative case. In this case,
since \emph{on} is the expected position of sitting with regards to chairs, the
preposition can even be dropped, as in (\ref{ex:prepunderspec}).

\begin{figure}[h]
\pex
\a\label{ex:lingnn}\begingl
	\gla Le @ yomareng kanka lingya rivanena. //
	\glb le= yoma=reng kanka-Ø ling-ya rivan-ena //
	\glc \PatTI{}= exist=\TsgI{}.\Aarg{} snow-\Top{} top-\Loc{}
		mountain-\Gen{} //
	\glft `There is snow on the top of the mountain.'\footnotemark //
\endgl

\a\label{ex:lingpr}\ljudge* \begingl
	\gla Ang @ nedraye lingya nedrānena. //
	\glb ang= nedra=ye.Ø ling-ya nedrān-na //
	\glc \AgtT{}= sit=\TsgF{}.\Top{} top-\Loc{} chair-\Gen{} //
	\glft `\ques{}She sits on the top of a chair.' //
\endgl
\xe
\end{figure}

\footnotetext{The corresponding sentence with a preposition is \xayr{le yomreNF
kMk liNF rivnFy}{Le yomareng kanka ling rivanya}{There is snow on top of the
mountain} (\PatTI{}=exist=\TsgI{}.\Aarg{} snow-\Top{} top mountain-\Loc{}).}

\begin{figure}[h]
\pex
\a\label{ex:lingpp}\begingl
	\gla Ang @ nedraye ling nedrānya. //
	\glb ang= nedra=ye.Ø ling nedrān-ya //
	\glc \AgtT{}= sit=\TsgF{}.\Top{} top chair-\Loc{} //
	\glft `She sits on a chair.' //
\endgl

\a\label{ex:prepunderspec}%
\begingl
	\gla Ang @ nedraye nedrānya. //
	\glb ang= nedra=ye.Ø nedrān-ya //
	\glc \AgtT{} sit=\TsgF{}.\Top{} chair-\Loc{} //
	\glft `She sits on a chair.' //
\endgl\xe
\end{figure}

With regards to (\ref{ex:lingnn}) it is also necessary to mention what
\citet{hagege2010} calls the `Proof by Anachrony Principle'
\citep[158--159]{hagege2010}. According to this principle, when an adposition
is very grammaticalized, speakers can use both the adposition and its
etymological ancestor side by side without taking offense in the double
occurrence. This is notably not the case in Ayeri, where something like
(\ref{ex:behindback1}) is not possible. Here \rayr{pNF}{pang} is used in both
its meanings so that the preposition \xayr{pNF}{pang}{behind} governs the
original noun \xayr{pNF}{pang}{back}.

\begin{figure}[h]
\pex
\a\label{ex:behindback1}\ljudge* \begingl
	\gla Le @ ranice ang @ Maha adanya pang pangya yena. //
	\glb le= ranit-ye ang= Maha adanya-Ø pang pang-ya yena //
	\glc \PatTI{}= hide-\TsgF{} \Aarg{}= Maha that-\Top{} back back-\Loc{} 
		\TsgF{}.\Gen{} //
	\glft `*Maha hides it at the back of her back.' //
\endgl

\a\label{ex:behindback2}\begingl
	\gla Le @ ranice ang @ Maha adanya pangya yena. //
	\glb le= ranit-ye ang= Maha adanya-Ø pang-ya yena //
	\glc \PatTI{}= hide-\TsgF{} \Aarg{}= Maha that-\Top{} back-\Loc{} 
		\TsgF{}.\Gen{} //
	\glft `Maha hides it at her back',\\
		\textit{or:} `Maha hides it behind herself.' //
\endgl
\xe
\end{figure}

Examples like (\ref{ex:lingpr}), on the other hand, show that there is
nonetheless a tendency in Ayeri towards grammaticalization\index{grammaticalization} of nouns which used
to be relational. Grammaticalization\index{grammaticalization} is visible in that formerly relational
nouns have become restricted in the way they can be used syntactically
\citep[174]{lehmann2015}. This specialization is also apparent in morphology
from the fact that prepositions in Ayeri, in spite of their nominal origin,
cannot be modified by adjectives\index{adjectives} and relative clauses\index{relative clause} like regular nouns. Thus,
for instance, while \rayr{AvnF}{avan} as a noun can mean `soil' or `ground' and
can be modified by semantically coherent adjectives\index{adjectives} like
\xayr{kbu}{kabu}{fertile}, the preposition \rayr{AvnF}{avan} cannot. Again, in
order to express (\ref{ex:avanprep}) in a grammatical way, one would have to
use \rayr{AvnF}{avan} as a relational noun, that is, \xayr{AvnFy kbu
similen}{avanya kabu similena}{at the fertile bottom of the country}
(bottom-\Loc{} fertile country-\Gen{}). The fact that topicalized\index{grammatical function!topic} heads lack
case marking makes adpositions derived from nouns, like \rayr{AvnF}{avan}
homophonous with the respective etymologically related preposition.

\begin{figure}[h]
\pex
\a\label{ex:avannn}\begingl
	\gla Sa @ yomareng avan kabu ibangya yana. //
	\glb sa= yoma=reng avan-Ø kabu ibang-ya yana //
	\glc \PatT{}= exist=\TsgI.\AargI{} ground-\Top{} fertile field-\Loc{} 
		\TsgM{}.\Gen{} //
	\glft `Fertile ground is on his field.' //
\endgl

\a\label{ex:avanprep}\ljudge* \begingl
	\gla Ang @ mican avan kabu similya //
	\glb ang= mit=yan.Ø avan kabu simil-ya //
	\glc \AgtT{}= live=\TplM{}.\Top{} bottom fertile country-\Loc{} //
	\glft `*They live at the fertile bottom of the country.' //
\endgl
\xe
\end{figure}

At the beginning of this section it was shown that prepositions in Ayeri cannot
receive number\index{number} and case\index{case} marking, which are otherwise typical features of nouns.
What is possible with regards to affixes, however, is adding quantifier\index{quantifiers}
suffixes\index{suffixes} to prepositions, since these suffixes\index{suffixes} are clitics\index{clitics} rather than
inflections, as shown in (\ref{ex:prepquant}); also compare
\autoref{subsec:clitics}, p.~\pageref{clitics_quant}.

\begin{figure}[h]
\ex\label{ex:prepquant}\begingl
	\gla Ang @ mitasaye pang-ikan mandayya tado. //
	\glb ang= mit-asa=ye.Ø pang=ikan manday-ya tado //
	\glc \AgtT{}= live-\Hab{}=\TsgF{}.\Top{} back=much forum-\Loc{} old //
	\glft `She used to live way behind the old forum.' //
\endgl\xe
\end{figure}

\begin{table}\centering
\caption{Prepositions (directional)}
\begin{tabu} to \linewidth {>{\itshape}l X X}
\tableheaderfont\toprule
\multicolumn{2}{c}{Preposition}
	& \fw{manga} + \Prep{}
	\\

\toprule

agonan
	& `outside'
	& `out'
	\\

avan
	& `at bottom'; +\,\Dat{}: `down'
	& `to the bottom'; +\,\Dat{}: `down to'
	\\

% dayrin
%	& `beside'
% 	& `to the side'
% 	\\

eyran
	& `under'
	& `under'
	\\

eyrarya
	& `over'
	& `across, over'
	\\

kayvo
	& `with, beside'
	& `along'
	\\

kong
	& `inside'
	& `into'
	\\

ling
	& `on top'; +\,\Dat{}: `up'
	& `onto, while'; +\,\Dat{}: `up to'
	\\

luga
	& `between'
	& `through, during, for\,+\,\textit{time}'
	\\

marin
	& `in front'
	& `to the front'
	\\

miday
	& `around'
	& `circling around'
	\\

nasay
	& `near'
	& `into the near'
	\\

nuveng
	& `left'
	& `to the left'
	\\

pang
	& `behind'
	& `behind, to the back'
	\\

patameng
	& `right'
	& `to the right'
	\\

\bottomrule
\end{tabu}

\label{tab:preposdyn}
\end{table}

\phantomsection\label{manga}
As demonstrated before, another quasi-inflection adpositions in Ayeri can host
is the directional marker \rayr{mN}{manga} (see \autoref{sec:typology}). While
most of the prepositions in \autoref{tab:prepos} have a static meaning,
\rayr{mN}{manga} indicates a motion in the direction of the respective
location, thus \xayr{koNF}{kong}{inside} becomes \xayr{mN koNF}{manga
kong}{into}, for instance. \autoref{tab:preposdyn} repeats the table of
prepositions above for the most part and gives the respective directional
meanings. The prepositions \rayr{mNsh}{mangasaha} and \rayr{mNsr}{mangasara}
are missing from this list and appear in the previous table instead, even
though they express motion rather than position, because they are only used in
this base form and cannot be prefixed\index{prefixes} by \rayr{mN}{manga}, which they already
contain. Note, however, that \rayr{mNsh}{mangasaha} and \rayr{mNsr}{mangasara}
are not synonymous to an adjunct\index{grammatical function!adjunct} in the dative\index{case!dative} and the genitive case\index{case!genitive},
respectively. Rather, the prepositions add a more deliberate or literal
meaning. This is illustrated by the difference between (\ref{ex:prepdat}) and
(\ref{ex:prepmgsh}).

\begin{figure}[h]
\pex
\a\label{ex:prepdat}\begingl
	\gla Ang @ nimpay kardangyam. //
	\glb ang= nimp=ay.Ø kardang-yam //
	\glc \AgtT{}= run=\Fsg{}.\Top{} school-\Dat{} //
	\glft `I'm running to (a/the) school.' \\
		(e.g. for class, or just up to the building) //
\endgl

\a\label{ex:prepmgsh}\begingl
	\gla Ang @ nimpay mangasaha kardangya. //
	\glb ang= nimp=ay.Ø mangasaha kardang-ya //
	\glc \AgtT{}= run=\Fsg{}.\Top{} towards school-\Loc{} //
	\glft `I'm running towards (a/the) school.' \\
		(up to the building) //
\endgl
\xe
\end{figure}

% \pex
% \a\begingl
% 	\gla Ang @ lampay kardangena. //
% 	\glb ang= walk=ay.Ø kardang-ena //
% 	\glc \AgtT{}= walk=\Fsg{}.\Top{} school-\Gen{} //
% 	\glft `I'm walking from (a/the) school.' \\
% 		(e.g. home, or somewhere else from there) //
% \endgl

% \a\begingl
% 	\gla Ang @ lampay mangasara kardangya. //
% 	\glb ang= lamp=ay.Ø mangasara kardang-ya //
% 	\glc \AgtT{}= walk=\Fsg{}.\Top{} away school-\Loc{} //
% 	\glft `I'm walking away from (a/the) school.' \\
% 		(away from the building) //
% \endgl
% \xe

Also note that while Germanic languages like English make frequent use of set
expressions which combine a verb with an intransitive preposition, such as
\fw{run away, go by, raise up, track down}, sometimes with rather idiomatic
meanings, this pattern does not occur as frequently in Ayeri. Some exceptions
are listed in (\ref{ex:particleverbs}).

\begin{figure}[h]
\pex\label{ex:particleverbs}
\a \xayr{\larger IlF/ mNsr}{il- mangasara}{surrender} (give away)
\a \xayr{\larger lMtF/ mNsr}{lant- mangasara}{distract} (lead away)
\a \xayr{\larger niMpF/ mNsr}{nimp- mangasara}{escape} (run away)
\a \xayr{\larger tpY/ djrinF}{tapy- dayrin}{save (valuable assets)} (put
	aside)
\a \xayr{\larger tpY/ midj}{tapy- miday}{put on} (put around)
\a \xayr{\larger tur/ mNsh}{tura- mangasaha}{forward} (send towards)
\xe
\end{figure}

These verbs\index{verbs} do not govern a prepositional object\index{grammatical function!adpositional object} in the locative case in their
idiomatic meaning, as displayed by (\ref{ex:mgsrobjdep}), in which
\rayr{btNimnF}{batangiman} and \rayr{s AgYaanF}{sa Ajān} do neither serve as
arguments of \rayr{lMtYo}{lanco} or \rayr{mNsr}{mangasara}, but of the phrasal
verb \rayr{lMtF/ mNsr}{lant- mangasara}.\footnote{Colloquially,
\rayr{mNsh}{mangasaha} and \rayr{mNsr}{mangasara} may be shortened to just
\rayr{sh}{saha} and \rayr{sr}{sara}, respectively.}

\begin{figure}[h]
\ex\label{ex:mgsrobjdep}\begingl
	\gla Ang @ lanco mangasara batangiman sa @ Ajān. //
	\glb ang= lant-yo mangasara batangiman-Ø sa= Ajān //
	\glc \AgtT{}= lead-\TsgN{} away mosquito-\Top{} \Parg{}= Ajān //
	\glft `The mosquito distracted Ajān.' //
\endgl\xe
\end{figure}

Very often, where the verbal expression in English\index{English} contains a preposition,
there is a separate verb\index{verbs} in Ayeri, as in (\ref{ex:separatevb}), or the same
verb\index{verbs} is used in Ayeri for both the plain English\index{English} verb and the one extended by a
preposition, as in (\ref{ex:doubleusevb}). In cases where the preposition does
not have a prepositional object\index{grammatical function!adpositional object} otherwise, its double nature as a noun comes to
the fore in that the preposition word will be treated like a noun if it is
denominal and carries the appropriate case\index{case} marker itself, like
\xayr{pNFymF}{pangyam}{to the back} does in (\ref{ex:ppasnoun}).

\begin{figure}
\begin{minipage}[t]{.6\remaining}
\pex\label{ex:separatevb}
	\a \xayr{\larger ApMdF/}{apand-}{descend, climb down}
%	\a \xayr{\larger dil/}{dila-}{figure out, find out}
	\a \xayr{\larger liNF/}{ling-}{ascend, mount, climb up}
%	\a \xayr{\larger ng/}{naga-}{watch after}
	\a \xayr{\larger phF/}{pah-}{remove, take away}
%	\a \xayr{\larger subFrF/}{subr-}{cease, give up}
\xe
\end{minipage}
~
\begin{minipage}[t]{.4\remaining}
\pex\label{ex:doubleusevb}
	\a \xayr{\larger k/}{ka-}{throw (away)}
	\a \xayr{\larger mtF/}{mat-}{warm (up)}
	\a \xayr{\larger sikFlF/}{sikl-}{rip (up)}
\xe
\end{minipage}
\end{figure}

\begin{figure}
\pex
\a\begingl
	\gla Ang @ sahayan manga @ pang nangaya. //
	\glb ang= saha=yan.Ø manga= pang nanga-ya //
	\glc \AgtT{}= go=\Tpl{}.\Top{} \Dir{}= back house-\Loc{} //
	\glft `They go behind the house.' //
\endgl

\a\label{ex:ppasnoun}\begingl
	\gla Ang @ sahayan pangyam. //
	\glb ang= saha=yan.Ø pangyam //
	\glc \AgtT{}= go=\Tpl{}.\Top{} back-\Dat{} //
	\glft `They go behind (it),'\\
		\textit{or:} `They go to the back.' //
\endgl
\xe
\end{figure}

\index{adpositions!prepositions|)}

\subsection{Postpositions}
\label{subsec:postpos}
\index{adpositions!postpositions|(}

\begin{table}[p]\centering
\caption{Postpositions}
\begin{tabu} to \linewidth {>{\itshape}l l X}
\tableheaderfont\toprule
\multicolumn{2}{c}{Postposition}
	& Etymology (or related to)
	\\

\toprule

da-nārya
	& `despite, in spite of'
	& \tayr{da-}{such} + \tayr{nārya}{but}
	\\

kayvay
	& `without'
	& \tayr{kayvo}{with} + \fw{-oy} (\Neg{})
	\\

masahatay
	& `since'
	& \fw{mə-} (\Pst{}) + \tayr{saha-}{come} + \tayr{taday}{time}
	\\

nasyam
	& `according to'
	& \tayr{nasyyam}{following}
	\\

pang
	& `beyond, after, past'
	& \tayr{pang}{back}
	\\

pesan
	& `until'
	& ---
	\\

ran
	& `against'
	& \emph{possibly} \tayr{ran}{from it}
	\\

rayu
	& `diagonally across'
	& \tayr{rayu}{slanted, oblique, skewed}
	\\
	
yamva
	& `instead of'
	& ---
	\\

\bottomrule
\end{tabu}

\label{tab:postpos}
\end{table}

While Ayeri mainly uses prepositions---which is by far the most common order\index{word order}
for VO languages \citep{wals95}---it also uses a number of postpositions, which
are given in \autoref{tab:postpos}. As can be read from the table,
postpositions do not usually have a nominal origin but are derived either from
other prepositions, from adverbial phrases, or even from an adjective\index{adjectives} in the
case of \rayr{ryu}{rayu}. The etymologies of \rayr{pesnF}{pesan} and
\rayr{ymFv}{yamva} are unclear to date.

The postposition \rayr{pNF}{pang} is special in that it also exist as a
preposition\index{adpositions!prepositions} meaning `behind, in the back of', though as a postposition it
acquires the related but slightly different meaning `beyond, after, past'. It
might thus better be treated as a homonym of the preposition\index{adpositions!prepositions} rather than as an
ambiposition \citep[115]{hagege2010}. Example (\ref{ex:pangprep}) illustrates
the use of \rayr{pNF}{pang} as a preposition\index{adpositions!prepositions}, (\ref{ex:pangpost}) the use of
\rayr{pNF}{pang} as a postposition. This is in contrast to typical
ambipositions such as German\index{German} \fw{wegen} `because of, due to', 
%in (\ref{ex:wegen}),
which has the same meaning in either position and the position variant is just
a matter of style.

\begin{figure}[h]
\pex
\a\label{ex:pangprep}\begingl
	\gla Sa @ lancāng pel manga @ pang penungya. //
	\glb sa= lant=yāng pel-Ø manga= pang penung-ya //
	\glc \PatT{}= lead=\TsgM{}.\Aarg{} horse-\Top{} \Dir{}= back 
		barn-\Loc{} //
	\glft `The horse, he leads it behind the stable.' //
\endgl

\a\label{ex:pangpost}\begingl
	\gla Lesyo pelang si sā @ nimpyong penungya pang yan. //
	\glb les-yo pel-ang si sā= nimp=yong penung-ya pang yan.Ø //
	\glc fall-\TsgN{} horse-\Aarg{} \Rel{} \CauT{}= 
		run=\TsgN{}.\Aarg{} stable-\Loc{} back \Tpl{}.\Top{} //
	\glft `The horse they raced past the barn fell.' //
\endgl
\xe
\end{figure}

% \begin{figure}[h]
% \pex\label{ex:wegen}%
% German:
% \a\label{ex:wegenprep}\begingl
% 	\gla wegen des schlechten wetters //
% 	\glb wegen des schlecht-en wetter-s //
% 	\glc because.of \Def{}.\Gen{}.\N{}.\Sg{} bad-\Gen{}.\N{}.\Sg{}.\Wk{}
% 		weather-\Gen{} //
% 	\glft `because of the bad weather' //
% \endgl

% \a\label{ex:wegenpost}\begingl
% 	\gla des schlechten wetters wegen //
% 	\glb des schlecht-en wetter-s wegen //
% 	\glc \Def{}.\Gen{}.\N{}.\Sg{} bad-\Gen{}.\N{}.\Sg{}.\Wk{} weather-\Gen{} 
% 		because.of //
% 	\glft (\fw{idem}) //
% \endgl
% \xe
% \end{figure}

Besides the difference in placement, the morphological properties of 
postpositions are the same as those of prepositions\index{adpositions!prepositions}. That is, where 
postpositions are derived from nouns at all, they do not receive case\index{case} and 
number\index{number} marking and cannot themselves be modified by adjectives\index{adjectives} or relative 
clauses. Generally, it is possible for them to be hosts of quantifier\index{quantifiers} clitics\index{clitics} 
where semantics permit it.

\index{adpositions!postpositions|)}

\subsection{Adpositions and time}

\begin{table}[p]\centering
\caption{Adpositions with temporal meaning}
\begin{tabu} to \linewidth {I X X}
\tableheaderfont\toprule
Adposition
	& Spatial meaning
	& Temporal meaning
	\\

\toprule

\tablesubheaderfont\multicolumn{3}{c}{Prepositions} \\

\midrule

kong
	& inside
	& within
	\\

ling
	& on top of
	& while
	\\

marin
	& in front of
	& before
	\\

manga luga
	& through
	& during
	\\

mangasaha
	& towards
	& in + \textit{time}
	\\

pang
	& behind
	& ago
	\\

\midrule

\tablesubheaderfont\multicolumn{3}{c}{Postpositions} \\

\midrule

masahatay
	& ---
	& since
	\\

pesan
	& ---
	& until
	\\

pang
	& beyond, after
	& after, past
	\\

\bottomrule
\end{tabu}

\label{tab:temppos}
\end{table}

It has been mentioned above that location also serves as the conceptual
metaphor for expressing temporal relationships. Notably the prepositions\index{adpositions!prepositions}
\xayr{koNF}{kong}{inside}, \xayr{liNF}{ling}{on}, \xayr{mrinF}{marin}{in front
of}, \xayr{mN lug}{manga luga}{through}, \xayr{mNsh}{mangasaha}{towards}, and
\xayr{pNF}{pang}{behind} come to mind as doubling for `within', `while',
`before', `during', `in + \emph{time}', and `ago', respectively (also see
\autoref{tab:temppos}). Since postpositions\index{adpositions!postpositions} are not primarily derived from
nouns, there are dedicated forms for expressing temporal relationships, namely,
\xayr{mshtj}{masahatay}{since}, \xayr{pesnF}{pesan}{until}, and, as the only
form with a double function, \xayr{pNF}{pang}{after, past}.

\begin{figure}[h]
\pex
\a\label{ex:kongtemp}\begingl
	\gla Miranang kong bihanya sam. //
	\glb mira=nang kong bihan-ya sam //
	\glc do=\Fpl{}.\Aarg{} inside week-\Loc{} two //
	\glft `We will do it within two weeks.' //
\endgl

\a\label{ex:mgshtemp}\begingl
	\gla Girenjang mangasaha pidimya-kay. //
	\glb girend=yang mangasaha pidim-ya=kay //
	\glc arrive=\TsgM{}.\Aarg{} towards hour-\Loc{}=few //
	\glft `He will arrive in a few hours.' //
\endgl

\a\label{ex:lingtemp}\begingl
	\gla Layaye-ikan ang @ Pila ling yeng pakur. //
	\glb laya-ye=ikan ang= Pila ling yeng pakur //
	\glc read-\TsgF{}=much \Aarg{}= Pila on \TsgF{}.\Aarg{} sick //
	\glft `Pila read a lot while she was sick.' //
\endgl
\xe
\end{figure}

Of the examples above, the use of \rayr{koNF}{kong} in (\ref{ex:kongtemp}) is
probably still closest to a local preposition\index{adpositions!prepositions} in that the time span is
conceptualized as a container, or the distance between two points. The use of
\rayr{mNsh}{mangasaha} in (\ref{ex:mgshtemp}), on the other hand, is more
idiomatic. While the prepositions\index{adpositions!prepositions} in these two examples each take an NP\index{phrase types!noun phrase}
complement, example (\ref{ex:lingtemp}) shows that it is also possible for
prepositions\index{adpositions!prepositions} expressing a temporal relationship to govern a subclause. This
ability is even more prominent with temporal postpositions\index{adpositions!postpositions} in that all of the
words listed in \autoref{tab:temppos} can be complemented by either an NP\index{phrase types!noun phrase} or a
clause. This is illustrated for \rayr{mshtj}{masahatay} in
(\ref{ex:mshtaycomps}).

\begin{figure}[h]
\pex\label{ex:mshtaycomps}
\a\label{ex:mshtaynp}\begingl
	\gla Ang @ manga @ hangya lakayperinya masahatay. //
	\glb ang= manga= hang=ya.Ø lakayperin-ya masahatay //
	\glc \AgtT{}= \Prog{}= stay=\TsgM{}.\Top{} solstice-\Loc{} since //
	\glft He has been staying since the solstice. //
\endgl

\a\label{ex:mshtays}\begingl
	\gla Yeng giday sarayāng masahatay. //
	\glb yeng giday sara=yāng masahatay //
	\glc \TsgF{}.\Aarg{} sad leave=\TsgM{}.\Aarg{} since //
	\glft `She has been sad since he left.' //
\endgl
\xe
\end{figure}

\index{semantic role!location|)}
\index{adpositions|)}

\section{Verbs}
\label{sec:verbs}
\index{verbs|(}

Besides nouns, verbs constitute the other main part of speech in Ayeri which
carries inflections. Verbs show person and number agreement\index{agreement}, but may also
inflect for tense, aspect, mood, and modality as grammatical categories of the
verb itself. Personal pronouns\index{pronouns!personal} may furthermore cliticize to the verb stem, and
the verb phrase\index{phrase types!verb phrase} is often also marked with a clitic indicating the topic\index{grammatical function!topic} of the
sentence and the topic\index{grammatical function!topic} NP's\index{phrase types!noun phrase} role in Ayeri's case system, which can be
interpreted as a second agreement\index{agreement} relation. Further clitics may indicate
reflexive actions, progressive aspect, likeness, logical connection, as well
as degree and measure. Verbs are thus probably the most versatile part of
speech on the one hand, but also the one with the heaviest workload on the
other. The following sections will dissect the morphology of verbs category by
category.
% Because verbs inhabit a central position in syntax and exhibit agreement
% morphology, it will be necessary in this section to merge syntax and
% morphology on occasion in order to describe morphosyntactic effects.

\subsection{Person marking}
\label{subsec:persnumagr}
\index{agreement|(}
\index{person|(}

\begin{table}\centering
\caption[Conjugation paradigm for \xayr{sobF/}{sob-}{learn, teach}]{Conjugation
paradigm for \xayr{sobF/}{sob-}{learn, teach} (monoconsonantal root)}

\begin{tabu} to \linewidth {X I[2] I[2] X[2]}
\tableheaderfont\toprule
Person
	& Topicalized%\footnotemark
	& Clitic agent
	& Translation
	\\

\toprule

\Fsg{}	& sobay		& sobyang	& `I learn'		\\
\Ssg{}	& sobva		& sobvāng	& `you learn'	\\
\TsgM{}	& sobya		& sobyāng	& `he learns'	\\
\TsgF{}	& sobye		& sobyeng	& `she learns'	\\
\TsgN{}	& sobyo		& sobyong	& `it learns'	\\
\TsgI{}	& sobara	& sobreng	& `it learns'	\\

\midrule

\Fpl{}	& sobayn	& sobnang	& `we learn'	\\
\Spl{}	& sobva		& sobvāng	& `you learn'	\\
\TplM{}	& sobyan	& sobtang	& `they learn'	\\
\TplF{}	& sobyen	& sobteng	& `they learn'	\\
\TplN{}	& sobyon	& sobtong	& `they learn'	\\
\TplI{}	& sobaran	& sobteng	& `they learn'	\\

\midrule

\Imp{}	& sobu!			& \multicolumn{2}{l}{`learn!'}					\\
\Hort{}	& sobu-sobu!	& \multicolumn{2}{l}{`let's learn!'}			\\
\Iter{}	& so-sob-		& \multicolumn{2}{l}{`learn again, relearn'}	\\
\Ptcp{}	& sobyam		& \multicolumn{2}{l}{`learning'}				\\
	
\bottomrule

\end{tabu}
\label{tab:monoconsconj}
\end{table}

% \footnotetext{Third-person topicalized forms are homonymous with third-person
% agreement forms.}

\begin{table}\centering
\caption[Conjugation paradigm for \xayr{AnFlF/}{anl-}{bring}]{Conjugation 
paradigm for \xayr{AnFlF/}{anl-}{bring} (biconsonantal root)}

\begin{tabu} to \linewidth {X I[2] I[2] X[2]}
\tableheaderfont\toprule
Person
	& Topicalized
	& Clitic agent
	& Translation
	\\

\toprule

\Fsg{}	& anlay		& anlyang	& `I bring'		\\
\Ssg{}	& anlava	& anlavāng	& `you bring'	\\
\TsgM{}	& anlya		& anlyāng	& `he brings'	\\
\TsgF{}	& anlye		& anlyeng	& `she brings'	\\
\TsgN{}	& anlyo		& anlyong	& `it brings'	\\
\TsgI{}	& anlara	& anlareng	& `it brings'	\\

\midrule

\Fpl{}	& anlayn	& anlanang	& `we bring'	\\
\Spl{}	& anlava	& anlavāng	& `you bring'	\\
\TplM{}	& anlyan	& anlatang	& `they bring'	\\
\TplF{}	& anlyen	& anlateng	& `they bring'	\\
\TplN{}	& anlyon	& anlatong	& `they bring'	\\
\TplI{}	& anlaran	& anlateng	& `they bring'	\\

\midrule

\Imp{}	& anlu!			& \multicolumn{2}{l}{`bring!'}					\\
\Hort{}	& anlu-anlu!	& \multicolumn{2}{l}{`let's bring!'}			\\
\Iter{}	& an-anl-		& \multicolumn{2}{l}{`bring again, bring back'}	\\
\Ptcp{}	& anlyam		& \multicolumn{2}{l}{`bringing'}				\\
	
\bottomrule

\end{tabu}
\label{tab:biconsconj}
\end{table}

\begin{table}\centering
\caption[Conjugation paradigm for \xayr{no/}{no-}{want}]{Conjugation 
paradigm for \xayr{no/}{no-}{want} (vocalic root)}

\begin{tabu} to \linewidth {X I[2] I[2] X[2]}
\tableheaderfont\toprule
Person
	& Topicalized
	& Clitic agent
	& Translation
	\\

\toprule

\Fsg{}	& noay		& noyang	& `I want'		\\
\Ssg{}	& nova		& novāng	& `you want'	\\
\TsgM{}	& noya		& noyāng	& `he wants'	\\
\TsgF{}	& noye		& noyeng	& `she wants'	\\
\TsgN{}	& noyo		& noyong	& `it wants'	\\
\TsgI{}	& noara		& noreng	& `it wants'	\\

\midrule

\Fpl{}	& noayn		& nonang	& `we want'		\\	
\Spl{}	& nova		& novāng	& `you want'	\\	
\TplM{}	& noyan		& notang	& `they want'	\\
\TplF{}	& noyen		& noteng	& `they want'	\\
\TplN{}	& noyon		& notong	& `they want'	\\
\TplI{}	& noaran	& noteng	& `they want'	\\

\midrule

\Imp{}	& nu!		& \multicolumn{2}{l}{`want!'}		\\
\Hort{}	& nu-nu!	& \multicolumn{2}{l}{`let's want!'}	\\
\Iter{}	& no-no-	& \multicolumn{2}{l}{`want again'}	\\
\Ptcp{}	& noyam		& \multicolumn{2}{l}{`wanting'}		\\

\bottomrule

\end{tabu}
\label{tab:vocconj}
\end{table}

\begin{table}\centering
\caption[Conjugation paradigm for \xayr{Ap/}{apa-}{laugh}]{Conjugation 
paradigm for \xayr{Ap/}{apa-}{laugh} (vocalic root in -a)}

\begin{tabu} to \linewidth {X I[2] I[2] X[2]}
\tableheaderfont\toprule
Person
	& Topicalized
	& Clitic agent
	& Translation
	\\

\toprule

\Fsg{}	& apāy		& apayang	& `I laugh'		\\
\Ssg{}	& apava		& apavāng	& `you laugh'	\\
\TsgM{}	& apaya		& apayāng	& `he laughs'	\\
\TsgF{}	& apaye		& apayeng	& `she laughs'	\\
\TsgN{}	& apayo		& apayong	& `it laughs'	\\
\TsgI{}	& apāra		& apareng	& `it laughs'	\\

\midrule

\Fpl{}	& apāyn		& apanang	& `we laugh'	\\	
\Spl{}	& apava		& apavāng	& `you laugh'	\\	
\TplM{}	& apayan	& apatang	& `they laugh'	\\
\TplF{}	& apayen	& apateng	& `they laugh'	\\
\TplN{}	& apayon	& apatong	& `they laugh'	\\
\TplI{}	& apāran	& apateng	& `they laugh'	\\

\midrule

\Imp{}	& apu!		& \multicolumn{2}{l}{`laugh!'}			\\
\Hort{}	& apu-apu!	& \multicolumn{2}{l}{`let's laugh!'}	\\
\Iter{}	& ap-apa-	& \multicolumn{2}{l}{`laugh again'}		\\
\Ptcp{}	& apayam	& \multicolumn{2}{l}{`laughing'}		\\

\bottomrule

\end{tabu}
\label{tab:vocconj2}
\end{table}

As described in \autoref{sec:markstrat}, Ayeri conjugates its main verbs
canonically in agreement with the agent NP\index{phrase types!noun phrase}. Verb conjugation as such is
extremely pervasive, to the point where verb roots cannot appear without
inflection. The basic conjugation paradigms are given in Tables
\ref{tab:monoconsconj}--\ref{tab:vocconj}. %\footnote{
Due to the agglutinating\index{agglutination} structure of Ayeri it makes little sense to list the
whole paradigm of verb inflection for all possible affix combinations here, as
the table would become unreasonably large. Instead, the various sections below
will contain examples of use for all affixes.%}

Agreement causes verbs to reflect grammatical categories of nominal entities,
thus, verbs show agreement in person (\First{}, \Second{}, \Third{}) and number\index{number}
(\Sg{}, \Pl{}); third persons are again differentiated by gender\index{gender} (\M{}, \F{},
\N{}, \Inan{}; compare \autoref{subsec:gender}). Verbs only have agreement
proper with third persons; their form, then, is the same as that of verbs with
topicalized\index{grammatical function!topic} pronominal inflection (see \autoref{subsec:perspro}).

Regarding person--number\index{number} inflection, verbs may be divided into three classes: 
monoconsonantal, biconsonantal, and vocalic stems. As discussed in 
\autoref{sec:phonotactics}, Ayeri restricts the number\index{number} of successive non-glide 
consonants to two, which has repercussions in the second person, since the 
conjugation suffix\index{suffixes} there is \rayr{/v}{-va}. Monoconsonantal roots are 
unaffected by this restriction, however, hence the conjugation suffixes\index{suffixes} can 
simply be appended as they are; this is illustrated with the verb 
\xayr{sobF/}{sob-}{teach, learn} in \autoref{tab:monoconsconj}. Verb stems 
ending in dental and velar plosives will naturally undergo palatalization in 
the third person animate, so for instance, the third person singular masculine 
of the verb \xayr{gurtF/}{gurat-}{answer} is \xayr{gurtFy}{guraca}{(he) 
answers}, and the third person feminine plural of \xayr{AbgF/}{abag-}{roam, 
wander} is \xayr{AbgFyenF}{abajen}{(they) roam, (they) wander}. Verbs whose 
stem ends in an affricate are treated as monoconsonantal roots as well, since 
the affricate occupies one consonant phoneme segment. Thus, the second 
person of \xayr{ItYF/}{ic-}{glide, slide} is not *\rayr{ItYv}{*icava}, but 
\xayr{ItYFv}{icva}{you glide, you slide}.

Since /v/ is neither a vowel nor a glide, an epenthetic \fw{-a-} is inserted
between the stem and the second-person suffix\index{suffixes} \rayr{/v}{-va} for verbs whose
stem ends in -CC.\footnote{A \emph{root} is understood here as the uninflected
verb morpheme, for instance, \rayr{AnFlF/}{anl-}, \rayr{ItYF/}{ic-},
\rayr{no/}{no-}, or \rayr{sob/}{sob-}. A \emph{stem} may contain inflections
and further inflectional affixes attach to it; it may also host clitics\index{clitics}. Roots
are thus counted as a subset of stems here.} This is illustrated in
\autoref{tab:biconsconj} for the verb \xayr{AnFlF/}{anl-}{bring}. The second
person conjugation of this verb is not *\rayr{AnFlFv}{*anlva}, since the
cluster \fw{-nlv-} is illegal, but \rayr{AnFlv}{anlava}. Since Ayeri treats two
successive instances of the same consonant as a single segment---there is no
gemination---verbs like \xayr{silFvF/}{silv-}{see} conjugate like
monoconsonantal roots with regards to consonant clusters. That is, the second
person of \rayr{silFvF/}{silv-} is not *\rayr{silFvv}{*silvava}, as one might
expect, but \rayr{silFv̔}{silvva}. A further exception to this are verbs
ending in -Cs, since -Cs-C- is commonly resyllabified as -C-sC- (see
\autoref{ch:phonology}, \autoref{fn:ssyl}). Thus, the second-person form of
\xayr{krFsF/}{kars-}{freeze} is not *\rayr{krFsv}{*karsava} as expected, but
\xayr{krFsFv}{karsva}{you freeze}.

Lastly, verb stems may end in a vowel, most commonly \fw{-a}. In these cases as
well, the conjugation suffixes\index{suffixes} may simply be appended to the stem. The
conjugation of this class is illustrated in \autoref{tab:vocconj} with the verb
\xayr{no/}{no}{want}. Verb stems ending in \fw{-a} undergo crasis regularly for
the first person suffixes\index{suffixes}, hence, the topicalized first-person singular form of
\xayr{Ap/}{apa-}{laugh} is \xayr{Apaaj}{apāy}{I laugh} (compare
\autoref{tab:vocconj2}). Verb stems ending in a diphthong in /ɪ/ are
treated as a hybrid of monoconsonantal and vocalic stems, since the diphthong's
final /ɪ/ is treated as /j/ before a vowel: \xayr{plyj}{palayay}{I rejoice},
\xayr{pljv}{palayva}{you rejoice}.

\index{clitics|(}
As mentioned above, the form of the third-person agreement suffixes\index{suffixes} on verbs is
essentially the same as that of topic-marked\index{grammatical function!topic} third-person pronominal clitics.
Any other person-marking on verbs except for third-person agreement is, in
fact, a topicalized\index{grammatical function!topic} pronoun\index{pronouns!personal} clitic, as we will see in the course of the
following discussion. Unlike English\index{English}, Ayeri does not use agent\index{case!agent} pronouns\index{pronouns} in
addition to person agreement\index{agreement} on verbs. Consider the two examples of English\index{English} in
(\ref{ex:engpersagr}).

\begin{figure}[h]
\pex\label{ex:engpersagr}%
English:
\a\label{ex:vbagrengnn}\begingl
	\gla John greets Mary. //
	\glb John greet-s Mary //
	\glc John greet-\Tsg{}.\Prs{} Mary //
\endgl

\a\label{ex:vbagrengpr}\begingl
	\gla He greets Mary. //
	\glb he greet-s Mary //
	\glc \TsgM{} greet-\Tsg{}.\Prs{} Mary //
\endgl
\xe
\end{figure}

In these examples, the verb has an agreement suffix\index{suffixes} \fw{-s} which indicates
third person singular, present tense\index{tense!present}, whether the subject of the sentence is a
noun (\fw{John}) or a pronoun (\fw{he}), which acts as a free morpheme in
English\index{English}. Now consider the Ayeri equivalents of these two examples in
(\ref{ex:ayrpersagr}), on the other hand.\footnote{Most of the following
account is taken nearly verbatim from a previously published blog article,
\citet{benung:verbagreement}. Some of the Ayeri examples used in the following
come from a list of samples I provided for a bachelor's thesis at the
University of Kent in March 2016, in private conversation, on request.%
% I do not know what the author made of them---the questionnaire I filled out
% initially indicated that the thesis was probably on the syntactic typology of
% fictional languages regarding typical word-order correlations (VO correlating
% with head-initial order etc.). I hope that my reflections here do not preempt
% or invalidate the author's analyses should they still be in the process of
% writing or their submitted thesis be in the process of evaluation and
% grading. I would certainly like to learn about their analysis of my examples.
}

\begin{figure}[h]
\pex\label{ex:ayrpersagr} % (1)
\a\label{ex:vbagr}\begingl
	\gla Ang @ manya {} @ Ajān sa @ Pila. //
	\glb ang= man-ya Ø= Ajān sa= Pila //
	\glc \AgtT{}= greet-\TsgM{} \Top{}= ​Ajān \Parg{}= Pila //
	\glc {} {} {} [\TsgM{}] {} [\TsgF{}] //
	\glft `Ajān greets Pila.' //
\endgl

\a\label{ex:vbclt}\begingl
	\gla Ang @ manya sa= Pila. //
	\glb ang= man=ya.Ø sa= ​Pila //
	\glc \AgtT{}= greet=\TsgM{}.\Top{} \Parg{}= ​Pila //
	\glc {} {} {} [\TsgF{}] //
	\glft `He greets Pila.' //
\endgl
\xe
\end{figure}

It is probably uncontroversial to analyze \rayr{/y}{-ya} in (\ref{ex:vbagr}) as
person agreement\index{agreement}: \rayr{AgYaanF}{Ajān} is a male name in Ayeri while
\rayr{pil}{Pila} is a female one; the verb inflects for a masculine third
person, which tells us that it agrees with the one doing the greeting,
\rayr{AgYaanF}{Ajān}. \rayr{AgYaanF}{Ajān} is also who this is about, which is
shown on the verb by marking for an agent\index{case!agent} topic\index{grammatical function!topic}. In (\ref{ex:vbagrspapr}),
there is only anaphoric reference to \rayr{AgYaanF}{Ajān}; the full agent\index{case!agent} NP\index{phrase types!noun phrase} is
not realized. Very broadly thus, the verb marking here seems to be like in
Spanish\index{Spanish}, where you can drop the subject pronoun; see (\ref{ex:spanpersagr}).
% \footnote{However, we will see that it is probably more complicated than 
% this.}

\begin{figure}[h]
\pex\label{ex:spanpersagr}% (2)
Spanish:
\a\label{ex:vbagrspann}\begingl
	\gla Juan saluda a María. //
	\glb Juan salud-a a María //
	\glc John greet-\Tsg{} \Acc{} Mary //
	\glft `John greets Mary.' //
\endgl

\a\label{ex:vbagrspapr}\begingl
	\gla Saluda a María. //
	\glb salud-a a María //
	\glc greet-\Tsg{} \Acc{} Mary //
	\glft `He greets Mary.' //
\endgl
\xe
\end{figure}

Example (\ref{ex:vbclt}) probably does not seem conspicious if we assume that
Ayeri is pro-drop, except that there is also topic\index{grammatical function!topic} marking for an agent\index{case!agent} there,
the controller of which I have so far assumed to be the person inflection on
the verb, in analogy with examples like (\ref{ex:lampyaang}).

\begin{figure}[h]
\ex\label{ex:lampyaang} % (3)
\begingl
	\gla Lampyāng. //
	\glb lamp=yāng //
	\glc walk=\TsgM{}.\Aarg{} //
	\glft `He walks.' //
\endgl\xe
\end{figure}

This raises the question whether in Ayeri, there is dropping of an agent\index{case!agent}
pronoun\index{pronouns} involved at all, which is why the person suffix\index{suffixes} in (\ref{ex:vbclt}) was
glossed as \emph{=ya.Ø} (\mbox{=\TsgM{}.\Top{}}) rather than just as \emph{-ya}
(-\TsgM{}). In turn, this question leads us to another characteristic of Ayeri
we need to consider, namely that the topic\index{grammatical function!topic} morpheme on noun phrases\index{phrase types!noun phrase} is zero.
That is, the absence of overt case\index{case} marking on a nominal element indicates that
it is a topic\index{grammatical function!topic}; the verb in turn marks the case\index{case} of the topicalized\index{grammatical function!topic} NP\index{phrase types!noun phrase} with a
(case-indicating) particle preceding it. Pronouns\index{pronouns!personal} as well show up in their
unmarked form when topicalized\index{grammatical function!topic}, which is why I am hesitant to analyze the
pronoun in (\ref{ex:protop}) as a clitic on the verb rather than as an
independent morpheme.\footnote{Also, perhaps a little untypically, topic\index{grammatical function!topic} NPs\index{phrase types!noun phrase} in
Ayeri are not usually pulled to the front\index{word order} of the phrase (at least not in the
written language; see \cite[120--122]{lehmann2015}), so topic-marked\index{grammatical function!topic} pronouns\index{pronouns}
stay \fw{in-situ}. Which NP\index{phrase types!noun phrase} constitutes the topic\index{grammatical function!topic} of the phrase is marked on
the verb right at the head of the clause. How and whether this can be justified
in terms of grammatical weight (see, for instance, \cite[95--98]{wasow1997})
remains to be seen.}

\begin{figure}[h]
\pex % (4)
\a\label{ex:fullsntc}\begingl
	\gla Sa @ manya ang @ Ajān {} @ Pila. //
	\glb sa= man-ya ang= ​Ajān Ø= ​Pila //
	\glc \PatT{}= greet-\TsgM{} \Aarg{}= ​Ajān \Top{}= ​Pila //
	\glft `It's Pila that Ajān greets.' //
\endgl

\a\label{ex:protop}\begingl
	\gla Sa @ manyāng ye. //
	\glb sa= man=yāng ye.Ø //
	\glc \PatT{}= greet=\TsgM{}.\Aarg{} \TsgF{}.\Top{} //
	\glft `It's her that he greets.' //
\endgl
\xe
\end{figure}

\phantomsection\label{patagr}
What is remarkable, then, is that \rayr{ye}{ye} (\TsgF{}.\Top{}) in
(\ref{ex:protop}) is the very same form that appears as an agreement morpheme
on the verb in (\ref{ex:ayrpronagr}), just like \rayr{/y}{-ya} (\TsgM{}) in
various examples above (also compare the examples in \autoref{subsec:perspro}).
This also holds for all other personal pronouns\index{pronouns!personal}. Moreover, 
\rayr{/yaaNF}{-yāng} as seen in examples (\ref{ex:lampyaang}) and 
(\ref{ex:protop}) may also be used as a free pronoun\index{pronouns} in equative statements
with predicative\index{grammatical function!predicative complement} nominals, as well as other such case-marked\index{case} personal forms, as
illustrated in (\ref{ex:indeppro}). As for case-marked\index{case} person suffixes\index{suffixes} on
verbs, the assumption so far has been that they are essentially clitics,
especially since the marking strategy displayed in (\ref{ex:passive}) is the
grammatical one in absence of an agent\index{case!agent} NP\index{phrase types!noun phrase} (compare
\autoref{clitics_postverb_person}, p.~\pageref{clitics_postverb_person}).

\begin{figure}[h]
\ex\label{ex:ayrpronagr} % (5)
\begingl
	\gla Ang @ purivaye yāy. //
	\glb ang= puriva=ye.Ø yāy //
	\glc \AgtT{}= smile=\TsgF{}.\Top{} \TsgM{}.\Loc{} //
	\glft `She smiles at him.' //
\endgl\xe
\end{figure}

\begin{figure}[h]
\ex\labels\label{ex:indeppro} % (6)
\begin{minipage}[t]{.5\remaining}
\tl\quad\begingl
	\gla Yeng mino. //
	\glb yeng mino //
	\glc \TsgF{}.\Aarg{} happy //
	\glft `She is happy.' //
\endgl
\end{minipage}
~
\begin{minipage}[t]{.5\remaining}
\tl\quad\begingl
	\gla Yāng naynay. //
	\glb yāng naynay //
	\glc \TsgM{}.\Aarg{} too //
	\glft `He is, too.' //
\endgl
\end{minipage}
\xe
\end{figure}

\begin{figure}[h]
\ex\labels\label{ex:passive} % (7)
\begin{minipage}[t]{.5\remaining}
\tl\quad\label{ex:manye}\begingl
	\gla Manye sa @ Pila. //
	\glb man-ye sa= ​Pila //
	\glc greet-\TsgF{} \Parg{}= ​Pila //
	\glft `Pila is being greeted.' //
\endgl
\end{minipage}
~
\begin{minipage}[t]{.5\remaining}
\tl\quad\label{ex:manyes}\begingl
	\gla Manyes. //
	\glb man=yes //
	\glc greet=\TsgF{}.\Parg{} //
	\glft `She is being greeted.' //
\endgl
\end{minipage}
\xe
\end{figure}

The verb here agrees\index{agreement} with the patient\index{case!patient}---or is it that person agreement suffixes\index{suffixes}
on verbs are generally clitics in Ayeri, even where they do not involve case
marking? There seems to be a gradient here between what looks like regular verb
agreement with the agent\index{case!agent} on the one hand, and agent\index{case!agent} or patient\index{case!patient} pronouns\index{pronouns} just
stacked onto the verb stem on the other hand. For an overview, compare
\autoref{tab:persinfltypes}. In this table, especially the middle,
transitional category is interesting in that what looks like verb agreement
superficially can still trigger topicalization\index{grammatical function!topic} marking, which is indicated in
column 2 by an index `$i$'. Note that this behavior only occurs in transitive\index{verbs!transitive}
contexts; there is no topic\index{grammatical function!topic} marking on the verb if the verb only has a single
NP\index{phrase types!noun phrase} dependent. Also consider that for example (b) in the type 3 transitive\index{verbs!transitive}
cell the question is, whether this should not better be analyzed as
\AgtT{}=…-\TsgM{}.\Top{} …-\Top{} …-\Parg{}, with co-indexing of the topic\index{grammatical function!topic} on 
the person inflection of the verb, making it structurally closer to type 2.

\afterpage{%
\clearpage%
\begin{landscape}\centering
\begin{table}[p]
\caption{Verb inflection types in Ayeri}
\begin{tabu} to \linewidth{B[2l] X[4] X[4] X[4]}
\tableheaderfont\toprule
%
	& Type 1: clitic pronouns
	& Type 2: Transitional
	& Type 3: verb agreement
\\

\toprule

Inflectional categories
	& person\newline
		number\newline
		case
	%
	& Person\newline
		number\newline
		case/topic
	%
	& Person\newline
		number
\\

\midrule

Examples (intransitive)
	& \fw{…=yāng}\newline
		…=\TsgM{}.\Aarg{}
	%
	& ---
	%
	& \fw{…-ya}$_i$ \fw{…-ang}$_i$\newline
		…-\TsgM{} …-\Aarg{}
\\

\midrule

Examples (transitive)
	& \fw{sa}$_i$ \fw{…=yāng …-Ø}$_i$\newline
		\PatT{}=…=\TsgM{}.\Aarg{} …-\Top{}
	%
	& \fw{ang}$_i$ \fw{…=ya.Ø}$_i$ \fw{…-as}\newline
		\AgtT{}=…=\TsgM{}.\Top{} …-\Parg{}
	%
	& \begin{tabu} to \linewidth {@{} l @{\quad} l @{}}
		a.	& \fw{ang}$_i$ \fw{…-ya}$_i$ \fw{…-Ø}$_i$ \fw{…-as} \\
			& \AgtT{}=…-\TsgM{} …-\Top{} …-\Parg{} \medskip
		\\
		
		b.	& \fw{sa}$_i$ \fw{…-ya}$_j$ \fw{…-ang}$_j$ \fw{…-Ø}$_i$ \\
			& \PatT{}=…-\TsgM{} …-\Aarg{} …-\Top{}
		\\
	\end{tabu}
\\

\bottomrule
\end{tabu}
\label{tab:persinfltypes}
\end{table}
\end{landscape}
\clearpage%
}

As for personal pronouns\index{pronouns!personal} fused with the verb stem like in the first column, 
\citet{corbett2006} points out that

\blockcquote[99--100]{corbett2006}{[i]n terms of syntax, pronominal affixes are
arguments of the verb; a verb with its pronominal affixes constitutes a full
sentence, and additional noun phrases are optional. If pronominal affixes are
the primary arguments, then they agree in the way that anaphoric pronouns agree
[…] In terms of morphology, pronominal affixes are bound to the verb; typically
they are obligatory […].}

\noindent This seems to be exactly what is going on, for instance, in
(\ref{ex:lampyaang}) and (\ref{ex:manyes}), where the verb forms a complete
sentence. It needs to be pointed out that Corbett includes an example from
Tuscarora, a native American polysynthetic language, in relation to the above
quotation. Ayeri should not be considered polysynthetic, however, since its
verbs generally do not exhibit relations with multiple NPs\index{phrase types!noun phrase}, at least as far as
person and number agreement is concerned
\citep[45--46]{comrie1989}.\footnote{The topic\index{grammatical function!topic} NP\index{phrase types!noun phrase} marked on the verb may be
different from the one with which the verb agrees in person and number\index{number}, so
technically, Ayeri verbs \emph{may} agree with more than one NP\index{phrase types!noun phrase} in a very
limited way (compare \autoref{sec:markstrat}). Still, I would not analyze this
as polypersonal agreement, since there is only canonical verb agreement with
one constituent, that is, the agent\index{case!agent} NP\index{phrase types!noun phrase}. Topic\index{grammatical function!topic} marking should, in my opinion, be
viewed as a separate agreement relation, as pointed out in the quoted section
above.}

Taking everything written above so far into account, it looks as though
Ayeri is in the process of grammaticalizing personal pronouns\index{pronouns!personal} into person
agreement\index{agreement} \parencites[42--45]{lehmann2015}[493--497]{vangelderen2011}.
\citet{corbett2006} illustrates an early stage of such a process with the
example in (\ref{ex:agrgen}).

\begin{figure}[h]
\pex\label{ex:agrgen}% (8)
Skou\index{Skou} \parencite[76--77]{corbett2006}:
\a\begingl
	\gla Ke móe ke=fue. \quad {\textup{(*}​Ke móe fue.\textup{)}} //
	\glb \TsgM{} fish \TsgM{}=​see.\TsgM{} {} //
	\glft `He saw a fish.' //
\endgl

\a\begingl
	\gla Pe móe pe=fu. \quad {\textup{(*}​Pe móe fu.\textup{)}} //
	\glb \TsgF{} fish \TsgF{}=​see.\TsgF{} {} //
	\glft `She saw a fish.' //
\endgl
\xe
\end{figure}

What \citet{vangelderen2011} calls the \emph{subject cycle}, the
\textcquote[493]{vangelderen2011}{oft-noted cline expressing that pronouns can
be reanalyzed as clitics and agreement markers} applies here, and as well in
Ayeri. However, while she continues to write that in
\textcquote[494]{vangelderen2011}{many languages, the agreement affix
resembles the emphatic pronoun and derives from it}, Ayeri does at least in
part the opposite and uses the case-unmarked\index{case} form of personal pronouns\index{pronouns!personal} for
what resembles verb agreement most closely. This, however, should not be too
controversial either, considering that, for instance, semantic bleaching and
phonetic erosion go hand in hand with grammaticalization\index{grammaticalization} 
\parencites[136--137]{lehmann2015}[497]{vangelderen2011}.

As pointed out above in (\ref{ex:passive}), Ayeri usually exhibits verbs as
agreeing\index{agreement} with agents\index{case!agent} and occasionally patients\index{case!patient} (but only in absence of agent\index{case!agent}
NPs\index{phrase types!noun phrase})---not topics\index{grammatical function!topic} as such. Ayeri, thus, has subject\index{grammatical function!subject} agreement. Agreement with a
patient\index{case!patient} NP\index{phrase types!noun phrase} may seem a little counterintuitive, but is licensed by Ayeri's
semantics-based case marking\index{case} which marks patient\index{semantic role!patient}-subjects\index{grammatical function!subject} of passive\index{voice!passive} clauses as
such; the agent case\index{case!agent} is not fully equivalent to a nominative which marks the
subject\index{grammatical function!subject} function. Formally, also, agent NPs\index{phrase types!noun phrase} usually follow\index{word order} the verb, and it
does not seem too unnatural to have an agreement relation between the verb and
the closest NP\index{phrase types!noun phrase} also when non-conjoined NPs\index{phrase types!noun phrase} are involved
\citep[180]{corbett2006}. This may serve as another explanation for why verbs
can agree with patients\index{semantic role!patient} as well if the agent NP\index{phrase types!noun phrase} is absent. Taking into account
that the grammaticalization\index{grammaticalization} process is still ongoing so that there is still
some relative freedom in how morphemes may be used if a paradigm has not yet
fully settled \citep[148--150]{lehmann2015} also makes this seem less strange.
Formally, thus, verbs simply become agreement targets of the closest
semantically plausible nominal constituent which can serve as a subject\index{grammatical function!subject}.

\begin{table}[tp]\centering
\caption[The syntax and morphology of pronominal affixes]{The syntax and 
morphology of pronominal affixes \citep[101]{corbett2006}}
{\tabulinesep=\itemsep
\begin{tabu} to \linewidth {B[28l,m] | X[24c,m] | X[24c,m] | X[24c,m]}

\tabucline[1pt]{1-4}

Syntax:\bigstrut
	& non-argument%\bigstrut
	& \multicolumn{2}{c}{argument}%\bigstrut}
\\

\hline

\mbox{Linguistic element:}%\bigstrut
	& `pure' agreement marker
	& pronominal affix%\bigstrut
	& free pronoun%\bigstrut
\\

\hline

Morphology:
	& \multicolumn{2}{c|}{inflectional form}%\bigstrut}
	& free form%\bigstrut
\\

\tabucline[1pt]{1-4}

\end{tabu}
}
\label{ex:typproaffx}
\end{table}

From the previous discussion of Ayeri's agreement and pronoun\index{pronouns!personal} morphology, it
may seem as though person agreement consists entirely of enclitic pronominal
affixes. The question is, how to determine and describe what actually happens
in terms of morphology. \citet{corbett2006} offers a typology\index{typology} along with test
criteria; compare \autoref{ex:typproaffx}. According to this typology\index{typology}, a
pronominal affix is syntactically an argument of the verb, but has the
morphology of an inflectional form (compare \autoref{clitics_postverb_person},
p.~\pageref{clitics_postverb_person}). If we compare this to the gradient given
in \autoref{tab:persinfltypes} above, it becomes evident that type I definitely
fulfills these criteria, and type 2 does so as well, in fact, in that there is
no agent\index{case!agent} NP\index{phrase types!noun phrase} that could serve as a controller if the verb inflection in type 2
were `merely' an agreement target. The inflection in type 3, on the other
hand, appears to have all hallmarks of agreement in that there is a controller
NP\index{phrase types!noun phrase} that triggers it, with the verb serving as an agreement target.

Moreover, the person marking on the verb is not a syntactic argument of the
verb in this case. As example (\ref{ex:manye}) shows, however, marking of type
3 permits the verb to mark more than one case role, which makes it slightly
atypical, although verbs can only carry a single instance of person marking
\citep[103]{corbett2006}. Regarding referentiality, the person suffixes\index{suffixes} on the
verb in \autoref{tab:persinfltypes}, columns 1 and 2 are independent means of
referring to discourse participants mentioned earlier, whereas the person
suffix\index{suffixes} in 3 needs support from an NP\index{phrase types!noun phrase} in the same clause as a source of
semantic features to share. This becomes apparent when comparing the examples
in (\ref{ex:agrctrl}) to each other.

\begin{figure}[h]
\pex\label{ex:agrctrl} % (9)
\a\label{ex:agttopclit}\begingl
	\gla Ajān … Ang @ manya sa @ Pila. //
	\glb Ajān … Ang= man=ya.Ø sa= ​Pila //
	\glc Ajān … \AgtT{}= greet=\TsgM{}.\Top{} \Parg{}= ​Pila //
	\glft `Ajān … He greets Pila.' //
\endgl

\a\label{ex:agtproclit}\begingl
	\gla Ajān … Sa @ manyāng {} @ Pila. //
	\glb Ajān … Sa= man=yāng Ø= Pila //
	\glc Ajān … \PatT{}= greet=\TsgM{}.\Aarg{} \Top{}= ​Pila //
	\glft `Ajān … It's Pila that he greets.' //
\endgl

\a\label{ex:wrongagr}\ljudge* \begingl
	\gla Ajān … Manya sa @ Pila. //
	\glb Ajān … Man-ya sa= ​Pila //
	\glc Ajān … greet-\TsgM{} \Parg{}= ​Pila //
\endgl
\xe
\end{figure}

Since person marking of the type 1 and 2 is \emph{referential}, as shown in
(\ref{ex:agttopclit}) and (\ref{ex:agtproclit}), it is best counted as
consisting of cliticized pronouns\index{pronouns!personal} \citep[103]{corbett2006}. Since mere
agreement as in type 3 needs support from an NP\index{phrase types!noun phrase} within the verb's scope\index{scope},
though, it does not have \emph{descriptive/lexical content} of its own. That
is, it \emph{only} serves a grammatical function \citep[104]{corbett2006}, not
strictly as an anaphora. This is why (\ref{ex:wrongagr}) is marked as
ungrammatical: the agreement suffix\index{suffixes} \rayr{/y}{-ya} itself does not define the
semantic features of the clause's subject\index{grammatical function!subject}; it requires a subject\index{grammatical function!subject} NP\index{phrase types!noun phrase} to exist
concurrently.

As for \citet{corbett2006}'s \emph{balance of information} criterion,
\autoref{tab:persinfltypes} also highlights differences in what information is
provided by the person marking. Nouns in Ayeri inherently bear information on
person, number, and gender, and all three types of person inflection on verbs
share these features. However, there are no additional grammatical features
indicated by the first two inflection types that are not expressed by noun
phrases\index{phrase types!noun phrase}, although under a very close understanding of \citet{corbett2006},
example (\ref{ex:verbplagr}) may still qualify as person-marking on the verb
realizing a grammatical feature shared with an NP\index{phrase types!noun phrase} that is not openly expressed
by the NP\index{phrase types!noun phrase}. \citet{corbett2006} writes that in the world's languages, this
frequently is number\index{number} \citep[105]{corbett2006}. This, however, does not apply to
Ayeri because the only time verbs display number\index{number} not expressed overtly by
inflection on a noun is in agreement like in type 3a, which is
exemplified by (\ref{ex:verbplagr}). Here, redundant plural\index{number!plural} marking on the
subject\index{grammatical function!subject} NP\index{phrase types!noun phrase} is omitted, but plural\index{number!plural} number\index{number} still surfaces in the agreement suffix\index{suffixes}
on the verb.\footnote{From an \Lfg{}\index{Lexical-functional grammar} point of view, the
number\index{number} feature of \rayr{kj}{kay} in (\ref{ex:verbplagr}) coalesces with the
semantic features provided by \rayr{AyonF}{ayon} in the maximal projection;
agreement is thus with the whole agent\index{case!agent} NP\index{phrase types!noun phrase} rather than just with
\rayr{AyonF}{ayon} as the NP's\index{phrase types!noun phrase} categorial head.}

\begin{figure}[h]
\ex\label{ex:verbplagr} % (10)
\begingl
	\gla Ang @ sahayan ayon kay kong nangginoya. //
	\glb ang= saha-yan ayon-Ø kay kong nanggino-ya //
	\glc \AgtT{}= come-\TplM{} man-\Top{} three into tavern-\Loc{} //
	\glft `Three men come into a pub.' //
\endgl\xe
\end{figure}

As discussed previously, verb marking of the types 1 and 2 is independent as a
reference, so there is \emph{unirepresentation} of the marked NP\index{phrase types!noun phrase}. In contrast,
verb marking of type 3 requires a controlling NP\index{phrase types!noun phrase} in the same clause to share
grammatical features with, so that there is \emph{multirepresentation} typical
of canonical agreement \citep[106]{corbett2006}.
% Note that unirepresentation as
% outlined here is probably different from pro-drop, as in this case 1 would
% expect sentences like (\ref{ex:wrongagr}) to be grammatical
% \citep[107]{corbett2006}.
A further property that hinges on types 1 and 2 being independent pronouns\index{pronouns!personal}
glued to verbs as clitics is that they are not coreferential with another NP\index{phrase types!noun phrase} of
the same grammatical relation, but are in complementary distribution\index{complementary distribution}, as
commonly assumed with pronominals \citep[108]{corbett2006}. Hence, either of
the two examples in (\ref{ex:subjclitcoref}) is ungrammatical.

\begin{figure}[h]
\pex\label{ex:subjclitcoref} % (11)
\a\ljudge* \begingl
	\gla Lampyāng ang @ Ajān. //
	\glb lamp=yāng ang= ​Ajān //
	\glc walk=\TsgM{}.\Aarg{} \Aarg{}= Ajān //
\endgl

\a\ljudge* \begingl	
	\gla Ang @ lampyāng {} @ Ajān. //
	\glb ang= lamp=yāng Ø= ​Ajān //
	\glc \AgtT{}= walk=\TsgM{}.\Aarg{} \Top{}= ​Ajān //
\endgl
\xe
\end{figure}

However, verb agreement with a free pronoun\index{pronouns} is also not possible even though it
might be expected according to \citep[109]{corbett2006}---also compare example
(\ref{ex:vbagrengpr}) above. Instead, the suffixed\index{suffixes} agent pronoun\index{pronouns!personal} replaces any
possible person agreement on the verb in (\ref{ex:proclitcoref}).\footnote{Also
see \autoref{clitics_postverb_person}, p.~\pageref{clitics_postverb_person}~ff.
for an analysis from a syntactic point of view.}

\begin{figure}[h]
\ex\labels\label{ex:proclitcoref} % (12)
\begin{minipage}[t]{.5\remaining}
\tl\quad\begingl
	\gla Lampyāng. //
	\glb lamp=yāng //
	\glc walk=\TsgM{} //
	\glft `He walks.' //
\endgl
\end{minipage}
~
\begin{minipage}[t]{.5\remaining}
\tl\quad\ljudge* \begingl	
	\gla Lampya yāng. //
	\glb lamp-ya yāng //
	\glc walk-\TsgM{} \TsgM{}.\Aarg{} //
	\glft \textit{Intended:} `He walks.' //
\endgl
\end{minipage}
\xe
\end{figure}

In conclusion, we may assert that Ayeri appears to be in the process of
grammaticalizing pronouns\index{pronouns!personal} as verb inflection, however, how far this
grammaticalization process has progressed is dependent on syntactic context.
Ayeri displays a full gamut from personal pronouns\index{pronouns!personal} (usually agents) glued to
verbs as clitics to agreement with coreferential NPs\index{phrase types!noun phrase} that is transparently
derived from these personal pronouns\index{pronouns!personal}. With the latter, the complication arises
that pronouns\index{pronouns} are not allowed as agreement controllers as one might expect, but
only properly nominal NPs\index{phrase types!noun phrase}. Information on agreement with committee nouns and
coordinated NPs\index{phrase types!noun phrase} with incongruent agreement features can be found in the section
on VPs.

\index{clitics|)}
\index{person}
\index{agreement|)}

\subsection{Tense}
\label{subsec:tense}
\index{tense|(}

Tense in Ayeri is often not explicitly marked, but has to be inferred from
context. However, where marked, Ayeri distinguishes past and future as
referring to past and future events, respectively. Both past and future tenses
come with three degrees each: near, recent/impending, and remote. Ayeri's
distinguishing three degrees of both past and future time is a little unusual
with regards to typology\index{typology} according to the survey conducted by
\citet[127]{dahl1985}. The decision for which subtier of the past and the
future to use is up to pragmatics, that is, there are no definitive and 
clear-cut lines. The near-time markers are most commonly used for immediate 
scope, that is, things which have just happened or will happen in a moment. 
The recent/impending-time markers may then be used for anything else which 
does not qualify as remote, that is, a long time into the past or the future 
from the point of view of the speaker.

\citet[117]{dahl1985} further notes that among the languages in the surveyed
sample, past tenses are mostly marked by suffixes\index{suffixes}, the marking of this category
being extremely common in addition. Ayeri may thus be a little unusual
crosslinguistically again by exclusively using prefixes\index{prefixes} for tense marking. This
makes sense, however, if we assume that historically, the tense prefixes\index{prefixes} once
were auxiliary verbs. Ayeri applies head-initial word order\index{word order} to subordinating
verbs, as we will see further below, so these prefixes\index{prefixes} may just have begun to
\emph{pro}cliticize instead of slipping into a position behind their head (that
is, Wackernagel's position).

Of the triad tense--aspect--mood this section will only cover basic uses of 
the marked tense categories, followed by a discussion of complex tense 
combinations such as past-in-future. The subsequent \autoref{subsec:aspect} 
will provide more insight into the morphological marking of aspectual\index{aspect} 
categories; \autoref{subsec:mood} deals with the morphology of mood\index{mood} marking in 
Ayeri.

\subsubsection{Present tense}
\index{tense!present|(}
Verbs in Ayeri are unmarked for present tense, since it is the normal mode of 
speaking. Besides being used to comment or report on current events, the 
present tense is also used to make statements of general truth:

\begin{figure}[h]
\ex\begingl
	\gla Sa @ arapyo tahanyamanang koyana nogalam-ikan. //
	\glb sa= arap-yo tahanyaman-ang koya-na nogalam-Ø=ikan //
	\glc \PatT{}= require-\TsgN{} writing-\Aarg{} book-\Gen{} 
		patience-\Top{}=much //
	\glft `Writing a book requires much patience.' //
\endgl\xe
\end{figure}

Moreover, Ayeri does not strictly mark its verbs for past tense in narrative 
discourses---verbs may thus appear as though with present-time reference in 
spite of recounting past events, whether historical or fictional. See the next 
subsection on the past tense.

\index{tense!present|)}

\subsubsection{Past tense}
\label{subsubsec:past}
\index{tense!past|(}
The past tense indicates actions which happened in the past if not further
modified. The three degrees of past tense are marked with \rayr{k/}{kə-}
(near/immediate), \rayr{m/}{mə-} (recent), and \rayr{v/}{və-} (remote), which
attach right in front\index{affix order} of a verb root. In spite of the customary spelling of the
past tense prefixes\index{prefixes} with \orth{ə}, which reflects pronunciation, they have an
underlying /a/ vowel in this place. This means that the vowel of the tense
prefixes\index{prefixes} coalesces\index{morphophonology} with a following /a/ to form a long vowel (see
\autoref{subsec:vowels}), which is demonstrated in example (\ref{ex:pst}).
\index{allomorphy}

\begin{figure}
\pex
\a\label{ex:npst}\begingl
	\gla Ang @ kəsilvay yes motonya. //
	\glb ang= kə-silv=ay.Ø yes moton-ya //
	\glc \AgtT{}= \NPst{}-see=\Fsg{}.\Top{} \TsgF{}.\Parg{} store-\Loc{} //
	\glft `I've just seen her at the store.' //
\endgl

\a\label{ex:pst}\begingl
	\gla Le @ mādruyāng ikan biratay. //
	\glb le= mə-adru=yāng ikan biratay-Ø //
	\glc \PatTI{}= \Pst{}-break=\TsgM{}.\Aarg{} wholly pot-\Top{} //
	\glft `The pot, he completely broke it.' //
\endgl

\a\label{ex:rpst}\begingl
	\gla Vəmittang edaya. //
	\glb və-mit=tang edaya //
	\glc \RPst{}-live=\TplM{}.\Aarg{} here //
	\glft `They lived here (a long time ago).' //
\endgl
\xe
\end{figure}

Note that the recent and the remote past tense are not generally marked if the 
past context is clear, for instance, when a past context has already been 
established in discourse. This may also happen explicitly by using a time 
adverbial such as \xayr{tml}{tamala}{yesterday} or \xayr{perikYnFy menNF 
pNF}{pericanya menang pang}{a hundred years ago}. In the presence of an 
explicit time adverbial, redundant tense marking is also dropped subsequently.

\begin{figure}
\ex\label{ex:pstnopst}\begingl
	\gla Ang @ kondayn kadanya terpasānley bihanya sarisa. //
	\glb ang= kond=ayn.Ø kadanya terpasān-ley bihan-ya sarisa //
	\glc \AgtT{}= eat=\Fpl{}.\Top{} together lunch-\PargI{} week-\Loc{}
		previous //
	\glft `We had lunch together last week.' //
\endgl\xe
\end{figure}

The reference to a past time frame is explicitly given in (\ref{ex:pstnopst})
by the adverbial phrase \xayr{bihnFy sris}{bihanya sarisa}{last week}, hence
the verb appears here simply as \rayr{koMdjnF}{kondayn}, rather than with
redundant past-tense marking as \rayr{mkoMdjnF}{məkondayn}. Since past tense is
often underspecified in Ayeri, the language also does not employ past forms in
narrative contexts like English\index{English}, among others, commonly does:

\begin{figure}[h!]
\ex\label{ex:neuromancer}
	The sky above the port was the color of television, tuned to a dead 
	channel. \tc{\citep[Ch.~1]{gibson:neuromancer}}
\xe
\end{figure}

This quote is, of course, the famous opening line of
\citeauthor{gibson:neuromancer}'s \citefield{gibson:neuromancer}{origyear}
novel \citetitle{gibson:neuromancer}, which never mentions any definite dates,
but is clearly set in a future world, maybe somewhere in the latter half of the
twenty-first century.\footnote{\citet{christian2017} reports that Gibson
himself pictured his novel as set around 2035, though that he had since
realized that this could not be right. One of the characters, the Finn,
\textquote{makes an offhand reference to the \enquote{Act of '53} as a law
[which] deals with the citizenship status of artificial intelligences}
(\cite{christian2017}; also compare \cite[Ch.~5]{gibson:neuromancer})---this is
very unlikely to refer to 1953.} Yet, however, \citeauthor{gibson:neuromancer}
recounts events which are logically happening in an imagined future as having
already happened in the past: he uses the past tense as a convention of
storytelling. What Ayeri, then, does in contrast to English\index{English}, is to basically
treat stories as though happening in the present\index{tense!present}; adverbials referring to past
time may, again, set up the correct time frame if required. Ayeri is in good
company here, since according to \citet{dahl1985}
\textcquote[113]{dahl1985}{[m]ore common than marking narrative contexts [...]
is not marking them---quite a considerable number of languages use unmarked
verb forms in narrative contexts}. The example in (\ref{ex:nonarrpst}) from an
Ayeri translation of the well-known Aesopian fable, `The North Wind and the
Sun' (compare \cite[39]{ipa2007}, and \autoref{sec:northwind}), illustrates
Ayeri's non-marking of tense on verbs in narrative contexts.

\begin{figure}
\ex\label{ex:nonarrpst}
\begingl
	\gla Ang @ manga @ ranyon adauyi {} @ Pintemis nay Perin, engyo mico 
		sinyāng luga toya, lingya si lugaya asāyāng si sitang-naykonyāng 
		kong tovaya mato. //
	\glb ang= manga= ran-yon adauyi Ø= Pintemis nay Perin eng-yo mico 
		sinya-ang luga toya ling-ya si luga-ya asāya-ang si 
		sitang-naykon=yāng kong tova-ya mato //
	\glc \AgtT{}= \Prog{}= argue-\TplN{} then \Top{}= {North Wind} and 
		Sun, be.more-\TsgN{} strong who-\Aarg{} among \TplN{}.\Loc{},
		while-\Loc{} \Rel{} pass-\TsgM{} traveler-\Aarg{} \Rel{}
		self-wrap=\TsgM{}.\Aarg{} inside cloak-\Loc{} warm. //
	\glft `The North Wind and the Sun were then arguing which among them is 
		stronger, all the while a traveler passed by who had wrapped 
		himself in a warm cloak.' //
\endgl
\xe
\end{figure}

\index{tense!past|)}

\subsubsection{Future tense}
\label{subsubsec:future}
\index{tense!future|(}

Future tense marks explicit references to future time in Ayeri, that is,
\textcquote[103]{dahl1985}{someone's plans, intentions or obligations}, as well
as predictions. The future prefixes\index{prefixes} behave analogous to the ones indicating
past tense: \rayr{p/}{pə-} indicates immediate/near future (\NFut{}),
\rayr{se/}{sə-} indicates impending future (\Fut{}), and \rayr{ni/}{ni-}
indicates remote future (\RFut{}). Underlying the reduced vowels in
\rayr{p/}{pə-} and \rayr{se/}{sə-} are /a/ and /e/, respectively, so that these
prefixes\index{prefixes} cause adjacent vowels of the same type to lengthen as
usual;\index{allomorphy} the same, of course, applies to \rayr{ni/}{ni-}
regarding /i/. The examples in (\ref{ex:futtnses}) show the future tense
markers in context.

\begin{figure}
\pex\label{ex:futtnses}
\a\label{ex:nfut}\begingl
	\gla Pəsahayang! //
	\glb pə-saha=yang //
	\glc \NFut{}-come=\Fsg{}.\Aarg{} //
	\glft `I'm coming (in a moment)!' //
\endgl

\a\label{ex:fut}\begingl
	\gla Ang @ səkarsayn kankaya. //
	\glb ang= sə-kars=ayn.Ø kanka-ya //
	\glc \AgtT{}= \Fut{}-freeze=\Fsg{}.\Top{} snow-\Loc{} //
	\glft `We will freeze in the snow.' //
\endgl

\a\label{ex:rfut}\begingl
	\gla Paronatang, nisa-sahaya dihakayāng. //
	\glb parona=tang ni-sa\til{}saha-ya dihakaya-ang //
	\glc believe=\TplM{}.\Aarg{} \RFut{}-\Iter{}\til{}come-\TsgM{} 
		prophet-\Aarg{} //
	\glft `They believe that the prophet will return (one day).' //
\endgl
\xe
\end{figure}

Like the past tense, the future is often not explicitly marked if the time 
frame is clear from context or has been clarified with such adverbials as 
\xayr{tsel}{tasela}{tomorrow}, \xayr{mNsh perikYnFy}{mangasaha pericanya}{in 
a year}, \xayr{metj}{metay}{sometime}, or \xayr{bihnY mrrY}{bihanya
mararya}{next week}, as in (\ref{ex:futadvbl}). It is possible as well to
explicitly mark the verb for future tense, for example, to make a promise, or
to otherwise emphasize that the future condition will come to pass, as
illustrated in (\ref{ex:futemph}).

\begin{figure}
\ex\label{ex:futadvbl}\begingl
	\gla Ang @ raypāy vaya bihanya mararya. //
	\glb ang= raypa=ay.Ø vaya bihan-ya mararya //
	\glc \AgtT{}= stop=\Fsg{}.\Top{} \Second{}.\Loc{} week-\Loc{} next //
	\glft `I'm stopping by you next week.' //
\endgl\xe
\end{figure}

\begin{figure}
\ex\label{ex:futemph}\begingl
	\gla Səsidejang tasela, diran. //
	\glb sə-sideg=yang tasela diran //
	\glc \Fut{}-repair=\Fsg{}.\Aarg{} tomorrow uncle //
	\glft `I \fw{will} repair it tomorrow, uncle.' //
\endgl\xe
\end{figure}

\index{tense!future|)}

\subsubsection{Past in past}
\index{tense!past|(}

So far, we have only dealt with tense marking from the point of view of the
present. However, it is also possible to refer to an event which precedes
another event in the past. Ayeri does not use auxiliary verbs, so its
morphological and pragmatic means of tense marking have to cover this relation
as well. To indicate pre-past events, it is customary to explicitly mark the
verb for past time in Ayeri, in difference to the common lack of morphological
marking for plain past tense. However, since it is possible for the
\rayr{m/}{mə-} prefix\index{prefixes} to be used to refer to `regular' past events from a
present point of view as well, context again has to provide that the deictic\index{deixis}
origin is a point in the past rather than the speaker's present\index{tense!present}.

\begin{figure}[h]
\ex\label{ex:pstpst}
\begingl
	\glpreamble \textsc{context:} Ajān's past travels //
	\gla Ya @ məsaraya iri maritay ang @ Ajān {} @ Tasankan //
	\glb ya= ma-sara-ya iri maritay ang= Ajān Ø= Tasankan //
	\glc \LocT{}= \Pst{}-go-\TsgM{} already before \Aarg{}= Ajān \Top{}= %
		Tasankan //
	\glft `Tasankan, Ajān had already gone there before.' //
\endgl
\xe
\end{figure}

The example in (\ref{ex:pstpst}) is essentially ambiguous as to the reference
point. The explicit tense marking draws attention to the fact that the event
definitely lies in the past and the adverbs\index{adverbs} underline this fact. Instead of
reading the sentence as referring to a pre-past event, it is equally possible
to read it from a present\index{tense!present}-time point of view as `Ajān has already gone to
Tasankan before', although under these circumstances, it would be more common
to leave the \rayr{m/}{mə-} out, as described in \autoref{subsubsec:past};
compare (\ref{ex:nopstmkr}).

\begin{figure}[h]
\ex\label{ex:nopstmkr}
\begingl
	\glpreamble \textsc{context:} Ajān's current traveling plans //
	\gla Ya @ saraya iri maritay ang @ Ajān {} @ Tasankan //
	\glb ya= sara-ya iri maritay ang= Ajān Ø= Tasankan //
	\glc \LocT{}= go-\TsgM{} already before \Aarg{}= Ajān \Top{}= %
		Tasankan //
	\glft `Tasankan, Ajān has already gone there before.' //
\endgl
\xe
\end{figure}

Likewise, it is possible to make plans in the past with the intention of them
coming to frutition only later, possibly at a point before the current time or
even further in the future\index{tense!future}. The English\index{English} idiom to express this time relation is
`was going to'; in Ayeri, the relation cannot be expressed by morphological
means, but only by lexical ones. Thus, \xayr{no/}{no-}{want; plan to} must be
used, together with explicit past marking. Since \rayr{no}{no} is used as a modal
particle\index{modals} in this context (see \autoref{subsec:modals}), inflection is placed on the content
verb. The time relation expressed in (\ref{ex:pstplan}) is, thus, essentially
that of a pre-past event again, since the planning of the action of buying took
place before the time of going to \rayr{tsMknF}{Tasankan}.

\begin{figure}[h]
\ex\label{ex:pstplan}
\begingl
	\glpreamble \textsc{context}: Ajān's having gone to Tasankan //
	\gla Ang @ no @ məinca tosantangyeley hiro yam @ Pila. //
	\glb ang= no= ma-int=ya tosantang-ye-ley hiro yam= Pila //
	\glc \AgtT{}= want= \Pst{}-buy=\TsgM{}.\Top{} earring-\Pl{}-\PargI{} new 
		\Dat{}= Pila //
	\glft `He had planned to buy new earrings for Pila.' //
\endgl
\xe
\end{figure}

\index{tense!past|)}

\subsubsection{Past in future}
\index{tense!future|(}
\index{tense!past|(}

It is also possible to refer to future actions or events which will already
have happened before a point further in the future. From the point of view of
the later event, the closer event will thus already lie in the past, forming
its prerequisite. As with future-in-past, there is no way in Ayeri to mark this
relation morphologically, but lexical means have to be used, that is, first and
foremost the adverb\index{adverbs} \xayr{Iri}{iri}{already}, which indicates that an action
has been completed in the past. As with other future actions, the time frame
must be inferred from context if it is not indicated explicitly by temporal
adverbs\index{adverbs} or future-tense marking (compare \autoref{subsubsec:future}). Strictly
speaking, (\ref{ex:futperf}) does not make it explicit whether Ajān \emph{will
arrive} before evening or \emph{will have arrived}. In order to indicate that
the action is complete, the cessative adverb\index{adverbs}
\xayr{myis}{mayisa}{be done; ready} may be added, as in (\ref{ex:futlexpfv}).

\begin{figure}[h]
\ex\label{ex:futperf}
\begingl
	\glpreamble \textsc{context}: Ajān's traveling to Tasankan //
	\gla Ang @ girenja iri nilay sirutayya tamala pesan. //
	\glb ang= girend=ya.Ø iri nilay sirutay-ya tamala pesan //
	\glc \AgtT{}= arrive=\TsgM{}.\Top{} already probably evening-\Loc{}
		tomorrow before //
	\glft `He will probably already (have) arrive(d) before 
		tomorrow evening.' //
\endgl
\xe
\end{figure}

\begin{figure}[h]
\ex\label{ex:futlexpfv}
\begingl
	\gla Girenjāng \textbf{mayisa} iri. //
	\glb girend=yāng mayisa iri //
	\glc arrive=\TsgM{}.\Aarg{} be.done already //
	\glft `He already has arrived', \\
		\textit{or:} `He will already have arrived.' //
\endgl
\xe
\end{figure}

\index{tense!past|)}
\index{tense!future|)}

\index{tense|)}

\subsection{Aspect}
\label{subsec:aspect}
\index{aspect|(}

Aspectually unmarked verb forms indicate general statements, which may be
completed or ongoing, depending on the meaning of the verb itself. Ayeri seems
not to make strict formal distinctions with regards to either, perfectivity or
lexical aspect. It needs to be noted, however, that at least to date, it is not
entirely clear how Ayeri treats perfectivity, which \citet[76]{dahl1985} in
reference to \citet[16]{comrie1976} characterizes as being based on the
conceptualization of actions or events as bounded or otherwise limited wholes,
versus a lack of closure. \citet{dahl1985} also notes that
\textcquote[69]{dahl1985}{it seems rather to be a typical situation that even
in individual languages, we cannot choose one member of the opposition
[perfective--imperfective] as being clearly unmarked}. He further argues that

\blockcquote[73]{dahl1985}{[t]he difficulty of deciding which member of the
opposition is marked and which is unmarked is connected with the tendency for
\textsc{pfv:ipfv} to be realized not by affixation or by periphrastic
constructions but rather by less straightforward morphological processes.}

In other words: it \emph{is} a difficult category to assess, in spite of being
\textcquote[69]{dahl1985}{often taken to be \enquote{the} category of aspect},
mostly since languages often do not realize it by straightforward means. In
Ayeri, the most tangible way of expressing completeness of an action is to
use adverbs\index{adverbs} like \xayr{myis}{mayisa}{ready, done}, \xayr{Iri}{iri}{already},
\xayr{IknF}{ikan}{completely, wholly} (also as an adjective); a quantifier\index{quantifiers} like
\xayr{/henF}{-hen}{all}; verbs like \xayr{smirF/}{samir-}{finish},
\xayr{pN/}{panga-}{end}, and \xayr{rjp/}{raypa-}{stop}; or an indefinite
pronoun\index{pronouns!indefinite} expressing entirety, like \xayr{EnY}{enya}{everything, everybody} in
(\ref{ex:entirepfv}).

\begin{figure}[h]
\ex\label{ex:entirepfv}\begingl
	\gla Le @ kondjeng enya. //
	\glb le= kond=yeng enya-Ø //
	\glc \PatTI{}= eat=\TsgF{}.\Aarg{} everything //
	\glft `She ate everything', \\
		\textit{or:} `She ate it all up.' //
\endgl\xe
\end{figure}

Apart from the more general dilemma of determining how perfectivity is
expressed in detail, Ayeri marks verbs openly by morphological means to
indicate progressive, habitual, and iterative actions---by their nature all
conceptualizing actions as being composed of a series of two or more related
actions of the same kind, though not necessarily implying a strong semantic
connection to the past\index{tense!past}. The following sections will discuss each of these
categories.

\subsubsection{Progressive}
\index{aspect!progressive|(}

In order to indicate an ongoing action explicitly, Ayeri employs the marker
\rayr{mN}{manga}, which we already saw with directional prepositions above
(\autoref{manga}). This clitic\index{clitics} attaches to the immediate left of the verb, as
displayed in (\ref{ex:presprog}).

\begin{figure}[h]
\ex\label{ex:presprog}\begingl
	\gla Ang @ manga @ ilye karonas nakajyam. //
	\glb ang= manga= il=ye.Ø karon-as naka-ye-yam //
	\glc \AgtT{}= \Prog{}= give=\TsgF{}.\Top{} water-\Parg{} 
		plant-\Pl{}-\Dat{} //
	\glft `She is giving water to the plants.' //
\endgl\xe
\end{figure}

Going by the data presented by \citet[91]{dahl1985}, Ayeri is typologically\index{typology}
unremarkable in marking progressive aspect with a periphrastic construction,
although it is remarkable in possessing morphological progressive marking at
all---morphological progressive marking only occurs in 27\pct{} of the
languages in \citet{dahl1985}'s sample. Typical of progressives, this form of
the verb is not limited to present\index{tense!present} contexts in Ayeri as exemplified in
(\ref{ex:presprog}) above. Instead, it is possible to also use the progressive
in past\index{tense!past} (\ref{ex:pastprog}) and future\index{tense!future} (\ref{ex:futprog}) contexts, the latter
being probably less typical, though.

\begin{figure}
\pex\label{ex:nonpresprog}
\a\label{ex:pastprog}\begingl
	\gla Ang @ manga @ gumya {} @ Ajān tadayya si ya @ kongaye ang @ Pila
		gumanga tamala. //
	\glb ang= manga= gum-ya Ø= Ajān taday-ya si ya= konga-ye ang= Pila 
		gumanga-Ø tamala //
	\glc \AgtT{}= \Prog{}= work-\Tsg{} \Top{}= Ajān time-\Loc{} \Rel{} \LocT{}=
		enter-\TsgF{} \Aarg{}= Pila workshop-\Top{} yesterday //
	\glft `Ajān was working when Pila entered the workshop yesterday.' //
\endgl

\a\label{ex:futprog}\begingl
	\gla Ang @ manga @ nimpay rangya tadayya si cunyo bekalang tasela. //
	\glb ang= manga= nimp=ay.Ø rang-ya taday-ya si cun-yo bekal-ang tasela //
	\glc \AgtT{}= \Prog{}= run=\Fsg{}.\Top{} home-\Loc{}
		time-\Loc{} \Rel{} begin-\TsgN{} festival-\Aarg{} tomorrow //
	\glft `I will be running home when when the festival starts 
		tomorrow.' //
\endgl
\xe
\end{figure}

Ignoring the constructedness of the above examples, the time adverb\index{adverbs} is located 
in the relative clause\index{relative clause} in both sentences in this case. For illustrative 
purposes, let us assume that a narrative context with the respective time 
frames has already been established in (\ref{ex:nonpresprog}). As noted above, 
Ayeri prefers not to mark every verb for tense\index{tense} explicitly when the context is 
clear already, insofar the argument that progressive aspect works independent 
of \fw{tense}\index{tense} needs corrobation; the question being whether constructions like 
\rayr{mN m/—}{manga mə-...} (\Prog{}=\Pst{}-...) are possible. Strictly 
speaking, there is nothing to prevent this construction, however, we have to 
wonder if it is actually \fw{natural} to phrase things this way. What can be 
said at least is that progressive marking is possible within a context 
referring to past\index{tense!past} or future\index{tense!future} actions and events irrespective of their explicit 
marking on the verb. Furthermore, the examples in (\ref{ex:nonpresprog}) 
illustrate a very typical use of the progressive as a structuring means, that 
is, an ongoing background action may be expressed using a progressive form, 
while an interrupting action receives no special marking (compare the past 
progressive in English\index{English}).

\index{aspect!progressive|)}

\subsubsection{Habitual}
\index{aspect!habitual|(}

Unlike the few instances of habitual marking in \citet{dahl1985}'s survey
\citep[96]{dahl1985}, Ayeri possesses a suffix\index{suffixes} for marking habitual actions on
the verb: \rayr{/As}{-asa}, where the first \fw{-a} replaces the terminal vowel
of a verb stem if present, compare example (\ref{ex:habvwl}) below.\index{morphophonology} The
habitual aspect in Ayeri stresses that an action is carried out as a habit,
that is, not just a few times, but with regular frequency. Essentially, verbs
marked with the habitual in Ayeri can be translated by adding the adverb\index{adverbs}
\fw{usually} in English\index{English} \citep[97]{dahl1985}. The habitual aspect is not
restricted to present\index{tense!present} actions or absolute statements like the one in
(\ref{ex:habcns}), but can also be used in past\index{tense!past} contexts to express that
something \fw{used to} be done in the past\index{tense!past}, as in (\ref{ex:habvwl}). While the
contexts are probably very few, there are no restrictions about using the
habitual also in contexts relating to future\index{tense!future} actions which are predicted to be
carried out habitually. Importantly, the verb root with habitual marking forms
a new verb stem to which affixes may be attached. This is relevant for mood\index{mood}
suffixes\index{suffixes}, which follow\index{affix order} aspectual marking.

\begin{figure}[h]
\pex
\a\label{ex:habcns}\begingl
	\gla Le kondasayāng hemaye pruyya nay napayya kayvay. //
	\glb le kond-asa=yāng hema-ye-Ø pruy-ya nay napay-ya kayvay //
	\glc \PatTI{} eat-\Hab{}=\TsgM{}.\Aarg{} egg-\Pl{}-\Top{} salt-\Loc{} 
		and pepper-\Loc{} without //
	\glft `He always eats his eggs without salt and pepper.' //
\endgl

\a\label{ex:habvwl}\begingl
	\gla Ang ajasāyn ranisungas tadayya si yāng ganas. //
	\glb ang aja-asa=ayn.Ø ranisung-as taday-ya si yāng gan-as //
	\glc \AgtT{} play-\Hab{}=\Fpl{}.\Top{} hide.and.seek-\Parg{} 
		time-\Loc{} \Rel{} \Fsg{}.\Aarg{} child-\Parg{} //
	\glft `We used to play hide-and-seek when I was a child.' //
\endgl
\xe
\end{figure}

\index{aspect!habitual|)}

\subsubsection{Iterative}
\label{subsubsec:iterative}
\index{aspect!iterative|(}

The iterative aspect marks actions that are repeated at least once by
reduplication\index{reduplication}. The equivalent in English\index{English} is to use the adverb \fw{again} or the
prefix \fw{re-}. Iterative reduplication\index{reduplication} in Ayeri is only partial, in that only
the initial CV- or VC- of a verb root is repeated---there are no verb roots
which consist of only a single consonant or vowel. Complications begin,
however, if the verb root starts with a consonant cluster (not unusual), or a
diphtong (rare). In the case of an intial consonant cluster, the cluster is
simplified to only include the first consonant; for initial diphthongs, there
is no necessity to include the first available consonant, since the secondary
vowel of a diphthong can by itself act as a semivowel to make up for the vowel
hiatus.

\begin{figure}[h]
\ex\labels\label{ex:itermorph}
	\begin{tabular}[t]{@{\tl\quad} l @{\enspace→\enspace} l @{\smallskip}}
	\xayr{\larger kut/}{kuta-}{thank}
		& \xayr{\larger ku/kut/}{ku-kuta-}{thank again}
		\\
	\xayr{\larger AmNF/}{amang-}{happen}
		& \xayr{\larger AmF/AmNF/}{am-amang-}{happen again}
		\\
	\xayr{\larger pFrMtF/}{prant-}{ask}
		& \xayr{\larger p/pFrMtF/}{pa-prant-}{ask again}
		\\
	\xayr{\larger AjrinF/}{ayrin-}{set}
		& \xayr{\larger Aj/AjrinF/}{ay-ayrin-}{set again}
		\\
	\end{tabular}
\xe
\end{figure}

The words listed in (\ref{ex:itermorph}) are examples of verbs and their
reduplicated form for the purpose of iterative marking. An example for each of
the previously mentioned onset types is included: \rayr{kut/}{kuta-}
exemplifies a CV onset, \rayr{AmNF/}{amang-} a VC one; \rayr{pFrMtF/}{prant-}
has a CCV onset which is simplified to CV in the reduplicated form\index{morphophonology}, and
\rayr{AjrinF/}{ayrin-} begins with a diphthong. The reduplicated stem in each
case functions as a new stem for other prefixes\index{prefixes}, that is, no morphological
material can go between the reduplicated part and the lexical stem proper.
Besides giving an example of the correct and incorrect order\index{word order} of attachment of
the past\index{tense!past} prefix\index{prefixes} \rayr{m/}{mə-} with a partially reduplicated verb, the example
in (\ref{ex:preford}) also shows that there is, again, no restriction on the
iterative aspect with regards to tense\index{tense}.

\begin{figure}[h]
\pex\label{ex:preford}
\a\begingl
	\gla Məku-kutayāng. //
	\glb mə-ku\til{}kuta=yāng //
	\glc \Pst{}-\Iter{}\til{}thank=\TsgM{}.\Aarg{} //
	\glft `He thanked again.' //
\endgl

\a\ljudge* \fw{Ku-məkutayāng.}
\xe
\end{figure}

Iterative reduplication\index{reduplication} is lexicalized at least in one verb, 
\xayr{s/sh/}{sa-saha-}{return}. Besides the meaning `again', iterative
reduplication\index{reduplication} may also indicate the meaning `back', as in (\ref{ex:backiter}).

\begin{figure}[h]
\ex\label{ex:backiter}\begingl
	\gla Ta-tapyu adaley! //
	\glb ta\til{}tapy-u ada-ley //
	\glc \Iter{}\til{}put-\Imp{} that-\PargI{} //
	\glft `Put that back!' //
\endgl\xe
\end{figure}

In addition to a simple iterative meaning, a frequentative meaning like `walk 
around', `cry all the time', or `keep asking' can be achieved by combining the 
iterative and progressive aspects\index{aspect!progressive}, that is, the verb is both modified by 
\rayr{mN}{manga} for progressive aspect\index{aspect!progressive} and partial initial reduplication\index{reduplication} for 
iterative aspect. Examples of this combination of aspectual marking are given
in (\ref{ex:progiter}).

% FIXME: Can these be nominalized? If so, how?
\begin{figure}[h]
\pex\label{ex:progiter}
\a\begingl
	\gla Ang @ manga @ la-lampay saha-sara manga @ luga bahisya-hen. //
	\glb ang= manga= la\til{}lamp=ay.Ø saha-sara manga= luga bahis-ya=hen //
	\glc \AgtT{}= \Prog{}= \Iter{}\til{}walk=\Fsg{}.\Top{} back.and.forth 
		\Dir{}= while day-\Loc{}=all //
	\glft `I was walking around back and forth all day long.' //
\endgl

\a\begingl
	\gla Ang @ manga @ si-sipye kimay sirutayya. //
	\glb ang= manga= si\til{}sip-ye kimay-Ø sirutay-ya //
	\glc \AgtT{}= \Prog{}= \Iter{}\til{}cry-\TsgF{} baby-\Top{} 
		night-\Loc{} //
	\glft `The baby, she is crying all the time at night.' //
\endgl

\a\begingl
	\gla Manga @ pa-prantu! //
	\glb manga= pa\til{}prant-u //
	\glc \Prog{}= \Iter{}\til{}ask-\Imp{} //
	\glft `Keep asking!' //
\endgl
\xe
\end{figure}

\index{aspect!iterative|)}

\subsubsection{Lexically marked aspectual categories}

Besides using morphological means, Ayeri expresses some aspectual categories by
way of lexical items, that is, verbs and adverbs\index{adverbs}. The relevant words in this
respect are the adverbs\index{adverbs} \xayr{sirimNF}{sirimang}{about to} (prospective) and
\xayr{myis}{mayisa}{ready; be done} (cessative), as well as 
the verb \xayr{kYunF/}{cun-}{begin, start} (inchoative).

\begin{figure}[h]
\pex
\a\label{ex:prospective}
\begingl
	\gla Saratang sirimang. //
	\glb sara=tang sirimang //
	\glc leave=\TplM{}.\Aarg{} about.to //
	\glft `They are about to leave.' //
\endgl

\a\label{ex:cessative}
\begingl
	\gla Konjang mayisa. //
	\glb kond=yang mayisa //
	\glc eat=\Fsg.\Aarg{} be.done //
	\glft `I am done eating.' //
\endgl

\a\label{ex:inchoative}
\begingl
	\gla Pəcunreng seyaryam. //
	\glb pə-cun=reng seyar-yam //
	\glc \NFut{}-begin=\TsgI{}.\Aarg{} rain-\Ptcp{} //
	\glft `It is going to start raining any moment.' //
\endgl
\xe
\end{figure}

Prospective \rayr{sirimNF}{sirimang} (\ref{ex:prospective}) and cessative
\rayr{myis}{mayisa} (\ref{ex:cessative}) are expressed by adverbs\index{adverbs} which are
regularly following verbs as their heads. They precede\index{word order} other adverbs\index{adverbs} due to a
higher amount of semantic bondedness, by tendency, than other descriptive
adverbs\index{adverbs}. For this reason, as well as for expressing a grammatical function
rather than lexical meaning with the original meaning still transparent, they
appear to be on the verge of grammaticalization\index{grammaticalization}. In contrast, the inchoative
verb \rayr{kYunF/}{cun-} (\ref{ex:inchoative}) is part of a periphrastic verb
construction, that is, \rayr{kYunF/}{cun-} requires a content-verb VP\index{phrase types!verb phrase} as a
complement\index{grammatical function!open complement} rather than an NP\index{phrase types!noun phrase}. The content/main verb appears in a non-finite\index{verbs!non-finite}
form marked by \rayr{/ymF}{-yam}, which will be described from a morphological
perspective in \autoref{subsec:participle}, and in \autoref{subsec:vps} from
that of syntax.

\index{aspect|)}

\subsection{Mood}
\label{subsec:mood}
\index{mood|(}

Besides various aspects\index{aspect}, Ayeri also marks mood other than realis: irrealis,
imperative, hortative, and negative. These are expressed by suffixes\index{suffixes} on
the verb and typically follow\index{affix order} aspectual\index{aspect} marking where it is expressed by a
suffix\index{suffixes}, that is, the habitual suffix\index{suffixes} \rayr{As/}{-asa}. The following
subsections will discuss each of the modal categories expressed by suffixes\index{suffixes};
modal particles proper will be discussed in \autoref{subsec:modals}.

\subsubsection{Irrealis}
\index{mood!irrealis|(}

Irrealis marking in Ayeri is indicated by the suffix\index{suffixes} \rayr{/ONF}{-ong} and 
marks that an action is thought of as hypothetical by the speaker, whether he 
or she expects it to be realized or not:

\begin{figure}[h]
\ex\label{ex:irrealis}\begingl
	\gla Sahongvāng edaya, ming @ silvongvāng sitang-vāri. //
	\glb saha-ong=vāng edaya ming= silv-ong=vāng sitang=vāri //
	\glb come-\Irr{}=\Second{}.\Aarg{} here can= see-\Irr{}=\Second{}.\Aarg{} 
		\Refl{}=\Second{}.\Ins{} //
	\glft `If you came/had come here, you could see/have seen it 
		yourself.' //
\endgl\xe
\end{figure}

As (\ref{ex:irrealis}) shows, irrealis marking is especially prominent in
conditional clauses which express a hypothetical cause and effect. Both
condition/protasis and consequence/apodosis are marked with the irrealis suffix\index{suffixes}
in this case. The example sentence also shows that, again, the initial vowel of
the suffix\index{suffixes} replaces the last vowel of the verb stem if there is one\index{morphophonology}, so that
\rayr{sh/}{saha-} becomes \rayr{shoNF/}{sahong-}, to which further mood
suffixes\index{suffixes} may be added, and finally, person\index{person} marking. The same suffix\index{suffixes},
\rayr{/ONF}{-ong}, is also used in other contexts expressing inactual events,
for instance, in reported speech such as in (\ref{ex:irrrepspch}), or in
complement clauses\index{complement clause} expressing a wish about the actualization of a hypothetical
event, as in (\ref{ex:irrwish}).

\begin{figure}[h]
\pex
\a\label{ex:irrrepspch}\begingl
	\gla Narayeng, ang @ menongye demās yena. //
	\glb nara=yeng ang= menu-ong=ye.Ø dema-as yena //
	\glc say=\TsgF{}.\Aarg{} \AgtT{}= visit-\Irr{}=\TsgF{}.\Top{} 
		aunt-\Parg{} \TsgF{}.\Gen{} //
	\glft `She said she were visiting her aunt.' //
\endgl

\a\label{ex:irrwish}\begingl
	\gla Hanuyang, koronongyang maritay. //
	\glb hanu=yang koron-ong=yang maritay //
	\glc wish=\Fsg{}.\Aarg{} know-\Irr{}=\Fsg{}.\Aarg{} before //
	\glft `I wish I had known this before.' //
\endgl
\xe
\end{figure}

Irrealis marking does not, however, appear in contexts that express
requirements on or wishes about a third person's actions, that is, typical
subjuctive contexts; the verb in the complement clause\index{complement clause} rather appears in the
indicative\index{mood!indicative} in these contexts. To add a sense of expectation of compliance about
the action, the modal\index{modals} \xayr{mY}{mya}{be supposed to, shall} may be added, see
\autoref{subsec:modals}. Example (\ref{ex:notirr}) gives a sentence expressing
requirement. As (\ref{ex:subjfalse}) shows, a rendition with the wished-for
action in the irrealis mood is ungrammatical, while the rendition with an
optional \rayr{mY}{mya} and an otherwise plain verb in (\ref{ex:myashall}) is
acceptable.

\begin{figure}[h]
\pex\label{ex:notirr}
\a\label{ex:subjfalse}\ljudge*\begingl
	\gla Arapnang, sa @ garongyāng hatay. //
	\glb arap=nang sa= gara-ong=yāng hatay-Ø //
	\glc require=\Fpl{}.\Aarg{} \PatT{}= call-\Irr{}=\TsgM{}.\Aarg{}
		police-\Top{} //
\endgl

\a\label{ex:myashall}\begingl
	\gla Arapnang, sa @ \textup{(}mya\textup{)} @ garayāng hatay. //
	\glb arap=nang sa= (mya=) gara=yāng hatay-Ø //
	\glc require=\Fpl{}.\Aarg{} \PatT{}= (shall=) call=\TsgM{}.\Aarg{}
		police-\Top{} //
	\glft `We require that he call the police.' //
\endgl
\xe
\end{figure}

\index{mood!irrealis|)}

\subsubsection{Negative}
\label{subsubsec:verbneg}
\index{mood!negative|(}
\index{negation|(}

The negative mood is used to negate verbs, which is separate from irrealis\index{mood!irrealis}
marking: negation of verbs is marked by the suffix\index{suffixes} \rayr{/Oj}{-oy}, which has
an allomorph\index{allomorphy} \fw{-u} before diphthongs in romanization and also in
pronunciation. The Tahano Hikamu\index{Tahano Hikamu} spelling is more conservative here and keeps
the spelling \ayr{/Oyj} \orth{-oyay} for [waɪ]
(\mbox{-\Neg{}=\Fsg{}.\Top{}}). Like the irrealis\index{mood!irrealis} suffix\index{suffixes}, the
negative suffix\index{suffixes} deletes the last vowel of the verb stem if present\index{morphophonology}, which is
exemplified in (\ref{ex:negallo}) besides this example showing the \fw{-u}
allomorph\index{allomorphy}. Moreover, example (\ref{ex:irrneg}) shows that negative marking
usually follows\index{affix order} irrealis\index{mood!irrealis} marking when suffixes\index{suffixes} are stacked: \rayr{/ONF}{-ong} +
\rayr{/Oj}{-oy} → \rayr{/ONoj}{-ongoy}.

\begin{figure}[h]
\pex
\a\label{ex:negative}\begingl
	\gla Ang @ silvoyyan nasiyamanas tan. //
	\glb ang= silv-oy=yan.Ø nasi-yam-an-as tan //
	\glc \AgtT{}= see-\Neg{}=\TplM{}.\Top{} approach-\Ptcp{}-\Nmlz{}-\Parg{} 
		\TplM{}.\Gen{} //
	\glft `They did not see them approaching.' //
\endgl

\a\label{ex:negallo}\begingl
	\gla Ang @ peguay kalam adaley! //
	\glb ang= pega-oy=ay.Ø kalam ada-ley //
	\glc \AgtT{}= steal-\Neg{}=\Fsg{}.\Top{} honestly that-\PargI{} //
	\glft `I didn't steal it, honestly!' //
\endgl

\a\label{ex:irrneg}\begingl
	\gla Tendongoyvang sarayam adaya. //
	\glb tend-ong-oy=vang sara-yam adaya //
	\glc dare-\Irr{}-\Neg{}=\Second{}.\Aarg{} go-\Ptcp{} there //
	\glft `You would not dare to go there.' //
\endgl
\xe
\end{figure}

If negated verbs appear together with negative indefinite pronouns\index{pronouns!indefinite} (compare 
\autoref{subsec:indefpro}), multiple negatives as displayed in
(\ref{ex:dblneg}) do not cancel each other out, but amplify the negation
instead. This is to say that Ayeri allows for multiple
negation as a means to emphasize the impossibility of something.

\begin{figure}[h]
\ex\label{ex:dblneg}
\begingl
	\gla Le @ gamaroyya tadoy ranyāng adanya. //
	\glb le= gamar-oy-ya tadoy ranyāng adanya-Ø //
	\glc \PatTI{}= manage-\Neg{}-\TsgM{} never nobody-\Aarg{} that-\Top{} //
	\glft `\emph{Nobody ever} managed that',\\
		\textit{literally:} `Nobody never didn't manage that.' //
\endgl
\xe
\end{figure}

\index{negation|)}
\index{mood!negative|)}

\subsubsection{Imperative}
\index{mood!imperative|(}

The imperative mood is used to mark orders to an unspecified second person\index{person},
that is, imperative verbs do not require an overt second person\index{person} agent\index{semantic role!agent}; if an
addressee is included, as in (\ref{ex:impmorph}a), it is unmarked for case\index{case},
see \autoref{subsec:uncased}. Moreover, no distinction\index{number} is made between singular
and plural second-person\index{person} addressees, so that the marker is \rayr{/U}{-u} in
either case. Like the other mood suffixes\index{suffixes}, the vowel of the imperative suffix\index{suffixes}
replaces the vowel of the verb stem if there is one\index{morphophonology}, as in
(\ref{ex:impmorph}b), where \rayr{gir/}{gira-} is shortened to
\rayr{girF/}{gir-} before appending the imperative marker.

\begin{figure}[h]
\ex\labels\label{ex:impmorph}
\begin{minipage}[t]{.5\remaining}
\tl\quad\begingl
	\gla Tangu yām, Yan! //
	\glb tang-u yām Yan //
	\glc listen-\Imp{} \Fsg{}.\Dat{} Yan //
	\glft `Listen to me, Yan!' //
\endgl
\end{minipage}
~
\begin{minipage}[t]{.5\remaining}
\tl\quad\begingl
	\gla Giru māy! //
	\glb gira-u māy //
	\glc hurry-\Imp{} \Int{} //
	\glft `Hurry up!' //
\endgl
\end{minipage}

% \a\begingl
% 	\gla Tangu yām, ledanye nā! //
% 	\glb tang-u yām ledan-ye nā //
% 	\glc listen-\Imp{} \Fsg{}.\Dat{} friend \Fsg{}.\Gen{} //
% 	\glft `Listen to me, my friends!' //
% \endgl
\xe
\end{figure}

Notably, imperative-marked verbs behave essentially as non-finite\index{verbs!non-finite} forms in that
they do not exhibit any agreement\index{agreement} in person, number, gender, and topic\index{grammatical function!topic}, and also
cannot act as hosts for clitic personal pronouns. Imperative verbs may be marked
for negative\index{mood!negative} and hortative\index{mood!hortative} mood, however. Hence, for instance, (\ref{ex:negimp})
is grammatical, while the examples in (\ref{ex:agrimp}) are not.

\begin{figure}[h]
\ex\label{ex:negimp}\begingl
	\gla Saroyu yas! //
	\glb sara-oy-u yas //
	\glc leave-\Neg{}-\Imp{} \Fsg{}.\Parg{} //
	\glft `Don't leave me!' //
\endgl\xe
\end{figure}

\begin{figure}[h]
\pex\label{ex:agrimp}
\a\label{ex:topimp}\ljudge*\begingl
	\gla Ya @ sa-sahu nanga! //
	\glb ya= sa\til{}saha-u nanga-Ø //
	\glc \LocT{}= \Iter{}\til{}go-\Imp{} house-\Top{} //
	\glft `Go back to the house!' //
\endgl

\a\label{ex:persimp}\ljudge*\begingl
	\gla Sa @ sutamuya kohanya tasela! //
	\glb sa= sutam-u=ya.Ø kohan-ya tasela //
	\glc \PatT{}= hang-\Imp{}=\TsgM{}.\Top{} sunrise-\Loc{} tomorrow //
	\glft `May he be hanged tomorrow at sunrise!' //
\endgl
\xe
\end{figure}

Example (\ref{ex:negimp}) simply expresses a negative\index{mood!negative} command, which is
unproblematic in terms of logic, since commands may be issued to act in a
certain way, or to refrain from this action. Example (\ref{ex:topimp}) shows
the imperative verb as preceded by a locative\index{case!locative} topic marker, which is not
logically impossible, but unacceptable by convention.\footnote{The translation
of `\citetitle{shelley:ozymandias}' in \autoref{sec:ozymandias} deviates from
this rule in the line \xayr{s silFvu gumo naa}{sa silvu gumo nā}{my works,
behold them}. This is poetic license, however.} Example (\ref{ex:persimp})
takes this one step further by displaying a cliticized object pronoun\index{pronouns!personal} in the
fashion of morphological passives\index{voice!passive} (compare \autoref{subsec:persnumagr}). This
is likewise ungrammatical, however, since imperatives generally imply a direct
order to a second-person\index{person} addressee, not an indirect order to arrange for a
third person\index{person} to be acted on.

\index{mood!imperative|)}

\subsubsection{Hortative}
\index{mood!hortative|(}

The hortative is a special kind of imperative\index{mood!imperative} which addresses a group 
including the speaker. Its implied referent is thus first-person\index{person} plural\index{number}. 
Again, it is not necessary to mark the verb for the addressee here. Since 
the hortative is related in meaning to the imperative\index{mood!imperative}, the verb also uses the 
imperative\index{mood!imperative} inflection with \rayr{/U}{-u}, but it is fully reduplicated in 
addition to mark the difference. As regards agreement\index{agreement} morphology, the same 
restrictions as with imperatives\index{mood!imperative} apply.

\begin{figure}[h]
\ex\labels\label{ex:hortmorph}
\begin{minipage}[t]{.5\remaining}
\tl\quad\begingl
	\gla Sahu! //
	\glb saha-u //
	\glc go-\Imp{} //
	\glft `Go!' //
\endgl
\end{minipage}
~
\begin{minipage}[t]{.5\remaining}
\tl\quad\begingl
	\gla Sahu-sahu umangya! //
	\glb sahu\til{}saha-u umang-ya //
	\glc \Hort{}\til{}go-\Imp{} beach-\Loc{} //
	\glft `Let's go to the beach!' //
\endgl
\end{minipage}
\xe
\end{figure}

(\ref{ex:hortmorph}a) again gives an example of an imperative\index{mood!imperative},
\rayr{shu}{sahu}, addressing a second person singular or plural.
(\ref{ex:hortmorph}b), on the other hand, shows the corresponding hortative
form, \rayr{shu/shu}{sahu-sahu}, in which a group including the speaker is
addressed.

\index{mood!hortative|)}

\index{mood|)}

\subsection{Modals}
\label{subsec:modals}
\index{modals|(}

\begin{table}
\caption{Modal verbs and particles}
\begin{tabu} to \linewidth {C[3] I[2] I[2] X[4]}
\tableheaderfont\toprule
Category
	& Verb
	& Particle
	& Translation
	\\
\toprule

ability
	& ming- % \ayr{miNF/} \fw{ming-}
	& ming % \ayr{miNF} \fw{ming}
	& `be able to, can'
	\\
	
\midrule
	
desire, intention
	& vac- % \ayr{vtYF/} \fw{vac-}
	& vaca % \ayr{vtY} \fw{vaca}
	& `like to'
	\\
	
	& no- % \ayr{no/} \fw{no-}
	& no % \ayr{no} \fw{no}
	& `want to'
	\\
	
\midrule

permission
	& kila- % \ayr{kil/} \fw{kila-}
	& kila % \ayr{kil} \fw{kila}
	& `be allowed to, may'
	\\
	
\midrule

requirement
	& ilta- % \ayr{IlFt/} \fw{ilta-}
	& ilta % \ayr{IlFt} \fw{ilta}
	& `need to'
	\\
	
\midrule

obligation
	& mya- % \ayr{mY/} \fw{mya-}
	& mya % \ayr{mY} \fw{mya}
	& `be supposed to, shall'
	\\
	
	& rua- % \ayr{ru\_a/} \fw{rua-}
	& rua % \ayr{ru\_a} \fw{rua}
	& `have to, must'
	\\
	
\midrule
	
continuation
	& div- % \ayr{divF/} \fw{div-}
	& diva % \ayr{div} \fw{diva}
	& `stay, remain'
	\\

\bottomrule
\end{tabu}
\label{tab:modverb}
\end{table}

Modals in Ayeri express the notions of ability, desire, permission,
requirement, obligation, and also of continuation, as indicated by
\autoref{tab:modverb}. They can generally act as both fully inflectable
intransitive verbs\index{verbs!intransitive}, as well as clitics\index{clitics} which occur in combination with fully
inflected content verbs.

\begin{figure}[h]
\pex
\a\label{ex:modalinvar}\begingl
	\gla Rua @ bahavāng baho, ang @ bihanoyya mirampaluy nas. //
	\glb rua= baha=vāng baho ang= bihan-oy=ya.Ø mirampaluy nas //
	\glc must= shout=\Second{}.\Aarg{} loudly \AgtT{}= 
		understand-\Neg{}=\TsgM{}.\Top{} otherwise \Fpl{}.\Parg{} //
	\glft `You have to shout loudly, otherwise he does not understand 
		us.'//
\endgl

\a\label{ex:modalinfl}\begingl
	\gla Ruasanang. //
	\glb rua-asa=nang //
	\glc must-\Hab{}=\Fpl{}.\Aarg{} //
	\glft `We usually have to.' //
\endgl
\xe
\end{figure}

As (\ref{ex:modalinvar}) shows, the modal does not inflect in combination with
another verb; as a clitic\index{clitics} it rather acts similar to a prefix\index{prefixes}, like the
progressive marker \rayr{mN}{manga}, which is also presumably deverbal
(compare \autoref{sec:typology}, \autoref{fn:mangaverb}). In difference to
\rayr{mN}{manga}, which as a preverbal element only serves a grammatical
function, the semantic component of the modals is still prevalent. This is
illustrated by (\ref{ex:modalinfl}), where \rayr{ru\_a/}{rua-} appears in its
function as an intransitive verb\index{verbs!intransitive} with the same meaning of strong obligation as
in (\ref{ex:modalinvar}), though it carries regular person and aspect
inflection here. Inflecting the modal in the context of cooccurrence with a
content verb is considered unacceptable, however, as (\ref{ex:modalinfl_2})
shows.

\begin{figure}[h]
\ex\label{ex:modalinfl_2}\ljudge*\begingl
	\gla Ruavāng bahayam baho. //
	\glb rua=vāng baha-yam baho //
	\glc must=\Second{}.\AgtT{} shout-\Ptcp{} loudly //
	\glft `You have to shout loudly.' //
\endgl\xe
\end{figure}

Regarding example (\ref{ex:modalinfl}) and the modal's ability to inflect,
Ayeri also has a verb that generally means `do', namely, \rayr{mir/}{mira-}.
However, it is not common to use this verb as a dummy to carry the inflection
instead of the modal verb like in (\ref{ex:modaldummy}) either. While such a
construction is not ungrammatical \fw{per se}, it is simply not the preferred
way to express intransitive\index{verbs!intransitive} modal verbs.

\begin{figure}[h]
\ex\label{ex:modaldummy}\ljudge\ques\begingl
	\gla Rua @ mirasanang. //
	\glb rua= mira-asa=nang //
	\glc must= do-\Hab{}=\Fpl{}.\Aarg{} //
	\glft `We usually have to.' //
\endgl\xe
\end{figure}

While most of the verbs listed in \autoref{tab:modverb} should look reasonable
to English speakers, Ayeri uses two verbs for modal particles\index{modals} which may seem
odd: \xayr{vtY}{vaca}{like to}, to express taking pleasure in doing something,
and \xayr{div}{diva}{stay, remain}, to express that the action is being
prolonged. The latter verb thus also has an aspectual\index{aspect} component to its meaning.

\begin{figure}[h]
\pex\label{ex:vacvaca}
\a\label{ex:vacfull}\begingl
	\gla Ang @ vacay betayley. //
	\glb ang= vac=ay.Ø betay-ley //
	\glc \AgtT{}= like=\Fsg{}.\Top{} berry-\PargI{} //
	\glft `I like berries.' //
\endgl

\a\label{ex:vacamod}\begingl
	\gla Ang @ vaca @ konday betayley. //
	\glb ang= vaca= kond=ay.Ø betay-ley //
	\glc \AgtT{}= like= eat=\Fsg{}.\Top{} berry-\PargI{} //
	\glft `I like to eat berries.' //
\endgl
\xe
\end{figure}

\begin{figure}[h]
\pex\label{ex:divdiva}
\a\label{ex:divfull}\begingl
	\gla Ang @ divay rangya tasela. //
	\glb ang= div=ay.Ø rang-ya tasela //
	\glc \AgtT{}= stay=\Fsg{}.\Top{} home-\Loc{} tomorrow //
	\glft `I will stay home tomorrow.' //
\endgl

\a\label{ex:divamod}\begingl
	\gla Ang @ diva @ bengya ku-danyās kebay. //
	\glb ang= diva= beng=ya.Ø ku=danya-as kebay //
	\glc \AgtT{}= stay= stand=\TsgM{}.\Top{} like=one-\Parg{} alone //
	\glft `He remained standing as the only one.' //
\endgl
\xe
\end{figure}

The fact that modal particles\index{modals} in Ayeri retain their verbal semantics in spite
of shedding verb morphology is probably even more obvious from examples
(\ref{ex:vacvaca}) and (\ref{ex:divdiva}), which show the alternation between
full-verb use in (a) and modal use in (b) for both \rayr{vtYF/}{vac-} and
\rayr{divF/}{div-}. In comparison to the other modals in 
\autoref{tab:modverb}, these two verbs in particular also stand out by virtue 
of their roots ending in a consonant instead of a vowel like in the other 
cases. This suggests that they may have been grammaticalized as modals 
only relatively recently, and there appears to be variation at least for 
\rayr{vtYF/}{vac-}, for instance, in (\ref{ex:vacmodfull}).

\begin{figure}[h]
\ex\label{ex:vacmodfull}\begingl
	\gla ... yam @ vacongyang ilisayam eda-koyās gan ... //
	\glb ... yam= vac-ong-yang ilisa-yam eda=koya-as gan-Ø ... //
	\glc {} \DatT{}= like-\Irr{}-\Fsg{}.\Aarg{} dedicate-\Ptcp{} 		
		this=book-\Parg{} child-\Top{} {} //
	\glft `... I would like to dedicate this book to the child ...' 
		\tc{\citep[1, 8]{benung:petitprince}} //
\endgl\xe
\end{figure}

Moreover, as illustrated previously in (\ref{ex:myashall}), \xayr{mY}{mya}{be
supposed to, shall} can be used to express indirect commands where English\index{English} may
use the subjunctive mood. Essentially, the function of this modal is that of
the jussive mood in that the speaker issues an order or request to arrange for
an action to happen instead of making a direct order to a second person. For
convenience, (\ref{ex:myashall}) is repeated here as (\ref{ex:myashall_2}).
While the version without \rayr{mY}{mya} is an indirect order, including the
modal adds a modicum of politeness by phrasing the indirect order as an
instruction. Essentially, thus, adding the modal has an effect comparable to
the use of the subjunctive `he call' instead of the indicative `he calls' in
the English\index{English} translation.

\begin{figure}[h]
\ex\label{ex:myashall_2}\begingl
	\gla Arapnang, sa @ \textup{(}mya\textup{)} @ garayāng hatay. //
	\glb arap=nang sa= (mya=) gara=yāng hatay-Ø //
	\glc require=\Fpl{}.\Aarg{} \PatT{}= (shall=) call=\TsgM{}.\Aarg{}
	police-\Top{} //
	\glft `We require that he call the police.' //
\endgl\xe
\end{figure}

In addition to this use, \rayr{mY}{mya} is also used in commands to third
persons\index{person}, whether direct or indirect. This use is displayed in
(\ref{ex:indirimp}). English\index{English} may use \fw{shall} here as an equivalent.

\begin{figure}[h]
\pex\label{ex:indirimp}
\a\begingl
	\gla Ningu cam, mya @ saratang. //
	\glb ning-u cam mya= sara=tang //
	\glc tell-\Imp{} \TplM{}.\Dat{} shall= leave=\TplM{}.\Aarg{} //
	\glft `Tell them to leave.' //
\endgl

% \a\begingl
% 	\gla Mya @ vehara nekanley. //
% 	\glb mya= veh-ara nekan-ley //
% 	\glc shall= build-\TsgI{} bridge-\PargI{} //
% 	\glft `A bridge shall be built.' //
% \endgl

\a\begingl
	\gla Mya @ yomāra makangreng. //
	\glb mya= yoma-ara makang-reng //
	\glc shall= exist-\TsgI{} light-\AargI{} //
	\glft `Let there be light.' //
\endgl
\xe
\end{figure}

\index{modals|)}

\subsection{Participle}
\label{subsec:participle}
\index{participle|(}
Besides the imperative---and, by extension, the hortative---Ayeri also
possesses another non-finite\index{verbs!non-finite} form called the participle.\footnote{It might as
well be referred to as an infinitive, but `participle' is now the established
term.} This form is marked by appending \rayr{/ymF}{-yam} to the verb root. The
participle is generally the form of verbal complements\index{grammatical function!open complement} of intransitive
subordinating verbs\index{verbs!intransitive}\index{verbs!control}\index{verbs!raising}. For instance, \xayr{kYunF/}{cun-}{begin} or
\xayr{mnNF/}{manang-}{avoid} both allow complementation with another verb, as
shown in (\ref{ex:vbcompl_1}).

\begin{figure}[h]
\pex\label{ex:vbcompl_1}
\a\label{ex:intrcompl_1}\begingl
	\gla Cunyo pero perinang makayam. // 
	\glb cun-yo pero perin-ang maka-yam // 
	\glc begin-\TsgN{} slowly sun-\Aarg{} shine-\Ptcp{} //
	\glft `The sun slowly began to shine.' //
\endgl

\a\label{ex:trcompl_1}\begingl
	\gla Manangye ang @ Nilan pengalyam badanas saha yena. //
	\glb manang-ye ang= Nilan pengal-yam badan-as saha yena //
	\glc avoid-\TsgF{} \Aarg{}= Nilan meet-\Ptcp{} father-\Parg{} in.law 
		\TsgF{}.\Gen{} //
	\glft `Nilan avoids to meet her father-in-law.' //
\endgl
\xe
\end{figure}

The subordinated verb may also be fronted into the position between the
subordinating verb and the subject, as in (\ref{ex:vbcompl_2}), especially when
the subordinate verb is intransitive\index{verbs!intransitive}, like \xayr{mkymF}{makayam}{shining} in
(\ref{ex:intrcompl_2}). As (\ref{ex:trcompl_2}) shows in comparison to
(\ref{ex:trcompl_2}), by fronting the subordinated verb, the arguments of the
subordinate verb become available for topicalization\index{grammatical function!topic}. Compare
\autoref{subsec:vps} for details on the syntactic operations possible with
subordinating verbs, that is, control verbs\index{verbs!control} (p.~\pageref{subsubsec:ctrlvb}) and
raising verbs\index{verbs!raising} (p.~\pageref{subsubsec:raisvb}).

\begin{figure}[h]
\pex\label{ex:vbcompl_2}
\a\label{ex:intrcompl_2}\begingl
	\gla Cunyo pero makayam perinang. // 
	\glb cun-yo pero maka-yam perin-ang // 
	\glc begin-\TsgN{} slowly shine-\Ptcp{} sun-\Aarg{} //
	\glft `The sun slowly began to shine.' //
\endgl

\a\label{ex:trcompl_2}\begingl
	\gla Sa @ manangye pengalyam ang @ Nilan badan saha yena. //
	\glb sa= manang-ye pengal-yam ang= Nilan badan-Ø saha yena //
	\glc \PatT{}= avoid-\TsgF{} meet-\Ptcp{} \Aarg{}= Nilan father-\Parg{}
		in.law \TsgF{}.\Gen{} //
	\glft `Her father-in-law, Nilan avoids to meet him.' //
\endgl
\xe
\end{figure}

% Since subordinated verbs may be transitive like in (\ref{ex:trcompl}), the
% problem of center-embedding arises when the agent NP of the subordinating
% verb is not simply a cliticized pronoun (see
% \autoref{clitics_postverb_person}, p.~\pageref{clitics_postverb_person};
% \ref{subsec:persnumagr}), since arguments of the subordinating verb follow
% the embedded clause as in (\ref{ex:intrcompl}):

% \pex[*=\ques\ques]
% \a\ljudge{\ques}\begingl
% 	\gla Ang pinyaya \textup{[}konjam inunas{\normalfont]} {} Yan sa 
% 		Pila. //
% 	\glb ang pinya-ya kond-yam inun-as Ø Yan sa Pila //
% 	\glc \AgtT{} ask-\TsgM{} eat-\Ptcp{} fish-\Parg{} \Top{} Yan \Parg{} 
% 		Pila //
% 	\glft `Yan asks Pila to eat the fish.' //
% \endgl

% \a\ljudge{\ques\ques}\begingl
% 	\gla Ang pinyaya \textup{[}ilyam koyaley ledanyam 
% 		yana{\normalfont]} {} Yan sa Pila. //
% 	\glb ang pinya-ya il-yam koya-ley ledan-yam yana Ø Yan sa Pila //
% 	\glc \AgtT{} ask-\TsgM{} give-\Ptcp{} book-\PargI{} friend-\Dat{} 
% 		\TsgM{}.\Gen{} \Top{} Yan \Parg{} Pila //
% 	\glft `Yan asks Pila to give the book to his friend.' //
% \endgl
% \xe

% In order to avoid too much complexity at the expense of ease of composition
% on the speaker's side, and intelligibility on the listener's, it is much
% preferable to express the embedded clause as a complement clause
% instead.\footnote{The German linguist Otto Behaghel (1854--1936) coined a
% number of laws---albeit with German in focus---three of which are relevant to
% information flow: \textcquote[4]{behaghel1932}{Das oberste Gesetz ist dieses,
% daß das geistig eng Zusammengehörige auch eng zusammengestellt wird.} [`The
% supreme law is such that the mentally closely related is also arranged in
% close proximity.']---\textcquote[4]{behaghel1932}{Ein zweites machtvolles
% Gesetz verlangt, daß das Wichtigere später steht als das Unwichtige,
% dasjenige, was zuletzt noch im Ohr klingen soll.} [`A second powerful law
% demands that more important information appear at a later point than what is
% less important: the which is supposed lastly to resonate in the listener's
% ear.']---\textcquote[6]{behaghel1932}{Gesetz der wachsenden Glieder […]; es
% besagt, daß von zwei Gliedern, soweit möglich, das kürzere vorausgeht, das
% längere nachsteht.} [`Law of the growing constituents […]; it signifies that
% of two constituents, if possible, the shorter one precedes, the longer one
% follows.'] Also compare \citet{wasow1997} on the cooperation between speaker
% and listener in the face of syntactically complex, `heavy' constituents.} The
% particle \rayr{d/}{da-} may be added to the formerly subordinating verb in
% order to signal that a complement clause is following.

% \pex
% \a\begingl
% 	\gla Ang da-pinyaya {} Yan sa Pila, \textup{[}le konjeng 
% 		inun\textup{]}. //
% 	\glb ang da=pinya-ya Ø Yan sa Pila le kond=yeng inun-Ø //
% 	\glc \AgtT{} such=ask-\TsgM{} \Top{} Yan \Parg{} Pila \PatTI{} 
% 		eat=\TsgF{}.\Aarg{} fish-\Top{} //
% 	\glft `Yan asks Pila to eat the fish.' //
% \endgl

% \a\begingl
% 	\gla Ang da-pinyaya {} Yan sa Pila, \textup{[}le ilyeng koya 
% 		ledanyam yana{\normalfont]}. //
% 	\glb ang da=pinya-ya Ø Yan sa Pila le il=yeng koya-Ø ledan-yam
% 		yana //
% 	\glc \AgtT{} such=ask-\TsgM{} \Top{} Yan \Parg{} Pila \PatTI{} 
% 		give-\TsgF{} book-\Top{} friend-\Dat{} \TsgM{}.\Gen{} //
% 	\glft `Yan asks Pila to give the book to his friend.' //
% \endgl
% \xe

\index{participle|)}

\subsection{Other affixes}
\index{clitics|(}

In the section on noun morphology we have already encountered a number of
proclitics\index{prefixes} that may attach to noun heads (see sections
\ref{subsec:clitics} and \ref{subsec:nounpref}). Some of these can also attach
to verbs. Furthermore, verbs may be modified by certain quantifier\index{quantifiers} clitics. The
latter are dealt with in more detail in \autoref{sec:quantifiers}; only a few
relevant examples will be given here.

\subsubsection{Prefixes}
\label{subsubsec:verbprefixes}
\index{prefixes|(}

We have already encountered the prefix \xayr{d/}{da-}{so, such} in the
previous section, as well as in the section on noun prefixes\index{prefixes} (see sections
\ref{clitics_prenoun_dem}, p.~\pageref{clitics_prenoun_dem}; 
\ref{subsec:nounpref}; and \ref{subsec:participle}). With nouns,
\xayr{d/}{da-}{such} patterns as a demonstrative with the deictic\index{deixis} prefixes\index{prefixes}
\xayr{Ed/}{eda-}{this} and \xayr{Ad/}{ada-}{that}. Distinguishing between near
and far is not possible with verbs,\footnote{Unless there were a distinction
between actions performed in the speaker's proximity and actions performed at a
distance. Ayeri, however, does not make such a distinction. A cursory web
search did not turn up evidence from natural languages either.\index{desiderata}} but pointing
out that something is happening `in this way, so' is still possible, hence
\rayr{d/}{da-} is also applicable to verbs. \rayr{d/}{da-} can thus act
essentially as a pro-verb. As a clitic, it leans on the verb, preceding all
other inflectional prefixes\index{prefixes}, that is, any tense\index{tense} prefixes that may possibly
precede\index{affix order} the verb root.

\begin{figure}[h]
\pex\label{ex:daproverb}
\a\begingl
	\gla Da-mingya ang @ Diyan. //
	\glb da=ming-ya ang= Diyan. //
	\glc so=can-\TsgM{} \Aarg{}= Diyan //
	\glft `Diyan can (do it).' //
\endgl

\a\begingl
	\gla Ang @ da-məpinyaya {} @ Yan sa @ Pila. //
	\glb ang= da=mə-pinya-ya Ø= Yan sa= Pila //
	\glc \AgtT{}= such=\Pst{}-ask-\TsgM{} \Top{}= Yan \Parg{}= Pila //
	\glft `Yan asked Pila to (do so).' //
\endgl
\xe
\end{figure}

Another possible use of the prefix \rayr{d/}{da-} with verbs is related to the
abbreviation of \xayr{dnY}{danya}{such one} as described in sections
\ref{clitics_preverb_da} (p.~\pageref{clitics_preverb_da}) and
\ref{subsec:dempro}, where the demonstrative part, \rayr{d/}{da-} may be split
off the pronoun\index{pronouns!demonstrative} and attached to the adjective directly to express `the
\textsc{adjective} one'. This practice has possibly been extended to verbs as
illustrated in (\ref{ex:daproverb}). Example (\ref{ex:redone}) from the
mentioned section is repeated here as (\ref{ex:redone_2} for the reader's
convenience. When \rayr{d/}{da-} is used as an abbreviation for
\rayr{dnYaasF}{danyās} (such.one-\Parg{}) or \rayr{dnYlej}{danyaley}
(such.one-\PargI{}), as in (\ref{ex:davb2}), it may also appear prefixed to the
verb.

\begin{figure}[h]
\ex\label{ex:redone_2}\begingl
	\gla Sa @ noyang da-tuvo. //
	\glb sa= no=yang da=tuvo-Ø //
	\glc \PatT{}= want=\Fsg{}.\Aarg{} such=red-\Top{} //
	\glft `I want the red one.' //
\endgl\xe
\end{figure}

\begin{figure}[h]
\ex\label{ex:davb2}\begingl
	\gla Mya @ da-vehoyyāng. //
	\glb mya= da=veh-oy=yāng //
	\glc shall= one=build-\Neg{}=\Tsg.\M{} //
	\glft `He is not supposed to build one.' //
\endgl\xe
\end{figure}

As mentioned above, \rayr{d/}{da-} can also be used in an expletive way, to
express `in this way' or `like that'. It does not encode an anaphoric relation
in this case, but merely serves as a discourse particle\index{discourse particles} to highlight the
action. \rayr{d/}{da-} in both examples in (\ref{ex:daexpl}) has a presentative
function rather than an anaphoric one.

\begin{figure}[h]
\pex\label{ex:daexpl}
\a\begingl
	\gla Da-sahāra seyaraneng. //
	\glb da=saha-ara seyaran-eng //
	\glc thus=come-\TsgI{} rain-\AargI{} //
	\glft `Here comes the rain.' //
\endgl

\a\begingl
	\gla Le @ no @ da-subroyya ang @ Hasanjan tiga kaytan yana. //
	\glb le= no= da=subr-oy-ya ang= Hasanjan tiga kaytan-Ø yana //
	\glc \PatT{}= want= there=give.up-\Neg{}-\TsgM{} \Aarg{}= Hasanjan 
		honorable right-\Top{} \TsgM{}.\Gen{} //
	\glft `Mr. Hasanjan did not want to cease his right just there.' //
\endgl
\xe
\end{figure}

Besides \rayr{d/}{da-}, verbs may also host the \xayr{ku/}{ku-}{like}
proclitic, which we have already seen with both nouns and adjectives (compare
sections \ref{subsec:clitics}, \ref{subsec:nounpref}, and
\ref{subsec:adjaffx}). The English\index{English} translation in context may rather be `as
though' than `like' here, as indicated in (\ref{ex:likecompl}), but the
function is the same: expressing alikeness and resemblance.

\begin{figure}[h]
\ex\label{ex:likecompl}\begingl
	\gla Misyeng, ang @ ku-tangoyye yās. //
	\glb mis=yeng ang= ku=tang-oy=ye.Ø yās //
	\glc act=\TsgF{}.\Aarg{} \AgtT{}= like=hear-\Neg{}=\TsgF{}.\Top{} 
		\TsgM{}.\Parg{} //
	\glft `She acts as though she does not hear him.' //
\endgl\xe
\end{figure}

As previously described (sections \ref{clitics_preverb_refl},
p.~\pageref{clitics_preverb_refl}, and \ref{subsec:reflrec}),
\xayr{sitNF/}{sitang}{self}, the reflexive clitic, can appear as a prefix on
verbs as well. This may be the case when the patient\index{semantic role!patient} of a transitive\index{verbs!transitive} sentence
signifies the same entity as the agent\index{semantic role!agent}. Example (\ref{ex:reflvb}) is repeated
here as (\ref{ex:reflvb_2}) for convenience.

\begin{figure}[h]
\ex\label{ex:reflvb_2}\begingl
	\gla Ang @ sitang-silvye puluyya. //
	\glb ang= sitang=silv=ye.Ø puluy-ya //
	\glc \AgtT{}= self=see=\TsgF{}.\Top{} mirror-\Loc{} //
	\glft `She sees herself in the mirror.' //
\endgl\xe
\end{figure}

The image of the agent\index{semantic role!agent} in the mirror is that of the agent\index{semantic role!agent} herself, so she is
seeing her own reflection. Both agent\index{semantic role!agent} and patient\index{semantic role!patient} thus refer to the same
person. This means that instead of using the reflexive object\index{grammatical function!primary object} pronoun\index{pronouns!reflexive}
\xayr{sitNF/yesF}{sitang-yes}{herself} (self-\TsgF{}.\Parg{}), it is possible
to drop the pronoun\index{pronouns!reflexive} and to place the reflexive prefix on the verb instead.

\index{prefixes|)}

\subsubsection{Suffixes}
\index{suffixes|(}

Besides hosting proclitics\index{prefixes}, verbs may also host enclitics, namely,
adverbial suffixes denoting degree, such as
\xayr{/Ani}{-ani}{not at all},
% \xayr{/ENF}{-eng}{rather},
\xayr{/IknF}{-ikan}{much},
% \xayr{/Ikoj}{-ikoy}{not much},
\xayr{/kj}{-kay}{a little}, or
% \xayr{/nm}{-nama}{just, only, merely},
\xayr{/NsF}{-ngas}{almost}
% \xayr{/nYm}{-nyama}{even}
(also see sections \ref{clitics_quant}, p.~\pageref{clitics_quant}, and
\ref{sec:quantifiers}). Some of these overlap with quantifiers\index{quantifiers}
applicable to nouns, and all of them are also applicable to adjectives. As
enclitics, these suffixes lean on the inflected verb, as shown in 
(\ref{ex:intsfvb}).

\begin{figure}[h]
\pex\label{ex:intsfvb}
\a\label{ex:verbquant}\begingl
	\gla Ang @ rua @ apaya-kay {} @ Latun adanyaya. //
	\glb ang= rua= apa-ya=kay Ø= Latun adanya-ya //
	\glc \AgtT{}= must= laugh-\TsgM{}=a.little \Top{}= Latun that.one-\Loc{} //
	\glft `Latun had to laugh a little at that.' //
\endgl

\a\begingl
	\gla Ya @ no @ narayang-nama va. //
	\glb ya= no= nara=yang=nama va.Ø //
	\glc \LocT{}= want= speak=\Fsg{}.\Aarg{}=just \Second{}.\Top{} //
	\glft `It is you I just want to talk to.' //
\endgl
\xe
\end{figure}

\index{suffixes|)}
\index{clitics|)}
\index{verbs|)}

\section{Adverbs}
\label{sec:adverbs}
\index{adverbs|(}

Adverbs in Ayeri are the counterparts of adjectives with regards to the
modification of verbs\index{verbs} and phrases. Like adjectives, they do not display
agreement, though attributive adverbs may as well take suffixes\index{suffixes} for comparison\index{comparison}
(`run \fw{faster}', `climb \fw{better}'). Adverbs may likewise be modified by
the usual quantifying\index{quantifiers} and grading suffixes\index{suffixes}, which have been analyzed here as
being adverbial in nature themselves here. Generally, there is no rigid
distinction between adverbs and adjectives, so the latter may easily be used as
the former. The following subsections will discuss the different kinds of
adverbs and their possible uses as modifiers.

\subsection{Attributive adverbs}

Attributive adverbs are words expressing the manner in which an action is
carried out, or the circumstances of an event. Like adjectives, adverbs follow\index{word order}
their heads, that is, verbs\index{verbs}. If near-grammaticalized adverbs are involved,
namely, adverbs whose function predominates over their semantic content,
attributive adverbs follow\index{word order} these. This case is illustrated in
(\ref{ex:funcadv}), where the attributive adjective \xayr{bnF}{ban}{good}
follows the more functional adverb \xayr{Iri}{iri}{already}. In
(\ref{ex:attradv}), on the other hand, the descriptive adjective
\xayr{tYbo}{cabo}{late} can directly follow the verb\index{verbs}. Further adverbs may
follow in decreasing order\index{word order} of semantic relation to their head. With regards to
grammaticalization\index{grammaticalization}, \citet[157\psqq]{lehmann2015} speaks of \emph{bondedness}
or \emph{fügungsenge} (`tightness of construction'): the closer the bond
between two juxtaposed terms is, the higher is its degree of
grammaticalization\index{grammaticalization}. This explains why \rayr{Iri}{iri} must follow the verb in
(\ref{ex:funcadv}) while descriptive adverbs less central to the verb's meaning
typically follow\index{word order} with increasing optionality.

\begin{figure}[h]
\pex
\a\label{ex:funcadv}\begingl
	\gla Ri @ rija iri \textbf{ban} ang @ Tapan palān yena. //
	\glb ri= rig-ya iri ban ang= Tapan palān-Ø yena //
	\glc \InsT{}= draw-\TsgM{} already well \Aarg{}= Tapan age-\Top{} 
		\TsgF{}.\Gen //
	\glft `For her age, Tapan already draws well.' //
\endgl

\a\label{ex:attradv}\begingl
	\gla Sahasaya \textbf{cabo} ang @ Niyas. //
	\glb saha-asa-ya cabo ang= Niyas //
	\glc come-\Hab{}-\TsgM{} late \Aarg{}= Niyas //
	\glft `Niyas is usually late.' //
\endgl
\xe
\end{figure}

Adverbs do not show agreement\index{agreement}, however, attributive adverbs can be negated\index{negation}.
This makes them very similar to adjectives, except that they do not modify
nouns. The negative\index{negation} suffix\index{suffixes} for attributive adverbs is \rayr{/Oj}{-oy}, which is
demonstrated in (\ref{ex:advneg}).

\begin{figure}[h]
\ex\label{ex:advneg}\begingl
	\gla Ersasayan napayoy ang @ Temisi. //
	\glb ers-asa-yan napay-oy ang= Temisi //
	\glc cook-\Hab{}-\TplM{} spicy-\Neg{} \Aarg{}= Northerner //
	\glft `The Northerners cook in an unspicy way.' //
\endgl\xe
\end{figure}

The adjective \xayr{npj}{napay}{spicy} has been seamlessly converted\index{derivation} into an
adjective here and negated to \xayr{npyoj}{napayoy}{unspicy(ly)}. The semantic
difference from the same sentence with the verb\index{verbs} negated\index{negation} instead of the adverb
in (\ref{ex:advneg_2}) is up to the choice of the speaker.

\begin{figure}[h]
\ex\label{ex:advneg_2}\begingl
	\gla Ersasoyyan napay ang @ Temisi. //
	\glb ers-asa-oy-yan napay ang= Temisi //
	\glc cook-\Hab{}-\Neg{}-\TplM{} spicy \Aarg{}= Northerner //
	\glft `The Northerners don't cook in a spicy way.' //
\endgl\xe
\end{figure}

\subsubsection{Comparison of adverbs}
\index{comparison|(}

Since actions may be gradable in the way they are carried out, it is 
possible to compare adverbs in the same way as adjectives. Here, however, only 
the particle-based strategy described in \autoref{subsec:adjcomp} 
can be used. In order to form the comparative, the enclitic\index{clitics} 
\rayr{/ENF}{-eng} is appended to the adverb as shown in (\ref{ex:advcomp}). The
superlative carries the enclitic\index{clitics} \rayr{/vaa}{-vā} as a marker, as in
(\ref{ex:advsupl}).

\begin{figure}[h]
\pex
\a\label{ex:advcomp}\begingl
	\gla Ang @ rije ban-eng {} @ Sipra na @ Tapan. //
	\glb ang= rig-ye ban=eng Ø= Sipra na= Tapan //
	\glc \AgtT{}= draw-\TsgF{} good=\Comp{} \Top{}= Sipra \Gen{}= Tapan //
	\glft `Sipra draws better than Tapan.' //
\endgl

\a\label{ex:advsupl}\begingl
	\gla Rije ban-vā ang @ Nava. //
	\glb rig-ye ban=vā ang= Nava //
	\glc draw-\TsgF{} good=\Supl{} \Aarg{}= Nava //
	\glft `Nava draws best.' //
\endgl
\xe
\end{figure}

\index{comparison|)}

\subsubsection{\fw{Māy} and \fw{voy}}
\label{subsubsec:maayvoy}
\index{questions|(}
% Maybe move this section to another, better fitting place later. For instance,
% question tags may better fit in with a discussion of question types within a
% chapter on sentence types.

The discourse particles\index{discourse particles} \xayr{maaj}{māy}{yes} and \xayr{voj}{voy}{no} can also
appear in the fashion of adverbs, though since they act mainly as functional
morphemes here, it is not possible for them to undergo comparison\index{comparison} in spite of
their attributive use. While \xayr{maaj}{māy}{yes} and \xayr{voj}{voy}{no}
normally express affirmative and negative\index{negation} responses as answers to closed
questions, \rayr{maaj}{māy}, for one, can be used adverbially as an
intensifier\index{intensifiers}, as in (\ref{ex:maayintens}). In a similar way, \rayr{voj}{voy} can
be used for negative\index{negation} intensification, which is demonstrated in 
(\ref{ex:voyintens}). The negative\index{negation} intensifier\index{intensifiers} replaces negation\index{mood!negative} on the verb\index{verbs} in
this case, though the verb\index{verbs} may still be negated\index{mood!negative} as well for very forceful
negation\index{negation}.

\begin{figure}[h]
\pex\label{ex:maayintens}
\a\begingl
	\gla Nay le @ konja māy epang ang @ Kaji nernan barina sebu! //
	\glb nay le= kond-ya māy epang ang= Kaji nernan-Ø bari-na sebu //
	\glc and \PatTI{}= eat-\TsgM{} \Int{} then \Aarg{}= Kaji piece-\Top{} 
		meat-\Gen{} rotten //
	\glft `And then Kaji totally ate the piece of rotten meat!' //
\endgl

\a\begingl
	\gla Yāng māy karomayās nārya. //
	\glb yāng māy karomaya-as nārya //
	\glc \TsgM{}.\Aarg{} \Int{} doctor-\Parg{} though // 
	\glft `He \emph{is} a doctor, though.' //
\endgl
\xe
\end{figure}

\begin{figure}[h]
\pex\label{ex:voyintens}
\a\begingl
	\gla Le @ vacyo voy veneyang kondan. //
	\glb le= vac-yo voy veney-ang kondan-Ø //
	\glc \PatTI{}= like-\TsgN{} \Int{}.\Neg{} dog-\Aarg{} food-\Top{} //
	\glft `The food, the dog did not like it at all.' //
\endgl

\a\begingl
	\gla Adareng voy bahisley niru. //
	\glb ada-reng voy bahis-ley niru //
	\glc that-\AargI{} \Int{}.\Neg{} day-\PargI{} bad // 
	\glft `That is not a bad day at all.' //
\endgl
\xe
\end{figure}

Besides this use, both \rayr{maaj}{māy} and \rayr{voj}{voy} can also be used in
tag questions, to reflect the expectation of the person asking with regards to
the answer. Example (\ref{ex:posexpect}) poses the question with the
expectation of an affirmative answer. This is indicated by using the
affirmative particle \rayr{maaj}{māy} after\index{word order} the verb\index{verbs}. Example 
(\ref{ex:posexpect}), on the other hand, indicates that the asker has doubts
about the issue in question and expects their opposite to decline. The negative\index{negation}
particle \rayr{voj}{voy} is placed in adverb position after\index{word order} the verb\index{verbs}
accordingly.

\begin{figure}[h]
\pex
\a\label{ex:posexpect}\begingl
	\gla Sa @ konjon māy patasjang keynam? //
	\glb sa= kond-yon māy patas-ye-ang keynam-Ø //
	\glc \PatT{}= eat-\TplN{} \Aff{} bear-\Pl{}-\Aarg{} people-\Top{} //
	\glft `People, bears eat them, don't they?' //
\endgl

\a\label{ex:negexpect}\begingl
	\gla Sa @ ginyon voy patasjang nimpur? //
	\glb sa= gin-yon voy patas-ye-ang nimpur-Ø //
	\glc \PatT{}= drink-\TplN{} \Neg{} bear-\Pl{}-\Aarg{} wine-\Top{} //
	\glft `Wine, bears don't drink it, do they?' //
\endgl
\xe
\end{figure}

\index{questions|)}

\subsection{Sentence adverbs}

Ayeri allows adverbs to modify sentences, for instance, to express the stance
of the speaker, to concede an argument, or simply to structure an argumentative
chain.

\subsubsection{Stance adverbs}

Adverbs indicating the stance of the speaker towards an assertion or a 
statement are, for instance:
\xayr{AMkYu}{ankyu}{really}, 
\xayr{kYuymF}{cuyam}{actually, indeed, in fact},
\xayr{klmF}{kalam}{honestly},
\xayr{kubnF}{kuban}{fortunately},
\xayr{kuniru}{kuniru}{unfortunately},
\xayr{nilj}{nilay}{probably},
\xayr{yomiNF}{yoming}{maybe, perhaps}.
These adverbs are usually placed after\index{word order} the verb\index{verbs} like any other attributive 
adverb, even though their scope\index{scope} is over the whole clause. It is also possible 
to place them towards the end of the clause they are used in, however. Example 
(\ref{ex:stanceadv}) gives an instance of either position.

\begin{figure}[h]
\pex\label{ex:stanceadv}
\a\begingl
	\gla Ang @ ming @ bengya kuban {} @ Tipal vahamya bavesangena nā. //
	\glb ang= ming= beng-ya kuban Ø= Tipal vaham-ya bavesang-ena nā //
	\glc \AgtT{}= can= attend-\TsgM{} fortunately \Top{}= Tipal party-\Loc{} 
		birthday-\Gen{} \Fsg{}.\Gen{} //
	\glft `Fortunately, Tipal can attend my birthday party.' //
\endgl

\a\label{ex:naaryaadv}\begingl
	\gla Sahayāng cabo-kay nilay nārya. //
	\glb saha=yāng cabo=kay nilay nārya //
	\glc come=\TsgM{} late=a.little probably though //
	\glft `He will probably come a little late, though. //
\endgl
\xe
\end{figure}

\subsubsection{Discourse-structuring adverbs}
\label{subsubsec:discourseadv}

Ayeri does not have a great number of concessive adverbs, that is,
\xayr{AreenF}{arēn}{however, anyway} and \xayr{naarY}{nārya}{although, though;
nevertheless} do most, if not all the work. Like adverbs expressing stance,
they may follow\index{word order} the verb\index{verbs} or be placed at the end of the clause. Example
(\ref{ex:naaryaadv}) above already showed an example of \rayr{naarY}{nārya}
being used as a sentence adverb. With regards to this word, it is important to
note that \rayr{naarY}{nārya} may also be used as a general contrastive
conjunction\index{conjunctions} which can mostly be translated as `but'. In this sense, its
placement in a clause creates a slight difference in meaning, as illustrated
by example (\ref{ex:naaryaplcmt}) below.

\begin{figure}[h]
\pex\label{ex:naaryaplcmt}
\a\label{ex:naaryaconj}\begingl
	\gla Garayang, nārya guraca ranyāng. //
	\glb gara=yang nārya gurat-ya ranya-ang //
	\glc call=\Fsg{}.\Aarg{} but answer-\TsgM{} nobody-\Aarg{} //
	\glft `I called, but nobody answered.' //
\endgl

\a\label{ex:naaryaadv2}\begingl
	\gla Garayang, guraca nārya ranyāng. //
	\glb gara=yang gurat-ya nārya ranya-ang //
	\glc call=\Fsg{}.\Aarg{} answer-\TsgM{} although nobody-\Aarg{} //
	\glft `I called, although nobody answered.' //
\endgl
\xe
\end{figure}

Besides the two adverbs mentioned above, there is also
\xayr{d/naarY}{da-nārya}{even though, in spite of, despite} as a postposition\index{adpositions!postpositions}
with a contrastive meaning (see \autoref{subsec:postpos}). As an adposition it
accepts either a noun phrase\index{phrase types!noun phrase} or a complementizer phrase\index{phrase types!complementizer phrase} (CP) as a complement\index{grammatical function!closed complement}.
In the latter case, which is shown in (\ref{ex:danaarya2}), there is no
locative case\index{case!locative} agreement\index{agreement} of the whole CP\index{phrase types!complementizer phrase} with the postposition\index{adpositions!postpositions}, since there is
no fitting agreement\index{agreement} target to attach it to.

\begin{figure}
\pex\label{ex:danaarya}
\a\label{ex:danaarya1}\begingl[glspace=.25em]
	\gla Ya @ precang nanga yena \textup{[\tsub{PP} [\tsub{NP}~} @ sarānya 
		yena~\textup{]} da-nārya~\textup{]}. //
	\glb ya= pret=yang nanga-Ø yena {} sarān-ya yena da-nārya //
	\glc \LocT{}= knock=\Fsg{}.\Aarg{} house-\Top{} \TsgF{}.\Gen{} {}
		absence-\Loc{} \TsgF{}.\Gen{} in.spite //
	\glft `I knocked at her house in spite of her absence.' //
\endgl

\a\label{ex:danaarya2}\begingl
	\gla Precang \textup{[\tsub{PP} [\tsub{CP}~} @ ang @ yomoyye rangya 
		yena~\textup{]} da-nārya~\textup{]}. //
	\glb pret=yang {} ang= yoma-oy=ye.Ø rang-ya yena da-nārya //
	\glc knock=\Fsg{}.\Aarg{} {} \AgtT{}= exist-\Neg{}=\TsgF{}.\Top{} 
		home-\Loc{} \TsgF{}.\Gen{} even.though //
	\glft `I knocked, even though she wasn't at home.' //
\endgl
\xe
\end{figure}

Further adverbs which are commonly used as adverbial expressions and which may 
appear in the presentation of arguments include:
\xayr{dermYmF}{deramyam}{after all},
\xayr{kjbunj}{kaybunay}{by the way},
\xayr{ku/nsY}{ku-nasya}{as follows},
\xayr{mennFy}{menanya}{on the one hand},
\xayr{mirMpluj}{mirampaluy}{otherwise},
\xayr{naareNF}{nāreng}{rather},
\xayr{njnj}{naynay}{(and) also, moreover, furthermore},
\xayr{pluNnY}{palunganya}{on the other hand},
\xayr{pMtY}{panca}{finally, eventually, in the end},
\xayr{pinYnF}{pinyan}{please},
\xayr{suhiNF}{suhing}{naturally, of course}.
It should be apparent by the complexity and relative length of some of these 
words that they are fossilized expressions, for instance, 
\xayr{dermYmF}{deramyam}{after all} transparently derives from 
\xayr{dermF}{deram}{matter of fact} declined for dative case
(\rayr{/ymF}{yam}, see \autoref{subsubsec:dative}); 
\rayr{ku/nsY}{ku-nasya} is derived from a phrase literally meaning `as 
(it) follows'; and \xayr{pluNnY}{palunganya}{on the other hand} literally 
means `in difference', from \xayr{pluNnF}{palungan}{difference, 
distinction}. Of the list given above, it may be noted that 
\xayr{pinYnF}{pinyan}{please} (from \xayr{pinY/}{pinya-}{ask}) is often found 
at the beginning of polite requests, as illustrated by (\ref{ex:request}).

\begin{figure}[h]
\ex\label{ex:request}
\begingl
	\gla Pinyan, sahu kongya! //
	\glb pinyan saha-u kong-ya //
	\glc please come-\Imp{} inside-\Loc{} //
	\glft `Please come inside!' //
\endgl
\xe
\end{figure}

\subsubsection{Conjunctive adverbs}
\label{subsubsec:conjadv}
\index{conjunctions|(}

The term `conjunctive adverb' here refers to sentence adverbs which have the
distribution of a conjunction. Whereas sentence adverbs are normally placed
either after\index{word order} the verb\index{verbs} or at the end of a clause, these words are usually found
as introducing clauses since they connect two otherwise independent statements
to show their relation to each other. Their meaning extends that of the
`pure', logical conjunctions \xayr{nj}{nay}{and} and \xayr{soyNF}{soyang}{or},
however.\footnote{Logical `not' is usually expressed by a negative\index{negation} suffix\index{suffixes} on
the adjective\index{adjectives} or the verb\index{verbs}, compare sections \ref{subsec:adjneg} and
\ref{subsubsec:verbneg}, respectively. For conjunctions proper, see
\autoref{sec:conjunctions}.} Part of this small class of words are the
expressions \xayr{bt}{bata}{if, whether},\footnote{Conditional protasis and
apodosis are often unmarked in Ayeri, however, it may still be desirable
occasionally to use a particle to indicate them explicitly.}
\xayr{kd}{kada}{then, thus},
\xayr{kd/kd}{kada-kada}{so that ... again},
\xayr{kdaare}{kadāre}{so that},
\xayr{njnj}{naynay}{moreover, furthermore, and also},
\xayr{naareNF}{nāreng}{(but) rather},
\xayr{naaroj}{nāroy}{but not},
\xayr{naarY}{nārya}{but, except that, though, yet},
\xayr{siniNF}{sining}{that is}, and 
\xayr{ynoymF}{yanoyam}{because, for, since}. Examples are provided by
(\ref{ex:conjadvs}). Regarding (\ref{ex:but}), it needs to be pointed out that
\rayr{naarY}{nārya} can also be used as a regular adverb. In those cases it is
considered to have less contrastive force, however: postposed
\rayr{naarY}{nārya} is best translated as `though, although' (compare
\autoref{subsubsec:discourseadv}).

\begin{figure}[h]
\pex\label{ex:conjadvs}
\a\begingl[glspace=.25em]
	\gla Le @ rimasayang kunang sirutayya, kadāre ming @ toryang ban-eng. //
	\glb le= rima-asa=yang kunang-Ø sirutay-ya kadāre ming= tor=yang 
		ban=eng //
	\glc \PatTI{}= shut-\Hab{}=\Fsg{}.\Aarg{} door-\Top{} night-\Loc{} 
		so.that can= sleep=\Fsg{}.\Aarg{} good=\Comp{} //
	\glft `I usually close the door at night so that I can sleep better.' //
\endgl

\a\label{ex:but}\begingl
	\gla Ilta @ toryeng, nārya da-kilisoyyon nilanjang yena. //
	\glb ilta= tor=yeng nārya da=kilis-oy-yon nilan-ye-ang yena //
	\glc need= sleep=\TsgF{}.\Aarg{} but so=allow-\Neg{}-\TplN{} 
		thought-\Pl{}-\Aarg{} \TsgF{}.\Gen{} //
	\glft `She needed to sleep, but her thoughts did not allow her to.' //
\endgl

\a\begingl
	\gla Ang @ ming @ hangoyya {} @ Yan padangas, yanoyam yāng pisu. //
	\glb ang= ming= hang-oy-ya Ø= Yan padang-as yanoyam yāng pisu //
	\glc \AgtT{}= can= keep-\Neg{}-\TsgM{} \Top{}= Yan mind-\Parg{} because 
		\TsgM{}.\Aarg{} tired //
	\glft `Yan cannot concentrate because he is tired.' //
\endgl
\xe
\end{figure}

Since verbs can be negated\index{negation} and reduplicated\index{reduplication} for grammatical purposes, the
adverbs \xayr{kd/kd}{kada-kada}{so that ... again} and \xayr{naaroj}{nāroy}{but
not} are mostly used with predicative adjectives\index{adjectives!predicative}, since equative statements
lack a verb\index{verbs} to apply verb morphology to. These two conjunctive adverbs thus can
convey the most important distinctions otherwise expressed by the verb\index{verbs} as a
substitute. This ability, however, is not a productive grammatical process, but
specific to \rayr{kd/kd}{kada-kada} and \rayr{naaroj}{nāroy}, respectively. An
example of each is given in (\ref{ex:drvconj}).

\begin{figure}[h]
\pex\label{ex:drvconj}
\a\label{ex:sothatagain}\begingl
	\gla Rua @ nibaya ang @ Pulan, kada-kada yāng sapin tadayya kivo. //
	\glb rua= niba-ya ang= Pulan kada\til{}kada yāng sapin taday-ya kivo //
	\glc must= rest-\TsgM{} \Aarg{}= Pulan \Iter{}\til{}so.that 
		\TsgM{}.\Aarg{} healthy time-\Loc{} little //
	\glft `Pulan must rest so that he will be healthy again very soon.' //
\endgl

\a\label{ex:butnot}\begingl
	\gla Yang temisena cuyam, nāroy yang petau. //
	\glb yang temis-ena cuyam nāroy yang petau //
	\glc \Fsg{}.\Aarg{} north-\Gen{} indeed but.not \Fsg{}.\Aarg{} stupid //
	\glft `I may be from the north, but I am not stupid.' //
\endgl
\xe
\end{figure}

As described above (compare \autoref{subsubsec:iterative}), partial
reduplication\index{reduplication} of the verb\index{verbs} expresses iterative aspect\index{aspect!iterative}, which in Ayeri is used
to mean `\textsc{verb} again' and `\textsc{verb} back', depending on context.
The reduplicated form \rayr{kd/kd}{kada-kada} as displayed in
(\ref{ex:sothatagain}) is irregular in its formation if we assume that it is
formed from \xayr{kdaare}{kadāre}{so that}; the regular outcome with
iterative reduplication\index{reduplication} applied would be *\rayr{k/kdaare}{*ka-kadāre}. As a
conjunction, however, it is relatively frequent, so it does not seem odd that
it has assumed a phonologically more simple, yet distinct form (compare, for
instance, \cite[11--12]{bybeehopper2001b}). The conjunctive adverb in
(\ref{ex:butnot}) exhibits likewise a slightly irregular formation if we
consider that it is essentially the negated form of \xayr{naarY}{nārya}{but};
the regular outcome would have been *\rayr{naarYoj}{*nāryoy}, which underwent
simplification to \rayr{naaroj}{nāroy}, presumably as well due to its
relatively high token frequency.

\subsection{Demonstrative adverbs}

\begin{table}[tp]\centering
\caption{Demonstratives relating to adverbial categories}
\begin{tabu} to .75\textwidth {C I X I X}
\tableheaderfont\toprule
Category
	& \multicolumn{2}{c}{Proximal}
	& \multicolumn{2}{c}{Distal}
	\\
\toprule

place
	& edaya
	& `here'
	& adaya
	& `there'
	\\
	
\midrule

time
	& edauyi
	& `now'
	& adauyi
	& `then'
	\\
	
\midrule

manner
	& edāre
	& `hereby'
	& adāre
	& `thereby'
	\\
	
\midrule

reason
	& edayam
	& `herefore'
	& adayam
	& `therefore'
	\\
	
\bottomrule

\end{tabu}
\label{tab:demadv}
\end{table}

Besides demonstrative pronouns\index{pronouns!demonstrative} like \xayr{AdnY}{adanya}{that (one)} (see
\autoref{subsec:dempro}), and indefinite pronouns\index{pronouns!indefinite} like
\xayr{yaarilF}{yāril}{for some reason; somewhere} (see
\autoref{subsec:indefpro}), Ayeri also possesses demonstrative pronouns\index{pronouns!demonstrative} for the
adverbial categories place, time, manner, and reason. The full paradigm is
given in \autoref{tab:demadv}. Compared to the paradigm for demonstrative
pronouns\index{pronouns!demonstrative} relating to persons or things, the paradigm of adverbial
demonstratives is incomplete in that forms with
\xayr{d/}{da-}{such} are unattested. Thus, instead of the hypothetical form
with \rayr{d/}{da-}, a full-NP\index{phrase types!noun phrase} adverbial with a generic noun\index{nouns!generic} has to be used:
*\rayr{dy}{*daya} → \xayr{d/ynoy}{da-yanoya}{in such a place}
(such-place-\Loc{}). Adverbial demonstratives are, like pronouns\index{pronouns}, in
complementary distribution\index{complementary distribution} with full NPs\index{phrase types!noun phrase}, since they are pro-forms. Thus, using
them as modifiers to NPs\index{phrase types!noun phrase} as in (\ref{ex:edayamod}) is not possible, while using
simple demonstrative \xayr{Ed/}{eda-}{this} together with a noun as in
(\ref{ex:edanp}) or using \xayr{Edy}{edaya}{here} as a pro-form fully replacing
the NP\index{phrase types!noun phrase} \xayr{Ed/nNy}{eda-nangaya}{in this house} as in (\ref{ex:edanyapro}) is
generally unproblematic.

\begin{figure}[h]
\pex
\a\ljudge*\label{ex:edayamod}\begingl
	\gla Ang @ mice {} @ Pada nangaya edaya. //
	\glb ang= mit-ye Ø= Pada nanga-ya edaya //
	\glc \AgtT{}= live-\TsgF{} \Top{}= Pada house-\Loc{} here //
\endgl

\a\label{ex:edanp}\begingl
	\gla Ang @ mice {} @ Pada eda-nangaya. //
	\glb ang= mit-ye Ø= Pada eda=nanga-ya //
	\glc \AgtT{}= live-\TsgF{} \Top{}= Pada this=house-\Loc{} //
	\glft `Pada lives in this house.' //
\endgl

\a\label{ex:edanyapro}\begingl
	\gla Mice ang @ Pada edaya. //
	\glb mit-ye ang= Pada edaya //
	\glc live-\TsgF{} \Aarg{}= Pada here //
	\glft `Pada lives here.' //
\endgl
\xe
\end{figure}

\index{conjunctions|)}
\index{adverbs|)}

\section{Numerals}
\label{sec:numerals}
\index{numerals|(}

The vast majority of the 196 sampled languages in \citet{wals131} either counts
in tens or employs a mixed vigesimal-decimal system, while only five languages
in the sample use a different base than 10. Ayeri uses a duodecimal system and
is thus very untypical compared to real-world languages in using a number\index{numerals} base
other than 10---none of the languages in \citet{wals131}'s sample are listed as
duodecimal.
% \footnote{I chose to use 12 as a numerical base because I simply wanted to
% toy with it. Also, I originally conceived of Ayeri speakers as humanoid but
% not necessarily human, which meant that they would not necessarily have
% evolved to have five fingers on each hand---for an earlier conlang of mine,
% Daléian, I used an octal system reasoning that speakers would only have four
% digits In any case, a duodecimal system could work reasonably well with human
% hands if you counted not only the fingers, but also the hands themselves.
% Finger-counting in Ayeri's duodecimal system would probably be similar to
% counting in the senary system of Nen described in \citet{evans2009} (as
% quoted in \cite{dixon2012}): \textcquote[73--74]{dixon2012}{In counting, Nen
% speakers `first count off the five fingers with a finger of their other hand,
% and then on the sixth they place their counting finger on the inside of the
% wrist'}.
Even though duodecimal numeral systems only occur rarely in natural languages,
they are not entirely unheard of. Thus, for instance, \citet{caingair2000}
report that in Maldivian, the \textcquote[21]{caingair2000}{numeral \fw{fas
doḷas} `60' (lit., `five twelves') comes from a duodecimal system that has all
but disappeared in the Maldives. This number system was used for special
purposes such as counting coconuts}.
% }

Ayeri's number words\index{numerals} are mostly semantic primes, that is, their meanings cannot
be readily recognized as derived from body parts \citep[74]{dixon2012} or from
internal arithmetic like 9 as `ten lacking one', for instance. The numerals
\xayr{kj}{kay}{three}, \xayr{Iri}{iri}{five}, and \xayr{henF}{hen}{eight} may
be an exception: as a quantifier\index{quantifiers}, \rayr{kj} {kay} means `a little,
few'; \rayr{Iri}{iri} means `already', which might refer to the fact that a
full hand has been counted off; and \rayr{henF}{hen} also means `all'. Ayeri
moreover appears extremely sophisticated in possessing a way of forming large
numerals by a theoretically open-ended, recursive process.

\subsection{Cardinal numerals}

Since people and concrete things are usually present in a countable manner, I
want to comment first on how countable entities are handled with regards to
numerals. After this, a discussion of how to express fractional amounts will
follow.

\subsubsection{Integers}

Cardinal numerals work much like adjectives in that they modify nouns. As
modifiers, they are placed after\index{word order} nouns. The full table of cardinal numerals
from $0 \times 12^0$ (0) to $11 \times 12^0$ (\elv) is given in
\autoref{tab:cardinals}.\footnote{For the sake of typographic simplicity, \ten\
and \elv\ will be used to mean $10 \times 12^0$ and $11 \times 12^0$,
respectively. An index `10' indicates base 10 explicitly, while an index `12'
indicates base 12.} An example of simple modification by a numeral is given in
(\ref{ex:nummod}).

\begin{figure}[h]
\ex\label{ex:nummod}
\begingl
	\gla Ang @ tenyaya pang bihanya yo soyang miye. //
	\glb ang= tenya=ya pang bihan-ya yo soyang miye //
	\glc \AgtT{}= die=\TsgM{}.\Top{} ago week-\Loc{} four or six //
	\glft `He died four or six weeks ago.' //
\endgl
\xe
\end{figure}

\begin{table}[p]\centering
\caption{Basic cardinal numerals}
\begin{tabu} to .75\linewidth {X[c] I X[c] I}
\toprule\tableheaderfont
Numeral
	& Word
	& Numeral
	& Word
	\\
\toprule

0
	& ja % \rayr{dY}{ja}
	& 6
	& miye % \rayr{miye}{miye}
	\\

1
	& men % \rayr{menF}{men}
	& 7
	& ito % \rayr{Ito}{ito}
	\\
	
2
	& sam % \rayr{smF}{sam}
	& 8
	& hen % \rayr{henF}{hen}
	\\
	
3
	& kay % \rayr{kj}{kay}
	& 9
	& veya % \rayr{vey}{veya}
	\\

4
	& yo % \rayr{yo}{yo}
	& \ten
	& mal % \rayr{mlF}{mal}
	\\

5
	& iri % \rayr{Iri}{iri}
	& \elv
	& tam % \rayr{tmF}{tam}
	\\

\bottomrule
\end{tabu}
\label{tab:cardinals}
\end{table}

In this example, the numeral \xayr{yo}{yo}{four} modifies the noun 
\xayr{bihnF}{bihan}{week}. Notably, however, plural\index{number!plural} marking is missing on the 
noun, since the notion of plurality\index{number!plural} is provided by the numeral itself; the 
numeral is thus normally sufficient to mark the whole NP\index{phrase types!noun phrase} as plural\index{number!plural}.

\begin{table}[p]\centering
\caption{Numerals for factors of 12}
\begin{tabu} to .75\linewidth {X[c] I X[c] I}
\toprule\tableheaderfont
Numeral
	& Word
	& Numeral
	& Word
	\\
\toprule

%
	& %
	& 60
	& miyelan % \rayr{miyelnF}{miyelan}
	\\

10 
	& menlan % \rayr{menFlnF}{menlan}
	& 70
	& itolan % \rayr{ItolnF}{itolan}
	\\

20 
	& samlan % \rayr{smFlnF}{samlan} 
	& 80
	& henlan % \rayr{henFlnF}{henlan}
	\\

30 
	& kaylan % \rayr{kjlnF}{kaylan}
	& 90
	& veyalan % \rayr{veylnF}{veyalan}
	\\

40 
	& yolan % \rayr{yolnF}{yolan}
	& \ten0
	& mallan % \rayr{mlFlnF}{mallan}
	\\

50 
	& irilan % \rayr{IrilnF}{irilan}
	& \elv0
	& tamlan % \rayr{tmFlnF}{tamlan}
	\\

\bottomrule
\end{tabu}
\label{tab:cardinalsten}
\end{table}

Multiples of $12^1$ between 10 and \elv0 are formed by appending the suffix\index{suffixes}
\rayr{/lnF}{\mbox{-lan}} to the numbers from 0 to \elv, which are given in
\autoref{tab:cardinalsten}. These numerals themselves act as heads for forming
compounds\index{compounds} with lower numerals to fill in the $12^0$ numerals 11, 12, 13, ...,
21, 22, 23, etc.\ Thus, one counts on from \xayr{menFlnF}{menlan}{dozen} in
the way illustrated by (\ref{ex:counting}).

\begin{figure}[h]
\ex\labels\label{ex:counting}
\begin{minipage}[t]{.5\remaining}
\tl\quad\begin{minipage}[t]{\linewidth}
	\rayr{\larger menFlnF/menF}{menlan-men} (11), \\
	\rayr{\larger menFlnF/smF}{menlan-sam} (12), \\
	\rayr{\larger menFlnF/kj}{menlan-kay} (13), \medskip
	
	etc.
	\end{minipage}
\end{minipage}
~
\begin{minipage}[t]{.5\remaining}
\tl\quad\begin{minipage}[t]{\linewidth}
	\rayr{\larger smFlnF/menF}{samlan-men} (21), \\
	\rayr{\larger smFlnF/smF}{samlan-sam} (22), \\
	\rayr{\larger smFlnF/kj}{samlan-kay} (23), \medskip
	
	etc.
	\end{minipage}
\end{minipage}
	
% \a %
% 	\rayr{\larger tmFlnF/menF}{tamlan-men} (\elv1), \\
% 	\rayr{\larger tmFlnF/smF}{tamlan-sam} (\elv2), \\
% 	\rayr{\larger tmFlnF/kj}{tamlan-kay} (\elv3), \medskip
% 	
%	etc.
\xe
\end{figure}

In order to form yet higher numbers, the suffix\index{suffixes} \rayr{/nNF}{-nang} is appended
to numerals: \rayr{menNF}{menang} (←~\xayr{menF}{men}{1} +
\rayr{/nNF}{-nang}), \xayr{smNF}{samang}{2} (←~\rayr{smF}{sam} +
\rayr{/nNF}{-nang}), \rayr{kjnNF}{kaynang} (←~\xayr{kj}{kay}{3} +
\rayr{/nNF}{-nang}), etc.\ While \rayr{menNF}{menang} is used for 100, higher
forms in the \fw{nang} series each multiply the numeral from which they are
derived by the factor of a duodecimal myriad (=~20\,736\tsub{10}). Thus the
series in (\ref{ex:myriads}) emerges.

\begin{figure}[h]
\ex[everyex={\tabcolsep=0em},]\label{ex:myriads}
	\begin{tabular}[t]
	{l @{\quad} l @{\quad} l}
	\rayr{\larger smNF}{samang}
		& $12^{(2-1) \times 4} = 12^{4}$
		& myriad
		\\
		
	\rayr{\larger kynNF}{kaynang}
		& $12^{(3-1) \times 4} = 12^{8}$
		& myriad myriads
		\\
		
	\rayr{\larger yonNF}{yonang}
		& $12^{(4-1) \times 4} = 12^{12}$
		& myriad myriad myriads
		\\
		
	\rayr{\larger IrinNF}{irinang}
		& $12^{(5-1) \times 4} = 12^{16}$
		& myriad myriad myriad myriads
		\\
	\end{tabular}
	
	\medskip etc.
\xe
\end{figure}

The numeral which the \fw{nang} series word is based on essentially indicates 
the number of myriad groups, thus, 1-\fw{nang} maximally contains 
\elv\elv\elv\elv; 2-\fw{nang} maximally contains 
\elv\elv\elv\elv\,\elv\elv\elv\elv; 3-\fw{nang} maximally contains 
\elv\elv\elv\elv\,\elv\elv\elv\elv\,\elv\elv\elv\elv, etc.\ Furthermore, the 
\fw{nang} series words serve as unit words, and thus can be modified by 
numerals again, for instance, as in (\ref{ex:unitmod}).

\begin{figure}[h]
\pex[glwordalign=center]\label{ex:unitmod}
\a\label{ex:menangunit}\begingl
	\gla menang sam veyalan @ -kay //
	\glb {100} {2} {90} {3} //
	\glft 293\tsub{12} = 399\tsub{10} //
\endgl

\a\label{ex:samangunit}\begingl
	\gla samang henlan @ -miye menang sam veyalan @ -kay //
	\glb {1\,0000} {80} {6} {100} {2} {90} {3} //
	\glft 86\,0293\tsub{12} = 2\,115\,471\tsub{10} //
\endgl
\xe
\end{figure}

In (\ref{ex:menangunit}), \rayr{smF}{sam} modifies \rayr{menNF}{menang} to 
indicate that there are two sets of 100\tsub{12}. Likewise, in 
(\ref{ex:samangunit}), \rayr{smNF}{samang} is modified by 
\rayr{henFlnF/miye}{henlan-miye} to mean 86\tsub{12} times 10\,000\tsub{12}.
Unit words like \rayr{menNF}{menang}, \rayr{smNF}{samang}, etc.\ may also be 
used as (inanimate\index{animacy}) nouns, so it is possible to speak of 
\xayr{menNYe}{menangye}{hundreds}.\label{hundreds} To express `hundreds of
people', however, the head of the genitive NP\index{phrase types!noun phrase} is pluralized exceptionally, even
if it is a \fw{plurale tantum}\index{number!plural}, like \xayr{kejnmF}{keynam}{people} in
(\ref{ex:keynampltant}). In (\ref{ex:keynampltant}), \rayr{kejnmF}{keynam} is
morphologically a singular form referring semantically to a multitude\index{number!plural}. It is
usually treated as a \textit{plurale tantum} in that it triggers plural\index{number!plural}
agreement\index{agreement} in spite of being morphologically singular, which is illustrated in
(\ref{ex:keynamunmkd}). In (\ref{ex:keynammkd}), the word still receives
otherwise redundant plural\index{number!plural} marking to express the difference in meaning from
(\ref{ex:keynamunmkd}).

\begin{figure}[h]
\pex\label{ex:keynampltant}
\a\label{ex:keynamunmkd}\begingl
	\gla Ang @ bengyon \textbf{keynam} menang kanānya {desay iray}. //
	\glb ang= beng-yon keynam-Ø menang kanān-ya {desay iray} //
	\glc \AgtT{}= attend-\TplN{} people-\Top{} hundred wedding-\Loc{} 
		royal //
	\glft `A hundred people attended the royal wedding.' //
\endgl

\a\label{ex:keynammkd}\begingl
	\gla Ang @ bengyon \textbf{keynamye} menang kanānya {desay iray}. //
	\glb ang= beng-yon keynam-ye-Ø menang kanān-ya {desay iray} //
	\glc \AgtT{}= attend-\TplN{} people-\Pl{}-\Top{} hundred wedding-\Loc{} 
		royal //
	\glft `Hundreds of people attended the royal wedding.' //
\endgl
\xe
\end{figure}

In order to indicate that myriad groups have been skipped, the conjunction\index{conjunctions}
\xayr{nj}{nay}{and} is used to avoid confusion, as shown in
(\ref{ex:numconfuse}), or simply to avoid having two single-digit numerals
following\index{word order} each other, as illustrated by (\ref{ex:numsgldig}).

\begin{figure}[h]
\pex[glwordalign=center]\label{ex:numconfuse}
\a\begingl
	\gla samang menang men henlan @ -miye //
	\glb {1\,0000} {100} {1} {80} {6} //
	\glft 186\,0000\tsub{12} = 5\,101\,056\tsub{10} //
\endgl

\a\begingl
	\gla samang menang men nay henlan @ -miye //
	\glb {1\,0000} {100} {1} and {80} {6} //
	\glft 100\,0086\tsub{12} = 2\,986\,086\tsub{10} //
\endgl
\xe
\end{figure}

\begin{figure}[h]
\pex~[glwordalign=center]\label{ex:numsgldig}
\a\ljudge\ques\begingl
	\gla menang mal ito //
	\glb {100} {\ten} {7} //
\endgl

\a\begingl
	\gla menang mal nay ito //
	\glb {100} {\ten} and {7} //
	\glft \ten07\tsub{12} = 1\,447\tsub{10} //
\endgl
\xe
\end{figure}

\subsubsection{Fractions}

\begin{table}[p]\centering
\caption[Simple fractions from $\frac{1}{2}$ to $\frac{1}{\elv}$]{Simple 
fractions from ¹⁄₂ to ¹⁄\tsub{\elv}}
\begin{tabu} to .75\linewidth {X[c] I X[c] I}
\toprule\tableheaderfont
Numeral
	& Word
	& Numeral
	& Word
	\\
\toprule

$\frac{1}{2}$
	& mesam % \rayr{mesmF}{mesam}
	& $\frac{1}{7}$
	& menito % \rayr{menito}{menito}
	\\ [.25\baselineskip]

$\frac{1}{3}$
	& menkay % \rayr{meMkj}{menkay}
	& $\frac{1}{8}$
	& menyen % \rayr{menFyenF}{menyen}
	\\ [.25\baselineskip]

$\frac{1}{4}$
	& menyo % \rayr{menFyo}{menyo}
	& $\frac{1}{9}$
	& menveya % \rayr{menFvey}{menveya}
	\\ [.25\baselineskip]

$\frac{1}{5}$
	& meniri % \rayr{meniri}{meniri}
	& $\frac{1}{\ten}$
	& memal % \rayr{memlF}{memal}
	\\ [.25\baselineskip]

$\frac{1}{6}$
	& memiye % \rayr{memiye}{memiye}
	& $\frac{1}{\elv}$
	& mentam % \rayr{meMtmF}{mentam}
	\\

\bottomrule
\end{tabu}
\label{tab:smallfrac}
\end{table}

So far, we have explored only whole numbers. Things can often be divided up
into smaller parts as well, though. The main way to express common fractions
like $\frac{1}{2}$, $\frac{1}{3}$, $\frac{1}{4}$, etc.\ is to prepend
\xayr{menF}{men}{one} to the denominator. The full paradigm of fractional
numerals from $\frac{1}{2}$ to $\frac{1}{\elv}$ is given in
\autoref{tab:smallfrac}. Note that a number of these fractions have slightly
irregular forms due to assimilation in consonant clusters. In order to
introduce a numerator, the fraction numeral is used as a unit word which is
modified by a regular cardinal numeral, as (\ref{ex:simplefrac}) shows.

\begin{figure}[h]
\pex\label{ex:simplefrac}
\a\begingl
	\gla Ang @ ilca {} @ Yan vadisānley mesam. //
	\glb ang= ilt-ya Ø= Yan vadisān-ley mesam //
	\glc \AgtT{}= buy-\TsgM{} \Top{}= Yan bread-\PargI{} half //
	\glft `Yan bought half a loaf of bread.' //
\endgl

\a\begingl
	\gla Ang @ ilce {} @ Mali sikanley menyo kay kipunena. //
	\glb ang= ilt-ye Ø= Mali sikan-ley menyo kay kipunena //
	\glc \AgtT{}= buy-\TsgM{} \Top{}= Mali pound-\PargI{} fourth three
		cheese-\Gen{} //
	\glft `Mali bought a three-quarter pound of cheese.' //
\endgl
\xe
\end{figure}

In order to express compound\index{compounds} numerals, \rayr{menF/}{men-} is prefixed\index{prefixes} to the
denominator head word, for instance as in (\ref{ex:hundredfrac_1}). However,
this may become confusing if numerators are used, so (\ref{ex:hundredfrac_2})
would be expressed less ambiguously as (\ref{ex:hundredfrac_3}) using the
ordinal form of the denominator.

\begin{figure}[h]
\pex
\a\label{ex:hundredfrac_1}\begingl
	\gla memallan-hen //
	\glb men-mallan-hen //
	\glb {$1/$ $10 \times 12^1 + 8$} //
	\glft $\frac{1}{\ten8_{12}}$ = $\frac{1}{128_{10}}$ //
\endgl

\a\label{ex:hundredfrac_2}\ljudge\ques\begingl
	\gla memenang ito menlan-yo kay //
	\glb men-menang ito menlan-yo kay //
	\glb {$1/$ $12^2$} {$7$} {$1 \times 12^1 + 4$} {$3$} //
	\glft $\frac{3}{714_{12}}$ = $\frac{3}{1024_{10}}$ //
\endgl

\a\label{ex:hundredfrac_3}\begingl
	\gla menangan ito menlan @ -yo nernanyena kay //
	\glb menang-an ito menlan -yo nernan-ye-na kay //
	\glc {$12^2$-\Nmlz{}} {$7$} {$1 \times 12^1$} {$+4$} part-\Pl{}-\Gen{} 
		$3$ //
	\glft `three of the 1\,024th part' //
\endgl
\xe
\end{figure}

\subsection{Ordinal numerals}

Ordinal numerals are formed by nominalization\index{nominalization}\index{derivation} from cardinal numerals. This may 
be another slightly odd strategy, however, it is in fact attested in Classical 
Tibetan\index{Classical Tibetan}, according to \citet{chungetal2014}, in reference to \citet{beyer1992}:

\blockcquote[626]{chungetal2014}{The suffix \fw{-pa} forms a noun from another 
noun, meaning `associated with N' (e.g. \fw{rta} `horse,' \fw{rta-pa} 
`horseman,' \fw{yi-ge} `letter,' \fw{yi-ge-pa} `one who holds a letter of 
office,' cf.\ \nocite{beyer1992} Beyer 1992: 117). When suffixed to ordinal 
numbers this suffix forms ordinals (e.g.\ \fw{gsum} `three,' \fw{gsum-pa} 
`third'; \fw{bcu} `ten,' \fw{bcu-pa} `tenth').}

\begin{table}\centering
\caption{Basic ordinal numerals}
\begin{tabu} to .75\linewidth {X[c] I X[c] I}
\toprule\tableheaderfont
Numeral
	& Word
	& Numeral
	& Word
	\\
\toprule

0th
	& jān % \rayr{dYaanF}{jān}
	& 6th
	& miyan % \rayr{miynF}{miyan}
	\\

1st
	& menan % \rayr{mennF}{menan}
	& 7th
	& itan % \rayr{ItnF}{itan}
	\\
	
2nd
	& saman % \rayr{smnF}{saman}
	& 8th
	& henan % \rayr{hennF}{henan}
	\\
	
3rd
	& kayan % \rayr{kynF}{kayan}
	& 9th
	& veyān % \rayr{veyaanF}{veyān}
	\\

4th
	& yan % \rayr{ynF}{yan}
	& \ten{}th
	& malan % \rayr{mlnF}{malan}
	\\

5th
	& iran % \rayr{Irni}{iran}
	& \elv{}th
	& taman % \rayr{tmnF}{taman}
	\\

\bottomrule
\end{tabu}
\label{tab:ordinals}
\end{table}

Unfortunately, neither \citet{chungetal2014} nor \citet{beyer1992} say whether
Classical Tibetan\index{Classical Tibetan} treats these derived forms as nouns or as numerals, or
whether it makes that distinction at all.\footnote{The collective wisdom of the
internet's conlanging community holds that one cannot truly innovate
grammatical structures; there is always a natural language which has evolved
the same construction, only with more complications. This is referred to as 
`\textsc{anadew}': `a nat[ural~]lang[uage] already dunnit except worse'
\citep{teoh2003}.} In Ayeri, ordinals are firmly treated as noun-like nominal
elements due to the derivational\index{derivation} suffix\index{suffixes} \rayr{/AnF}{-an} (compare
\autoref{subsec:nominalization}). Since nominals are the heads of NPs\index{phrase types!noun phrase}, this
also means that the ordinal numeral forms the head of the NP\index{phrase types!noun phrase} it occurs in,
instead of modifying the entity being counted like an ordinal numeral does.
This is illustrated in (\ref{ex:ord}). The paradigm for the ordinal numerals
from 0 to \elv\ can be found in
\autoref{tab:ordinals}.

\begin{figure}[h]
\pex\label{ex:ord}
\a\label{ex:ordanaph}\begingl
	\gla Ang @ Mahān menanas. //
	\glb ang= Mahān menan-as //
	\glc \Aarg{}= Mahān first-\Parg{} //
	\glft `Mahan is the first.' //
\endgl

\a\label{ex:ordrel}\begingl
	\gla Ang @ Mahān menanas si girenjāng. //
	\glb ang= Mahān menan-as si girend=yāng //
	\glc \Aarg{}= Mahān first-\Parg{} \Rel{} arrive=\TsgM{}.\Aarg{} //
	\glft `Mahān is the first to arrive.' //
\endgl

\a\label{ex:ordadj}\begingl
	\gla Ang @ girenja ku-menan diyan {} @ Mahān bahalanya //
	\glb ang= girend-ya ku=menan diyan Ø= Mahān bahalan-ya //
	\glc \Aarg{}= arrive-\TsgM{} like=first worthy \Top{}= Mahān 
		finish-\Loc{} //
	\glft `Mahān arrives at the finish as a worthy first.' //
\endgl

\a\label{ex:ordgen}\begingl
	\gla Ang @ tavya {} @ Mahān menanas ganyena yana. //
	\glb ang= tav-ya Ø= Mahān menan-as gan-ye-na yana //
	\glc \AgtT{}= get-\TsgM{} \Top{}= Mahān first-\Parg{} child-\Pl{}-\Gen{} 
		\TsgM{}.\Gen{} //
	\glft `Mahān gets his first child', \\
		\textit{literally:} `Mahān gets the first of his children.' //
\endgl
\xe
\end{figure}

As (\ref{ex:ordanaph}) shows, the ordinal numeral may serve as an anaphora
meaning `the $n$th (one)'. In these cases, animacy\index{animacy} is determined by the word
the ordinal references for purposes of case marking\index{case} and agreement\index{agreement}. Since
ordinals are treated as nominals, they can also be modified by both a relative
clause\index{relative clause}, as (\ref{ex:ordrel}) shows, and an adjective\index{adjectives}, as shown in
(\ref{ex:ordadj}). In order to include an entity whose rank in a series is
given, the counted entity appears as a genitive attribute; compare
(\ref{ex:ordgen}).

So far, only single-digit ordinals have been described. In order to form higher
ordinals, the head unit word receives the nominalizer\index{nominalization} with the rest of the term
trailing as a modifier, otherwise the number word\index{numerals} as such is nominalized\index{nominalization}.
Essentially, an ordinal in the `teens'
% \footnote{More specifically, $a \times 12^1 + b$ with $\{a,b \in \textbf{Z}
% \mid 0 < a,b < 12^1\}$.}
behaves like a `tight' noun compound\index{compounds}, while ordinals involving unit words for
powers of 12 higher than 1 behave as `loose' compounds\index{compounds} (compare
\autoref{subsubsec:endocomp}, p.~\pageref{loosecomp}). In order to illustrate,
the whole numeral \rayr{kjlnF/miye}{kaylan-miye} in (\ref{ex:ordtightcomp})
is nominalized and inflected for case, yielding
\rayr{kjlnF/miynFlej}{kaylan-miyanley}. This is analogous to such nouns as
\xayr{betjniMpurF}{betaynimpur}{grape} (literally `wine-berry'), which inflects
as a single unit---a `tight' compound\index{compounds}.

\begin{figure}[h]
\pex
\a\label{ex:ordtightcomp}\begingl
	\gla Adareng kaylan-miyanley bahisyena pericanena. //
	\glb ada-reng kaylan-miye-an-ley bahis-ye-na perican-ena //
	\glc that-\AargI{} {$3 \times 12^1 + 6$-\Nmlz{}-\PargI{}} 
		day-\Pl{}-\Gen{} year-\Gen{} //
	\glft `It is the 36th (=\,42nd) day of the year.' //
\endgl

\a\label{ex:ordloosecomp}\begingl
	\gla Adareng menanganley kaylan-miye bahisyena pericanena. //
	\glb ada-reng menang-an-ley kaylan-miye bahis-ye-na perican-ena //
	\glc that-\AargI{} {$12^2$-\Nmlz{}-\PargI{}} {$3 \times 12^1 
		+ 6$} day-\Pl{}-\Gen{} year-\Gen{} //
	\glft `It is the 136th (=\,186th) day of the year.' //
\endgl
\xe
\end{figure}

In (\ref{ex:ordloosecomp}), on the other hand, only the first unit word,
\rayr{menNF}{menang} is nominalized and inflected, yielding \rayr{menNnFlej}
{menanganley} with \rayr{kjlnF/miye}{kaylan-miye} following it uninflected.
This is analogous to \xayr{rlmpNF}{ralamapang}{fingernail} which is
transparently made up of \xayr{rlnF}{ralan}{nail} and \xayr{mpNF}{mapang}
{finger} and for which only the first constituent inflects, for instance,
\xayr{rlnYen mpNF}{ralanyena mapang}{of the fingernails} (nail-\Pl{}-\Gen{}
finger)---a `loose' compound\index{compounds}.

\subsection{Multiplicative numerals}

Whereas ordinals are derived from cardinal numerals by nominalization,
multiplicative numerals are derived\index{derivation} from ordinals (compare
\autoref{tab:ordinals}) in turn by putting them in the dative case\index{case!dative}: the suffix\index{suffixes}
\rayr{/ymF}{yam} is added to the ordinal form of the numeral. The resulting
multiplicative numeral can thus be used as an adverbial meaning `for the $n$th
time', or as an adverb\index{adverbs} meaning `$n$ times'. Context helps to disambiguate\index{ambiguity}
between the two, as well as temporal adverbs\index{adverbs} like \xayr{Iri}{iri}{already}. An
example of both uses of a multiplicative numeral is given by 
(\ref{ex:multnumwo}).

\begin{figure}[h]
\pex\label{ex:multnumwo}
\a\begingl
	\gla Linkaya iri ang @ Anang kayanyam. //
	\glb linka-ya iri ang= Anang kayanyam //
	\glc try-\TsgM{} already \Aarg{}= Anang third-\Dat{} //
	\glft `Anang already tries it for the third time.' //
\endgl

\a\begingl
	\gla Linkaya iri kayanyam ang @ Anang. //
	\glb linka-ya iri kayanyam ang= Anang //
	\glc try-\TsgM{} already three.times \Aarg{}= Anang //
	\glft `Anang already tried it three times.' //
\endgl
\xe
\end{figure}

Compound\index{compounds} multiplicative numerals are treated as analogous to ordinals, that is,
for composite numerals smaller than $12^2$, the derivational\index{derivation} marking is placed
at the end of the composite numeral. Conversely, for composite numerals of
orders of magnitude above $12^1$, the head of the phrase receives all the
marking that makes it a multiplicative numeral while the rest trails
uninflected as a modifier; see (\ref{ex:mutlnuminfl}).

\begin{figure}[h]
\pex\label{ex:mutlnuminfl}
\a\begingl
	\gla kaylan-tamanyam //
	\glb kay-lan-tam-an-yam //
	\glc {$3 \times 12^1 + 11$-\Nmlz{}-\Dat{}} //
	\glft `3\elv{} (=\,47\tsub{10}) times' //
\endgl

\a\begingl
	\gla menanganyam men samlan-kay //
	\glb menang-an-yam men sam-lan-kay //
	\glc {$12^2$-\Nmlz{}-\Dat{}} {$1$} {$2 \times 12^1 + 3$} //
	\glft `123 (=\,171\tsub{10}) times' //
\endgl
\xe
\end{figure}

\subsection{Distributive numerals}

Distributive numerals are formed similar to multiplicative numerals in that
they are based on a derivation\index{derivation} of the respective ordinal numeral, which itself
has the form of a nominalized\index{nominalization} cardinal numeral (compare
\autoref{tab:ordinals}). The derivational\index{derivation} affix in this case is the instrumental\index{case!instrumental}
marker, \rayr{/Eri}{-eri} (compare \autoref{subsubsec:instrumental},
p.~\pageref{subsubsec:instrumental}).
Distributive numerals refer to groups of $n$, as example (\ref{ex:distnum})
shows.

\begin{figure}[h]
\ex\label{ex:distnum}
\begingl
	\gla Ang @ sarayon burangjang kong besonya samaneri. //
	\glb ang= sara-yon burang-ye-yang kong beson-ya sam-an-eri //
	\glc \AgtT{}= go-\TplN{} animal-\Pl{}-\Aarg{} inside ship-\Loc{} 
		two-\Nmlz{}-\Ins{} //
	\glft `The animals went inside the ship two by two.' //
\endgl
\xe
\end{figure}

The formation of composite numerals mirrors that of multiplicative
numerals, in that composite numerals below $12^1$ are treated as single units
whereas composite numerals of orders of magnitude larger than $12^1$ mark only
the head word while the remainder of the phrase follows\index{word order} as an uninflected
modifier. (\ref{ex:distnuminfl}) provides an example of the inflected
distributive numeral.

\begin{figure}[h]
\pex\label{ex:distnuminfl}
\a\begingl
	\gla henlan-yaneri //
	\glb hen-lan-yo-an-eri //
	\glc {$8 \times 12^1 + 4$-\Nmlz{}-\Ins{}} //
	\glft `84 by 84 (=\,100\tsub{10})' //
\endgl

\a\begingl
	\gla menanganeri miye tamlan-yo //
	\glb menang-an-eri miye tam-lan-yo //
	\glc {$12^2$-\Nmlz{}-\Ins{}} {$6$} {$11 \times 12^1 + 4$} //
	\glft `6\elv{}4 by 6\elv{}4 (=\,1\,000\tsub{10})' //
\endgl
\xe
\end{figure}

\subsection{Number ranges}

Ayeri treats cardinal numerals more like adjectives than nouns, so using means
of case marking\index{case} is not possible. On the other hand, adpositions take both NPs\index{phrase types!noun phrase}
and CPs\index{phrase types!complementizer phrase} as complements\index{grammatical function!closed complement}, so that an adjective should be able to act as a
complement of an adposition as well. Since the numeral in the PP\index{phrase types!prepositional phrase} is treated
like an adjective, it is not marked for locative case\index{case!locative}, since adjectives do not
inflect for nominal categories (compare \autoref{sec:adjectives}). Ranges of
cardinal numbers may hence be expressed using the postposition
\xayr{pesnF}{pesan}{(up) until}. When counting starts at \xayr{menF}{men}{one},
this numeral may be dropped, like in English\index{English} `count to ten' instead of `count
from one to ten'; compare (\ref{ex:numstretch}).

\begin{figure}[h]
\ex\label{ex:numstretch}
\begingl
	\gla Kurye ang @ Pila \textup{(}men\textup{)} tam pesan. //
	\glb kur-ye ang= Pila (men) tam pesan //
	\glc count-\TsgF{} \Aarg{}= Pila (1) \elv{} until //
	\glft `Pila counts from 1 to \elv{} (=\,1 … 11\tsub{10}).' //
\endgl
\xe
\end{figure}

Since ordinal numerals are treated as nouns, they may receive case marking\index{case}.
This means that, in contrast to cardinal numerals, it is possible to express a
range using a combination of the genitive and the dative case\index{case!dative}, or again
\rayr{pesnF}{pesan} with its prepositional object\index{grammatical function!adpositional object} in the locative case\index{case!locative}.
Context is needed to disambiguate\index{ambiguity} whether the dative\index{case!dative} form of the numeral is a
multiplicative derivation\index{derivation} or an actual ordinal numeral in the dative case\index{case!dative}.
Examples of this are given in (\ref{ex:rangecase}).

\begin{figure}[h]
\pex\label{ex:rangecase}
\a\label{ex:rangecase1}\begingl
	\gla Ang @ gumasaya samanena pidimyena da-malanyam. //
	\glb ang= gum-asa=ya saman-ena bahis-ye-na da=malan-yam //
	\glc \AgtT{}= work-\Hab{}=\TsgM{}.\Top{} second-\Gen{} hour-\Pl{}-\Gen{} 
		one=tenth-\Dat{} //
	\glft `He usually works from the second hour to the tenth.' //
\endgl

\a\label{ex:rangecase2}\begingl
	\gla Ang @ yomaya {} @ Magay diyan edaya henanena bahisyena da-menlananya 
		pesan. //
	\glb ang= yoma-ya Ø= Magay diyan edaya henan-ena bahis-ye-na 
		da-menlanan-ya pesan //
	\glc \AgtT{}= exist-\TsgM{} \Top{}= Magay worthy here eighth-\Gen{} 
		day-\Pl{}-\Gen{} one=dozenth-\Loc{} until //
	\glft `Mr. Magay is here from the eighth to the dozenth day.' //
\endgl
\xe
\end{figure}

\xayr{smnen}{samanena}{from the first} in (\ref{ex:rangecase1}) and
\xayr{hennen}{henanena}{from the eighth} in (\ref{ex:rangecase2}) use the
genitive case\index{case!genitive} marker \rayr{/En}{-ena} (compare \autoref{subsubsec:genitive}) to
indicate the starting point. \xayr{d/mlnYmF}{da-malanyam}{to the tenth one} and
\xayr{d/menFlnFy pesnF}{da-menlanya pesan}{up until the dozenth one} indicate 
the end points. Since \rayr{menFlnF}{menlan} in (\ref{ex:rangecase2}) is 
embedded in a PP\index{phrase types!prepositional phrase} headed by the postposition \rayr{pesnF}{pesan}, it appears in 
the locative case\index{case!locative} instead of the dative case\index{case!dative} like \rayr{mlnF}{malan} in 
(\ref{ex:rangecase1}).

\index{numerals|)}

\section{Quantifiers and Intensifiers}
\label{sec:quantifiers}
\index{quantifiers|(}
\index{intensifiers|(}
\index{clitics|(}

The most common words expressing degree or quantity (both subsumed under the
label `quantifier' here) do not only follow\index{word order} verbs, nouns, adpositions,
adjectives, or other adverbs, but they cliticize to them, that is, they are
dependent morphemes (compare \autoref{clitics_quant},
p.~\pageref{clitics_quant}). The word stem---a lexical head which is usually
inflected except in the case of adjectives\index{adjectives}---serves as the host for the clitic
in all these cases. Examples of degree and quantifier suffixes\index{suffixes} and how they
interact with different parts of speech were already given in all the relevant
sections; an example from each section is repeated here in (\ref{ex:posquant})
for convenience. As we will see below, there are common quantifiers which
behave like regular, free words as well. It is possible to combine both the
suffixed\index{suffixes} and the free kinds with other quantifiers as long as those quantifiers
permit modification with regards to degree.

\begin{figure}[h]
\pex\label{ex:posquant}
\a\label{ex:nounquant2}\begingl
	\glpreamble With a noun (\ref{ex:nounquant}): //
	\gla Ajayon ganang-ikan kivo. //
	\glb aja-yon gan-ang=ikan kivo. //
	\glc play-\TsgN{} child-\Aarg{}=many small //
	\glft `Many small children are playing.' //
\endgl

\a\label{ex:adjquant2}\begingl
	\glpreamble With an adjective\index{adjectives} (\ref{ex:adjquant}): //
	\gla Eda-prikanreng napay-eng //
	\glb eda=prikan-reng {napay eng} //
	\glc this=soup-\AargI{} {spicy rather} //
	\glft `This soup is rather spicy.' //
\endgl

\a\label{ex:prepquant2}\begingl
	\glpreamble With an adposition\index{adpositions} (\ref{ex:prepquant}): //
	\gla Ang @ mitasaye pang-ikan mandayya tado. //
	\glb ang= mit-asa=ye.Ø pang=ikan manday-ya tado //
	\glc \AgtT{}= live-\Hab{}=\TsgF{}.\Top{} back=much forum-\Loc{} old //
	\glft `She used to live way behind the old forum.' //
\endgl

\a\label{ex:verbquant2}\begingl
	\glpreamble With a verb (\ref{ex:verbquant}): //
	\gla Ang @ rua @ apaya-kay {} @ Latun adanyaya. //
	\glb ang= rua= apa-ya=kay Ø= Latun adanya-ya //
	\glc \AgtT{}= must= laugh=\TsgM{}=a.little \Top{}= Latun that.one-\Loc{} //
	\glft `Latun had to laugh a little at that.' //
\endgl
\xe
\end{figure}

A number of quantifiers can be used to express both quantity and degree.
Especially prominent in this regard is \rayr{/IknF}{-ikan}, which encompasses
all of `many', `much' and `very', as displayed in examples (\ref{ex:nounquant2})
and (\ref{ex:prepquant2}): in the former case it appears as a quantifier of a
countable entity (\xayr{gnNF/IknF}{ganang-ikan}{many children}) and in the
latter case as an intensifier (\xayr{pNF/IknF}{pang-ikan}{way behind}). The
complete set of degree and quantifier suffixes\index{suffixes} is listed in
\autoref{tab:quantifiers}.

\begin{table}[tp]\centering
\caption{Degree and quantity suffixes}
\begin{tabu} to .75\linewidth {>{\itshape}l X X}
\toprule\tableheaderfont
Suffix
	& Degree
	& Quantity
	\\

\toprule
	
-ani % \rayr{/Ani}{-ani}
	& not at all
	& none at all
	\\

-aril % \rayr{/ArilF}{-aril}
	& 
	& some
	\\

-eng % \rayr{/ENF}{-eng}
	& rather, more
	& more
	\\
	
-hen % \rayr{/henF}{-hen}
	& completely
	& all, every, each
	\\

-ikan % \rayr{/IknF}{-ikan}
	& much, very
	& many, much
	\\
	
-ikoy % \rayr{/Ikoj}{-ikoy}
	& not very, less
	& not many, not much
	\\
	
-ing % \rayr{/INF}{-ing}
	& so
	&
	\\
	
-kay % \rayr{/kj}{-kay}
	& a bit, little
	& few
	\\

-ma % \rayr{/m}{-ma}
	& enough
	& enough
	\\

-mas % \rayr{/msF}{-mas}
	& some kind of
	&
	\\

-nama % \rayr{/nm}{-nama}
	& just, merely
	& just, only
	\\
	
-ngas % \rayr{/NsF}{-ngas}
	& almost
	&
	\\

-nyama % \rayr{/nYmF}{-nyama}
	& even
	&
	\\
	
-vā % \rayr{/vaa}{-vā}
	& most
	& most
	\\

-ven % \rayr{/venF}{-ven}
	& pretty, quite
	&
	\\

\bottomrule
\end{tabu}
\label{tab:quantifiers}
\end{table}

Grading and quantifying\index{quantifiers} expressions which deviate form the pattern of 
cliticization and instead are used as independent words are, most notably:
\xayr{AMkYu}{ankyu}{really},
\xayr{diriNF}{diring}{several},
\xayr{EkeNF}{ekeng}{over-, overly, too},
\xayr{heNsF}{hengas}{almost all},
\xayr{IknF/IknF}{ikan-ikan}{altogether, totally},
\xayr{IknFvaanFy}{ikanvānya}{at most, by and large},
% \xayr{kjvj}{kayvay}{without},
\xayr{kgnF}{kagan}{excessively, far too},
\xayr{menikneNF}{menikaneng}{another (one more)},
\xayr{midj}{miday}{approximately},
\xayr{pluNF}{palung}{another (a different kind)},
\xayr{regnFdej}{regandey}{bit by bit, gradually},
\xayr{sno}{sano}{both},
\xayr{vrYaanY}{varyānya}{at least}.
Besides, adjectives\index{adjectives} denoting a degree, like \xayr{IpnF}{ipan}{drastic, extreme,
radical} can of course also be used as intensifiers by way of adverbial uses.
The conversion is not explicitly marked. \rayr{IpnF}{ipan} in
(\ref{ex:adjasadv}) can thus also be used to mean `extremely' rather than
`extreme'.

\begin{figure}[h]
\ex\label{ex:adjasadv}
\begingl
	\gla Yang valuy ipan, sa @ silvyang va. //
	\glb yang valuy ipan sa= silv=yang va.Ø //
	\glc \Fsg{}.\Aarg{} glad extremely \PatT{}= see=\Fsg{}.\Aarg{} 
		\Second{}.\Top{} //
	\glft `I'm extremely glad to see you.' //
\endgl
\xe
\end{figure}

\index{clitics|)}
\index{intensifiers|)}
\index{quantifiers|)}

\section{Conjunctions}
\label{sec:conjunctions}
\index{conjunctions|(}

Section \ref{subsubsec:conjadv} already dealt with conjunctive adverbs as
sentence adverbs and their conjunction-like behavior. The present section is
about the `purely logical' conjunctions \xayr{nj}{nay}{and} and
\xayr{soyNF}{soyang}{or}, as well as their combination with 
\xayr{kmo}{kamo}{equal(ly), likewise} to form correlative conjunctions.

\subsection{Simple conjunction and disjunction}
\index{coordination|(}

Coordination is commonly achieved by the conjunction \xayr{nj}{nay}{and}. It is
placed in between the conjuncts, and works on all syntactic levels. Namely, it
may coordinate lexical heads, as well as phrases, and whole clauses. The
example sentences in (\ref{ex:and}) are ordered\index{word order} by increasing level of
coordination: (\ref{ex:andheads}) combines two adjective-phrase (AP) heads,
\xayr{trnF}{taran}{quiet} and \xayr{stYo}{saco}{cool}, which together make up
the predicative AP that is equated to \xayr{nNaaNF}{nangāng}{a/the house}. In
(\ref{ex:andphrases}), then, two agent NPs, \xayr{ynNF}{yanang}{a/the boy} and
\xayr{lyNF}{layang}{a/the girl}, together form the subject of the verb
\xayr{AgYynF}{ajayan}{play}. Lastly, (\ref{ex:andclauses}) shows two main
clauses coordinated, that is, \xayr{naaNF pisu}{nāng pisu}{we are tired} on the
one hand, and \xayr{tpnFnNF}{tapannang}{we are thirsty} on the other.

\begin{figure}[h]
\pex\label{ex:and}
\a\label{ex:andheads}\begingl
	\gla \textup{[\tsub{AP}~[\tsub{A}~} @ Taran @ \textup{]} nay 
		\textup{[\tsub{A}~} @ saco @ \textup{]]} nangāng. //
	\glb {} Taran {} nay {} saco {} nanga-ang //
	\glc {} quiet {} and {} cool {} house-\Aarg{} //
	\glft `The house is quiet and cool.' //
\endgl

\a\label{ex:andphrases}\begingl
	\gla Ajayan \textup{\tsub{NP}~[\tsub{N}~} @ yanang @ \textup{]} nay 
		\textup{[\tsub{N}~} @ layang @ \textup{]].} //
	\glb aja-yan {} yan-ang {} nay {} lay-ang {} //
	\glc play-\TplM{} {} boy-\Aarg{} {} and {} girl-\Aarg{} {} //
	\glft `The boy and the girl are playing.' //
\endgl

\a\label{ex:andclauses}\begingl
	\gla \textup{[\tsub{S}~} @ nāng pisu @ \textup{]} nay 
		\textup{[\tsub{IP}~} @ tapannang \textup{].} //
	\glb {} nāng pisu {} nay {} tapan-nang {} //
	\glc {} \Fpl{}.\Aarg{} tired {} and {} be.thirsty-\Fpl{}.\Aarg{} {} //
	\glft `We are tired and are thirsty.' //
\endgl
\xe
\end{figure}

Just as \rayr{nj}{nay} expresses \emph{con}junction, \xayr{soyNF}{soyang}{or} 
expresses \emph{dis}junction. It is likewise placed between two disjuncts and 
works at all levels as well---lexical heads, phrases, and clauses. Inclusive 
and exclusive `or' are not formally distinguished in Ayeri by the disjunction 
\rayr{soyNF}{soyang} alone, so context is necessary to contrast between them. 
Alternatively, a construction akin to English `either ... or' may be used to 
make the distinction explicit (see \autoref{subsec:corrconj}).

\begin{figure}[h]
\pex\label{ex:or}
\a\label{ex:orheads}\begingl
	\gla Pasyyang, yāng \textup{[\tsub{AP}~[\tsub{A}~} @ mino @ 
		\textup{]} soyang \textup{[\tsub{A}~} @ giday @ 
		\textup{]].} //
	\glb pasy=yang yāng {} mino {} soyang {} giday {} //
	\glc wonder=\Fsg{}.\Aarg{} \TsgM{}.\Aarg{} {} happy {} or {} sad {} //
	\glft `I wonder whether he is happy or sad.' //
\endgl

\a\label{ex:orphrases}\begingl
	\gla Le @ no @ ginvāng \textup{[\tsub{NP}~[\tsub{N}~} @ karon @ \textup{]} 
		soyang \textup{[\tsub{N}~} @ gali @ \textup{]]?} //
	\glb le= no= gin=vāng {} karon-Ø {} soyang {} gali-Ø {} //
	\glc \PatTI{}= want= drink=\Second{}.\Aarg{} {} water-\Top{} {} or {} 
		juice-\Top{} {} //
	\glft `Do you want to drink water or juice?' //
\endgl

\a\label{ex:orclauses}\begingl
	\gla \textup{[\tsub{IP}~} @ Beratu edauyi @ \textup{]} soyang
		\textup{[\tsub{IP}~} @ sa-sahu rangya \textup{]!} //
	\glb {} berata-u edauyi {} soyang {} sa\til{}saha-u rang-ya {} //
	\glc {} decide-\Imp{} now {} or {} return-\Imp{} home-\Loc{} {} //
	\glft `Decide now or go home!' //
\endgl
\xe
\end{figure}

As above, (\ref{ex:or}) shows different syntactic contexts for 
\rayr{soyNF}{soyang}. In (\ref{ex:orheads}), two adjectives, 
\xayr{mino}{mino}{happy} and \xayr{gidj}{giday}{sad} are put in opposition as
phrasal heads making up a predicative AP\index{phrase types!adjective phrase}. Then, in (\ref{ex:orphrases}), the
choice is between two nouns, \xayr{kronF}{karon}{water} and
\xayr{gli}{gali}{juice}, which jointly form the object of
\xayr{ginvaaNF}{ginvāng}{you drink}. Lastly, in (\ref{ex:orclauses}), two main 
clauses are in opposition---either disjunct forms a complete sentence on its 
own.

\index{coordination|)}

\subsection{Complex conjunction and disjunction}
\label{subsec:corrconj}

English\index{English} has a number of conjunctions made up of multiple parts which work
together as one expression. Among these are, notably,
\fw{as ...\ as},
\fw{both ...\ and},
\fw{either ...\ or},
\fw{neither ...\ nor},
\fw{rather ...\ than}, and
\fw{the ...\ the}.
Ayeri uses the adverb\index{adverbs} \xayr{kmo}{kamo}{equally, same, likewise} together with a
conjunction for many of these. \xayr{kmo—nj}{kamo ...\ nay}{equally ...\ and}
is equivalent to `both ...\ and': the correlative construction emphasizes that
two options are equal to each other. \xayr{sno}{sano}{both} may be used as a
synonym to \rayr{kmo}{kamo} as well here; compare (\ref{ex:bothand}).
Alternatively, it is possible to use a construction with
\xayr{njnj}{naynay}{(and) also}, as in (\ref{ex:andaswell}).

\begin{figure}[h]
\pex
\a\label{ex:bothand}\begingl
	\gla Ang @ vacay kamo piyuley nay obanley. //
	\glb ang= vac=ay.Ø kamo piyu-ley nay oban-ley //
	\glc \AgtT{}= like=\Fsg{}.\Top{} equally grain-\PargI{} and
		bean-\PargI{} //
	\glft `I like both grains and beans.' //
\endgl

\a\label{ex:andaswell}\begingl
	\gla Ang @ vacay piyuley, obanley naynay. //
	\glb ang= vac=ay.Ø piyu-ley oban-ley naynay //
	\glc \AgtT{}= like=\Fsg{}.\Top{} grain-\PargI{} bean-\PargI{} also //
	\glft `I like grains and also beans.' //
\endgl
\xe
\end{figure}

The example in (\ref{ex:bothand}) may be translated more literally as `I like
grains and beans equally', with two NPs in alternation, both being objects of a
transitive verb\index{verbs!transitive}, \xayr{vtYF/}{vac-}{like}. With predicative adjectives\index{adjectives!predicative}, the
verb \xayr{km/}{kama-}{(be) equal} may be used, which (\ref{ex:bothandpred})
shows, also compare \autoref{subsec:eqs}.

\begin{figure}[h]
\ex\label{ex:bothandpred}
\begingl
	\gla Ang @ kamayan mabo nay giday. //
	\glb ang= kama=yan.Ø mabo nay giday //
	\glc \AgtT{}= be.equal=\TplM{}.\Top{} hungry and thirsty //
	\glft `They are both hungry and thirsty.' //
\endgl
\xe
\end{figure}

\rayr{km/}{kama-} is one of Ayeri's copular verbs\index{verbs!comparative} used to express equality 
between two properties of its subject. The literal meaning of
(\ref{ex:bothandpred}) is thus, roughly, `They are as hungry as they are
thirsty'. The construction slightly differs from that used to do comparison\index{comparison} of
NPs\index{phrase types!noun phrase}, however, in that the conjunction \rayr{nj}{nay} is placed between both
predicative terms here. In order to express literal `be ...\ as ...\ as', thus,
the conjunction is dropped, as in (\ref{ex:asas}).

\begin{figure}[h]
\ex\label{ex:asas}
\begingl
	\gla Kamareng matikan helanas agonanya. //
	\glb kama=reng matikan helan-as agonan-ya //
	\glc be.equal=\TsgI{}.\Aarg{} hot oven-\Parg{} outside-\Loc{} //
	\glft `It's as hot as an oven outside.' //
\endgl
\xe
\end{figure}

% I've never thought of the following before, but I don't see why it shouldn't 
% make sense!

\rayr{kmo—nj}{kamo ...\ nay} is used to express `the ...\ the', that is, a 
proportional or antiproportional relationship between two amounts, sizes, 
or properties; using \xayr{sno}{sano}{both} here is judged less fitting. In 
order to express a relationship of equal increase/decrease in this way, 
conjuncts are additionally marked with the comparative\index{comparison} suffix\index{suffixes} 
\xayr{/ENF}{-eng}{more, rather} or its opposite, \xayr{/Ikoj}{-ikoy}{less}. See
(\ref{ex:thethe}) for an example of the former.

\begin{figure}[h]
\ex\label{ex:thethe}
\begingl
	\gla Ang @ tavyan kamo nakēng nay konjāng-eng. //
	\glb ang= tav=yan.Ø equal nake=eng nay kond=yāng=eng //
	\glc \AgtT{}= become=\TsgM{}.\Top{} equally tall=\Comp{} and 
		eat=\TsgM{}.\Aarg{}=more //
	\glft `The taller they get, the more they eat.' //
\endgl
\xe
\end{figure}

The type of correlative conjunction which selects one of two alternatives but
not both---that is, exclusive `or' (\textsc{xor})---is expressed by the
construction \xayr{kmo—soyNF}{kamo ...\ soyang}{equally ...\ or}, as
illustrated by (\ref{ex:eitheror}). For its negative\index{negation} opposite, `neither ...\
nor', the verb\index{verbs} is negated\index{mood!negative}. \rayr{miry}{miraya} and \rayr{kmyoNF}
{kamayong} in (\ref{ex:eitheror}) thus need to change to \rayr{mirojy}{miroyya}
and \rayr{kmojyoNF}{kamoyyong} to express the sentences' negative\index{negation} counterpart.
% negation must be used, which is displayed in (\ref{ex:neithernor}).

\begin{figure}[h]
\pex\label{ex:eitheror}
\a\label{ex:eitherorvb}\begingl
	\gla Ang @ miraya kamo Ajān adaley eda-konkyanya soyang da-mararya. //
	\glb ang= mira-ya kamo Ajān adaley eda=konkyan-ya soynag da=mararya //
	\glc \AgtT{}= do-\TsgM{} equally Ajān that-\PargI{} this=month-\Loc{} 
		or such-next //
	\glft `Ajān does it either this month or next.' //
\endgl

\a\label{ex:eitherorpred}\begingl
	\gla Kamayong mabo soyang krito mirampaluy. //
	\glb kama=yong mabo soyang krito mirampaluy //
	\glc be.equal=\TsgN{}.\Aarg{} hungry or angry otherwise //
	\glft `They are either hungry or otherwise angry.' //
\endgl

\xe
\end{figure}

% \begin{figure}[h]
% \pex\label{ex:neithernor}
% \a\label{ex:neithernorvb}\begingl
% 	\gla Ang @ tahoyye kamo {} @ Sipra netuas soyang kinās. //
% 	\glb ang= taha-oy-ye kamo Ø= Sipra netu-as soyang kinās //
% 	\glc \AgtT{}= have-\Neg{}-\TsgF{} equally \Top{}= Sipra brother-\Parg{} 
% 		or sister-\Parg{} //
% 	\glft `Sipra has neither a brother nor a sister.' //
% \endgl

% \a\label{ex:neithernorpred}\begingl
% 	\gla Ang @ kamuay layas soyang veno. //
% 	\glb ang= kama-oy=ay.Ø lay-as soyang veno //
% 	\glc \AgtT{}= be.equal-\Neg{}=\Fsg{}.\Aarg{} girl-\Parg{} or beautiful //
% 	\glft `I am neither a girl nor beautiful.' //
% \endgl
% \xe
% \end{figure}

\index{conjunctions|)}
