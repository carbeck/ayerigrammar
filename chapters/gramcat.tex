\chapter{Grammatical categories}
\label{ch:gramcat}

While the previous chapter was about general mechanisms of morphological
marking in Ayeri, this chapter will dive into the various parts of speech in
order to describe their morphology with a closer look. I will begin with nouns
as the main carriers of meaning, then deal with other parts of speech that
regularly feature in the noun phrase or in combination with it---pronouns,
adjectives, and adpositions. Following this, there will be a discussion of
verbs and adverbs before moving on to numerals and conjunctions.

\section{Nouns}
\label{sec:nouns}
\index{nouns|(}

Nouns in Ayeri have \emph{gender} and \emph{number} as their inherent
grammatical properties. Besides common nouns, there are, of course, also proper
nouns (that is, names) and nominalizations. Nouns, as the heads of NPs, are
also assigned \emph{case} by the verb, which is a third grammatical property
they display. For an illustration of the declension paradigms, compare Tables
\ref{tab:anideclcons} to \ref{tab:inandeclvow}.

\begin{table}[t]
\caption[Declension paradigm for \xayr{bdnF}{badan}{father}]{Declension 
paradigm for \xayr{bdnF}{badan}{father} (animate; consonantal root)}
\begin{tabu} to \linewidth {X[1] I[2] X[4] I[2] X[4]}
\tableheaderfont\toprule

	& \multicolumn2{c}{Singular}
	& \multicolumn2{c}{Plural}
	\\

\midrule
	
\Top{}
	& badan
	& `the father'
	%
	& badanye
	& `the fathers'
	\\

\midrule

\Aarg{}
	& badanang
	& `father'
	%
	& badanjang
	& `fathers'
	\\

\Parg{}
	& badanas
	& `father' (obj.)
	%
	& badanjas
	& `fathers' (obj.)
	\\

\Dat{}
	& badanyam
	& `to the father'
	%
	& badanjyam
	& `to the fathers'
	\\

\midrule

\Gen{}
	& badanena
	& `of the father'
	%
	& badanyena
	& `of the fathers'
	\\
	
\Loc{}
	& badanya
	& `at/in the father'
	%
	& badanjya
	& `at/in the fathers'
	\\

\Caus{}
	& badanisa
	& `due to the father'
	%
	& badanjisa
	& `due to the fathers'
	\\

\Ins{}
	& badaneri
	& `with the father'
	%
	& badanyeri
	& `with the fathers'
	\\

\bottomrule
\end{tabu}
\label{tab:anideclcons}
\end{table}

\begin{table}[t]
\caption[Declension paradigm for \xayr{maav}{māva}{mother}]{Declension 
paradigm for \xayr{maav}{māva}{mother} (animate; vocalic root)}
\begin{tabu} to \linewidth {X[1] I[2] X[4] I[2] X[4]}
\tableheaderfont\toprule

	& \multicolumn2{c}{Singular}
	& \multicolumn2{c}{Plural}
	\\

\midrule
	
\Top{}
	& māva
	& `the mother'
	%
	& māvaye
	& `the mothers'
	\\

\midrule

\Aarg{}
	& māvāng
	& `mother'
	%
	& māvajang
	& `mothers'
	\\

\Parg{}
	& māvās
	& `mother' (obj.)
	%
	& māvajas
	& `mothers' (obj.)
	\\

\Dat{}
	& māvayam
	& `to the mother'
	%
	& māvajyam
	& `to the mothers'
	\\

\midrule

\Gen{}
	& māvana
	& `of the mother'
	%
	& māvayena
	& `of the mothers'
	\\
	
\Loc{}
	& māvaya
	& `at/in the mother'
	%
	& māvajya
	& `at/in the mothers'
	\\

\Caus{}
	& māvaisa
	& `due to the mother'
	%
	& māvajisa
	& `due to the mothers'
	\\

\Ins{}
	& māvari
	& `with the mother'
	%
	& māvayeri
	& `with the mothers'
	\\

\bottomrule
\end{tabu}
\label{tab:anideclvow}
\end{table}

\begin{table}[t]
\caption[Declension paradigm for \xayr{kirinF}{kirin}{street}]{Declension 
paradigm for \xayr{kirinF}{kirin}{street} (inanimate; consonantal root)}
\begin{tabu} to \linewidth {X[1] I[2] X[4] I[2] X[4]}
\tableheaderfont\toprule

	& \multicolumn2{c}{Singular}
	& \multicolumn2{c}{Plural}
	\\

\midrule
	
\Top{}
	& kirin
	& `the street'
	%
	& kirinye
	& `the streets'
	\\

\midrule

\Aarg{}
	& kirinreng
	& `street'
	%
	& kirinyereng
	& `streets'
	\\

\Parg{}
	& kirinley
	& `street' (obj.)
	%
	& kirinyeley
	& `streets' (obj.)
	\\

\Dat{}
	& kirinyam
	& `to the street'
	%
	& kirinjyam
	& `to the streets'
	\\

\midrule

\Gen{}
	& kirinena
	& `of the street'
	%
	& kirinyena
	& `of the streets'
	\\
	
\Loc{}
	& kirinya
	& `at/in the street'
	%
	& kirinjya
	& `at/in the streets'
	\\

\Caus{}
	& kirinisa
	& `due to the street'
	%
	& kirinjisa
	& `due to the streets'
	\\

\Ins{}
	& kirineri
	& `with the street'
	%
	& kirinyeri
	& `with the streets'
	\\

\bottomrule
\end{tabu}
\label{tab:inandeclcons}
\end{table}

\begin{table}[t]
\caption[Declension paradigm for \xayr{per}{pera}{measure}]{Declension 
paradigm for \xayr{per}{pera}{measure} (inanimate; vocalic root)}
\begin{tabu} to \linewidth {X[1] I[2] X[4] I[2] X[4]}
\tableheaderfont\toprule

	& \multicolumn2{c}{Singular}
	& \multicolumn2{c}{Plural}
	\\

\midrule
	
\Top{}
	& pera
	& `the measure'
	%
	& peraye
	& `the measures'
	\\

\midrule

\Aarg{}
	& perareng
	& `measure'
	%
	& perayereng
	& `measures'
	\\

\Parg{}
	& peraley
	& `measure' (obj.)
	%
	& perayeley
	& `measures' (obj.)
	\\

\Dat{}
	& perayam
	& `to the measure'
	%
	& perajyam
	& `to the measures'
	\\

\midrule

\Gen{}
	& perana
	& `of the measure'
	%
	& perayena
	& `of the measures'
	\\
	
\Loc{}
	& peraya
	& `at/in the measure'
	%
	& perajya
	& `at/in the measures'
	\\

\Caus{}
	& peraisa
	& `due to the measure'
	%
	& perajisa
	& `due to the measures'
	\\

\Ins{}
	& perari
	& `with the measure'
	%
	& perayeri
	& `with the measures'
	\\

\bottomrule
\end{tabu}
\label{tab:inandeclvow}
\end{table}

\subsection{Gender}
\label{subsec:gender}
\index{gender|(}

% \begin{figure}[tb]\centering
% \begin{forest}
% where n children=0{tier=word}{}, shorter edges,
% [grammatical gender
% 	[animate
% 		[masculine]
% 		[feminine]
% 		[neuter]
% 	]
% 	[inanimate]
% ]
% \end{forest}
% \caption{Grammatical genders in Ayeri}
% \label{fig:gramgend}
% \end{figure}

Grammatical gender in Ayeri consists of two tiers which are subdivided into
four classes based on a mixture of semantic and ontological properties, see 
(\ref{ex:gramgend}).

\begin{figure}
\ex\label{ex:gramgend}
\begin{forest}
where n children=0{tier=word}{}, shorter edges,
[grammatical gender
	[animate
		[masculine]
		[feminine]
		[neuter]
	]
	[inanimate]
]
\end{forest}
\xe
\end{figure}

The animate gender refers, broadly speaking, to 
entities that are considered alive or are closely associated with living
things, such as events, concepts, or activities executed or connected to them.
The `masculine' and `feminine' subcategories are applied to humans, animals
whose sex is known (for example on behalf of breeding them or keeping them as
pets), and gods---basically anything that shows sexual dimorphism or is assumed
to be an exponent of it, as well as nouns referring to such entities in a
functional way, for instance, \xayr{bdnF}{badan}{father} and
\xayr{maav}{māva}{mother}. The remainder falls into the `neuter' 
category---plants, for instance, body parts, or animals whose sex is unknown. 
The `inanimate' category typically contains materials and things, such as 
tools. Furthermore, animals and plants change their category to inanimate as 
well if they serve as food. There are exceptions to either group, where 
elements appear in them for no obviously discernable reason. In order to 
illustrate, (\ref{ex:noungendex}) gives a few examples of each category.

\begin{figure}
\pex\label{ex:noungendex}
	\a Animate masculine:\medskip\\
		\xayr{\larger bdnF}{badan}{father}, 
		\xayr{\larger netu}{netu}{brother}, 
		\xayr{\larger AguynF}{aguyan}{rooster}, 
		\rayr{\larger AgYaanF}{Ajān}, 
		\rayr{\larger ltunF}{Latun}
		% FIXME: bull? stallion? dog?
	
	\a Animate feminine:\medskip\\
		\xayr{\larger maav}{māva}{mother}, 
		\xayr{\larger kin}{kina}{sister}, 
		\xayr{\larger Aguvj}{aguvay}{hen}, 
		\rayr{\larger mh}{Maha}, 
		\rayr{\larger tFraanj}{Trānay}
		% FIXME: cow? mare? bitch?
	
	\a Animate neuter:\medskip\\
		\xayr{\larger AdNF}{adang}{palm tree},
		\xayr{\larger bino}{bino}{color},
		\xayr{\larger IkmF}{ikam}{deer},
		\xayr{\larger kdaanF}{kadān}{harvest},
		\xayr{\larger tYaanF}{cān}{love},
		\xayr{\larger nN}{nanga}{house}, 
		\xayr{\larger tMpu}{tampu}{luck},
		\xayr{\larger yil}{yila}{foot}
	
	\needspace{2\baselineskip}
	\a Inanimate:\medskip\\
		\xayr{\larger AhlF}{ahal}{sand},
		\xayr{\larger hem}{hema}{egg},
		\xayr{\larger khnF}{kahan}{spear},
		\xayr{\larger meluNF}{melung}{yogurt},
		\xayr{\larger nusaanF}{nusān}{damage},
		\xayr{\larger pyutaanF}{payutān}{mathematics}
\xe
\end{figure}

There are also a number of duplicates like French \fw{le livre} `the book' and
\fw{la livre} `the pound', for instance, \ayr{bnnF} \fw{banan} (an.) 
`kindness, charity' or \ayr{bino} \fw{bino} (an.) `color' on the one hand, and
\ayr{bnnF} \fw{banan} (inan.)\ `quality' or \ayr{bino} \emph{bino} (inan.)\ 
`paint' on the other. Gender is reified by case marking as well as verb
agreement; it is not possible to read the gender of a noun from its
phonological makeup. (\ref{ex:gender}) illustrates differences in case marking
and agreement (inherent information on grammatical features underneath the
NPs).

\begin{figure}
\pex\label{ex:gender}
\a\label{ex:gender1}\begingl
	\gla Ang @ konja badan hemaley. //
	\glb ang= kond-ya badan-Ø hema-ley //
	\glc {} {} {\tiny [\TsgM{}.\An{}}] {\tiny [\TsgI{}]} //
	\glc \AgtT{}.\An{}= eat-\TsgM{}.\An{} father-\Top{} egg-\PargI{} //
	\glft `Father eats an egg.' //
\endgl

\a\label{ex:gender2}\begingl
	\gla Sa @ tombara kahanreng burang. //
	\glb sa= tomb-ara kahan-reng burang-Ø //
	\glc {} {} {\tiny [\TsgI{}]} {\tiny [\TsgN{}.\An{}]} //
	\glc \PatT{}.\An{}= kill-\TsgI{} spear-\AargI{} animal-\Top{} //
	\glft `The animal, the spear kills it.' //
\endgl
\xe
\end{figure}

In example (\ref{ex:gender1}), the noun in the agent NP,
\xayr{bdnF}{badan}{father}, bears the features [\Gend{}~\M{}, \Anim{}~$+$],
which triggers the animate agent topic agreement marker \rayr{ANF}{ang} on the
verb, since the agent NP is also topicalized. The verb also agrees in person
and number with the agent NP by way of the person marker \rayr{/y}{-ya} for
third person singular masculine. The object of the sentence,
\xayr{hem}{hema}{egg}, on the other hand bears the feature
[\Gend{}~\Inan{}, \Anim{}~$-$], so it receives the inanimate patient case
marker \rayr{/lej}{-ley} rather than its animate counterpart \rayr{/AsF}{-as}.

In (\ref{ex:gender2}), on the other hand, we see an inanimate agent,
\xayr{khnF}{kahan}{spear}, so the verb receives the marker \rayr{/Ar}{-ara} 
for third person singular inanimate rather than its animate neuter counterpart
\rayr{/yo}{-yo}. The (non-topicalized) NP's case marking shows that the agent 
of the clause is inanimate: \rayr{khnF}{kahan} carries the marker
\rayr{/reNF}{-reng}, which marks it as an inanimate agent. The object of the 
sentence, \xayr{burNF}{burang}{animal}, is also the topic, hence topic
agreement on the verb uses the marker \rayr{s}{sa} according to the NP being
animate, rather than its inanimate counterpart \rayr{le}{le}.

\index{gender|)}

\subsection{Number}
\index{number|(}

Ayeri only distinguishes singular and plural in nouns, which receive plural 
marking; verbs, then, agree with agent NPs in number in the canonical case. 
Ordinarily, nouns in Ayeri are countable, however, there is also a group of 
uncountable nouns as well as a (small) group of nouns which are always plural. 
As above, I will list a few words from each group for illustration:

%\needspace{2\baselineskip}
\begin{figure}[h]
\pex[everyex={\tabcolsep=0em},]
	\a Countable nouns:\label{ex:plurals}\medskip\\
		\begin{tabular}[t]
		{l @{\enspace---\enspace} l}
		
		\xayr{\larger AgYmF}{ajam}{toy}
			& \xayr{\larger AgYmFye}{ajamye}{toys}, %
			\medskip \\
				
		\xayr{\larger devo}{devo}{head}
			& \xayr{\larger devoye}{devoye}{heads}, %
			\medskip \\
				
		\xayr{\larger InunF}{inun}{fish}
			& \xayr{\larger InunFye}{inunye}{fish} (pl.),%
			\medskip \\
				
		\xayr{\larger netu}{netu}{brother}
			& \xayr{\larger netuye}{netuye}{brothers};
			\\
		\end{tabular}
	
	\a Uncountable nouns:\medskip\\
		\xayr{\larger AhlF}{ahal}{sand}, 
		\xayr{\larger bkj}{bakay}{stuff}, 
		\xayr{\larger ghaanF}{gahān}{hope}, 
		\xayr{\larger miNnF}{mingan}{ability};
	
	\a Plurale tantum nouns:\medskip\\
		\xayr{\larger burNF}{burang}{lifestock, 
			cattle},\footnotemark~
		\xayr{\larger gneNnF}{ganengan}{siblings}, 
		\xayr{\larger kejnmF}{keynam}{people}, 
		\xayr{\larger tNF}{tang}{ears}.
\xe
\end{figure}

\footnotetext{Specifically in this meaning; \rayr{burNF}{burang} can also 
simply mean `animal', in which case there is a plural form 
\xayr{burNFye}{burangye}{animals}.}

Most concrete things that exist as discrete entities are countable, also, for
instance, animals and lifestock. Fish, deer, sheep etc.\ are thus countable,
unlike in English; pants, pliers, scissors, glasses, etc.\ are by default
singular, also unlike in English. Uncountable, on the other hand, are materials
in general or abstract concepts. There are also a number of nouns which are
plural by default, most notably entities which often occur in groups, but there
is as well the odd word for which there seems to be no reason to be included in
this group, for instance, \xayr{bino}{bino}{paint}, and
\xayr{giMbj}{gimbay}{sorrows}. A few body parts are also \fw{plurale tantum}
nouns, especially those which occur in pairs (\xayr{niv}{niva}{eye} is a
notable exception).

\index{morphophonology!of the plural marker}\index{plural}
As demonstrated in (\ref{ex:plurals}), the noun plural marker is 
\rayr{/ye}{-ye}, which in native orthography also occurs in the variant 
\ayr{*Ye} or \ayr{ʲ*e} due to allography. As described above
(\autoref{pluralmorph}, p.~\pageref{pluralmorph}), the plural marker may also
be reduced to [dʒ] \orth{-j} before case suffixes beginning with /j/ or with a
vowel other than /e/, like \rayr{/ANF}{-ang} (\Aarg{}) or \rayr{/ymF}{-yam}
(\Dat{}), as demonstrated in (\ref{ex:pluralaltern}). For \fw{pluralia tantum},
to express a singular entity, it is always possible to use a genitive phrase
like \xayr{—/En menF}{…-ena men}{one of …} (…-\Gen{} one), like in
(\ref{ex:dualplural}).

\begin{figure}[h]
\ex\labels\label{ex:pluralaltern}
	\begin{tabular}[t]{@{\tl\quad} l @{\enspace→\enspace} l @{\smallskip}}
	\rayr{\larger dirnNF}{diranang} (uncle-\Aarg{})
		+ \rayr{\larger /ye}{-ye} (\Pl{})
		& \rayr{\larger dirnFye\_aNF}{diranjang} (uncle-\Pl{}-\Aarg{}),
		\\
	\rayr{\larger dirnen}{diranena} (uncle-\Gen{})
		+ \rayr{\larger /ye}{-ye} (\Pl{})
		& \rayr{\larger dirnFyen}{diranyena} (uncle-\Pl{}-\Gen{}),
		\\
	\rayr{\larger dirnFymF}{diranyam} (uncle-\Dat{})
		+ \rayr{\larger /ye}{-ye} (\Pl{})
		& \rayr{\larger dirnFyeymF}{diranjyam} (uncle-\Pl{}-\Dat{}).
		\\
	\end{tabular}
\xe
\end{figure}

\begin{figure}[h]
\pex\label{ex:dualplural}
\a\begingl
	\gla Nupayon tangang nā. //
	\glb nupa-yon tang-ang nā //
	\glc hurt-\TplN{} ears-\Aarg{} \Fsg{}.\Gen{} //
	\glft `My ears hurt.' //
\endgl

\a\label{ex:gensubj}\begingl
	\gla Na @ nupareng tang men nā. //
	\glb na= nupa=reng tang-Ø men nā //
	\glc \GenT{}= hurt=\TsgI{}.\Aarg{} ears-\Top{} one \Fsg{}.\Gen{} //
	\glft `Of my ears, one is hurting.' //
\endgl
\xe
\end{figure}

Number in nouns can also be manipulated by quantifiers which attach to declined
nouns as suffixes---or rather, enclitics. In this case, when plurality is
indicated by the quantifier, the noun is not additionally marked for number;
the verb, however, keeps agreeing in number. This is illustrated in
(\ref{ex:quantplur}).

\begin{figure}[h]
\pex\label{ex:quantplur}
\a\begingl
	\gla Ajayon ganjang kivo. //
	\glb aja-yon gan-ye-ang kivo //
	\glc play-\TplN{} child-\Pl{}-\Aarg{} small //
	\glft `The small children are playing.' //
\endgl
	
\a\label{ex:nounquant}\begingl
	\gla Ajayon ganang-ikan kivo. //
	\glb aja-yon gan-ang=ikan kivo. //
	\glc play-\TplN{} child-\Aarg{}=many small //
	\glft `Many small children are playing.' //
\endgl
\xe
\end{figure}

Likewise, when nouns are modified by numerals,\index{numerals} plurality is not
normally marked again on the noun. In example (\ref{ex:plurnorm}), we see a
plural noun, \xayr{nN}{nanga}{house}, and in (\ref{ex:plurnum}) the same phrase
is repeated with plurality implied by the use of a numeral,
\xayr{smF}{sam}{two}. The plural noun itself appears unmarked in its singular
form in this case. An exception to this is the use of numeral powers, like
\xayr{lnF}{lan}{dozen}, \xayr{menNF}{menang}{gross}, etc.\ in an unspecified
way, like `dozens of people'. To convey that the numeral is not to be
understood as a precise value, the modified noun appears in the plural---even
if it is a \fw{plurale tantum} like \xayr{kejnmF}{keynam}{people} in
(\ref{ex:keynamplur}).

\begin{figure}[h]
\pex
\a\label{ex:plurnorm}\begingl
	\gla Ang @ no @ vehya sitang-yām nangajas veno nay hiro. //
	\glb ang= no= veh=ya.Ø sitang=yām nanga-ye-as veno nay hiro //
	\glc \AgtT{}= want= build=\TsgM.\Top{} self=\TsgM{}.\Dat{} 
		house-\Pl{}-\Parg{} pretty and new //
	\glft `He wants to build himself pretty new houses.' //
\endgl

\a\label{ex:plurnum}\begingl
	\gla Ang @ no @ vehya sitang-yām nangās sam veno nay hiro. //
	\glb ang= no= veh=ya.Ø sitang=yām nanga-as sam veno nay hiro //
	\glc \AgtT{}= want= build=\TsgM.\Top{} self=\TsgM{}.\Dat{} house-\Parg{} 
		two pretty and new //
	\glft `He wants to build himself two pretty new houses.' //
\endgl
\xe
\end{figure}

\begin{figure}
\ex\label{ex:keynamplur}%
\begingl
	\gla Bengyon keynamjang menang. //
	\glb beng-yon keynam-ye-ang menang //
	\glc attend-\TsgN{} people-\Pl{}-\Aarg{} gross //
	\glft `Hundreds of people attended.' //
\endgl\xe
\end{figure}

% This is a new rule; earlier, names were treated as countable but still
% carried special case marking. I found this slightly weird, however, so, let
% us simply assert this new rule, which should make things more consistent. The
% odd case of a pluralized name could still be explained as individual
% variation, though I can't think of an example where this was ever an issue.
%
As we have seen in various examples above, proper nouns in Ayeri do not receive
inflection for case by suffixes as common nouns do, and for the purpose of
number they are treated as uncountable in Ayeri---they resist inflection by
suffixation, marking their special status.\footnote{Many common names in Ayeri
are derived from regular words in the language, so the language needs to have a
way to distinguish between regular use and use as a name. For instance, the
name \rayr{ynF}{Yan} also means `boy, son' as a common noun.} However, they can
still be modified by quantifiers and quantifying clitics; verb agreement as
well can be used to indicate plurality, compare (\ref{ex:verbplur}).

\begin{figure}[h]
\pex\label{ex:verbplur}
\a\begingl
	\gla Sahayan cabo ekeng ang @ Yan. //
	\glb saha-yan cabo ekeng ang= Yan //
	\glc come-\TplM{} late too \Aarg{}= Yan //
	\glft `The Yans are coming too late.' //
\endgl

\a\begingl
	\gla Ang @ apateng sa @ Yan-ikan. //
	\glb ang= apa=teng sa= Yan=ikan //
	\glc \AgtT{}= laugh=\TplF{}.\Aarg{} \Parg{}= Yan=all //
	\glft `They laughed at (all) the Yans.' //
\endgl
\xe
\end{figure}

\index{number|)}

\subsection{Case}
\label{subsec:case}
\index{cases|(}

As demonstrated in the declension tables at the beginning of this section
(Tables \ref{tab:anideclcons}--\ref{tab:inandeclvow}), Ayeri's NPs are marked
for case, which is governed by the verb or assigned to adjuncts freely
depending on their purpose or meaning. Since in Ayeri, case marking is at least
partially based on semantics rather than purely on function or structure. This
causes a few exceptions, so it is better, in my opinion, not to use the classic
labels of nominative (S/A) and accusative (O), or of absolutive (S/P) and
ergative (O) for the first two core roles. Instead, I will be using the terms
`agent' and `patient', which I hope brings about some more clarity, especially
when discussing the mentioned exceptions later on. For a discussion of how
Ayeri deals with subjecthood, see \autoref{subsec:subjecthood}.

\subsubsection{Agent}
\label{subsubsec:agent}
\index{cases!agent|(}

What I call `agent' here is, to quote \citet{fillmore1968},
\textcquote[46]{fillmore1968}{the case of the typically animate perceived
instigator of the action identified by the verb}. \citet{fillmore1968}
himself qualifies this definition, however, in that the \textcquote[46,
footnote 31]{fillmore1968}{escape qualification `typically' expresses my
awareness that contexts which I will say require agents are sometimes occupied
by `inanimate' nouns like \fw{robot} or `human institution' nouns like
\fw{nation}}. \citet{payne1997} summarizes on prototypical agents with regards
to their topicality that a \textcquote[151]{payne1997}{less technical way of
expressing this fact is to say that people identify with and like to talk about
things that act, move, control events, and have power}.

Agents in Ayeri frequently embody the properties quoted by both
\citet{fillmore1968} and \citet{payne1997} in this regard, including 
\citet{fillmore1968}'s caveat. However, importantly, `agent' in Ayeri is 
a macrorole that may be applied to, for instance, instruments, experiencers,
and less typical actors as well, specifically, in absence of more prototypical
candidates for agenthood in a sentence. It thus comes very close to a
nominative, except that it does not need to be locus of the sentence's
topic\index{topic}---although agents very typically are topics, as
\citet[151]{payne1997} goes on to note.

The agent is marked by the suffix \rayr{/ANF}{-ang} for animate referents and
the suffix \rayr{/reNF}{-reng} for inanimate referents; names and verbal topic
agreement are marked by the clitic case markers \rayr{ANF}{ang} and
\rayr{ENF}{eng}, respectively. See (\ref{ex:agtmarking1}) and
(\ref{ex:agtmarking2}) for examples of each marker.

\begin{figure}
\pex\label{ex:agtmarking1}
\a\begingl
	\gla \textbf{Ang} @ tinkaya {} @ \textbf{Yan} kunangley. //
	\glb ang= tinka-ya Ø= Yan kunang-ley //
	\glc \AgtT{}= open-\TsgM{} \Top{}= Yan door-\PargI{} //
	\glft `Yan opens the door.' //
\endgl

\a\begingl
	\gla Le @ tinkaya \textbf{ayonang} kunang. //
	\glb le= tinka-ya ayon-ang kunang-Ø //
	\glc \PatT{}= open-\TsgM{} man-\Aarg{} door-\Top{} //
	\glft `The door is opened by a/the man',\\
		\textit{or:} `The door, a/the man opens it.' //
\endgl
\xe
\end{figure}

\begin{figure}
\pex\label{ex:agtmarking2}
\a\begingl
	\gla \textbf{Eng} @ tinkāra \textbf{tinkay} kunangley. //
	\glb eng= tinka-ara tinkay-Ø kunang-ley //
	\glc \AgtTI{}= open-\TsgI{} key-\Top{} door-\PargI{} //
	\glft `The key opens the door.' //
\endgl

\a\begingl
	\gla Tinkāra \textbf{kunangreng}. //
	\glb tinka-ara kunang-reng //
	\glc open-\TsgI{} door-\AargI{} //
	\glft `The door opens.' //
\endgl
\xe
\end{figure}

In predicative constructions, the constituent which a quality is assigned to or
which a judgment is made about is also assigned the agent case, as
(\ref{ex:prednpagt}) shows. With regards to constituents' roles in ditransitive
argument frames, donors are represented by agents in Ayeri as well, since they
are the origin of whatever is conceptually passed on to the recipient party,
compare (\ref{ex:ditragt}). Moreover, as (\ref{ex:causagt}) shows, the causee
is marked as an agent, not as a patient, since that would be semantically
incongrouous.

\begin{figure}[h]
\pex\label{ex:prednpagt}
\a\begingl
	\gla Tado \textbf{tinkayreng}. //
	\glb tado tinkay-reng //
	\glc old key-\AargI{} //
	\glft `The key is old.' //
\endgl

\a\begingl
	\gla \textbf{Ang} @ \textbf{Yan} nimpayās ban. //
	\glb ang= Yan nimpaya-as ban //
	\glc \Aarg{}= Yan runner-\Parg{} good //
	\glft `Yan is a good runner.' //
\endgl
\xe
\end{figure}

\begin{figure}[h]
\ex\label{ex:ditragt}%
\begingl
	\gla Le @ ilya \textbf{ang} @ \textbf{Yan} tinkay yam @ Cānlay. //
	\glb le= il-ya ang= Yan tinkay-Ø yam= Cānlay //
	\glc \PatT{}= give-\TsgM{} \Aarg{}= Yan key-\Top{} \Dat{}= Cānlay //
	\glft `The key, Yan gives it to Cānlay.' //
\endgl\xe
\end{figure}

\begin{figure}[h]
\ex\label{ex:causagt}
\begingl
	\gla Sā @ tinkaya \textbf{ang} @ \textbf{Yan} kunangley yan. //
	\glb sā= tinka-ya ang= Yan kunang-ley yan.Ø //
	\glc \CauT{}= open-\TsgM{} \Aarg{}= Yan door-\PargI{} \TsgM{}.\Top{} //
	\glft `They make Yan open a/the door',\\
		\textit{or:} `Because of them, Yan opens the door.' //
\endgl
\xe
\end{figure}

\index{cases!agent|)}

\subsubsection{Patient}
\index{cases!patient|(}

Patients are less of a definitional problem than agents, since in transitive
sentences, they are very typically undergoers, that is, the constituent that is
acted on, affected, or produced by the action expressed by the verb. The
patient case is thus the one assigned by default to direct objects---but also
to predicative nominals. Animate patients are marked by \rayr{/AsF}{-as},
inanimate ones by \rayr{/lej}{-ley}; for names and verbal topic agreement, the
markers are \rayr{s}{sa} and \rayr{le}{le}, respectively, compare
(\ref{ex:patmarking1}) and (\ref{ex:patmarking2}). In ditransitive sentences
like the one in (\ref{ex:ditrpat}), the theme is represented by the patient.

\begin{figure}
\pex\label{ex:patmarking1}
\a\begingl
	\gla Ang @ silvye {} @ Briha \textbf{sa} @ \textbf{Taryan}. //
	\glb ang= silv-ye Ø= Briha sa= Taryan //
	\glc \AgtT{}= see-\TsgF{} \Top{}= Briha \Parg{}= Taryan//
	\glft `Briha sees Taryan.' //
\endgl

\a\begingl
	\gla \textbf{Sa} @ manye ang @ Briha {} @ \textbf{Taryan}. //
	\glb sa= man-ye ang= Briha Ø= Taryan //
	\glc \PatT{}= greet-\TsgF{} \Aarg{}= Briha \Top{}= Taryan //
	\glft `Taryan is greeted by Briha',\\
		\textit{or:} `Taryan, Briha greets him.' //
\endgl
\xe
\end{figure}

\begin{figure}
\pex\label{ex:patmarking2}
\a\begingl
	\gla Ang @ rimaye {} @ Briha \textbf{kunangley}. //
	\glb ang= rima-ye Ø= Briha kunang-ley //
	\glc \AgtT{}= close-\TsgF{} \Top{}= Briha door-\PargI{} //
	\glft `Briha closes a/the door.' //
\endgl

\a\begingl
	\gla \textbf{Le} @ rimaye ang @ Briha \textbf{kunang}. //
	\glb le= rima-ye ang= Briha kunang-Ø //
	\glc \PatTI{}= close-\TsgF{} \Aarg{}= Briha door-\Top{} //
	\glft `The door is closed by Briha',\\
		\textit{or:} `The door, Briha closes it.' //
\endgl
\xe
\end{figure}

\begin{figure}
\ex\label{ex:ditrpat}
\begingl
	\gla Ang @ ilya {} @ Taryan \textbf{koyaley} yam @ Kandan. //
	\glb ang= il-ya Ø Taryan= koya-ley yam= Kandan //
	\glc \AgtT{}= give-\TsgM{} \Top{}= Taryan book-\PargI{} \Dat{}= Kandan //
	\glft `Taryan gives Kandan a book.' //
\endgl
\xe
\end{figure}

As the translations of the examples above show, topicalizing the patient can be
used to create an effect similar to English's passive voice, except that the
patient will not become marked by the agent case for logical reasons---this is
a notable difference from the nominative. Even if the agent NP is omitted to
form a passive in (\ref{ex:agtnotnom}), the patient NP will not be changed to
the agent case, since that would reverse the direction of action.

\begin{figure}
\ex\label{ex:agtnotnom}\begingl
	\gla Manya sa @ Taryan. {} Manya ang @ Taryan. //
	\glb man-ya sa= Taryan ≠ Man-ya ang= Taryan //
	\glc greet-\TsgM{} \Parg{}= Taryan {} greet-\TsgM{} \Aarg{}= Taryan //
	\glft `Taryan is greeted.' ≠ `Taryan greets.' //
\endgl\xe
\end{figure}

(\ref{ex:agtnotnom}) shows that the case of the NP will not change, however,
the verb will: it now agrees with the next argument in line, the patient NP. It
will not do so, however, if the order of arguments is simply scrambled, as in
(\ref{ex:verbscram}). This is to say that the verb does not simply agree with
whichever NP follows it, even if it can be assumed that verb agreement in Ayeri
developed along similar lines in-world, which will become especially apparent
in the discussion of pronouns.\footnote{Mismatches in agreement in connection
to scrambling such as exemplified by (\ref{ex:scramfalse}) are to be expected,
however. \citet{corbett2006}, notes that with regards to agreement in NP
conjuncts, \textcquote[62]{corbett2006}{distant agreement is rare, and that
agreement with the nearest noun phrase or agreement with all (resolution) is
much more common}. If there were an extensive corpus of texts written by Ayeri
speakers, it might be interesting to gather statistics on the number of words
between target and controller in relation to the prevalence of agreement
mismatches.}

\begin{figure}[h]
\pex[aboveglftskip=2em]\label{ex:verbscram}
\a\label{ex:scramcorr}\begingl
	\gla Sa @ manye {} @ Taryan ang @ Briha. //
	\glb sa= man-ye Ø= Taryan ang= Briha //
	\glc \PatT{}= greet-\tikzmark{target}\TsgF{} \Top{}= Taryan 
		\Aarg{}= \tikzmark{controller}Briha //
	\glft `Taryan is greeted by Briha',\\
		\textit{or:} `Taryan, Briha greets him.' //
\endgl

\a\label{ex:scramfalse}\ljudge* \begingl
	\gla Sa @ manya {} @ Taryan ang= Briha. //
	\glb sa= man-ya Ø= Taryan ang= Briha //
	\glc \PatT{}= greet-\tikzmark{target2}\TsgM{} \Top{}= %
		\tikzmark{controller2}Taryan \Aarg{}= Briha //
\endgl\xe
\begin{tikzpicture}[remember picture, overlay]
	\coordinate [below right=.25em and 1em   of {pic cs:controller}] (A);
	\coordinate [below right=1em   and 1em   of {pic cs:controller}] (B);
	\coordinate [below right=1em   and .75em of {pic cs:target}    ] (C);
	\coordinate [below right=.25em and .75em of {pic cs:target}    ] (D);
	\draw [-latex] (A) -- (B) -- (C) -- (D);
	\node (label) at ($(B)!0.5!(C)$) [below] {\tiny\itshape person 
		agreement};
	
	\coordinate [below right=.25em and 1em   of {pic cs:controller2}] (A);
	\coordinate [below right=1em   and 1em   of {pic cs:controller2}] (B);
	\coordinate [below right=1em   and .75em of {pic cs:target2}    ] (C);
	\coordinate [below right=.25em and .75em of {pic cs:target2}    ] (D);
	\draw [-latex, dashed] (A) -- (B) -- (C) -- (D);
	\node (label) at ($(B)!0.5!(C)$) [below] {\tiny\itshape *person 
		agreement};
\end{tikzpicture}
\end{figure}

Besides being the default case for direct objects, the patient case is also 
assigned to predicative nominals, by analogy with transitive sentences and in 
spite of the likening nature of the construction, compare (\ref{ex:predpat}).

\begin{figure}[h]
\ex\label{ex:predpat}%
\begingl
	\gla Ang @ Yan \textbf{nimpayās} ban. //
	\glb ang= Yan nimpaya-as ban //
	\glc \Aarg{}= Yan runner-\Parg{} good //
	\glft `Yan is a good runner.' //
\endgl\xe
\end{figure}

\index{cases!patient|)}

\subsubsection{Dative}
\label{subsubsec:dative}
\index{cases!dative|(}

The most typical use of the dative is for the recipient NP in a ditransitive
clause; as such, it may be a recipient proper or the entity to whose benefit
the action is carried out. The dative can furthermore be used to mark movement
toward a place. The case suffix for datives is \rayr{/ymF}{-yam} for both
animate and inanimate entities. Names and verbal topic agreement are marked
equally by \rayr{ymF}{yam}. Verbs do not exhibit person agreement with dative
NPs, since experiencers are treated as agents.

\begin{figure}[h]
\pex\label{ex:datregular}
\a\begingl
	\gla Ang @ ilya {} @ Taryan koyaley \textbf{ayonyam}. //
	\glb ang= il-ya Ø= Taryan koya-ley ayon-yam //
	\glc \AgtT{}= give-\TsgM{} \Top{}= Taryan book-\PargI{} 
		man-\Dat{} //
	\glft `Taryan gives a book to the man.' //
\endgl

\a\begingl
	\gla Ang @ ilya {} @ Taryan koyaley \textbf{yam} @ \textbf{Kandan}. //
	\glb ang= il-ya Ø= Taryan koya-ley yam= Kandan //
	\glc \AgtT{}= give-\TsgM{} \Top{}= Taryan book-\PargI{} \Dat{}= Kandan //
	\glft `Taryan gives Kandan a book.' //
\endgl

\a\begingl
	\gla \textbf{Yam} @ ilya ang @ Taryan koyaley \textbf{ayon}. //
	\glb yam= il-ya ang= Taryan koya-ley ayon-Ø //
	\glc \DatT{}= give-\TsgM{} \Aarg{}= Taryan book-\PargI{} man-\Top{} //
	\glft `The man is given a book by Taryan',\\
		\textit{or:} `The man, Taryan gives him a book.' //
\endgl
\xe
\end{figure}

The three examples in (\ref{ex:datregular}) show the regular use of the dative
as the case the recipient of the theme appears in. It is also possible for
dative NPs to appear as topics---person agreement is unaffected by this,
though, since topicalization and subject marking are different processes in
Ayeri.

As mentioned above, the dative can also take on an allative meaning insofar as
it marks the target of a motion, as displayed in (\ref{ex:datloc}). As an
extension of this means, the adpositional object may as well appear in the
dative, since Ayeri cannot distinguish, for instance, `up' from `to the top of'
with just the preposition, in this case \xayr{liNF}{ling}{on top of}. With the
adpositional object in the locative case (see below), the phrase in
(\ref{ex:datlocprep}) would imply that the man were literally going to the top
of the temple, that is, possibly ending up on its roof.

\begin{figure}[h]
\pex
\a\label{ex:datloc}\begingl
	\gla Ang @ nimpye lay \textbf{māvayam} yena. //
	\glb ang= nimp-ye lay-Ø māva-yam yena //
	\glc \AgtT{}= run-\TsgF{} girl-\Top{} mother-\Dat{} \TsgF{}.\Gen{} //
	\glft `The girl runs to her mother.' //
\endgl

\a\label{ex:datlocprep}\begingl
	\gla Ang @ saraya ayon manga @ ling \textbf{natrangyam}. //
	\glb ang= sara-ya ayon-Ø manga= ling natrang-yam //
	\glc \AgtT{}= go-\TsgM{} man-\Top{} \Dir{}= top temple-\Dat{} //
	\glft `The man goes up to the temple.' //
\endgl
\xe
\end{figure}

Lastly, the dative case is also used to mark resultative NPs, that is, NPs
which express the result of an action performed on the semantic patient of a
clause. This not only includes syntactic objects, but also patient-subjects of
agentless sentences and the subjects of unaccusative verbs
\citep{perlmutter1978}, that is, verbs whose syntactic subject is not
performing the action expressed by the verb, but undergoing it. The resultative
dative NP is fronted to occur after the verb in contrast to regular recipients,
beneficiaries, or goals. A clause may thus contain two dative NPs. These,
however, are still required to be functionally unique. That is, one may not
have two recipients or two resultatives in the same clause.

\begin{figure}[h]
\ex\label{ex:resultdat}\begingl
	\gla Ang @ visya nernanjyam {} @ Niyas seygoley ganyam. //
	\glb ang= vis-ya nernan-ye-yam Ø= Niyas seygo-ley gan-yam //
	\glc \AgtT{}= cut-\TsgM{} piece-\Pl{}-\Dat{} \Top{}= Niyas apple-\PargI{}
		child-\Dat{} //
	\glft `Niyas cuts the apple into pieces for the child.' //
\endgl\xe
\end{figure}

Hence, the first dative NP in (\ref{ex:resultdat}),
\xayr{nerFnnFyeymF}{nernanjyam}{(in)to pieces}, expresses the result of cutting
the object of the clause, \xayr{sejgolej}{seygoley}{apple}. The second dative
NP, \xayr{gnFymF}{ganyam}{for the child}, expresses the (optional) beneficiary
of the action.

\index{cases!dative|)}

\subsubsection{Genitive}
\label{subsubsec:genitive}
\index{cases!genitive|(}

The genitive is used to mark possessors; attributive genitives follow the
possessee. It can also be used for ablative meanings, that is, to mark the
place from which a motion originates, in analogy to the dative's allative use.
The genitive is marked on common nouns with the suffix \rayr{/n}{-na}. If a
noun stem ends in a consonant, the marker becomes \rayr{/En}{-ena}, compare
Tables \ref{tab:anideclcons}--\ref{tab:inandeclvow} above. Names and verbal
topic agreement are marked by \rayr{n}{na}. There is no animacy distinction in
the genitive case. Examples for the genitive case markers are given in
(\ref{ex:genmarking}).

\begin{figure}[h]
\pex\label{ex:genmarking}
\a\begingl
	\gla Pakur ledanang \textbf{netuna} nā. //
	\glb pakur ledan-ang netu-na nā //
	\glc sick friend-\Aarg{} brother-\Gen{} \Fsg{}.\Gen{} //
	\glft `My brother's friend is sick.' //
\endgl

% \a\begingl
% 	\gla Kopo dilengyereng \textbf{ajānena}. //
% 	\glb kopo dileng-ye-reng ajān-ena //
% 	\glc difficult rule-\Pl{}-\AargI{} game-\Gen{} //
% 	\glft `The rules of the game are difficult.' //
% \endgl

\a\begingl
	\gla Ang nakasyo tamo ibangya \textbf{na} @ \textbf{Niyas}. //
	\glb ang nakas-yo tamo-Ø ibang-ya na= Niyas //
	\glc \AgtT{} grow-\TsgN{} wheat-\Top{} field-\Loc{} \Gen{}= Niyas //
	\glft `There is wheat growing on Niyas's field.' //
\endgl

\a\begingl
	\gla \textbf{Na} @ nakasyo tamoang ibangya {} @ \textbf{Niyas}. //
	\glb na= nakas-yo tamo-ang ibang-ya Ø= Niyas //
	\glc \GenT{}= grow-\TsgN{} wheat-\Aarg{} field-\Loc{} \Top{}= Niyas //
	\glft `Regarding Niyas, there is wheat growing on his field.' //
\endgl
\xe
\end{figure}

Futhermore, Ayeri does not make a distinction between alienable and inalienable
possession at least in the formal language, so that typically inalienable
things such as body parts, relatives and family members, or personal items and
tools are all treated as described in (\ref{ex:genmarking}). Consider
(\ref{ex:inalposs}) for an illustration of various inalienable things. However,
inalienably possessed NPs may still appear without any possessor marked in less
formal language. Besides the mentioned items, this also extends to \xayr{rNF}
{rang}{home}.

\begin{figure}
\ex\label{ex:inalposs}
\begingl
	\gla Ang @ puntaye māva \textbf{nā} mitrangas \textbf{yena} sembari 
		\textbf{yena}. //
	\glb ang= punta-ye māva-Ø nā mitrang-as yena semba-ri 
		yena //
	\glc \AgtT{}= brush-\TsgF{} mother-\Top{} \Fsg{}.\Gen{} hair-\Parg{} 
		\TsgF{}.\Gen{} comb-\Ins{} \TsgF{}.\Gen{} //
	\glft `My mother is brushing her hair with her comb.' //
\endgl\xe
\end{figure}

The above examples show the regular use of the genitive as a marker of
possession. Apart from possession, however, the genitive can also be used to
mark origin, that is, it has a secondary function as an ablative. This is shown
in (\ref{ex:genabl}).

\begin{figure}[h]
\ex\label{ex:genabl}%
\begingl
	\gla Ang @ sahaya {} @ Vetayan \textbf{rimanena}. //
	\glb ang= saha-ya Ø= Vetayan riman-ena //
	\glc \AgtT{}= come-\TsgM{} \Top{}= Vetayan city-\Gen{} //
	\glft `Vetayan comes from the city.' //
\endgl
\xe
\end{figure}

\index{cases!genitive|)}

\subsubsection{Locative}
\index{cases!locative|(}

The locative marks basic locations, often the default that is associated with a
verb. It is also the case in which adpositional objects normally appear,
besides the special cases using the dative and the genitive mentioned above.
Common nouns are marked by \rayr{/y}{-ya};\footnote{Older texts still exhibit
an allomorph \rayr{/E\_a}{-ea}, used especially in combination with the plural
suffix \rayr{/ye}{ye}, giving \rayr{/yee\_a}{-yēa}. The modern language uses
\rayr{/yey}{-jya}.} names and verbal topic agreement use the marker 
\rayr{y}{ya}. There is no difference made between animate and inanimate 
referents in the locative.

\begin{figure}
\pex\label{ex:locplain}
\a\label{ex:locnedra}\begingl
	\gla Ang @ nedraya paray \textbf{hinya}. //
	\glb ang= nedra-ya paray-Ø hin-ya //
	\glc \AgtT{}= sit-\TsgM{} cat-\Top{} box-\Loc{} //
	\glft `The cat sits in the box.' //
\endgl

\a\label{ex:locnara}\begingl
	\gla Ang @ naraya {} @ Ajān \textbf{ya} @ \textbf{Kaman}. //
	\glb ang= nara-ya Ø= Ajān ya= Kaman //
	\glc \AgtT{}= speak-\TsgM{} \Top{}= Ajān \Loc{}= Kaman //
	\glft `Ajān speaks to Kaman.' //
\endgl

\a\label{ex:locmit}\begingl
	\gla \textbf{Ya} @ mica ang @ Kaman {} @ \textbf{Visamhinang}. //
	\glb ya= mit-ya ang= Kaman Ø= Visamhinang //
	\glc \LocT{}= live-\TsgM{} \Aarg{}= Kaman \Top{}= Visamhinang //
	\glft `Kaman lives in Visamhinang',\\
		\textit{or:} `Visamhinang is where Kaman lives.' //
\endgl
\xe
\end{figure}

The example sentences in (\ref{ex:locplain}) show locative NPs that are not
further specified by adpositions so that the correct interpretation may be
dependent on context and the experience of the addressee. Example
(\ref{ex:locnedra}) is an instance of this circumstance, in that experience
tells that cats like to sit inside boxes, so further specifying the position
with the preposition \xayr{koNF}{kong}{inside} would be emphasizing that the
cat is not sitting just anywhere, but really \emph{inside} the box as opposed
to on top of it, for instance. The sentence in example (\ref{ex:expladp}) has
the cat sitting on top of the box.

\begin{figure}[h]
\ex\label{ex:expladp}
\begingl
	\gla Ang @ nedraya paray ling hinya. //
	\glb ang= nedra-ya paray-Ø ling hin-ya //
	\glc \AgtT{}= sit-\TsgM{} cat-\Top{} on.top box-\Loc{} //
	\glft `The cat sits on the box.' //
\endgl\xe
\end{figure}

Ayeri also has a number of postpositions. This does not change the fact that
the adpositional object is marked for locative case, however, as we see in
(\ref{ex:locpostpos}), where the adpositional object,
\xayr{tenYnF}{tenyan}{death} is marked for locative case governed by the
postposition \xayr{pesnF}{pesan}{until}.

\begin{figure}[h]
\ex\label{ex:locpostpos}%
\begingl
	\gla Ang @ mican edaya \textbf{tenyanya} tan pesan. //
	\glb ang= mit-yan edaya tenyan-ya tan pesan //
	\glc \AgtT{}= live-\TplM{} here death-\Loc{} \TplM{}.\Gen{} until //
	\glft `They lived here until their death.' //
\endgl\xe
\end{figure}

\index{cases!locative|)}

\subsubsection{Causative}
\label{subsubsec:causative}
\index{cases!causative|(}

The causative marks the cause or causer of an action, the instigator or the
reason on behalf of which an agent is acting. It is thus similar to the agent
case, though it does not replace it in Ayeri; verbs do not exhibit person
agreement with causers even though their action logically supersedes or
precedes that of the agent in the embedded event. \citet{dixon2000} writes that
a \textcquote[30]{dixon2000}{causer refers to someone or something (which can
be an event or state) that initiates or controls the activity. This is the
defining property of the syntactic--semantic function A (transitive subject)}.
According to \citet[176]{comrie1989}, the causee---the agent of the event
controlled by the causer---normally takes the highest place in the hierarchy of
syntactic constituents that is not already filled, in this case, by the causer.
This observation, however, is complicated by Ayeri's more or less
semantics-based case marking as well as topicalization. In the following, I
will give examples of nominal marking for cause as before; a discussion of the
morphosyntax of Ayeri's morphological causative constructions will be deferred
to the section on valency-increasing operations, compare
\autoref{subsubsec:valincr}.

Causers or causes are marked by \rayr{/Is}{-isa} for common nouns; names and
verbal topic agreement use the marker \rayr{saa}{sā}. As stated above, verbs do
not agree with causers even though they have agent-like semantics. There is no
animacy distinction in the marking of causers. Examples of the case marker in
its various positions are provided by (\ref{ex:caumarking}).

\begin{figure}[h]
\pex\label{ex:caumarking}
\a\begingl
	\gla Ang @ rua @ sarāyn \textbf{seyaranisa}. //
	\glb ang= rua= sara=ayn.Ø seyaran-isa //
	\glc \AgtT{}= must= leave=\Fpl{}.\Top{} rain-\Caus{} //
	\glft `We had to leave due to the rain.' //
\endgl

\a\begingl
	\gla Ang @ yomāy edaya \textbf{sā} @ \textbf{Apican}. //
	\glb ang= yoma=ay.Ø edaya sā= Apican //
	\glc \AgtT{}= be=\Fsg{}.\Top{} here \Caus{}= Apican //
	\glft `I am here because of Apican.' //
\endgl

\a\label{ex:caustop}\begingl
	\gla \textbf{Sā} @ nimpvāng hakasley \textbf{yan}. //
	\glb sā= nimp=vāng hakas-ley yan.Ø //
	\glc \CauT{}= run=\Ssg{}.\Aarg{} mile-\PargI{} \TplM{}.\Top{} //
	\glft `You run a mile because of them',\\
		\textit{or:} `They make you run a mile.' //
\endgl
\xe
\end{figure}

Regarding the typological oddities mentioned above, example (\ref{ex:caustop}) 
shows what happens in Ayeri with regards to the marking of causers. 
Essentially, the causer topic was grammaticalized to express a causative 
relationship.

\index{cases!causative|)}

\subsubsection{Instrumental}
\label{subsubsec:instrumental}
\index{cases!instrumental|(}

The instrumental marks the means by which an action is carried out by an agent.
This can be a tool as well as an animate being by whose help the action is
brought about. The instrumental, thus, marks secondary agents in effect. Verbs,
however, never show person agreement with instrumental NPs. Common nouns are
marked by \rayr{/ri}{-ri} when ending in a vowel and by \rayr{/Eri}{-eri}
when ending in a consonant; names and verbal topic agreement are marked by
\rayr{ri}{ri}. With nouns ending in \fw{-e}, as well as the plural marker
\rayr{/ye}{-ye}, there is variation regarding whether \rayr{/ri}{-ri} or 
\rayr{/Eri}{-eri} is used, so that both \rayr{/yeri}{-yeri} and \rayr{/yeeri}
{-yēri} may be found as plural forms. In passive-like constructions, it is not
grammatical to reintroduce the agent as an instrumental; the agent simply
remains in the clause in this case, though as a non-topic constituent. Examples
for the case markers are given in (\ref{ex:insmarking}).

\begin{figure}
\pex\label{ex:insmarking}
% \a\begingl
% 	\gla Ang @ visye {} @ Pila seygoley \textbf{tihangeri} yena. //
% 	\glb ang= vis-ye Ø= Pila seygo-ley tihang-eri yena. //
% 	\glc \AgtT{}= cut-\TsgF{} \Top{}= Pila apple-\PargI{} 
% 		knife-\Ins{} \TsgF{}.\Gen{} //
% 	\glft `Pila cuts an apple with her knife.' //
% \endgl
%
\a\begingl
	\gla Ang @ lihoyya-ma badan \textbf{nihanyeri} 
		\textbf{\textup{(}nihanyēri\textup{)}}. //
	\glb ang= liha-oy-ya=ma badan-Ø nihan-ye-ri (nihan-ye-eri) //
	\glc \AgtT{}= earn-\Neg{}-\TsgM{}=enough father-\Top{} 
		nihan-\Pl{}-\Ins{} (nihan-\Pl{}-\Ins) //
	\glft `Father did not earn enough with his fruits.' //
\endgl

\a\begingl
	\gla Ang @ lingya {} @ Mindan mehiras \textbf{ri} @ \textbf{Kadijān}. //
	\glb ang= ling-ya Ø= Mindan mehir-as ri= Kadijān. //
	\glc \AgtT{}= climb.up-\TsgM{} \Top{}= Mindan tree-\Parg{} 
		\Ins{}= Kadijān //
	\glft `Mindan climbs a tree with Kadijān's help.' //
\endgl

\a\begingl
	\gla \textbf{Ri} @ tavya gino ang @ Kan \textbf{nimpur}. //
	\glb ri= tav-ya gino ang= Kan nimpur-Ø //
	\glc \InsT{}= become-\TsgM{} drunk \Aarg{}= Kan wine-\Top{} //
	\glft `The wine, Kan becomes drunk on it.' //
\endgl
\xe
\end{figure}

The instrumental may also be used for cases where the instrumental NP acts as a
nominal adjunct describing an attribute of its antecedent head, as in
(\ref{ex:nounadjc}).

\begin{figure}[h]
\ex\label{ex:nounadjc}
\begingl
	\gla Ang @ pegayo sinya kasuley \textbf{bariri} nā? //
	\glb ang= pega-yo sinya-Ø kasu-ley bari-ri nā //
	\glc \AgtT{}= steal-\TsgN{} who-\Top{} basket-\PargI{} 
		meat-\Ins{} \Fsg{}.\Gen{} //
	\glft `Who stole my basket of meat?' //
\endgl\xe
\end{figure}

Here, \xayr{bri}{bari}{meat} is marked as an instrumental since it serves as an
attribute of \xayr{ksu}{kasu}{basket}. The instrumental NP describes what its
antecedent contains or entails more specifically: it is a basket \fw{with} meat
in it. Note, however, that this use of the instrumental is different from
expressing accompaniment. Thus, it is not possible to use the sentence in
(\ref{ex:wrongcomit}) to express `Ajān comes (together) with Pila'.

\ex\label{ex:wrongcomit}\ljudge* \begingl
	\gla Ang @ sahaya {} @ Ajān \textbf{ri} @ \textbf{Pila}. //
	\glb ang= saha-ya Ø= Ajān ri= Pila //
	\glc \AgtT{}= come-\TsgM{} \Top{}= Ajān \Ins{}= Pila //
\endgl\xe

The sentence in (\ref{ex:wrongcomit}) would instead imply that Pila helps Ajān
to come, for example, because he has a sprained ankle and thus needs support to
get around. To express accompaniment, instead, the preposition
\xayr{kjvo}{kayvo}{with, along, beside} has to be used; the prepositional
object appears in the locative case, as usual, then, compare
(\ref{ex:comitwith}).

\begin{figure}[h]
\ex\label{ex:comitwith}%
\begingl
	\gla Ang @ sahaya {} @ Ajān \textbf{kayvo} \textbf{ya} @ \textbf{Pila}. //
	\glb ang= saha-ya Ø= Ajān kayvo ya= Pila //
	\glc \AgtT{}= come-\TsgM{} \Top{}= Ajān with \Loc{}= Pila //
	\glft `Ajān comes (together) with Pila.' //
\endgl\xe
\end{figure}

Theoretically, it should be possible as well to use the instrumental together
with prepositions for some kind of prolative meaning. The adposition would
indicate the place \emph{by way of} a motion is happening, as in
(\ref{ex:viains}).

\begin{figure}[h]
\ex\label{ex:viains}
\begingl
	\gla Ang @ pukay manga @ luga \textbf{lahaneri}. //
	\glb ang= puk=ay.Ø manga= luga lahan-eri //
	\glc \AgtT{}= jump=\Fsg{}.\Top{} \Dir{}= top fence-\Ins{} //
	\glft `I jump over the fence.' //
\endgl\xe
\end{figure}

This use of the instrumental is unattested in previous translations into Ayeri,
however, but could be considered a stylistic alternative---in the case of the
example above, to the construction with the word for `over',
\rayr{EjrrY}{eyrarya} in (\ref{ex:vialoc}).

\begin{figure}[h]
\ex\label{ex:vialoc}
\begingl
	\gla Ang @ pukay manga @ eyrarya lahanya. //
	\glb ang= puk=ay.Ø manga= eyrarya lahan-ya //
	\glc \AgtT{}= jump=\Fsg{}.\Top{} \Dir{}= over fence-\Loc{} //
	\glft `I jump over the fence.' //
\endgl\xe
\end{figure}

A more literal translation of \rayr{mN lug lhneri}{manga luga lahaneri} is `by
way of the top of the fence', though without the verbosity of the English
translation, since both ways to express the circumstance are about equally long
in Ayeri.

\index{cases!instrumental|)}

\subsubsection{Case-unmarked nouns}
\label{subsec:uncased}

Case morphology is applied to nouns in Ayeri basically whenever nouns serve as
complements or adjuncts, though there are a number of exceptions to this rule,
as we will see below. For one, the case-unmarked form is the citation form, not
the one declined for agent. As a first exception, the unmarked form can be
found when addressing people---one might speak of an unmarked vocative, as
illustrated in (\ref{ex:vocative}).

\begin{figure}[h]
\pex\label{ex:vocative}
\a\label{ex:vocnoun}\begingl
	\gla Raypu, \textbf{petāya}! //
	\glb raypa-u petāya //
	\glc stop-\Imp{} idiot //
	\glft `Stop it, you idiot!' //
\endgl

\a\label{ex:vocname}\begingl
	\gla Sahu edaya, \textbf{Diras}! //
	\glb saha-u edaya Diras //
	\glc come-\Imp{} here Diras //
	\glft `Come here, Diras!' //
\endgl
\xe
\end{figure}

Imperative forms have underlying second-person agents, so both the `idiot' in
(\ref{ex:vocnoun}) and Diras in (\ref{ex:vocname}) would be the implied agents
of their sentences, yet neither the noun nor the name are marked by the agent
markers \rayr{/ANF}{-ang} and \rayr{ANF}{ang}, respectively, since the
addressees occur as appositions. Another case where nouns are not necessarily
marked for case is attested in translations for the prefix
\xayr{ku/}{ku-}{like, as though} when the phrase acts as a depictive secondary
predicate, adn thus similar to an adverb (compare \autoref{subsubsec:depict},
p.~\pageref{subsubsec:depict}). This is exemplified by
(\ref{ex:depictapnp}).

\begin{figure}[h]
\pex\label{ex:depictapnp}
\a\label{ex:kuudhr}\begingl
	\gla … nay ang @ mya @ rankyon sitanyās \textbf{ku-netu}. //
	\glb … nay ang= mya= rank=yon.Ø sitanya-as ku=netu //
	\glc … and \AgtT{}= shall= treat=\TplN{}.\Top{} 
		each.other-\Parg{} like=brother //
	\glft `… and they shall treat each other like brothers.'\footnotemark%
	\tc{\citep{benung:udhr}}//
\endgl

\a\label{ex:kukafka}\begingl
	\gla … ang @ nunaya \textbf{ku-vipin} … //
	\glb … ang= nuna=ya.Ø ku=vipin … //
	\glc … \AgtT{}= fly=\TsgM{}.\Top{} like=bird … //
	\glft `… he (would) fly like a bird …'%
	\tc{\citep[14]{becker:kafka:imperial}}//
\endgl
\xe
\end{figure}

\footnotetext{The original English text this was translated from has 
\textcquote[Article 1]{udhr}{and should act towards one another in a spirit of 
brotherhood}.}

Strikingly, in example (\ref{ex:kuudhr}), \xayr{netu}{netu}{brother} in 
\xayr{ku/netu}{ku-netu}{like brothers} is not even inflected for plural;
likewise, \xayr{ku/vipinF}{ku-vipin}{like a bird} in (\ref{ex:kukafka}) is not
inflected for case. The depictive NP in (\ref{ex:kuudhr}) is also a little
unusual in that it does not occur after the verb in the position of an adverb
as depictives usually would.

Nouns may also be unmarked if they act as modifiers in a compound and the head
is marked for the NP's case and number, for instance as in
(\ref{ex:compunmkd}). Here, \xayr{mpNF}{mapang}{finger}, the modifier in the
compound, acts in the way of an adjective in that `fingernail' is not used as a
syntactic unit as far as case marking goes. Instead, the case marker appears on
the compound's head, \xayr{rlnF}{ralan}{nail}. Compounds will be described in
more detail in \autoref{subsec:compounds}.

\begin{figure}[h]
\ex\label{ex:compunmkd}
\begingl
	\gla ralanyeri mapang //
	\glb ralan-ye-ri mapang //
	\glc nail-\Pl{}-\Ins{} finger //
	\glft `with the fingernails' //
\endgl\xe
\end{figure}

Lastly, and probably most importantly, nouns appear superficially unmarked if
topicalized, since the topic marker is a null-morpheme (\fw{-Ø}) if viewed
systematically. We have already seen numerous examples of this above, but 
(\ref{ex:topzeromkd}) gives an example again explicitly.

\begin{figure}[h]
\ex\label{ex:topzeromkd}
\begingl
	\gla Saru-nama, ang nupoyya \textbf{veney} aruno vās. //
	\glb sar-u=nama ang nupa-oy-ya veney-Ø aruno vās //
	\glc go-\Imp{}=just \AgtT{} hurt-\Neg{}-\TsgM{} dog-\Top{} brown 
		\Ssg{}.\Parg{} //
	\glft `Just go, the brown dog won't hurt you.' //
\endgl\xe
\end{figure}

\index{cases|)}

\subsection{Prefixes on nouns}
\label{subsec:nounpref}
\index{prefixes!on nouns|(}

All of the nominal morphology we have so far dealt with in this section was
suffixing. As mentioned in the previous section already
(p.~\pageref{nounprefixes}), there are also a number of prefixes which can be
applied to nouns. I have just given two examples of the prefix
\xayr{ku/}{ku-}{like, as though} above, but \rayr{ku/}{ku-} applies not only
to nouns, but can be combined with other parts of speech as well. As discussed
in \autoref{subsec:clitics} (p.~\pageref{clitics_prenoun_ku}~ff.), it behaves
in the way of a special clitic in \citet{zwicky1977}'s terminology, since no
corresponding full form exists in its place. (\ref{ex:kucasemkd}) cites another
example from the Ayeri translation of Kafka's short story \enquote{Eine
kaiserliche Botschaft} to illustrate.

\begin{figure}[h]
\ex\label{ex:kucasemkd}
\label{ex:kukafka2}\begingl
	\gla … saylingyāng kovaro naynay, ku-ranyāng palung. //
	\glb … sayling=yāng kovaro naynay ku=ranya-ang palung //
	\glc … progress=\TsgM{}.\Aarg{} easy also like=nobody-\Aarg{} else //
	\glft `… he also got on easily, like nobody else.'%
	\tc{\citep[12]{becker:kafka:imperial}}//
\endgl\xe
\end{figure}

In this example, we can see \rayr{ku/}{ku-} attaching to a properly inflected
NP. The NP \xayr{rnYaaNF pluNF}{ranyāng palung}{nobody else} is case-marked for
agent since it can be understood to refer to the verb
\xayr{sjliNF/}{sayling-}{progress} in the main clause, so \xayr{rnYaaNF
pluNF}{ranyāng palung}{nobody else} can replace \xayr{/yaaNF}{-yāng}{he} in the
main clause. While this section deals mainly with prefixes on nouns, it should
be mentioned for completeness that \rayr{ku/}{ku-} may also appear as a suffix
under certain conditions. As discussed in \autoref{subsec:clitics}
(p.~\pageref{clitics_prenoun_ku}~ff.), \rayr{ku/}{ku-} moves to the end of the
noun phrase when a proper-noun is marked by a case-marker particle. Example
(\ref{ex:kuposvar}) repeats (\ref{ex:clitics_34}) from the previous chapter for
convenience.

\begin{figure}[h]
\pex\label{ex:kuposvar}
\a\begingl
	\gla Ang @ lentava sa @ Tagāti diyan-ku. //
	\glb ang= lenta=va.Ø sa= Tagāti diyan=ku //
	\glc \AgtT{}= sound=\Second{}.\Top{} \Parg{}= Tagāti worthy=like //
	\glft `You sound like Mr. Tagāti.' //
\endgl

\a\begingl
	\gla Ang @ lentava sa @ Tagāti diyan-ku nay diranas yana. //
	\glb ang= lenta=va.Ø sa= Tagāti diyan=ku nay diran-as yana //
	\glc \AgtT{}= sound=\Second{}.\Top{} \Parg{}= Tagāti worthy=like and 
		uncle-\Parg{} \TsgM{}.\Gen{} //
	\glft `You sound like Mr. Tagāti and his uncle.' //
\endgl

\a\begingl
	\gla Sa @ lentavāng ku-​Tagāti diyan. //
	\glb sa= lenta=vāng ku=​Tagāti diyan //
	\glc \PatT{}= sound=\Second{}.\Aarg{} like=​Tagāti worthy //
	\glft `Like Mr. Tagāti you sound.' //
\endgl
\xe
\end{figure}

Besides \rayr{ku/}{ku-}, there are also the demonstrative prefixes
\xayr{d/}{da-}{such}, \xayr{Ed/}{eda-}{this}, and \xayr{Ad/}{ada-}{that},
which have already been mentioned in the previous section as well (see
\autoref{subsec:clitics}, p.~\pageref{clitics_prenoun_dem}). The demonstrative
prefixes undergo crasis with nouns beginning with \fw{a-}, that is, they form
phonological words with their hosts for all means and purposes. An example of
this is given in (\ref{ex:noundemclit}), where \xayr{Ed/}{eda-}{this} merges
with \xayr{AyonF}{ayon}{man} to become \xayr{EdaayonF}{edāyon}{this man}. The
demonstrative prefixes are special clitics since no contemporary free form
exists.

\begin{figure}[h]
\pex\label{ex:noundemclit}
\a\begingl
	\gla da-nanga kāryo //
	\glb da=nanga kāryo //
	\glc such=house big //
	\glft `such a big house' //
\endgl

\a\begingl
	\gla edāyon nake //
	\glb eda=ayon nake //
	\glc this=man tall //
	\glft `this tall man' //
\endgl

\a\begingl
	\gla ada-envan alingo //
	\glb ada=envan alingo //
	\glc that=woman clever //
	\glft `that clever woman' //
\endgl
\xe
\end{figure}

Moreover, there is a proclitic \rayr{me/}{mə-} in complementary distribution
with the demonstrative prefixes. This particle adds a meaning along the lines
of `just any', `whatsoever', `some' to the noun. Note that this clitic is
distinct from the morpheme indicating an inspecific quantity,
\xayr{/ArilF}{-aril}{some}. Uncharacteristically of a clitic, but also like the
deictic clitics, \rayr{me/}{mə-} forms a long vowel if the noun it leans on
begins with an /e/. An example of this is given in (\ref{ex:menoun}).

\begin{figure}[h]
\pex\label{ex:menoun}
\a\begingl
	\gla Ang @ lampyo mə-veney kayvo kirinya. //
	\glb ang= lamp-yo mə=veney-Ø kayvo kirin-ya //
	\glc \AgtT{}= walk-\TsgN{} some=dog-\Top{} along street-\Loc{} //
	\glft `Some dog is walking along the street.' //
\endgl

\a\begingl
	\gla Ang @ noyan mēntānley pegamayayam. //
	\glb ang= no=yan mə=entān-ley pegamaya-yam //
	\glc \AgtT{}= want=\TsgM{}.\Top{} some=punishment-\PargI{} 
		thief-\Dat{} //
	\glft `They demanded some kind of punishment for the thief.' //
\endgl
\xe
\end{figure}

\index{prefixes!on nouns|)}

\subsection{Compounding}
\label{subsec:compounds}
\index{compounds|(}

With regards to the classification of compounds, \citet{bauer2001} gives some 
helpful typological guidelines. Besides the compound types recognized by 
Sanskrit grammarians---endocentric (\fw{tatpuruṣa}), coordinative 
(\fw{dvandva}), adjectival-endo\-cent\-ric (\fw{kar\-ma\-dhā\-ra\-ya}), and 
exocentric (\fw{bahuvrīhi})---he also adds synthetic compounds, which Sanskrit 
did not have \citep[697]{bauer2001}. Overall, he finds that determinative, or 
endocentric, compounds are the most common ones in the languages of the world 
\citep[697]{bauer2001}, especially if the head refers to a location or source 
of sorts \citep[702]{bauer2001}.

\citet{gaeta2008}, then, adds to \citet{bauer2001}'s research, based on a
larger sample of grammars surveyed, that compounds for the largest part
correlate with the constituent order of the language, both regarding the order
of verb and object and that of noun and genitive \citep[129--133]{gaeta2008}.
Mismatches in headedness occur, but appear to constitute the minority of cases
and may often be explained through historical changes in syntax; he discerns
for one that \textcquote[135]{gaeta2008}{morphology is not autonomous from
syntax}, and that secondly, \textcquote[135]{gaeta2008}{[s]yntax seems to be
the motor of change, which may be then reflected in compounds}. Thirdly, he
finds that lexical conservativism causes atavisms to linger on, reflecting the
syntax of earlier stages of the language \citep[138--139]{gaeta2008}.

\index{typology!of compounds}
For the purpose of gaining at least a little insight into which types of
compounds Ayeri allows---besides endocentric compounds---a small,
non-exhaustive survey was conducted based on 130 compounds from the Ayeri
dictionary \citep[Dictionary]{benung}; \autoref{tab:comptyp} shows the various
compound classes and the number of words for each. `Harmonic' and
`disharmonic', respectively, refer to the order of elements; the order is
`harmonic' if it is following the normal constituent order of the language and
`disharmonic' if it is at odds with it \citep{gaeta2008}.

\begin{table}[t]
\caption[Compounds in the Ayeri dictionary]{Compounds in the Ayeri dictionary 
\citep{benung} and their classification (n\,=\,130)}
\begin{tabu} to \linewidth {X[3.5l] X[c] X[c] X[c] X[c] X[c] X[c]}
\tableheaderfont\toprule
Type
	& \multicolumn2{c}{Harmonic}
	& \multicolumn2{c}{Disharmonic}
	& \multicolumn2{c}{Total}
	\\
\toprule

Endocentric (N\,+\,N)
	& 67
	& 51.54\pct
	& 2
	& 1.54\pct
	& 69
	& 53.08\pct
	\\
	
Endocentric (N\,+\,Adj)
	& 18
	& 13.85\pct
	& 4
	& 3.08\pct
	& 22
	& 16.92\pct
	\\

Synthetic (V\,+\,N)
	& 16
	& 12.31\pct
	& 4
	& 3.08\pct
	& 20
	& 15.38\pct
	\\

Coordinative (N\,+\,N)
	& 9
	& 6.92\pct
	& \multicolumn2{c}{---}
	& 9
	& 6.92\pct
	\\
	
Exocentric (N\,+\,N)
	& 1
	& 0.77\pct
	& 3
	& 2.31\pct
	& 4
	& 3.08\pct
	\\
	
\midrule

Unclear
	& 6
	& 4.62\pct
	& \multicolumn2{c}{---}
	& 6
	& 4.62\pct
	\\
	
\midrule

Total
	& 117
	& 90.00\pct
	& 13
	& 10.00\pct
	& 130
	& 100\pct
	\\
	
\bottomrule
\end{tabu}
\label{tab:comptyp}
\end{table}

Unsurprisingly, the largest number of compound nouns in the sample were
endocentric compounds of the regular kind, which means that, just as genitive
attributes follow nouns, noun compounds are headed left. Especially compounds
with adjectives are interesting insofar as this is also the normal order for
free adjectives, so to illustrate, some tests will be necessary to show that
these adjectives form a unit with the head noun and are unable to undergo
comparison, for instance. Synthetic compounds exist in Ayeri and produce nouns.
These are compounds in which \textcquote[701]{bauer2001}{the modifying element
in the compound is (usually) interpreted as an argument of the verb from which
the head is derived}. There are also a number of coordinative compounds. This
group, however, is lexicalized and not productive. Exocentric compounds
constitute the minority of the sample. In the following, I will give examples
for each type. It needs to be noted as well that unlike Germanic languages,
Ayeri does not allow compounds of arbitrary length to be strung together, like
in the ridiculous but no less real example from (former) German legislation in
(\ref{ex:REUeAUeG}) \parencite[see, for instance,][]{sz:rindfleisch}.

\begin{figure}[h]
\ex\label{ex:REUeAUeG}%
German:\medskip \\
\begingl%
	\gla %
Rind\-fleisch\-­eti\-ket\-tie\-rungs\-­über\-wa\-chungs\-­auf\-gaben\-über\-tra%
\-gungs\-gesetz//
	\glb rind-fleisch-etikettierung-s-überwachung-s­-aufgabe-n%
		-übertragung-s-gesetz //
	\glc cow-meat-labeling-\Lnk{}-supervision-\Lnk{}-duty-\Lnk{}%
		-delegation-\Lnk{}-law//
	\glft `law on the delegation of duties in the supervision of beef 
		labeling' //
\endgl\xe
\end{figure}

In stark contrast, Ayeri allows only two elements in compounds. Furthermore,
this section on compounds is located within the section on nouns because Ayeri
almost only possesses compounds involving nouns, and the majority of these also
results in a noun.

\subsubsection{Endocentric compounds}
\label{subsubsec:endocomp}
\index{compounds!endocentric|(}

To start with the largest group, endocentric/\fw{tatpuruṣa} compounds, the bulk
of these compounds combines two nouns, one of which is the head which is
modified by a dependent noun. Since Ayeri exhibits a rather strict head-initial
word order, it comes as no surprise, following \citet{gaeta2008}, that most of
these compounds follow this order strictly: the second noun modifies the first,
which is opposite of how English, for instance, typically operates. Examples
from Ayeri are given in (\ref{ex:endonoun}).

\begin{figure}[h]
\ex\label{ex:endonoun}\labels
	\begin{tabular}[t]{@{\tl\quad} l @{\enspace←\enspace} l @{\smallskip}}
	\xayr{\larger betjniMpurF}{betaynimpur}{grape}
		& \xayr{\larger betj}{betay}{berry}
		+ \xayr{\larger niMpurF}{nimpur}{wine}
		\\
	\xayr{\larger krirynF}{karirayan}{vertigo}
		& \xayr{\larger krF}{kar}{fear}
		+ \xayr{\larger IrynF}{irayan}{height}\footnotemark
		\\
	\xayr{\larger pikunMdiNF}{pikunanding}{mustache}
		& \xayr{\larger piku}{piku}{beard}
		+ \xayr{\larger nMdiNF}{nanding}{lips}
		\\
	\xayr{\larger tpjperinF}{tapayperin}{sunblind}
		& \xayr{\larger tpj}{tapay}{screen}
		+ \xayr{\larger perinF}{perin}{sun}
		\\
	\end{tabular}
\xe
\end{figure}

\footnotetext{\rayr{IrynF}{irayan}, however, is a transparent nominalization 
of \xayr{Irj}{iray}{high}.}

The example words in (\ref{ex:endonoun}) show that the relationships between 
the modifier and the head are various: a grape is a berry \emph{used} to 
make wine from \parencite[compare][702]{bauer2001}; vertigo is the fear 
\emph{of} height; a mustache is a beard \fw{located} over the lips 
\parencite[702]{bauer2001}; and a sunblind is a screen \fw{against} the 
sun.
% \footnote{Further examples include:
% \xayr{AvnMdirunF}{avanandirun}{square root}, lit. `base-square'; 
% \xayr{bidmihnye}{bidamihanaye}{xylophone}, lit. `block-wood-\Pl{}';
% \xayr{bgmFtupoj}{bagamtupoy}{dragon}, lit. `lizard-fire'; 
% \xayr{binMpdNF}{binampadang}{memory}, lit. `picture-mind'; 
% \xayr{burNu\_in}{buranguina}{elephant}, lit. `animal-nose'; 
% \xayr{dgmiMdoj}{dagamindoy}{menu}, lit. `choose-card'; 
% \xayr{dlMpsiNF}{dalampasing}{giraffe}, lit. `cow-neck'; 
% \xayr{drMdevo}{darandevo}{skull}, lit. `bone-head'; 
% \xayr{deveMthaanF}{deventahān}{alphabet}, lit. `system-writing'; 
% \xayr{glimehirF}{galimehir}{resin, tar}, lit. `juice-tree'; 
% \xayr{koybhisF}{koyabahis}{diary}, lit. `book-day'; 
% \xayr{ltuMkem}{latunkema}{tiger}, lit. `lion-stripe'; 
% \xayr{lonupt}{lonupata}{poultice}, lit. `bandage-mash'; 
% \xayr{mliMkronF}{malinkaron}{coast, seashore}, lit. `shore-sea'; 
% \xayr{mehirFgtNF}{mehirgatang}{ovaries}, lit. `tree-womb'; 
% \xayr{mehisiNj}{mehisingay}{conifer}, lit. `tree-needle'; 
% \xayr{mikYnFsitemF}{micansitem}{electric}, lit. `power-lightning'; 
% \xayr{mirMthnF}{mirantahan}{typeface}, lit. `kind-writing'; 
% \xayr{mirMthaanF}{mirantahān}{spelling}, lit. `way-writing'; 
% \xayr{mitFrmtau}{mitramatau}{pubic hair}, lit. `hair-tangle'; 
% \xayr{mitFrnvsNF}{mitranavasang}{axillary hair}, lit. `hair-sweat'; 
% \xayr{nrMbesuhej}{narambesuhey}{dictionary}, lit. `word-list'; 
% \xayr{niMpurivnF}{nimpurivan}{vinyard}, lit. `wine-mountain'; 
% \xayr{ptyelNF}{patayelang}{concrete}, lit. `mash-stone'; 
% \xayr{pikulkj}{pikulakay}{goatee}, lit. `beard-chin'; 
% \xayr{prihiNumo}{prihingumo}{desk}, lit. `table-work'; 
% \xayr{rgMterFpeNF}{raganterpeng}{diameter}, lit. `line-middle'; 
% \xayr{rlmpNF}{ralamapang}{fingernail}, lit. `nail-finger'; 
% \xayr{ridspj}{ridasapay}{glove}, lit. `sock-hand'; 
% \xayr{sNumospoj}{sangumosapoy}{ticket office}, lit. `office-ticket'; 
% \xayr{svtkNF}{savatakang}{tank}, lit. `cart-armor'; 
% \xayr{sayMprl}{sayamparal}{urine hole}, lit. `hole-penis'; 
% \xayr{syNu\_in}{sayanguina}{nostril}, lit. `hole-nose'; 
% \xayr{syniv}{sayaniva}{eye socket}, lit. `hole-eye'; 
% \xayr{selNblN}{selangbalang}{search engine}, lit. `machine-search'; 
% \xayr{selMkurnF}{selangkuran}{computer}, lit. `machine-counting'; 
% \xayr{sepFrkronF}{seprakaron}{ditch}, lit. `cleft-water'; 
% \xayr{similitnF}{similitan}{borderland}, lit. `land-margin-\Nmlz{}'; 
% \xayr{similitj}{similitay}{republic}, lit. `land-democracy'; 
% \xayr{sirjyil}{sirayyila}{knee}, lit. `joint-foot'; 
% \xayr{sirjtinu}{siraytinu}{elbow}, lit. `joint-arm'; 
% \xayr{sirukronF}{sirukaron}{starfish}, lit. `star-water'; 
% \xayr{sirusitFrmF}{sirusitram}{comet}, lit. `star-tail'; 
% \xayr{sirutj}{sirutay}{night}, lit. `star-time'; 
% \xayr{sitNlugaanF}{sitanglugān}{incest}, lit. `self-entry'; 
% \xayr{trFtrihimF}{tartarihim}{tobacco}, lit. `pipe-weed'; 
% \xayr{tepilFpihaanF}{tepilpihān}{fester}, lit. `sore-pus'; 
% \xayr{tFreMdpNisF}{trendapangis}{bank}, lit. `hall-money'; 
% \xayr{tuptinu}{tupatinu}{fathom}, lit. `length-arm'; 
% \xayr{veb\_osnF}{vebaosan}{slug}, lit. `snail-slime'; 
% \xayr{veMkubesonF}{venkubeson}{navy}, lit. `army-ship'; 
% \xayr{vinimyonF}{vinimayon}{monkey}, lit. `forest-man'; 
% \xayr{yno\_avnF}{yanoavan}{area, region}, lit. `place-ground'; 
% \xayr{yelNFssaanF}{yelangsasān}{cobblestone}, lit. `stone-way'; 
% \xayr{yenukrFdNF}{yenukardang}{classmates}, lit. `group-school'; 
% \xayr{yutnjkonF}{yutanaykon}{foreskin}, lit. `skin-cover'.
% }
\citet{bauer2001} mentions that \textquote{there may be special morphophonemic
processes which apply between the elements of compounds}, such as
\textcquote[695]{bauer2001}{phonological merger[s] between the elements of the
compound}. This also occasionally happens in Ayeri, as the example words in
(\ref{ex:endonounmod}) show.

\begin{figure}[h]
\pex\label{ex:endonounmod}
	\a \xayr{\larger AvrrnF}{avararan}{wetland} \\
		← \xayr{\larger AvnF}{avan}{ground}
		+ \xayr{\larger rro}{raro}{wet}
		+ \rayr{\larger /AnF}{-an} (\Nmlz{})
	\a \xayr{\larger mehimitFrNF}{mehimitrang}{fiber tree} \\
		← \xayr{\larger mehirF}{mehir}{tree}
		+ \xayr{\larger mitFrNF}{mitrang}{hair, fiber}
	\a \xayr{\larger niNMpinmF}{ningampinam}{bedtime story} \\
		← \xayr{\larger niNnF}{ningan}{story}
		+ \xayr{\larger pinmF}{pinam}{bed}
	\a \xayr{\larger pdilmikYnF}{padilamican}{gravitational force} \\
		← \xayr{\larger pdilnF}{padilan}{attraction}
		+ \xayr{\larger mikYnF}{mican}{force, power}
\xe
\end{figure}

There is a modicum of alteration happening in all of the heads of the example 
words in (\ref{ex:endonounmod}), mostly nasals assimilating to the stop or 
nasal which the modifier begins with (/n/~+~/p/~→~/mp/, /n/~+~/m/~→~/m/), 
though \rayr{AvrrnF}{avararan} and \rayr{mehimitFrNF}{mehimitrang} even delete 
whole coda segments.
% Stuff may even be mashed together completely, but examples??
\citet[703]{bauer2001} notes that very commonly, genitive and plural markers 
may form linking elements, though he also gives examples of languages which
allow other case markers on the modifying element in languages with head-final
order; individual languages may allow even more case inflection. However, this
appears not to happen in Ayeri. The only element that comes up time and again
in between the two halves of compounds is the nominalizer \rayr{/AnF}{-an},
which signifies that the head is being formed by a nominalized root, such as in
\rayr{pdilmikYnF}{padilamican}, where \xayr{pdilnF}{padilan}{attraction} is a 
nominalization of \xayr{pdilF/}{padil-}{attract}, or in 
\rayr{niNMpinmF}{ningampinam}, where \xayr{niNnF}{ningan}{story} is derived 
from the verb \xayr{niNF/}{ning-}{tell}. However, since Ayeri is head-initial
and possessive phrases are dependent marking, genitive or other case marking
would be expected on the second element, not the first. Case marking on a
compound, however, does not inflect just the modifier, but the whole NP, as 
(\ref{ex:compinfl}) shows.

\begin{figure}[h]
\ex\label{ex:compinfl}\begingl
	\gla Ang @ ningya sipikanena koyabahisena. //
	\glb ang= ning-ya sipik-an-ena koyabahis-ena //
	\glc \AgtT{}= talk-\TsgM{}.\Top{} keep-\Nmlz{}-\Gen{} book.day-\Gen{} //
	\glft `He talks about keeping a journal.' //
\endgl\xe
\end{figure}

\rayr{koybhisen}{koyabahisena} in this example is not to be interpreted as 
`book of day(s)' but as `of a day-book'. Inflection between the parts of a
compound can happen nonetheless, though. In compounds which are formed \fw{ad
hoc} or which are otherwise transparent in their composition (`loose'
compounds\label{loosecomp}), inflection often is deferred to the head noun
instead of the edge of the compound as a whole; the modifier is treated as an
adjunct in this case, and stays uninflected. An example of this is given in 
(\ref{ex:nouncompdiv}).

\begin{figure}[h]
\ex\label{ex:nouncompdiv}\begingl
	\gla Sa @ trayeng tipin ralanyeri mapang yena. //
	\glb sa= tra=yeng tipin-Ø ralan-ye-ri mapang yena //
	\glc \PatT{}= scratch=\TsgF{}.\Aarg{} itch-\Top{} nail-\Pl{}-\Ins{} 
		finger \TsgF{}.\Gen{} //
	\glft `The itch, she scratches it with her fingernails.' //
\endgl\xe
\end{figure}

Besides noun modifiers, there are also compounds where the modifier is an 
adjective. In classical Sanskrit terminology, this type is called 
\fw{karmadhāraya} \citep[698--699]{bauer2001}.\footnote{\citet{bauer2001} 
also mentions that appositional compounds like \fw{maid-servant}, \fw{woman
doctor} and \fw{fighter-bomber} are counted in this category
\citep[699]{bauer2001}. Ayeri, however, does not possess such formations in
particular.} Examples in Ayeri include those listed in
(\ref{ex:ayrnounadjcomp}). In all of these cases, the adjective forms a unified
lexeme with the head noun, hence it is not comparable, as the examples in
(\ref{ex:nounadjcompsupl}) show.
% \footnote{Further examples include: 
% \xayr{bhisino}{bahisino}{holiday, day off}, lit. `day-free'; 
% \xayr{dikuMtrinF}{dikuntaring}{bureaucracy}, lit. `passion-administrative'; 
% \xayr{leMtMkusNF}{lentankusang}{diphthong}, lit. `sound-double'; 
% \xayr{nNbnY}{nangabanya}{hospital}, lit. `house-sick'; 
% \xayr{naraaMtiynF}{narāntiyan}{conlang}, lit. `language-created-\Nmlz{}'; 
% \xayr{rohMpraanF}{rohamparān}{snack}, lit. `bite-quick-\Nmlz{}'; 
% \xayr{rohMkivo}{rohankivo}{snack}, lit. `bite-small'; 
% \xayr{sNumirj}{sangumiray}{ministry, authority}, lit. `office-high'; 
% \xayr{tbMpehu}{tabampehu}{lower jaw}, lit. `jaw-loose'
% \xayr{tabnikp}{tabanikapa}{upper jaw}, lit. `jaw-attached'.
% }

\begin{figure}[h]
\ex\labels\label{ex:ayrnounadjcomp}
	\begin{tabular}[t]{@{\tl\quad} l @{\enspace←\enspace} l @{\smallskip}}
	\xayr{\larger krFdNirj}{kardangiray}{university}
		& \xayr{\larger krFdNF}{kardang}{school}
		+ \xayr{\larger Irj}{iray}{high} \\
		
	\xayr{\larger mrsFhri}{marashari}{witticism}
		& \xayr{\larger mrsF}{maras}{phrase}
		+ \xayr{\larger hri}{hari}{pithy} \\
		
	\xayr{\larger silFvniknF}{silvanikan}{overview}
		& \xayr{\larger silFvnF}{silvan}{view}
		+ \xayr{\larger IknF}{ikan}{whole} \\
		
	\xayr{\larger vipimkaarY}{vipimakārya}{crow}
		& \xayr{\larger vipinF}{vipin}{bird}
		+ \xayr{\larger mkaarY}{makārya}{black} \\
	\end{tabular}
\xe
\end{figure}

\begin{figure}[h]
\ex\label{ex:nounadjcompsupl}\labels
\begin{tabular}[t]{@{} l @{\quad} l @{\hspace{2em}} l}

		& \textsc{comparative}			& \textsc{superlative} \medskip \\

	\tl	& \ljudge*\fw{kardangiray-eng}	& \fw{kardangiray-vā} \\
		& kardang-iray=eng				& kardang-iray=vā \\
		& school-[high=\Comp{}]			& school-*[high=\Supl{}] \\
		& `higher-school'				& `highest-school' \medskip \\

	\tl	& \ljudge*\fw{marashari-eng}	& \fw{marashari-vā} \\
		& maras-hari=eng				& maras-hari=vā \\
		& phrase-[pithy=\Comp{}]		& phrase-*[pithy=\Supl{}] \\
		& `pithier-phrase'				& `pithiest-phrase' \\
\end{tabular}
\xe
\end{figure}

In fact, it \emph{is} possible to form \rayr{krFdNirj/vaa}{kardangiray-vā} and
\rayr{mrsFhti/vaa}{marasari-vā}, but they mean `most universities' and `most
witticisms', that is, \rayr{/vaa}{-vā} here does not mark the adjectival part
as a superlative form; the suffix modifies the noun--adjective compound as a
whole: \textit{$[$school-high$]$=most}, \textit{$[$phrase-pithy$]$=most}.
\xayr{/ENF}{-eng}{rather} as a quantifier does not combine with nouns, which is
why the first examples in (\ref{ex:nounadjcompsupl}ab) are both ungrammatical
\fw{per se}.

Since the meaning of noun--adjective compounds is often idiomatic, they also
cannot be divided as shown above in (\ref{ex:nouncompdiv}), since a
\xayr{krFdNirj}{kardangiray}{university} is not a
\xayr{krFdNF}{kardang}{school} which is \xayr{Irj}{iray}{high} in the literal
sense, but a school of the highest tier. \rayr{krFdNen Irj}{kardangena iray}
(school-\Gen{} high), then, can only be interpreted in the literal sense, `of
the high school', but not as `of the university', which thus can only be
\rayr{krFdNiryen}{kardangirayena}.

In the sample, there were also a few compounds which were categorized as
noun--noun combinations and which look as though they violate head-initial
order. All of these involve \xayr{sitNF}{sitang}{self} as a modifier, for
instance, as in (\ref{ex:sitangcomp}).

\begin{figure}
\ex\labels\label{ex:sitangcomp}
	\begin{tabular}[t]{@{\tl\quad} l @{\enspace←\enspace} l @{\smallskip}}
	\xayr{\larger sitNFleMtnF}{sitanglentan}{vowel}
		& \xayr{\larger sitNF}{sitang}{self}
		+ \xayr{\larger leMtnF}{lentan}{sound}
		\\
	\xayr{\larger sitNFpronaanF}{sitangparonān}{self-confidence}
		& \xayr{\larger sitNF}{sitang}{self}
		+ \xayr{\larger pronaanF}{paronān}{faith}
		\\
	\xayr{\larger sitNFtenYnF}{sitangtenyan}{suicide}
		& \xayr{\larger sitNF}{sitang}{self}
		+ \xayr{\larger tenYnF}{tenyan}{death}
		\\
	\end{tabular}
\xe
\end{figure}

\rayr{sitNF}{sitang} does not exist as a noun by itself in Ayeri, the word for 
`self' is its nominalization, \rayr{sitNnF}{sitangan}. Nonetheless, it looks 
as though it could have plausibly been a noun once. However, this noun 
may have been grammaticalized into a reflexive morpheme of a more 
general kind, which in turn gave rise to the form \rayr{sitNnF}{sitangan} as a 
renovation.\footnote{A little bit of language history would certainly simplify 
things here and lend them credence. Let us simply assume that 
\rayr{sitNF}{sitang} used to be a noun meaning something like `self' at a 
previous stage of Ayeri and was repurposed as a reflexive prefix. 
\citet{lehmann2015} quotes a few examples of what he calls `autophoric' nouns 
that came to be used as reflexive pronouns in their respective language: 
\textcquote[45--46]{lehmann2015}{Typical examples are Sanskrit \fw{tan} 
`body, person' and \fw{ātmán} `breath, soul', Buginese \fw{elena} `body',
Okinawan \fw{dūna} `body', !Xu \fw{l’esi} `body', Basque \fw{burua} `head',
Abkhaz \fw{a-xə̀} `the head'. In their respective languages, all these nouns
are translation equivalents of English \fw{self}}. Thus, it would not be out of
line at all to assume such a grammaticalization path for Ayeri as well.} The
reflexive \rayr{sitNF}{sitang} is used---as we have seen in the previous
chapter---as a prefix, so there are two ways to intepret these formations:
first, \rayr{sitNF}{sitang} may be the reflexive prefix here and thus the
compound follows the normal syntactic order; or second, the order of elements
is reversed and thus may reflect an earlier stage of Ayeri where
\rayr{sitNF}{sitang} was still a noun and modifiers could still appear in front
of their heads, at least optionally so \citep[133--137]{gaeta2008}.

There are a number of genuinely reversed endocentric compounds as well,
however, in which the modifier comes first and the head last. There are only a
few of these in the sample; (\ref{ex:endocomp}) lists all of them.

\begin{figure}[h]
\ex\labels\label{ex:endocomp}
	\begin{tabular}[t]{@{\tl\quad} l @{\enspace←\enspace} l @{\smallskip}}
	\xayr{\larger bript}{baripata}{ground meat}
		& \xayr{\larger bri}{bari}{meat}
		+ \xayr{\larger pt}{pata}{mash}
		\\
	\xayr{\larger kjvoleMtnF}{kayvolentan}{consonant}
		& \xayr{\larger kjvo}{kayvo}{with}
		+ \xayr{\larger leMtnF}{lentan}{sound}
		\\
	\xayr{\larger maavgneNF}{māvaganeng}{mother's siblings}
		& \xayr{\larger maav}{māva}{mother}
		+ \xayr{\larger gneNF}{ganeng}{siblings}
		\\
	\xayr{\larger mtinMdiNF}{matinanding}{labia}
		& \xayr{\larger mtiknF}{matikan}{hot}
		+ \xayr{\larger nMdiNF}{nanding}{lips}
		\\
	\xayr{\larger muyvirNF}{muyavirang}{brass}
		& \xayr{\larger muy}{muya}{false}
		+ \xayr{\larger AvirNF}{avirang}{gold}
		\\
	\xayr{\larger tonisjtNF}{tonisaytang}{self-assured}
		& \xayr{\larger tonis}{tonisa}{assured}
		+ \ques{}\,\xayr{\larger sitNnF}{sitangan}{self}
		\\
	\end{tabular}
\xe
\end{figure}

% Given the discussion of \rayr{sitNF}{sitang} above, one word among the
% examples above whose origin is not quite clear is
% \rayr{tonisjtNF}{tonisaytang}, which appears to contain a deviant form of
% either \rayr{sitNF}{sitang} or \rayr{sitNnF}{sitangan}, which is preceded by
% the adjective \xayr{tonis}{tonisa}{assured, ascertained}.

All of the previously mentioned compounds involving nominal elements formed 
nouns, though there are also a few denominal compounds in the sample. This 
process is not productive, however, and interestingly, only noun--adjective 
combinations appear in this group. These are listed in (\ref{ex:denomcomp}).

\begin{figure}[h]
\ex\labels\label{ex:denomcomp}
	\begin{tabular}[t]{@{\tl\quad} l @{\enspace←\enspace} l @{\smallskip}}
	\xayr{\larger mirMpluj}{mirampaluy}{otherwise}
		& \xayr{\larger mirnF}{miran}{way}
		+ \ques{}\,\xayr{\larger pluNF}{palung}{different}
		\\
	\xayr{\larger pdbnY}{padabanya}{insane}
		& \xayr{\larger pdNF}{padang}{mind}
		+ \xayr{\larger bny}{banaya}{sick}
		\\
	\xayr{\larger teMkris/}{tenkarisa-}{be scared to death}
		& \xayr{\larger tenF}{ten}{life}
		+ \xayr{\larger kris}{karisa}{frightened}
		\\
	\end{tabular}
\xe
\end{figure}

As for the examples in (\ref{ex:denomcomp}), \rayr{mirMpluj}{mirampaluy} is an
adverb whose modifier is probably a mangling of \rayr{pluNF}{palung}.
\rayr{pdbnY}{padabanya} is an adjective meaning `insane' rather than the
expected `insanity' (instead: \rayr{pdbnYaanF}{padabanyān}). Lastly,
\rayr{teMkris/}{tenkarisa-} acts as a verb, possibly from conversion or
reinterpretation, since the suffix \rayr{/Is}{-isa} also forms morphological
causatives of a number of verbs. Besides these irregularities, there is also at
least one noun compound which uses a postposition as an adjectival modifier,
given in (\ref{ex:nounpostposcomp}). This compound must be derived from the
phrase \xayr{silFvnFy kjvj}{silvanya kayvay}{without sight} (see-\Nmlz{}-\Loc{}
without), though here as well, the word roots are simply juxtaposed, as is the
common way to form compounds in Ayeri.

\begin{figure}[h]
\ex\label{ex:nounpostposcomp}
	\xayr{\larger silFvMkjvj}{silvankayvay}{blindness} 
	← \xayr{\larger silFvnF}{silvan}{sight}
	+ \xayr{\larger kjvj}{kayvay}{without}
\xe
\end{figure}

\index{compounds!endocentric|)}

\subsubsection{Synthetic compounds}
\index{compounds!synthetic|(}

According to \citet{bauer2001}, (semi-)synthetic compounds, or verbal(-nexus)
compounds, are compounds that have \textcquote[701]{bauer2001}{been variously
defined as being based on word-groups or syntactic constructions
\citep[2]{botha1984}, or as compounds whose head elements are derived from
verbs \citep[3607]{lieber1994}}. Examples of this type in English would include
\fw{truck-driver}, \fw{peace-keeping}, and \fw{home-made}. He mentions also 
that synthetic compounds have been mainly discussed with regards to Germanic
languages, but that according to \citet[3608]{lieber1994}, the phenomenon is
much more widespread.

Ayeri possesses compounds like this as well, and the regular case again follows
the constituent order, here that of verbs and nouns: Ayeri is a VO language,
and thus the verb as the head of the compound is usually found on the left side
with its nominal modifier following it \citep[compare][129--133]{gaeta2008},
compare (\ref{ex:verbnouncomp}).

\begin{figure}[h]
\ex\labels\label{ex:verbnouncomp}
	\begin{tabular}[t]{@{\tl\quad} l @{\enspace←\enspace} l @{\smallskip}}
	\xayr{\larger AnFlgonnF}{anlagonan}{pronunciation}
		& \xayr{\larger AnFlF/}{anl-}{bring}
		+ \xayr{\larger AgonnF}{agonan}{outside}
		\\
	\xayr{\larger npkronF}{napakaron}{acid}
		& \xayr{\larger npF/}{nap-}{burn}
		+ \xayr{\larger kronF}{karon}{water}
		\\
	\xayr{\larger npperinF}{napaperin}{sunburn}
		& \xayr{\larger npF/}{nap-}{burn}
		+ \xayr{\larger perinF}{perin}{sun}
		\\
	\xayr{\larger telFbssaanF}{telbasasān}{waysign}
		& \xayr{\larger telFb/}{telba-}{show}
		+ \xayr{\larger ssaanF}{sasān}{way}
		\\
	\end{tabular}
\xe
\end{figure}

Here as well, the relations between verb and noun are various, that is, the
nominal modifier is not simply the direct object of the verb: to pronounce
something means to bring it \emph{to} the outside; a sunburn is a burn
\emph{caused by} the sun; and a waysign \emph{shows} the way
(\rayr{ssaanF}{sasān} is the object here). Even though \rayr{kronF}{karon} may
serve as an agent (or a causer) of the burning effect of acid (similar for
\xayr{npperinF}{napaperin}{sunburn}), the verb-first order is justified here as
well, since verbs always come first in Ayeri sentences, and any other NPs,
whether actor or undergoer, are following.
% \footnote{Further examples include:
% \xayr{bimkNnF}{bimakangan}{photo}, lit. `paint-light-\Nmlz{}'; 
% \xayr{IlgonnF}{ilagonan}{edition}, lit. `give-out-\Nmlz{}'; 
% \xayr{lMtmidj}{lantamiday}{diversion}, lit. `lead-around'; 
% \xayr{nbisFmaavy}{nabismāvaya}{motherfucker}, lit. `fuck-mother-\Agtz{}'; 
% \xayr{nrkhu}{narakahu}{phone}, lit. `speak-far'; 
% \xayr{srsjliNF}{sarasayling}{progress}, lit. `go-further'; 
% \xayr{silFvkhu}{silvakahu}{TV}, lit. `see-far'; 
% \xayr{silFvmrinnF}{silvamarinan}{preview}, lit. `see-before-\Nmlz{}'; 
% \xayr{telFbgonnF}{telbagonan}{advertisement}, lit. `show-out'; 
% \xayr{vliktu}{valikatu}{masochist}, lit. `enjoys pain'.
% }

Just as with endocentric compounds, there are a number of seeming exceptions to
the verb-first order of synthetic compounds. These are just as far and few
between, however, and whether they should all be counted as noun--verb
combinations is also questionable, since they appear to all be formed with
nominalized verbs. The verbal element may thus be only indirectly verbal for
the purposes of compounding. If interpreted as noun--noun combinations, the
nominal first element would reasonably form the head again for some of the
words in (\ref{ex:compvbrev}).

\begin{figure}[h]
\pex\label{ex:compvbrev}
	\a \xayr{\larger mripuMtymF}{maripuntayam}{spread} \\
		← \xayr{\larger mrinF}{marin}{surface}
		+ \xayr{\larger puMt/}{punta-}{stroke}
		+ \rayr{\larger /ymF}{-yam} (\Dat{})
	\a \xayr{\larger ssnFlekaanF}{sasanlekān}{labyrinth} \\
		← \xayr{\larger ssaanF}{sasān}{way}
		+ \xayr{\larger lek/}{leka-}{guess}
		+ \rayr{\larger /AnF}{-an} (\Nmlz{})
	\a \xayr{\larger selNnunaanF}{selangnunān}{plane} \\
		← \xayr{\larger selNF}{selang}{machine}
		+ \xayr{\larger nun/}{nuna-}{fly}
		+ \rayr{\larger /AnF}{-an} (\Nmlz{})
	\a \xayr{\larger siMturaanF}{sinturān}{radio} \\
		← \xayr{\larger siMto}{sinto}{wave}
		+ \xayr{\larger tur/}{tura-}{send}
		+ \rayr{\larger /AnF}{-an} (\Nmlz{})
\xe
\end{figure}

\rayr{mripuMtymF}{maripuntayam} is special in that it contains the dative 
suffix \rayr{/ymF}{-yam} which is lexicalized as a part of the word: something
made or intended for spreading on a surface. A few more such verbal derivations
can be found, though not compounds, among others in those words listed in 
(\ref{ex:yamderiv}).

\begin{figure}[h]
\ex\label{ex:yamderiv}\labels
	\begin{tabular}[t]{@{\tl\quad} l @{\enspace←\enspace} l @{\smallskip}}
	\xayr{\larger gFrenYmF}{grenyam}{extremity}
		& \xayr{\larger gFren/}{gren-}{reach out}
		\\
	\xayr{\larger lugymF}{lugayam}{password}
		& \xayr{\larger lug/}{luga-}{go through} 
		\\
	\xayr{\larger shymF}{sahayam}{future}
		& \xayr{\larger sh/}{saha-}{come}
		\\
	\end{tabular}
\xe
\end{figure}

There is also \xayr{mripuMt/}{maripunta-}{spread over} as the verb corresonding
to \rayr{mripuMtymF}{maripuntayam}, though its meaning is less specific. Here
as well, however, the verbal part is last instead of first. For the other
example words (\ref{ex:compvbrev}b--d), an interpretation of the second part as
a deverbal noun is possible: a labyrinth as a way or path which requires
guessing, a plane as a machine for flight, and radio as a transmission of
waves. In the latter case, \rayr{siMturaanF}{sinturān}, however, the head is
still on the wrong side even if one interprets all of the above examples as
noun--noun compounds with a deverbal element.

\index{compounds!synthetic|)}

\subsubsection{Coordinative compounds}
\index{compounds!coordinative|(}

Coordinative compounds are a very small group among the sample drawn from the
dictionary, and not a very productive one. \citet{bauer2001} defines this class
as having \textcquote[699]{bauer2001}{two or more words in a coordinate
relationship, such that the entity denoted is the totality of the entities
denoted by each of the elements}. He cautions that they are very easily
confused with appositional (also \fw{karmadhāraya}) compounds in that both
types of compound allow inserting an \fw{and} between both elements. The
nominal coordinative compounds included in the sample are listed in
(\ref{ex:ayrdvand}).

\begin{figure}[h]
\ex\labels\label{ex:ayrdvand}
	\begin{tabular}[t]{@{\tl\quad} l @{\enspace←\enspace} l @{\smallskip}}
	\xayr{\larger baaːm}{bāmā}{mom-and-dad}
		& \xayr{\larger baa(baa)}{bā(bā)}{dad}
		+ \xayr{\larger maa(maa)}{mā(mā)}{mom}
		\\
	\xayr{\larger pFrujnpj}{pruynapay}{seasoning}
		& \xayr{\larger pruj}{pruy}{salt}
		+ \xayr{\larger npj}{napay}{pepper}
		\\
	\xayr{\larger spjyil}{sapayyila}{hands-and-feet}
		& \xayr{\larger spj}{sapay}{hand}
		+ \xayr{\larger yil}{yila}{foot}
		\\
	\xayr{\larger simileno}{simileno}{horizon}
		& \xayr{\larger similF}{simil}{country}
		+ \xayr{\larger leno}{leno}{sky}
		\\
	\xayr{\larger sitemFrugonF}{sitemrugon}{thunderstorm}
		& \xayr{\larger sitemF}{sitem}{lightning}
		+ \xayr{\larger rugonF}{rugon}{thunder}
		\\
	\xayr{\larger vekmFdekej}{vekamdekey}{dishes}
		& \xayr{\larger vekmF}{vekam}{plate}
		+ \xayr{\larger dekej}{dekey}{fork}
		\\
	\end{tabular}
\xe
\end{figure}

None of the two elements recognizably forms the head in these examples, but
both elements are typical components of the thing the compound signifies.
\citet[699]{bauer2001} mentions that coordinative adjective compounds are rare,
or at least rarely documented in the grammars he surveyed. In our sample, only
the compound in (\ref{ex:adjadjcomp}) is included. This compound forms a noun
from the combination of two adjectives, insofar it is relevant to this section
even though the component parts are not nouns.

\begin{figure}[h]
\ex\label{ex:adjadjcomp}
	\xayr{\larger mkgisu}{makagisu}{twilight}
		← \xayr{\larger mk}{maka}{light}
		+ \xayr{\larger gisu}{gisu}{dark}
\xe
\end{figure}

The sample also includes the two words in (\ref{ex:verbverbcomp}), which are,
however, neither made up from nouns, nor do they form a noun in combination.
Instead, they are technically verbs combining to form directional adverbs and
have been exceptionally included here for completeness.

\begin{figure}[h]
\ex\labels\label{ex:verbverbcomp}
	\begin{tabular}[t]{@{\tl\quad} l @{\enspace←\enspace} l @{\smallskip}}
	\xayr{\larger mNsh}{mangasaha}{towards}
		& \xayr{\larger mN/}{manga-}{move}
		+ \xayr{\larger sh/}{saha-}{come}
		\\
	\xayr{\larger mNsr}{mangasara}{away}
		& \xayr{\larger mN/}{manga-}{move}
		+ \xayr{\larger sr}{sara-}{go}
		\\
	\end{tabular}
\xe
\end{figure}

\index{compounds!coordinative|)}

\subsubsection{Exocentric compounds}
\index{compounds!exocentric|(}

In exocentric compounds, the modifier is not a hyponym of its head
\citep[700]{bauer2001}, which means that the modifier is not describing a
property that more closely determines its head. So while a \fw{dog house} is a
type of house made for dogs, the head of an \fw{egghead} is neither for eggs,
nor containing eggs, nor made of eggs; instead, it refers to an egg-shaped
skull metaphorically. And while a \fw{bluecollar} may wear a blue shirt
professionally, the referent it signifies is not a type of collar, but the
relationship is metonymical in that the blue collar is part of the guise of the
signified entity as a whole. The sample from the Ayeri dictionary contains a
few compounds of this kind as well, listed in (\ref{ex:exocentcomp}). Again,
however, it is not a very productive group.

\begin{figure}[h]
\ex\labels\label{ex:exocentcomp}
	\begin{tabular}[t]{@{\tl\quad} l @{\enspace←\enspace} l @{\smallskip}}
	\xayr{\larger AvnFyonNF}{avanyonang}{artery}
		& \xayr{\larger AvnF}{avan}{bottom, down}
		+ \xayr{\larger yonNF}{yonang}{stream}
		\\
	\xayr{\larger bjtMdevo}{baytandevo}{headache}
		& \xayr{\larger bjtNF}{baytang}{blood}
		+ \xayr{\larger devo}{devo}{head}
		\\
	\xayr{\larger linFyonNF}{linyonang}{vein}
		& \xayr{\larger liNF}{ling}{top, up}
		+ \xayr{\larger yonNF}{yonang}{steam}
		\\
	\xayr{\larger siMdjnN}{sindaynanga}{address}
		& \xayr{\larger sindj}{sinday}{number}
		+ \xayr{\larger nN}{nanga}{house}
		\\
	\end{tabular}
\xe
\end{figure}

What is striking here is that only one out of four examples shows the expected
head-initial order: \rayr{siMdjnN}{sindaynanga}. The other three examples all
have the head component on the right side, preceded by a modifier. However,
what all of these have in common, is that they are only metaphorically or
metonymically describing the thing they signify: veins and arteries are not
literally streams going up or down (they are a kind of stream flowing in
different directions, however, so these are probably on the borderline between
exocentric and endocentric); a headache is related to the head, but has not
directly to do with being made of or containing blood (the rationale behind
this being a superstition that you have too much blood in your head, which is
said to cause the pain); and a house number may be part of an address, but is
in a \fw{pars pro toto} relationship to it.

\index{compounds!exocentric|)}

\subsubsection{A few mysterious cases}

The following words from our sample were either undeterminable as to their 
composition due to parts of the word not being clear regarding one of their 
constituent parts, either because I tweaked the constituent so much as to not 
be readily recognizable anymore, or because I forgot to make an entry in the 
dictionary, or even deleted or changed it. The words in question are listed in 
(\ref{ex:mystcomp}).

\begin{figure}[h]
\ex\labels\label{ex:mystcomp}
	\begin{tabular}[t]{@{\tl\quad} l @{\enspace←\enspace} l @{\smallskip}}
	\xayr{\larger btNimnF}{batangiman}{mosquito}
		& \xayr{\larger bjtNF}{baytang}{blood}
		+ ?
		\\
	\xayr{\larger kirinlNF}{kirinalang}{avenue}
		& \xayr{\larger kirinF}{kirin}{street}
		+ ?
		\\
	\xayr{\larger niNMbkrF}{ningambakar}{telltale}
		& \xayr{\larger niNnF}{ningan}{story}
		+ ?
		\\
	\xayr{\larger rgyesuj}{ragayesuy}{grid}
		& \xayr{\larger rgnF}{ragan}{line}
		+ ?
		\\
	\xayr{\larger terjmino}{teraymino}{melancholic}
		& ?
		+ \xayr{\larger mino}{mino}{happy}
		\\
	\xayr{\larger vetjsno}{vetaysano}{fare}
		& ?
		+ \rayr{\larger ssaanF}{sasān} (earlier \rayr{\larger 
			ssno}{sasano}) `way'
		\\
	\end{tabular}
\xe
\end{figure}

For all of the components represented by a question mark, there is no 
corresponding dictionary entry. At least in \rayr{bjtNimnF}{baytangiman}, the 
*\rayr{ImnF}{*iman} part looks as though it could be a noun due to the 
\rayr{/AnF}{-an} nominalizer suffix. *\rayr{terj}{*teray} in 
\rayr{terjmino}{teraymino} might also be an adjective supposed to mean `sad' 
(which would make it an adjectival coordinative compound), although the 
dictionary entry for that is \rayr{gidj}{giday}. Even though parts of all 
these words are unclear, they all seem to follow the correct syntactic order, 
judging by those parts that are identifiable. And even in the case of 
\rayr{vetjsno}{vetaysano}, which is missing the first part, it can be 
reasonably assumed that the identifiable part, *\rayr{sno}{*sano}, is the 
modifier, and *\rayr{vetj}{vetay} may have once been intended to mean `money' 
or `fee' or something along these lines.

With the exception of \rayr{niNMbkrF}{ningambakar}, all of the mystery words 
were entered into the dictionary in 2006. Digging through old archives and 
translations, I could determine at least that *\rayr{bkrF}{*bakar} was once 
intended to mean `lie', and *\rayr{terj}{*teray} was indeed intended to 
mean `sad'.

\index{compounds|)}

\subsection{Reduplication}
\index{reduplication|(}

\citet{wiltshiremarantz2000} write that it has been suggested that 
reduplication serves an iconic function, 
\textcquote[561]{wiltshiremarantz2000}{with the repetition of phonological 
material indicating a repetition or intensity in the semantics}, so with 
regards to nouns it mainly serves to indicate plurality of various kinds. 
However, they find that in fact, reduplication serves all kinds of functions, 
also ones without iconic meanings, and mention Agta, an Austronesian language 
of the Philippines, which uses reduplication to form diminutives 
\citep[6--9]{healey1960}. As we have seen in \autoref{subsec:reduplication} 
above, so does Ayeri, and it is justified in doing so since there is 
real-world evidence for this use of reduplication. A few examples of diminutive
reduplication are given in (\ref{ex:dimredup}).

\begin{figure}[h]
\ex\labels\label{ex:dimredup}
	\begin{tabular}[t]{@{\tl\quad} l @{\enspace→\enspace} l @{\smallskip}}
	\xayr{\larger limu}{limu}{shirt}
		& \xayr{\larger limu/limu}{limu-limu}{little shirt}
		\\
	\xayr{\larger nN}{nanga}{house}
		& \xayr{\larger nN/nN}{nanga-nanga}{little house}
		\\
	\xayr{\larger spj}{sapay}{hand}
		& \xayr{\larger spj/spj}{sapay-sapay}{little hand}
		\\
	\xayr{\larger venej}{veney}{dog}
		& \xayr{\larger venej/venej}{veney-veney}{little dog}
		\\
	\end{tabular}
\xe
\end{figure}

Diminutive reduplication involves full-stem reduplication in Ayeri. Besides the
productive use of reduplication for diminutive marking, there are a number of
diminutive formations which have been lexicalized, such as in the examples
given in (\ref{ex:lexdimredup}). There are also at least two documented cases
where the reduplicated root is not a noun, but the reduplication results in a
noun; compare (\ref{ex:nomzredup}).

\begin{figure}[h]
\ex\labels\label{ex:lexdimredup}
	\begin{tabular}[t]{@{\tl\quad} l @{\enspace→\enspace} l @{\smallskip}}
	\xayr{\larger Agu}{agu}{chicken}
		& \xayr{\larger Agu/Agu}{agu-agu}{chick}
		\\
	\xayr{\larger gnF}{gan}{child}
		& \xayr{\larger gnF/gnF}{gan-gan}{grandchild}
		\\
	\xayr{\larger psiNF}{pasing}{tube}
		& \xayr{\larger psiNF/psiNF}{pasing-pasing}{straw}
		\\
	\xayr{\larger poyu}{poyu}{cheek; bacon}
		& \xayr{\larger poyu/poyu}{poyu-poyu}{butt}
		\\
	\end{tabular}
\xe
\end{figure}

\begin{figure}[h]
\ex\labels\label{ex:nomzredup}
	\begin{tabular}[t]{@{\tl\quad} l @{\enspace→\enspace} l @{\smallskip}}
	\xayr{\larger kusNF}{kusang}{double (adj.)}
		& \xayr{\larger kusNF/kusNF}{kusang-kusang}{model}
		\\
	\xayr{\larger veh/}{veh-}{build}
		& \xayr{\larger veh/veh}{veha-veha}{tinkering}
		\\
	\end{tabular}
\xe
\end{figure}

Reduplicated nouns behave like regular nouns with regards to inflection, that 
is, they receive prefixes and suffixes just like the simplexes from which they 
are derived. This is illustrated in (\ref{ex:diminfl}) for \xayr{venej/venej}
{veney-veney}{little dog}, from \xayr{venej}{veney}{dog}.

\begin{figure}[h]
\ex\label{ex:diminfl}\begingl
	\gla Puco mino \textbf{veney-veneyang}. //
	\glb puk-yo mino veney\til{}veney-ang //
	\glc jump-\TsgN{} happily \Dim{}\til{}dog-\Aarg{} //
	\glft `The little dog is jumping happily.' //
\endgl\xe
\end{figure}

In (\ref{ex:diminfl}), the reduplicated noun \rayr{venej/venej}{veney-veney} is
marked as an agent in that the agent suffix \rayr{/ANF}{-ang} is appended to
the noun as a unit \emph{after} reduplicating the noun stem. In other words,
the following formation in which the root is reduplicated along with its
declension suffix is ungrammatical for the purpose of forming a diminutive:
*\rayr{\larger veneyNF/veneyNF}{*veneyang-veneyang}. Likewise, the reduplicated
form is not treated in the way of an endocentric compound, so case and
plural marking cannot be appended to the first element: *\rayr{\larger veneyNF
venej}{*veneyang veney}.

While ordinary nouns undergo full reduplication to form a diminutive, in 
compounds, only the head is reduplicated, unless the compound is strongly 
lexicalized or has an idiomatic meaning going beyond that of its components. 
(\ref{ex:compredup}) shows the simple case of a transparent endocentric 
compound.

\begin{figure}[h]
\ex\label{ex:compredup}\begingl
	\gla Ya @ yomayo mehir-mehirang seygo veno kay pang nanga nana. //
	\glb ya= yoma-yo mehir\til{}mehir-ang seygo veno kay pang nanga-Ø nana //
	\glc \LocT{}= be-\TsgN{} \Dim{}\til{}tree-\Aarg{} apple pretty three 
		back house-\Top{} \Fpl{}.\Gen{} //
	\glft `There are three pretty little apple trees behind our house.' //
\endgl\xe
\end{figure}

In this example, being endearing or otherwise small is treated as a property of
the head, \xayr{mehirF}{mehir}{tree}, not of the whole compound 
\xayr{mehirFsejgo}{mehirseygo}{apple tree}, or the dependent, 
\xayr{sejgo}{seygo}{apple}---after all, an apple tree which is small is 
rather a small tree with apples on it than a tree with small apples. The 
avoidance of the fully reduplicated form 
\rayr{mehirFsejgo/mehirFsejgo}{mehirseygo-mehirseygo} is probably related to
the notion of economy of expression.

\index{reduplication|)}

\subsection{Nominalization}
\label{subsec:nominalization}
\index{nominalization|(}

Some accidental ways of deriving nouns have been mentioned above, for instance,
some reduplicated non-nominal roots like \xayr{kusNF}{kusang}{double} or
\xayr{veh/}{veha-}{build} may form nouns. However, Ayeri also has some 
dedicated morphology to derive nouns from other parts of speech. The most 
common and highly productive way to derive a noun, is the suffix 
\rayr{/AnF}{-an}. The examples in (\ref{ex:vb-nn}) illustrate some derivations 
from verbs, and (\ref{ex:adj-nn}) shows derivations from adjectives to nouns. 
As \xayr{kuhnF}{kuhan}{oar} shows, the nominalization may have an idiomatic 
meaning.

\begin{figure}[h]
\ex\label{ex:vb-nn}\labels
	\begin{tabular}[t]{@{\tl\quad} l @{\enspace→\enspace} l @{\smallskip}}
	\xayr{\larger blNF/}{balang-}{search (v.)}
		& \xayr{\larger blNnF}{balangan}{search (n.)}
		\\
	\xayr{\larger kuhF/}{kuh-}{row}
		& \xayr{\larger kuhnF}{kuhan}{oar}
		\\
	\xayr{\larger rigF/}{rig-}{draw}
		& \xayr{\larger rignF}{rigan}{drawing}
		\\
	\xayr{\larger vehF/}{veh-}{build}
		& \xayr{\larger vehnF}{vehan}{building}
		\\
	\end{tabular}
\xe
\end{figure}

\begin{figure}[h]
\ex\label{ex:adj-nn}\labels
	\begin{tabular}[t]{@{\tl\quad} l @{\enspace→\enspace} l @{\smallskip}}
	\xayr{\larger Apitu}{apitu}{clean}
		& \xayr{\larger Apitu\_anF}{apituan}{cleanliness}
		\\
	\xayr{\larger gir}{gira}{urgent}
		& \xayr{\larger giraanF}{girān}{hurry}
		\\
	\xayr{\larger pkisF}{pakis}{serious}
		& \xayr{\larger pkisnF}{pakisan}{seriousness}
		\\
	\xayr{\larger vp}{vapa}{skillful}
		& \xayr{\larger vpnF}{vapan}{skill}
		\\
	\end{tabular}
\xe
\end{figure}

Occasionally, it may even happen that a noun is derived from a noun with a 
related but sometimes more basic meaning using the nominalizer 
\rayr{/AnF}{-an}. This process, however, is not productive, so compared to
deverbalization and deadjectivization, examples of this derivation strategy are
few. (\ref{ex:nn-nn}) gives examples of double nominalizations.

\begin{figure}[h]
\ex\label{ex:nn-nn}\labels
	\begin{tabular}[t]{@{\tl\quad} l @{\enspace→\enspace} l @{\smallskip}}
	\xayr{\larger AgYmF}{ajam}{toy}
		& \xayr{\larger AgYmnF}{ajaman}{game}
		\\
	\xayr{\larger kelNF}{kelang}{chain}
		& \xayr{\larger kelNnF}{kelangan}{connection}
		\\
	\xayr{\larger nN}{nanga}{house}
		& \xayr{\larger nNaanF}{nangān}{household}
		\\
	\xayr{\larger tenF}{ten}{life}
		& \xayr{\larger tennF}{tenan}{soul}
		\\
	\end{tabular}
\xe
\end{figure}

There are also some apparent nominalizations in \rayr{/AmF}{-am} and 
\rayr{/ANF}{-ang}, although these are irregular and non-productive; compare 
(\ref{ex:nounamderiv}) and (\ref{ex:nounangderiv}). At least the \rayr{/AmF}
{-am} derivations in (\ref{ex:nounamderiv}) seem to have a connotation of being
tools used for the action they derive from; the \rayr{/ANF}{-ang} derivations
listed seem to derive a more abstract related term. As mentioned, however,
these tendencies are not entirely regular.

\begin{figure}
\ex\labels\label{ex:nounamderiv}
	\begin{tabular}[t]{@{\tl\quad} l @{\enspace→\enspace} l @{\smallskip}}
	\xayr{\larger AgY/}{aja-}{play}
		& \xayr{\larger AgYmF}{ajam}{toy}
		\\
	\xayr{\larger ginF/}{gin-}{drink}
		& \xayr{\larger ginmF}{ginam}{glass}
		\\
	\xayr{\larger mikF/}{mik-}{poison (v.)}
		& \xayr{\larger mikmF}{mikam}{poison (n.), venom}
		\\
	\xayr{\larger nun/}{nuna-}{fly}
		& \xayr{\larger nunmF}{nunam}{feather}
		\\
	\end{tabular}
\xe
\end{figure}

\begin{figure}
\ex\labels\label{ex:nounangderiv}
	\begin{tabular}[t]{@{\tl\quad} l @{\enspace→\enspace} l @{\smallskip}}
	\xayr{\larger bjh/}{bayha-}{rule}
		& \xayr{\larger bjhNF}{bayhang}{government}
		\\
	\xayr{\larger hp}{hapa}{remaining}
		& \xayr{\larger hpNF}{hapang}{remainder}
		\\
	\xayr{\larger kd/}{kada-}{collect}
		& \xayr{\larger kdNF}{kadang}{committee; alliance}
		\\
	\xayr{\larger mim}{mima}{possible}
		& \xayr{\larger mimNF}{mimang}{access}
		\\
	\end{tabular}
\xe
\end{figure}

Agentive nouns can be formed from regular nouns with the suffix 
\rayr{/my}{-maya}, compare the examples in (\ref{ex:mayaregular}). An 
epenthetic /a/ may be introduced to break up consonant clusters that would
otherwise be either difficult to pronounce or violating phonotactics. When the
stem of the word to which the agentive suffix is attached ends in a consonant
or /Ca/, it is also often found fused with the root, sometimes with the first
/a/ of \fw{-Caya} lengthened, compare (\ref{ex:mayairregular}). Specifically
feminine agentive nouns can be derived with the related suffix
\rayr{/vy}{-vaya}; two examples of this are given in (\ref{ex:vaya}).

\begin{figure}[h]
\ex\label{ex:mayaregular}\labels
	\begin{tabular}[t]{@{\tl\quad} l @{\enspace→\enspace} l @{\smallskip}}
	\xayr{\larger AnFlF/}{anl-}{bring}
		& \xayr{\larger AnFlmy}{anlamaya}{waiter}
		\\
	\xayr{\larger hor}{hora}{sin}
		& \xayr{\larger hormy}{horamaya}{sinner}
		\\
	\xayr{\larger nsY/}{nasy-}{follow}
		& \xayr{\larger nsYmy}{nasyamaya}{follower}
		\\
	\xayr{\larger teb/}{teba-}{bake}
		& \xayr{\larger tebmy}{tebamaya}{baker}
		\\
	\end{tabular}
\xe
\end{figure}

\begin{figure}[h]
\ex\label{ex:mayairregular}\labels
	\begin{tabular}[t]{@{\tl\quad} l @{\enspace→\enspace} l @{\smallskip}}
	\xayr{\larger As/}{asa-}{travel}
		& \xayr{\larger Asaay}{asāya}{traveler}
		\\
	\xayr{\larger IbutF/}{ibut-}{trade}
		& \xayr{\larger Ibuty}{ibutaya}{trader, merchant}
		\\
	\xayr{\larger lMtF/}{lant-}{lead}
		& \xayr{\larger lMty}{lantaya}{leader; driver}
		\\
	\xayr{\larger tNF/}{tang-}{listen}
		& \xayr{\larger tNy}{tangaya}{listener}
		\\
	\end{tabular}
\xe
\end{figure}

\begin{figure}[h]
\ex\label{ex:vaya}\labels
	\begin{tabular}[t]{@{\tl\quad} l @{\enspace→\enspace} l @{\smallskip}}
	\xayr{\larger gnF}{gan}{child}
		& \xayr{\larger gnFvy}{ganvaya}{governess}
		\\
	\xayr{\larger lnY}{lanya}{king}
		& \xayr{\larger lnFvy}{lanvaya}{queen}
		\\
	\end{tabular}
\xe
\end{figure}

Besides the agentive suffixes, there is also a derivative suffix for makers of
things, \rayr{/Ati}{-ati} (contracting to \rayr{/AtYF/}{-ac-} before a vowel),
though this is not too productive, and sometimes irregular, as
\xayr{sirFtNti}{sirtangati}{youth} in (\ref{ex:makerderiv}) shows. Moreover,
there are instances of nominalization where a tool of sorts is derived with a
suffix \rayr{/(E)rYnF}{-(e)ryan}, which is related to the instrumental suffix
\rayr{/Eri}{-eri} in combination with the nominalizer \rayr{/AnF}{-an}; compare
(\ref{ex:toolderiv}).

\begin{figure}[h]
\ex\labels\label{ex:makerderiv}
	\begin{tabular}[t]{@{\tl\quad} l @{\enspace→\enspace} l @{\smallskip}}
	\xayr{\larger giMdi}{gindi}{poem}
		& \xayr{\larger giMdti}{gindati}{poet}
		\\
	\xayr{\larger sirFtNF}{sirtang}{young}
		& \xayr{\larger sirFtNti}{sirtangati}{youth}
		\\
	\xayr{\larger thnF/}{tahan-}{write}
		& \xayr{\larger thnti}{tahanati}{scribe}
		\\
	\xayr{\larger vehimF}{vehim}{piece of clothing}
		& \xayr{\larger vehimti}{vehimati}{tailor}
		\\
	\end{tabular}
\xe
\end{figure}

\begin{figure}[h]
\ex\labels\label{ex:toolderiv}
	\begin{tabular}[t]{@{\tl\quad} l @{\enspace→\enspace} l @{\smallskip}}
	\xayr{\larger gurF/}{gur-}{turn}
		& \xayr{\larger gurFynF}{guryan}{coil, cylinder}
		\\
	\xayr{\larger misF/}{mis-}{behave}
		& \xayr{\larger miserYnF}{miseryan}{method, strategy}
		\\
	\xayr{\larger npF/}{nap-}{burn}
		& \xayr{\larger nperYnF}{naperyan}{tinder}
		\\
	\xayr{\larger pr/}{pra-}{glitter, gleam}
		& \xayr{\larger pFrrYnF}{praryan}{spark}
		\\
	\end{tabular}
\xe
\end{figure}

\index{gerund|(}

While \rayr{/AnF}{-an} derives nouns from verbs to produce nouns that act as
such in every way, it may sometimes be preferable to refer to the action itself
by a noun, compare (\ref{ex:enggerund}) for an example from English. In
(\ref{ex:devnouneng}), \fw{building} is simply a noun derived from the verb
\fw{build}. It acts as a noun in every way, for example, it can serve as a 
subject and object, it can be pluralized, it can take determiners, and can be 
modified by adjectives.

% \begin{figure}[h]
\pex\label{ex:enggerund}%
	English:
	\a\label{ex:devnouneng} \fw{Manhattan is famous for its tall
		\textbf{buildings}.}
	\a\label{ex:gerundeng} \fw{\textbf{Building} a house is an expensive
		endeavor.}
\xe
% \end{figure}

The form of \fw{building} in (\ref{ex:gerundeng}), however, is a gerund, and as
such underlies the restriction that it cannot be pluralized
\citep[35]{payne1997}. As we have seen at the beginning of this section on
nominalization, Ayeri can derive \xayr{vehnF}{vehan}{building, construction}
from the verb \xayr{vehF/}{veh-}{build}, which acts like every other common
noun, much like in the English example in (\ref{ex:devnouneng}).

\begin{figure}
\pex\label{ex:vbnomz}
\a\label{ex:nomz-sbj-adj}\begingl
	\gla Lesāra sirimang \textbf{vehānreng} \textbf{tado}. //
	\glb lesa-ara sirimang vehān-reng tado //
	\glc collapse-\TsgI{} about.to building-\AargI{} old//
	\glft `The old building is about to collapse.' //
\endgl

\a\label{ex:nomz-obj-det}\begingl
	\gla Le @ vacyang \textbf{eda-vehān}. //
	\glb le= vac=yang eda=vehān-Ø //
	\glc \PatTI{}= like-\Fsg{}.\Aarg{} this=building-\Top{} //
	\glft `This building, I like it.' //
\endgl

\a\label{ex:nomz-pl-poss}\begingl
	\gla Ang @ latayo bayhang \textbf{vehānyeley} \textbf{yona}. //
	\glb ang= lata-yo bayhang-Ø vehān-ye-ley yona //
	\glc \AgtT{}= sell-\TsgN{} government-\Top{} 
		building-\Pl{}-\PargI{} \TsgN{}.\Gen{} //
	\glft `The government is selling its buildings.' //
\endgl

\a\label{ex:nomz-qty}\begingl
	\gla Le @ ming @ kuysāran \textbf{vehān-kay} dirasyam ran. //
	\glb le= ming= kuysa-aran vehān-Ø=kay diras-yam ran //
	\glc \PatTI{}= can= compare-\TplI{} building-\Top=few 
		splendor-\Dat{} \TsgI{}.\Gen{} //
	\glft `Few buildings can compare to its splendor.' //
\endgl
\xe
\end{figure}

The examples in (\ref{ex:vbnomz}) condense several properties into one for
illustration. For instance, (\ref{ex:nomz-sbj-adj}) shows that \rayr{vehaanF}
{vehān} can serve as the subject of a clause, and that it can as well
be modified by an adjective---the choice of adjectives is not subject to any
distributional restrictions other than those imposed by the semantic frame of
\textsc{house}. In the next example, (\ref{ex:nomz-obj-det}),
\rayr{vehaanF}{vehān} serves as the object of the clause and is being
determined by the demonstrative prefix \xayr{Ed/}{eda-}{this}. The third
example, (\ref{ex:nomz-pl-poss}), shows \rayr{vehaanF}{vehān} both pluralized
and modified by a possessive pronoun, \xayr{yon}{yona}{of it}. And finally, in
(\ref{ex:nomz-qty}) we see \rayr{vehaanF}{vehān} quantified by the enclitic
\xayr{/kj}{-kay}{few}.

\begin{figure}[h]
\ex\label{ex:kafkagerund}\begingl
	\gla … nay ang @ pətangongva ankyu \textbf{haruyamanas} nanang … //
		% megayena yana kunangya vana. //
	\glb … nay ang= pə-tang-ong=va.Ø ankyu haru-yam-an-as nanang … //
		% mega-ye-na yana kunang-ya vana //
	\glc … and \AgtT{}= \NFut{}-hear-\Irr{}=\Ssg{}.\Top{} truly 
		beat-\Ptcp{}-\Nmlz{}-\Parg{} great … // % fist-\Pl{}-\Gen{} 
		% \TsgM{}.\Gen{} door-\Loc{} \Ssg{}.\Gen{} //
	\glft `… and you would indeed hear the magnificent beating …' //
		% at your door very soon.' //
\endgl\xe
\end{figure}

Similar to the English example in (\ref{ex:gerundeng}), Ayeri can also derive
nouns from the participle of a verb describing the action as such---a gerund.
(\ref{ex:kafkagerund}) again draws on the Ayeri translation of Kafka's short
story \enquote{Eine kaiserliche Botschaft} \citep[2, 14]{becker:kafka:imperial}
for an example. The annotations to this translation contain a comment on the
grammatical rules which operate in this passage, more specifically also on the
gerund derivation \xayr{hruymnF}{haruyaman}{beating}:

\blockcquote[14--15]{becker:kafka:imperial}{Furthermore, I wrote
\fw{haruyaman} `beating' instead of \fw{haruan} `beat(ing)' because I wanted to
emphasize the process of beating as an incomplete action. This is possible here
because the word is not topicalized and neither is it marked as a dative, which
would also require \fw{haruyamanyam} `beat-\Ptcp{}-\Nmlz{}-\Dat{}' to become
\fw{haruanyam} `beat-\Nmlz{}-\Dat{}' (the participle marker \fw{-yam} is 
derived from the dative case ending \fw{-yam}).}

We can read from this description that the participle marker in Ayeri has
possibly been grammaticalized from the dative case marker, or that it is at
least synchronically homonymous. In order for case marking to operate, this
formation has to be nominalized, which is done in the usual way by appending
\rayr{/AnF}{-an}, thus yielding the suffix cluster \rayr{/ymnF}{-yaman} for the
derivation of verbs as gerunds. If the gerund is marked for dative case, the
suffix cluster *\rayr{/ymnFymF}{*-yamanyam} basically undergoes haplology to a
simple nominalized form with the suffix cluster \rayr{/AnFymF}{-anyam}. See
(\ref{ex:datnmlz}) for an example.

\begin{figure}[h]
\ex\label{ex:datnmlz}%
\begingl
	\gla haru- {} haruyam {} haruyaman {} *haruyamanyam {} haruanyam //
	\glb haru- → haru-yam → haru-yam-an → haru-yam-an-yam → 
		haru-an-yam //
	\glc beat {} beat-\Ptcp{} {} beat-\Ptcp{}-\Nmlz{} {} 
		beat-\Ptcp{}-\Nmlz{}-\Dat{} {} beat-\Nmlz{}-\Dat{} //
\endgl\xe
\end{figure}

The comment on the translation also makes a little note on the gerund being
possible because the word is not topicalized. This is based on an old rule that
gerunds cannot be topicalized unless nominalized first, however, usage has
since changed so that earlier, \rayr{hruymF}{haruyam} would have constituted
the gerund form, while even by the time of translating the short story, it had
changed to \rayr{hruymnF}{haruyaman}. This is encountered in
(\ref{ex:exuperygerund}), an example from the partial translation of
Saint-Exupéry's story \enquote{Le petit prince} \citep[3, 13]
{benung:petitprince}. A more literal translation of this sentence would be `The
distinguishing of China and Arizona, I knew it at first sight', so the whole
passage \rayr{pluNFymnF — n byokivo}{palungyaman … na Bayokivo} forms the topic
of the sentence here, headed by the gerund
\xayr{pluNFymnF}{palungyaman}{distinguishing}. According to the old rule
obliquely quoted in the comment to the passage in (\ref{ex:kafkagerund}), this
should not be possible. As mentioned before, though, use has changed.

\begin{figure}[h]
\ex\label{ex:exuperygerund}\begingl
	\gla Sa @ koronyang \textbf{palungyaman} na @ Baysānterpeng nay na @
		Bayokivo menaneri nivānyena. //
	\glb sa= koron=yang palung-yam-an-Ø na= Baysānterpeng nay na= 
		Bayokivo menan-eri nivān-ye-na //
	\glc \PatT{}= knew=\Fsg{}.\Aarg{} distinguish-\Ptcp{}-\Nmlz{}-\Top{} 
		\Gen{}= Realm.Middle and \Gen{}= Spring.Little first-\Ins{} 
		glimpse-\Pl{}-\Gen{} //
	\glft `I knew the difference between China and Arizona at first sight.' //
\endgl\xe
\end{figure}

A rule we can gather from (\ref{ex:exuperygerund}) is that gerunds are treated
as animate nouns. Since they are impersonal, they trigger neuter agreement on
verbs. They can also be the objects of sentences. The passage in
(\ref{ex:kafkagerund}) furthermore illustrates that gerunds can be modified by
adjectives. The example in (\ref{ex:scimethgerund}) shows a gerund used as an
agent-subject as well \citep{benung:scientificmethod}.

\begin{figure}[h]
\ex\label{ex:scimethgerund}\begingl
	\gla \textbf{Dilayamanang} kalamena bahalanas ayonena … //
	\glb dila-yam-an-ang kalam-ena bahalan-as ayon-ena … //
	\glc find.out-\Ptcp{}-\Nmlz{}-\Aarg{} truth-\Gen{} goal-\Parg{} 
		man-\Gen{} … //
	\glft `(If) finding out the truth is the goal of the man …' //
\endgl\xe
\end{figure}

What all the passages on gerunds quoted before show is that gerunds in Ayeri do
not behave like transitive verbs as in English. Thus, what would be the object
of the former verb appears in the genitive case in Ayeri. As in English,
however, gerunds in Ayeri cannot be pluralized; compare (\ref{ex:grndplur}). It
is possible, however, to quantify gerunds insofar as the quantifier does not
imply countable quantities of the action. Moreover, it is possible for gerunds
to be modified by possessors. The two sentences in (\ref{ex:grndmod}) exemplify
this use.

\begin{figure}[h]
\ex\label{ex:grndplur}\ljudge*\begingl
	\gla Noyo \textbf{vehayamanjang} nangayena. //
	\glb noyo veha-yam-an-ye-ang nanga-ye-na //
	\glc expensive build-\Ptcp{}-\Nmlz{}-\Pl{}-\Aarg{} house-\Pl{}-\Gen{} //
	\glft `*The buildings of houses are expensive.' //
\endgl\xe
\end{figure}

\begin{figure}[h]
\pex\label{ex:grndmod}
\a\begingl
	\gla Ang @ lugayan \textbf{delacamanas-ikan} kayanya pang. //
	\glb ang= luga=yan.Ø delak-yam-an-as=ikan kayan-ya pang //
	\glc \AgtT{}= go.through=\TplM{}.\Top{} 
		suffer-\Ptcp{}-\Nmlz{}-\Parg{}=much war-\Loc{} after //
	\glft `They went through a lot of suffering after the war.' //
\endgl

\a\begingl
	\gla Krico \textbf{malyyamanang} muya \textbf{tan}. //
	\glb krit-yo maly-yam-an-ang muya tan //
	\glc annoy-\TsgN{} sing-\Ptcp{}-\Nmlz{}-\Aarg{} wrong \TplM{}.\Gen{} //
	\glft `Their off singing is annoying.' //
\endgl
\xe
\end{figure}

\index{gerund|)}
\index{nominalization|)}

\index{nouns|)}

\section{Pronouns}
\index{pronouns|(}

Ayeri possesses different kinds of pronouns in the sense that there is a closed
class of words which contains anaphora of various types---personal pronouns,
demonstrative pronouns, interrogative pronouns, relative pronouns, as well as
reflexive and reciprocal expressions. Each class of pronouns will be discussed
in the following.

\subsection{Personal pronouns}
\label{subsec:perspro}
\index{pronouns!personal|(}

\begin{table}[tp]\centering
\caption{Personal pronouns}

\begin{tabu} to \linewidth{l X[c] X[c] X[c] X[c] X[c] X[c] X[c] X[c]}
\tableheaderfont\toprule
Person
	& \Top{}
	& \Aarg{}
	& \Parg{}
	& \Dat{}
	& \Gen{}
	& \Loc{}
	& \Caus{}
	& \Ins{}
	\\
\toprule

\Fsg{}
	& ay	% \Top{}
	& yang	% \Aarg{}
	& yas	% \Parg{}
	& yām	% \Dat{}
	& nā	% \Gen{}
	& yā	% \Loc{}
	& sā	% \Caus{}
	& rī	% \Ins{}
	\\
	
\midrule

\Ssg{}
	& va	% \Top{}
	& vāng	% \Aarg{}
	& vās	% \Parg{}
	& vayam	% \Dat{}
	& vana	% \Gen{}
	& vaya	% \Loc{}
	& vasa	% \Caus{}
	& vari	% \Ins{}
	\\

\midrule

\TsgM{}
	& ya	% \Top{}
	& yāng	% \Aarg{}
	& yās	% \Parg{}
	& yayam	% \Dat{}
	& yana	% \Gen{}
	& yāy	% \Loc{}
	& yasa	% \Caus{}
	& yari	% \Ins{}
	\\

\TsgF{}
	& ye	% \Top{}
	& yeng	% \Aarg{}
	& yes	% \Parg{}
	& yeyam	% \Dat{}
	& yena	% \Gen{}
	& yea	% \Loc{}
	& yesa	% \Caus{}
	& yeri	% \Ins{}
	\\

\TsgN{}
	& yo	% \Top{}
	& yong	% \Aarg{}
	& yos	% \Parg{}
	& yoyam	% \Dat{}
	& yona	% \Gen{}
	& yoa	% \Loc{}
	& yosa	% \Caus{}
	& yori	% \Ins{}
	\\

\TsgI{}
	& ra	% \Top{}
	& reng	% \Aarg{}
	& rey	% \Parg{}
	& rayam	% \Dat{}
	& ran	% \Gen{}
	& raya	% \Loc{}
	& rasa	% \Caus{}
	& rari	% \Ins{}
	\\

\midrule

\Fpl{}
	& ayn	% \Top{}
	& nang	% \Aarg{}
	& nas	% \Parg{}
	& nyam	% \Dat{}
	& nana	% \Gen{}
	& nyā	% \Loc{}
	& nisa	% \Caus{}
	& ni	% \Ins{}
	\\
	
\midrule

\Spl{}
	& va	% \Top{}
	& vāng	% \Aarg{}
	& vās	% \Parg{}
	& vayam	% \Dat{}
	& vana	% \Gen{}
	& vaya	% \Loc{}
	& vasa	% \Caus{}
	& vari	% \Ins{}
	\\

\midrule

\TplM{}
	& yan	% \Top{}
	& tang	% \Aarg{}
	& tas	% \Parg{}
	& cam	% \Dat{}
	& tan	% \Gen{}
	& ca	% \Loc{}
	& tis	% \Caus{}
	& ti	% \Ins{}
	\\

\TplF{}
	& yen	% \Top{}
	& teng	% \Aarg{}
	& tes	% \Parg{}
	& teyam	% \Dat{}
	& ten	% \Gen{}
	& teya	% \Loc{}
	& tēs	% \Caus{}
	& teri	% \Ins{}
	\\

\TplN{}
	& yon	% \Top{}
	& tong	% \Aarg{}
	& tos	% \Parg{}
	& toyam	% \Dat{}
	& ton	% \Gen{}
	& toya	% \Loc{}
	& tōs	% \Caus{}
	& tori	% \Ins{}
	\\

\TplI{}
	& ran	% \Top{}
	& teng	% \Aarg{}
	& tey	% \Parg{}
	& racam	% \Dat{}
	& ten	% \Gen{}
	& raca	% \Loc{}
	& ratas	% \Caus{}
	& ray	% \Ins{}
	\\

\bottomrule
\end{tabu}
\label{tab:perspro}
\end{table}

As \autoref{tab:perspro} shows, Ayeri possesses quite a large number of
personal pronouns with (maybe unnaturally) little syncretism between the
different paradigm slots overall (the second person is a notable exception);
there are also no gaps in the paradigm. Ayeri's personal pronouns reflect the
grammatical features also found in nouns, that is, number, gender, and case,
and person is added to that. The individual forms range from completely fused
to fully transparent even within the same case paradigm, for instance,
\xayr{yaamF}{yām}{(to/for) me} (\Fsg{}.\Dat{}) on the one hand, and 
\xayr{yymF}{yayam}{(to/for) him} (transparently \TsgM{}-\Dat{}) on the other. 
Originally, all pronouns have been regular formations based on the respective
unmarked pronominal element listed in the \Top{} column of
\autoref{tab:perspro} declined by adding a case suffix (see
\autoref{subsec:case}). Use has caused many of these formations to contract and
erode as grammaticalization progressed, for instance the first person agent and
third person animate masculine plural pronouns; compare (\ref{ex:prongen}).

\begin{figure}[h]
\pex\label{ex:prongen}
\a\begingl
	\gla ayang → yang //
	\glb ay-ang {} yang //
	\glc \makebox[\widthof{\Tsg{}-\M{}-\Pl{}-\Gen{}}][l]{\Fsg{}-\Aarg{}} {} 
		\Fsg{}.\Aarg{} //
\endgl

\a\begingl
	\gla iyatena → tan //
	\glb iy-a-t-ena {} tan //
	\glc \Tsg{}-\M{}-\Pl{}-\Gen{} {} \TsgM{}.\Gen{}\footnotemark //
\endgl
\xe
\end{figure}

\footnotetext{Strictly speaking, this could as well be glossed as \fw{t<a>n} 
(\Tsg{}.\Gen{}<\M{}>). I chose to gloss the pronoun in the above way, however, 
in order to not overly complicate things.}

The plural series used to be derived by adding \rayr{/nF}{-n} or, in the third 
person, \rayr{/tF/}{\mbox{-t-}} to the pronoun stem, which can still be easily 
observed in the unmarked pronouns as well as in the alternation between 
\rayr{yF/}{y-} and \rayr{tF/}{t-} in the third person pronouns. The same goes 
for the gender-marking thematic vowel in the animate third person pronouns, 
which has been retained as a distinctive feature even in the non-core pronouns 
despite sometimes heavy modifications. A further interesting property of Ayeri 
is that synchronically, singular and plural are distinguished, except for the 
second person, where the forms are the same, basically like in English. 
\citet{lehmann2015} explains, however, that this is not an unusual route for 
languages to take:

\blockcquote[42]{lehmann2015}{New pronouns, especially for the second person 
singular, are often obtained by shifting pronouns around in the paradigm,
especially by substituting marked forms for unmarked ones. This explains, for
instance, the use of [...]\ English \fw{you} for the second person singular
[...]}

The second person singular subject pronoun of English used to be \fw{thou}, 
cognate to German \fw{du}, which can still be found in Shakespeare, for 
instance. Something along the lines of English \fw{you} as a second 
person plural pronoun replacing second person singular \fw{thou} by way of a 
deferential singular use of a plural pronoun \citep[you, pron., adj., and 
n.]{oed} may have happened in Ayeri as well.

\begin{figure}[h]
\pex\label{ex:perspro}
\a\label{ex:pronfull}\begingl
	\gla Ang @ harya {} @ Paradan tandās kaleri. //
	\glb ang= har-ya Ø= Paradan tanda-as kal-eri //
	\glc \AgtT{}= beat-\TsgM{} \Top{}= Paradan fly-\Parg{} rag-\Ins{} //
	\glft `Paradan, he beats the fly with a rag.' //
\endgl

\a\label{ex:pronagt}\begingl
	\gla Sa @ haryāng tanda kaleri. //
	\glb sa= har=yāng tanda-Ø kal-eri //
	\glc \PatT{}= beat=\TsgM{}.\Aarg{} fly-\Top{} rag-\Ins{} //
	\glft `The fly, he beats it with a rag.' //
\endgl

\a\label{ex:pronpat}\begingl
	\gla Ang @ harya {} @ Paradan yos kaleri. //
	\glb ang= har-ya Ø= Paradan yos kal-eri //
	\glc \AgtT{}= beat-\TsgM{} \Top{}= Paradan \TsgN{}.\Parg{} rag-\Ins{} //
	\glft `Paradan, he beats it with a rag.' //
\endgl

\a\label{ex:pronins}\begingl
	\gla Ang @ harya {} @ Paradan tandās rari. //
	\glb ang= har-ya Ø= Paradan tanda-as rari //
	\glc \AgtT{}= beat-\TsgM{} \Top{}= Paradan fly-\Parg{} \TsgI{}.\Ins{} //
	\glft `Paradan, he beats the fly with it.' //
\endgl
\xe
\end{figure}

The personal pronouns are used in just the same way as their full-NP
counterparts would be, also in the non-core cases. (\ref{ex:pronfull}) shows a
sentence with full subject and object NPs; (\ref{ex:pronagt}) shows a variation
of the sentence with the agent, \rayr{prdnF}{Paradan}, replaced by the third
person singular masculine agent pronoun \xayr{yaaNF}{yāng}{he}. In
(\ref{ex:pronpat}), then, the patient, \xayr{tMdaasF}{tandās}{fly}, is replaced
with the third person singular neuter patient pronoun \rayr{yosF}{yos}. In
(\ref{ex:pronins}), lastly, the instrument, \xayr{kleri}{kaleri}{with a rag} is
replaced with the third person singular inanimate instrumental pronoun
\xayr{rri}{rari}{with it}. Furthermore, complex NPs are in complementary
distribution with pronouns, since pronouns are anaphora for NPs. Thus also, an
NP which contains an adjective is wholly replaced by a personal-pronoun DP 
(determiner phrase), as in (\ref{ex:procompldist}).

\begin{figure}[h]
\pex\label{ex:procompldist}
\a\begingl
	\gla Ang @ ninye vehimley veno. //
	\glb ang= nin=ye.Ø vehim-ley veno //
	\glc \AgtT{}= wear=\TsgF{}.\Top{} dress-\PargI{} beautiful //
	\glft `She wears a beautiful dress.' //
\endgl

\a\ljudge* \begingl
	\gla Ang @ ninye adaley veno. //
	\glb ang= nin=ye.Ø ada-ley veno //
	\glc \AgtT{}= wear=\TsgF{}.\Top{} that-\PargI{} beautiful //
	\glft `*She wears a beautiful it.' //
\endgl

\a\begingl
	\gla Ang @ ninye adaley. //
	\glb ang= nin=ye.Ø ada-ley //
	\glc \AgtT{}= wear=\TsgF{}.\Top{} that-\PargI{} //
	\glft `She wears it.' //
\endgl
\xe
\end{figure}

Comparing the example sentences in (\ref{ex:perspro}) with the \Top{} column
in \autoref{tab:perspro} an important property of personal pronouns becomes 
apparent. That is, the `unmarked' (or rather, zero-marked) pronoun forms are 
also the ones showing as verb agreement. An important difference in this 
respect, however, is that the third person singular inanimate verb agreement 
marker is not \rayr{/r}{-ra}, but \rayr{/Ar}{-ara}. The following two examples 
illustrate the parallel more clearly---observe the person marking on the verb 
in (\ref{ex:verbinfl1}) and the corresponding object pronouns in 
(\ref{ex:verbinfl2}).

\begin{figure}
\pex\label{ex:verbinfl1}
\a\begingl
	\gla Sa @ man\textbf{ya} ang @ Ajān {} @ Pila. //
	\glb sa= man-ya ang= ​Ajān Ø= ​Pila //
	\glc \PatT{}= greet-\TsgM{} \Aarg{}= ​Ajān \Top{}= ​Pila //
	\glft `Pila, Ajān greets her.' //
\endgl

\a\begingl
	\gla Sa @ man\textbf{ye} ang @ Pila {} @ Ajān. //
	\glb sa= man-ye ang= Pila Ø= ​Ajān //
	\glc \PatT{}= greet-\TsgF{} \Aarg{}= Pila \Top{}= ​Ajān //
	\glft `Ajān, she greets him.' //
\endgl

\xe
\end{figure}

\begin{figure}
\pex\label{ex:verbinfl2}
\a\begingl
	\gla Sa @ manye ang @ Pila \textbf{ya}. //
	\glb sa= man-ye ang= Pila ya.Ø //
	\glc \PatT{}= greet-\TsgF{} \Aarg{}= Pila \TsgM{}.\Top{} //
	\glft `Pila, she greets him.' //
\endgl

\a\begingl
	\gla Sa @ manya ang @ Ajān \textbf{ye}. //
	\glb sa= man-ya ang= ​Ajān ye.Ø //
	\glc \PatT{}= greet-\TsgM{} \Aarg{}= ​Ajān \TsgF{}.\Top{} //
	\glft `Ajān, he greets her.' //
\endgl
\xe
\end{figure}

Another important property of both pronouns and verbs is that agent pronouns
(and patient pronouns under certain circumstances) replace person agreement by
cliticizing to the verb stem. Since person agreement morphology is a domain of
verbs, it will be dealt with in more detail in the chapter on verbs proper. 
Example (\ref{ex:agtproclt_1}) again has full subject and object NPs; the verb
displays \rayr{/y}{-ya} as the agreement suffix for the masculine agent NP.
Example (\ref{ex:agtproclt_2}), then, replaces the agent NP with a pronoun.
This is not expressed by a free form like \fw{he}, though, but as a pronominal
clitic, \xayr{/yaaNF}{-yāng}{he}.

\begin{figure}[h]
\pex\label{ex:agtproclt}
\a\label{ex:agtproclt_1}%
\begingl
	\gla Sa @ man\textbf{ya} \textbf{ang} @ \textbf{Ajān} {} @ Pila. //
	\glb sa= man-ya ang= ​Ajān Ø= ​Pila //
	\glc \PatT{}= greet-\TsgM{} \Aarg{}= ​Ajān \Top{}= ​Pila //
	\glft `Pila, Ajān greets her.' //
\endgl

\a\label{ex:agtproclt_2}%
\begingl
	\gla Sa @ man\textbf{yāng} {} @ Pila. //
	\glb sa= man=yāng Ø= ​Pila //
	\glc \PatT{}= greet=\TsgM{}.\Aarg{} \Top{}= Pila //
	\glft `Pila, he greets her.' //
\endgl
\xe
\end{figure}

\index{pronouns!personal|)}
\index{pronouns!possessive|(}

\phantomsection\label{phsec:possadj}
Possessive pronouns are special compared to regular personal pronouns in that
they act basically as possessive adjectives. Pronominal uses of possessive
pronouns need \rayr{d/}{da-} as a supporting particle, however. The main use
for the genitive pronouns in \autoref{tab:perspro} is to show possession. This
means that unlike personal pronouns, they are by themselves not in
complementary distribution with nominal NPs, compare (\ref{ex:procompldist}).
Instead, they may be used as modifiers like, or alongside, adjectives, as
(\ref{ex:adjgen}) shows.

\begin{figure}[h]
\ex\label{ex:adjgen}%
\begingl
	\gla nangaya ledo nā //
	\glb nanga-ya ledo nā //
	\glc house-\Loc{} blue \Fsg{}.\Gen{} //
	\glft `in my blue house' //
\endgl
\xe
\end{figure}

Yet, however, possessives do not fully share the properties of adjectives,
namely, they cannot be compared (*\xayr{naa/ENF}{*nā-eng}{*myer},
*\xayr{naa/vaa}{*nā-vā}{*myest}). However, fronting them in predicative
statements like the one in (\ref{ex:genpred}) is possible even without the
deictic particle. Alternatively, a phrasal construction with
\xayr{vilFyNF/}{vilyang-}{belong}, as indicated in (\ref{ex:genphrase}), may be
used.

\begin{figure}[h]
\pex\label{ex:genpred}
\a\label{ex:genpred_1}\begingl
	\gla Ada-nangāng da-nā. //
	\glb ada=nanga-ang da-nā //
	\glc that=house-\Aarg{} one=\Fsg{}.\Gen{} //
	\glft `That house is mine.' //
\endgl

\a\label{ex:genpred_2}\begingl
	\gla Nā ada-nangāng. //
	\glb Nā ada=nanga-ang //
	\glc \Fsg{}.\Gen{} that=house-\Aarg{} //
	\glft `Mine is that house.' //
\endgl
\xe
\end{figure}

\begin{figure}[h]
\ex\label{ex:genphrase}%
\begingl
	\gla Ang @ vilyangyo ada-nanga yas. //
	\glb ang= vilyang-yo ada=nanga-Ø yas //
	\glc \AgtT{}= belong-\TsgN{} that=house-\Top{} \Fsg{}.\Parg{} //
	\glft `That house belongs to me.' //
\endgl
\xe
\end{figure}

\index{pronouns!possessive|)}

\subsection{Demonstrative pronouns}
\label{subsec:dempro}
\index{pronouns!demonstrative|(}

\begin{table}[tp]\centering
\caption{Demonstrative pronouns}

\begin{tabu} to .75\linewidth{l X[c] X[c] X[c]}
\tableheaderfont\toprule

Case
	& Proximal
	& Distal
	& Indefinite
	\\
\toprule

\Top{}
	& edanya
	& adanya
	& danya
	\\
	
\midrule
	
\Aarg{}
	& edanyāng
	& adanyāng
	& \emph{danyāng}
	\\

\Aarg{}.\Inan{}
	& edareng, \emph{edanyareng}
	& adareng, adanyareng
	& \emph{danyareng}
	\\

\Parg{}
	& edanyās
	& adanyās
	& danyās
	\\

\Parg{}.\Inan{}
	& edaley
	& \emph{adaley}
	& danyaley
	\\

\Dat{}
	& \emph{edayam}
	& adayam
	& \emph{danyayam}
	\\

\midrule

\Gen{}
	& edanyana
	& adanyana
	& danyana
	\\
	
\Loc{}
	& \emph{edanyaya}
	& adanyaya
	& \emph{danyaya}
	\\
	
\Caus{}
	& \emph{edanyasa}
	& \emph{adanyasa}
	& \emph{danyasa}
	\\
	
\Ins{}
	& \emph{edanyari}
	& \emph{adanyari}
	& \emph{danyari}
	\\

\bottomrule
\end{tabu}
\label{tab:detpro}
\end{table}

Demonstrative pronouns in Ayeri are formed with the demonstrative 
prefixes: \xayr{Ed/}{eda-}{this} (proximal), \xayr{Ad/}{ada-}{that} 
(distal), and \xayr{d/}{da-}{such} (indefinite). These are combined with a 
morpheme \rayr{nY}{nya}, which is related to the word for `person', 
\rayr{nYaanF}{nyān}. \autoref{tab:detpro} gives the declined forms for all of 
them. Those forms attested in the corpus gathered from dictionary entries and 
example texts also used for the syllable structure analyses in 
\autoref{sec:phonotactics} appear in upright type, those that should be 
grammatical as well otherwise are given in italic type. The corpus is very 
small, but the prevalence of some forms is possibly reflecting varying degrees 
of grammaticalization at least to some extent. \autoref{tab:detprontokenfq} 
gives the token frequencies of the various attested forms.

\begin{table}[tp]\centering
\caption{Token frequencies of attested demonstrative pronouns}

\begin{tabu} to .75\linewidth {>{\itshape}X[2l] X[2l] X[1c] X[1c]}
\tableheaderfont\toprule

Pronoun
	& Gloss
	& Absolute
	& Relative
	\\

\toprule

edanya
	& this.\Top{}
	& 1
	& 1.69\pct
	\\

adanya
	& that.\Top{}
	& 9
	& 15.25\pct
	\\

danya
	& such.\Top{}
	& 1
	& 1.69\pct
	\\

\midrule

edanyāng
	& this.\Aarg{}
	& 4
	& 6.78\pct
	\\

adanyāng
	& that.\Aarg{}
	& 8
	& 13.56\pct
	\\

edareng
	& this.\AargI{}
	& 3
	& 5.08\pct
	\\

adareng
	& that.\AargI{}
	& 15
	& 25.42\pct
	\\

adanyareng
	& that.\AargI{}
	& 1
	& 1.69\pct
	\\

\midrule

edanyās
	& this.\Parg{}
	& 1
	& 1.69\pct
	\\

adanyās
	& that.\Parg{}
	& 2
	& 3.39\pct
	\\

danyās
	& such.\Parg{}
	& 2
	& 3.39\pct
	\\

edaley
	& this.\PargI{}
	& 2
	& 3.39\pct
	\\

danyaley
	& such.\PargI{}
	& 2
	& 3.39\pct
	\\

\midrule

adayam
	& that.\Dat{}
	& 3
	& 5.08\pct
	\\

\midrule

edanyana
	& this.\Gen{}
	& 1
	& 1.69\pct
	\\

adanyana
	& that.\Gen{}
	& 2
	& 3.39\pct
	\\

danyana
	& such.\Gen{}
	& 1
	& 1.69\pct
	\\

\midrule

adanyaya
	& that.\Loc{}
	& 1
	& 1.69\pct
	\\

\bottomrule

\textup{Total}
	& 
	& 59
	& 100\pct
	\\

\bottomrule
\end{tabu}
\label{tab:detprontokenfq}
\end{table}

Of all the cases, the agent demonstratives have the highest token frequency at
a combined 52.5\pct{}, especially the distal pronouns are very frequent in the
sample. Moreover, the distal inanimate agent demonstative occurs twice as often
as its animate counterpart, the shortened form \xayr{AdreNF}{adareng}{that
(one)} being far more current than the full form \rayr{AdnYreNF}{adanyareng}.
Interestingly, the shortened form \xayr{EdreNF}{edareng}{this one} is also the
only one attested for the inanimate proximate agent; similarly, the only dative
demonstrative attested once is shortened as well: \xayr{AdymF}{adayam}{(to/for)
that}. For non-core cases, only `long' demonstratives are attested, albeit
sparingly so.

Regarding the variation between `long' and `short' forms, it is not surprising
that those demonstratives with a high frequency of use are eroded in some way:
it seems that Ayeri prefers them to stay trisyllabic, which is achieved by
dropping the \rayr{nY}{nya} part.\footnote{According to the so-called Zipf's
law, word length and token frequency correlate in that the most frequently used
words in a language also tend to be the shortest \citep[25--27]{zipf1935}.} A
further reason for dropping the \rayr{nY}{nya} part especially in the inanimate
demonstratives may be that it is perceived as a marker of animacy---it has been
noted above already that it is related to the word \xayr{nYaanF}{nyān}{person}.
Both factors, high frequency and semantic mismatch, may thus encourage
contraction. Still, the question of high frequency especially of 
\rayr{AdreNF}{adareng} remains. It may be explained by looking at a few 
typical examples of this word in context, however; see (\ref{ex:demexpl}).

\begin{figure}[h]
\pex[glspace=0.5em]\label{ex:demexpl}
\a\begingl
	\gla Nay ang @ nelyo-ikan sungkorankihas, adareng tono. //
	\glb nay ang= nel-yo=ikan sungkorankihas ada-reng tono //
	\glc and \AgtT{}= help-\TsgN{}=much geography that-\AargI{} certain //
	\glft `And geography, that's for sure, helped me a lot.'%
		\tc{\citep[13]{benung:petitprince}} //
\endgl

\a\begingl
	\gla Adareng merambay-ikan, le @ sundalvāng sasān {vana ...} //
	\glb ada-reng merambay=ikan le= sundal=vāng sasān-Ø {vana ...} //
	\glc that-\AargI{} useful=very \PatTI{}= lose=\Ssg{}.\Aarg{} way-\Top{} 
		{\Ssg{}.\Gen{} ...} //
	\glft `It’s very useful if you get lost [...]'%
		\tc{\citep[14]{benung:petitprince}} //
\endgl

\a\begingl
	\gla Adareng danyaley segasena boa tinka. //
	\glb ada-reng danya-ley segas-ena boa tinka //
	\glc that-\AargI{} such-\PargI{} snake-\Gen{} boa closed //
	\glft `The one of the closed boa snake.'\footnotemark%
		\tc{\citep[22]{benung:petitprince}} //
\endgl
\xe
\end{figure}

\footnotetext{More literal translations of this sentence are `That is the one 
of the closed boa snake' or `That is one of a closed boa snake'.}

In all of the example sentences in (\ref{ex:demexpl}),
\xayr{AdreNF}{adareng}{that (one)} serves as a dummy pronoun together with a
predicative adjective or NP, which is the main reason why it occurs so
frequently. This is to say, Ayeri prefers the demonstrative pronoun
\rayr{AdreNF}{adareng} as the dummy agent in predicative contexts over the
personal pronoun \xayr{reNF}{reng}{it}. Otherwise, however, demonstrative
pronouns work regularly as deictic anaphora: `this', `that', and `such (a)',
except that as nominal elements they are declined for case---but not for number
or animacy, which is a notable difference between demonstrative pronouns and
personal pronouns. Example (\ref{ex:demproanaph1}) illustrates the use of the
indefinite demonstrative pronoun, \xayr{dnY}{danya}{(such) one} in reference to
the singular NP \xayr{nNaasF}{nangās}{house}; (\ref{ex:demproanaph2}) gives an
example of a demonstrative pronoun in an oblique case, \xayr{AdnYri}{adanyari}
{due to that}, with reference to the plural NP
\rayr{Ed/migorjye}{eda-migorayye}{these flowers}. In the latter example, the
pronoun does not inflect for its antecedent's \textsc{\Num{}ber} feature.

\begin{figure}[h]
\pex\label{ex:demproanaph1}
\a\begingl
	\gla Ang @ vehya {} @ Ajān nangās. //
	\glb ang= veh-ya Ø= Ajān nanga-as //
	\glc \AgtT{}= build-\TsgM{} \Top{}= Ajān house-\Parg{} //
	\glft `Ajān builds a house.' //
\endgl

\a\begingl
	\gla Nangās? Sa @ vehyāng may danya. //
	\glb nanga-as sa= veh=yāng may danya-Ø //
	\glc house-\Parg{} \PatT{}= build=\TsgM{}.\Aarg{} \Aff{} such-\Top{} //
	\glft `A house? He builds one indeed.' //
\endgl

\xe
\end{figure}

\begin{figure}[h]
\pex\label{ex:demproanaph2}
\a\begingl
	\gla Sā @ hasuyeng eda-migorayye. //
	\glb sā= hasu=yeng eda=migoray-ye-Ø //
	\glc \CauT{}= sneeze=\TsgF{}.\Aarg{} this=flower-\Pl{}-\Top{} //
	\glft `These flowers make her sneeze.' //
\endgl

\a\begingl
	\gla Ang @ tipinyon nivaye yena adanyari naynay. //
	\glb ang= tipin-yon niva-ye-Ø yena adanya-ri naynay //
	\glc \AgtT{}= itch-\TplN{} eye-\Pl{}-\Top{} \TsgF{}.\Gen{} that-\Caus{} 
		as.well //
	\glft `Her eyes are itching due to that/them/those [the flowers] as 
		well.' //
\endgl
\xe
\end{figure}

As mentioned in the previous chapter (\autoref{nounprefixes},
p.~\pageref{nounprefixes}), the prefix \xayr{d/}{da-}{such, so} can combine
with a range of syntactic phrase types, but most notably NPs, to serve as an
indefinite demonstrative meaning `such (a)', as in (\ref{ex:danoun}).

\begin{figure}[h]
\ex\label{ex:danoun}%
\begingl
	\gla Adareng da-dipakanas. //
	\glb adareng da=dipakan-as //
	\glc that-\AargI{} such=pity-\Parg{} //
	\glft `That is such a pity.' //
\endgl
\xe
\end{figure}

\rayr{d/}{da-} can be used to express English `one' in the sense of a deictic
anaphora as well. Thus, in order to express `the \textsc{adjective} one', it
may be necessary to use the full demonstrative pronoun, \rayr{dnY}{danya},
since adjectives themselves do not decline, and Ayeri largely avoids undeclined
NPs. An example is given in (\ref{ex:danyaadj}). Also see
\autoref{subsec:uncased} above for examples of situations where nouns regularly
do not exhibit case marking. It is also possible, however, to abbreviate
\rayr{dnY}{danya} to the prefixed form \rayr{d/}{da-}, which may be
complemented by adjectives and possessive pronouns alike. The adjective or
pronoun basically forms a complex anaphora, then, which in most circumstances
can be marked for case and topic like any other nominal element, as
demonstrated in (\ref{ex:redone}).

\begin{figure}
\pex\label{ex:danyaadj}
\a\begingl
	\gla Silvyo ku-mino-ing danyāng kivo. //
	\glb silv-yo ku=mino=ing danya-ang kivo //
	\glc look-\TsgN{} like=happy=so such-\Aarg{} little //
	\glft `The little one looks so happy.' //
\endgl

\a\label{ex:danyatop}\begingl
	\gla Sa @ noyang danya tuvo. //
	\glb sa= no=yang danya-Ø tuvo //
	\glc \PatT{}= want=\Fsg{}.\Aarg{} such-\Top{} red //
	\glft `I want the red one.' //
\endgl
\xe
\end{figure}

\begin{figure}[h]
\ex\label{ex:redone}\begingl
	\gla Sa @ noyang da-tuvo. //
	\glb sa= no=yang da=tuvo.Ø //
	\glc \PatT{}= want=\Fsg{}.\Aarg{} such=red.\Top{} //
	\glft `I want the red one.' //
\endgl\xe
\end{figure}

If incorporated in this way, the adjective cannot take comparison morphology:
(\ref{ex:demadjsupl1}) is not possible since inflections cannot be appended to
clitics (if we analyze \rayr{/ENF}{-eng} and \rayr{/vaa}{-vā} as such in this
context); and the meaning of (\ref{ex:demadjsupl2}) differs from what was
intended, since the \rayr{/vaa}{-vā} clitic is appended not to the adjective
but to the composite nominal as such.

\begin{figure}[h]
\pex
\a\label{ex:demadjsupl1}\ljudge*\begingl
	\gla da-tuvo-vāley //
	\glb da=tuvo=vā-ley //
	\glc one=red=\Supl{}-\PargI{} //
	\glft \textit{Intended:} `the reddest one' //
\endgl

\a\label{ex:demadjsupl2}\ljudge\excl\begingl
	\gla da-tuvoley-vā //
	\glb da=tuvo-ley=vā //
	\glc one=red-\PargI{}=most/*\Supl{} //
	\glft `most red ones' \\
		\textit{Intended:} `the reddest one' //
\endgl
\xe
\end{figure}

\index{pronouns!demonstrative|)}

\subsection{Interrogative pronouns}
\label{subsec:interpro}
\index{pronouns!interrogative|(}

\begin{table}\centering
\caption{Interrogative pronouns}
\begin{tabu} to \linewidth {l l X}
\tableheaderfont\toprule
Pronoun
	& Literal meaning
	& Idiomatic meaning
	\\

\toprule

\fw{sinya} % \rayr{sinY}{sinya}
	& `which one' (\tayr{nyān}{person}) %\xayr{nYaanF}{nyān}{person})
	& `who', `what', `which'
	\\

\midrule

\fw{sikan} % \rayr{siknF}{sikan}
	& `how much' (\tayr{ikan}{much}) %\xayr{IknF}{ikan}{much})
	& `how much', `how many'
	\\

\fw{sikay} % \rayr{sikj}{sikay}
	& `with what' (\tayr{kayvo}{with}) %\xayr{kjvo}{kayvo}{with})
	& `how' (tool, circumstance)
	\\

\fw{simin} % \rayr{siminF}{simin}
	& `which way' (\tayr{miran}{way}) %\xayr{mirnF}{miran}{way})
	& `how' (way, procedure)
	\\

\fw{sitaday} % \rayr{sitdj}{sitaday}
	& `which time' (\tayr{taday}{time}) %\xayr{tdj}{taday}{time})
	& `when'
	\\

\fw{siyan} % \rayr{siynF}{siyan}
	& `which place' (\tayr{yano}{place}) %\xayr{yno}{yano}{place})
	& `where'
	\\

\bottomrule
\end{tabu}
\label{tab:interpro}
\end{table}

The intererrogative pronouns are all formed with \rayr{si/}{si-}, combined with
a lexical element or a case marker; \rayr{si/}{si-} is also related to the
relativizer \rayr{si}{si}. The interrogative pronouns are listed in
\autoref{tab:interpro}. All interrogative pronouns share the property that they
are placed \fw{in situ}. That is, they appear in the same position as the
phrase they stand in for, so there will not be movement of the question word to
the front as in English. Additionally, impersonal interrogative pronouns cannot
be topicalized since they also do not inflect for case, which preempts the
difference between zero-marked topicalized and overtly case-marked
untopicalized forms. This is illustrated in (\ref{ex:qprondist}).

\begin{figure}[h]
\pex\label{ex:qprondist}
\a\begingl
	\gla Sa @ petigavāng inun sikan? //
	\glb sa= petiga=vāng inun-Ø sikan //
	\glc \PatT{}= catch=\Ssg{}.\Aarg{} fish-\Top{} how.much //
	\glft `How much fish did you catch?' //
\endgl

\a\begingl
	\gla Sa-sahavāng sitaday? //
	\glb sa\til{}saha=vāng sitaday //
	\glc \Iter{}\til{}come=\Ssg{}.\Aarg{} when //
	\glft `When will you return?' //
\endgl
\xe
\end{figure}

In the table on interrogative pronouns above, \xayr{sinY}{sinya}{who, what,
which} is seperated from the other pronouns because it behaves differently.
Namely, it can be declined for all cases according to the syntactic or semantic
role of the NP it replaces, and it can also be topicalized, since the element
asked about is likely high in discourse salience; compare (\ref{ex:qprotop}).

\begin{figure}[h]
\pex\label{ex:qprotop}
\a\begingl
	\gla Ang @ yomayo sinya adaya?\footnotemark //
	\glb ang= yoma-yo sinya-Ø adaya //
	\glc \AgtT{}= exist-\TsgN{} who-\Top{} there //
	\glft `Who is there?' //
\endgl

\a\begingl
	\gla Sa @ narayeng sinya? //
	\glb sa= nara=yeng sinya-Ø //
	\glc \PatT{}= say=\TsgF{}.\Aarg{} what-\Top{} //
	\glft `What did she say?' //
\endgl
\xe
\end{figure}

\footnotetext{This may be shortened to just \xayr{sinYaaNF Ady?}{sinyāng 
adaya?}{who (is) there?} (who-\Aarg{} there).}

\begin{table}[tp]\centering
\caption{Declension paradigm for \xayr{sinY}{sinya}{who, what}}
\begin{tabu} to \linewidth {l l X}
\tableheaderfont\toprule
Case
	& Pronoun
	& Translation
	\\

\toprule

\Top{}
	& \fw{sinya} % \rayr{sinY}{sinya}
	& `who', `what'
	\\

\midrule

\Aarg{}
	& \fw{sinyāng} % \rayr{sinYaaNF}{sinyāng}
	& `who', `what'
	\\

\AargI{}
	& \fw{sinyareng} % \rayr{sinYreNF}{sinyareng}
	& `who', `what'
	\\
\Parg{}
	& \fw{sinyās} % \rayr{sinYaasF}{sinyās}
	& `whom', `what'
	\\
\PargI{}
	& \fw{sinyaley} % \rayr{sinYlej}{sinyaley}
	& `whom', `what'
	\\
\Dat{}
	& \fw{sinyayam} % \rayr{sinYymF}{sinyayam}
	& `for/to whom', `for/to what'
	\\

\midrule

\Gen{}
	& \fw{sinyana} % \rayr{sinYn}{sinyana}
	& `whose', `from whom', `from what'
	\\

\Loc{}
	& \fw{sinyaya} % \rayr{sinYy}{sinyaya}
	& `in/at/on whom', `in/at/on what'
	\\

\Caus{}
	& \fw{sinyisa} % \rayr{sinYis}{sinyisa}
	& `due to/because of whom', `due to/because of what'
	\\

\Ins{}
	& \fw{sinyari} % \rayr{sinYri}{sinyari}
	& `by whose help', `with what'
	\\

\bottomrule
\end{tabu}
\label{tab:sinya}
\end{table}

Ayeri does not strictly distinguish animate from inanimate referents in its
interrogative pronouns, so there is no distinction between `who' and `what'.
\rayr{sinY}{sinya} and/or the verb will instead inflect according to context
and to the speaker's expectations or knowledge (compare \autoref{tab:sinya}).
Thus, there is also no dedicated question word for `why', since in Ayeri one
can simply ask `due to what/whom' by inflecting \rayr{sinY}{sinya};
\rayr{sinYis}{sinyisa} is \rayr{sinY}{sinya} marked for causative case by the
suffix \rayr{/Is}{-isa}. Declension of \rayr{sinY}{sinya} for different
purposes is shown in (\ref{ex:sinyacase}).

\begin{figure}[h]
\pex\label{ex:sinyacase}
\a\begingl
	\gla Le @ kayāng adanya sinyayam? //
	\glb le= ka=yāng adanya-Ø sinya-yam //
	\glc \PatTI{}= throw.away=\TsgM{}.\Aarg{} that-\Top{} what-\Dat{} //
	\glft `Why (= what for) did he throw that away?' //
\endgl

\a\begingl
	\gla Ang @ prantoyva sinyisa? //
	\glb ang= prant-oy=va.Ø sinya-isa //
	\glc \AgtT{}= ask-\Neg{}=\Ssg{}.\Top{} what-\Caus{} //
	\glft `Why (= because of what) did you not ask?' //
\endgl
\xe
\end{figure}

While there is no single, dedicated word for `why', Ayeri distinguishes between
two kinds of `how': \rayr{siminF}{simin}, on the one hand, asks about the way
by which---or the circumstances under which---an action is carried out, see
(\ref{ex:simin}). \rayr{sikj}{sikay}, on the other hand, asks for the means or
tools used to carry out an action, see (\ref{ex:sikay}). Thus, the correct
answer to the question in (\ref{ex:simin}) needs to treat the process of making
bread, since \rayr{siminF}{simin} asks about the way of doing something; a
correct answer to the question in (\ref{ex:sikay}), on the other hand, will
likely mention grinding utensils, like a mill or a pestle.

\begin{figure}[h]
\pex
\a\label{ex:simin}\begingl
	\gla Le @ tiyavāng vadisān simin? //
	\glb le= tiya=vāng vadisān-Ø simin //
	\glc \PatTI{}= make=\Ssg{}.\Aarg{} bread-\Top{} how //
	\glft `How do you make bread?' //
\endgl

\a\label{ex:sikay}\begingl
	\gla Le @ peralvāng sagan sikay? //
	\glb le= peral=vāng sagan-Ø sikay //
	\glc \PatTI{}= grind=\Ssg{}.\Aarg{} flour-\Top{} how //
	\glft `How do you grind flour?' //
\endgl
\xe
\end{figure}

Comparing Tables \ref{tab:interpro} and \ref{tab:sinya}, strikingly, there are
two possbilities to express `where'---lexical \rayr{siynF}{siyan} and synthetic
\rayr{sinYy}{sinyaya}. These, however, are not strictly interchangable, even
though some variation is to be expected. While \rayr{siynF}{siyan} refers to
\emph{places} in general, the \rayr{sinY}{sinya} series refers to
\emph{discourse participants} both animate and inanimate more specifically, as
shown in (\ref{ex:siyansinya}).

\begin{figure}[h]
\pex\label{ex:siyansinya}
\a\begingl
	\gla Saravāng siyan? --- Ya @ Sikatay. //
	\glb sara=vāng siyan --- ya= Sikatay //
	\glc go=\Ssg{}.\Aarg{} where --- \Loc{}= Sikatay //
	\glft `\,\enquote{Where are you going?}---\enquote{To Sikatay.}\,' //
\endgl

\a\begingl
	\gla Ya @ divvāng sinya? --- Ya @ Haki. //
	\glb ya= div=vāng sinya-Ø --- ya= Haki //
	\glc \LocT{}= stay=\Ssg{}.\Aarg{} who-\Top{} --- \Loc{}= Haki //
	\glft `\,\enquote{Who are you staying with?}---\enquote{At Haki's}\,' //
\endgl
\xe
\end{figure}

\index{pronouns!interrogative|)}

\subsection{Indefinite pronouns}
\label{subsec:indefpro}
\index{pronouns!indefinite|(}

\citet[56]{haspelmath1997} notes how descriptions of languages often do not
document indefinite pronouns---whether they simply do not exist in this
language or whether they escaped the author's attention remains unknown in
these cases. It may thus be duly noted here that Ayeri does indeed possess
indefinite pronouns.\footnote{Since it is an invented language, the value of
this assertion to linguistic typology remains doubtful, however.} In order to
classify languages, \citet{haspelmath1997} generalizes the map displayed in
\autoref{fig:haspeltab} based on a sample of 100 languages from all continents,
although he notes that this sample has a European bias due to the availability
of data \citep[2]{haspelmath1997}. Languages typically form continguous areas
on the map, even though they may carve it up quite differently, and with
overlaps between the different semantic groupings 1--9.

An interesting question that \citet{haspelmath1997} poses towards the end of
his book is whether there are any correlations between word order typology and
the preference for generic nouns (`person', `thing', `place', `time', `manner')
or, for instance, interrogative-based systems \citep[239--241]{haspelmath1997}.
From \citet{haspelmath1997}'s concluding statistics it looks as though there is
a slight preference of languages with which Ayeri shares basic typological
traits---such as verb-initial, verb--object, and noun--genitive word order,
also having prepositions---for basing indefinite pronouns on generic nouns.
\citet{haspelmath1997} concedes that these seeming correlations are skewed by
areal effects, \textcquote[241]{haspelmath1997}{because indefinite pronouns
have a strongly areal distribution}.\footnote{The map in \citetitle{wals}
\citep{wals46A} suggests areal clusters at least for generic-noun based systems
in Africa and Southeast Asia. \citetitle{wals} classifies 60\pct{} of the
sampled languages as possessing interrogative-based indefinite pronouns, with
evidence for this type quoted for all continents except Africa. The next
smaller group, generic-noun based, falls behind at 26\pct. The lack of evidence
for the interrogative type in Africa despite being the most frequent one in the
set may be due to the unavailability of data. Crossreferencing
constituent-order and indefinite-pronoun systems did not yield a result which
obviously suggested a correlation.} He still presumes, however, that word-order
typology may have an effect on the formation of indefinites insofar as it
correlates with grammaticalization more generally \citep[239]{haspelmath1997}.

\begin{figure}[tp]\centering
\scalebox{.9}{%
\begin{tikzpicture}[x=5em]
\node (1) at (1,3) {(1)};
\node (2) at (2,3) {(2)};
\node (3) at (3,3) {(3)};
\node (4) at (4,4) {(4)};
\node (5) at (4,2) {(5)};
\node (6) at (5,4) {(6)};
\node (7) at (6,5) {(7)};
\node (8) at (5,2) {(8)};
\node (9) at (6,1) {(9)};
%
\draw (1) -- (2);
\draw (2) -- (3);
\draw (3) -- (4);
\draw (3) -- (5);
\draw (4) -- (5);
\draw (4) -- (6);
\draw (5) -- (8);
\draw (6) -- (7);
\draw (6) -- (8);
\draw (8) -- (9);
%
\node[haspanno]            (l1) at (1) {specific known};
\node[haspanno]            (l2) at (2) {specific unknown};
\node[haspanno]            (l3) at (3) {irrealis non-specific};
\node[haspanno, above=2ex] (l4) at (4) {question};
\node[haspanno]            (l5) at (5) {conditional};
\node[haspanno, above=2ex] (l6) at (6) {indirect negation};
\node[haspanno]            (l7) at (7) {direct negation};
\node[haspanno]            (l8) at (8) {comparative};
\node[haspanno]            (l9) at (9) {free choice};
\end{tikzpicture}
}
\caption[The implicational map for indefinite pronoun functions]{The 
implicational map for indefinite pronoun functions \citep[4]{haspelmath1997}}
\label{fig:haspeltab}
\end{figure}

\begin{table}\centering
\caption{Indefinite pronouns}

\begin{tabu} to \linewidth {C[2] X[3c] X[3c] X[3c]}
\toprule\tableheaderfont

Property
	& every
	& some
	& none
	\\
\toprule
	
person
	& enya % every/any
	& arilinya % some
	& ranya % none
	\\
	
thing
	& enya % every/any
	& arilinya, arilya % some
	& ranya % none
	\\
\midrule
	
place
	& yanen % every/any
	& yāril % some
	& yanoy % none
	\\
\midrule
	
time
	& tadayen % every/any
	& tajaril; metay % some
	& tadoy; jānyam % none
	\\
\midrule
	
manner
	& arēn % every/any
	& miranaril % some
	& aremoy % none
	\\
	
\midrule

reason
	& --- % every/any
	& yāril % some
	& --- % none
	\\

\bottomrule

\end{tabu}

\label{tab:indeftab}
\end{table}

\citet{haspelmath1997} mentions generic nouns, and these can be combined with
the quantifying expressions `every', `any', `some', and `none' into an array
like the one presented in \autoref{tab:indeftab}. Ayeri does not distinguish
`every' from `any' as English does; there is also no distinction in polarity
(affirmative versus negative) the way English has it. See (\ref{ex:englpol})
for an example.

\begin{figure}[h]
\pex\label{ex:englpol}%
	English:
	\a\ljudge* \fw{I don't know something about this.}
	\a \fw{I don't know anything about this.}
\xe
\end{figure}

Likewise, Ayeri does not distinguish between animate and inanimate indefinite 
referents. The same pronouns are used for either, although the shortening of 
\rayr{ArilinY}{arilinya}, \rayr{ArilY}{arilya}, can only be used for 
inanimates, similar to the distinction in the demonstrative pronouns between 
\xayr{AdnYaaNF}{adanyāng}{that one} (that.one-\Aarg{}) and 
\xayr{AdnYreNF}{adareng}{that one} (that.one-\AargI{}; see 
\autoref{subsec:dempro}). Two further features stand out, however.

\phantomsection\label{indefprocomp}
Firstly, most of the pronouns in the chart have a lexical part---Ayeri's
indefinite pronouns are based on generic nouns. Thus, the pronouns referring to
people and things all have the \rayr{/nY}{-nya} element in common, which we
also find in the interrogative and demonstrative pronouns, and which also
appears in the word \xayr{nYaanF}{nyān}{person}. In the same way, the pronouns
related to the notion of place have a \rayr{y/}{ya-} or \rayr{ynF/}{yan-} part,
which we also find in \xayr{yno}{yano}{place}.\footnote{\rayr{yno}{yano} itself
is an old nominalization and very likely related as a morpheme to the locative
suffix \rayr{/y}{-ya}.} In a regular continuation of this pattern, the
indefinite pronouns of time all have an element related to
\xayr{tdj}{taday}{time} in common, which is obscured somewhat by palatalization
in \rayr{tdYrilF}{tajaril}. The exception to this series, then, is
\rayr{dYaanFymF}{jānyam}, which is the multiplicative numeral formed from
\xayr{dY}{ja}{zero}, thus means `zero times' or `not once' rather than 
`never', although it can also be used emphatically for the latter. The series
of manner pronouns is an absolute exception in that it must be a residue from
an older layer of grammaticalization since \rayr{Are/}{are-} is not a
recognizable morpheme in the modern language.\footnote{I probably made this up
as I was going, many years ago, and without considering systematic
implications, since I was unaware of them at the time.} \rayr{mirnrilF}
{miranaril} is a regular formation of \xayr{mirnF}{miran}{way, manner} combined
with the quantifier (!) for indefinite amounts, \xayr{/ArilF}{-aril}{some}.

This observation leads to the second regular feature, that is, affixes as 
modifiers to generic nouns. The `every' series regularly features the 
morpheme \rayr{EnF}{en}, either prefixed or suffixed, which is related to the 
quantifier \xayr{/henF}{-hen}{every, all, each} and can presumably be found 
even on \rayr{AreenF}{arēn} in spite of its obscure lexical base. In the same 
manner, the series related to inspecific generic-noun referents is marked by 
the affix \rayr{ArilF}{aril} which, as we have just seen above, is otherwise 
used to refer to inspecific quantities, for instance, 
\xayr{vdiːsnF/ArilF}{vadisān-aril}{some bread} (bread=some). In the case of 
\rayr{mirnrilF}{miranaril}, the suffix seems somewhat of an 
odd choice, since manner is not a quantifiable variable in the same way people,
things, locations, or moments are. Possibly, it is chosen rather in analogy
with the other pronouns in this series than on semantic grounds. In any event,
\rayr{metj}{metay} has the semantically more `proper' \rayr{me/}{me-} prefix,
relating it to absolute inspecificity.\footnote{Compare German
\fw{irgendjemand} and French \fw{n'importe qui} `no matter who'.} This 
alternation is employed to distinguish between the meaning of `sometime', that 
is, occurring once at an unspecified point in time, and 
\xayr{tdYrilF}{tajaril}{sometimes}, which refers to repeated occurrence at
inspecific times. The alternation between \rayr{mirnrilF}{miranaril} and
regularly derived \rayr{me/mirnF}{mə-miran} can be leveraged to express a
specificity difference as well. While the former suggests that an action is
carried out or an event is happening by means of a specific, though unknown
procedure, the latter suggests just any possible procedure. Lastly, the
negative series is reguarly marked by the negative suffix \rayr{/Oj}{-oy},
which also occurs with adjectives and verbs (see sections \ref{subsec:adjneg}
and \ref{subsubsec:verbneg}). An outlier in this series is the 
person/thing-related indefinite pronoun, \rayr{rnY}{ranya}. The etymologic 
connections of the \rayr{r}{ra} part are not presently known, perhaps the 
postposition \xayr{rnF}{ran}{against} is related.

The chart in \autoref{tab:indeftab} only tells half the truth by not giving any
information on use contexts for the individual forms, so how do they fit in
with the chart from \citet{haspelmath1997} quoted at the beginning of this
section? Regarding the functions of indefinite pronouns annotated to the
numbers on the map, \citet{haspelmath1997} gives the example sentences in
(\ref{ex:indeftypo}, which, however, mostly only give one example for either
the `person' or `thing' category at a time. It is up to the reader to
generalize from this \citep[2--3]{haspelmath1997}.\footnote{These appear here
reordered according to numerical order. The book lists them according to their
logical order as tracing the map, the enumeration somewhat confusingly tied in
with the running enumeration of examples.}

\begin{figure}[h]
\pex[labeltype=numeric]\label{ex:indeftypo}
\a specific, known to the speaker: \smallskip\\ % 1
	\textit{\underline{Somebody} called while you were away: guess who!}
	
\a specific, unknown to the speaker: \smallskip\\ % 2
	\textit{I heard \underline{something}, but I couldn't tell what kind of 
	sound it was.}
	
\a non-specific, irrealis: \smallskip\\ % 3
	\textit{Please try \underline{somewhere} else.}
	
\a polar question: \smallskip\\ % 4
	\textit{Did \underline{anybody} tell you anything about it?}
	
\a conditional protasis: \smallskip\\ % 5
	\textit{If you see \underline{anything}, tell me immediately.}
	
\a indirect negation: \smallskip\\ % 6
	\textit{I don't think that \underline{anybody} knows the answer.}
	
\a direct negation: \smallskip\\ % 7
	\textit{\underline{Nobody} knows the answer.}
	
\a standard of comparison: \smallskip\\ % 8
	\textit{In Freiburg the weather is nicer than \underline{anywhere} in 
	Germany.}
	
\a free choice: \smallskip\\ % 9
	\textit{\underline{Anybody} can solve this simple problem.}
\xe
\end{figure}

\begin{figure}\centering
\scalebox{.9}{%
\begin{tikzpicture}[x=5em]
\node (1) at (1,3) {(1)};
\node (2) at (2,3) {(2)};
\node (3) at (3,3) {(3)};
\node (4) at (4,4) {(4)};
\node (5) at (4,2) {(5)};
\node (6) at (5,4) {(6)};
\node (7) at (6,5) {(7)};
\node (8) at (5,2) {(8)};
\node (9) at (6,1) {(9)};
%
\draw (1) -- (2);
\draw (2) -- (3);
\draw (3) -- (4);
\draw (3) -- (5);
\draw (4) -- (5);
\draw (4) -- (6);
\draw (5) -- (8);
\draw (6) -- (7);
\draw (6) -- (8);
\draw (8) -- (9);
%
\node[haspanno]            (l1) at (1) {specific known};
\node[haspanno]            (l2) at (2) {specific unknown};
\node[haspanno]            (l3) at (3) {irrealis non-specific};
\node[haspanno, above=2ex] (l4) at (4) {question};
\node[haspanno]            (l5) at (5) {conditional};
\node[haspanno, above=2ex] (l6) at (6) {indirect negation};
\node[haspanno]            (l7) at (7) {direct negation};
\node[haspanno]            (l8) at (8) {comparative};
\node[haspanno]            (l9) at (9) {free choice};
%
\draw[semithick] (0.5,5.0) -- (3.0,5.0) -- (4.0,5.0) -- (5.5,5.0) 
-- (5.5,3.0) -- (4.5,3.0) -- (4.5,1.0) -- (4.0,1.0) -- (3.0,1.0) -- (0.5,1.0) 
-- (0.5,5.0);
%
\draw[dotted, semithick] (2.75,3.5) -- (3.0,3.5) -- (4.0,4.5) -- (4.25,4.5) 
-- (4.25,1.5) -- (4.0,1.5) -- (3.0,2.5) -- (2.75,2.5) 
-- (2.75,3.5);
%
\node[draw, loosely dotted, semithick, fit=(1) (2)] {};
%
\node[draw, dashed, semithick, fit=(6) (7)] {};
%
\node[draw, dash dot, semithick, fit=(8) (9)] {};
%
\node[haspanno2, below=.25em, anchor=north west,] at (0.50,1.00)
	{↑ `some' series};
\node[haspanno2, above=.75em, anchor=south west,] at (0.75,3.25)
	{plain generic nouns ↓};
\node[haspanno2, above=.25em, anchor=south west,] at (2.75,3.50) {mə- ↓};
\node[haspanno2, below=.25em, anchor=north west,] at (5.50,3.50)
	{↑ `none' series};
\node[haspanno2, below=.25em, anchor=north west,] at (4.75,0.5)
	{↑ `every' series};
\end{tikzpicture}
}
\caption[Map of indefinite pronoun functions in Ayeri]{Map of indefinite 
pronoun functions in Ayeri}
\label{fig:haspeltabayr}
\end{figure}

As we have seen in \autoref{tab:indeftab} above, Ayeri does not make a
difference between `every' and `any', which is why the `some' series can be
applied to all of (1)--(5); it can also be used for indirect negation (6). The
pronouns from the `none' column, then, are used to express direct negation (7).
Since double negation---that is, agreement in negation between verbs and
indefinite pronouns for purposes of emphasis rather than double negation in the
strictly logical sense---is possible, the `none' series may also be employed
for indirect negation (6). Moreover, Ayeri uses the `every' series for both
standard of comparison (8) and free choice (9). Besides this, 
absolute-indefinite \rayr{me/}{me-} can be used for (3)--(6) in combination 
with a (generic) noun to attach to. It needs to be noted that only the 
indefinite pronouns with person or thing reference (those including 
\rayr{nY}{nya}) decline; they can also be topicalized. The other indefinites, 
relating to place, time and manner, are indeclinable and also cannot be topics 
for this reason.
%
% Off the top of my head I'm not sure if I've done this before or if it is a 
% new rule I made up here for the purpose of spicing things up a little:
%
For the `specific' categories (1) and (2) it is furthermore possible to use the
plain generic nouns, \xayr{nYaanF}{nyān}{person}, \xayr{linY}{linya}{thing},
\xayr{yno}{yano}{place}, \xayr{tdj}{taday}{time}, \xayr{mirnF}{miran}{way}, 
however. \autoref{fig:haspeltabayr} shows the groupings for Ayeri; 
(\ref{ex:indefex}) gives examples of all types.

\needspace{3\baselineskip}
\pex[labeltype=numeric,interpartskip=1em]\label{ex:indefex}
\a specific, known to the speaker:\vspace{.5em} % 1
	\beginsubsub
	\b{a.} \begingl
		\gla Ang @ sahaya \textbf{arilinya}, leku, sinyāng adaley! //
		\glb ang= saha-ya arilinya-Ø lek-u sinya-ang ada-ley //
		\glc \AgtT{}= come-\TsgM{} someone-\Top{} guess-\Imp{} 
			who-\Aarg{} that-\PargI{} //
		\glft `Someone came, guess who it is!' //
		\endgl\vspace{.5em}
		
	\b{b.} \begingl
		\gla Le @ ilta ningyang \textbf{linya} vayam. //
		\glb le= ilta ning=yang linya-Ø vayam //
		\glc \PatTI{}= need tell=\Fsg{}.\Aarg{} thing-\Top{} 
			\Ssg{}.\Dat{} //
		\glft `I need to tell you something.' //
		\endgl
	\endsubsub

\needspace{3\baselineskip}
\a specific, unknown to the speaker:\vspace{.5em} % 2
	\beginsubsub
	\b{a.} \begingl
		\gla Ang @ pegaya \textbf{arilinya} pangisley nā. //
		\glb ang= pega-ya arilinya-Ø pangis-ley nā //
		\glc \AgtT{}= steal-\TsgM{} someone-\Top{} money-\PargI{} 
			\Fsg{}.\Gen{} //
		\glft `Someone stole my money.' //
		\endgl\vspace{.5em}
		
	\b{b.} \begingl
		\gla Ang @ sarayan \textbf{yanoya} agon. //
		\glb ang= sara=yan yano-ya agon //
		\glc \AgtT{}= go=\TplM{}.\Top{} place-\Loc{} foreign //
		\glft `They are going somewhere foreign.' //
		\endgl
	\endsubsub

\needspace{3\baselineskip}	
\a non-specific, irrealis:\vspace{.5em} % 3
	\beginsubsub
	\b{a.} \begingl
		\gla Pinyan, prantu \textbf{yāril} palung. //
		\glb pinyan prant-u yāril palung //
		\glc please ask-\Imp{} somewhere different //
		\glft `Please ask somewhere else.' //
		\endgl\vspace{.5em}
		
	\b{b.} \begingl
		\gla Le @ ilta @ miranang adanya \textbf{mə-}miraneri 
			palung. //
		\glb le= ilta= mira=nang adanya-Ø mə-miran-eri palung //
		\glc \PatTI{}= need= do=\Fsg{}.\Aarg{} that.one-\Top{} 
			some-way-\Ins{} different //
		\glft `We need to do that in some other way.' //
		\endgl
	\endsubsub

\needspace{3\baselineskip}	
\a polar question:\vspace{.5em} % 4
	\beginsubsub
	\b{a.} \begingl
		\gla Ang @ koronva \textbf{arilinyaley} edanyana? //
		\glb ang= koron=va.Ø arilinya-ley edanya-na //
		\glc \AgtT{}= know=\Ssg{}.\Top{} something-\PargI{} 
			this.one-\Gen{} //
		\glft `Do you know anything about this?' //
		\endgl\vspace{.5em}
		
	\b{b.} \begingl
		\gla Yomaya \textbf{mə-}nyānang si ang @ vaca mirongya 
			edanyaley? //
		\glb yoma-ya mə-nyān-ang si ang= vaca mira-ong=ya.Ø
			edanya-ley //
		\glc exist-\TsgM{} some-person-\Aarg{} \Rel{} \AgtT{}= 
			like do-\Irr{}=\TsgM{}.\Top{} this-\PargI{} //
		\glft `Is there \emph{anyone} who would like to do this?' //
		\endgl
	\endsubsub

\needspace{3\baselineskip}
\a conditional protasis:\vspace{.5em} % 5
	\beginsubsub
	\b{a.} \begingl
		\gla Ang @ ming pengalayn sitanyās \textbf{yāril}, adareng 
			pray-ven. //
		\glb ang= ming pengal=ayn.Ø sitanya-as yāril ada-reng 
			pray=ven //
		\glc \AgtT{}= can meet-\Fpl{}.\Top{} each.other-\Parg{} 
			somewhere that-\AargI{} great=pretty //
		\glft `If we can meet somewhere that would be pretty great.' //
		\endgl\vspace{.5em}
		
	\b{b.} \begingl
		\gla Sa @ na-naravāng \textbf{mə-}lentan, ang @ haray vās! //
		\glb sa= na\til{}nara=vāng mə-lentan-Ø ang= har=ay.Ø vās //
		\glc \PatT{}= \Iter{}\til{}say=\Ssg{}.\Aarg{} some-sound-\Top{} 
			\AgtT{}= punch-\Fsg{}.\Top{} \Ssg{}.\Parg{} //
		\glft `You make any more sound, I'm gonna punch you!' //
		\endgl
	\endsubsub

\needspace{3\baselineskip}	
\a indirect negation:\vspace{.5em} % 6
	\beginsubsub
	\b{a.} \begingl
		\gla Paronoyyang, ang @ no @ tahaya \textbf{arilinya} adaley. //
		\glb paron-oy=yang ang= no= taha-ya arilinya-Ø ada-ley //
		\glc believe-\Neg{}=\Fsg{}.\Aarg{} \AgtT{}= want= 
			have-\Tsg{}.\M{} anyone-\Top{} that-\PargI{} //
		\glft `I don't think anyone wants to have that.' //
		\endgl\vspace{.5em}
	
	\b{b.} \begingl
		\gla Paronoyyang, le @ ming @ sungvāng adanya \textbf{yanoy}. //
		\glb paron-oy=yang le= ming= sung=vāng adanya-Ø yanoy //
		\glc believe-\Neg{}=\Fsg{}.\Aarg{} \PatTI{}= can= 
			find=\Ssg{}.\Aarg{} that.one-\Top{} nowhere //
		\glft `I don't think you can find that \emph{anywhere}.' //
		\endgl
	\endsubsub

\needspace{3\baselineskip}	
\a direct negation:\vspace{.5em} % 7
	\beginsubsub
	\b{a.} \begingl
		\gla Ang @ koronya \textbf{ranya} guratanley. //
		\glb ang= koron-ya ranya-Ø guratan-ley //
		\glc \AgtT{}= know-\TsgM{} nobody-\Top{} answer-\PargI{} //
		\glft `Nobody knows the answer.' //
		\endgl\vspace{.5em}
		
	\b{b.} \begingl
		\gla Le @ ming @ sungvāng adanya \textbf{yanoy}. //
		\glb le= ming= sung=vāng adanya-Ø yanoy //
		\glc \PatTI{}= can= find=\Ssg{}.\Aarg{} that.one-\Top{} nowhere //
		\glft `You can't find that anywhere.' //
		\endgl
	\endsubsub

\needspace{3\baselineskip}	
\a standard of comparison:\vspace{.5em} % 8
	\beginsubsub
	\b{a.} \begingl
		\gla Sa @ engyeng larau \textbf{enya} palung. //
		\glb sa= eng=yeng larau enya-Ø palung //
		\glc \PatT{}= be.more=\TsgF{}.\Aarg{} nice anyone different //
		\glft `She is nicer than anyone else.' //
		\endgl\vspace{.5em}
		
	\b{b.} \begingl
		\gla Ang @ engyo ban eda-riman \textbf{yanen} palung. //
		\glb ang= eng-yo ban eda=riman-Ø yanen palung //
		\glc \AgtT{}= be.more-\TsgN{} good this=city-\Top{} anywhere 
			different //
		\glft `This city is better than anywhere else.' //
		\endgl
	\endsubsub

\needspace{3\baselineskip}
\a free choice:\vspace{.5em} % 9
	\beginsubsub
	\b{a.} \begingl
		\gla Ang @ ming @ guraca \textbf{enya} eda-prantanley. //
		\glb ang= ming= gurat-ya enya-Ø eda=prantan-ley //
		\glc \AgtT{}= can= answer-\TsgM{} anyone-\Top{} 
				this=question-\PargI{} //
		\glft `Anyone can answer this question.' //
		\endgl\vspace{.5em}
		
	\b{b.} \begingl
		\gla Epayeng \textbf{tadayen} si sa @ pinyaya ye ang @ Tapan. //
		\glb epa=yeng tadayen si sa= pinya-ya ye ang= Tapan //
		\glc refuse=\TsgF{}.\Aarg{} everytime \Rel{} \PatT{}=
				ask-\TsgM{} \TsgF{}.\Top{} \Aarg{}= Tapan //
		\glft `She refused everytime Tapan asked her.' //
		\endgl
	\endsubsub
	
\xe

\index{pronouns!indefinite|)}

\subsection{Relative pronouns}
\label{subsec:relpro}
\index{pronouns!relative|(}

\begin{table}[tp]\centering
\caption{Relative pronouns}

\begin{tabu} to \linewidth {l X[c] X[c] X[c] X[c] X[c] X[c]}
\tableheaderfont\toprule
Case
	& Pronoun
	& \multicolumn{5}{c}{Pronoun with secondary inflection}
	\\

\cmidrule{3-7}
	& 
	& \Dat{}
	& \Gen{}
	& \Loc{}
	& \Caus{}
	& \Ins{}
	\\
	
\toprule

Ø
	& si % Ø
	& siyām % \Dat{}
	& sinā % \Gen{}
	& siyā % \Loc{}
	& sisā % \Caus{}
	& sirī % \Ins{}
	\\

\midrule

\Aarg{}
	& sang % Ø
	& sangyam % \Dat{}
	& sangena % \Gen{}
	& sangya % \Loc{}
	& sangisa % \Caus{}
	& sangeri % \Ins{}
	\\

\Aarg{}.\Inan{}
	& sireng % Ø
	& sirengyam % \Dat{}
	& sirengena % \Gen{}
	& sirengya % \Loc{}
	& sirengisa % \Caus{}
	& sirengeri % \Ins{}
	\\
	
\Parg{}
	& sas % Ø
	& sasyam % \Dat{}
	& sasena % \Gen{}
	& sasya % \Loc{}
	& sasisa % \Caus{}
	& saseri % \Ins{}
	\\

\Parg{}.\Inan{}
	& siley % Ø
	& sileyyam % \Dat{}
	& sileyena % \Gen{}
	& sileyya % \Loc{}
	& sileyisa % \Caus{}
	& sileyeri % \Ins{}
	\\

\Dat{}
	& siyam % Ø
	& siyamyam % \Dat{}
	& siyamena % \Gen{}
	& siyamya % \Loc{}
	& siyamisa % \Caus{}
	& siyameri % \Ins{}
	\\

\midrule

\Gen{}
	& sina/sena % Ø
	& sinayam % \Dat{}
	& sinana % \Gen{}
	& sinaya % \Loc{}
	& sinaisa % \Caus{}
	& sinari % \Ins{}
	\\
	
\Loc{}\footnotemark
	& siya % Ø
	& siyayam % \Dat{}
	& siyana % \Gen{}
	& siyaya % \Loc{}
	& siyaisa % \Caus{}
	& siyari % \Ins{}
	\\
	
\Caus{}
	& sisa % Ø
	& sisayam % \Dat{}
	& sisana % \Gen{}
	& sisaya % \Loc{}
	& sisaisa % \Caus{}
	& sisari % \Ins{}
	\\
	
\Ins{}
	& seri % Ø
	& seriyam % \Dat{}
	& serina % \Gen{}
	& seriya % \Loc{}
	& serīsa % \Caus{}
	& seriri % \Ins{}
	\\

\bottomrule
\end{tabu}
\label{tab:relpro}
\end{table}

\footnotetext{The contracted form \fw{sijya} for \rayr{siyy}{siyaya} is
attested once, compare \citet[12]{becker:kafka:imperial}. Likewise, it should
be possible for \rayr{siyymF}{siyayam} to contract to \fw{sijyam}. The native
spelling of both the long and the contracted forms would not differ, though,
since contracted \rayr{/ye} {-ye} is also still spelled that way in spite of
the difference in pronunciation.}

As described before, Ayeri connects relative clauses to main clauses with the
relativizer \rayr{si}{si}. This relativizer can be declined for case in
accordance with the relative clause's head in the matrix clause. The respective
forms can be gathered from \autoref{tab:relpro} (column `Pronoun').

\begin{figure}[h]
\pex
\a\label{ex:n-rel}\begingl
	\gla Eryyo tarela natrangās si tado. //
	\glb ery-yo tarela natranga-as si tado //
	\glc use-\TsgN{} still temple-\Parg{} \Rel{} old //
	\glft `The temple, which is old, is still being used.' //
\endgl

\a\label{ex:n-adj-rel}\begingl
	\gla Edanyāng ayonas sirtang sas ang @ sihabaya mondoas nana. //
	\glb edanya-ang ayon-as sirtang si-as ang= sihaba=ya mondo-as nana //
	\glc this-\Aarg{} man-\Parg{} young \Rel{}-\Parg{} 
		\AgtT{}= tend=\TsgM{}.\Top{} garden-\Parg{} \Fpl{}.\Gen{} //
	\glft `This is the young man who tends our garden.' //
\endgl
\xe
\end{figure}

As explained in \autoref{sec:markstrat}, if the relativizer is immediately 
following its lexical head, only the base form \rayr{si}{si} is used, which is 
illustrated in (\ref{ex:n-rel}). Here, the head of the relative clause is
\xayr{ntFrNaasF}{natrangās}{the temple}, which is immediately followed by the
relative clause. If word material is intervening, however, as in
(\ref{ex:n-adj-rel}), the relative pronoun may be inflected to agree in case
with its antecedent in more formal language for referential clarity:
\rayr{ssF}{sas} agrees in case with \rayr{AyonsF}{ayonas} two words over to the
left. Relative pronouns do not agree in number with their heads, though, and in
gender only insofar as it is relevant to nominal case inflection, that is,
agents and patients are distinguished for animacy.

A special property of the relative pronoun is that it can be declined for its 
role in the relative clause as well to express more complex relationships 
between the main clause and the relative clause. The respective forms can be 
found in the columns titled `pronoun with secondary inflection' in 
\autoref{tab:relpro}. The token frequency of the actually occurring complex 
relative pronouns in the very small corpus gathered from example texts and 
dictionary entries (see \autoref{sec:phonotactics}) is given in 
\autoref{tab:relprotokenfreq}.

\begin{table}[tp]\centering
\caption{Token frequencies of attested complex relative pronouns}

\begin{tabu} to .75\linewidth {>{\itshape}X[2l] X[2l] X[1c]}
\tableheaderfont\toprule

Pronoun & Gloss & Absolute \\

\toprule

siyā	& \Rel{}.Ø.\Loc{} & 7 \\
sirī	& \Rel{}.Ø.\Ins{} & 3 \\
sinā	& \Rel{}.Ø.\Gen{} & 1 \\
siyām	& \Rel{}.Ø.\Dat{} & 1 \\

\bottomrule

\textup{Total}	& & 12 \\

\bottomrule
\end{tabu}
\label{tab:relprotokenfreq}
\end{table}

Compared to the unmarked relativizer \rayr{si}{si}, which occurs 50 times in
the sample (out of 80), the complex relative pronouns have a very low
frequency. This is not surprising, since `for whom', `by which', etc.\ are
quite specialized expressions. It also seems that those forms unmarked for
their antecedent are preferred, since these are the only ones attested. The
sample is really much too small to make actually meaningful judgments here,
however. Complex relative pronouns are illustrated in (\ref{ex:relcompl}).
Importantly, a complex relative pronoun cannot form the topic of the relative
clause even though it is marked for case according to the relative clause's
syntactic domain. Furthermore, the relative pronoun cannot receive inflection
for an agent or a patient of the embedded clause. Compare (\ref{ex:reltop}) to
(\ref{ex:relpat}) for examples.

\begin{figure}
\pex\label{ex:relcompl}
\a\begingl[glspace=.33em]
	\gla Le @ vacyang koya sileyya ang @ layāy adanyana. //
	\glb le= vac=yang koya-Ø si-ley-ya ang= laya=ay.Ø adanya-na //
	\glc \PatTI{}= like=\Fsg{}.\Aarg{} book-\Top{} \Rel{}-\PargI{}-\Loc{}
	\AgtT{}= read=\Fsg{}.\Top{} that-\Gen{} //
	\glft `I like the book in which I read about it.' //
\endgl

\a\label{ex:reldat}\begingl
	\gla Ya @ saratang yano siyām sarasatang. //
	\glb ya= sara=tang yano-Ø si-Ø-yām sara-asa=tang //
	\glc \LocT{}= go=\TplM{}.\Aarg{} place-\Top{} \Rel{}-\Loc{}-\Dat{} 
		go-\Hab{}=\TplM{}.\Aarg{} //
	\glft `They went to the place to which they always went.' //
\endgl
\xe
\end{figure}

\begin{figure}
\ex\label{ex:reltop}
% \a\begingl
\ljudge* \begingl
	\gla Mica edaya sobayāng {si \textup{(\ques{}\textit{sī})}} na @ ihayang 
		koyaley. //
	\glb mit-ya edaya sobaya-ang si-Ø-Ø na= iha=yang koya-ley //
	\glc live-\TsgM{} here teacher-\Aarg{} \Rel{}-\Aarg{}-\Top{} \GenT{}= 
		borrow=\Fsg{}.\Aarg{} book-\PargI{} //
% \endgl
% 
% \a\begingl
% 	\gla Mica edaya sobayāng sinā ang ihāy koyaley. //
% 	\glb mit-ya edaya sobaya-ang si-Ø-nā ang iha=ay.Ø koya-ley //
% 	\glc live-\TsgM{} here teacher-\Aarg{} \Rel{}-\Aarg{}-\Gen{} \AgtT{} 
% 		borrow=\Fsg{}.\Top{} book-\PargI{} //
	\glft `Here lives the teacher from whom I borrowed a book.' //
\endgl
\xe
\end{figure}

\begin{figure}
\ex\label{ex:relagt}
% \a\begingl
\ljudge* \begingl
	\gla Mica edaya sobayāng sāng le @ sobya payutān yām. //
	\glb mit-ya edaya sobaya-ang si-Ø-ang le= sob-ya payutān-Ø yām //
	\glc live-\TsgM{} here teacher-\Aarg{} \Rel{}-\Aarg{}-\Aarg{} \PatTI{}= 
		teach-\TsgM{} math-\Top{} \Fsg{}.\Dat{} //
% \endgl
% 
% \a\begingl
% 	\gla Mica edaya sobayāng si le sobyāng payutān yām. //
% 	\glb mit-ya edaya sobaya-ang si le sob=yāng payutān-Ø yām //
% 	\glc live-\TsgM{} here teacher-\Aarg{} \Rel{} \PatTI{} 
% 		teach=\TsgM{}.\Aarg{} math-\Top{} \Fsg{}.\Dat{} //
	\glft `Here lives the teacher who taught me math.' //
\endgl
\xe
\end{figure}

Example (\ref{ex:reltop}) shows a sentence in which the relative pronoun, 
ungrammatically, forms the controller of topic agreement on the verb in the 
relative clause: \rayr{n}{na} as a genitive topic is supposed to refer to 
\xayr{sobyaaNF}{sobayāng}{teacher} in the matrix clause by way of the
relativizer \rayr{si}{si}. This relativizer would then necessarily carry a
zero-morpheme topic marker. There is no resumptive pronoun in the relative
clause, however, so the relativizer itself forms the anaphora in the relative
clause referring to the relativized argument in the matrix clause. This is not
possible.

In (\ref{ex:relagt}), the relative pronoun *\rayr{saaNF}{*sāng} carries no
overt case agreement since it follows its antecedent (*\rayr{sNNF}{*sangang}
otherwise)---the long vowel identifies it as the agent of the relative clause;
the verb agrees accordingly. There is no resumptive agent pronoun here either,
so the relative pronoun stands in for the agent NP that would be necessary if
the relative clause were an independent sentence. Using a relative pronoun as
an agent-NP replacement in this sentence is likewise ungrammatical, though, and
so is verb agreement with the declined relative pronoun. Similarly, in
(\ref{ex:relpat}), the relative pronoun carries case marking for the patient of
the relative clause, since the agent of the matrix clause serves as the patient
NP of the embedded clause. This is not grammatical either.

\begin{figure}
\ex\label{ex:relpat}
% \a\begingl
\ljudge* \begingl
	\gla Mica edaya sobayāng sās ya @ kradasayang kardang. //
	\glb mit-ya edaya sobaya-ang si-Ø-as ya= krad-asa=yang kardang-Ø //
	\glc live-\TsgM{} here teacher-\Aarg{} \Rel{}-\Aarg{}-\Parg{} \LocT{}=
		hate-\Hab{}=\Fsg{}.\Aarg{} school-\Top{} //
% \endgl
% 
% \a\begingl
% 	\gla Mica edaya sobayāng si ya kradasayang (yas) kardang. //
% 	\glb mit-ya edaya sobaya-ang si ya krad-asa=yang (yas) kardang-Ø //
% 	\glc live-\TsgM{} here teacher-\Aarg{} \Rel{} \LocT{}
% 		hate-\Hab{}=\Fsg{}.\Aarg{} (\TsgM{}.\Parg{}) school-\Top{} //
	\glft `Here lives the teacher whom I used to hate in school.' //
\endgl
\xe
\end{figure}

Altogether, it seems that in Ayeri, core arguments of intransitive and
transitive clauses---agents and patients---cannot precede the embedded verb of
a relative clause; the verb firmly forms the head of the embedded clause in
this regard. The relative pronoun also cannot receive secondary marking for
agents or patients, and neither can it stand in directly as the agent and
patient NP of the relative clause, respectively. It is interesting in this
regard that Ayeri \emph{does} allow this for recipients, however, maybe since
by their nature as goals they carry something of a locative connotation
(compare (\ref{ex:reldat})) and are thus less tightly integrated with verbs,
occupying a middle ground between core arguments and adverbials like the
locative proper.\footnote{This would be interesting to explore in terms of
grammaticalization, since it is possible that this behavior reflects a stage of
the language before \rayr{/ymF}{-yam} had been grammaticalized as the dative
marker. In this respect, it would as well be necessary to explore whether the
similarity between the dative marker \rayr{/ymF}{-yam} and the locative marker
\rayr{/y}{-ya} is indeed etymological or merely incidental.}

\index{pronouns!relative|)}

\subsection{Reflexives and reciprocals}
\label{subsec:reflrec}
\index{pronouns!reflexive|(}

As mentioned previously, Ayeri forms its reflexives with the prefix 
\rayr{sitNF/}{sitang-} in combination with a personal pronoun, compare 
(\ref{ex:reflpat}). If the agent of the action is the same as the reflexive 
patient---that is, the agent acts on itself---the reflexive prefix can also 
migrate onto the verb instead, which is demonstrated in (\ref{ex:reflvb}).

\begin{figure}
\ex\label{ex:reflpat}\begingl
	\gla Ang @ silvye sitang-yes puluyya. //
	\glb ang= silv=ye.Ø sitang=yes puluy-ya //
	\glc \AgtT{}= see=\TsgF{}.\Top{} self=\TsgF{}.\Parg{} mirror-\Loc{} //
	\glft `She sees herself in the mirror.' //
\endgl\xe
\end{figure}

Doing the same with a non-patient pronoun does not work, however. Thus, the
sentence in (\ref{ex:reflvb}) with the reflexive \rayr{sitNF/}{sitang} marked
on the verb is not equivalent to the one in (\ref{ex:reflloc}). Here,
\rayr{sitNF/}{sitang-} appears together with a personal pronoun in the locative
case, even though here as well, the agent and the locative pronoun refer to the
same entity. It may be noted furthermore that the genitive/possessive pronoun
series conveys the meaning of `one's own', which is completely regular in
meaning (`of X-self'), compare (\ref{ex:emphposs}).

\begin{figure}
\ex\label{ex:reflvb}\begingl
	\gla Ang @ sitang-silvye puluyya. //
	\glb ang= sitang=silv=ye.Ø puluy-ya //
	\glc \AgtT{}= self=see=\TsgF{}.\Top{} mirror-\Loc{} //
	\glft `She sees herself in the mirror.' //
\endgl\xe
\end{figure}

\begin{figure}
\ex\label{ex:reflloc}\begingl
	\gla Ang @ silvye sitang-yea puluyya. //
	\glb ang= silv=ye.Ø sitang=yea puluy-ya //
	\glc \AgtT{}= look=\TsgF{}.\Top{} self=\TsgF{}.\Loc{} mirror-\Loc{} //
	\glft `She looks at herself in the mirror.' //
\endgl\xe
\end{figure}

\begin{figure}
\ex\label{ex:emphposs}%
\begingl
	\gla Le @ no @ eryongyang pakay sitang-nā. //
	\glb le= no= ery-ong=yang pakay-Ø sitang=nā //
	\glc \PatTI{}= want= use-\Irr{}=\Fsg{}.\Aarg{} umbrella-\Top{} 
		self=\Fsg{}.\Gen{} //
	\glft `I'd like to use my own umbrella.' //
\endgl\xe
\end{figure}

\rayr{sitNF}{sitang} is also used to carry quantifiers referring to a
pronominal suffix as in (\ref{ex:sitangquant}). Appending a quantifier directly
to the conjugated verb itself can be ambiguous; compare
(\ref{ex:nositangquant}). It appears that \rayr{sitNF}{sitang} does not act as
the controller of the verbal topic marker, however. This is illustrated also by
the ability of \rayr{sitNF}{sitang} and a non-topic agent pronominal suffix to
appear side by side, as in (\ref{ex:sitangnotop}).\footnote{For an analysis
from the point of view of syntax, refer to \autoref{subsubsec:expsitang} 
(p.~\pageref{subsubsec:expsitang}).} As described previously, lexical NPs and
pronominal suffixes on the verb are mutually exclusive; see
\autoref{clitics_postverb_person} (p.~\pageref{clitics_postverb_person}). The
correct answer to the question, \xayr{ANF koronFy sinY gurtnF?}{Ang koronya
sinya guratan?}{Who knows the answer?}, is \xayr{yNF/nYm}{Yang-nyama}{Even I},
with the quantifier clitic leaning on the free pronoun directly, however, since
there is no referential ambiguity in this. Introducing an adverb shows that the
reflexive--quantifier compound follows the conjugated verb and its adjuncts, as
in (\ref{ex:sitangqtorder}).

\begin{figure}
\pex
\a\label{ex:sitangquant}\begingl
	\gla Ang @ koronay sitang-nyama guratanley. //
	\glb ang= koron=ay.Ø sitang=nyama guratan-ley //
	\glc \AgtT{}= know=\Fsg{}.\Top{} self=even answer-\PargI{} //
	\glft `Even I know the answer.' //
\endgl

\a\label{ex:nositangquant}\ljudge\excl\begingl
	\gla Ang @ koronay-nyama guratanley. //
	\glb ang= koron=ay.Ø=nyama guratan-ley //
	\glc \AgtT{}= know=\Fsg{}.\Top{}=even answer-\PargI{} //
	\glft `I even know the answer.' \\
		\textit{Intended:} `Even I know the answer.' //
\endgl
\xe
\end{figure}

\begin{figure}
\ex\label{ex:sitangnotop}%
\begingl
	\gla Le @ koronyang sitang-nyama guratan. //
	\glb le= koron=yang sitang=nyama guratan-Ø //
	\glc \PatTI{}= know=\Fsg{}.\Aarg{} self=even answer-\Top{} //
	\glft `The answer, even I know it.' //
\endgl\xe
\end{figure}

\begin{figure}
\ex\label{ex:sitangqtorder}%
\begingl
	\gla Nimpyāng para-ma sitang-nama. //
	\glb nimp=yāng para=ma sitang=nama //
	\glc run=\TsgM{}.\Aarg{} quick=enough self=only //
	\glft `Only he is running quickly enough.' //
\endgl\xe
\end{figure}

\index{pronouns!reflexive|)}
\index{pronouns!reciprocal|(}

Besides reflexive pronouns, Ayeri also has a reciprocal pronoun,
\xayr{sitnY}{sitanya}{each other}. This pronoun acts the same as other pronouns
and can be inflected according to its function in the clause, as
(\ref{ex:recippro}) shows.

\begin{figure}
\pex\label{ex:recippro}
\a\begingl
	\gla Ang @ narayan {} @ Ajān nay Pila sitanyaya. //
	\glb ang= nara-yan Ø= Ajān nay Pila sitanya-ya //
	\glc \AgtT{}= talk-\TplM{} \Top{}= Ajān and Pila each.other-\Loc{} //
	\glft `Ajān and Pila talk to each other.' //
\endgl

\a\begingl
	\gla Sa @ ming @ tangtang sitanya. //
	\glb sa= ming= tang=tang sitanya-Ø //
	\glc \PatT{}= can= hear=\TplM{}.\Aarg{} each.other-\Top{} //
	\glft `They can hear each other.' //
\endgl
\xe
\end{figure}

\index{pronouns!reciprocal|)}
\index{pronouns|)}

\section{Adjectives}
\label{sec:adjectives}
\index{adjectives|(}

Adjectives are one of the parts of speech in Ayeri which do not inflect for any
of the grammatical properties of their heads, that is, there is no agreement
relation between adjectives and nominal heads. They do inflect for comparison
under certain circumstances, however, and can also take various affixes that
modify the meaning of the adjective stem.

\subsection{Comparison}
\label{subsec:adjcomp}
\index{comparison!of adjectives|(}

In cases where a comparee is left unexpressed or the patient forms the 
standard of comparison, Ayeri uses clitic suffixes on adjectives. The suffixes 
involved are \rayr{/ENF}{-eng} (\Comp{}) and \rayr{/vaa}{-vā} (\Supl{}):

\pex\label{ex:sfxcomp}
\a\label{ex:sfxcomp2}\begingl
	\gla Yeng ganyena men si alingo-eng. //
	\glb yeng gan-ye-na men si alingo=eng //
	\glc \TsgF{}.\Aarg{} child-\Pl{}-\Gen{} one \Rel{} clever=\Comp{} //
	\glft `She is one of the more clever children.' //
\endgl

\a\label{ex:sfxcomp1}\begingl
	\gla Ang tavya {} Diyan tingracas ban-eng na Maha. //
	\glb ang tav-ya Ø Diyan tingrati-as ban=eng na Maha //
	\glc \AgtT{} become-\TsgM{} \Top{} Diyan musician-\Parg{} good=\Comp{} 
		\Gen{} Maha //
	\glft `Diyan became a better musician than Maha.' //
\endgl

%%% Try to express this with the analytic construction with eng-, and you'll 
%%% fail?! > Diyan became a better musician than Maha---only possible with a 
%%% relative clause: Diyan became a musician who is better than Maha. (Sa 
%%% tavya ang Diyan tingrati si ang engya ban sa Maha)

\a\label{ex:sfxsupl}\begingl
	\gla Naratang, yāng pokamayās para-vā. //
	\glb nara=tang yāng pokamaya-as para-vā //
	\glc say=\TplM{}.\Aarg{} \TsgM.\Aarg{} shooter-\Parg{} fast-\Supl{} //
	\glft `They said he is the fastest shooter.' //
\endgl\xe

In (\ref{ex:sfxcomp2}) the comparee is missing, while in (\ref{ex:sfxcomp1}),
the quality under comparison, \xayr{tiNrti\_asF bnF/ENF}{tingracas ban-eng}{a
better musician}, is a patient NP; the standard, \rayr{mh}{Maha}, is expressed
by an adverbial genitive NP. The example in (\ref{ex:sfxsupl}) similarly
expresses an absolute without giving a group of entities to draw from. In all
these cases, it is, however, also possible to use a more complex analytic
construction using verbs.
% FIXME: COVER THIS IN SECTION ON EXISTENTIAL CLAUSES AND COPULAIC VERBS

\index{comparison!of adjectives|)}

\subsection{Negation}
\label{subsec:adjneg}
\index{negation!of adjectives|(}

Adjectives in Ayeri can be negated in two ways: categorially with 
\rayr{/ArY}{-arya}, and pragmatically with \rayr{/Oj}{-oy}. These correspond to
English \fw{un-}, and \fw{in-, il-, ir-} etc.\ for categorial negation, and to
\fw{not} for pragmatic negation. \rayr{/Oj}{-oy} absorbs the vowel of the root 
it is attached to if said root ends in a vowel.

\ex\label{ex:adjarya}\begingl
	\gla Telbaya miseryanang ku-ardārya. //
	\glb telba-ya miseryan-ang ku=arda-arya //
	\glc show-\TsgM{} method-\Aarg{} like=suitable-\Neg{} //
	\glft `The method proved unsuitable.' //
\endgl\xe

\ex~\label{ex:adjoy}\begingl
	\gla Pakoy eda-yanoreng. //
	\glb paka-oy eda=yano-reng //
	\glc safe-\Neg{} this=place-\AargI{} //
	\glft `This place is not safe.' //
\endgl\xe

Example (\ref{ex:adjarya}) displays an adjective which carries the categorial
negation marker \rayr{/ArY}{-arya}; the adjective in (\ref{ex:adjoy}) carries
the simple, pragmatic negation marker \rayr{/Oj}{-oy}. Which one to use is up
to the speaker, since both negate the described property. The categorial marker
puts an emphasis more on expressing a general opposite, while the pragmatic
marker simply negates, so that it is not necessarily implied that the negative
state persists. The place that is \xayr{pkoj}{pakoy}{not safe} now is not
necessarily \xayr{pkaarY}{pakārya}{unsafe} in general, but simply not safe in
the context of the here and now of the utterance.

Besides \fw{ad hoc} derivation of categorial negatives with \rayr{/ArY}{-arya},
there are also a few lexicalized instances. These have an idiomatic meaning and
the negator or the word itself may be irregularly reduced. Examples are, among
others:

\ex\labels
	\begin{tabular}[t]{@{\tl\quad} l @{\enspace→\enspace} l @{\smallskip}}
	\xayr{\larger bnF}{ban}{good}
		& \xayr{\larger bny}{banaya}{ill, sick}
		\\
	\xayr{\larger kovro}{kovaro}{easy}
		& \xayr{\larger kovrY}{kovarya}{awkward}
		\\
	\xayr{\larger sirimNF}{sirimang}{straight}
		& \xayr{\larger sirimy}{sirimaya}{passive}
		\\
	\end{tabular}
\xe

\index{negation!of adjectives|)}

\subsection{Adjectivization}

Adjectives in Ayeri are very commonly zero derivations, that is, there is
rather free conversion between nouns and adjectives,\footnote{Adjectives and
split-off modifiers in noun--noun (compare \autoref{subsubsec:endocomp})
compounds are thus similar at least superficially.} for instance:

\ex\labels
	\begin{tabular}[t]{@{\tl\quad} l @{\enspace\til\enspace} l 
		@{\smallskip}}
	\xayr{\larger Ayeri}{Ayeri}{Ayeri}
		& \xayr{\larger Ayeri}{Ayeri}{Ayeri}
		\\
	\xayr{\larger dis}{disa}{soap, lye}
		& \xayr{\larger dis}{disa}{soapy, alkaline}
		\\
	\xayr{\larger gino}{gino}{drink}
		& \xayr{\larger gino}{gino}{drunk}
		\\
	\xayr{\larger phmj}{pahamay}{danger}
		& \xayr{\larger phmj}{pahamay}{dangerous}
		\\
	\xayr{\larger seMpj}{sempay}{peace}
		& \xayr{\larger seMpj}{sempay}{peaceful}
		\\
	\end{tabular}
\xe

Adjectives can also be derived from verbs with the causative suffix 
\rayr{/Is}{-isa}, which often corresponds to adjectives derived from the 
past participle form---the meaning is often, but not necessarily, relating to 
an achieved state. The suffix may change the last vowel to \rayr{U}{u} or drop 
it; a specific pattern to these changes is not recognizable. The derivations 
may be idiomatic occasionally, as some derivations in the example below show.

\ex\labels
	\begin{tabular}[t]{@{\tl\quad} l @{\enspace→\enspace} l @{\smallskip}}
	\xayr{\larger kelNF/}{kelang-}{connect}
		& \xayr{\larger kelNisu}{kelangisu}{connected, related}
		\\
	\xayr{\larger pluNF/}{palung-}{distinguish}
		& \xayr{\larger pluNis}{palungisa}{various}
		\\
	\xayr{\larger suMdl/}{sundala-}{lose}
		& \xayr{\larger suMdlisu}{sundalisu}{lost}
		\\
	\xayr{\larger thnF/}{tahan-}{write}
		& \xayr{\larger thnisF}{tahanis}{literary}
		\\
	\xayr{\larger ves/}{vesa-}{give birth}
		& \xayr{\larger vesis}{vesisa}{native}
		\\
	\end{tabular}
\xe

There are also at least two words where an \rayr{/Is}{-isa} adjective is 
derived not from a verb, but a word of a different part of speech---in this 
case, a noun, and another adjective:

\ex\labels
	\begin{tabular}[t]{@{\tl\quad} l @{\enspace→\enspace} l @{\smallskip}}
	\xayr{\larger ApinF}{apin}{luck}
		& \xayr{\larger Apinis}{apinisa}{lucky}
		\\
	\xayr{\larger Irj}{iray}{high}
		& \xayr{\larger Iryisu}{irayisu}{exalting}
		\\
	\end{tabular}
\xe

\subsection{Other affixes}
\label{subsec:adjaffx}

As with nouns, other affixes which can be attached to adjectives as clitic 
hosts, are the prefix \rayr{ku/}{ku-}, expressing semblance, as well as 
quantifying and grading suffixes, of which the suffixes used to express 
comparative and superlative are, essentially, a grammaticalized variety, since 
\rayr{/ENF}{-eng} can also be used like `rather'.

\ex\begingl
	\gla Ku-pikisu paray-parayang. //
	\glb ku=pikisu paray\til{}paray-ang //
	\glc like=scared \Dim{}\til{}cat-\Aarg{} //
	\glft `The kitten is like scared.' //
\endgl\xe

\ex~\label{ex:adjquant}\begingl
	\gla Napay-eng eda-prikanreng. //
	\glb napay=eng eda=prikan-reng //
	\glc spicy=rather this=soup-\AargI{} //
	\glft `This soup is rather spicy.' //
\endgl\xe

\index{adjectives|)}


\section{Adpositions}
\label{sec:adpositions}
\index{adpositions|(}

Adpositions are another part of speech in Ayeri whose stem itself does not 
inflect. Ayeri's most basic adpositions are derived from relational nouns, 
which is likely the reason why Ayeri mostly employs prepositions, with 
postpositions and ambipositions being less important placement patterns 
\parencites[110--111]{hagege2010}[81\psqq]{lehmann2015}. Adpositions in their 
most basic use trigger locative marking on the governed NP, the prepositional
object; for allative and ablative meanings, the prepositional object may also
appear in the dative and the genitive case, respectively, as described in
\autoref{subsubsec:dative}.\footnote{Even a prolative use together with the
instrumental is thinkable.} The cognitive metaphor `time equals space' with the
future conceptualized as lying ahead and the past behind also holds in Ayeri,
so that some of the words describing locations also double to describe temporal
relations.

\subsection{Prepositions}
\label{subsec:prepositions}
\index{prepositions|(}

\begin{table}[tp]\centering
\caption{Prepositions (simple)}
\begin{tabu} to \linewidth {I[3] X[4] X[6]}
\tableheaderfont\toprule
\multicolumn{2}{c}{Preposition}
	& Etymology (or related to)
	\\

\toprule

\ayr{\upshape AgonnF}
agonan
	& outside
	& \xayr{AgonnF}{agonan}{outside}
	\\
	
\ayr{\upshape AvnF}
avan
	& bottom, ground
	& \xayr{AvnF}{avan}{ground, bottom; soil}
	\\

% \ayr{\upshape djrinF}
% dayrin
% 	& side
% 	& \xayr{djrinF}{dayrin}{waist}
% 	\\

\ayr{\upshape EjrnF}
eyran
	& under, below
	& \xayr{EjrnF}{eyran}{sole}
	\\

\ayr{\upshape EjrrY}
eyrarya
	& over
	& \xayr{EjrnF}{eyran}{sole} + \rayr{/ArY}{-arya} (\Neg{})
	\\

\ayr{\upshape kjvo}
kayvo
	& with, beside\footnotemark
	& \xayr{kjvF/}{kayv-}{accompany}
	\\

\ayr{\upshape koNF}
kong
	& inside, within
	& \xayr{koNF}{kong}{inside}
	\\
	
\ayr{\upshape liNF}
ling
	& on
	& \xayr{liNF}{ling}{top}
	\\

\ayr{\upshape lug}
luga
	& among, between
	& \xayr{lug/}{luga-}{pass, penetrate}
	\\

\ayr{\upshape mNsh}
mangasaha
	& towards, in\,+\,\emph{time}
	& \xayr{mN sh/}{manga saha-}{coming}
	\\

\ayr{\upshape mNsr}
mangasara
	& away
	& \xayr{mN sr/}{manga sara-}{going}
	\\

\ayr{\upshape mrinF}
marin
	& front, on (walls etc.)
	& \xayr{mrinF}{marin}{face, surface}
	\\

\ayr{\upshape midj}
miday
	& around
	& \xayr{midj/}{miday-}{surround}
	\\

\ayr{\upshape nsj}
nasay
	& near, close
	& \xayr{nsj}{nasay}{proximity}
	\\

\ayr{\upshape nuveNF}
nuveng
	& left
	& \xayr{nuho}{nuho}{liver}
	\\

\ayr{\upshape pNF}
pang
	& behind, ago
	& \xayr{pNF}{pang}{back}
	\\

\ayr{\upshape ptmeNF}
patameng
	& right
	& \xayr{ptmF}{patam}{heart}
	\\

\bottomrule
\end{tabu}

\label{tab:prepos}
\end{table}

\footnotetext{There is also a preposition \xayr{djrin}{dayrin}{side} listed in 
the dictionary, however, this has never seen much use. Instead, 
\rayr{kjvo}{kayvo} has come to cover `beside, to the side of' as well.}

\autoref{tab:prepos} gives all the words in Ayeri which may be used as
prepositions. As mentioned above, most of these are derived transparently from
nouns, so they have probably been grammaticalized relatively recently---their
non-preposition meaning is still transparent, they are still phonologically
rather complex, and some of them are even polysyllabic in spite of not being
composed and covering rather basic meanings.\footnote{Unsurprisingly,
\citet[129]{hagege2010} references Zipf regarding speech economy and token
frequency. According to \citet[134--141]{lehmann2015}, the phonological
integrity of morphemic units reduces as grammaticalization is progressing (with
token frequency increasing due to increasing obligatoriness).
\citet{bybeehopper2001b} see the reason for phonological reduction of highly
frequent phonological material \textcquote[11]{bybeehopper2001b}{in the
automatization of neuro-motor sequences […]. Such reductions are systematic
across speakers; that is, they do not respresent \enquote{sloppy} or
\enquote{lazy} speech}. Hence, for example, English's most basic prepositions
are extremely short and simple words, for instance, \fw{of, at, in}, which
derive from the slightly more complex PIE forms \fw{*h₂ep-ó, *h₂ed, *h₁en(-i)},
respectively \citep[1, 39, 269]{kroonen2013}. Since adpositions frequently
grammaticalize into case markers, it may be assumed that the phonologically
much more simple case affixes of Ayeri constitute an older layer of basic
adpositions. Their non-suffixed forms may be remnants of this use.} Since these
nouns have ceased to function as common nouns in this context due to
grammaticalization, however, it is not possible to inflect them in the way
described in \autoref{sec:nouns}. Thus, for example, while it is possible to
say (\ref{ex:lingnn}), it is not really possible to say (\ref{ex:lingpr}):

\pex
\a\label{ex:lingnn}\begingl
	\gla Le yomareng kanka lingya rivanena. //
	\glb le yoma=reng kanka-Ø ling-ya rivan-ena //
	\glc \PatTI{} exist=\TsgI{}.\Aarg{} snow-\Top{} top-\Loc{}
		mountain-\Gen{} //
	\glft `There is snow on the top of the mountain.'\footnotemark //
\endgl

\a\label{ex:lingpr}\ljudge* \begingl
	\gla Ang nedraye lingya nedrānena. //
	\glb ang nedra=ye.Ø ling-ya nedrān-na //
	\glc \AgtT{} sit=\TsgF{}.\Top{} top-\Loc{} chair-\Gen{} //
	\glft `\ques{}She sits on the top of a chair.' //
\endgl

\xe

\footnotetext{The corresponding sentence with a preposition is \xayr{le yomreNF
kMk liNF rivnFy}{Le yomareng kanka ling rivanya}{There is snow on top of the
mountain} (\PatTI{} exist=\TsgI{}.\Aarg{} snow-\Top{} top mountain-\Loc{}).}

\noindent Instead, the grammatical way to express (\ref{ex:lingpr}) is the 
following, using \rayr{liNF}{ling} as a preposition with the object in the 
locative case:

\ex\begingl
	\gla Ang nedraye ling nedrānya. //
	\glb ang nedra=ye.Ø ling nedrān-ya //
	\glc \AgtT{} sit=\TsgF{}.\Top{} top chair-\Loc{} //
	\glft `She sits on a chair.' //
\endgl\xe

In this case, since \emph{on} is the expected position of sitting with regards 
to chairs, the preposition can even be dropped:

\ex\begingl
	\gla Ang nedraye nedrānya. //
	\glb ang nedra=ye.Ø nedrān-ya //
	\glc \AgtT{} sit=\TsgF{}.\Top{} chair-\Loc{} //
	\glft `She sits on a chair.' //
\endgl\xe

With regards to (\ref{ex:lingnn}) it is also necessary to mention what 
\citet{hagege2010} calls the `Proof by Anachrony Principle' 
\citep[158--159]{hagege2010}. According to this principle, when an adposition 
is very grammaticalized, speakers can use both the adposition and its 
etymological ancestor side by side without taking offense in the double 
occurrence. This is notably not the case in Ayeri, where it is not possible to 
say things like (\ref{ex:behindback1}), where \rayr{pNF}{pang} is used in both 
its meanings so that the preposition \xayr{pNF}{pang}{behind} governs the 
original noun \xayr{pNF}{pang}{back}.

\pex
\a\label{ex:behindback1}\ljudge* \begingl
	\gla Le ranice ang Maha adanya pang pangya yena. //
	\glb le ranit-ye ang Maha adanya-Ø pang pang-ya yena //
	\glc \PatTI{} hide-\TsgF{} \Aarg{} Maha that-\Top{} back back-\Loc{} 
		\TsgF{}.\Gen{} //
	\glft `*Maha hides it at the back of her back.' //
\endgl

\a\label{ex:behindback2}\begingl
	\gla Le ranice ang Maha adanya pangya yena. //
	\glb le ranit-ye ang Maha adanya-Ø pang-ya yena //
	\glc \PatTI{} hide-\TsgF{} \Aarg{} Maha that-\Top{} back-\Loc{} 
		\TsgF{}.\Gen{} //
	\glft `Maha hides it at her back,'\\
		\textit{or:} `Maha hides it behind herself.' //
\endgl

\xe

Examples like (\ref{ex:lingpr}), on the other hand, show that there is
nonetheless a tendency in Ayeri towards grammaticalization of nouns which used
to be relational. Grammaticalization is visible in that formerly relational
nouns have become restricted in the way they can be used syntactically
\citep[174]{lehmann2015}. This specialization is also apparent in morphology
from the fact that prepositions in Ayeri, in spite of their nominal origin,
cannot be modified by adjectives and relative clauses like regular nouns. Thus,
for instance, while \rayr{AvnF}{avan} as a noun can mean `soil' or `ground' and
can be modified by semantically coherent adjectives like
\xayr{kbu}{kabu}{fertile}, the preposition \rayr{AvnF}{avan} cannot. Again, a
grammatical way to express (\ref{ex:avanprep}) would have to use
\rayr{AvnF}{avan} as a relational noun, that is, \xayr{AvnFy kbu
similen}{avanya kabu similena}{at the fertile bottom of the country}
(bottom-\Loc{} fertile country-\Gen{}). The fact that topicalized heads lack
case marking makes adpositions derived from nouns, like \rayr{AvnF}{avan}
homophonous with the respective etymologically related preposition.

\pex
\a\label{ex:avannn}\begingl
	\gla Sa yomareng avan kabu ibangya yana. //
	\glb sa yoma=reng avan-Ø kabu ibang-ya yana //
	\glc \PatT{} exist=\TsgI.\AargI{} ground-\Top{} fertile field-\Loc{} 
		\TsgM{}.\Gen{} //
	\glft `Fertile ground is on his field.' //
\endgl

\a\label{ex:avanprep}\ljudge* \begingl
	\gla Ang mican avan kabu similya //
	\glb ang mit=yan.Ø avan kabu simil-ya //
	\glc \AgtT{} live=\TplM{}.\Top{} bottom fertile country-\Loc{} //
	\glft `*They live at the fertile bottom of the country.' //
\endgl

\xe

At the beginning of this section it was shown that prepositions in Ayeri cannot
receive number and case marking, which are otherwise typical features of nouns.
What is possible with regards to affixes, however, is adding degree suffixes to
prepositions, since these suffixes are clitics rather than inflections 
(compare \autoref{subsec:clitics}, p.~\pageref{clitics_quant}):

\ex\label{ex:prepquant}\begingl
	\gla Ang mitasaye pang-ikan mandayya tado. //
	\glb ang mit-asa=ye.Ø pang=ikan manday-ya tado //
	\glc \AgtT{} live-\Hab{}=\TsgF{}.\Top{} back=much forum-\Loc{} old //
	\glft `She used to live way behind the old forum.' //
\endgl\xe

\begin{table}[tp]\centering
\caption{Prepositions (directional)}
\begin{tabu} to \linewidth {X[2] X[3]}
\tableheaderfont\toprule
Preposition
	& \fw{manga} + \Prep{}
	\\

\toprule

\xayr{AgonnF}{agonan}{outside}
	& out
	\\

\xayr{AvnF}{avan}{at bottom}
	& to the bottom; \textit{with \Dat{}/\Gen{}:} down to/from
	\\

% \xayr{djrinF}{dayrin}{beside}
% 	& to the side
% 	\\

\xayr{EjrnF}{eyran}{under}
	& under
	\\

\xayr{EjrrY}{eyrarya}{over}
	& across, over
	\\

\xayr{kjvo}{kayvo}{with, beside}
	& along
	\\

\xayr{koNF}{kong}{inside}
	& into
	\\

\xayr{liNF}{ling}{on top}
	& onto, while; \textit{with \Dat{}/\Gen{}:} up to/from
	\\

\xayr{lug}{luga}{between}
	& through, during, for\,+\,\textit{time}
	\\

% \ayr{\upshape mNsh}
% mangasaha
% 	& 
% 	\\
% 
% \ayr{\upshape mNsr}
% mangasara
% 	& 
% 	\\

\xayr{mrinF}{marin}{in front}
	& to the front
	\\

\xayr{midj}{miday}{around}
	& circling around
	\\

\xayr{nsj}{nasay}{near}
	& into the near
	\\

\xayr{nuveNF}{nuveng}{left}
	& to the left
	\\

\xayr{pNF}{pang}{behind}
	& behind, to the back
	\\

\xayr{ptmeNF}{patameng}{right}
	& to the right
	\\

\bottomrule
\end{tabu}

\label{tab:preposdyn}
\end{table}

\phantomsection\label{manga}
As demonstrated before, another quasi-inflection adpositions in Ayeri can carry
is the directional marker \rayr{mN}{manga} (see \autoref{sec:typology}). While
most of the prepositions in \autoref{tab:prepos} have a static meaning,
\rayr{mN}{manga} indicates a motion in the direction of the respective
location, thus \xayr{koNF}{kong}{inside} becomes \xayr{mN koNF}{manga
kong}{into}, for instance. \autoref{tab:preposdyn} repeats the table of
prepositions above for the most part and gives the respective directional
meanings. The prepositions \rayr{mNsh}{mangasaha} and \rayr{mNsr}{mangasara}
are missing from this list and appear in the previous table instead, even
though they express motion rather than position, because they are only used in
this base form and cannot be prefixed by \rayr{mN}{manga}, which they already
contain. Note, however, that \rayr{mNsh}{mangasaha} and \rayr{mNsr}{mangasara}
are not synonymous to an adjunct in the dative and the genitive case,
respectively. Rather, the prepositions add a more deliberate or literal
meaning:

\pex
\a\begingl
	\gla Ang nimpay kardangyam. //
	\glb ang nimp=ay.Ø kardang-yam //
	\glc \AgtT{} run=\Fsg{}.\Top{} school-\Dat{} //
	\glft `I'm running to (a/the) school.' \\
		(e.g. for class, or just up to the building) //
\endgl

\a\begingl
	\gla Ang nimpay mangasaha kardangya. //
	\glb ang nimp=ay.Ø mangasaha kardang-ya //
	\glc \AgtT{} run=\Fsg{}.\Top{} towards school-\Loc{} //
	\glft `I'm running towards (a/the) school.' \\
		(up to the building) //
\endgl
\xe

\pex~
\a\begingl
	\gla Ang lampay kardangena. //
	\glb ang walk=ay.Ø kardang-ena //
	\glc \AgtT{} walk=\Fsg{}.\Top{} school-\Gen{} //
	\glft `I'm walking from (a/the) school.' \\
		(e.g. home, or somewhere else from there) //
\endgl

\a\begingl
	\gla Ang lampay mangasara kardangya. //
	\glb ang lamp=ay.Ø mangasara kardang-ya //
	\glc \AgtT{} walk=\Fsg{}.\Top{} away school-\Loc{} //
	\glft `I'm walking away from (a/the) school.' \\
		(away from the building) //
\endgl
\xe

Also note that while Germanic languages like English make frequent use of set
expressions which combine a verb with an intransitive preposition, such as
\fw{run away, go by, raise up, track down}, sometimes with rather idiomatic
meanings, this pattern does not occur as frequently in Ayeri. Some exceptions
are:

\pex\label{ex:particleverbs}
\a \xayr{\larger IlF/ mNsr}{il- mangasara}{surrender} (give away),
\a \xayr{\larger lMtF/ mNsr}{lant- mangasara}{distract} (lead away),
\a \xayr{\larger niMpF/ mNsr}{nimp- mangasara}{escape} (run away),
\a \xayr{\larger tpY/ djrinF}{tapy- dayrin}{save (valuable assets)} (put
	aside),
\a \xayr{\larger tpY/ midj}{tapy- miday}{put on} (put around),
\a \xayr{\larger tur/ mNsh}{tura- mangasaha}{forward} (send towards).
\xe

These verbs do not govern a prepositional object in the locative case in their
idiomatic meaning, as displayed by the next example, in which
\rayr{btNimnF}{batangiman} and \rayr{s AgYaanF}{sa Ajān} do neither serve as
arguments of \rayr{lMtYo}{lanco} or \rayr{mNsr}{mangasara}, but of the phrasal
verb \rayr{lMtF/ mNsr}{lant- mangasara}:\footnote{Colloquially,
\rayr{mNsh}{mangasaha} and \rayr{mNsr}{mangasara} may be shortened to just
\rayr{sh}{saha} and \rayr{sr}{sara}, respectively.}

\ex\begingl
	\gla Ang lanco mangasara batangiman sa Ajān. //
	\glb ang lant-yo mangasara batangiman-Ø sa Ajān //
	\glc \AgtT{} lead-\TsgN{} away mosquito-\Top{} \Parg{} Ajān //
	\glft `The mosquito distracted Ajān.' //
\endgl\xe

Very often, where the verb expression in English contains a preposition, there 
is a separate verb in Ayeri, or the same verb is used in Ayeri for both the 
plain English verb and the one extended by a preposition:

\pex
	\a \xayr{\larger ApMdF/}{apand-}{descend, climb down},
	\a \xayr{\larger dil/}{dila-}{figure out, find out},
	\a \xayr{\larger liNF/}{ling-}{ascend, mount, climb up},
	\a \xayr{\larger ng/}{naga-}{watch after},
	\a \xayr{\larger phF/}{pah-}{remove, take away},
	\a \xayr{\larger subFrF/}{subr-}{cease, give up}.
\xe

\pex~
	\a \xayr{\larger k/}{ka-}{throw (away)},
	\a \xayr{\larger mtF/}{mat-}{warm (up)},
	\a \xayr{\larger sikFlF/}{sikl-}{rip (up)}.
\xe

In cases where the preposition does not have a prepositional object otherwise,
its double nature as a noun comes to the fore in that the preposition word will
be treated like a noun if it is denominal and carries the appropriate case
marker itself:

\pex
\a\begingl
	\gla Ang sahayan manga pang nangaya. //
	\glb ang saha=yan.Ø manga pang nanga-ya //
	\glc \AgtT{} go=\Tpl{}.\Top{} \Dir{} back house-\Loc{} //
	\glft `They go behind the house.' //
\endgl

\a\begingl
	\gla Ang sahayan pangyam. //
	\glb ang saha=yan.Ø pangyam //
	\glc \AgtT{} go=\Tpl{}.\Top{} back-\Dat{} //
	\glft `They go behind (it),'\\
		\textit{or:} `They go to the back.' //
\endgl

\xe

\index{prepositions|)}

\subsection{Postpositions}
\label{subsec:postpos}
\index{postpositions|(}

\begin{table}[tp]\centering
\caption{Postpositions}
\begin{tabu} to \linewidth {I[3] X[4] X[6]}
\tableheaderfont\toprule
\multicolumn{2}{c}{Postposition}
	& Etymology (or related to)
	\\

\toprule

\ayr{\upshape d/naarY}
da-nārya
	& despite, in spite of
	& \xayr{d/}{da-}{such} + \xayr{naarY}{nārya}{but}
	\\

\ayr{\upshape kjvj}
kayvay
	& without
	& \xayr{kjvo}{kayvo}{with} + \rayr{/Oj}{-oy} (\Neg{})
	\\

\ayr{\upshape mshtj}
masahatay
	& since
	& \rayr{m/}{mə-} (\Pst{}) + \xayr{sh/}{saha-}{come} + 
		\xayr{tdj}{taday}{time}
	\\

\ayr{\upshape nsYmF}
nasyam
	& according to
	& \xayr{nsYymF}{nasyyam}{following}
	\\

\ayr{\upshape pNF}
pang
	& beyond, after, past
	& \xayr{pNF}{pang}{back}
	\\

\ayr{\upshape pesnF}
pesan
	& until
	& ---
	\\

\ayr{\upshape rnF}
ran
	& against
	& \emph{possibly} \xayr{rnF}{ran}{from it}
	\\

\ayr{\upshape ryu}
rayu
	& diagonally across
	& \xayr{ryu}{rayu}{slanted, oblique, skewed}
	\\
	
\ayr{\upshape ymFv}
yamva
	& instead of
	& ---
	\\

\bottomrule
\end{tabu}

\label{tab:postpos}
\end{table}

While Ayeri mainly uses prepositions---which is by far the most common order
for VO languages \citep{wals95}---it also uses a number of postpositions, which
are given in \autoref{tab:postpos}. As can be read from the table,
postpositions do not usually have a nominal origin but are derived either from
other prepositions, from adverbial phrases, or even from an adjective in the
case of \rayr{ryu}{rayu}. The etymologies of \rayr{pesnF}{pesan} and
\rayr{ymFv}{yamva} are unclear to date.

The postposition \rayr{pNF}{pang} is special in that it also exist as a
preposition meaning `behind, in the back of', though as a postposition it
acquires the related but slightly different meaning `beyond, after, past'. It
might thus be better treated as a homonym to the preposition rather than as an
ambiposition \citep[115]{hagege2010}. Example (\ref{ex:pangprep}) illustrates
the use of \rayr{pNF}{pang} as a preposition, (\ref{ex:pangpost}) the use of
\rayr{pNF}{pang} as a postposition. This is in contrast to typical
ambipositions like German \fw{wegen} `because of, due to' in (\ref{ex:wegen}),
which has the same meaning in either position and the position variant is just
a matter of style.

\pex
\a\label{ex:pangprep}\begingl
	\gla Sa lancāng pel manga pang penungya. //
	\glb sa lant=yāng pel-Ø manga pang penung-ya //
	\glc \PatT{} lead=\TsgM{}.\Aarg{} horse-\Top{} \Dir{} back 
		barn-\Loc{} //
	\glft `The horse, he leads it behind the stable.' //
\endgl

\a\label{ex:pangpost}\begingl
	\gla Lesyo pelang si sā nimpyong penungya pang yan. //
	\glb les-yo pel-ang si sā nimp=yong penung-ya pang yan.Ø //
	\glc fall-\TsgN{} horse-\Aarg{} \Rel{} \CauT{} 
		run=\TsgN{}.\Aarg{} stable-\Loc{} back \Tpl{}.\Top{} //
	\glft `The horse they raced past the barn fell.' //
\endgl

\xe

\pex~\label{ex:wegen}%
German:
\a\label{ex:wegenprep}\begingl
	\gla wegen des schlechten Wetters //
	\glb wegen des schlecht-en Wetter-s //
	\glc because.of \Def{}.\Gen{}.\N{}.\Sg{} bad-\Gen{}.\N{}.\Sg{} 
		weather-\Gen{} //
	\glft `because of the bad weather' //
\endgl

\a\label{ex:wegenpost}\begingl
	\gla des schlechten Wetters wegen //
	\glb des schlecht-en Wetter-s wegen //
	\glc \Def{}.\Gen{}.\N{}.\Sg{} bad-\Gen{}.\N{}.\Sg{} weather-\Gen{} 
		because.of //
	\glft (idem) //
\endgl

\xe

Besides the difference in placement, the morphological properties of 
postpositions are the same as those of prepositions. That is, where 
postpositions are derived from nouns at all, they do not receive case and 
number marking and cannot themselves be modified by adjectives or relative 
clauses. Generally, it is possible for them to be hosts of quantifier clitics 
where semantics permit it.

\index{postpositions|(}

\subsection{Adpositions and time}
\index{adpositions!of time|(}

\begin{table}[tp]\centering
\caption{Adpositions with temporal meaning}
\begin{tabu} to \linewidth {X X X}
\tableheaderfont\toprule
Adposition
	& Spatial meaning
	& Temporal meaning
	\\

\toprule

\tablesubheaderfont\multicolumn{3}{c}{P r e p o s i t i o n s} \\

\midrule

\rayr{koNF}{kong}
	& inside
	& within
	\\

\rayr{liNF}{ling}
	& on top of
	& while
	\\

\rayr{mrinF}{marin}
	& in front of
	& before
	\\

\rayr{mN lug}{manga luga}
	& through
	& during
	\\

\rayr{mNsh}{mangasaha}
	& towards
	& in + \textit{time}
	\\

\rayr{pNF}{pang}
	& behind
	& ago
	\\

\midrule

\tablesubheaderfont\multicolumn{3}{c}{P o s t p o s i t i o n s} \\

\midrule

\rayr{mshtj}{masahatay}
	& ---
	& since
	\\

\rayr{pesnF}{pesan}
	& ---
	& until
	\\

\rayr{pNF}{pang}
	& beyond, after
	& after, past
	\\

\bottomrule
\end{tabu}

\label{tab:temppos}
\end{table}

It has been mentioned above that location also serves as the conceptual
metaphor for expressing temporal relationships. Notably the prepositions
\xayr{koNF}{kong}{inside}, \xayr{liNF}{ling}{on}, \xayr{mrinF}{marin}{in front
of}, \xayr{mN lug}{manga luga}{through}, \xayr{mNsh}{mangasaha}{towards}, and
\xayr{pNF}{pang}{behind} come to mind as doubling for `within', `while',
`before', `during', `in + \emph{time}', and `ago', respectively (also see
\autoref{tab:temppos}). Since postpositions are not primarily derived from
nouns, there are dedicated forms for expressing temporal relationships, namely,
\xayr{mshtj}{masahatay}{since},
\xayr{pesnF}{pesan}{until}, and as the only form with a double function,
\xayr{pNF}{pang}{after, past}.

\pex
\a\label{ex:kongtemp}\begingl
	\gla Miranang kong bihanya sam. //
	\glb mira=nang kong bihan-ya sam //
	\glc do=\Fpl{}.\Aarg{} inside week-\Loc{} two //
	\glft `We will do it within two weeks.' //
\endgl

\a\label{ex:mgshtemp}\begingl
	\gla Girenjang mangasaha pidimya-kay. //
	\glb girend=yang mangasaha pidim-ya=kay //
	\glc arrive=\TsgM{}.\Aarg{} towards hour-\Loc{}=few //
	\glft `He will arrive in a few hours.' //
\endgl

\a\label{ex:lingtemp}\begingl
	\gla Layaye-ikan ang Pila ling yeng pakur. //
	\glb laya-ye=ikan ang Pila ling yeng pakur //
	\glc read-\TsgF{}=much \Aarg{} Pila on \TsgF{}.\Aarg{} sick //
	\glft `Pila read a lot while she was sick.' //
\endgl

\xe

Of the examples above, the use of \rayr{koNF}{kong} in (\ref{ex:kongtemp}) is
probably still closest to a local preposition in that the time span is
conceptualized as a container, or the distance between two points. The use of
\rayr{mNsh}{mangasaha} in (\ref{ex:mgshtemp}), on the other hand, is more
idiomatic. While the prepositions in these two examples each govern an NP,
example (\ref{ex:lingtemp}) shows that it is also possible for prepositions
expressing a temporal relationship to govern a subclause. This ability is even
more prominent with temporal postpositions in that all of the words listed
above can govern either an NP or a clause, for instance,
\rayr{mshtj}{masahatay}:

\pex
\a\label{ex:mshtaynp}\begingl
	\gla Ang manga hangya lakayperinya masahatay. //
	\glb ang manga hang=ya.Ø lakayperin-ya masahatay //
	\glc \AgtT{} \Prog{} stay=\TsgM{}.\Top{} solstice-\Loc{} since //
	\glft He has been staying since the solstice. //
\endgl

\a\label{ex:mshtays}\begingl
	\gla Yeng giday sarayāng masahatay. //
	\glb yeng giday sara=yāng masahatay //
	\glc \TsgF{}.\Aarg{} sad leave=\TsgM{}.\Aarg{} since //
	\glft `She has been sad since he left.' //
\endgl

\xe

\index{adpositions!of time|)}

\index{adpositions|)}

\section{Verbs}
\label{sec:verbs}
\index{verbs|(}

Besides nouns, verbs constitute the other main part of speech in Ayeri which
carries inflections. Verbs show person and number agreement, but may also
inflect for tense, aspect, mood, and modality as grammatical categories of the
verb itself. Personal pronouns may furthermore cliticize to the verb stem, and
the verb phrase is also often marked with a clitic indicating the topic of the
sentence and the topic NP's role in Ayeri's case system, which can be
interpreted as a second agreement relation. Further clitics may indicate
reflexive actions, progressive aspect, likeness, logical connection, as well
as degree and measure. Verbs are thus probably the most versatile part of
speech on the one hand, but also the one with the heaviest workload on the
other. The following sections will dissect the morphology of verbs category by
category.
% Since cliticization is a phrase-level process \citep{klavans1985},
% it will only be touched on briefly here.
Because verbs inhabit a central 
position in syntax and exhibit agreement morphology, it will be necessary in
this section to merge syntax and morphology on occasion in order to describe
morphosyntactic effects.

\subsection{Person marking}
\label{subsec:persnumagr}
\index{agreement!person|(}
\index{agreement!number|(}

\begin{table}[tp]\centering
\caption[Conjugation paradigm for \xayr{sobF/}{sob-}{learn, teach}]{Conjugation
paradigm for \xayr{sobF/}{sob-}{learn, teach} (monoconsonantal root)}

\begin{tabu} to \linewidth {X I[2] I[2] X[2]}
\tableheaderfont\toprule
Person
	& Topicalized\footnotemark
	& Clitic agent
	& Translation
	\\

\toprule

\Fsg{}	& sobay		& sobyang	& `I learn'		\\
\Ssg{}	& sobva		& sobvāng	& `you learn'	\\
\TsgM{}	& sobya		& sobyāng	& `he learns'	\\
\TsgF{}	& sobye		& sobyeng	& `she learns'	\\
\TsgN{}	& sobyo		& sobyong	& `it learns'	\\
\TsgI{}	& sobara	& sobreng	& `it learns'	\\

\midrule

\Fpl{}	& sobayn	& sobnang	& `we learn'	\\
\Spl{}	& sobva		& sobvāng	& `you learn'	\\
\TplM{}	& sobyan	& sobtang	& `they learn'	\\
\TplF{}	& sobyen	& sobteng	& `they learn'	\\
\TplN{}	& sobyon	& sobtong	& `they learn'	\\
\TplI{}	& sobaran	& sobteng	& `they learn'	\\

\midrule

\Imp{}	& sobu!			& \multicolumn{2}{l}{`learn!'}					\\
\Hort{}	& sobu-sobu!	& \multicolumn{2}{l}{`let's learn!'}			\\
\Iter{}	& so-sob-		& \multicolumn{2}{l}{`learn again, relearn'}	\\
\Ptcp{}	& sobyam		& \multicolumn{2}{l}{`learning'}				\\
	
\bottomrule

\end{tabu}
\label{tab:monoconsconj}
\end{table}

\footnotetext{Third-person topicalized forms are homonymous with third-person
agreement forms.}

\begin{table}[tp]\centering
\caption[Conjugation paradigm for \xayr{AnFlF/}{anl-}{bring}]{Conjugation 
paradigm for \xayr{AnFlF/}{anl-}{bring} (biconsonantal root)}

\begin{tabu} to \linewidth {X I[2] I[2] X[2]}
\tableheaderfont\toprule
Person
	& Topicalized
	& Clitic agent
	& Translation
	\\

\toprule

\Fsg{}	& anlay		& anlyang	& `I bring'		\\
\Ssg{}	& anlava	& anlavāng	& `you bring'	\\
\TsgM{}	& anlya		& anlyāng	& `he brings'	\\
\TsgF{}	& anlye		& anlyeng	& `she brings'	\\
\TsgN{}	& anlyo		& anlyong	& `it brings'	\\
\TsgI{}	& anlara	& anlareng	& `it brings'	\\

\midrule

\Fpl{}	& anlayn	& anlanang	& `we bring'	\\
\Spl{}	& anlava	& anlavāng	& `you bring'	\\
\TplM{}	& anlyan	& anlatang	& `they bring'	\\
\TplF{}	& anlyen	& anlateng	& `they bring'	\\
\TplN{}	& anlyon	& anlatong	& `they bring'	\\
\TplI{}	& anlaran	& anlateng	& `they bring'	\\

\midrule

\Imp{}	& anlu!			& \multicolumn{2}{l}{`bring!'}					\\
\Hort{}	& anlu-anlu!	& \multicolumn{2}{l}{`let's bring!'}			\\
\Iter{}	& an-anl-		& \multicolumn{2}{l}{`bring again, bring back'}	\\
\Ptcp{}	& anlyam		& \multicolumn{2}{l}{`bringing'}				\\
	
\bottomrule

\end{tabu}
\label{tab:biconsconj}
\end{table}

\begin{table}[tp]\centering
\caption[Conjugation paradigm for \xayr{no/}{no-}{want}]{Conjugation 
paradigm for \xayr{no/}{no-}{want} (vocalic root)}

\begin{tabu} to \linewidth {X I[2] I[2] X[2]}
\tableheaderfont\toprule
Person
	& Topicalized
	& Clitic agent
	& Translation
	\\

\toprule

\Fsg{}	& noay		& noyang	& `I want'		\\
\Ssg{}	& nova		& novāng	& `you want'	\\
\TsgM{}	& noya		& noyāng	& `he wants'	\\
\TsgF{}	& noye		& noyeng	& `she wants'	\\
\TsgN{}	& noyo		& noyong	& `it wants'	\\
\TsgI{}	& noara		& noreng	& `it wants'	\\

\midrule

\Fpl{}	& noayn		& nonang	& `we want'		\\	
\Spl{}	& nova		& novāng	& `you want'	\\	
\TplM{}	& noyan		& notang	& `they want'	\\
\TplF{}	& noyen		& noteng	& `they want'	\\
\TplN{}	& noyon		& notong	& `they want'	\\
\TplI{}	& noaran	& noteng	& `they want'	\\

\midrule

\Imp{}	& nu!		& \multicolumn{2}{l}{`want!'}		\\
\Hort{}	& nu-nu!	& \multicolumn{2}{l}{`let's want!'}	\\
\Iter{}	& no-no-	& \multicolumn{2}{l}{`want again'}	\\
\Ptcp{}	& noyam		& \multicolumn{2}{l}{`wanting'}		\\

\bottomrule

\end{tabu}
\label{tab:vocconj}
\end{table}

\begin{table}[tp]\centering
\caption[Conjugation paradigm for \xayr{Ap/}{apa-}{laugh}]{Conjugation 
paradigm for \xayr{Ap/}{apa-}{laugh} (vocalic root in -a)}

\begin{tabu} to \linewidth {X I[2] I[2] X[2]}
\tableheaderfont\toprule
Person
	& Topicalized
	& Clitic agent
	& Translation
	\\

\toprule

\Fsg{}	& apāy		& apayang	& `I laugh'		\\
\Ssg{}	& apava		& apavāng	& `you laugh'	\\
\TsgM{}	& apaya		& apayāng	& `he laughs'	\\
\TsgF{}	& apaye		& apayeng	& `she laughs'	\\
\TsgN{}	& apayo		& apayong	& `it laughs'	\\
\TsgI{}	& apāra		& apareng	& `it laughs'	\\

\midrule

\Fpl{}	& apāyn		& apanang	& `we laugh'	\\	
\Spl{}	& apava		& apavāng	& `you laugh'	\\	
\TplM{}	& apayan	& apatang	& `they laugh'	\\
\TplF{}	& apayen	& apateng	& `they laugh'	\\
\TplN{}	& apayon	& apatong	& `they laugh'	\\
\TplI{}	& apāran	& apateng	& `they laugh'	\\

\midrule

\Imp{}	& apu!		& \multicolumn{2}{l}{`laugh!'}			\\
\Hort{}	& apu-apu!	& \multicolumn{2}{l}{`let's laugh!'}	\\
\Iter{}	& ap-apa-	& \multicolumn{2}{l}{`laugh again'}		\\
\Ptcp{}	& apayam	& \multicolumn{2}{l}{`laughing'}		\\

\bottomrule

\end{tabu}
\label{tab:vocconj2}
\end{table}

As described in \autoref{sec:markstrat}, Ayeri conjugates its main verbs,
canonically in agreement with the agent NP, and verb conjugation as such is
extremely pervasive. The basic conjugation paradigms are given in Tables
\ref{tab:monoconsconj}--\ref{tab:vocconj}.\footnote{Due to the agglutinating
structure of Ayeri it makes little sense to list the whole paradigm of verb
inflection for all possible affix combinations here, as the table would become
unreasonably large. Instead, the various sections below will contain examples
of use for all affixes.} Agreement causes verbs to reflect grammatical
categories of nominal entities, thus, verbs show agreement in person
(\First{}, \Second{}, \Third{}) and number (\Sg{}, \Pl{}); third persons are
again differentiated by gender (\M{}, \F{}, \N{}, \Inan{}; compare
\autoref{subsec:gender}). Verbs only have agreement proper with third persons;
their form, then, is the same as that of verbs with topicalized pronominal
inflection (see \autoref{subsec:perspro}).

Regarding person--number inflection, verbs may be divided into three classes: 
monoconsonantal, biconsonantal, and vocalic stems. As discussed in 
\autoref{sec:phonotactics}, Ayeri restricts the number of successive non-glide 
consonants to two, which has repercussions in the second person, since the 
conjugation suffix there is \rayr{/v}{-va}. Monoconsonantal roots are 
unaffected by this restriction, however, hence the conjugation suffixes can 
simply be appended as they are; this is illustrated with the verb 
\xayr{sobF/}{sob-}{teach, learn} in \autoref{tab:monoconsconj}. Verb stems 
ending in dental and velar plosives will naturally undergo palatalization in 
the third person animate, so for instance, the third person singular masculine 
of the verb \xayr{gurtF/}{gurat-}{answer} is \xayr{gurtFy}{guraca}{(he) 
answers}, and the third person feminine plural of \xayr{AbgF/}{abag-}{roam, 
wander} is \xayr{AbgFyenF}{abajen}{(they) roam, (they) wander}. Verbs whose 
stem ends in an affricate are treated as monoconsonantal roots as well, since 
the affricate occupies one consonant phoneme segment. Thus, the second 
person of \xayr{ItYF/}{ic-}{glide, slide} is not *\rayr{ItYv}{*icava}, but 
\xayr{ItYFv}{icva}{you glide, you slide}.

Since /v/ is neither a vowel nor a glide, as present in the non--second person
suffixes, an epenthetic \fw{-a-} is inserted between the stem and the
second-person suffix for verbs whose stem ends in -CC.\footnote{A \emph{root}
is understood here as the uninflected verb morpheme, for instance,
\rayr{AnFlF/}{anl-}, \rayr{ItYF/}{ic-}, \rayr{no/}{no-}, or \rayr{sob/}{sob-}.
A \emph{stem} may contain inflections and further inflectional affixes attach
to it; it may also host clitics.} This is illustrated in
\autoref{tab:biconsconj} for the verb \xayr{AnFlF/}{anl-}{bring}. The second
person conjugation of this verb is not *\rayr{AnFlFv}{*anlva}, since the
cluster \fw{-nlv-} is illegal, but \rayr{AnFlv}{anlava}. Since Ayeri treats
two successive instances of the same consonant as a single segment---there is
no gemination---verbs like \xayr{silFvF/}{silv-}{see} conjugate like
monoconsonantal roots with regards to consonant clusters. That is, the second
person of \rayr{silFvF/}{silv-} is not *\rayr{silFvv}{*silvava}, as one might
expect, but \rayr{silFvFv}{silvva}. A further exception to this are verbs
ending in -Cs, since -Cs-C- is commonly resyllabified as -C-sC- (see
\autoref{ch:phonology}, \autoref{fn:ssyl}). Thus, the second-person form of
\xayr{krFsF/}{kars-}{freeze} is not *\rayr{krFsv}{*karsava} as expected, but
\xayr{krFsFv}{karsva}{you freeze}.

Lastly, verb stems may end in a vowel, most commonly \fw{-a}. In these cases
as well, the conjugation suffixes may simply be appended to the stem. The
conjugation of this class is illustrated in \autoref{tab:vocconj} with the
verb \xayr{no/}{no}{want}. Verb stems ending in \fw{-a} undergo the regular
vowel lengthening process for the first person suffixes, hence, the
topicalized first-person singular form of \xayr{Ap/}{apa-}{laugh} is
\xayr{Apaaj}{apāy}{I laugh} (compare \autoref{tab:vocconj2}). Verb stems
ending in a diphthong in /ɪ/ are essentially treated as a hybrid of
monoconsonantal and vocalic stems, since the diphthong's final /ɪ/ is treated
as /j/ before a vowel: \xayr{plyj}{palayay}{I rejoice},
\xayr{pljv}{palayva}{you rejoice}.

As mentioned above, the person marking on verbs is essentially the same as the
topic-marked personal pronouns. This has further ramifications for
person-marking on verbs, however, in that in canonical cases, even fully
case-marked agent pronouns may act as person marking by means of cliticization.
Thus, any person-marking on verbs except third-person agreement is, in fact, a
topicalized pronoun clitic not only by diachronic origin. Unlike English, Ayeri
does not use agent pronouns in addition to person agreement on verbs. Consider
these two example sentences in English:

\pex
\a\label{ex:vbagrengnn}\begingl
	\gla John greets Mary. //
	\glb John greet-s Mary //
	\glc John greet-\Tsg{}.\Prs{} Mary //
\endgl

\a\label{ex:vbagrengpr}\begingl
	\gla He greets Mary. //
	\glb he greet-s Mary //
	\glc \TsgM{} greet-\Tsg{}.\Prs{} Mary //
\endgl

\xe

In these examples, the verb has an agreement suffix \fw{-s} which indicates
third person singular, present tense, whether the subject of the sentence is a
noun (\fw{John}) or a pronoun (\fw{he}), which acts as a free morpheme in
English. Now consider the Ayeri equivalents of these two examples, on the other
hand:\footnote{Most of the following account is taken nearly verbatim from a
previously published blog article, \citet{benung:verbagreement}. Some of the
Ayeri examples used in the following come from a list of samples I provided for
a bachelor's thesis at the University of Kent in March 2016, in private
conversation, on request.%
% I do not know what the author made of them---the questionnaire I filled out
% initially indicated that the thesis was probably on the syntactic typology of
% fictional languages regarding typical word-order correlations (VO correlating
% with head-initial order etc.). I hope that my reflections here do not preempt
% or invalidate the author's analyses should they still be in the process of
% writing or their submitted thesis be in the process of evaluation and
% grading. I would certainly like to learn about their analysis of my examples.
}

\pex % (1)
\a\label{ex:vbagr}\begingl
	\gla Ang manya {} Ajān sa Pila. //
	\glb ang man-ya Ø Ajān sa Pila //
	\glc \AgtT{} greet-\TsgM{} \Top{} ​Ajān[\TsgM{}] \Parg{} Pila[\TsgF{}]//
	\glft `Ajān greets Pila.' //
\endgl

\a\label{ex:vbclt}\begingl
	\gla Ang manya sa Pila. //
	\glb ang man=ya.Ø sa ​Pila //
	\glc \AgtT{} greet=\TsgM{}.\Top{} \Parg{} ​Pila[\TsgF{}] //
	\glft `He greets Pila.' //
\endgl

\xe

It is probably uncontroversial to analyze \rayr{/y}{-ya} in (\ref{ex:vbagr}) as
person agreement: \rayr{AgYaanF}{Ajān} is a male name in Ayeri while
\rayr{pil}{Pila} is a feminine one; the verb inflects for a masculine third
person, which tells us that it agrees with the one doing the greeting, Ajān.
Ajān is also who this is about, which is shown on the verb by marking for an
agent topic. In the second case, there is only anaphoric reference to Ajān; the
full agent NP is not realized. Very broadly thus, the verb marking here seems 
to be like in Spanish, where you can drop the subject pronoun:
% \footnote{However, we will see that it is probably more complicated than 
% this.}

\pex% (2)
Spanish:
\a\label{ex:vbagrspann}\begingl
	\gla Juan saluda a María. //
	\glb Juan salud-a a María //
	\glc John greet-\Tsg{} \Acc{} Mary //
	\glft `John greets Mary.' //
\endgl

\a\label{ex:vbagrspapr}\begingl
	\gla Saluda a María. //
	\glb salud-a a María //
	\glc greet-\Tsg{} \Acc{} Mary //
	\glft `He greets Mary.' //
\endgl

\xe

Example (\ref{ex:vbclt}) probably does not seem conspicious if we assume that
Ayeri is pro-drop, except that there is also topic marking for an agent there,
the controller of which I have so far assumed to be the person inflection on
the verb, in analogy with examples like the following:

\ex\label{ex:lampyaang} % (3)
\begingl
	\gla Lampyāng. //
	\glb lamp=yāng //
	\glc walk=\TsgM{}.\Aarg{} //
	\glft `He walks.' //
\endgl\xe

This raises the question whether in Ayeri, there is dropping of an agent
pronoun involved at all, which is why the person suffix in (\ref{ex:vbclt}) was
glossed as \emph{=ya.Ø} (\mbox{=\TsgM{}.\Top{}}) rather than just as \emph{-ya}
(-\TsgM{}). In turn, this question leads us to consider another characteristic
of Ayeri, namely that the topic morpheme on noun phrases is zero. That is, the
absence of overt case marking on a nominal element indicates that it is a
topic; the verb in turn marks the case of the topicalized NP with a (case)
particle preceding it. Pronouns as well show up in their unmarked form when
topicalized, which is why I am hesitant to analyze the pronoun in
(\ref{ex:protop}) as a clitic on the VP rather than an independent
morpheme:\footnote{Also, perhaps a little untypically, topic NPs in Ayeri are
not usually pulled to the front of the phrase (at least not in the written
language; see \cite[120--122]{lehmann2015}), so topic-marked pronouns stay 
in-situ. Which NP constitutes the topic of the phrase is marked on the verb 
right at the head of the clause. How and whether this can be justified in 
terms of grammatical weight (see, for instance, \cite[95--98]{wasow1997}) 
remains to be seen.}

\pex % (4)
\a\label{ex:fullsntc}\begingl
	\gla Sa manya ang Ajān {} Pila. //
	\glb sa man-ya ang ​Ajān Ø ​Pila //
	\glc \PatT{} greet-\TsgM{} \Aarg{} ​Ajān \Top{} ​Pila //
	\glft `It's Pila that Ajān greets.' //
\endgl

\a\label{ex:protop}\begingl
	\gla Sa manyāng ye. //
	\glb sa man=yāng ye.Ø //
	\glc \PatT{} greet=\TsgM{}.\Aarg{} \TsgF{}.\Top{} //
	\glft `It's her that he greets.' //
\endgl

\xe

What is remarkable, then, is that \rayr{ye}{ye} (\TsgF{}.\Top{}) is the very 
same form that appears as an agreement morpheme on the verb, just like 
\rayr{/y}{-ya} (\TsgM{}) in various examples above (also compare the examples 
in \autoref{subsec:perspro}):

\ex % (5)
\begingl
	\gla Ang purivaye yāy. //
	\glb ang puriva=ye.Ø yāy //
	\glc \AgtT{} smile=\TsgF{}.\Top{} \TsgM{}.\Loc{} //
	\glft `She smiles at him.' //
\endgl\xe

This also holds for all other personal pronouns. Moreover, 
\rayr{/yaaNF}{-yāng} as seen in examples (\ref{ex:lampyaang}) and 
(\ref{ex:protop}) may also be used as a free pronoun in equative statements 
with predicative nominals, as well as other such case-marked personal forms:

\pex % (6)
\a\begingl
	\gla Yeng mino. //
	\glb yeng mino //
	\glc \TsgF{}.\Aarg{} happy //
	\glft `She is happy.' //
\endgl
	
\a\begingl
	\gla Yāng naynay. //
	\glb yāng naynay //
	\glc \TsgM{}.\Aarg{} too //
	\glft `He is, too.' //
\endgl

\xe

\phantomsection\label{patagr} As for case-marked person suffixes on verbs, the
assumption so far has been  that they are essentially clitics, especially
since the following marking  strategy is the grammatical one in absence of an
agent NP (compare \autoref{clitics_postverb_person}, 
p.~\pageref{clitics_postverb_person}):

\pex\label{ex:passive} % (7)
\a\label{ex:manye}\begingl
	\gla Manye sa Pila. //
	\glb man-ye sa ​Pila //
	\glc greet-\TsgF{} \Parg{} ​Pila //
	\glft `Pila is being greeted.' //
\endgl
	
\a\label{ex:manyes}\begingl
	\gla Manyes. //
	\glb man=yes //
	\glc greet=\TsgF{}.\Parg{} //
	\glft `She is being greeted.' //
\endgl

\xe

The verb here agrees with the patient---or is it that person agreement suffixes
on verbs are generally clitics in Ayeri, even where they do not involve case
marking? There seems to be a gradient here between what looks like regular verb
agreement with the agent on the one hand, and agent or patient pronouns just
stacked onto the verb stem on the other hand. For an overview, compare
\autoref{tab:persinfltypes}. In this table, especially the middle,
transitional category is interesting in that what looks like verb agreement
superficially can still govern topicalization marking, which is indicated in
column II by an index `1'. Note that this behavior only occurs in transitive
contexts; there is no topic marking on the verb if the verb only has a single
NP dependent. Also consider that for example (b) in the type III transitive
cell the question is, whether this should not better be analyzed as
\AgtT{} …-\TsgM{}.\Top{} …-\Top{} …-\Parg{}, with co-indexing of the topic on 
the person inflection of the verb, making it structurally closer to type II.

\afterpage{%
\clearpage% Flush earlier floats (otherwise order might not be correct)
\begin{landscape}\centering
\mbox{}\vfill
\captionof{table}{Verb inflection types in Ayeri}
~\\
\begin{tabu} to \linewidth{H[2l] X[4] X[4] X[4]}
\tableheaderfont\toprule
%
	& Type I: Clitic pronouns
	& Type II: Transitional
	& Type III: Verb agreement
\\

\toprule

Inflectional categories
	& Person\newline
		Number\newline
		Case
	%
	& Person\newline
		Number\newline
		Case/Topic
	%
	& Person\newline
		Number
\\

\midrule

Examples (intransitive)
	& …=yāng\newline
		…=\TsgM{}.\Aarg{}
	%
	& ---
	%
	& …-ya₁ …-ang₁\newline
		…-\TsgM{} …-\Aarg{}
\\

\midrule

Examples (transitive)
	& sa₁ …=yāng …-Ø₁\newline
		\PatT{} …=\TsgM{}.\Aarg{} …-\Top{}
	%
	& ang₁ …=ya.Ø₁ …-as\newline
		\AgtT{} …=\TsgM{}.\Top{} …-\Parg{}
	%
	& \begin{tabu} to \linewidth {X[1] X[9]}
		a.	& ang₁ …-ya₁ …-Ø₁ …-as\newline
			\AgtT{} …-\TsgM{} …-\Top{} …-\Parg{} \medskip
		\\
		
		b.	& a₁ …-ya₂ …-ang₂ …-Ø₁\newline
			\PatT{} …-\TsgM{} …-\Aarg{} …-\Top{}
		\\
	\end{tabu}
\\

\bottomrule

\end{tabu}
\label{tab:persinfltypes}
\mbox{}\vfill
\end{landscape}
\clearpage
}

As for personal pronouns fused with the verb stem like in the first column, 
\citet{corbett2006} points out that

\blockcquote[99--100]{corbett2006}{In terms of syntax, pronominal affixes are
arguments of the verb; a verb with its pronominal affixes constitutes a full
sentence, and additional noun phrases are optional. If pronominal affixes are
the primary arguments, then they agree in the way that anaphoric pronouns agree
[…] In terms of morphology, pronominal affixes are bound to the verb; typically
they are obligatory […].}

\noindent This seems to be exactly what is going on for instance in
(\ref{ex:lampyaang}) and (\ref{ex:manyes}), where the verb forms a complete
sentence. It needs to be pointed out that Corbett includes an example from
Tuscarora, a native American polysynthetic language, in relation to the above
quotation. Ayeri should not be considered polysynthetic, however, since its
verbs generally do not exhibit relations with multiple NPs, at least as far as
person and number agreement is involved
\citep[45--46]{comrie1989}.\footnote{The topic NP marked on the verb may be a
different from the one with which the verb agrees in person and number, so
technically, Ayeri verbs \emph{may} agree with more than one NP in a very
limited way (compare \autoref{sec:markstrat}). Still, I would not analyze this
as polypersonal agreement, since there is only canonical verb agreement with
one constituent, that is, the agent NP. Topic marking should, in my opinion, be
viewed as a separate agreement relation, as pointed out in the quoted section
above.}

Taking everything written above so far into account, it looks much as though 
Ayeri is in the process of grammaticalizing personal pronouns into person 
agreement \parencites[42--45]{lehmann2015}[493--497]{vangelderen2011}. 
\citet{corbett2006} illustrates an early stage of such a process:

\pex% (8)
Skou \parencite[76--77]{corbett2006}:
\a\begingl
	\gla Ke móe ke=fue. \quad {\textup{(*}​Ke móe fue.\textup{)}} //
	\glb \TsgM{} fish \TsgM{}=​see.\TsgM{} {} //
	\glft `He saw a fish.' //
\endgl

\a\begingl
	\gla Pe móe pe=fu. \quad {\textup{(*}​Pe móe fu.\textup{)}} //
	\glb \TsgF{} fish \TsgF{}=​see.\TsgF{} {} //
	\glft `She saw a fish.' //
\endgl

\xe

What \citet{vangelderen2011} calls the \emph{subject cycle}, the
\textcquote[493]{vangelderen2011}{oft-noted cline expressing that pronouns can
be reanalyzed as clitics and agreement markers} applies here, and as well in
Ayeri. However, while she continues to say that in
\textcquote[494]{vangelderen2011}{many languages, the agreement affix
resembles the emphatic pronoun and derives from it}, Ayeri does at least in
part the opposite and uses the case-unmarked form of personal pronouns for
what resembles verb agreement most closely. This, however, should not be too
controversial either, considering that, for instance, semantic bleaching and
phonetic erosion go hand in hand with grammaticalization 
\parencites[136--137]{lehmann2015}[497]{vangelderen2011}.

As pointed out above in (\ref{ex:passive}), Ayeri usually exhibits verbs as
agreeing with agents and occasionally patients, not topics as such. This may
be a little counterintuitive since the relation between topics and subjects is
close, but is possibly due to the fact that the unmarked word order is VAP.
This means that agent NPs usually follow the verb, and does not seem too
unnatural to have an agreement relation between the verb and the closest NP
also when non-conjoined NPs are involved \citep[180]{corbett2006}. This
conveniently explains why verbs can agree with patients as well if the agent
NP is absent. Taking into account that the grammaticalization process is still
ongoing so that there is still some relative freedom in how morphemes may be
used if a paradigm has not yet fully settled \citep[148--150]{lehmann2015}
also makes this seem less strange. Verbs simply become agreement targets of
the closest semantically plausible nominal constituent. Ayeri seems to be
shifting from topics to subjects, and as a consequence the bond between agents
and verbs is strengthened due to their usual adjacency; developing verb
agreement with agents may be seen as symptomatic of this change.\footnote{When
translating things in Ayeri, I find myself very often using agent topics,
which may be because I am used to subjects proper. Supposing that this is also
what Ayeri prefers in-universe, it would make sense to assume the usual
grammaticalization path by which topics become subjects, thereby also leading
to subject-verb agreement by means of resumptive pronouns referring back to
left-dislocated topics 
\parencites[121--122]{lehmann2015}[499--500]{vangelderen2011}.
\citet[120]{lehmann2015} gives colloquial French \fw{Jean, je l'ai vu hier}
`John, I saw him yesterday' as an example here: the object clitic \fw{l'} (←
\fw{le} `him') may well develop into an agreement affix (also see
\citet[498]{vangelderen2011} on a dialect of Spanish in which, she argues,
this has happened). Ayeri in its present form does not use clitic doubling
\citep[153--161]{spencerluis2012}, however.}

\begin{table}[tp]\centering
\caption[The syntax and morphology of pronominal affixes]{The syntax and 
morphology of pronominal affixes \citep[101]{corbett2006}}
{\tabulinesep=\itemsep
\begin{tabu} to \linewidth {B[28l,m] | X[24c,m] | X[24c,m] | X[24c,m]}

\tabucline[1pt]{1-4}

Syntax:\bigstrut
	& non-argument%\bigstrut
	& \multicolumn{2}{c}{argument}%\bigstrut}
\\

\hline

\mbox{Linguistic element:}%\bigstrut
	& `pure' agreement marker
	& pronominal affix%\bigstrut
	& free pronoun%\bigstrut
\\

\hline

Morphology:
	& \multicolumn{2}{c|}{inflectional form}%\bigstrut}
	& free form%\bigstrut
\\

\tabucline[1pt]{1-4}

\end{tabu}
}
\label{ex:typproaffx}
\end{table}

Signs so far point towards Ayeri's person agreement in fact being more likely
enclitic pronominal affixes.
%, even what I had been thinking of as person agreement before 
% (that is, suffixes on the verb that only encode person and
% number, but not case).
The question is, then, how this might be corrobated. \citet{corbett2006}
offers a typology here, see \autoref{ex:typproaffx}. According to this
typology, a pronominal affix is syntactically an argument of the verb but has
the morphology of an inflectional form (compare
\autoref{clitics_postverb_person}, p.~\pageref{clitics_postverb_person}). If
we compare this to the gradient given in \autoref{tab:persinfltypes} above, it
becomes evident that type I definitely fulfills these criteria, and type II
does so as well, in fact, in that there is no agent NP that could serve as a
controller if the verb inflection in type II were `merely' an agreement
target. The inflection in type III, on the other hand, appears to have all
hallmarks of agreement in that there is a controller NP that triggers it, with
the verb serving as an agreement target. Moreover, the person marking on the
verb is not a syntactic argument of the verb in this case. As example
(\ref{ex:manye}) shows, however, marking of type III permits the verb to mark
more than one case role, which makes it slightly atypical, although verbs can
only carry a single instance of person marking \citep[103]{corbett2006}.
Regarding referentiality, the person suffixes on the verb in table 1, columns
I and II are independent means of referring to discourse participants
mentioned earlier, whereas the person suffix in III needs support from an NP
in the same clause as a source of morphological features to share:

\pex % (9)
\a\label{ex:agttopclit}\begingl
	\gla Ajān … Ang manya sa Pila. //
	\glb Ajān … Ang man=ya.Ø sa ​Pila //
	\glc Ajān … \AgtT{} greet=\TsgM{}.\Top{} \Parg{} ​Pila //
	\glft `Ajān … He greets Pila.' //
\endgl

\a\label{ex:agtproclit}\begingl
	\gla Ajān … Sa manyāng {} Pila. //
	\glb Ajān … Sa man=yāng Ø Pila //
	\glc Ajān … \PatT{} greet=\TsgM{}.\Aarg{} \Top{} ​Pila //
	\glft `Ajān … It's Pila that he greets.' //
\endgl

\a\label{ex:wrongagr}\ljudge* \begingl
	\gla Ajān … Manya sa Pila. //
	\glb Ajān … Man-ya sa ​Pila //
	\glc Ajān … greet-\TsgM{} \Parg{} ​Pila //
\endgl

\xe

Since person marking of the type I and II is \emph{referential}, as shown in
example (\ref{ex:agttopclit}) and (\ref{ex:agtproclit}), it can be counted as
a cliticized pronoun \citep[103]{corbett2006}. Pronouns in Ayeri can also
refer to non-people---there are both a `neuter' gender for non-people
considered living (or being closely associated with living things), and an
`inanimate' gender for the whole rest of things (compare
\autoref{subsec:gender}). Since mere agreement as in type III needs support
from an NP within the verb's scope, though, it does not have
\emph{descriptive/lexical content} of its own. That is, it \emph{only} serves
a grammatical function \citep[104]{corbett2006}, not strictly as an anaphora.
As for \citet{corbett2006}'s \emph{balance of information} criterion,
\autoref{tab:persinfltypes} also highlights differences in what information is
provided by the person marking. Nouns in Ayeri inherently bear information on
person, number, and gender, and all three types of person inflection on verbs
share these features. However, there are no extra grammatical features
indicated by the first two inflection types that are not expressed by noun
phrases, although under a very close understanding of \citet{corbett2006}, the
following example (\ref{ex:verbplagr}) may still qualify as person-marking on
the verb realizing a grammatical feature shared with an NP that is not openly
expressed by the NP. \citet{corbett2006} writes that in the world's languages, 
this frequently is number \citep[105]{corbett2006}. This, however, does not 
apply to Ayeri because the only time verbs display number not expressed 
overtly by inflection on a noun is in agreement like in type 
III\,(a):\footnote{From a Lexical-Functional Grammar point of view, the number 
feature of \rayr{kj}{kay} in (\ref{ex:verbplagr}) coalesces with the semantic 
features provided by \rayr{AyonF}{ayon} in the maximal projection; agreement 
is thus with the whole agent NP rather than just with \rayr{AyonF}{ayon} as 
the NP's categorial head.}

\ex\label{ex:verbplagr} % (10)
\begingl
	\gla Ang sahayan ayon kay kong nangginoya. //
	\glb ang saha-yan ayon-Ø kay kong nanggino-ya //
	\glc \AgtT{} come-\TplM{} man-\Top{} three into tavern-\Loc{} //
	\glft `Three men come into a pub.' //
\endgl\xe

As shown above, verb marking of the types I and II is independent as a
reference, so there is \emph{unirepresentation} of the marked NP. In contrast,
verb marking of type III requires a controlling NP in the same clause to share
grammatical features with, so that there is \emph{multirepresentation} typical
of canonical agreement \citep[106]{corbett2006}.
% Note that unirepresentation as
% outlined here is probably different from pro-drop, as in this case I would
% expect sentences like (\ref{ex:wrongagr}) to be grammatical
% \citep[107]{corbett2006}.
A further property that hinges on types I and II
being independent pronouns glued to verbs as clitics is that they are not
coreferential with another NP of the same grammatical relation, but are in
complementary distribution, as commonly assumed with pronominals
\citep[108]{corbett2006}. Hence, either of these two examples is ungrammatical:

\pex % (11)
\a\ljudge* \begingl
	\gla Lampyāng ang Ajān. //
	\glb lamp=yāng ang ​Ajān //
	\glc walk=\TsgM{}.\Aarg{} \Aarg{} Ajān //
\endgl

\a\ljudge* \begingl	
	\gla Ang lampyāng {} Ajān. //
	\glb ang lamp=yāng Ø ​Ajān //
	\glc \AgtT{} walk=\TsgM{}.\Aarg{} \Top{} ​Ajān //
\endgl

\xe

However, verb agreement with a free pronoun is also not possible even though
it might be expected according to \citep[109]{corbett2006}---also compare
example (\ref{ex:vbagrengpr}) above. Instead, the agent pronoun replaces any
possible person agreement on the verb (see \autoref{clitics_postverb_person},
p.~\pageref{clitics_postverb_person}~ff., for an attempt to explain this effect
from a syntactic point of view):

\pex % (12)
\a\begingl
	\gla Lampyāng. //
	\glb lamp=yāng //
	\glc walk=\TsgM{} //
	\glft `He walks.' //
\endgl

\a\ljudge* \begingl	
	\gla Lampya yāng. //
	\glb lamp-ya yāng //
	\glc walk-\TsgM{} \TsgM{}.\Aarg{} //
\endgl

\xe

In conclusion, we may assert that Ayeri appears to be in the process of
grammaticalizing pronouns as verb infletions, however, how far this
grammaticalization process has progressed is dependent on syntactic context.
Ayeri displays a full gamut from personal pronouns (usually agents) glued to
verbs as clitics to agreement with coreferential NPs that is transparently
derived from these personal pronouns. With the latter, the complication arises
that coreferential pronoun NPs are not allowed as agreement controllers as one
might expect, but only properly nominal NPs. Slight oddities with regards to
Austronesian alignment---Ayeri's actors bear more similarities to subjects than
expected, but still without fully conflating the two notions---can possibly be
explained by a strengthening of the verb-agent relationship pointed out as a
grammaticalization process in this article as well. Information on agreement
with committee nouns and coordinated NPs with incongruent agreement features
can be found in the section on VPs.

\index{agreement!number|)}
\index{agreement!person|)}

\subsection{Tense}
\label{subsec:tense}
\index{tense|(}

Tense in Ayeri is often not explicitly marked, but has to be inferred from
context. However, where marked, Ayeri distinguishes past and future as
referring to past and future events, respectively. Both past and future tenses
come with three degrees each: near, recent/impending, and remote. Ayeri's
distinguishing three degrees of both past and future time is a little unusual
with regards to typology according to the survey conducted by
\citet[127]{dahl1985}. The decision for which subtier of the past and the
future to use is up to pragmatics, that is, there are no definitive and 
clear-cut lines. The near-time markers are most commonly used for immediate 
scope, that is, things which have just happened or will happen in a moment. 
The recent/impending-time markers may then be used for anything else which 
does not qualify as remote, that is, a long time into the past or the future 
from the point of view of the speaker.

\citet[117]{dahl1985} further notes that among the languages in the surveyed
sample, past tenses are mostly marked by suffixes, the marking of this category
being extremely common in addition. Ayeri may thus be a little unusual
crosslinguistically again by exclusively using prefixes for tense marking. This
makes sense, however, if we assume that historically, the tense prefixes once
were auxiliary verbs. Ayeri applies head-initial word order to subordinating
verbs, as we will see further below, so these prefixes may just have begun to
\emph{pro}cliticize instead of slipping into a position behind their head (that
is, Wackernagel's position).

Of the triad tense--aspect--mood this section will only cover basic uses of 
the marked tense categories, followed by a discussion of complex tense 
combinations such as past-in-future. The subsequent \autoref{subsec:aspect} 
will provide more insight into the morphological marking of aspectual 
categories; \autoref{subsec:mood} deals with the morphology of mood marking in 
Ayeri.

\subsubsection{Present tense}
\index{tense!present|(}
Verbs in Ayeri are unmarked for present tense, since it is the normal mode of 
speaking. Besides being used to comment or report on current events, the 
present tense is also used to make statements of general truth:

\ex\begingl
	\gla Sa arapyo tahanyamanang koyana nogalam-ikan. //
	\glb sa arap-yo tahanyaman-ang koya-na nogalam-Ø=ikan //
	\glc \PatT{} require-\TsgN{} writing-\Aarg{} book-\Gen{} 
		patience-\Top{}=much //
	\glft `Writing a book requires much patience.' //
\endgl\xe

Moreover, Ayeri does not strictly mark its verbs for past tense in narrative 
discourses---verbs may thus appear as though with a present-time reference in 
spite of recounting past events, whether historical or fictional. See the next 
subsection on the past tense.

\index{tense!present|)}

\subsubsection{Past tense}
\label{subsubsec:past}
\index{tense!past|(}
The past tense indicates actions in the past if not further modified. The three
degrees of past tense are marked with \rayr{k/}{kə-} (near/immediate),
\rayr{m/}{mə-} (recent), and \rayr{v/}{və-} (remote), which attach right in
front of a verb root. In spite of the customary spelling of the past tense
prefixes with \orth{ə}, which reflects pronunciation, they have an underlying
/a/ vowel in this place. This means that the vowel of the tense prefixes
coalesces with a following /a/ to form a long vowel (see
\autoref{subsec:vowels}), which is demonstrated in example (\ref{ex:pst})
below:\index{allomorphy}

\pex
\a\label{ex:npst}\begingl
	\gla Ang kəsilvay yes motonya. //
	\glb ang kə-silv=ay.Ø yes moton-ya //
	\glc \AgtT{} \NPst{}-see=\Fsg{}.\Top{} \TsgF{}.\Parg{} store-\Loc{} //
	\glft `I've just seen her at the store.' //
\endgl

\a\label{ex:pst}\begingl
	\gla Le mādruyāng ikan biratay. //
	\glb le mə-adru=yāng ikan biratay-Ø //
	\glc \PatTI{} \Pst{}-break=\TsgM{}.\Aarg{} wholly pot-\Top{} //
	\glft `The pot, he completely broke it.' //
\endgl

\a\label{ex:rpst}\begingl
	\gla Vəmittang edaya. //
	\glb və-mit=tang edaya //
	\glc \RPst{}-live=\TplM{}.\Aarg{} here //
	\glft `They lived here (a long time ago)' //
\endgl

\xe

Note that the recent and the remote past tense are not generally marked if the 
past context is clear, for instance, when a past context has already been 
established in discourse. This may also happen explicitly by using a time 
adverbial such as \xayr{tml}{tamala}{yesterday} or \xayr{perikYnFy menNF 
pNF}{pericanya menang pang}{a hundred years ago}. In the presence of an 
explicit time adverbial, redundant tense marking is also dropped subsequently:

\ex\begingl
	\gla Ang kondayn kadanya terpasānley bihanya sarisa. //
	\glb ang kond=ayn.Ø kadanya terpasān-ley bihan-ya sarisa //
	\glc \AgtT{} eat=\Fpl{}.\Top{} together lunch-\PargI{} week-\Loc{}
		previous //
	\glft `We had lunch together last week.' //
\endgl\xe

The reference to a past time frame is explicitly given in this example by the
adverbial phrase \xayr{bihnFy sris}{bihanya sarisa}{last week}, hence the verb
appears here simply as \rayr{koMdjnF}{kondayn}, rather than with redundant
past-tense marking as \rayr{mkoMdjnF}{məkondayn}. Since past tense is often
underspecified in Ayeri, the language also does not employ past forms in
narrative contexts like English, among others, commonly does:

\ex\label{ex:neuromancer}
	The sky above the port was the color of television, tuned to a dead 
	channel. \tc{\citep[9]{gibson:neuromancer}}
\xe

This quote is, of course, the first sentence of 
\citet{gibson:neuromancer}'s novel \citetitle{gibson:neuromancer}, which 
never mentions any definite dates, but is clearly set in a future world, maybe 
somewhere in the latter half of the twenty-first century.\footnote{%
\citet{christian2017} reports that Gibson himself pictured his novel as set 
around 2035, though he has since realized that this cannot be right. One of 
the characters, the Finn, \textquote{makes an offhand reference to the 
\enquote{Act of ‘53} as a law [which] deals with the citizenship status of 
artificial intelligences} (\cite{christian2017}; also compare 
\cite[92]{gibson:neuromancer})---this is very unlikely to refer to 1953.} Yet, 
however, \citet{gibson:neuromancer} recounts events which are logically 
happening in an imagined future as having already happened in the past: he 
uses the past tense as a convention of storytelling. What Ayeri, then, does in 
contrast to English is to basically treat stories as though happening in the 
present; adverbials referring to past time may, again, set up the correct time 
frame if required. Ayeri is in good company here, since according to 
\citet{dahl1985} \textcquote[113]{dahl1985}{[m]ore common than marking 
narrative contexts [...] is not marking them---quite a considerable number of 
languages use unmarked verb forms in narrative contexts}. This, however, is 
yet different from a narrative present, that is, the use of present tense 
within a past context, which languages like English may use in narrative 
contexts to increase the feeling of immediacy and thus raise suspense. The 
following example from an Ayeri translation of the well-known Aesopian fable, 
`The North Wind and the Sun' (compare \cite[39]{ipa2007}), illustrates 
Ayeri's non-marking of tense on verbs in narrative contexts:

\ex
\begingl
	\gla Ang manga ranyon adauyi {} Pintemis nay {} Perin, engyo mico 
		sinyāng luga toya, lingya si lugaya asāyāng si sitang-naykonyāng 
		kong tovaya mato. //
	\glb ang manga ran-yon adauyi Ø Pintemis nay Ø Perin eng-yo mico 
		sinya-ang luga toya ling-ya si luga-ya asāya-ang si 
		sitang-naykon=yāng kong tova-ya mato //
	\glc \AgtT{} \Prog{} argue-\TplN{} then \Top{} {North Wind} and 
		\Top{} Sun, be.more-\TsgN{} strong who-\Aarg{} among 
		\TplN{}.\Loc{}, while-\Loc{} \Rel{} pass-\TsgM{} 
		traveler-\Aarg{} \Rel{} self-wrap=\TsgM{}.\Aarg{} inside 
		cloak-\Loc{} warm. //
	\glft `The North Wind and the Sun were then arguing which among them is 
		stronger, all the while a traveler passed by who had wrapped 
		himself in a warm cloak.' //
\endgl
\xe

\index{tense!past|)}

\subsubsection{Future tense}
\label{subsubsec:future}
\index{tense!future|(}

Future tense marks explicit references to future time in Ayeri, that is,
\textcquote[103]{dahl1985}{someone's plans, intentions or obligations}, as well
as predictions. The future prefixes behave analogously to the ones indicating
past tense: \rayr{p/}{pə-} indicates immediate/near future (\NFut{}),
\rayr{se/}{sə-} indicates impending future (\Fut{}), and \rayr{ni/}{ni-}
indicates remote future (\RFut{}). Underlying the reduced vowels in
\rayr{p/}{pə-} and \rayr{se/}{sə-} are /a/ and /e/, respectively, so that these
prefixes cause adjacent vowels of the same type to lengthen as
usual;\index{allomorphy} the same, of course, applies to \rayr{ni/}{ni-}
regarding /i/. The following examples show the future tense markers in context:

\pex
\a\label{ex:nfut}\begingl
	\gla Pəsahayang! //
	\glb pə-saha=yang //
	\glc \NFut{}-come=\Fsg{}.\Aarg{} //
	\glft `I'm coming (in a moment)!' //
\endgl

\a\label{ex:fut}\begingl
	\gla Ang səkarsayn kankaya. //
	\glb ang sə-kars=ayn.Ø kanka-ya //
	\glc \AgtT{} \Fut{}-freeze=\Fsg{}.\Top{} snow-\Loc{} //
	\glft `We will freeze in the snow.' //
\endgl

\a\label{ex:rfut}\begingl
	\gla Paronatang, nisa-sahaya dihakayāng. //
	\glb parona=tang ni-sa\til{}saha-ya dihakaya-ang //
	\glc believe=\TplM{}.\Aarg{} \RFut{}-\Iter{}\til{}come-\TsgM{} 
		prophet-\Aarg{} //
	\glft `They believe that the prophet will return (one day).' //
\endgl

\xe

Like the past tense, the future is often not explicitly marked if the time 
frame is clear from context or has been clarified with such adverbials as 
\xayr{tsel}{tasela}{tomorrow}, \xayr{mNsh perikYnFy}{mangasaha pericanya}{in 
a year}, or \xayr{metj}{metay}{sometime}:

\ex\begingl
	\gla Ang raypāy vaya bihanya mararya. //
	\glb ang raypa=ay.Ø vaya bihan-ya mararya //
	\glc \AgtT{} stop=\Fsg{}.\Top{} \Ssg{}.\Loc{} week-\Loc{} next //
	\glft `I'm stopping by you next week.' //
\endgl\xe

It is possible here to explicitly mark the verb for future tense as well, for
example, to make a promise, or to otherwise emphasize that the future condition
will come to pass:

\ex\begingl
	\gla Səsidejang tasela, diran. //
	\glb sə-sideg=yang tasela diran //
	\glc \Fut{}-repair=\Fsg{}.\Aarg{} tomorrow uncle //
	\glft `I \fw{will} repair it tomorrow, uncle.' //
\endgl\xe

\index{tense!future|)}

\subsubsection{Past in past}
\index{tense!past|(}

So far, we have only dealt with tense marking from the point of view of the
present. However, it is also possible to refer to an event which precedes
another event in the past. Ayeri makes little use of auxiliary verbs, and thus
the regular morphological and pragmatic means of tense marking have to cover
this relation as well. In order to indicate pre-past events, it is customary to
explicitly mark the verb for past time in Ayeri, in difference to the common
lack of morphological marking for plain past tense. However, as it is possible
for the \rayr{m/}{mə-} prefix to be used to refer to `regular' past events from
a present point of view as well, context again has to provide the information
that the frame of reference is past in this case, rather than the speaker's
present.

\ex
\begingl
	\glpreamble \textsc{context:} Ajān's past travels //
	\gla Ya məsaraya iri maritay ang Ajān {} Tasankan //
	\glb ya ma-sara-ya iri maritay ang Ajān Ø Tasankan //
	\glc \LocT{} \Pst{}-go-\TsgM{} already before \Aarg{} Ajān \Top{} %
		Tasankan //
	\glft `Tasankan, Ajān had already gone there before.' //
\endgl
\xe

The above example is essentially ambiguous as to the reference point. The
explicit tense marking draws attention to the fact that the event definitely
lies in the past and the adverbs underline this fact. Instead of reading the
sentence as referring to a pre-past event, it is equally possible to read it
from a present-time point of view as `Ajān has already gone to Tasankan
before', although under these circumstances, it would be more common to leave
the \rayr{m/}{mə-} out, as described in \autoref{subsubsec:past}:

\ex
\begingl
	\glpreamble \textsc{context:} Ajān's current traveling plans //
	\gla Ya saraya iri maritay ang Ajān {} Tasankan //
	\glb ya sara-ya iri maritay ang Ajān Ø Tasankan //
	\glc \LocT{} go-\TsgM{} already before \Aarg{} Ajān \Top{} %
		Tasankan //
	\glft `Tasankan, Ajān already went there before.' //
\endgl
\xe

Likewise, it is possible to make plans in the past with the intention of them 
coming to frutition only later, possibly at a point before the current time or 
even further in the future. The English idiom to express this time relation is 
`was going to'; in Ayeri, the relation cannot be expressed by morphological 
means, but only by lexical ones. Thus, \xayr{no/}{no-}{want; plan to} must 
be used, together with explicit past marking. Since \rayr{no/}{no-} is a modal 
particle (see \autoref{subsec:modals}), inflection is placed on the content 
verb.

\ex
\begingl
	\glpreamble \textsc{context}: Ajān's having gone to Tasankan //
	\gla Ang no məinca tosantangyeley hiro yam Pila. //
	\glb ang no ma-int=ya tosantang-ye-ley hiro yam Pila //
	\glc \AgtT{} want \Pst{}-buy=\TsgM{}.\Top{} earring-\Pl{}-\PargI{} new 
		\Dat{} Pila //
	\glft `He had planned to buy new earrings for Pila.' //
\endgl
\xe

The time relation expressed here is, thus, essentially that of a pre-past 
event again, since the planning of the action of buying took place before 
the time of going to Tasankan.

\index{tense!past|)}

\subsubsection{Past in future}
\index{tense!future|(}
\index{tense!past|(}

Of course, it is also possible to refer to future actions or events which will
already have happened before a point further in the future. From the point of
view of the later event, the closer event will thus already lie in the past,
forming its prerequisite. As with future-in-past, there is no way in Ayeri to
mark this relation morphologically, but lexical means have to be used, that is,
first and foremost the adverb \xayr{Iri}{iri}{already}, which indicates that an
action has been completed in the past. As with other future actions, the time
frame must be inferred from context if it is not indicated explicitly by
temporal adverbs or future-tense marking (compare
\autoref{subsubsec:future}).

\ex
\begingl
	\glpreamble \textsc{context}: Ajān's traveling to Tasankan //
	\gla Ang girenja iri nilay sirutayya tamala pesan. //
	\glb ang girend=ya.Ø iri nilay sirutay-ya tamala pesan //
	\glc \AgtT{} arrive=\TsgM{}.\Top{} already 
		probably evening-\Loc{} tomorrow before //
	\glft `He will probably already (have) arrive(d) before 
		tomorrow evening.' //
\endgl
\xe

Strictly speaking, the above example does not make it explicit whether Ajān 
\emph{will arrive} before evening or \emph{will have arrived}. In order to 
indicate that the action is all but complete, the cessative adverb 
\xayr{myis}{mayisa}{be done; ready} may be added:

\ex
\begingl
	\gla Girenjāng \textbf{mayisa} iri. //
	\glb girend=yāng mayisa iri //
	\glc arrive=\TsgM{}.\Aarg{} be.done already //
	\glft `He already has arrived', or: `He will already have arrived.' //
\endgl
\xe

\index{tense!past|)}
\index{tense!future|)}

\index{tense|)}

\subsection{Aspect}
\label{subsec:aspect}
\index{aspect|(}

Aspectually unmarked verb forms indicate general statements, which may be
completed or ongoing, depending on the meaning of the verb itself. Ayeri seems
not to make strict formal distinctions with regards to either, perfectivity or
lexical aspect. It needs to be noted, however, that at least to date, it is not
entirely clear how Ayeri fares with regards to conceptualizing perfectivity,
which \citet[76]{dahl1985} in reference to
\citet[16]{comrie1976} characterizes as being based on the conceptualization of
actions or events as bounded or otherwise limited wholes, versus a lack of
closure. \citet{dahl1985} also notes that \textcquote[69]{dahl1985}{it
seems rather to be a typical situation that even in individual languages, we
cannot choose one member of the opposition [perfective--imperfective] as being
clearly unmarked}. He further argues that

\blockcquote[73]{dahl1985}{The difficulty of deciding which member of the
opposition is marked and which is unmarked is connected with the tendency for
\textsc{pfv:ipfv} to be realized not by affixation or by periphrastic
constructions but  rather by less straightforward morphological processes.}

In other words: it \emph{is} a difficult category to assess, in spite of being
\textcquote[69]{dahl1985}{often taken to be \enquote{the} category of aspect},
mostly since languages often do not realize it by straightforward means. In
Ayeri, the most tangible way of expressing completeness of an action is to use
adverbs like \xayr{myis}{mayisa}{ready, done}, \xayr{Iri}{iri}{already},
\xayr{IknF}{ikan}{completely, wholly} (also as an adjective); a quantifier like
\xayr{/henF}{-hen}{all}; verbs like \xayr{smirF/}{samir-}{finish},
\xayr{pN/}{panga-}{end}, and \xayr{rjp/}{raypa-}{stop}; or an indefinite
pronoun like \xayr{EnY}{enya}{everything, everybody}:

\ex\begingl
	\gla Le kondjeng enya. //
	\glb le kond=yeng enya-Ø //
	\glc \PatTI{} eat=\TsgF{}.\Aarg{} everything //
	\glft `She ate everything.' or: `She ate it all up.' //
\endgl\xe

Apart from the more general dilemma of determining how perfectivity is
expressed in detail, Ayeri marks verbs openly by morphological means to
indicate progressive, habitual, and iterative actions---by their nature all
conceptualizing actions as being composed of a series of two or more related
actions of the same kind, though not necessarily implying a strong semantic
connection to the past. The following sections will discuss each of these
categories.

\subsubsection{Progressive}
\index{aspect!progressive|(}

In order to indicate an ongoing action explicitly, Ayeri employs the marker
\rayr{mN}{manga}, which we have already seen with directional prepositions
above (\autoref{manga}). This marker is a clitic within the verb phrase and
precedes the verb word:

\ex\label{ex:presprog}\begingl
	\gla Ang manga ilye karonas nakajyam. //
	\glb ang manga il=ye.Ø karon-as naka-ye-yam //
	\glc \AgtT{} \Prog{} give=\TsgF{}.\Top{} water-\Parg{} 
		plant-\Pl{}-\Dat{} //
	\glft `She is giving water to the plants.' //
\endgl\xe

Going by the data presented by \citet[91]{dahl1985}, Ayeri is typologically
unremarkable in marking progressive aspect with a periphrastic construction,
although it is remarkable in possessing morphological progressive marking at
all---it only occurs in 27\pct{} of the languages in \citet{dahl1985}'s
sample. Typical of progressives, this form of the verb is not limited to
present contexts in Ayeri as exemplified in (\ref{ex:presprog}) above. Instead,
it is possible to also use the progressive in past (\ref{ex:pastprog}) and
future (\ref{ex:futprog}) contexts, the latter being probably less typical,
though:

\pex\label{ex:nonpresprog}
\a\label{ex:pastprog}\begingl
	\gla Ang manga gumya {} Ajān tadayya si ya kongaye ang Pila gumanga 
		tamala. //
	\glb ang manga gum-ya Ø Ajān taday-ya si ya konga-ye ang Pila 
		gumanga-Ø tamala //
	\glc \AgtT{} \Prog{} work-\Tsg{} \Top{} Ajān time-\Loc{} \Rel{} \LocT{} 
		enter-\TsgF{} \Aarg{} Pila workshop-\Top{} yesterday //
	\glft `Ajān was working when Pila entered the workshop yesterday.' //
\endgl

\a\label{ex:futprog}\begingl
	\gla Ang manga nimpay rangya nā tadayya si cunyo bekalang tasela. //
	\glb ang manga nimp=ay.Ø rang-ya nā taday-ya si cun-yo bekal-ang 
		tasela //
	\glc \AgtT{} \Prog{} run=\Fsg{}.\Top{} home-\Loc{} \Fsg{}.\Gen{} 
		time-\Loc{} \Rel{} begin-\TsgN{} festival-\Aarg{} tomorrow //
	\glft `I will be running home when when the festival starts 
		tomorrow.' //
\endgl

\xe

Ignoring the constructedness of the above examples, the time adverb is located 
in the relative clause in both sentences in this case. For illustrative 
purposes, let us assume that a narrative context with the respective time 
frames has already been established in (\ref{ex:nonpresprog}). As noted above, 
Ayeri prefers to not mark every verb for tense explicitly when the context is 
clear already, insofar the argument that progressive aspect works independent 
of \fw{tense} needs corrobation; the question being whether constructions like 
\rayr{mN m/—}{manga mə-...} (\Prog{} \Pst{}-...) are possible. Strictly 
speaking, there is nothing to prevent this construction, however, we have to 
wonder if it is actually \fw{natural} to phrase things this way. What can be 
said at least is that progressive marking is possible within a context 
referring to past or future actions and events irrespective of their explicit 
marking on the verb. Furthermore, the examples in (\ref{ex:nonpresprog}) 
illustrate a very typical use of the progressive as a structuring means, that 
is, an ongoing background action may be expressed using a progressive form, 
while an interrupting action receives no special marking (compare the past 
progressive in English).

\index{aspect!progressive|)}

\subsubsection{Habitual}
\index{aspect!habitual|(}

Unlike the few instances of habitual marking in \citet{dahl1985}'s survey
\citep[96]{dahl1985}, Ayeri possesses a suffix for marking habitual actions on
the verb: \rayr{/As}{-asa}, where the first \fw{-a} replaces the terminal vowel
of a verb stem if present, compare example (\ref{ex:habvwl}) below. The
habitual aspect in Ayeri stresses that an action is carried out as a habit,
that is, not just a few times, but with regular frequency. Essentially, verbs
marked with the habitual in Ayeri can be translated by adding the adverb
\fw{usually} in English \citep[97]{dahl1985}. The habitual aspect is not
restricted to present actions or absolute statements like the one in
(\ref{ex:habcns}), but can also be used in past contexts to express that
something \fw{used to} be done in the past as in (\ref{ex:habvwl}).
While the contexts are probably very few, there are no restrictions about using
the habitual also in contexts relating to future actions which are predicted to
be carried out habitually. The following sentences illustrate typical contexts
in which the habitual may be used:

\pex
\a\label{ex:habcns}\begingl
	\gla Le kondasayāng hemaye pruyya nay napayya kayvay. //
	\glb le kond-asa=yāng hema-ye-Ø pruy-ya nay napay-ya kayvay //
	\glc \PatTI{} eat-\Hab{}=\TsgM{}.\Aarg{} egg-\Pl{}-\Top{} salt-\Loc{} 
		and pepper-\Loc{} without //
	\glft `He always eats his eggs without salt and pepper.' //
\endgl

\a\label{ex:habvwl}\begingl
	\gla Ang ajasāyn ranisungas tadayya si yāng ganas. //
	\glb ang aja-asa=ayn.Ø ranisung-as taday-ya si yāng gan-as //
	\glc \AgtT{} play-\Hab{}=\Fpl{}.\Top{} hide.and.seek-\Parg{} 
		time-\Loc{} \Rel{} \Fsg{}.\Aarg{} child-\Parg{} //
	\glft `We used to play hide-and-seek when I was a child.' //
\endgl

\xe

Importantly, the verb root with habitual marking forms a new verb stem to which
affixes may be attached. This is relevant to mood suffixes, which follow
aspectual marking.

\index{aspect!habitual|)}

\index{aspect|)}

\subsubsection{Iterative}
\label{subsubsec:iterative}
\index{aspect!iterative|(}

The iterative aspect marks actions that are repeated at least once by
reduplication. The equivalent in English is to use the adverb \fw{again} or the
prefix \fw{re-}. Iterative reduplication in Ayeri is only partial, in that only
the initial CV- or VC- of a verb root is repeated---there are no verb roots
which consist of only a single consonant or vowel. Complications begin,
however, if the verb root starts with a consonant cluster (not unusual), or a
diphtong (rare). In the case of an intial consonant cluster, the cluster is
simplified to only include the first consonant; for initial diphthongs, there
is no necessity to include the first available consonant, since the secondary
vowel of a diphthong can by itself act as a semivowel to make up for the vowel
hiatus.

\ex\labels
	\begin{tabular}[t]{@{\tl\quad} l @{\enspace→\enspace} l @{\smallskip}}
	\xayr{\larger kut/}{kuta-}{thank}
		& \xayr{\larger ku/kut/}{ku-kuta-}{thank again}
		\\
	\xayr{\larger AmNF/}{amang-}{happen}
		& \xayr{\larger AmF/AmNF/}{am-amang-}{happen again}
		\\
	\xayr{\larger pFrMtF/}{prant-}{ask}
		& \xayr{\larger p/pFrMtF/}{pa-prant-}{ask again}
		\\
	\xayr{\larger AjrinF/}{ayrin-}{set}
		& \xayr{\larger Aj/AjrinF/}{ay-ayrin-}{set again}
		\\
	\end{tabular}
\xe

The reduplicated stem works as a new stem for other prefixes, that is, no 
morphological material can go between the reduplicated part and the lexical 
stem proper; the following example also shows that there is, again, no 
restriction on the iterative aspect with regards to tense:

\ex\begingl
	\gla Məku-kutayāng. \quad \textup{(*}Ku-məkutayāng\textup{)} //
	\glb mə-ku\til{}kuta=yāng //
	\glc \Pst{}-\Iter{}\til{}thank=\TsgM{}.\Aarg{} //
	\glft `He thanked again.' //
\endgl\xe

Iterative reduplication is lexicalized at least in one verb, 
\xayr{s/sh/}{sa-saha-}{return}. Besides the meaning of `again', iterative 
reduplication may also indicate the meaning `back', for instance in the 
following example:

\ex\begingl
	\gla Ta-tapyu adaley! //
	\glb ta\til{}tapy-u ada-ley //
	\glc \Iter{}\til{}put-\Imp{} that-\PargI{} //
	\glft `Put that back!' //
\endgl\xe

In addition to a simple iterative meaning, a frequentative meaning like `walk 
around', `cry all the time', or `keep asking' can be achieved by combining the 
iterative and progressive aspects, that is, the verb is both modified by 
\rayr{mN}{manga} for progressive aspect and partial initial reduplication for 
iterative aspect:

% FIXME: Can these be nominalized? If so, how?
\pex
\a\begingl
	\gla Ang manga la-lampay saha-sara manga luga bahisya-hen. //
	\glb ang manga la\til{}lamp=ay.Ø saha-sara manga luga bahis-ya=hen //
	\glc \AgtT{} \Prog{} \Iter{}\til{}walk=\Fsg{}.\Top{} back.and.forth 
		\Dir{} while day-\Loc{}=all //
	\glft `I was walking around back and forth all day long.' //
\endgl

\a\begingl
	\gla Ang manga si-sipye kimay sirutayya. //
	\glb ang manga si\til{}sip-ye kimay.Ø sirutay-ya //
	\glc \AgtT{} \Prog{} \Iter{}\til{}cry-\TsgF{} baby.\Top{} 
		night-\Loc{} //
	\glft `The baby, she is crying all the time at night.' //
\endgl

\a\begingl
	\gla Manga pa-prantu! //
	\glb manga pa\til{}prant-u //
	\glc \Prog{} \Iter{}\til{}ask-\Imp{} //
	\glft `Keep asking!' //
\endgl

\xe

\index{aspect!iterative|)}

\subsubsection{Lexically marked aspectual categories}

Besides using morphological means, Ayeri expresses some aspectual categories by
way of lexical items, that is, verbs and adverbs. The relevant words in this
respect are the adverbs \xayr{sirimNF}{sirimang}{about to} (prospective) and
\xayr{myis}{mayisa}{ready; be done} (cessative), as well as 
the verb \xayr{kYunF/}{cun-}{begin, start} (inchoative):

\ex\label{ex:prospective}\begingl
	\gla Saratang sirimang. //
	\glb sara=tang sirimang //
	\glc leave=\TplM{}.\Aarg{} about.to //
	\glft `They are about to leave.' //
\endgl\xe

\ex~\label{ex:cessative}\begingl
	\gla Konjang mayisa. //
	\glb kond=yang mayisa //
	\glc eat=\Fsg.\Aarg{} be.done //
	\glft `I am done eating.' //
\endgl\xe

\ex~\label{ex:inchoative}\begingl
	\gla Pəcunreng seyaryam. //
	\glb pə-cun=reng seyar-yam //
	\glc \NFut{}-begin=\TsgI{}.\Aarg{} rain-\Ptcp{} //
	\glft `It is going to start raining any moment.' //
\endgl\xe

Prospective \rayr{sirimNF}{sirimang} (\ref{ex:prospective}) and cessative
\rayr{myis}{mayisa} (\ref{ex:cessative}) are expressed by adverbs which are
regularly following verbs as their heads. They precede other adverbs due to a
higher amount of semantic bondedness, by tendency, than other descriptive
adverbs. For this reason, as well as for expressing a grammatical function rather
than lexical meaning with the original meaning still transparent, they appear
to be on the verge of grammaticalization. In contrast, the verb inchoative 
\rayr{kYunF/}{cun-} (\ref{ex:inchoative}) is part of a periphrastic verb 
construction, that is, \rayr{kYunF/}{cun-} requires a content-verb VP as a 
complement rather than an NP. The content/main verb appears in a non-finite 
form marked by \rayr{/ymF}{-yam}, which will be described below.

\subsection{Mood}
\label{subsec:mood}
\index{mood|(}

Besides various aspects, Ayeri also marks mood other than realis: irrealis,
imperative, hortative, and negative. These are expressed by suffixes on
the verb and typically follow aspectual marking where it is expressed by a
sufffix, that is, the habitative suffix \rayr{As/}{-asa}. The following
subsections will discuss each of the modal categories expressed by suffixes;
modals proper will be discussed in \autoref{subsec:modals}.

\subsubsection{Irrealis}
\index{mood!irrealis|(}

Irrealis marking in Ayeri is indicated by the suffix \rayr{/ONF}{-ong} and 
marks that an action is thought of as hypothetical by the speaker, whether he 
or she expects it to be fulfilled or not:

\ex\label{ex:irrealis}\begingl
	\gla Sahongvāng edaya, ming silvongvāng sitang-vāri. //
	\glb saha-ong=vāng edaya ming silv-ong=vāng sitang=vāri //
	\glb come-\Irr{}=\Ssg{}.\Aarg{} here can see-\Irr{}=\Ssg{}.\Aarg{} 
		\Refl{}=\Ssg{}.\Ins{} //
	\glft `If you came/had come here, you could see/have seen it  
		yourself.' //
\endgl\xe

As (\ref{ex:irrealis}) shows, irrealis marking is especially prominent in
conditional clauses which express a hypothetical cause and effect. Both
condition/protasis and consequence/apodosis are marked with the irrealis suffix
in this case. The example sentence also shows that, again, the initial vowel of
the suffix replaces the last vowel of the verb stem, if there is one, so that
\rayr{sh/}{saha-} becomes \rayr{shoNF/}{sahong-}, to which further mood
suffixes may be added, and finally, person marking.

The same suffix, \rayr{/ONF}{-ong}, is also used in other contexts expressing 
inactual events, for instance, in reported speech, or complement clauses 
expressing a wish about the actualization of a hypothetical event:

\ex\begingl
	\gla Narayeng, ang menongye demās yena. //
	\glb nara=yeng ang menu-ong=ye.Ø dema-as yena //
	\glc say=\TsgF{}.\Aarg{} \AgtT{} visit-\Irr{}=\TsgF{}.\Top{} 
		aunt-\Parg{} \TsgF{}.\Gen{} //
	\glft `She said she were visiting her aunt.' //
\endgl\xe

\ex~\begingl
	\gla Hanuyang, koronongyang maritay. //
	\glb hanu=yang koron-ong=yang maritay //
	\glc wish=\Fsg{}.\Aarg{} know-\Irr{}=\Fsg{}.\Aarg{} before //
	\glft `I wish I had known this before.' //
\endgl\xe

Irrealis marking does not, however, appear in contexts that express
requirements on or wishes about a third person's actions, that is, typical
subjuctive contexts; the verb in the complement clause rather appears in the
indicative in these contexts. To add a sense of expectation of compliance about
the action, the modal \xayr{mY}{mya}{be supposed to} may be added, see
\autoref{subsec:modals}.

\pex
\a\ljudge*\begingl
	\gla Arapnang, sa garongyāng hatay. //
	\glb arap=nang sa gara-ong=yāng hatay-Ø //
	\glc require=\Fpl{}.\Aarg{} \PatT{} call-\Irr{}=\TsgM{}.\Aarg{} 
		police-\Top{} //
\endgl

\a\label{ex:myashall}\begingl
	\gla Arapnang, sa {\normalfont (}mya{\normalfont )} garayāng hatay. //
	\glb arap=nang sa (mya) gara=yāng hatay-Ø //
	\glc require=\Fpl{}.\Aarg{} \PatT{} (be.supposed.to) 
		call=\TsgM{}.\Aarg{} police-\Top{} //
	\glft `We require that he call the police.' //
\endgl
\xe

\index{mood!irrealis|)}

\subsubsection{Negative}
\label{subsubsec:verbneg}
\index{mood!negative|(}
\index{negation!of verbs|(}

The negative mood is used to negate verbs, which is separate from irrealis
marking: negation of verbs is marked by the suffix \rayr{/Oj}{-oy}, which has
an allomorph \fw{-u} before diphthongs in romanization and also in
pronunciation. The Tahano Hikamu spelling is more conservative here and keeps
the spelling \ayr{/Oyj} \orth{-oyay} for /-uay/
(\mbox{-\Neg{}=\Fsg{}.\Top{}}).\index{allomorphy} Like the irrealis suffix, the
negative suffix deletes the last vowel of the verb stem if present, which is
exemplified in (\ref{ex:negallo}) besides this example showing the \fw{-u}
allomorph. Moreover, example (\ref{ex:irrneg}) shows that negative marking
usually follows irrealis marking when suffixes are stacked: \rayr{/ONF}{-ong} +
\rayr{/Oj}{-oy} → \rayr{/ONoj}{-ongoy}.

\pex
\a\label{ex:negative}\begingl
	\gla Ang silvoyyan nasiyamanas tan. //
	\glb ang silv-oy=yan.Ø nasi-yam-an-as tan //
	\glc \AgtT{} see-\Neg{}=\TplM{}.\Top{} approach-\Ptcp{}-\Nmlz{}-\Parg{} 
		\TplM{}.\Gen{} //
	\glft `They did not see them approaching.' //
\endgl

\a\label{ex:negallo}\begingl
	\gla Ang peguay kalam adaley!  //
	\glb ang pega-oy=ay.Ø kalam ada-ley //
	\glc \AgtT{} steal-\Neg{}=\Fsg{}.\Top{} honestly that-\PargI{} //
	\glft `I didn't steal it, honestly!' //
\endgl

\a\label{ex:irrneg}\begingl
	\gla Ang tendongoyva sarayam adaya. //
	\glb ang tend-ong-oy=va.Ø sara-yam adaya //
	\glc \AgtT{} dare-\Irr{}-\Neg{}=\Ssg{}.\Top{} go-\Ptcp{} there //
	\glft `You would not dare to go there.' //
\endgl

\xe

If negated verbs appear together with negative indefinite pronouns (compare 
\autoref{subsec:indefpro}), multiple negatives do not cancel each other out, 
but amplify the negation instead. This is to say that Ayeri allows for multiple
negation\index{negation!multiple negation} as a means to emphasize the 
impossibility of something.

\ex
\begingl
	\gla Le gamaroyya tadoy ranyāng adanya. //
	\glb le gamar-oy-ya tadoy ranyāng adanya-Ø //
	\glc \PatTI{} manage-\Neg{}-\TsgM{} never nobody-\Aarg{} that-\Top{} //
	\glft `Nobody ever managed that',\\
		literally: `Nobody never didn't manage that.' //
\endgl
\xe

\index{negation!of verbs|)}
\index{mood!negative|)}

\subsubsection{Imperative}
\index{mood!imperative|(}

The imperative mood is used to mark orders to an unspecified second person,
that is, imperative verbs do not require an overt second person agent; if an
addressee is included, it is oblique and unmarked for case, see
\autoref{subsec:uncased}. Moreover, no distinction is made between singular and
plural second-person addressees, so that the marker is \rayr{/U}{-u} in either
case. Like the other mood suffixes, the vowel of the imperative suffix replaces
the vowel of the verb stem if there is one.

\pex
\a\begingl
	\gla Giru māy! //
	\glb gira-u māy //
	\glc hurry-\Imp{} \Int{} //
	\glft `Hurry up!' //
\endgl

\a\begingl
	\gla Tangu yām, Yan! //
	\glb tang-u yām Yan //
	\glc listen-\Imp{} \Fsg{}.\Dat{} Yan //
	\glft `Listen to me, Yan!' //
\endgl

\a\begingl
	\gla Tangu yām, ledanye nā! //
	\glb tang-u yām ledan-ye nā //
	\glc listen-\Imp{} \Fsg{}.\Dat{} friend \Fsg{}.\Gen{} //
	\glft `Listen to me, my friends!' //
\endgl

\xe

It is important to note that imperative-marked verbs behave essentially as
infinite forms in that they do not exhibit any  agreement in person, number,
gender, and topic, and also cannot act as hosts  for clitic personal pronouns.
Imperative verbs may be marked for negative and  hortative, however. Hence,
for instance, (\ref{ex:negimp}) is grammatical,  while the examples in
(\ref{ex:agrimp}) are not.

\ex\label{ex:negimp}\begingl
	\gla Saroyu yas! //
	\glb sara-oy-u yas //
	\glc leave-\Neg{}-\Imp{} \Fsg{}.\Parg{} //
	\glft `Don't leave me!' //
\endgl\xe

\pex~\label{ex:agrimp}
\a\label{ex:topimp}\ljudge*\begingl
	\gla Ya sa-sahu nanga! //
	\glb ya sa\til{}saha-u nanga-Ø //
	\glc \LocT{} \Iter{}\til{}go-\Imp{} house-\Top{} //
	\glft `Go back to the house!' //
\endgl

\a\label{ex:persimp}\ljudge*\begingl
	\gla Sa sutamuya kohanya tasela! //
	\glb sa sutam-u=ya.Ø kohan-ya tasela //
	\glc \PatT{} hang-\Imp{}=\TsgM{}.\Top{} sunrise-\Loc{} tomorrow //
	\glft `May he be hanged tomorrow at sunrise!' //
\endgl
\xe

Example (\ref{ex:negimp}) simply expresses a negative command, which is
unproblematic in terms of logic, since commands may be issued to act in a
certain way, or to forgo this action. Example (\ref{ex:topimp}) shows the
imperative verb as preceded by a locative topic marker, which is not logically
impossible, but unacceptable by convention.\footnote{The translation of
`\citetitle{shelley:ozymandias}' in \autoref{sec:ozymandias} deviates from
this rule in the line \xayr{s silFvu gumo naa}{sa silvu gumo nā}{my works,
regard them}. This is poetic license, however.} Example (\ref{ex:persimp})
takes  this one step further in displaying a cliticized object pronoun in the
fashion  of morphological passives (compare \autoref{subsec:persnumagr}, page
\pageref{patagr}).

\index{mood!imperative|)}

\subsubsection{Hortative}
\index{mood!hortative|(}

The hortative is a special kind of imperative which addresses a group 
including the speaker. Its implied referent is thus first-person plural. 
Again, it is not necessary to mark the verb for the addressee here. As 
the hortative is related in meaning to the imperative, the verb also uses the 
imperative inflection with \rayr{/U}{-u}, but it is fully reduplicated in 
addition to mark the difference. As regards agreement morphology, the same 
restrictions as those of imperatives apply.

\pex
\a\begingl
	\gla Sahu! //
	\glb saha-u //
	\glc go-\Imp{} //
	\glft `Go!' //
\endgl

\a\begingl
	\gla Sahu-sahu umangya! //
	\glb sahu\til{}saha-u umang-ya //
	\glc \Hort{}\til{}go-\Imp{} beach-\Loc{} //
	\glft `Let's go to the beach!' //
\endgl
\xe

\index{mood!hortative|)}

\index{mood|)}

\subsection{Modals}
\label{subsec:modals}
\index{modals|(}

\begin{table}
\caption{Modal verbs and particles}
\begin{tabu} to \linewidth {C[3] X[2] X[2] X[4]}
\tableheaderfont\toprule
Category
	& Verb
	& Particle
	& Translation
	\\
\toprule

ability
	& \ayr{miNF/} \fw{ming-}
	& \ayr{miNF} \fw{ming}
	& `be able to, can'
	\\
	
\midrule
	
desire, intention
	& \ayr{vtYF/} \fw{vac-}
	& \ayr{vtY} \fw{vaca}
	& `like to'
	\\
	
	& \ayr{no/} \fw{no-}
	& \ayr{no} \fw{no}
	& `want to'
	\\
	
\midrule

permission
	& \ayr{kil/} \fw{kila-}
	& \ayr{kil} \fw{kila}
	& `be allowed to, may'
	\\
	
\midrule

requirement
	& \ayr{IlFt/} \fw{ilta-}
	& \ayr{IlFt} \fw{ilta}
	& `need to'
	\\
	
\midrule

obligation
	& \ayr{mY/} \fw{mya-}
	& \ayr{mY} \fw{mya}
	& `be supposed to, shall'
	\\
	
	& \ayr{ru\_a/} \fw{rua-}
	& \ayr{ru\_a} \fw{rua}
	& `have to, must'
	\\
	
\midrule
	
continuation
	& \ayr{divF/} \fw{div-}
	& \ayr{div} \fw{diva}
	& `stay, remain'
	\\

\bottomrule
\end{tabu}
\label{tab:modverb}
\end{table}

Modals in Ayeri express the notions of ability, desire, permission,
requirement, obligation, and also of continuation, as indicated by
\autoref{tab:modverb}. They can generally act as both fully inflectable
intransitive verbs, as well as clitics which occur in combination with fully
inflected content verbs:

\pex
\a\label{ex:modalinvar}\begingl
	\gla Rua bahavāng baho, ang bihanoyya mirampaluy nas. //
	\glb rua baha=vāng baho ang bihan-oy=ya.Ø mirampaluy nas //
	\glc must shout=\Ssg{}.\Aarg{} loudly \AgtT{} 
		understand-\Neg{}=\TsgM{}.\Top{} otherwise \Fpl{}.\Parg{} //
	\glft `You have to shout loudly, otherwise he does not understand 
		us.'//
\endgl

\a\label{ex:modalinfl}\begingl
	\gla Ruasanang. //
	\glb rua-asa=nang //
	\glc must-\Hab{}=\Fpl{}.\Aarg{} //
	\glft `We usually have to.' //
\endgl

\xe

As (\ref{ex:modalinvar}) shows, the modal does not inflect in combination with
another verb; as a clitic it rather acts similarly to a prefix, like the
progressive marker \rayr{mN}{manga}, which is also presumably deverbal
(compare \autoref{sec:typology}, \autoref{fn:mangaverb}). In difference to
\rayr{mN}{manga}, which as a preverbal element only serves a grammatical
function, the semantic component of the modals is still prevalent, as is shown
by (\ref{ex:modalinfl}), where \rayr{ru\_a/}{rua-} appears in its function as
an intransitive verb with the same meaning of strong obligation as in
(\ref{ex:modalinvar}), though it carries regular person and aspect inflection
here. Inflecting the modal in the context of cooccurrence with a content verb
is, however, considered unacceptable:

\ex\ljudge*\begingl
	\gla Ruavāng bahayam baho. //
	\glb rua=vāng baha-yam baho //
	\glc must=\Ssg{}.\AgtT{} shout-\Ptcp{} loudly //
	\glft `You have to shout loudly.' //
\endgl\xe

Regarding example (\ref{ex:modalinfl}) and its ability to inflect, Ayeri also
has a verb that generally means `do', namely, \rayr{mir/}{mira-}. However, it
is not common to use this as a dummy verb to carry the inflection instead of
the modal verb either. While such a construction is not ungrammatical \fw{per
se}, it is simply not the preferred way to express intransitive modal verbs:

\ex\ljudge\ques\begingl
	\gla Rua mirasanang. //
	\glb rua mira-asa=nang //
	\glc must do-\Hab{}=\Fpl{}.\Aarg{} //
	\glft `We usually have to.' //
\endgl\xe

While most of the verbs listed in \autoref{tab:modverb} should look 
reasonable to English speakers, Ayeri uses two verbs for modal particles which 
may seem odd: \xayr{vtY}{vaca}{like to}, to express taking pleasure in doing 
something, and \xayr{div}{diva}{stay, remain}, to express that the action is 
being prolonged.\footnote{The verb stems indeed end in a consonant while the 
modal particles need an epenthetic \fw{-a} to form permissible words.} The 
latter verb thus also has an aspectual component to its meaning.

\pex\label{ex:vacvaca}
\a\label{ex:vacfull}\begingl
	\gla Ang vacay betayley. //
	\glb ang vac=ay.Ø betay-ley //
	\glc \AgtT{} like=\Fsg{}.\Top{} berry-\PargI{} //
	\glft `I like berries.' //
\endgl

\a\label{ex:vacamod}\begingl
	\gla Ang vaca konday betayley. //
	\glb ang vaca kond=ay.Ø betay-ley //
	\glc \AgtT{} like eat=\Fsg{}.\Top{} berry-\PargI{} //
	\glft `I like to eat berries.' //
\endgl
\xe

\pex~\label{ex:divdiva}
\a\label{ex:divfull}\begingl
	\gla Ang divay rangya nā tasela. //
	\glb ang div=ay.Ø rang-ya nā tasela //
	\glc \AgtT{} stay=\Fsg{}.\Top{} home-\Loc{} \Fsg{}.\Gen{} //
	\glft `I will stay home tomorrow.' //
\endgl

\a\label{ex:divamod}\begingl
	\gla Ang diva bengya ku-danyās kebay. //
	\glb ang diva beng=ya.Ø ku=danya-as kebay //
	\glc \AgtT{} stay stand=\TsgM{}.\Top{} like=one-\Parg{} alone //
	\glft `He remained standing as the only one.' //
\endgl
\xe

The fact that modal particles in Ayeri retain their verbal semantics in spite 
of shedding verb morphology is probably even more obvious from the above 
examples (\ref{ex:vacvaca}) and (\ref{ex:divdiva}), which show the alternation 
between full-verb use (a) and modal use (b) for both \rayr{vtYF/}{vac-} and 
\rayr{divF/}{div-}. In comparison to the other modals in 
\autoref{tab:modverb}, these two verbs in particular also stand out by virtue 
of their roots ending in a consonant instead of a vowel like in the other 
cases. This suggests that they may have been grammaticalized as modals 
only relatively recently, and there appears to be variation at least for 
\rayr{vtYF/}{vac-}, for instance:

\ex\begingl
	\gla ... yam vacongyang ilisayam eda-koyās gan ... //
	\glb ... yam vac-ong-yang ilisa-yam eda=koya-as gan-Ø ... //
	\glc {} \DatT{} like-\Irr{}-\Fsg{}.\Aarg{} dedicate-\Ptcp{} 		
		this=book-\Parg{} child-\Top{} {} //
	\glft `... I would like to dedicate this book to the child ...' 
		\tc{\citep[1, 8]{benung:petitprince}} //
\endgl\xe

Moreover, as illustrated previously in (\ref{ex:myashall}), \xayr{mY}{mya}{be 
supposed to, shall} can be used to express indirect commands where English may 
use the subjunctive mood; essentially the function of this modal is that of 
the jussive mood. For convenience, the above example will be repeated here:

\ex\begingl
	\gla Arapnang, sa {\normalfont (}mya{\normalfont )} garayāng hatay. //
	\glb arap=nang sa (mya) gara=yāng hatay-Ø //
	\glc require=\Fpl{}.\Aarg{} \PatT{} (be.supposed.to) 
		call=\TsgM{}.\Aarg{} police-\Top{} //
	\glft `We require that he call the police.' //
\endgl\xe

In addition to this use, \rayr{mY}{mya} is also used in commands to third 
persons, whether direct or indirect. English may use \fw{shall} here as an 
equivalent.

\pex
\a\begingl
	\gla Ningu cam, mya saratang. //
	\glb ning-u cam mya sara=tang //
	\glc tell-\Imp{} \TplM{}.\Dat{} shall leave=\TplM{}.\Aarg{} //
	\glft `Tell them to leave.' //
\endgl

\a\begingl
	\gla Mya vehara nekanley. //
	\glb mya veh-ara nekan-ley //
	\glc shall build-\TsgI{} bridge-\PargI{} //
	\glft `A bridge shall be built.' //
\endgl

\a\begingl
	\gla Mya yomāra makangreng. //
	\glb mya yoma-ara makang-reng //
	\glc shall exist-\TsgI{} light-\AargI{} //
	\glft `Let there be light.' //
\endgl
\xe

\index{modals|)}

\subsection{Participle}
\label{subsec:participle}
\index{participle|(}
Besides the imperative---and, by extension, the hortative---Ayeri also 
possesses another infinite form called the participle. This form is marked by 
appending \rayr{/ymF}{-yam} to the verb root. The participle is generally the 
form of verbal complements of intransitive subordinating verbs other than 
modal verbs (compare \autoref{subsec:modals}). For instance,  
\xayr{kYunF/}{cun-}{begin} or \xayr{mnNF/}{manang-}{avoid} both allow 
complementation with another verb:

\pex
\a\label{ex:intrcompl}\begingl
	\gla Cunyo makayam perinang. // 
	\glb cun-yo maka-yam perin-ang // 
	\glc begin-\TsgN{} shine-\Ptcp{} sun-\Aarg{} //
	\glft `The sun began to shine.' //
\endgl

\a\label{ex:trcompl}\begingl
	\gla Manangyeng pengalyam badanas saha yena. //
	\glb manang=yeng pengal-yam badan-as saha yena //
	\glc avoid=\TsgF{}.\Aarg{} meet-\Ptcp{} father-\Parg{} in.law 
		\TsgF{}.\Gen{} //
	\glft `She avoids to meet her father-in-law.' //
\endgl
\xe

Since subordinated verbs may be transitive like in (\ref{ex:trcompl}), the
problem of center-embedding arises when the agent NP of the subordinating verb
is not simply a cliticized pronoun (see \autoref{clitics_postverb_person},
p.~\pageref{clitics_postverb_person}; \ref{subsec:persnumagr}), since
arguments of the subordinating verb follow the embedded clause as in
(\ref{ex:intrcompl}):

\pex[*=\ques\ques]
\a\ljudge{\ques}\begingl
	\gla Ang pinyaya {\normalfont [}konjam inunas{\normalfont]} {} Yan sa 
		Pila. //
	\glb ang pinya-ya kond-yam inun-as Ø Yan sa Pila //
	\glc \AgtT{} ask-\TsgM{} eat-\Ptcp{} fish-\Parg{} \Top{} Yan \Parg{} 
		Pila //
	\glft `Yan asks Pila to eat the fish.' //
\endgl

\a\ljudge{\ques\ques}\begingl
	\gla Ang pinyaya {\normalfont [}ilyam koyaley ledanyam 
		yana{\normalfont]} {} Yan sa Pila. //
	\glb ang pinya-ya il-yam koya-ley ledan-yam yana Ø Yan sa Pila //
	\glc \AgtT{} ask-\TsgM{} give-\Ptcp{} book-\PargI{} friend-\Dat{} 
		\TsgM{}.\Gen{} \Top{} Yan \Parg{} Pila //
	\glft `Yan asks Pila to give the book to his friend.' //
\endgl
\xe

In order to avoid too much complexity at the expense of ease of composition on
the speaker's side, and intelligibility on the listener's, it is much
preferable to express the embedded clause as a complement clause
instead.\footnote{The German linguist Otto Behaghel (1854--1936) coined a
number of laws---albeit with German in focus---three of which are relevant to
information flow: \textcquote[4]{behaghel1932}{Das oberste Gesetz ist dieses,
daß das geistig eng Zusammengehörige auch eng zusammengestellt wird.} [`The
supreme law is such that the mentally closely related is also arranged in close
proximity.']---\textcquote[4]{behaghel1932}{Ein zweites machtvolles Gesetz
verlangt, daß das Wichtigere später steht als das Unwichtige, dasjenige, was
zuletzt noch im Ohr klingen soll.} [`A second powerful law demands that more
important information appear at a later point than what is less important: the
which is supposed lastly to resonate in the listener's
ear.']---\textcquote[6]{behaghel1932}{Gesetz der wachsenden Glieder […]; es
besagt, daß von zwei Gliedern, soweit möglich, das kürzere vorausgeht, das
längere nachsteht.} [`Law of the growing constituents […]; it signifies that of
two constituents, if possible, the shorter one precedes, the longer one
follows.'] Also compare \citet{wasow1997} on the cooperation between speaker
and listener in the face of syntactically complex, `heavy' constituents.} The
particle \rayr{d/}{da-} may be added to the formerly subordinating verb in
order to signal that a complement clause is following.

\pex
\a\begingl
	\gla Ang da-pinyaya {} Yan sa Pila, {\normalfont [}le konjeng 
		inun{\normalfont ]}. //
	\glb ang da=pinya-ya Ø Yan sa Pila le kond=yeng inun-Ø //
	\glc \AgtT{} such=ask-\TsgM{} \Top{} Yan \Parg{} Pila \PatTI{} 
		eat=\TsgF{}.\Aarg{} fish-\Top{} //
	\glft `Yan asks Pila to eat the fish.' //
\endgl

\a\begingl
	\gla Ang da-pinyaya {} Yan sa Pila, {\normalfont [}le ilyeng koya 
		ledanyam yana{\normalfont]}. //
	\glb ang da=pinya-ya Ø Yan sa Pila le il=yeng koya-Ø ledan-yam
		yana //
	\glc \AgtT{} such=ask-\TsgM{} \Top{} Yan \Parg{} Pila \PatTI{} 
		give-\TsgF{} book-\Top{} friend-\Dat{} \TsgM{}.\Gen{} //
	\glft `Yan asks Pila to give the book to his friend.' //
\endgl
\xe

\index{participle|)}

\subsection{Other affixes}

In the section on noun morphology we have already encountered a number of
clitic prefixes that may attach to noun heads (see \autoref{subsec:clitics},
\autoref{subsec:nounpref}),  and some of these can also attach to verbs.
Furthermore, verbs may also be  modified by certain adverbial quantifier
clitics. The latter are dealt with in  more detail in the section on adverbs;
only a few relevant examples will be  given here.

\subsubsection{Prefixes}
\label{subsubsec:verbprefixes}

We have already encountered the prefix \xayr{d/}{da-}{so, such} in the
previous section, as well as in the section on noun prefixes (see
\autoref{clitics_prenoun_dem}, p.~\pageref{clitics_prenoun_dem}; 
\ref{subsec:nounpref}; \ref{subsec:participle}). With nouns,
\xayr{d/}{da-}{such} patterns as a demonstrative with the deictic prefixes
\xayr{Ed/}{eda-}{this} and \xayr{Ad/}{ada-}{that}. Distinguishing between near
and far is not possible with verbs,\footnote{Unless you distinguish between
actions performed in the speaker's proximity versus ones that are performed at
a distance. Ayeri, however, does not make such a distinction.} but pointing
out that something is happening `in this way', `so' is still possible, hence
\rayr{d/}{da-} is also applicable to verbs. \rayr{d/}{da-} can thus act
essentially as a pro-verb. As a clitic, it leans on the verb, preceding all
other inflectional prefixes, that is, any tense prefixes that may possibly
precede the verb root.

\pex\label{ex:daproverb}
\a\begingl
	\gla Da-mingya ang Diyan. //
	\glb da=ming-ya ang Diyan. //
	\glc so=can-\TsgM{} \Aarg{} Diyan //
	\glft `Diyan can (do it).' //
\endgl

\a\begingl
	\gla Ang da-məpinyaya {} Yan sa Pila. //
	\glb ang da=mə-pinya-ya Ø Yan sa Pila //
	\glc \AgtT{} such=\Pst{}-ask-\TsgM{} \Top{} Yan \Parg{} Pila //
	\glft `Yan asked Pila to (do so).' //
\endgl

\xe

Another possible use of the prefix \rayr{d/}{da-} with verbs is related to the
colloquial abbreviation of \xayr{dnY}{danya}{such one} as described in
sections \autoref{clitics_preverb_da} (p.~\pageref{clitics_preverb_da}) and
\autoref{subsec:dempro}, where the demonstrative part, \rayr{d/}{da-} may be
split off the pronoun and attached in front of the adjective directly to
express `the \textsc{adj} one'. This practice has possibly been extended to
verbs in analogy to the use just illustrated in (\ref{ex:daproverb}). Example
(\ref{ex:redone}) from the mentioned section is repeated here for the reader's
convenience:

\ex\begingl
	\gla Sa noyang da-tuvo. //
	\glb sa no=yang da=tuvo.Ø //
	\glc \PatT{} want=\Fsg{}.\Aarg{} such=red.\Top{} //
	\glft `I want the red one.' //
\endgl\xe

When \rayr{d/}{da-} is used as an abbreviation for \rayr{dnYaasF}{danyās} 
(such.one-\Parg{}) or \rayr{dnYlej}{danyaley} (such.one-\PargI{}), as in the 
following example, it may also appear prefixed to the verb:

\ex\begingl
	\gla Mya da-vehoyyāng. //
	\glb mya da=veh-oy=yāng //
	\glc supposed.to one=build-\Neg{}=\Tsg.\M{} //
	\glft `He is not supposed to build one.' //
\endgl\xe

As mentioned above, \rayr{d/}{da-} can also be used in an expletive way, to 
express `in this way' or `like that'. It does not encode an anaphoric relation 
in this case, but merely serves as a discourse particle to highlight the action.

\pex
\a\begingl
	\gla Da-sahāra seyaraneng. //
	\glb da=saha-ara seyaran-eng //
	\glc thus=come-\TsgI{} rain-\AargI{} //
	\glft `Here comes the rain.' //
\endgl

\a\begingl
	\gla Le no da-subroyya ang Hasanjan tiga kaytan yana. //
	\glb le no da=subr-oy-ya ang Hasanjan tiga kaytan-Ø yana //
	\glc \PatT{} want there=give.up-\Neg{}-\TsgM{} \Aarg{} Hasanjan 
		honorable right-\Top{} \TsgM{}.\Gen{} //
	\glft `Mr. Hasanjan did not want to cease his right just there.' //
\endgl

\xe

Besides \rayr{d/}{da-}, verbs may also take the \xayr{ku/}{ku-}{like} prefix,
which we have already seen with both nouns and adjectives (compare
\autoref{subsec:clitics}, \ref{subsec:nounpref}, \ref{subsec:adjaffx}). The
English translation in  context may rather be `as though' than `like' here,
but the function is the  same: expressing alikeness and resemblance.

\ex\begingl
	\gla Misyeng, ang ku-tangoyye yās. //
	\glb mis=yeng ang ku=tang-oy=ye.Ø yās //
	\glc act=\TsgF{}.\Aarg{} \AgtT{} like=hear-\Neg{}=\TsgF{}.\Top{} 
		\TsgM{}.\Parg{} //
	\glft `She acts as though she does not hear him.' //
\endgl\xe

As previously described (compare \autoref{clitics_preverb_refl},
p.~\pageref{clitics_preverb_refl}, and \ref{subsec:reflrec}),
\xayr{sitNF/}{sitang}{self}, the reflexive clitic, can appear as a prefix on
verbs as well. This may be the case when the patient/undergoer of a
transitive sentence signifies the same entity as the actor. Example
(\ref{ex:reflvb}) is repeated here for convenience:

\ex\begingl
	\gla Ang sitang-silvye puluyya. //
	\glb ang sitang=silv=ye.Ø puluy-ya //
	\glc \AgtT{} self=see=\TsgF{}.\Top{} mirror-\Loc{} //
	\glft `She sees herself in the mirror.' //
\endgl\xe

The image of the agent in the mirror is that of the agent herself, so she is
seeing her own reflection. Both agent and patient thus reference the same
person, which means that instead of using the reflexive object pronoun
\xayr{sitNF/yesF}{sitang-yes}{herself} (self-\TsgF{}.\Parg{}), it is possible
to drop the pronoun and to place the reflexive prefix on the verb instead.

\subsubsection{Suffixes}

Besides taking clitic prefixes, verbs may also take clitic suffixes, namely,
adverbial suffixes denoting degree, such as \xayr{/Ani}{-ani}{not at all},
\xayr{/ENF}{-eng}{rather}, \xayr{/IknF}{-ikan}{much},  \xayr{/Ikoj}{-ikoy}{not
much}, \xayr{/kj}{-kay}{a little},  \xayr{/nm}{-nama}{just, only, merely},
\xayr{/NsF}{-ngas}{almost}, and  \xayr{/nYm}{-nyama}{even} (see
\autoref{clitics_quant}, p.~\pageref{clitics_quant}). Some of these overlap
with quantifiers applicable to nouns, and all of them are also applicable to
adjectives. As enclitics, these suffixes lean on the inflected verb:

\pex
\a\label{ex:verbquant}\begingl
	\gla Ang rua apaya-kay {} Latun adanyaya. //
	\glb ang rua apa-ya=kay Ø Latun adanya-ya //
	\glc \AgtT{} must laugh-\TsgM{}=a.little \Top{} Latun that.one-\Loc{} //
	\glft `Latun had to laugh a little at that.' //
\endgl

\a\begingl
	\gla Ya no narayang-nama va. //
	\glb ya no nara=yang=nama va.Ø //
	\glc \LocT{} want speak=\Fsg{}.\Aarg{}=just \Ssg{}.\Top{} //
	\glft `It is you I just want to talk to.' //
\endgl
\xe

\index{verbs|)}

\section{Adverbs}
\label{sec:adverbs}
\index{adverbs|(}

Adverbs in Ayeri are the counterparts of adjectives with regards to
modification of verbs and phrases. Like adjectives, they do not display
agreement, though attributive adverbs may take suffixes for comparison (`run
\fw{faster}', `climb \fw{better}'). Adverbs may equally be modified by the
usual degree suffixes. Generally, there is no rigid distinction between adverbs
and adjectives, so the latter may easily be used as the former. The following
subsections will discuss the different kinds of adverbs and their possible uses
as modifiers.

\subsection{Attributive adverbs}
\index{adverbs!attributive|(}

Attributive adverbs are words expressing the manner in which an action is
carried out, or the circumstances of an event. Like adjectives, adverbs follow
their heads, that is, verbs. If near-grammaticalized adverbs are involved,
namely, adverbs whose function predominates over their semantic content,
attributive adverbs follow these. This case is illustrated in
(\ref{ex:funcadv}), where the attributive adjective \xayr{bnF}{ban}{good}
follows the more functional adverb \xayr{Iri}{iri}{already}. In
(\ref{ex:attradv}), on the other hand, the descriptive adjective
\xayr{tYbo}{cabo}{late} can directly follow the verb. Further adverbs may
follow in decreasing order of semantic relation to their head. With regards to
grammaticalization, \citet[157\psqq]{lehmann2015} speaks of \emph{bondedness}
or \emph{fügungsenge} (`closeness of construction'): the closer the bond
between two juxtaposed terms is, the higher is its degree of
grammaticalization. This explains why \rayr{Iri}{iri} must follow the verb in
(\ref{ex:funcadv}) while descriptive adverbs less central to the verb's meaning
typically follow with increasing optionality.

\pex
\a\label{ex:funcadv}\begingl
	\gla Ri rija iri \textbf{ban} ang Tapan palān yena. //
	\glb ri rig-ya iri ban ang Tapan palān-Ø yena //
	\glc \InsT{} draw-\TsgM{} already well \Aarg{} Tapan age-\Top{} 
		\TsgF{}.\Gen //
	\glft `For her age, Tapan already draws well.' //
\endgl

\a\label{ex:attradv}\begingl
	\gla Sahasaya \textbf{cabo} ang Niyas. //
	\glb saha-asa-ya cabo ang Niyas //
	\glc come-\Hab{}-\TsgM{} late \Aarg{} Niyas //
	\glft `Niyas is usually late.' //
\endgl

\xe

Adverbs do not show agreement, however, attributive adverbs can be negated.
This makes them very similar to adjectives, except that they do not modify
nouns. The negative suffix for attributive adverbs is \rayr{/Oj}{-oy}, which is
demonstrated in (\ref{ex:advneg}).

\ex\label{ex:advneg}\begingl
	\gla Ersasayan napayoy ang Temisi. //
	\glb ers-asa-yan napay-oy Ø Temisi //
	\glc cook-\Hab{}-\TplM{} spicy-\Neg{} \Aarg{} Northerner //
	\glft `The Northerners cook in an unspicy way.' //
\endgl\xe

The adjective \xayr{npj}{napay}{spicy} has been seamlessly converted into an
adjective here and negated to \xayr{npyoj}{napayoy}{unspicy(ly)}. The semantic
difference from the same sentence with the verb negated instead of the adverb
is marginal and up to the choice of the speaker:

\ex\label{ex:advneg_2}\begingl
	\gla Ersasoyyan napay ang Temisi. //
	\glb ers-asa-oy-yan napay Ø Temisi //
	\glc cook-\Hab{}-\Neg{}-\TplM{} spicy \Aarg{} Northerner //
	\glft `The Northerners don't cook in a spicy way.' //
\endgl\xe

\subsubsection{Comparison of adverbs}
\index{comparison!of adverbs|(}

Since actions are usually gradable in the way they are carried out, it is 
possible to compare adverbs in the same way as adjectives. Here, however, only 
the particle-based strategy described in \autoref{subsec:adjcomp} 
can be used.

\pex
\a\label{ex:advcomp}\begingl
	\gla Ang rije ban-eng {} Sipra na Tapan. //
	\glb ang rig-ye ban=eng Ø Sipra na Tapan //
	\glc \AgtT{} draw-\TsgF{} good=\Comp{} \Top{} Sipra \Gen{} Tapan //
	\glft `Sipra draws better than Tapan.' //
\endgl

\a\label{ex:advsupl}\begingl
	\gla Rije ban-vā ang Nava. //
	\glb rig-ye ban=vā ang Nava //
	\glc draw-\TsgF{} good=\Supl{} \Aarg{} Nava //
	\glft `Nava draws best.' //
\endgl
\xe

\index{comparison!of adverbs|)}

In order to form the comparative (\ref{ex:advcomp}), the suffix 
\rayr{/ENF}{-eng} is appended to the adverb; the superlative (\ref{ex:advsupl})
carries the suffix \rayr{/vaa}{-vā} as a marker.

\subsubsection{\fw{Māy} and \fw{voy}}
\label{subsubsec:maayvoy}
\index{questions!tag questions|(}
% Maybe move this section to another, better fitting place later. For instance,
% question tags may better fit in with a discussion of question types within a
% chapter on sentence types.

The discourse particles \xayr{maaj}{māy}{yes} and \xayr{voj}{voy}{no} can 
also appear as adverbs, though since they act mainly as functional morphemes 
here, it is not possible for them to undergo comparison in spite of their 
attributive use. While \xayr{maaj}{māy}{yes} and \xayr{voj}{voy}{no} normally 
express affirmative and negative responses as answers to closed questions, 
\rayr{maaj}{māy}, for one, can be used adverbially as an intensifier 
(\ref{ex:maayintens}). In a similar way, \rayr{voj}{voy} can be used for 
negative intensification (\ref{ex:voyintens}). The negative intensifier 
replaces negation on the verb in this case, though the verb may still be 
negated as well for very forceful negation.

\pex\label{ex:maayintens}
\a\begingl
	\gla Nay le konja māy epang ang Kaji nernan barina sebu! //
	\glb nay le kond-ya māy epang ang Kaji nernan-Ø bari-na sebu //
	\glc and \PatTI{} eat-\TsgM{} \Int{} then \Aarg{} Kaji piece-\Top{} 
		meat-\Gen{} rotten //
	\glft `And then Kaji totally ate the piece of rotten meat!' //
\endgl

\a\begingl
	\gla Yāng māy karomayās nārya. //
	\glb yāng māy karomaya-as nārya //
	\glc \TsgM{}.\Aarg{} \Int{} doctor-\Parg{} though // 
	\glft `He \emph{is} a doctor, though.' //
\endgl
\xe

\pex~\label{ex:voyintens}
\a\begingl
	\gla Le vacyo voy veneyang kondan. //
	\glb le vac-yo voy veney-ang kondan-Ø //
	\glc \PatTI{} like-\TsgN{} \Int{}.\Neg{} dog-\Aarg{} food-\Top{} //
	\glft `The food, the dog did not like it at all.' //
\endgl

\a\begingl
	\gla Adareng voy bahisley niru. //
	\glb ada-reng voy bahis-ley niru //
	\glc that-\AargI{} \Int{}.\Neg{} day-\PargI{} bad // 
	\glft `That is not a bad day at all.' //
\endgl
\xe

Besides this use, both \rayr{maaj}{māy} and \rayr{voj}{voy} can also be used in
tag questions, to reflect the expectation of a person asking with regards to
the answer:

\pex
\a\label{ex:posexpect}\begingl
	\gla Sa konjon māy patasjang keynam? //
	\glb sa kond-yon māy patas-ye-ang keynam-Ø //
	\glc \PatT{} eat-\TplN{} \Aff{} bear-\Pl{}-\Aarg{} people-\Top{} //
	\glft `People, bears eat them, don't they?' //
\endgl

\a\label{ex:negexpect}\begingl
	\gla Sa ginyon voy patasjang nimpur? //
	\glb sa gin-yon voy patas-ye-ang nimpur-Ø //
	\glc \PatT{} drink-\TplN{} \Neg{} bear-\Pl{}-\Aarg{} wine-\Top{} //
	\glft `Wine, bears don't drink it, do they?' //
\endgl
\xe

Example (\ref{ex:posexpect}) poses the question with the expectation of an 
affirmative answer. This is indicated by using the affirmative particle 
\rayr{maaj}{māy} after the verb. Example (\ref{ex:posexpect}), on the other 
hand, indicates that the asker has doubts about the issue in question and 
expects their opposite to decline. The negative particle \rayr{voj}{voy} 
is placed in adverb position after the verb accordingly.

\index{questions!tag questions|)}
\index{adverbs!attributive|)}

\subsection{Degree and quantity}
\label{subsec:quantifiers}
\index{adverbs!degree and quantity|(}

While attributive adverbs follow their heads as independent words, the most
common adverbs expressing degree and quantity (quantifiers) do not only follow
verbs, nouns, adpositions, adjectives, or other adverbs, but they cliticize on
them, that is, they are dependent morphemes (compare \autoref{clitics_quant},
p.~\pageref{clitics_quant}). The word stem---a lexical head  which is usually
inflected except in the case of adjectives---serves as the  host for the
clitic in all these cases. Examples of degree and quantifier  suffixes and how
they interact with different parts of speech were already  given in all the
relevant sections; an example from each section is repeated  here for
convenience. As we will see below, there are common grading and  quantifying
adverbs which behave like regular adverbs as well. It is possible  to combine
both the suffixed and the free kinds with other adverbs as long as  those
adverbs permit modification with regards to degree and/or quantity.  Purely
functional adverbs like \xayr{Iroj}{iroy}{not yet} or
\xayr{sirimNF}{sirimang}{about to} may allow degree adverbs to modify them at
least.

\pex
\a\label{ex:nounquant2}\begingl
	\glpreamble With a noun (\ref{ex:nounquant}): //
	\gla Ajayon ganang-ikan kivo. //
	\glb aja-yon gan-ang=ikan kivo. //
	\glc play-\TsgN{} child-\Aarg{}=many small //
	\glft `Many small children are playing.' //
\endgl

\a\label{ex:adjquant2}\begingl
	\glpreamble With an adjective (\ref{ex:adjquant}): //
	\gla Eda-prikanreng napay-eng //
	\glb eda=prikan-reng {napay eng} //
	\glc this=soup-\AargI{} {spicy rather} //
	\glft `This soup is rather spicy.' //
\endgl

\a\label{ex:prepquant2}\begingl
	\glpreamble With an adposition (\ref{ex:prepquant}): //
	\gla Ang mitasaye pang-ikan mandayya tado. //
	\glb ang mit-asa=ye.Ø pang=ikan manday-ya tado //
	\glc \AgtT{} live-\Hab{}=\TsgF{}.\Top{} back=much forum-\Loc{} old //
	\glft `She used to live way behind the old forum.' //
\endgl

\a\label{ex:verbquant2}\begingl
	\glpreamble With a verb (\ref{ex:verbquant}): //
	\gla Ang rua apaya-kay {} Latun adanyaya. //
	\glb ang rua apa-ya=kay Ø Latun adanya-ya //
	\glc \AgtT{} must laugh=\TsgM{}=a.little \Top{} Latun that.one-\Loc{} //
	\glft `Latun had to laugh a little at that.' //
\endgl

\xe

A number of quantifier adverbs can be used to express both quantity and 
degree, especially prominent in this regard is \rayr{/IknF}{-ikan}, which 
comprises all of `many', `much' and `very', as displayed in examples 
(\ref{ex:nounquant2}) and (\ref{ex:prepquant2}), where in the former case it 
appears as a quantifier of a countable entity 
(\xayr{gnNF/IknF}{ganang-ikan}{many children}) and in the latter case as a 
degree adverb (\xayr{pNF/IknF}{pang-ikan}{way behind}). The complete set of 
degree and quantifier suffixes is listed in \autoref{tab:quantifiers}. Note 
that not all suffixes have both a grading and a quantifying meaning.

\begin{table}[tp]\centering
\caption{Adverbial degree and quantifier suffixes}
\begin{tabu} to \linewidth {X X X}
\toprule\tableheaderfont
Suffix
	& Degree
	& Quantity
	\\

\toprule
	
\rayr{/Ani}{-ani}
	& not at all
	& none at all
	\\

\rayr{/ArilF}{-aril}
	& 
	& some
	\\

\rayr{/ENF}{-eng}
	& rather, more
	& more
	\\
	
\rayr{/henF}{-hen}
	& completely
	& all, every, each
	\\

\rayr{/IknF}{-ikan}
	& much, very
	& many, much
	\\
	
\rayr{/Ikoj}{-ikoy}
	& not very, less
	& not many, not much
	\\
	
\rayr{/INF}{-ing}
	& so
	&
	\\
	
\rayr{/kj}{-kay}
	& a bit, little
	& few
	\\

\rayr{/m}{-ma}
	& enough
	& enough
	\\

\rayr{/msF}{-mas}
	& some kind of
	&
	\\

\rayr{/nm}{-nama}
	& just, merely
	& just, only
	\\
	
\rayr{/NsF}{-ngas}
	& almost
	&
	\\

\rayr{/nYmF}{-nyama}
	& even
	&
	\\
	
\rayr{/vaa}{-vā}
	& most
	& most
	\\

\rayr{/venF}{-ven}
	& pretty, quite
	&
	\\

\bottomrule
\end{tabu}
\label{tab:quantifiers}
\end{table}

Grading and quantifying expressions which deviate form the pattern of 
cliticization and instead are used as independent words are, most notably:
\xayr{AMkYu}{ankyu}{really},
\xayr{diriNF}{diring}{several},
\xayr{EkeNF}{ekeng}{over-, overly, too},
\xayr{heNsF}{hengas}{almost all},
\xayr{IknF/IknF}{ikan-ikan}{altogether, totally},
\xayr{IknFvaanFy}{ikanvānya}{at most, by and large},
% \xayr{kjvj}{kayvay}{without},
\xayr{kgnF}{kagan}{excessively, far too},
\xayr{menikneNF}{menikaneng}{another (one more)},
\xayr{midj}{miday}{approximately},
\xayr{pluNF}{palung}{another (a different kind)},
\xayr{regnFdej}{regandey}{bit by bit, gradually},
\xayr{sno}{sano}{both},
\xayr{vrYaanY}{varyānya}{at least}.
Besides, adjectives denoting a degree, like \xayr{IpnF}{ipan}{drastic, extreme,
radical} can of course also be used as adverbial modifiers. Since adverbs do
not inflect, the conversion happens without the need to explicitly mark it.
\rayr{IpnF}{ipan} can thus also be used to mean `extremely':

\ex
\begingl
	\gla Yang valuy ipan, sa silvyang va. //
	\glb yang valuy ipan sa silv=yang va.Ø //
	\glc \Fsg{}.\Aarg{} glad extremely \PatT{} see=\Fsg{}.\Aarg{} 
		\Ssg{}.\Top{} //
	\glft `I'm extremely glad to see you.' //
\endgl
\xe

\index{adverbs!degree and quantity|)}

\subsection{Sentence adverbs}
\index{adverbs!sentence adverbs|(}

Ayeri allows adverbs to modify sentences, for instance, to express the stance
of the speaker, to concede an argument, or simply to structure an argumentative
chain.

\subsubsection{Stance adverbs}

Adverbs indicating the stance of the speaker towards an assertion or a 
statement are, for instance:
\xayr{AMkYu}{ankyu}{really}, 
\xayr{kYuymF}{cuyam}{actually, indeed, in fact},
\xayr{klmF}{kalam}{honestly},
\xayr{kubnF}{kuban}{fortunately},
\xayr{kuniru}{kuniru}{unfortunately},
\xayr{nilj}{nilay}{probably},
\xayr{yomiNF}{yoming}{maybe, perhaps}.
These adverbs are usually placed after the verb like any other attributive 
adverb, even though their scope is over the whole clause. It is also possible 
to place them towards the end of the clause they are used in, however. Example 
(\ref{ex:stanceadv}) gives an example of either position.

\pex\label{ex:stanceadv}
\a \begingl
	\gla Ang ming bengya kuban {} Tipal vahamya bavesangena nā. //
	\glb ang ming beng-ya kuban Ø Tipal vaham-ya bavesang-ena nā //
	\glc \AgtT{} can attend-\TsgM{} fortunately \Top{} Tipal party-\Loc{} 
		birthday-\Gen{} \Fsg{}.\Gen{} //
	\glft `Fortunately, Tipal can attend my birthday party.' //
\endgl

\a\label{ex:naaryaadv} \begingl
	\gla Sahayāng cabo-kay nilay nārya. //
	\glb saha=yāng cabo=kay nilay nārya //
	\glc come=\TsgM{} late=a.little probably though //
	\glft `He will probably come a little late, though. //
\endgl
\xe

\subsubsection{Discourse-structuring adverbs}
\label{subsubsec:discourseadv}

Ayeri does not have a great number of concessive adverbs, that is,
\xayr{AreenF}{arēn}{however, anyway} and \xayr{naarY}{nārya}{although, though;
nevertheless} do most, if not all the work. Like adverbs expressing stance,
they may follow the verb or be placed at the end of the clause. Example
(\ref{ex:naaryaadv}) above already shows an example of \rayr{naarY}{nārya}
being used as a sentence adverb. With regards to this word, it is important to
note that \rayr{naarY}{nārya} may also be used as a general contrastive
conjunction which can mostly be translated as `but'. In this sense, its
placement in a clause creates a slight difference in meaning, as illustrated
by example (\ref{ex:naaryaplcmt}) below.\footnote{Possibly it is easier not to
distinguish between conjunction and adverb at all in this case and instead to
treat \rayr{naarY}{nārya} as an adverb with a general constrastive meaning
which can exceptionally be found at the beginning of a clause as
well.\label{fn:naaryapos}}

\pex\label{ex:naaryaplcmt}
\a\label{ex:naaryaconj}\begingl
	\gla Garayang, nārya guraca ranyāng. //
	\glb gara=yang nārya gurat-ya ranya-ang //
	\glc call=\Fsg{}.\Aarg{} but answer-\TsgM{} nobody-\Aarg{} //
	\glft `I called, but nobody answered.' //
\endgl

\a\label{ex:naaryaadv2}\begingl
	\gla Garayang, guraca nārya ranyāng. //
	\glb gara=yang gurat-ya nārya ranya-ang //
	\glc call=\Fsg{}.\Aarg{} answer-\TsgM{} although nobody-\Aarg{} //
	\glft `I called, although nobody answered.' //
\endgl

\xe

Besides the two adverbs mentioned above, there is also
\xayr{d/naarY}{da-nārya}{even though, in spite of, despite} as a postposition
with a contrastive meaning (see \autoref{subsec:postpos}). As an adposition it
accepts either a noun phrase or a complement phrase (CP) as a complement. In
the latter case, which is shown in (\ref{ex:danaarya2}), there is no locative
case agreement of the whole CP with the postposition, since there is no fitting
agreement target to attach it to.

\pex\label{ex:danaarya}
\a\label{ex:danaarya1}\begingl
	\gla Ya precang nanga yena {\normalfont[\tsup{PP} [\tsup{NP}} @ sarānya 
		yena @ {\normalfont]} da-nārya @ {\normalfont]}. //
	\glb ya pret=yang nanga-Ø yena {} sarān-ya yena {} da-nārya {} //
	\glc \LocT{} knock=\Fsg{}.\Aarg{} house-\Top{} \TsgF{}.\Gen{} {}
		absence-\Loc{} \TsgF{}.\Gen{} {} in.spite {} //
	\glft `I knocked at her house in spite of her absence.' //
\endgl

\a\label{ex:danaarya2}\begingl
	\gla Precang {\normalfont[\tsup{PP} [\tsup{CP}} @ ang yomoyye rangya 
		yena @ {\normalfont]} da-nārya @ {\normalfont]}. //
	\glb pret=yang {} ang yoma-oy=ye.Ø rang-ya yena {} da-nārya {} //
	\glc knock=\Fsg{}.\Aarg{} {} \AgtT{} exist-\Neg{}=\TsgF{}.\Top{} 
		home-\Loc{} \TsgF{}.\Gen{} {} even.though {} //
	\glft `I knocked, even though she wasn't at home.' //
\endgl

\xe

Further adverbs which are commonly used as adverbial expressions and which may 
appear in the presentation of arguments include:
\xayr{dermYmF}{deramyam}{after all},
\xayr{kjbunj}{kaybunay}{by the way},
\xayr{ku/nsY}{ku-nasya}{as follows},
\xayr{mennFy}{menanya}{on the one hand},
\xayr{mirMpluj}{mirampaluy}{otherwise},
\xayr{naareNF}{nāreng}{rather},
\xayr{njnj}{naynay}{(and) also, moreover, furthermore},
\xayr{pluNnY}{palunganya}{on the other hand},
\xayr{pMtY}{panca}{finally, eventually, in the end},
\xayr{pinYnF}{pinyan}{please},
\xayr{suhiNF}{suhing}{naturally, of course}.
It should be apparent by the complexity and relative length of some of these 
words that they are fossilized expressions, for instance, 
\xayr{dermYmF}{deramyam}{after all} transparently derives from 
\xayr{dermF}{deram}{matter of fact} declined for dative case 
(\rayr{/ymF}{yam}, see \autoref{subsubsec:dative}); 
\rayr{ku/nsY}{ku-nasya} is derived from a phrase literally meaning `as 
(it) follows'; and \xayr{pluNnY}{palunganya}{on the other hand} literally 
means `in difference', from \xayr{pluNnF}{palungan}{difference, 
distinction}. Of the list given above, it may be noted that 
\xayr{pinYnF}{pinyan}{please} (from \xayr{pinY/}{pinya-}{ask}) is often found 
at the beginning of polite requests:

\ex
\begingl
	\gla Pinyan, sahu kongya! //
	\glb pinyan saha-u kong-ya //
	\glc please come-\Imp{} inside-\Loc{} //
	\glft `Please come inside!' //
\endgl
\xe

\subsubsection{Conjunctive adverbs}
\label{subsubsec:conjadv}

The term `conjunctive adverb' here refers to sentence adverbs which have the
distribution of a conjunction. Whereas sentence adverbs are normally placed
either after the verb or at the end of a clause, these words are usually found
as introducing clauses since they connect two otherwise independent statements
to show their relation to each other. Their meaning extends that of the
`pure', logical conjunctions \xayr{nj}{nay}{and} and \xayr{soyNF}{soyang}{or},
however.\footnote{Logical `not' is usually expressed by a negative suffix on
the adjective or the verb, compare sections \ref{subsec:adjneg} and
\ref{subsubsec:verbneg}, respectively. For conjunctions proper, see
\autoref{sec:conjunctions}.} Part of this small class of words are the
expressions \xayr{bt}{bata}{if, whether},\footnote{Conditional protasis and
apodosis are usually unmarked in Ayeri, however, it may still be desirable
occasionally to use a particle to indicate them explicitly.}
\xayr{bt}{bata}{if, whether}
\xayr{kd}{kada}{then, thus},
\xayr{kd/kd}{kada-kada}{so that ... again},
\xayr{kdaare}{kadāre}{so that},
\xayr{njnj}{naynay}{moreover, furthermore, and also},
\xayr{naareNF}{nāreng}{(but) rather},
\xayr{naaroj}{nāroy}{but not},
\xayr{naarY}{nārya}{but, except that, though, yet},
\xayr{siniNF}{sining}{that is}, and 
\xayr{ynoymF}{yanoyam}{because, for, since}.

\pex
\a\begingl
	\gla Le rimasayang kunang sirutayya, kadāre ming toryang ban-eng. //
	\glb le rima-asa=yang kunang-Ø sirutay-ya kadāre ming tor=yang 
		ban=eng //
	\glc \PatTI{} shut-\Hab{}=\Fsg{}.\Aarg{} door-\Top{} night-\Loc{} 
		so.that can sleep=\Fsg{}.\Aarg{} good=\Comp{} //
	\glft `I usually close the door at night so that I can sleep better.' //
\endgl

\a\label{ex:but}\begingl
	\gla Ilta toryeng, nārya da-kilisoyyon nilanjang yena. //
	\glb ilta tor=yeng nārya da=kilis-oy-yon nilan-ye-ang yena //
	\glc need sleep=\TsgF{}.\Aarg{} but so=allow-\Neg{}-\TplN{} 
		thought-\Pl{}-\Aarg{} \TsgF{}.\Gen{} //
	\glft `She needed to sleep, but her thoughts did not allow her to.' //
\endgl

\a\begingl
	\gla Ang ming hangoyya {} Yan padangas, yanoyam yāng pisu. //
	\glb ang ming hang-oy-ya Ø Yan padang-as yanoyam yāng pisu //
	\glc \AgtT{} can keep-\Neg{}-\TsgM{} \Top{} Yan mind-\Parg{} because 
		\TsgM{}.\Aarg{} tired //
	\glft `Yan cannot concentrate because he is tired.' //
\endgl

\xe

Regarding (\ref{ex:but}), it needs to be pointed out that \rayr{naarY}{nārya} 
can also be used as a regular adverb. In those cases it is considered to have 
less contrastive force, however: postposed \rayr{naarY}{nārya} is best 
translated as `though, although' (compare \autoref{subsubsec:discourseadv}).

Since verbs can be negated and reduplicated for grammatical purposes, the
adverbs \xayr{kd/kd}{kada-kada}{so that ... again} and \xayr{naaroj}{nāroy}{but
not} are mostly used with predicative adjectives, since equative statements
lack a verb to apply verb morphology to. These two conjunctive adverbs thus can
convey the most important distinctions otherwise expressed by the verb as a
substitute. This ability, however, is not a productive grammatical process, but
specific to \rayr{kd/kd}{kada-kada} and \rayr{naaroj}{nāroy}, respectively.

\pex
\a\label{ex:sothatagain}\begingl
	\gla Rua nibaya ang Pulan, kada-kada yāng sapin tadayya kivo. //
	\glb rua niba-ya ang Pulan kada\til{}kada yāng sapin taday-ya kivo //
	\glc must rest-\TsgM{} \Aarg{} Pulan \Iter{}\til{}so.that 
		\TsgM{}.\Aarg{} healthy time-\Loc{} little //
	\glft `Pulan must rest so that he will be healthy again very soon.' //
\endgl

\a\label{ex:butnot}\begingl
	\gla Yang temisena cuyam, nāroy yang petau. //
	\glb yang temis-ena cuyam nāroy yang petau //
	\glc \Fsg{}.\Aarg{} north-\Gen{} indeed but.not \Fsg{}.\Aarg{} stupid //
	\glft `I may be from the north, but I am not stupid.' //
\endgl

\xe

As described above (compare \autoref{subsubsec:iterative}), partial
reduplication of the verb expresses iterative aspect, which in Ayeri is used
to mean `\textsc{verb} again' and `\textsc{verb} back', depending on context.
The reduplicated form \rayr{kd/kd}{kada-kada} as displayed in
(\ref{ex:sothatagain}) is irregular in its formation if we assume that it is
formed from \xayr{kdaare}{kadāre}{so that}; the regular outcome with
iterative reduplication applied would be *\rayr{k/kdaare}{*ka-kadāre}. As a
conjunction, however, it is relatively frequent, so it does not seem odd that
it has assumed a phonologically more  simple, yet distinct form (compare, for
instance,  \cite[11--12]{bybeehopper2001b}). The conjunctive adverb in
(\ref{ex:butnot}) exhibits likewise a slightly irregular formation if we
consider that it is essentially the negated form of \xayr{naarY}{nārya}{but};
the regular outcome would have been *\rayr{naarYoj}{*nāryoy}, which underwent
simplification to \rayr{naaroj}{nāroy}, presumably as well due to its
relatively high token frequency.

\index{adverbs!sentence adverbs|)}

\subsection{Demonstrative adverbs}
\index{adverbs!demonstrative adverbs|(}

\begin{table}[tp]\centering
\caption{Demonstratives relating to adverbial categories}
\begin{tabu} to .75\textwidth {C I X I X}
\tableheaderfont\toprule
Category
	& \multicolumn{2}{c}{Proximal}
	& \multicolumn{2}{c}{Distal}
	\\
\toprule

place
	& edaya
	& `here'
	& adaya
	& `there'
	\\
	
\midrule

time
	& edauyi
	& `now'
	& adauyi
	& `then'
	\\
	
\midrule

manner
	& edāre
	& `hereby'
	& adāre
	& `thereby'
	\\
	
\midrule

reason
	& edayam
	& `herefore'
	& adayam
	& `therefore'
	\\
	
\bottomrule

\end{tabu}
\label{tab:demadv}
\end{table}

Besides demonstrative pronouns like \xayr{AdnY}{adanya}{that (one)} (see
\autoref{subsec:dempro}), and indefinite pronouns like
\xayr{yaarilF}{yāril}{for some reason; somewhere} (see
\autoref{subsec:indefpro}), Ayeri also possesses demonstrative pronouns for the
adverbial categories place, time, manner, and reason. The full paradigm is
given in \autoref{tab:demadv}. Compared to the paradigm for demonstrative
pronouns relating to persons or things, the paradigm of adverbial
demonstratives is incomplete in that forms with
\xayr{d/}{da-}{such} are unattested. Thus, instead of the hypothetical form
with \rayr{d/}{da-}, a full-NP adverbial with a generic noun has to be used:
*\rayr{dy}{*daya} → \xayr{d/ynoy}{da-yanoya}{in such a place}
(such-place-\Loc{}). Adverbial demonstratives are, like pronouns, in
complementary distribution with full NPs, since they are pro-forms. Thus, using
them as modifiers to NPs as in (\ref{ex:edayamod}) is not possible, while using
simple demonstrative \xayr{Ed/}{eda-}{this} together with a noun as in
(\ref{ex:edanp}) or using \xayr{Edy}{edaya}{here} as a pro-form fully replacing
the NP \xayr{Ed/nNy}{eda-nangaya}{in this house} as in (\ref{ex:edanyapro}) is
generally unproblematic.

\pex
\a\ljudge*\label{ex:edayamod}\begingl
	\gla Ang mice {} Pada nangaya edaya. //
	\glb ang mit-ye Ø Pada nanga-ya edaya //
	\glc \AgtT{} live-\TsgF{} \Top{} Pada house-\Loc{} here //
\endgl

\a\label{ex:edanp}\begingl
	\gla Ang mice {} Pada eda-nangaya. //
	\glb ang mit-ye Ø Pada eda=nanga-ya //
	\glc \AgtT{} live-\TsgF{} \Top{} Pada this=house-\Loc{} //
	\glft `Pada lives in this house.' //
\endgl

\a\label{ex:edanyapro}\begingl
	\gla Mice ang Pada edaya. //
	\glb mit-ye ang Pada edaya //
	\glc live-\TsgF{} \Aarg{} Pada here //
	\glft `Pada lives here.' //
\endgl

\xe

\index{adverbs!demonstrative adverbs|)}

\index{adverbs|)}

\section{Numerals}
\label{sec:numerals}
\index{numerals|(}

The vast majority of the 196 sampled languages in \citet{wals131} either counts
in tens or employs a mixed vigesimal-decimal system, while only five languages
in the sample use a different base than 10. Ayeri uses a duodecimal system and
is thus very untypical compared to real-world languages in using a number base
other than 10---none of the languages in \citet{wals131}'s sample are listed as
duodecimal.\footnote{I chose to use 12 as a numerical base because I simply
wanted to toy with it. Also, I originally conceived of Ayeri speakers as
humanoid but not necessarily human, which meant that they would not need to
have evolved to have five fingers on each hand---for an earlier fictional
language of mine, Daléian, I used an octal system reasoning that speakers would
only have four digits. In any case, a duodecimal system could work reasonably
well with human hands if you counted not only the fingers, but also the hands
themselves. Finger-counting in Ayeri's duodecimal system would probably be
similar to counting in the senary system of Nen described in
\citet{evans2009} (as quoted in \cite{dixon2012}):
\textcquote[73--74]{dixon2012}{In counting, Nen speakers `first count off the
five fingers with a finger of their other hand, and then on the sixth they
place their counting finger on the inside of the wrist'}. Even though
duodecimal numeral systems only occur rarely in in natural languages, they are
not entirely unheard of. Thus, for instance, \citet{caingair2000} report that
in Maldivian, the \textcquote[21]{caingair2000}{decade plus numeral system is
currently in fashion, but with some remnants of an older system as well. The
numeral \fw{fas doḷas} `60' (lit., `five twelves') comes from a duodecimal
system that has all but disappeared in the Maldives. This number system was
used for special purposes such as counting coconuts}.} Ayeri's number words
are mostly semantic primes, that is, their meanings cannot be readily
recognized as derived from body parts \citep[74]{dixon2012} or from internal
arithmetic like 9 as `ten lacking one', for instance. The numerals
\xayr{kj}{kay}{three}, \xayr{Iri}{iri}{five}, and \xayr{henF}{hen}{eight} may
be an exception: as a quantifying adverb, \rayr{kj}{kay} means `a little, few';
\rayr{Iri}{iri} means `already', which might refer to the fact that a full hand
has been counted off; and \rayr{henF}{hen} also means `all'. Ayeri moreover
appears extremely sophisticated as far as the upper limit of counting systems
is concerned in possessing a way of forming large numerals by a theoretically
open-ended, recursive process.

\subsection{Cardinal numerals}
\index{numerals!cardinal|(}

Since people and concrete things are usually present in a countable manner, I
want to comment first on how countable entities are handled with regards to
numerals. After this, a discussion of how to express fractional amounts will
follow.

\subsubsection{Integers}
\index{numerals!integers|(}

Cardinal numerals work much like adjectives in that they modify nouns. As
modifiers, they are placed after nouns. The full table of cardinal numerals
from $0 \times 12^0$ (0) to $11 \times 12^0$ (\elv) is given in
\autoref{tab:cardinals}.\footnote{For the sake of typographic simplicity, \ten\
and \elv\ will henceforth be used to mean $10 \times 12^0$ and $11 \times
12^0$, respectively. An index `10' after a figure indicates base 10, while an
index `12' indicates base 12.} An example of simple modification by a numeral
is given in (\ref{ex:nummod}):

\ex\label{ex:nummod}
\begingl
	\gla Ang tenyaya pang bihanya yo soyang miye. //
	\glb ang tenya=ya pang bihan-ya yo soyang miye //
	\glc \AgtT{} die=\TsgM{}.\Top{} ago week-\Loc{} four or six //
	\glft `He died four or six weeks ago.' //
\endgl
\xe

\begin{table}[tp]\centering
\caption{Basic cardinal numerals}
\begin{tabu} to .75\linewidth {X[c] X[c] X[c] X[c]}
\toprule\tableheaderfont
Numeral
	& Word
	& Numeral
	& Word
	\\
\toprule

0
	& \rayr{dY}{ja}
	& 6
	& \rayr{miye}{miye}
	\\

1
	& \rayr{menF}{men}
	& 7
	& \rayr{Ito}{ito}
	\\
	
2
	& \rayr{smF}{sam}
	& 8
	& \rayr{henF}{hen}
	\\
	
3
	& \rayr{kj}{kay}
	& 9
	& \rayr{vey}{veya}
	\\

4
	& \rayr{yo}{yo}
	& \ten
	& \rayr{mlF}{mal}
	\\

5
	& \rayr{Iri}{iri}
	& \elv
	& \rayr{tmF}{tam}
	\\

\bottomrule
\end{tabu}
\label{tab:cardinals}
\end{table}

In this example, the numeral \xayr{yo}{yo}{four} modifies the noun 
\xayr{bihnF}{bihan}{week}. Notably, however, plural marking is missing on the 
noun, since the notion of plurality is provided by the numeral itself; the 
numeral is thus normally sufficient to mark the whole NP as plural.

\begin{table}[tp]\centering
\caption{Numerals for factors of 12}
\begin{tabu} to .75\linewidth {X[1c] X[2c] X[1c] X[2c]}
\toprule\tableheaderfont
Numeral
	& Word
	& Numeral
	& Word
	\\
\toprule

&
& 60 & \rayr{miyelnF}{miyelan} \\

10 & \rayr{menFlnF}{menlan}
& 70 & \rayr{ItolnF}{itolan} \\

20 & \rayr{smFlnF}{samlan}  
& 80 & \rayr{henFlnF}{henlan} \\

30 & \rayr{kjlnF}{kaylan}
& 90 & \rayr{veylnF}{veyalan} \\

40 & \rayr{yolnF}{yolan}
& \ten0 & \rayr{mlFlnF}{mallan} \\

50 & \rayr{IrilnF}{irilan}
& \elv0 & \rayr{tmFlnF}{tamlan} \\

\bottomrule
\end{tabu}
\label{tab:cardinalsten}
\end{table}

Multiples of $12^1$ between 10 and \elv0 are formed by appending the suffix
\rayr{/lnF}{\mbox{-lan}} to the numbers from 0 to \elv, which are given in
\autoref{tab:cardinalsten}. These numerals themselves act as heads for forming
compounds with lower numerals to fill in the $12^0$ numerals 11, 12, 13, ...,
21, 22, 23, etc.\ Thus, one counts on from \xayr{menFlnF}{menlan}{dozen} in
the  following way:

\pex
\a %
	\rayr{\larger menFlnF/menF}{menlan-men} (11), \\
	\rayr{\larger menFlnF/smF}{menlan-sam} (12), \\
	\rayr{\larger menFlnF/kj}{menlan-kay} (13),  \medskip
	
	etc.

\a %
	\rayr{\larger smFlnF/menF}{samlan-men} (21), \\
	\rayr{\larger smFlnF/smF}{samlan-sam} (22), \\
	\rayr{\larger smFlnF/kj}{samlan-kay} (23),  \medskip
	
	etc.
	
\a %
	\rayr{\larger tmFlnF/menF}{tamlan-men} (\elv1), \\
	\rayr{\larger tmFlnF/smF}{tamlan-sam} (\elv2), \\
	\rayr{\larger tmFlnF/kj}{tamlan-kay} (\elv3), \medskip
	
	etc.
\xe

In order to form yet higher numbers, the suffix \rayr{/nNF}{-nang} is appended
to numerals: \rayr{menNF}{menang} (←~\xayr{menF}{men}{1} +
\rayr{/nNF}{-nang}), \xayr{smNF}{samang}{2} (←~\rayr{smF}{sam} +
\rayr{/nNF}{-nang}), \rayr{kjnNF}{kaynang} (←~\xayr{kj}{kay}{3} +
\rayr{/nNF}{-nang}), etc.\ While \rayr{menNF}{menang} is used for 100, higher
forms in the \fw{nang} series each multiply the numeral from which they are
derived by the factor of a duodecimal myriad (=~20\,736\tsub{10}). Thus, we get
the following series:

\ex[everyex={\tabcolsep=0em},]
	\begin{tabular}[t]
	{l @{\quad} l @{\quad} l}
	\rayr{\larger smNF}{samang}
		& $12^{(2-1) \times 4} = 12^{4}$
		& myriad
		\\
		
	\rayr{\larger kynNF}{kaynang}
		& $12^{(3-1) \times 4} = 12^{8}$
		& myriad myriads
		\\
		
	\rayr{\larger yonNF}{yonang}
		& $12^{(4-1) \times 4} = 12^{12}$
		& myriad myriad myriads
		\\
		
	\rayr{\larger IrinNF}{irinang}
		& $12^{(5-1) \times 4} = 12^{16}$
		& myriad myriad myriad myriads
		\\
	\end{tabular}
	
	\medskip etc.
\xe

The numeral which the \fw{nang} series word is based on essentially indicates 
the number of myriad groups, thus, 1-\fw{nang} maximally contains 
\elv\elv\elv\elv; 2-\fw{nang} maximally contains 
\elv\elv\elv\elv\,\elv\elv\elv\elv; 3-\fw{nang} maximally contains 
\elv\elv\elv\elv\,\elv\elv\elv\elv\,\elv\elv\elv\elv, etc.\ Furthermore, the 
\fw{nang} series words serve as unit words, and thus can be modified by 
numerals again, for instance:

\pex[glwordalign=center]
\a\label{ex:menangunit}\begingl
	\gla menang sam veyalan @ -kay //
	\glb {100} {2} {90} {3} //
	\glft 293\tsub{12} = 399\tsub{10} //
\endgl

\a\label{ex:samangunit}\begingl
	\gla samang henlan @ -miye menang sam veyalan @ -kay //
	\glb {1\,0000} {80} {6} {100} {2} {90} {3} //
	\glft 86\,0293\tsub{12} = 2\,115\,471\tsub{10} //
\endgl

\xe

In (\ref{ex:menangunit}), \rayr{smF}{sam} modifies \rayr{menNF}{menang} to 
indicate that there are two sets of 100\tsub{12}. Likewise, in 
(\ref{ex:samangunit}), \rayr{smNF}{samang} is modified by 
\rayr{henFlnF/miye}{henlan-miye} to mean 86\tsub{12} times 10\,000\tsub{12}.
Unit words like \rayr{menNF}{menang}, \rayr{smNF}{samang}, etc.\ may also be 
used as (inanimate) nouns, so it is possible to speak of  
\xayr{menNYe}{menangye}{hundreds} \label{hundreds}. To express `hundreds of
people', however, the head of the genitive NP is pluralized exceptionally, even
if it is a \fw{plurale tantum}:

\pex\label{ex:keynampltant}
\a\label{ex:keynamunmkd}\begingl
	\gla Ang bengyon \textbf{keynam} menang kanānya {desay iray}. //
	\glb ang beng-yon keynam-Ø menang kanān-ya {desay iray} //
	\glc \AgtT{} attend-\TplN{} people-\Top{} hundred wedding-\Loc{} 
		royal //
	\glft `A hundred people attended the royal wedding.' //
\endgl

\a\label{ex:keynammkd}\begingl
	\gla Ang bengyon \textbf{keynamye} menang kanānya {desay iray}. //
	\glb ang beng-yon keynam-ye-Ø menang kanān-ya {desay iray} //
	\glc \AgtT{} attend-\TplN{} people-\Pl{}-\Top{} hundred wedding-\Loc{} 
		royal //
	\glft `Hundreds of people attended the royal wedding.' //
\endgl

\xe

In (\ref{ex:keynampltant}), \rayr{kejnmF}{keynam} is morphologically a
singular form referring semantically to a multitude. It is usually treated as
a \textbf{plurale tantum} in that it triggers plural agreement in spite of
being morphologically singular, which is illustrated in
(\ref{ex:keynamunmkd}). In (\ref{ex:keynammkd}), the word still receives
otherwise redundant plural marking to express the difference in meaning from
(\ref{ex:keynamunmkd}).

In order to indicate that myriad groups have been skipped, the conjunction
\xayr{nj}{nay}{and} is used to avoid confusion, as shown in
(\ref{ex:numconfuse}), or simply to avoid having two single-digit numerals
following each other, as illustrated by (\ref{ex:numsgldig}).

\pex[glwordalign=center]\label{ex:numconfuse}
\a\begingl
	\gla samang menang men henlan @ -miye //
	\glb {1\,0000} {100} {1} {80} {6} //
	\glft 186\,0000\tsub{12} = 5\,101\,056\tsub{10} //
\endgl

\a\begingl
	\gla samang menang men nay henlan @ -miye //
	\glb {1\,0000} {100} {1} and {80} {6} //
	\glft 100\,0086\tsub{12} = 2\,986\,086\tsub{10} //
\endgl

\xe

\pex~[glwordalign=center]\label{ex:numsgldig}
\a\ljudge\ques\begingl
	\gla menang mal ito //
	\glb {100} {\ten} {7} //
\endgl

\a\begingl
	\gla menang mal nay ito //
	\glb {100} {\ten} and {7} //
	\glft \ten07\tsub{12} = 1\,447\tsub{10} //
\endgl

\xe

\index{numerals!integers|)}

\subsubsection{Fractions}
\index{numerals!fractions|(}

\begin{table}[tp]\centering
\caption[Simple fractions from $\frac{1}{2}$ to $\frac{1}{\elv}$]{Simple 
fractions from ¹⁄₂ to ¹⁄\tsub{\elv}}
\begin{tabu} to .75\linewidth {X[1c] X[2c] X[1c] X[2c]}
\toprule\tableheaderfont
Numeral
	& Word
	& Numeral
	& Word
	\\
\toprule

$\frac{1}{2}$ & \rayr{mesmF}{mesam}
& $\frac{1}{7}$ & \rayr{menito}{menito} \\ [.25\baselineskip]

$\frac{1}{3}$ & \rayr{meMkj}{menkay}
& $\frac{1}{8}$ & \rayr{menFyenF}{menyen} \\ [.25\baselineskip]

$\frac{1}{4}$ & \rayr{menFyo}{menyo}
& $\frac{1}{9}$ & \rayr{menFvey}{menveya} \\ [.25\baselineskip]

$\frac{1}{5}$ & \rayr{meniri}{meniri}
& $\frac{1}{\ten}$ & \rayr{memlF}{memal} \\ [.25\baselineskip]

$\frac{1}{6}$ & \rayr{memiye}{memiye}
& $\frac{1}{\elv}$ & \rayr{meMtmF}{mentam} \\

\bottomrule
\end{tabu}
\label{tab:smallfrac}
\end{table}

So far, we have explored only integers, that is, whole numbers. Since things
can often be divided up into smaller sections as well, this section is going to
deal with how to express fractional amounts. The main way to express common
fractions like $\frac{1}{2}$, $\frac{1}{3}$, $\frac{1}{4}$, etc.\ is to prepend
\xayr{menF}{men}{one} to the denominator; the full paradigm for $\frac{1}{2}$
to $\frac{1}{\elv}$ is given in \autoref{tab:smallfrac}. Note that a number of
these fractions have slightly irregular forms due to assimilation in consonant
clusters. In order to introduce a numerator, the fraction numeral is used as a
unit word which is modified by a regular cardinal numeral, as
(\ref{ex:simplefrac}) shows.

\pex\label{ex:simplefrac}
\a\begingl
	\gla Ang ilca {} Yan vadisānley mesam. //
	\glb ang ilt-ya Ø Yan vadisān-ley mesam //
	\glc \AgtT{} buy-\TsgM{} \Top{} Yan bread-\PargI{} half //
	\glft `Yan bought half a loaf of bread.' //
\endgl

\a\begingl
	\gla Ang ilce {} Mali sikanley menyo kay kipunena. //
	\glb ang ilt-ye Ø Mali sikan-ley menyo kay kipunena //
	\glc \AgtT{} buy-\TsgM{} \Top{} Mali pound-\PargI{} fourth three
		cheese-\Gen{} //
	\glft `Mali bought a three-quarter pound of cheese.' //
\endgl
\xe

In order to express compound numerals, \rayr{menF/}{men-} may still be added to
the denominator head word, for instance

\ex
\begingl
	\gla memallan-hen //
	\glb men-mallan-hen //
	\glb {$1/$ $10 \times 12^1 + 8$} //
	\glft $\frac{1}{\ten8_{12}}$ = $\frac{1}{128_{10}}$ //
\endgl
\xe

\noindent However, this may become confusing if numerators are used, so %

\ex
\ljudge\ques\begingl
	\gla memenang ito menlan-yo kay //
	\glb men-menang ito menlan-yo kay //
	\glb {$1/$ $12^2$} {$7$} {$1 \times 12^1 + 4$} {$3$} //
	\glft $\frac{3}{714_{12}}$ = $\frac{3}{1024_{10}}$ //
\endgl
\xe

\noindent would be expressed less ambiguously as

\ex
\begingl
	\gla menangan ito menlan @ -yo nernanyena kay //
	\glb menang-an ito menlan -yo nernan-ye-na kay //
	\glc {$12^2$-\Nmlz{}} {$7$} {$1 \times 12^1$} {$+4$} part-\Pl{}-\Gen{} 
		$3$ //
	\glft `three of the 1\,024th part' //
\endgl
\xe

\noindent using the ordinal form of the denominator.

\index{numerals!fractions|)}

\index{numerals!ordinal|)}

\subsection{Ordinal numerals}
\index{numerals!ordinal|(}

Ordinal numerals are formed by nominalization from cardinal numerals. This may 
be another slightly odd strategy, however, it is in fact attested in Classical 
Tibetan, according to \citet{chungetal2014}, in reference to \citet{beyer1992}:

\blockcquote[626]{chungetal2014}{The suffix \fw{-pa} forms a noun from another 
noun, meaning `associated with N' (e.g. \fw{rta} `horse,' \fw{rta-pa} 
`horseman,' \fw{yi-ge} `letter,' \fw{yi-ge-pa} `one who holds a letter of 
office,' cf. \nocite{beyer1992} Beyer 1992: 117). When suffixed to ordinal 
numbers this suffix forms ordinals (e.g. \fw{gsum} `three,' \fw{gsum-pa} 
`third'; \fw{bcu} `ten,' \fw{bcu-pa} `tenth').}

\begin{table}[tp]\centering
\caption{Basic ordinal numerals}
\begin{tabu} to .75\linewidth {X[c] X[c] X[c] X[c]}
\toprule\tableheaderfont
Numeral
	& Word
	& Numeral
	& Word
	\\
\toprule

0th
	& \rayr{dYaanF}{jān}
	& 6th
	& \rayr{miynF}{miyan}
	\\

1st
	& \rayr{mennF}{menan}
	& 7th
	& \rayr{ItnF}{itan}
	\\
	
2nd
	& \rayr{smnF}{saman}
	& 8th
	& \rayr{hennF}{henan}
	\\
	
3rd
	& \rayr{kynF}{kayan}
	& 9th
	& \rayr{veyaanF}{veyān}
	\\

4th
	& \rayr{ynF}{yan}
	& \ten{}th
	& \rayr{mlnF}{malan}
	\\

5th
	& \rayr{Irni}{iran}
	& \elv{}th
	& \rayr{tmnF}{taman}
	\\

\bottomrule
\end{tabu}
\label{tab:ordinals}
\end{table}

Unfortunately, neither \citet{chungetal2014} nor \citet{beyer1992} say whether
Classical Tibetan treats these derived forms as nouns or as numerals, or
whether it makes that distinction at all. In Ayeri, ordinals are firmly treated
as noun-like nominal elements due to the derivational suffix
\rayr{/AnF}{-an} (compare \autoref{subsec:nominalization}). Since nominals are
the heads of NPs, this also means that the ordinal numeral forms the head of
the NP it occurs in, instead of modifying the entity being counted like an
ordinal numeral does. This is illustrated in (\ref{ex:ord}) below. The paradigm
for the ordinal numerals from 0 to \elv\ can be found in
\autoref{tab:ordinals}.

\pex\label{ex:ord}
\a\label{ex:ordanaph}\begingl
	\gla Ang Mahān menanas. //
	\glb ang Mahān menan-as //
	\glc \Aarg{} Mahān first-\Parg{} //
	\glft `Mahan is the first.' //
\endgl

\a\label{ex:ordrel}\begingl
	\gla Ang Mahān menanas si girenjāng. //
	\glb ang Mahān menan-as si girend=yāng //
	\glc \Aarg{} Mahān first-\Parg{} \Rel{} arrive=\TsgM{}.\Aarg{} //
	\glft `Mahān is the first to arrive.' //
\endgl

\a\label{ex:ordadj}\begingl
	\gla Ang girenja {} Mahān bahalanya ku-menan diyan //
	\glb ang girend-ya Ø Mahān bahalan-ya ku=menan diyan //
	\glc \Aarg{} arrive-\TsgM{} \Top{} Mahān finish-\Loc{} like=first 
		worthy //
	\glft `Mahān arrives at the finish as a worthy first.' //
\endgl

\a\label{ex:ordgen}\begingl
	\gla Ang tavya {} Mahān menanas ganyena yana. //
	\glb ang tav-ya Ø Mahān menan-as gan-ye-na yana //
	\glc \AgtT{} get-\TsgM{} \Top{} Mahān first-\Parg{} child-\Pl{}-\Gen{} 
		\TsgM{}.\Gen{} //
	\glft `Mahān gets his first child', \\
		literally: `Mahān gets the first of his children.' //
\endgl

\xe

As (\ref{ex:ordanaph}) shows, the ordinal may serve as an anaphora meaning `the
$n$th (one)'. In these cases, animacy is determined by the word the ordinal
references, as far as case marking and person agreement are concerned. Since
ordinals are treated as nominals, they can also be modified by both a relative
clause, as (\ref{ex:ordrel}) shows, and an adjective, as shown in
(\ref{ex:ordadj}). In order to include an entity whose rank in a series is
given, the counted entity appears as a genitive attribute, which is illustrated
by (\ref{ex:ordgen}).

So far, only single-digit ordinals have been described. In order to form higher
ordinals, the head unit word receives the nominalizer with the rest of the term
trailing as a modifier, otherwise the number word as such is nominalized.
Essentially, an ordinal in the `teens'\footnote{More specifically, $a \times
12^1 + b$ with $\{a,b \in \textbf{Z} \mid 0 < a,b < 12^1\}$.} behaves like a
`tight' noun compound, while ordinals involving unit words for powers of 12
higher than 1 behave as `loose' compounds (compare
\autoref{subsubsec:endocomp}, p.~\pageref{loosecomp}). 

\pex
\a\label{ex:ordtightcomp}\begingl
	\gla Adareng kaylan-miyanley bahisyena pericanena. //
	\glb ada-reng kaylan-miye-an-ley bahis-ye-na perican-ena //
	\glc that-\AargI{} {$3 \times 12^1 + 6$-\Nmlz{}-\PargI{}} 
		day-\Pl{}-\Gen{} year-\Gen{} //
	\glft `It is the 36th (=\,42nd) day of the year.' //
\endgl

\a\label{ex:ordloosecomp}\begingl
	\gla Adareng menanganley kaylan-miye bahisyena pericanena. //
	\glb ada-reng menang-an-ley kaylan-miye bahis-ye-na perican-ena //
	\glc that-\AargI{} {$12^2$-\Nmlz{}-\PargI{}} {$3 \times 12^1 
		+ 6$} day-\Pl{}-\Gen{} year-\Gen{} //
	\glft `It is the 136th (=\,186th) day of the year.' //
\endgl

\xe

In (\ref{ex:ordtightcomp}), the whole numeral \rayr{kjlnF/miye}{kaylan-miye} is
nominalized and inflected for case, yielding
\rayr{kjlnF/miynFlej}{kaylan-miyanley}. This is analogous to such nouns as
\xayr{betjniMpurF}{betaynimpur}{grape} (literally `wine-berry'), which inflects
as a single unit---a `tight' compound. In (\ref{ex:ordloosecomp}), on the other
hand, only the first unit word, \rayr{menNF}{menang} is nominalized and
inflected, yielding \rayr{menNnFlej}{menanganley} with
\rayr{kjlnF/miye}{kaylan-miye} following it uninflected. This is analogous to
\xayr{rlmpNF}{ralamapang}{fingernail} which is transparently made up of
\xayr{rlnF}{ralan}{nail} and \xayr{mpNF}{mapang}{finger} and for which only the
first constituent inflects, for instance, \xayr{rlnYen mpNF}{ralanyena
mapang}{of the fingernails} (nail-\Pl{}-\Gen{} finger)---a `loose' compound.

\index{numerals!cardinal|)}

\subsection{Multiplicative numerals}
\index{numerals!multiplicative|(}

Whereas ordinals are derived from cardinal numerals by nominalization,
multiplicative numerals are derived from ordinals (compare
\autoref{tab:ordinals}) in turn by putting them in the dative case: the suffix
\rayr{/ymF}{yam} is added to the ordinal form of the numeral. The resulting
multiplicative numeral can thus be used as an adverbial meaning `for the $n$th
time', or as an adverb meaning `$n$ times'. Context helps to disambiguate
between the two, as well as temporal adverbs like \xayr{Iri}{iri}{already}.

\pex
\a\begingl
	\gla Linkaya iri ang Anang kayanyam. //
	\glb linka-ya iri ang Anang kayanyam //
	\glc try-\TsgM{} already \Aarg{} Anang third-\Dat{} //
	\glft `Anang already tries it for the third time.' //
\endgl

\a\begingl
	\gla Linkaya iri kayanyam ang Anang. //
	\glb linka-ya iri kayanyam ang Anang //
	\glc try-\TsgM{} already three.times \Aarg{} Anang //
	\glft `Anang already tried it three times.' //
\endgl

\xe

Compound multiplicative numerals are treated analogously to ordinals, that is, 
for composite numerals smaller than $12^2$, the derivational marking is placed 
at the end of the composite numeral. Conversely, for composite numerals 
of orders of magnitude above $12^1$, the head of the phrase receives all 
the marking that makes it a multiplicative numeral while the rest trails 
uninflected as a modifier:

\pex
\a\begingl
	\gla kaylan-tamanyam //
	\glb kay-lan-tam-an-yam //
	\glc {$3 \times 12^1 + 11$-\Nmlz{}-\Dat{}} //
	\glft `3\elv{} (=\,47\tsub{10}) times' //
\endgl

\a\begingl
	\gla menanganyam men samlan-kay //
	\glb menang-an-yam men sam-lan-kay //
	\glc {$12^2$-\Nmlz{}-\Dat{}} {$1$} {$2 \times 12^1 + 3$} //
	\glft `123 (=\,171\tsub{10}) times' //
\endgl

\xe

\index{numerals!multiplicative|)}

\subsection{Distributive numerals}
\index{numerals!distributive|(}

Distributive numerals are formed similar to multiplicative numerals in that
they are based on a derivation of the respective ordinal numeral, which itself
has  the form of a nominalized cardinal numeral (compare
\autoref{tab:ordinals}). The derivative affix in this case is the
instrumental marker, \rayr{/Eri}{-eri}  (compare
\autoref{subsubsec:instrumental}). Distributive numerals refer to groups of
$n$, as example (\ref{ex:distnum}) shows.

\ex\label{ex:distnum}
\begingl
	\gla Ang sarayon burangjang kong besonya samaneri. //
	\glb ang sara-yon burang-ye-yang kong beson-ya sam-an-eri //
	\glc \AgtT{} go-\TplN{} animal-\Pl{}-\Aarg{} inside ship-\Loc{} 
		two-\Nmlz{}-\Ins{} //
	\glft `The animals went inside the ship two by two.' //
\endgl
\xe

The formation of composite numerals mirrors that of multiplicative numerals, in
that composite numerals below $12^1$ are treated as single units whereas
composite numerals of orders of magnitude larger than $12^1$ mark only the head
word while the remainder of the phrase follows as an uninflected modifier.

\pex
\a\begingl
	\gla henlan-yaneri //
	\glb hen-lan-yo-an-eri //
	\glc {$8 \times 12^1 + 4$-\Nmlz{}-\Ins{}} //
	\glft `84 by 84 (=\,100\tsub{10})' //
\endgl

\a\begingl
	\gla menanganeri miye tamlan-yo //
	\glb menang-an-eri miye tam-lan-yo //
	\glc {$12^2$-\Nmlz{}-\Ins{}} {$6$} {$11 \times 12^1 + 4$} //
	\glft `6\elv{}4 by 6\elv{}4 (=\,1\,000\tsub{10})' //
\endgl

\xe

\index{numerals!distributive|)}

\subsection{Number ranges}
\index{numerals!ranges|(}

Ranges of cardinal numbers may be viewed as conceptually similar to stretches
of way, hence they are predestined to be expressed by prepositional phrases,
or in Ayeri, by any of the cases which can be used for locative purposes
(dative, genitive, locative). However, Ayeri treats cardinal numerals more
like adjectives than nouns, so using means of case marking is not possible. On
the other hand, adpositions take both NPs and CPs as complements, so that an
adjective should be able to act as a complement to an adposition as well.
Since the numeral in the PP is treated like an adjective, it is not marked for
locative case, since adjectives do not inflect for nominal categories (compare
\autoref{sec:adjectives}). Ranges of cardinal numbers may hence be expressed
using the postposition \xayr{pesnF}{pesan}{(up) until}. When counting starts
at \xayr{menF}{men}{one}, this numeral may be dropped, like in English `count
to ten' instead of `count from one to ten'.

\ex
\begingl
	\gla Kurye ang Pila {\normalfont (}men{\normalfont )} tam pesan. //
	\glb kur-ye ang Pila (men) tam pesan //
	\glc count-\TsgF{} \Aarg{} Pila (1) \elv{} until //
	\glft `Pila counts from 1 to \elv{} (=\,1 … 11\tsub{10}).' //
\endgl
\xe

Since ordinal numerals are treated as nouns, they may receive case marking.
This means that, in contrast to cardinal numerals, it is possible to express a
range using a combination of the genitive and the dative case, or again
\rayr{pesnF}{pesan} with its prepositional object in the locative case.
Context is needed to disambiguate whether the dative form of the numeral is a
multiplicative derivation or an actual ordinal numeral in the dative case.
Examples for this are given in (\ref{ex:rangecase}).

\pex\label{ex:rangecase}
\a\label{ex:rangecase1}\begingl
	\gla Ang gumasaya samanena pidimyena da-malanyam. //
	\glb ang gum-asa=ya saman-ena bahis-ye-na da=malan-yam //
	\glc \AgtT{} work-\Hab{}=\TsgM{}.\Top{} second-\Gen{} hour-\Pl{}-\Gen{} 
		such=tenth-\Dat{} //
	\glft `He usually works from the second hour to the tenth.' //
\endgl

\a\label{ex:rangecase2}\begingl
	\gla Ang yomaya {} Magay diyan edaya henanena bahisyena da-menlananya 
		pesan. //
	\glb ang yoma-ya Ø Magay diyan edaya henan-ena bahis-ye-na 
		da-menlanan-ya pesan //
	\glc \AgtT{} exist-\TsgM{} \Top{} Magay worthy here eighth-\Gen{} 
		day-\Pl{}-\Gen{} such-dozenth-\Loc{} until //
	\glft `Mr. Magay is here from the eighth to the dozenth day.' //
\endgl

\xe

\xayr{smnen}{samanena}{from the first} in (\ref{ex:rangecase1}) and
\xayr{hennen}{henanena}{from the eighth} in (\ref{ex:rangecase2}) use the
genitive case marker \rayr{/En}{-ena} (compare \autoref{subsubsec:genitive}) to
indicate the starting point. \xayr{d/mlnYmF}{da-malanyam}{to the tenth one} and
\xayr{d/menFlnFy pesnF}{da-menlanya pesan}{up until the dozenth one} indicate 
the end points. Since \rayr{menFlnF}{menlan} in (\ref{ex:rangecase2}) is 
embedded in a PP headed by the postposition \rayr{pesnF}{pesan}, it appears in 
the locative case instead of the dative case like \rayr{mlnF}{malan} in 
(\ref{ex:rangecase1}).

\index{numerals!ranges|)}

\index{numerals|)}

\section{Conjunctions}
\label{sec:conjunctions}
\index{conjunctions|(}

\autoref{subsubsec:conjadv} already dealt with conjunctive adverbs as sentence
adverbs and their conjunction-like behavior. The present section is about the
`purely logical' conjunctions \xayr{nj}{nay}{and} and
\xayr{soyNF}{soyang}{or}, as well as their combination with 
\xayr{kmo}{kamo}{equal(ly)} to form correlative conjunctions.

\subsection{Simple conjunction and disjunction}

\index{conjunctions!nay@\fw{nay}|(}

Coordination is commonly achieved by the conjunction \xayr{nj}{nay}{and}. It is
placed in between the conjuncts, and works on all syntactic levels. Namely, it
may coordinate lexical heads, as well as phrases, and whole clauses.

\pex\label{ex:and}
\a\label{ex:andheads}\begingl
	\gla {\normalfont [\tsup{AP}[\tsup{A}} @ Taran @ {\normalfont ]} nay  
		{\normalfont [\tsup{A}} @ saco @ {\normalfont ]]} nangāng. //
	\glb {} Taran {} nay {} saco {} nanga-ang //
	\glc {} quiet {} and {} cool {} house-\Aarg{} //
	\glft `The house is quiet and cool.' //
\endgl

\a\label{ex:andphrases}\begingl
	\gla Ajayan {\normalfont [\tsup{NP}} @ yanang @ {\normalfont ]} nay 
{\normalfont [\tsup{NP}} @ layang @ {\normalfont ].} //
	\glb aja-yan {} yan-ang {} nay {} lay-ang {} //
	\glc play-\TplM{} {} boy-\Aarg{} {} and {} girl-\Aarg{} {} //
	\glft `The boy and the girl are playing.' //
\endgl

\a\label{ex:andclauses}\begingl
	\gla {\normalfont [\tsup{S}} @ nāng pisu @ {\normalfont ]} nay 
{\normalfont [\tsup{S}} @ tapannang {\normalfont ].} //
	\glb {} nāng pisu {} nay {} tapan-nang {} //
	\glc {} \Fpl{}.\Aarg{} tired {} and {} be.thirsty-\Fpl{}.\Aarg{} {} //
	\glft `We are tired and are thirsty.' //
\endgl

\xe

The example sentences in (\ref{ex:and}) are ordered by increasing level of
coordination: (\ref{ex:andheads}) combines two adjective-phrase (AP) heads,
\xayr{trnF}{taran}{quiet} and \xayr{stYo}{saco}{cool}, which together make up
the predicative AP that is equated to \xayr{nNaaNF}{nangāng}{a/the house}. In
(\ref{ex:andphrases}), then, two agent NPs, \xayr{ynNF}{yanang}{a/the boy} and
\xayr{lyNF}{layang}{a/the girl}, together form the subject of the verb
\xayr{AgYynF}{ajayan}{play}. Lastly, (\ref{ex:andclauses}) shows two main
clauses coordinated, that is, \xayr{naaNF pisu}{nāng pisu}{we are tired} on the
one hand, and \xayr{tpnFnNF}{tapannang}{we are thirsty} on the other.

\index{conjunctions!nay@\fw{nay}|)}

\index{conjunctions!soyang@\fw{soyang}|(}

Just as \rayr{nj}{nay} expresses \emph{con}junction, \xayr{soyNF}{soyang}{or} 
expresses \emph{dis}junction. It is equally placed between two disjuncts and 
equally works at all levels---lexical heads, phrases, and clauses. Inclusive 
and exclusive `or' are not formally distinguished in Ayeri by the disjunction 
\rayr{soyNF}{soyang} alone, so context is necessary to contrast between them. 
Alternatively, a construction akin to English `either ... or' may be used to 
make the distinction explicit (see \autoref{subsec:corrconj}).

\pex\label{ex:or}
\a\label{ex:orheads}\begingl
	\gla Pasyyang, yāng {\normalfont [\tsup{AP}[\tsup{A}} @ mino @ 
		{\normalfont ]} soyang {\normalfont [\tsup{A}} @ giday @ 
		{\normalfont ]].} //
	\glb pasy=yang yāng {} mino {} soyang {} giday {} //
	\glc wonder=\Fsg{}.\Aarg{} \TsgM{}.\Aarg{} {} happy {} or {} sad {} //
	\glft `I wonder whether he is happy or sad.' //
\endgl

\a\label{ex:orphrases}\begingl
	\gla Le no ginvāng {\normalfont [\tsup{NP}} @ karon @ {\normalfont ]} 
		soyang {\normalfont [\tsup{NP}} @ gali @ {\normalfont ]?} //
	\glb le no gin=vāng {} karon-Ø {} soyang {} gali-Ø {} //
	\glc \PatTI{} want drink=\Ssg{}.\Aarg{} {} water-\Top{} {} or {} 
		juice-\Top{} {} //
	\glft `Do you want to drink water or juice?' //
\endgl

\a\label{ex:orclauses}\begingl
	\gla {\normalfont [\tsup{S}} @ Beratu edauyi @ {\normalfont ]} soyang
		{\normalfont [\tsup{S}} @ sa-sahu rangya {\normalfont ]!} //
	\glb {} berata-u edauyi {} soyang {} sa\til{}saha-u rang-ya {} //
	\glc {} decide-\Imp{} now {} or {} return-\Imp{} home-\Loc{} {} //
	\glft `Decide now or go home!' //
\endgl

\xe

As above, (\ref{ex:or}) shows different syntactic contexts for 
\rayr{soyNF}{soyang}. In (\ref{ex:orheads}), two adjectives, 
\xayr{mino}{mino}{happy} and \xayr{gidj}{giday}{sad} are put in opposition as
phrasal heads making up a predicative AP. Then, in (\ref{ex:orphrases}), the
choice is between two NPs, \xayr{le—kronF}{le ... karon}{water} and
\xayr{le—gli}{le ... gali}{juice}, which jointly form the object of 
\xayr{ginvaaNF}{ginvāng}{you drink}. Lastly, in (\ref{ex:orclauses}), two main 
clauses are in opposition---either disjunct forms a complete sentence on its 
own.

\index{conjunctions!soyang@\fw{soyang}|)}

\subsection{Complex conjunction and disjunction}
\label{subsec:corrconj}

English has a number of correlative conjunctions, that is, conjunctions made up
of multiple parts which work together as one expression. Among these are,
notably,
\fw{as ... as},
\fw{both ... and},
\fw{either ... or},
\fw{neither ... nor},
\fw{rather ... than}, and
\fw{the ... the}.
Ayeri uses the adverb \xayr{kmo}{kamo}{equally, same} together with a 
conjunction for many of these.

\index{conjunctions!kamonay@\fw{kamo ... nay}|(}

\xayr{kmo—nj}{kamo ... nay}{equally ... and} is equivalent to 
`both ... and': the correlative construction emphasizes that two options are 
equal to each other. Syntactically, resulting sentences are equal to those
presented in (\ref{ex:and}). \xayr{sno}{sano}{both} may be used as a synonym 
to \rayr{kmo}{kamo} as well here.

\ex\label{ex:bothand}
\begingl
	\gla Ang vacay kamo piyuley nay obanley. //
	\glb ang vac=ay.Ø kamo piyu-ley nay oban-ley //
	\glc \AgtT{} like=\Fsg{}.\Top{} equally grain-\PargI{} and
		bean-\PargI{} //
	\glft `I like both grains and beans.' //
\endgl
\xe

Alternatively, it is possible to use a construction with 
\xayr{njnj}{naynay}{(and) also}:

\ex
\begingl
	\gla Ang vacay piyuley, obanley naynay. //
	\glb ang vac=ay.Ø piyu-ley oban-ley naynay //
	\glc \AgtT{} like=\Fsg{}.\Top{} grain-\PargI{} bean-\PargI{} also //
	\glft `I like grains and also beans.' //
\endgl
\xe

The example in (\ref{ex:bothand}) may be translated more literally as `I like 
grains and beans equally', with two NPs in alternation, both being objects of a 
transitive verb, \xayr{vtYF/}{vac-}{like}. With predicative adjectives, the 
verb \xayr{km/}{kama-}{(be) equal} may be used:

\ex\label{ex:bothandpred}
\begingl
	\gla Ang kamayan mabo nay giday. //
	\glb ang kama=yan.Ø mabo nay giday //
	\glc \AgtT{} be.equal=\TplM{}.\Top{} hungry and thirsty //
	\glft `They are both hungry and thirsty.' //
\endgl
\xe

\rayr{km/}{kama-} is one of Ayeri's copular verbs used to express equality 
between two properties of its subject. The literal meaning of 
(\ref{ex:bothandpred}) is thus, roughly, `They are as hungry as they are 
thirsty'. The construction slightly differs from that used to do comparison of 
NPs, however, in that the conjunction \rayr{nj}{nay} is placed between both 
predicative terms here. In order to express literal `be ... as ... as', thus, 
the conjunction is dropped:

\ex\label{ex:asas}
\begingl
	\gla Kamareng matikan helanas agonanya. //
	\glb kama=reng matikan helan-as agonan-ya //
	\glc be.equal=\TsgI{}.\Aarg{} hot oven-\Parg{} outside-\Loc{} //
	\glft `It's as hot as an oven outside.' //
\endgl
\xe

% I've never thought of the following before, but I don't see why it shouldn't 
% make sense!

\rayr{kmo—nj}{kamo ... nay} is used to express `the ... the', that is, a 
proportional or antiproportional relationship between two amounts, sizes, 
or properties; using \xayr{sno}{sano}{both} here is judged less fitting. In 
order to express a relationship of equal increase/decrease in this way, 
conjuncts are additionally marked with the comparative suffix 
\xayr{/ENF}{-eng}{more, rather} or its opposite, \xayr{/Ikoj}{-ikoy}{less}:

\ex\label{ex:thethe}
\begingl
	\gla Ang tavyan kamo nakēng nay konjāng-eng. //
	\glb ang tav=yan.Ø equal nake-eng nay kond=yāng=eng //
	\glc \AgtT{} become=\TsgM{}.\Top{} equally tall-\Comp{} and 
		eat=\TsgM{}.\Aarg{}=more //
	\glft `The taller they get, the more they eat.' //
\endgl
\xe

\index{conjunctions!kamonay@\fw{kamo ... nay}|)}

\index{conjunctions!kamosoyang@\fw{kamo ... soyang}|(}

The type of correlative conjunction which selects one of two alternatives but
not both---that is, exclusive `or' (\textsc{xor})---is expressed by the
construction \xayr{kmo—soyNF}{kamo ... soyang}{equally ... or}, as illustrated
by (\ref{ex:eitheror}).\footnote{Interestingly, it looks as though I slightly
plagiarized English here in idea, albeit unwittingly: the etymology of
\fw{either} is given as being from Old English \fw{ǽghwæðer, ǽgðer}, from
Germanic \fw{*aiwon} `always' + \fw{*gihwaþaro-z} `each of two', cf.
\citet[either, adj. (and pron.) and adv. (and conj.)]{oed}. On the other hand,
the collective wisdom of the internet's language invention community holds that
one cannot truly innovate grammatical structures; there will always be a
natural language which evolved a given construction before, and possibly with
more complications involved. This situation is referred to as
`\textsc{anadew}', an acronym for `a nat[ural~]lang[uage] already dunnit except
worse' \citep{teoh2003}.} For its negative opposite, `neither ... nor',
negation must be used, which is displayed in (\ref{ex:neithernor}).

\pex\label{ex:eitheror}
\a\label{ex:eitherorvb}\begingl
	\gla Ang miraya kamo Ajān adaley eda-konkyanya soyang da-mararya. //
	\glb ang mira-ya kamo Ajān adaley eda=konkyan-ya soynag da=mararya //
	\glc \AgtT{} do-\TsgM{} equally Ajān that-\PargI{} this=month-\Loc{} 
		or such-next //
	\glft `Ajān does it either this month or next.' //
\endgl

\a\label{ex:eitherorpred}\begingl
	\gla Kamayong mabo soyang krito mirampaluy. //
	\glb kama=yong mabo soyang krito mirampaluy //
	\glc be.equal=\TsgN{}.\Aarg{} hungry or angry otherwise //
	\glft `They are either hungry or otherwise angry.' //
\endgl

\xe

\pex~\label{ex:neithernor}
\a\label{ex:neithernorvb}\begingl
	\gla Ang tahoyye kamo {} Sipra netuas soyang kinās. //
	\glb ang taha-oy-ye kamo Ø Sipra netu-as soyang kinās //
	\glc \AgtT{} have-\Neg{}-\TsgF{} equally \Top{} Sipra brother-\Parg{} 
		or sister-\Parg{} //
	\glft `Sipra has neither a brother nor a sister.' //
\endgl

\a\label{ex:neithernorpred}\begingl
	\gla Ang kamuay layas soyang veno. //
	\glb ang kama-oy=ay.Ø lay-as soyang veno //
	\glc \AgtT{} be.equal-\Neg{}=\Fsg{}.\Aarg{} girl-\Parg{} or beautiful //
	\glft `I am neither a girl nor beautiful.' //
\endgl

\xe

\index{conjunctions!kamosoyang@\fw{kamo ... soyang}|)}

\index{conjunctions|)}
