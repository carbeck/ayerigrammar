% kate: word-wrap true;

\chapter{Grammatical categories}

While the previous chapter was about general mechanisms of marking in Ayeri, 
this chapter will dive into the various parts of speech in order to define 
their morphology with a closer look. I will begin with nouns as the main 
carriers of meaning, then deal with other parts of speech that regularly 
feature in the noun phrase---pronouns, adjectives, and adpositions. Following 
this, there will be a discussion of verbs and adverbs before moving on to 
numerals and conjunctions.

\section{Nouns}
\index{nouns|(}

Nouns in Ayeri have \emph{gender} and \emph{number} as their inherent 
grammatical properties. Besides common nouns, there are, of course, also proper 
nouns (i.e. names) and deverbal nouns. Nouns, as the heads of NPs, are also 
assigned \emph{case} by the VP, which is a third grammatical property they 
display. For an illustration of the declension paradigms, compare Figures 
\ref{fig:anideclcons}–\ref{fig:inandeclvow}.

\begin{figure}[ht]
\caption[Declension paradigm for Ayeri \xayr{bdnF}{badan}{father}]{Declension 
paradigm for Ayeri \xayr{bdnF}{badan}{father} (animate; consonantal root)}
\begin{tabu} to \linewidth {X[1] I[2] X[4] I[2] X[4]}
\tableheaderfont\toprule

	& \multicolumn2{c}{Singular}
	& \multicolumn2{c}{Plural}
	\\

\midrule
	
\Top{}
	& badan
	& `the father'
	%
	& badanye
	& `the fathers'
	\\

\midrule

\Aarg{}
	& badanang
	& `father'
	%
	& badanjang
	& `fathers'
	\\

\Parg{}
	& badanas
	& `father' (obj.)
	%
	& badanjas
	& `fathers' (obj.)
	\\

\Dat{}
	& badanyam
	& `to the father'
	%
	& badanjyam
	& `to the fathers'
	\\

\midrule

\Gen{}
	& badanena
	& `of the father'
	%
	& badanyena
	& `of the fathers'
	\\
	
\Loc{}
	& badanya
	& `at the father'
	%
	& badanjya
	& `at the fathers'
	\\

\Caus{}
	& badanisa
	& `due to the father'
	%
	& badanjisa
	& `due to the fathers'
	\\

\Ins{}
	& badaneri
	& `with the father'
	%
	& badanyeri
	& `with the fathers'
	\\

\bottomrule
\end{tabu}
\label{fig:anideclcons}
\end{figure}
~
\begin{figure}[ht]
\caption[Declension paradigm for Ayeri \xayr{mv}{māva}{mother}]{Declension 
paradigm for Ayeri \xayr{mv}{māva}{mother} (animate; vocalic root)}
\begin{tabu} to \linewidth {X[1] I[2] X[4] I[2] X[4]}
\tableheaderfont\toprule

	& \multicolumn2{c}{Singular}
	& \multicolumn2{c}{Plural}
	\\

\midrule
	
\Top{}
	& māva
	& `the mother'
	%
	& māvaye
	& `the mothers'
	\\

\midrule

\Aarg{}
	& māvāng
	& `mother'
	%
	& māvajang
	& `mothers'
	\\

\Parg{}
	& māvās
	& `mother' (obj.)
	%
	& māvajas
	& `mothers' (obj.)
	\\

\Dat{}
	& māvayam
	& `to the mother'
	%
	& māvajyam
	& `to the mothers'
	\\

\midrule

\Gen{}
	& māvana
	& `of the mother'
	%
	& māvayena
	& `of the mothers'
	\\
	
\Loc{}
	& māvaya
	& `at the mother'
	%
	& māvajya
	& `at the mothers'
	\\

\Caus{}
	& māvaisa
	& `due to the mother'
	%
	& māvajisa
	& `due to the mothers'
	\\

\Ins{}
	& māvari
	& `with the mother'
	%
	& māvayeri
	& `with the mothers'
	\\

\bottomrule
\end{tabu}
\label{fig:anideclvow}
\end{figure}

\begin{figure}[ht]
\caption[Declension paradigm for Ayeri \xayr{kirinF}{kirin}{street}]{Declension 
paradigm for Ayeri \xayr{kirinF}{kirin}{street} (inanimate; consonantal root)}
\begin{tabu} to \linewidth {X[1] I[2] X[4] I[2] X[4]}
\tableheaderfont\toprule

	& \multicolumn2{c}{Singular}
	& \multicolumn2{c}{Plural}
	\\

\midrule
	
\Top{}
	& kirin
	& `the street'
	%
	& kirinye
	& `the streets'
	\\

\midrule

\Aarg{}
	& kirinreng
	& `street'
	%
	& kirinyereng
	& `streets'
	\\

\Parg{}
	& kirinley
	& `street' (obj.)
	%
	& kirinyeley
	& `streets' (obj.)
	\\

\Dat{}
	& kirinyam
	& `to the street'
	%
	& kirinjyam
	& `to the streets'
	\\

\midrule

\Gen{}
	& kirinena
	& `of the street'
	%
	& kirinyena
	& `of the streets'
	\\
	
\Loc{}
	& kirinya
	& `at the street'
	%
	& kirinjya
	& `at the streets'
	\\

\Caus{}
	& kirinisa
	& `due to the street'
	%
	& kirinjisa
	& `due to the streets'
	\\

\Ins{}
	& kirineri
	& `with the street'
	%
	& kirinyeri
	& `with the streets'
	\\

\bottomrule
\end{tabu}
\label{fig:inandeclcons}
\end{figure}
~
\begin{figure}[ht]
\caption[Declension paradigm for Ayeri \xayr{per}{pera}{measure}]{Declension 
paradigm for Ayeri \xayr{per}{pera}{measure} (inanimate; vocalic root)}
\begin{tabu} to \linewidth {X[1] I[2] X[4] I[2] X[4]}
\tableheaderfont\toprule

	& \multicolumn2{c}{Singular}
	& \multicolumn2{c}{Plural}
	\\

\midrule
	
\Top{}
	& pera
	& `the measure'
	%
	& peraye
	& `the measures'
	\\

\midrule

\Aarg{}
	& perareng
	& `measure'
	%
	& perayereng
	& `measures'
	\\

\Parg{}
	& peraley
	& `measure' (obj.)
	%
	& perayeley
	& `measures' (obj.)
	\\

\Dat{}
	& perayam
	& `to the measure'
	%
	& perajyam
	& `to the measures'
	\\

\midrule

\Gen{}
	& perana
	& `of the measure'
	%
	& perayena
	& `of the measures'
	\\
	
\Loc{}
	& peraya
	& `at the measure'
	%
	& perajya
	& `at the measures'
	\\

\Caus{}
	& peraisa
	& `due to the measure'
	%
	& perajisa
	& `due to the measures'
	\\

\Ins{}
	& perari
	& `with the measure'
	%
	& perayeri
	& `with the measures'
	\\

\bottomrule
\end{tabu}
\label{fig:inandeclvow}
\end{figure}

\subsection{Gender}
\index{gender|(}

Grammatical gender in Ayeri consists of two tiers which are subdivided into 
four classes based on a mixture of semantic and epistemic properties:

\begin{figure}[h]
\caption{Grammatical genders in Ayeri}\centering
\begin{forest}
where n children=0{tier=word}{}
[grammatical gender
	[animate
		[masculine]
		[feminine]
		[neuter]
	]
	[inanimate]
]
\end{forest}
\label{fig:gramgend}
\end{figure}

The animate gender refers, broadly speaking, to entities that are considered 
alive or are closely associated with living entities, such as events, concepts, 
or activities executed by living things. The `masculine' and `feminine' 
subcategories are applied to humans, animals whose sex is known (for example on 
behalf of breeding them or keeping them as pets), and gods---basically anything 
that shows sexual dimorphism or is assumed to be an exponent of it as well as 
nouns referring to such entities in a functional way, for instance, 
\xayr{bdnF}{badan}{father} and \xayr{maav}{māva}{mother}. The remainder falls 
into the `neuter' category---plants, for instance, body parts, or animals whose 
sex is unknown. The `inanimate' category typically contains objects such as 
tools or materials. Furthermore, animals and plants change their category to 
inanimate as well if they serve as food. There are exceptions to either group, 
where elements appear in them for no obviously discernable reason. In order to 
illustrate, here are a few examples for each category:

\pex
	\a Animate masculine:\\
		\xayr{\larger bdnF}{badan}{father}, 
		\xayr{\larger netu}{netu}{brother}, 
		\xayr{\larger AguynF}{aguyan}{rooster}, 
		\rayr{\larger AgYaanF}{Ajān}, 
		\rayr{\larger ltunF}{Latun};
		% FIXME: bull? stallion? dog?
	
	\a Animate feminine:\\
		\xayr{\larger maav}{māva}{mother}, 
		\xayr{\larger kin}{kina}{sister}, 
		\xayr{\larger Aguvj}{aguvay}{hen}, 
		\rayr{\larger mh}{Maha}, 
		\rayr{\larger tFraanj}{Trānay};
		% FIXME: cow? mare? bitch?
	
	\a Animate neuter:\\
		\xayr{\larger AdNF}{adang}{palm tree},
		\xayr{\larger bino}{bino}{color},
		\xayr{\larger IkmF}{ikam}{deer},
		\xayr{\larger kdaanF}{kadān}{harvest},
		\xayr{\larger tYaanF}{cān}{love},
		\xayr{\larger nN}{nanga}{house}, 
		\xayr{\larger tMpu}{tampu}{luck},
		\xayr{\larger yil}{yila}{foot};
	
	\a Inanimate:\\
		\xayr{\larger AhlF}{ahal}{sand},
		\xayr{\larger hem}{hema}{egg},
		\xayr{\larger khnF}{kahan}{spear},
		\xayr{\larger meluNF}{melung}{yogurt},
		\xayr{\larger nusaanF}{nusān}{damage},
		\xayr{\larger pyutaanF}{payutān}{mathematics}.
\xe

There are also a number of doublets like French \fw{le livre} `the book' and 
\fw{la livre} `the pound', for instance, \ayr{bnnF} \fw{banan} (an.) `kindness, 
charity' or \ayr{bino} \fw{bino} (an.) `color' on the one hand, and 
\ayr{bnnF} \fw{banan} (inan.) `quality' or \ayr{bino} \emph{bino} (inan.) 
`paint' on the other. Gender is reified by case marking as well as verb 
agreement; it is not possible to read the gender of a noun from its phonological 
makeup. The following example illustrates differences in case marking and 
agreement (inherent information on grammatical features underneath the NPs):

\pex
\a\label{ex:gender1}\begingl
	\gla Ang konja badan hemaley. //
	\glb Ang kond-ya badan-Ø hema-ley //
	\glc {} {} {\tiny [\TsgM{}.\An{}.\Aarg{}]} {\tiny [\TsgI{}.\Parg{}]} //
	\glc \AgtT{}.\An{} eat-\TsgM{}.\An{} father-\Top{} egg-\PargI{} //
	\glft `Father eats an egg.' //
\endgl

\a\label{ex:gender2}\begingl
	\gla Sa tombara kahanreng burang. //
	\glb Sa tomb-ara kahan-reng burang-Ø //
	\glc {} {} {\tiny [\TsgI{}.\Aarg{}]} {\tiny [\TsgN{}.\An{}.\Parg{}]} //
	\glc \PatT{}.\An{} kill-\TsgI{} spear-\AargI{} animal-\Top{} //
	\glft `The animal, the spear kills it.' //
\endgl

\xe

In example (\ref{ex:gender1}), the noun in the agent NP, 
\xayr{bdnF}{badan}{father}, bears the features \textsc{[+\,animate, 
+\,masculine]}, which triggers the animate agent topic agreement marker 
\rayr{ANF}{ang} on the verb, since the agent NP is also topicalized. The verb 
also agrees in person and number with the agent NP by way of the person marker 
\rayr{/y}{-ya} for third person singular masculine. The object of the sentence, 
\xayr{hem}{hema}{egg}, on the other hand bears the feature 
\textsc{[–\,animate]}, so it receives the inanimate patient case marker 
\rayr{/lej}{-ley} rather than its animate counterpart \rayr{/AsF}{-as}.

In (\ref{ex:gender2}), on the other hand, we see an inanimate agent, 
\xayr{khnF}{kahan}{spear}, so the verb receives the marker \rayr{/Ar}{-ara} for 
third person singular inanimate rather than its animate neuter counterpart 
\rayr{/yo}{-yo}. That the agent of the clause is inanimate is also shown by the 
(non-topicalized) NP's case marking: \rayr{khnF}{kahan} carries the marker 
\rayr{/reNF}{-reng}, which marks it as an inanimate agent. The object of the 
sentence, \xayr{burNF}{burang}{animal}, is also the topic, hence topic agreement 
on the verb uses the marker \rayr{s}{sa} according to the NP being animate, 
rather than its inanimate counterpart \rayr{le}{le}.

\index{gender|)}

\subsection{Number}
\index{number|(}

Ayeri only distinguishes singular and plural in nouns, which receive plural 
marking; verbs, then, agree with agent NPs in number in the canonical case. 
Ordinarily, nouns in Ayeri are countable, however, there is also a group of 
uncountable nouns as well as a (small) group of nouns which are always plural. 
As above, I will list a few words from each group in the following example:

\pex
	\a\label{ex:plurals} Countable nouns:\\[0.5\baselineskip]
		\makebox[7em][l]{\xayr{\larger AgYmF}{ajam}{toy}}
			\xayr{\larger AgYmFye}{ajamye}{toys}, %
				\\[0.5\baselineskip]
		\makebox[7em][l]{\xayr{\larger devo}{devo}{head}}
			\xayr{\larger devoye}{devoye}{heads}, %
				\\[0.5\baselineskip]
		\makebox[7em][l]{\xayr{\larger InunF}{inun}{fish}}
			\xayr{\larger InunFye}{inunye}{fish} (pl.),%
				\\[0.5\baselineskip]
		\makebox[7em][l]{\xayr{\larger netu}{netu}{brother}}
			\xayr{\larger netuye}{netuye}{brothers};
	
	\a Uncountable nouns:\\
		\xayr{\larger AhlF}{ahal}{sand}, 
		\xayr{\larger bkj}{bakay}{stuff}, 
		\xayr{\larger ghaanF}{gahān}{hope}, 
		\xayr{\larger miNnF}{mingan}{ability};
	
	\a Plurale tantum nouns:\\
		\xayr{\larger burNF}{burang}{lifestock, cattle},\footnotemark~%
		\xayr{\larger gneNnF}{ganengan}{siblings}, 
		\xayr{\larger kejnmF}{keynam}{people}, 
		\xayr{\larger tNF}{tang}{ears}.
\xe

\footnotetext{Specifically in this meaning; \rayr{burNF}{burang} can also simply 
mean `animal', in which case there is a plrual form 
\xayr{burNFye}{burangye}{animals}.}

Most concrete things that exist as clearly separate entities are countable, 
also, for instance, animals and lifestock---fish, deer, sheep etc. are thus 
countable, unlike in English; pants, pliers, scissors, glasses, etc. are by 
default singular as well. Uncountable, on the other hand, are materials in 
general or abstract concepts. There is also a number of nouns which is plural 
by default, most notably entities which often occur in groups, but there is as 
well the odd word for which there seems to be no reason to be included in this 
group, for instance, \xayr{\larger bino}{bino}{paint}, and \xayr{\larger 
giMbj}{gimbay}{sorrows}. A few body parts are also plurale tantum nouns, 
especially those which occur in pairs (\xayr{niv}{niva}{eye} is a notable 
exception).

As demonstrated in (\ref{ex:plurals}), the noun plural marker is 
\rayr{/ye}{-ye}, which in native orthography also occurs in the variant 
\ayr{*Ye} or \ayr{ʲ*e}. As described above (\autoref{pluralmorph}, 
p.~\pageref{pluralmorph}), the plural marker may also be reduced to [dʒ] 
\orth{-j} before case suffixes that begin with a vowel other than /e/ or /j/, 
like \rayr{/ANF}{-ang} (\Aarg{}) or \rayr{/ymF}{-jam} (\Dat{}):

\pex
	\a \rayr{\larger AgYmFreNF}{ajamreng} (toy-\AargI{})
		+ \rayr{\larger /ye}{-ye} \\[0.5\baselineskip]
		→ \rayr{\larger AgYmFyereNF}{ajamyereng} (toy-\Pl{}-\AargI{}),
	\a \rayr{\larger AgYmFymF}{ajamyam} (toy-\Dat{})
		+ \rayr{\larger /ye}{-ye} \\[0.5\baselineskip]
		→ \rayr{\larger AgYmFyeymF}{ajamjyam} (toy-\Pl{}-\Dat{}).
\xe

For plurale tantum nouns, to express a singular entity, it is always possible to 
use a genitive phrase like \xayr{—/En menF}{…-ena men}{one of …} (…-\Gen{} 
one), for instance:

\pex
\a\begingl
	\gla Nupayon tangang nā. //
	\glb Nupa-yon tang-ang nā //
	\glc hurt-\TplN{} ears-\Aarg{} \Fsg{}.\Gen{} //
	\glft `My ears hurt.' //
\endgl

\a\begingl
	\gla Na nupareng tang nā men. //
	\glb Na nupa-reng tang-Ø nā men //
	\glc \GenT{} hurt-\TsgI{}.\Aarg{} ears-\Top{} \Fsg{}.\Gen{} one //
	\glft `One of my ears, it hurts.' //
\endgl
\xe

\index{number|)}

\index{nouns|)}