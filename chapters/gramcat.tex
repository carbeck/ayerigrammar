% kate: word-wrap true;

\chapter{Grammatical categories}

While the previous chapter was about general mechanisms of marking in Ayeri, 
this chapter will dive into the various parts of speech in order to define 
their morphology with a closer look. I will begin with nouns as the main 
carriers of meaning, then deal with other parts of speech that regularly 
feature in the noun phrase---pronouns, adjectives, and adpositions. Following 
this, there will be a discussion of verbs and adverbs before moving on to 
conjunctions.

\section{Nouns}
\index{nouns|(}

Nouns in Ayeri have \emph{gender} and \emph{number} as their inherent 
grammatical properties. Besides common nouns, there are, of course, also proper 
nouns (i.e. names) and deverbal nouns.

\subsection{Gender}
\index{gender|(}

Grammatical gender in Ayeri consists of two tiers which are subdivided into 
four classes based on a mixture of semantic and epistemic properties:

\begin{figure}[h]
\caption{Grammatical genders in Ayeri}\centering
\begin{forest}
where n children=0{tier=word}{}
[grammatical gender
	[animate
		[masculine]
		[feminine]
		[neuter]
	]
	[inanimate]
]
\end{forest}
\label{fig:gramgend}
\end{figure}

The animate gender refers, broadly speaking, to entities that are considered 
alive or are closely associated with living entities, such as events, concepts, 
or activities executed by living things. The `masculine' and `feminine' 
subcategories are applied to humans, animals whose sex is known (for example on 
behalf of breeding them or keeping them as pets), and gods---basically anything 
that shows sexual dimorphism or is assumed to be an exponent of it as well as 
nouns referring to such entities in a functional way, for instance, 
\xayr{bdnF}{badan}{father} and \xayr{maav}{māva}{mother}. The remainder falls 
into the `neuter' category---plants, for instance, body parts, or animals whose 
sex is unknown. The `inanimate' category typically contains objects such as 
tools or materials. Furthermore, animals and plants change their category to 
inanimate as well if they serve as food. There are exceptions to either group, 
where elements appear in them for no obviously discernable reason. In order to 
illustrate, here are a few examples for each category:

\pex
	\a Animate masculine:\\
		\xayr{\larger bdnF}{badan}{father}, 
		\xayr{\larger netu}{netu}{brother}, 
		\xayr{\larger AguynF}{aguyan}{rooster}, 
		\rayr{\larger AgYaanF}{Ajān}, 
		\rayr{\larger ltunF}{Latun};
		% FIXME: bull? stallion? dog?
	
	\a Animate feminine:\\
		\xayr{\larger maav}{māva}{mother}, 
		\xayr{\larger kin}{kina}{sister}, 
		\xayr{\larger Aguvj}{aguvay}{hen}, 
		\rayr{\larger mh}{Maha}, 
		\rayr{\larger tFraanj}{Trānay};
		% FIXME: cow? mare? bitch?
	
	\a Animate neuter:\\
		\xayr{\larger AdNF}{adang}{palm tree},
		\xayr{\larger bino}{bino}{color},
		\xayr{\larger IkmF}{ikam}{deer},
		\xayr{\larger kdaanF}{kadān}{harvest},
		\xayr{\larger tYaanF}{cān}{love},
		\xayr{\larger nN}{nanga}{house}, 
		\xayr{\larger tMpu}{tampu}{luck},
		\xayr{\larger yil}{yila}{foot};
	
	\a Inanimate:\\
		\xayr{\larger AhlF}{ahal}{sand},
		\xayr{\larger hem}{hema}{egg},
		\xayr{\larger khnF}{kahan}{spear},
		\xayr{\larger meluNF}{melung}{yogurt},
		\xayr{\larger nusaanF}{nusān}{damage},
		\xayr{\larger pyutaanF}{payutān}{mathematics}.
\xe

There are also a number of doublets like French \fw{le livre} `the book' and 
\fw{la livre} `the pound', for instance, \ayr{bnnF} \fw{banan} (an.) `kindness, 
charity' or \ayr{bino} \fw{bino} (an.) `color' on the one hand, and 
\ayr{bnnF} \fw{banan} (inan.) `quality' or \ayr{bino} \emph{bino} (inan.) 
`paint' on the other. Gender is reified by case marking as well as verb 
agreement; it is not possible to read the gender of a noun from its phonological 
makeup. The following example illustrates differences in case marking and 
agreement:

\pex\label{ex:gender1}
\a\label{ex:gender1a}\begingl
	\gla Ang konja badan hemaley. //
	\glb Ang kond-ya badan-Ø hema-ley //
	\glc \AgtT{} eat-\TsgM{} father-\Top{} egg-\PargI{} //
	\glft `Father eats an egg.' //
\endgl

\a\label{ex:gender1b}\begingl
	\gla *Eng konje badan hemās. //
	\glb *Eng kond-ye badan-Ø hema-as //
	\glc *\AgtTI{} eat-\TsgF{} father-\Top{} egg-\Parg{} //
\endgl
\xe

\pex~\label{ex:gender2}
\a\label{ex:gender2a}\begingl
	\gla Sa tombara kahanreng burang. //
	\glb Sa tomb-ara kahan-reng burang-Ø //
	\glc \PatT{} kill-\TsgI{} spear-\AargI{} animal-\Parg{} //
	\glft `The animal, the spear kills it.' //
\endgl

\a\label{ex:gender2b}\begingl
	\gla *Le tombyo kahanang burang. //
	\glb *Le tomb-yo kahan-ang burang-Ø //
	\glc *\PatTI{} kill-\TsgN{} spear-\AargI{} animal-\Top{} //
\endgl
\xe

In both example groups, (\ref{ex:gender1}) and (\ref{ex:gender2}), we first see 
a grammatical sentence with its respective case marking on nouns and verb 
agreement and then the same sentence with everything reversed to the opposite 
tier. In (\ref{ex:gender1a}), the noun in the agent NP 
\xayr{bdnF}{badan}{father} bears the features 
\textsc{[+\,animate +\,masculine]}, which triggers the animate agent topic 
agreement marker \rayr{ANF}{ang} on the verb, since the agent NP is also 
topicalized, and the verb agrees with the person marker \rayr{/y}{-ya} 
for third person singular masculine. The object of the sentence, 
\xayr{hem}{hema}{egg}, on the other hand bears the feature 
\textsc{[–\,animate]}, so it receives the inanimate patient case marker 
\rayr{/lej}{-ley} rather than its animate counterpart \rayr{/AsF}{-as}. In 
(\ref{ex:gender2a}), on the other hand, we have a \textsc{[–\,animate]} agent, 
\xayr{khnF}{kahan}{spear}, so the verb agrees with the marker \rayr{/ar}{-ara} 
for third person singular inanimate rather than its animate 
neuter counterpart \rayr{/yo}{-yo}. That the agent of the clause is inanimate 
is also shown in the NP's case marking: \rayr{khnF}{kahan} carries the marker 
\rayr{/reNF}{-reng} which marks it as an inanimate agent. The object of the 
sentence, \xayr{burNF}{burang}{animal}, is also the topic, hence the verb 
agrees with the marker \rayr{s}{sa} for the feature \textsc{[+\,animate]} 
rather than its inanimate counterpart \rayr{le}{le}.

\index{gender|)}

\subsection{Number}
\index{number|(}

...

\index{number|)}

\index{nouns|)}