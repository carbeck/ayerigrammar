% kate: word-wrap true;

\chapter{Grammatical categories}

While the previous chapter was about general mechanisms of marking in Ayeri, 
this chapter will dive into the various parts of speech in order to define 
their morphology with a closer look. I will begin with nouns as the main 
carriers of meaning, then deal with other parts of speech that regularly 
feature in or in combination with the noun phrase---pronouns, adjectives, and 
adpositions. Following this, there will be a discussion of verbs and adverbs 
before moving on to numerals and conjunctions.

\section{Nouns}
\label{sec:nouns}
\index{nouns|(}

Nouns in Ayeri have \emph{gender} and \emph{number} as their inherent 
grammatical properties. Besides common nouns, there are, of course, also proper 
nouns (that is, names) and nominalizations. Nouns, as the heads of NPs, are 
also assigned \emph{case} by the verb, which is a third grammatical property 
they display. For an illustration of the declension paradigms, compare Figures 
\ref{fig:anideclcons}–\ref{fig:inandeclvow}.

\begin{figure}[ht]
\caption[Declension paradigm for \xayr{bdnF}{badan}{father}]{Declension 
paradigm for \xayr{bdnF}{badan}{father} (animate; consonantal root)}
\begin{tabu} to \linewidth {X[1] I[2] X[4] I[2] X[4]}
\tableheaderfont\toprule

	& \multicolumn2{c}{Singular}
	& \multicolumn2{c}{Plural}
	\\

\midrule
	
\Top{}
	& badan
	& `the father'
	%
	& badanye
	& `the fathers'
	\\

\midrule

\Aarg{}
	& badanang
	& `father'
	%
	& badanjang
	& `fathers'
	\\

\Parg{}
	& badanas
	& `father' (obj.)
	%
	& badanjas
	& `fathers' (obj.)
	\\

\Dat{}
	& badanyam
	& `to the father'
	%
	& badanjyam
	& `to the fathers'
	\\

\midrule

\Gen{}
	& badanena
	& `of the father'
	%
	& badanyena
	& `of the fathers'
	\\
	
\Loc{}
	& badanya
	& `at the father'
	%
	& badanjya
	& `at the fathers'
	\\

\Caus{}
	& badanisa
	& `due to the father'
	%
	& badanjisa
	& `due to the fathers'
	\\

\Ins{}
	& badaneri
	& `with the father'
	%
	& badanyeri
	& `with the fathers'
	\\

\bottomrule
\end{tabu}
\label{fig:anideclcons}
\end{figure}
~
\begin{figure}[ht]
\caption[Declension paradigm for \xayr{maav}{māva}{mother}]{Declension 
paradigm for \xayr{maav}{māva}{mother} (animate; vocalic root)}
\begin{tabu} to \linewidth {X[1] I[2] X[4] I[2] X[4]}
\tableheaderfont\toprule

	& \multicolumn2{c}{Singular}
	& \multicolumn2{c}{Plural}
	\\

\midrule
	
\Top{}
	& māva
	& `the mother'
	%
	& māvaye
	& `the mothers'
	\\

\midrule

\Aarg{}
	& māvāng
	& `mother'
	%
	& māvajang
	& `mothers'
	\\

\Parg{}
	& māvās
	& `mother' (obj.)
	%
	& māvajas
	& `mothers' (obj.)
	\\

\Dat{}
	& māvayam
	& `to the mother'
	%
	& māvajyam
	& `to the mothers'
	\\

\midrule

\Gen{}
	& māvana
	& `of the mother'
	%
	& māvayena
	& `of the mothers'
	\\
	
\Loc{}
	& māvaya
	& `at the mother'
	%
	& māvajya
	& `at the mothers'
	\\

\Caus{}
	& māvaisa
	& `due to the mother'
	%
	& māvajisa
	& `due to the mothers'
	\\

\Ins{}
	& māvari
	& `with the mother'
	%
	& māvayeri
	& `with the mothers'
	\\

\bottomrule
\end{tabu}
\label{fig:anideclvow}
\end{figure}

\begin{figure}[ht]
\caption[Declension paradigm for \xayr{kirinF}{kirin}{street}]{Declension 
paradigm for \xayr{kirinF}{kirin}{street} (inanimate; consonantal root)}
\begin{tabu} to \linewidth {X[1] I[2] X[4] I[2] X[4]}
\tableheaderfont\toprule

	& \multicolumn2{c}{Singular}
	& \multicolumn2{c}{Plural}
	\\

\midrule
	
\Top{}
	& kirin
	& `the street'
	%
	& kirinye
	& `the streets'
	\\

\midrule

\Aarg{}
	& kirinreng
	& `street'
	%
	& kirinyereng
	& `streets'
	\\

\Parg{}
	& kirinley
	& `street' (obj.)
	%
	& kirinyeley
	& `streets' (obj.)
	\\

\Dat{}
	& kirinyam
	& `to the street'
	%
	& kirinjyam
	& `to the streets'
	\\

\midrule

\Gen{}
	& kirinena
	& `of the street'
	%
	& kirinyena
	& `of the streets'
	\\
	
\Loc{}
	& kirinya
	& `at the street'
	%
	& kirinjya
	& `at the streets'
	\\

\Caus{}
	& kirinisa
	& `due to the street'
	%
	& kirinjisa
	& `due to the streets'
	\\

\Ins{}
	& kirineri
	& `with the street'
	%
	& kirinyeri
	& `with the streets'
	\\

\bottomrule
\end{tabu}
\label{fig:inandeclcons}
\end{figure}
~
\begin{figure}[ht]
\caption[Declension paradigm for \xayr{per}{pera}{measure}]{Declension 
paradigm for \xayr{per}{pera}{measure} (inanimate; vocalic root)}
\begin{tabu} to \linewidth {X[1] I[2] X[4] I[2] X[4]}
\tableheaderfont\toprule

	& \multicolumn2{c}{Singular}
	& \multicolumn2{c}{Plural}
	\\

\midrule
	
\Top{}
	& pera
	& `the measure'
	%
	& peraye
	& `the measures'
	\\

\midrule

\Aarg{}
	& perareng
	& `measure'
	%
	& perayereng
	& `measures'
	\\

\Parg{}
	& peraley
	& `measure' (obj.)
	%
	& perayeley
	& `measures' (obj.)
	\\

\Dat{}
	& perayam
	& `to the measure'
	%
	& perajyam
	& `to the measures'
	\\

\midrule

\Gen{}
	& perana
	& `of the measure'
	%
	& perayena
	& `of the measures'
	\\
	
\Loc{}
	& peraya
	& `at the measure'
	%
	& perajya
	& `at the measures'
	\\

\Caus{}
	& peraisa
	& `due to the measure'
	%
	& perajisa
	& `due to the measures'
	\\

\Ins{}
	& perari
	& `with the measure'
	%
	& perayeri
	& `with the measures'
	\\

\bottomrule
\end{tabu}
\label{fig:inandeclvow}
\end{figure}

\subsection{Gender}
\label{subsec:gender}
\index{gender|(}

\begin{figure}[hb]
\caption{Grammatical genders in Ayeri}\centering
\begin{forest}
where n children=0{tier=word}{}
[grammatical gender
	[animate
		[masculine]
		[feminine]
		[neuter]
	]
	[inanimate]
]
\end{forest}
\label{fig:gramgend}
\end{figure}

Grammatical gender in Ayeri consists of two tiers which are subdivided into 
four classes based on a mixture of semantic and ontological properties, see 
\autoref{fig:gramgend}. The animate gender refers, broadly speaking, to 
entities that are considered alive or are closely associated with living things, 
such as events, concepts, or activities executed or connected to them. The 
`masculine' and `feminine' subcategories are applied to humans, animals whose 
sex is known (for example on behalf of breeding them or keeping them as pets), 
and gods---basically anything that shows sexual dimorphism or is assumed to be 
an exponent of it as well as nouns referring to such entities in a functional 
way, for instance, \xayr{bdnF}{badan}{father} and \xayr{maav}{māva}{mother}. The 
remainder falls into the `neuter' category---plants, for instance, body parts, 
or animals whose sex is unknown. The `inanimate' category typically contains 
materials and things, such as tools. Furthermore, animals and plants change 
their category to inanimate as well if they serve as food. There are exceptions 
to either group, where elements appear in them for no obviously discernable 
reason. In order to illustrate, here are a few examples for each category:

\pex
	\a Animate masculine:\\
		\xayr{\larger bdnF}{badan}{father}, 
		\xayr{\larger netu}{netu}{brother}, 
		\xayr{\larger AguynF}{aguyan}{rooster}, 
		\rayr{\larger AgYaanF}{Ajān}, 
		\rayr{\larger ltunF}{Latun};
		% FIXME: bull? stallion? dog?
	
	\a Animate feminine:\\
		\xayr{\larger maav}{māva}{mother}, 
		\xayr{\larger kin}{kina}{sister}, 
		\xayr{\larger Aguvj}{aguvay}{hen}, 
		\rayr{\larger mh}{Maha}, 
		\rayr{\larger tFraanj}{Trānay};
		% FIXME: cow? mare? bitch?
	
	\a Animate neuter:\\
		\xayr{\larger AdNF}{adang}{palm tree},
		\xayr{\larger bino}{bino}{color},
		\xayr{\larger IkmF}{ikam}{deer},
		\xayr{\larger kdaanF}{kadān}{harvest},
		\xayr{\larger tYaanF}{cān}{love},
		\xayr{\larger nN}{nanga}{house}, 
		\xayr{\larger tMpu}{tampu}{luck},
		\xayr{\larger yil}{yila}{foot};
	
	\a Inanimate:\\
		\xayr{\larger AhlF}{ahal}{sand},
		\xayr{\larger hem}{hema}{egg},
		\xayr{\larger khnF}{kahan}{spear},
		\xayr{\larger meluNF}{melung}{yogurt},
		\xayr{\larger nusaanF}{nusān}{damage},
		\xayr{\larger pyutaanF}{payutān}{mathematics}.
\xe

There are also a number of doublets like French \fw{le livre} `the book' and 
\fw{la livre} `the pound', for instance, \ayr{bnnF} \fw{banan} (an.) `kindness, 
charity' or \ayr{bino} \fw{bino} (an.) `color' on the one hand, and 
\ayr{bnnF} \fw{banan} (inan.) `quality' or \ayr{bino} \emph{bino} (inan.) 
`paint' on the other. Gender is reified by case marking as well as verb 
agreement; it is not possible to read the gender of a noun from its 
phonological makeup. The following example illustrates differences in case 
marking and agreement (inherent information on grammatical features underneath 
the NPs):

\pex
\a\label{ex:gender1}\begingl
	\gla Ang konja badan hemaley. //
	\glb Ang kond-ya badan-Ø hema-ley //
	\glc {} {} {\tiny [\TsgM{}.\An{}}] {\tiny [\TsgI{}]} //
	\glc \AgtT{}.\An{} eat-\TsgM{}.\An{} father-\Top{} egg-\PargI{} //
	\glft `Father eats an egg.' //
\endgl

\a\label{ex:gender2}\begingl
	\gla Sa tombara kahanreng burang. //
	\glb Sa tomb-ara kahan-reng burang-Ø //
	\glc {} {} {\tiny [\TsgI{}]} {\tiny [\TsgN{}.\An{}]} //
	\glc \PatT{}.\An{} kill-\TsgI{} spear-\AargI{} animal-\Top{} //
	\glft `The animal, the spear kills it.' //
\endgl

\xe

In example (\ref{ex:gender1}), the noun in the agent NP, 
\xayr{bdnF}{badan}{father}, bears the features \textsc{[+\,animate, 
+\,masculine]}, which triggers the animate agent topic agreement marker 
\rayr{ANF}{ang} on the verb, since the agent NP is also topicalized. The verb 
also agrees in person and number with the agent NP by way of the person marker 
\rayr{/y}{-ya} for third person singular masculine. The object of the sentence, 
\xayr{hem}{hema}{egg}, on the other hand bears the feature 
\textsc{[–\,animate]}, so it receives the inanimate patient case marker 
\rayr{/lej}{-ley} rather than its animate counterpart \rayr{/AsF}{-as}.

In (\ref{ex:gender2}), on the other hand, we see an inanimate agent, 
\xayr{khnF}{kahan}{spear}, so the verb receives the marker \rayr{/Ar}{-ara} for 
third person singular inanimate rather than its animate neuter counterpart 
\rayr{/yo}{-yo}. The (non-topicalized) NP's case marking shows that the agent 
of the clause is inanimate: \rayr{khnF}{kahan} carries the marker 
\rayr{/reNF}{-reng}, which marks it as an inanimate agent. The object of the 
sentence, \xayr{burNF}{burang}{animal}, is also the topic, hence topic 
agreement on the verb uses the marker \rayr{s}{sa} according to the NP being 
animate, rather than its inanimate counterpart \rayr{le}{le}.

\index{gender|)}

\subsection{Number}
\index{number|(}

Ayeri only distinguishes singular and plural in nouns, which receive plural 
marking; verbs, then, agree with agent NPs in number in the canonical case. 
Ordinarily, nouns in Ayeri are countable, however, there is also a group of 
uncountable nouns as well as a (small) group of nouns which are always plural. 
As above, I will list a few words from each group for illustration:

\pex
	\a Countable nouns:\label{ex:plurals}\\[0.5\baselineskip]
		\makebox[6.5em][l]{\xayr{\larger AgYmF}{ajam}{toy}}
			\makebox[2em][c]{---}
			\xayr{\larger AgYmFye}{ajamye}{toys}, %
				\\[0.5\baselineskip]
		\makebox[6.5em][l]{\xayr{\larger devo}{devo}{head}}
			\makebox[2em][c]{---}
			\xayr{\larger devoye}{devoye}{heads}, %
				\\[0.5\baselineskip]
		\makebox[6.5em][l]{\xayr{\larger InunF}{inun}{fish}}
			\makebox[2em][c]{---}
			\xayr{\larger InunFye}{inunye}{fish} (pl.),%
				\\[0.5\baselineskip]
		\makebox[6.5em][l]{\xayr{\larger netu}{netu}{brother}}
			\makebox[2em][c]{---}
			\xayr{\larger netuye}{netuye}{brothers};
	
	\a Uncountable nouns:\\
		\xayr{\larger AhlF}{ahal}{sand}, 
		\xayr{\larger bkj}{bakay}{stuff}, 
		\xayr{\larger ghaanF}{gahān}{hope}, 
		\xayr{\larger miNnF}{mingan}{ability};
	
	\a Plurale tantum nouns:\\
		\xayr{\larger burNF}{burang}{lifestock, 
			cattle},\footnotemark~
		\xayr{\larger gneNnF}{ganengan}{siblings}, 
		\xayr{\larger kejnmF}{keynam}{people}, 
		\xayr{\larger tNF}{tang}{ears}.
\xe

\footnotetext{Specifically in this meaning; \rayr{burNF}{burang} can also simply 
mean `animal', in which case there is a plural form 
\xayr{burNFye}{burangye}{animals}.}

Most concrete things that exist as discrete entities are countable, also, for 
instance, animals and lifestock---fish, deer, sheep etc. are thus countable, 
unlike in English; pants, pliers, scissors, glasses, etc. are by default 
singular as well. Uncountable, on the other hand, are materials in general or 
abstract concepts. There are also a number of nouns which are plural by 
default, 
most notably entities which often occur in groups, but there is as well the odd 
word for which there seems to be no reason to be included in this group, for 
instance, \xayr{bino}{bino}{paint}, and \xayr{giMbj}{gimbay}{sorrows}. A few 
body parts are also plurale tantum nouns, especially those which occur in pairs 
(\xayr{niv}{niva}{eye} is a notable exception).

\index{morphophonology!of the plural marker}\index{plural}
As demonstrated in (\ref{ex:plurals}), the noun plural marker is 
\rayr{/ye}{-ye}, which in native orthography also occurs in the variant 
\ayr{*Ye} or \ayr{ʲ*e}. As described above (\autoref{pluralmorph}, 
p.~\pageref{pluralmorph}), the plural marker may also be reduced to [dʒ] 
\orth{-j} before case suffixes that begin with /j/ or with a vowel other than 
/e/, like \rayr{/ANF}{-ang} (\Aarg{}) or \rayr{/ymF}{-yam} (\Dat{}):

\pex
	\a \rayr{\larger dirnFANF}{diranang} (uncle-\Aarg{})
		+ \rayr{\larger /ye}{-ye} (\Pl{}) %\\[0.5\baselineskip]
		→ \rayr{\larger dirnFye\_aNF}{diranjang} (uncle-\Pl{}-\Aarg{}),
	\a \rayr{\larger dirnen}{diranena} (uncle-\Gen{})
		+ \rayr{\larger /ye}{-ye} (\Pl{}) %\\[0.5\baselineskip]
		→ \rayr{\larger dirnFyen}{diranyena} (uncle-\Pl{}-\Gen{}),
	\a \rayr{\larger dirnFymF}{diranyam} (uncle-\Dat{})
		+ \rayr{\larger /ye}{-ye} (\Pl{}) %\\[0.5\baselineskip]
		→ \rayr{\larger dirnFyeymF}{diranjyam} (uncle-\Pl{}-\Dat{}).
\xe

For pluralia tantum, to express a singular entity, it is always possible to 
use a genitive phrase like \xayr{—/En menF}{…-ena men}{one of …} (…-\Gen{} 
one), for instance:

\pex
\a\begingl
	\gla Nupayon tangang nā. //
	\glb Nupa-yon tang-ang nā //
	\glc hurt-\TplN{} ears-\Aarg{} \Fsg{}.\Gen{} //
	\glft `My ears hurt.' //
\endgl

\a\label{ex:gensubj}\begingl
	\gla Na nupareng tang nā men. //
	\glb Na nupa=reng tang-Ø nā men //
	\glc \GenT{} hurt=\TsgI{}.\Aarg{} ears-\Top{} \Fsg{}.\Gen{} one //
	\glft `Of my ears, it hurts one.' //
\endgl
\xe

Number in nouns can also be manipulated by quantifiers which attach to declined 
nouns as suffixes. In this case, when plurality is indicated by the 
quantifier, the noun is not additionally marked for number; the verb, however, 
keeps agreeing in number:

\pex
\a\begingl
	\gla Ajayon ganjang kivo. //
	\glb Aja-yon gan-ye-ang kivo //
	\glc play-\TsgN{} child-\Pl{}-\Aarg{} small //
	\glft `The small children are playing.' //
\endgl
	
\a\begingl
	\gla Ajayon ganang-ikan kivo. //
	\glb Aja-yon gan-ang=ikan kivo. //
	\glc play-\TsgN{} child-\Aarg{}=many small //
	\glft `Many small children are playing.' //
\endgl

\xe

Likewise, when nouns are modified by numerals,\index{numerals} plurality is not 
normally marked again on the noun. In example (\ref{ex:plurnorm}), we see a 
plural noun, \xayr{nN}{nanga}{house}, and in (\ref{ex:plurnum}) the same phrase 
is repeated again with plurality implied by the use of a numeral, 
\xayr{smF}{sam}{two}; the plural noun itself appears unmarked in its singular 
form in this case.

\pex
\a\label{ex:plurnorm}\begingl
	\gla Ang no vehya sitang-yām nangajas veno nay hiro. //
	\glb Ang no veh=ya.Ø sitang=yām nanga-ye-as veno nay hiro //
	\glc \AgtT{} want build-\TsgM.\Top{} self=\TsgM{}.\Dat{} 
		house-\Pl{}-\Parg{} pretty and new //
	\glft `He wants to build himself pretty new houses.' //
\endgl

\a\label{ex:plurnum}\begingl
	\gla Ang no vehya sitang-yām nangās sam veno nay hiro. //
	\glb Ang no veh=ya.Ø sitang=yām nanga-as sam veno nay hiro //
	\glc \AgtT{} want build-\TsgM.\Top{} self=\TsgM{}.\Dat{} house-\Parg{} 
		two pretty and new //
	\glft `He wants to build himself two pretty new houses.' //
\endgl

\xe

An exception to this is the use of words for the numeral powers, 
like \xayr{lnF}{lan}{dozen}, \xayr{menNF}{menang}{gross}, 
\xayr{smNF}{samang}{myriad}, etc. in an unspecified way like `dozens 
of people'. In this case, to convey that the numeral is not to be understood as 
a precise value, the modified noun will appear in the plural---even if 
it is a plurale tantum like \xayr{kejnmF}{keynam}{people}:

\ex\begingl
	\gla Bengyon keynamjang menang. //
	\glb Beng-yon keynam-ye-ang menang //
	\glc attend-\TsgN{} people-\Pl{}-\Aarg{} gross //
	\glft `Hundreds of people attended.' //
\endgl\xe

% This is a new rule; earlier names were treated as countable but still carried 
% special case marking. I found this slightly weird, however so, let us simply 
% assert this new rule, which should make things more consistent. The odd 
% case of a pluralized name could still be explained as individual variation, 
% though I can't think of an example where this was ever an issue.
%
As we have seen in various examples above, proper nouns in Ayeri do not 
receive inflection for case by suffixes as common nouns do, and for the 
purpose of number they are treated as uncountable in Ayeri---they resist 
inflection by suffixation, marking their special status.\footnote{Many common 
names in Ayeri are derived from regular words in the language, so the language 
needs to have a way to distinguish between regular use and use as a name. For 
instance, the name \rayr{ynF}{Yan} also means `boy, son' as a common noun.} 
However, they can still be modified by quantifiers and quantifying suffixes; 
verb agreement as well can be used to indicate plurality:

\pex
\a\begingl
	\gla Sahayan cabo ekeng ang Yan. //
	\glb Saha-yan cabo ekeng ang Yan //
	\glc come-\TplM{} late too \Aarg{} Yan //
	\glft `The Yans are coming too late.' //
\endgl

\a\begingl
	\gla Ang apatang sa Yan-ikan. //
	\glb Ang apa=teng sa Yan=ikan //
	\glc \AgtT{} laugh=\TplF{}.\Aarg{} \Parg{} Yan=all //
	\glft `They laughed at (all) the Yans.' //
\endgl

\xe

\index{number|)}

\subsection{Case}
\label{subsec:case}
\index{cases|(}

As demonstrated in the declension tables at the beginning of this section 
(Figures \ref{fig:anideclcons}–\ref{fig:inandeclvow}), Ayeri's NPs are marked 
for case, which is governed by the verb. Since Ayeri uses a split alignment 
system with some additional complications, it is not very straightforward, in 
my opinion, to use the classical labels of nominative (S/A) and accusative (O), 
or of absolutive (S/P) and ergative (O) for the first two core roles. Hence, I 
will be using the terms `agent' and `patient', which I hope brings about some 
more clarity, especially when discussing the mentioned complications later on.


\subsubsection{Agent}
\index{cases!agent|(}

What I call `agent' here is, to quote \citet{fillmore1968}, 
\textcquote[46]{fillmore1968}{the case of the typically animate perceived 
instigator of the action identified by the verb}. \citeauthor{fillmore1968} 
himself qualifies this definition, however, in that the \textcquote[46, 
footnote 31]{fillmore1968}{escape qualification `typically' expresses my 
awareness that contexts which I will say require agents are sometimes occupied 
by `inanimate' nouns like robot or `human institution' nouns like nation}. 
\citet{payne1997} summarizes on prototypical agents with regards to 
their topicality that a \textcquote[151]{payne1997}{less technical way of 
expressing this fact is to say that people identify with and like to talk about 
things that act, move, control events, and have power}.

Agents in Ayeri frequently embody the properties quoted by both 
\citeauthor{fillmore1968} and \citeauthor{payne1997} in this regard, including 
\citeauthor{fillmore1968}'s caveat. However, importantly, `agent' in Ayeri is a 
macrorole that may be applied to, for instance, instruments, experiencers, and 
less typical actors as well, specifically, in absence of more prototypical 
candidates for agenthood in a sentence. It thus comes very close to a 
nominative, except that it does not need to be locus of the sentence's 
topic\index{topic}---although agents very typically are topics, as 
\citet[151]{payne1997} goes on to note.\footnote{This is the main reason I 
spoke of `complications' above: Ayeri's notion of `subject' is somewhat 
problematic due to topicalization, which is why I try to avoid complicating 
terminology by using `nominative' for agent topics and `ergative' for agent 
non-topics, and `accusative' for patient non-topics and `absolutive' for patient 
topics, respectively.} Thus, the first NP after the verb in all of the following 
examples is treated as an agent; the agent is marked by the suffix 
\rayr{/ANF}{-ang} for animate referents and the suffix \rayr{/reNF}{-reng} for 
inanimate referents; names and verbal topic agreement are marked by 
\rayr{ANF}{ang} and \rayr{ENF}{eng}, respectively:

\pex
\a\begingl
	\gla \textbf{Ang} tinkaya \textbf{{}} \textbf{Yan} kunangley. //
	\glb \textbf{Ang} tinka-ya \textbf{Ø} \textbf{Yan} kunang-ley //
	\glc \textbf{\AgtT{}} open-\TsgM{} \textbf{\Top{}} \textbf{Yan} 
		door-\PargI{} //
	\glft `Yan opens the door.' //
\endgl

\a\begingl
	\gla Le tinkaya \textbf{ayonang} kunang. //
	\glb Le tinka-ya \textbf{ayon-ang} kunang-Ø //
	\glc \PatT{} open-\TsgM{} \textbf{man-\Aarg{}} door-\Top{} //
	\glft `The door is opened by a/the man',\\
		or: `The door, a/the man opens it.' //
\endgl

\a\begingl
	\gla \textbf{Eng} tinkāra \textbf{tinkay} kunangley. //
	\glb \textbf{Eng} tinka-ara \textbf{tinkay-Ø} kunang-ley //
	\glc \textbf{\AgtTI{}} open-\TsgI{} \textbf{key-\Top{}} door-\PargI{} //
	\glft `The key opens the door.' //
\endgl

\a\begingl
	\gla Tinkāra \textbf{kunangreng}. //
	\glb Tinka-ara \textbf{kunang-reng} //
	\glc open-\TsgI{} \textbf{door-\AargI{}} //
	\glft `The door opens.' //
\endgl

\a\begingl
	\gla Sā tinkaya \textbf{ang} \textbf{Yan} kunangley yan. //
	\glb Sā tinka-ya \textbf{ang} \textbf{Yan} kunang-ley yan.Ø //
	\glc \CauT{} open-\TsgM{} \textbf{\Aarg{}} \textbf{Yan} door-\PargI{} 
		\TsgM{}.\Top{} //
	\glft `They make Yan open a/the door',\\
		or: `Because of them, Yan opens the door.' //
\endgl

\xe

In predicative constructions, the constituent which a quality is assigned to or 
about which a judgement is made is also assigned the agent case:

\pex
\a\begingl
	\gla Tado \textbf{tinkayreng}. //
	\glb Tado \textbf{tinkay-reng} //
	\glc old \textbf{key-\AargI{}} //
	\glft `The key is old.' //
\endgl

\a\begingl
	\gla \textbf{Ang} \textbf{Yan} nimpayās ban. //
	\glb \textbf{Ang} \textbf{Yan} nimpaya-as ban //
	\glc \textbf{\Aarg{}} \textbf{Yan} runner-\Parg{} good //
	\glft `Yan is a good runner.' //
\endgl

\xe

\index{cases!agent|)}

With regards to constituents' roles in ditransitive verb frames, donors are 
represented by agents in Ayeri as well, since they are the origin of whatever 
is conceptually passed on to the recipient party:

\ex\begingl
	\gla Le ilya \textbf{ang} \textbf{Yan} tinkay yam Cānlay. //
	\glb Le il-ya \textbf{ang} \textbf{Yan} tinkay-Ø yam Cānlay //
	\glc \PatT{} give-\TsgM{} \textbf{\Aarg{}} \textbf{Yan} key-\Top{} 
		\Dat{} Cānlay //
	\glft `The key, Yan gives it to Cānlay.' //
\endgl\xe

\subsubsection{Patient}
\index{cases!patient|(}

Patients are less of a definitional problem than agents, since in transitive 
sentences, they are very typically undergoers, that is, the constituent that is 
acted on, affected, or produced by the action expressed by the verb. The 
patient case is thus the one assigned by default to direct objects---but also 
to predicative nominals. In ditransitive sentences, the theme is represented by 
the patient. Animate patients are marked by \rayr{/AsF}{-as}, inanimate ones by 
\rayr{/lej}{-ley}; for names and verbal topic agreement, the markers are 
\rayr{s}{sa} and \rayr{le}{le}, respectively:

\pex
\a\begingl
	\gla Ang silvye {} Briha \textbf{sa} \textbf{Taryan}. //
	\glb Ang silv-ye Ø Briha \textbf{sa} \textbf{Taryan} //
	\glc \AgtT{} see-\TsgF{} \Top{} Briha \textbf{\Parg{}} \textbf{Taryan}//
	\glft `Briha sees Taryan.' //
\endgl

\a\begingl
	\gla \textbf{Sa} manye ang Briha \textbf{{}} \textbf{Taryan}. //
	\glb \textbf{Sa} man-ye ang Briha \textbf{Ø} \textbf{Taryan} //
	\glc \textbf{\PatT{}} greet-\TsgF{} \Aarg{} Briha \textbf{\Top{}} 
		\textbf{Taryan} //
	\glft `Taryan is greeted by Briha',\\
		or: `Taryan, Briha greets him.' //
\endgl

\xe

\pex~
\a\begingl
	\gla Ang rimaye {} Briha \textbf{kunangley}. //
	\glb Ang rima-ye Ø Briha \textbf{kunang-ley} //
	\glc \AgtT{} close-\TsgF{} \Top{} Briha \textbf{door-\PargI{}} //
	\glft `Briha closes a/the door.' //
\endgl

\a\begingl
	\gla \textbf{Le} rimaye ang Briha \textbf{kunang}. //
	\glb \textbf{Le} rima-ye ang Briha \textbf{kunang-Ø} //
	\glc \textbf{\PatTI{}} close-\TsgF{} \Aarg{} Briha 
		\textbf{door-\Top{}} //
	\glft `The door is closed by Briha',\\
		or: `The door, Briha closes it.' //
\endgl

\xe

\ex~
\begingl
	\gla Ang ilya {} Taryan \textbf{koyaley} yam Kandan. //
	\glb Ang il-ya Ø Taryan \textbf{koya-ley} yam Kandan //
	\glc \AgtT{} give-\TsgM{} \Top{} Taryan \textbf{book-\PargI{}} \Dat{} 
		Kandan //
	\glft `Taryan gives Kandan a book.' //
\endgl

\xe

As the translations of the examples above show, topicalizing the patient can be 
used to create an effect similar to English's passive voice, except that the 
patient will not become marked by the agent case for logical reasons---this is 
a notable difference from the nominative. Even if the agent NP is omitted, the 
patient NP will not be changed to the agent case, since that would reverse the 
direction of action:

\ex\begingl
	\gla Manya sa Taryan. ≠ Manya ang Taryan. //
	\glb Man-ya sa Taryan {} Man-ya ang Taryan //
	\glc greet-\TsgM{} \Parg{} Taryan {} greet-\TsgM{} \Aarg{} Taryan //
	\glft `Taryan is greeted.' ≠ `Taryan greets.' //
\endgl\xe

This example shows that the case of the NP will not change, however, the 
verb will: it now agrees with the next argument in line, the patient NP. It 
will not do so, however, if the order of arguments is just scrambled, as 
exemplified by (\ref{ex:verbscram}). This is to say that the verb does not 
simply agree with whichever NP follows it, even if it can be assumed that verb 
agreement in Ayeri developed along similar lines in-world, which will become 
especially apparent in the discussion of pronouns.\footnote{Mismatches in 
agreement in connection to scrambling such as exemplified by 
(\ref{ex:scramfalse}) are to be expected, however, since the brain can only 
handle so much information between the controller and the target of an 
agreement relationship. \citet{corbett2006}, notes that with regards to 
agreement in NP conjuncts, \textcquote[62]{corbett2006}{distant agreement is 
rare, and that agreement with the nearest noun phrase or agreement with all 
(resolution) is much more common}. If there were an extensive corpus of 
texts written by Ayeri speakers, it might be interesting to gather statistics 
on the number of words between target and controller in relation to the 
prevalence of agreement mismatches.}

\tikzstyle{every picture}+=[remember picture]
\pex[aboveglftskip=2em]\label{ex:verbscram}
\a\label{ex:scramcorr}\begingl
	\gla Sa manye {} Taryan ang Briha. //
	\glb Sa man-ye Ø Taryan ang Briha //
	\glc \PatT{} greet\tikz\node[na](target){-\TsgF{}}; \Top{} Taryan 
		\Aarg{} \tikz\node[na](controller){Briha}; //
	\glft `Taryan is greeted by Briha',\\
		or: `Taryan, Briha greets him.' //
\endgl

\a\label{ex:scramfalse}\ljudge* \begingl
	\gla Sa manya {} Taryan ang Briha. //
	\glb Sa man-ya Ø Taryan ang Briha //
	\glc \PatT{} greet\tikz\node[na](target2){-\TsgM{}}; \Top{} 
		\tikz\node[na](controller2){Taryan}; \Aarg{} Briha //
\endgl\xe
\begin{tikzpicture}[overlay]
	\coordinate [below=.25em of controller] (A);
	\coordinate [below=1em of controller] (B);
	\coordinate [below=1em of target] (C);
	\coordinate [below=.25em of target] (D);
	\draw [-latex] (A) -- (B) -- (C) -- (D);
	\node (label) at ($(B)!0.5!(C)$) [below] {\tiny\itshape person 
		agreement};
	
	\coordinate [below=.25em of controller2] (A);
	\coordinate [below=1em of controller2] (B);
	\coordinate [below=1em of target2] (C);
	\coordinate [below=.25em of target2] (D);
	\draw [-latex, dotted] (A) -- (B) -- (C) -- (D);
	\node (label) at ($(B)!0.5!(C)$) [below] {\tiny\itshape *person 
		agreement};
\end{tikzpicture}

Besides being the default case for direct objects, the patient case is also 
assigned to predicative nominals, by analogy with transitive sentences and in 
spite of the likening nature of the construction:

\ex\begingl
	\gla Ang Yan \textbf{nimpayās} ban. //
	\glb Ang Yan \textbf{nimpaya-as} ban //
	\glc \Aarg{} Yan \textbf{runner-\Parg{}} good //
	\glft `Yan is a good runner.' //
\endgl\xe

\index{cases!patient|)}

\subsubsection{Dative}
\label{subsubsec:dative}
\index{cases!dative|(}

The most typical use of the dative is for the recipient NP in a ditransitive 
clause; as such, it may be a recipient proper or the entity to whose benefit 
the action is carried out. A number of transitive verbs also use the dative 
for their object, for example, when it is the target of address. The dative can 
furthermore be used to mark movement toward a place. The case suffix for 
datives is \rayr{/ymF}\fw{-yam} for both animate and inanimate entities. Names 
and verbal topic agreement are marked equally by \rayr{ymF}{yam}. Verbs do not 
exhibit person agreement with dative NPs, since experiencers are treated as 
agents.

\pex\label{ex:datregular}
\a\begingl
	\gla Ang ilya {} Taryan koyaley \textbf{ayonyam}. //
	\glb Ang il-ya Ø Taryan koya-ley \textbf{ayon-yam} //
	\glc \AgtT{} give-\TsgM{} \Top{} Taryan book-\PargI{} 
		\textbf{man-\Dat{}} //
	\glft `Taryan gives a book to the man.' //
\endgl

\a\begingl
	\gla Ang ilya {} Taryan koyaley \textbf{yam} \textbf{Kandan}. //
	\glb Ang il-ya Ø Taryan koya-ley \textbf{yam} \textbf{Kandan} //
	\glc \AgtT{} give-\TsgM{} \Top{} Taryan book-\PargI{} \textbf{\Dat{}} 
		\textbf{Kandan} //
	\glft `Taryan gives Kandan a book.' //
\endgl

\a\begingl
	\gla \textbf{Yam} ilya ang Taryan koyaley \textbf{ayon}. //
	\glb \textbf{Yam} il-ya ang Taryan koya-ley \textbf{ayon-Ø} //
	\glc \textbf{\DatT{}} give-\TsgM{} \Aarg{} Taryan book-\PargI{} 
		\textbf{man-\Top{}} //
	\glft `The man is given a book by Taryan',\\
		or: `The man, Taryan gives him a book.' //
\endgl

\xe

The three examples in (\ref{ex:datregular}) show the regular use of the dative 
as the case the recipient of the theme appears in. What distinguishes Ayeri 
from a pure split-S language is that all constituents can serve as topics, not 
just agents and patients with regards to their function as syntactic subjects. 
Thus, it is also possible for dative NPs to appear as topics---person 
agreement is unaffected by this, though. The following example shows the 
addressee of a speech act in the dative case; the message is treated as the 
theme which is passed on:

\ex
\begingl
	\gla Ang ningye māva ninganas \textbf{ganyam} yena. //
	\glb Ang ning-ye māva-Ø ningan-as \textbf{gan-yam} yena //
	\glc \AgtT{} tell-\TsgF{} mother-\Top{} story-\Parg{} 
		\textbf{child-\Dat{}} \TsgF{}.\Gen{} //
	\glft `The mother tells her child a story.' //
\endgl
\xe

As mentioned above, the dative can also take on an allative meaning insofar as 
it marks the target of a motion, as displayed in (\ref{ex:datloc}). As an 
extention of this means, the adpositional object may as well appear in the 
dative, since Ayeri cannot distinguish, for instance, `up' from `to the top of' 
with just the preposition, in this case \xayr{liNF}{ling}{on top of}. With the 
adpositional object in the locative case (see below), the phrase in 
(\ref{ex:datlocprep}) would imply that the man were literally going to the top 
of the temple, that is, possibly ending up on its roof.

\pex
\a\label{ex:datloc}\begingl
	\gla Ang nimpye lay \textbf{māvayam} yena. //
	\glb Ang nimp-ye lay-Ø \textbf{māva-yam} yena //
	\glc \AgtT{} run-\TsgF{} girl-\Top{} \textbf{mother-\Dat{}} 
		\TsgF{}.\Gen{} //
	\glft `The girl runs to her mother.' //
\endgl

\a\label{ex:datlocprep}\begingl
	\gla Ang saraya ayon manga ling \textbf{natrangyam}. //
	\glb Ang sara-ya ayon-Ø manga ling \textbf{natrang-yam} //
	\glc \AgtT{} go-\TsgM{} man-\Top{} \Dyn{} top \textbf{temple-\Dat{}} //
	\glft `The man goes up to the temple.' //
\endgl

\xe

\index{cases!dative|)}

\subsubsection{Genitive}
\label{subsubsec:genitive}
\index{cases!genitive|(}

The genitive is used to mark possessors; attributive genitives follow the 
possessee. It can also be used for ablative meanings, that is, to mark the 
place from which a motion originates, in analogy to the dative's allative use. 
The genitive is marked on common nouns with the suffix \rayr{/n}{-na}. If a 
noun stem ends in a consonant, the marker becomes \rayr{/En}{-ena}, compare 
Figures \ref{fig:anideclcons}–\ref{fig:inandeclvow} above. Names and verbal 
topic agreement are marked by \rayr{n}{na}. There is no animacy distinction in 
the genitive case.

\pex
\a\begingl
	\gla Pakur ledanang \textbf{netuna} nā. //
	\glb Pakur ledan-ang \textbf{netu-na} nā //
	\glc sick friend-\Aarg{} \textbf{brother-\Gen{}} \Fsg{}.\Gen{} //
	\glft `My brother's friend is sick.' //
\endgl

\a\begingl
	\gla Kopo dilengyereng \textbf{ajānena}. //
	\glb Kopo dileng-ye-reng \textbf{ajān-ena} //
	\glc difficult rule-\Pl{}-\AargI{} \textbf{game-\Gen{}} //
	\glft `The rules of the game are difficult.' //
\endgl

\a\begingl
	\gla Ang nakasyo tamo ibangya \textbf{na} \textbf{Niyas}. //
	\glb Ang nakas-yo tamo-Ø ibang-ya \textbf{na} \textbf{Niyas} //
	\glc \AgtT{} grow-\TsgN{} wheat-\Top{} field-\Loc{} \textbf{\Gen} 
		\textbf{Niyas} //
	\glft `There is wheat growing on Niyas's field.' //
\endgl

\a\begingl
	\gla \textbf{Na} nakasyo tamoang ibangya \textbf{{}} \textbf{Niyas}. //
	\glb \textbf{Na} nakas-yo tamo-ang ibang-ya \textbf{Ø} \textbf{Niyas} //
	\glc \textbf{\GenT{}} grow-\TsgN{} wheat-\Aarg{} field-\Loc{} 
		\textbf{\Top} \textbf{Niyas} //
	\glft `Regarding Niyas, there is wheat growing on his field.' //
\endgl

\xe

Futhermore, Ayeri does not make a distinction between alienable and inalienable 
possession, so that typically inalienable things such as body parts, 
relatives and family members, or personal items and tools are all treated 
as described above. Consider the following example for illustration:

\ex\begingl
	\gla Ang puntaye māva \textbf{nā} mitrangas \textbf{yena} sembari 
		\textbf{yena}. //
	\glb Ang punta-ye māva-Ø \textbf{nā} mitrang-as \textbf{yena} semba-ri 
		\textbf{yena} //
	\glc \AgtT{} brush-\TsgF{} mother-\Top{} \textbf{\Fsg{}.\Gen{}} 
		hair-\Parg{} \textbf{\TsgF{}.\Gen{}} comb-\Ins{} 
		\textbf{\TsgF{}.\Gen{}} //
	\glft `My mother is brushing her hair with her comb.' //
\endgl\xe

The above examples show the regular use of the genitive as a marker of 
possession. The following examples, on the other hand, show the genitive in its 
ablative function, first without qualification by a preposition, then with the 
preposition \xayr{AvnF}{avan}{at the bottom of}, which together with the 
genitive assumes the meaning `down from':

\pex
\a\begingl
	\gla Ang sahaya {} Vetayan \textbf{rimanena}. //
	\glb Ang saha-ya Ø Vetayan \textbf{riman-ena} //
	\glc \AgtT{} come-\TsgM{} \Top{} Vetayan \textbf{city-\Gen{}} //
	\glft `Vetayan comes from the city.' //
\endgl

\a\begingl
	\gla Sahu manga avan \textbf{mehirena}, Niva! //
	\glb Saha-u manga avan \textbf{mehir-ena}, Niva //
	\glc come-\Imp{} \Dyn{} at.bottom \textbf{tree-\Gen{}}, Niva //
	\glft `Come down from the tree, Niva!' //
\endgl

\xe

\index{cases!genitive|)}

\subsubsection{Locative}
\index{cases!locative|(}

The locative marks basic locations, often the default that is associated with a 
verb. It is also the case in which adpositional objects normally appear, 
besides the special cases using the dative and the genitive mentioned above. 
Common nouns are marked by \rayr{/y}{-ya};\footnote{Older texts still exhibit 
an allomorph \rayr{/E\_a}{-ea}, used especially in combination with the plural 
suffix \rayr{/ye}, giving \rayr{/yee\_a}{-yēa}. The modern language uses 
\rayr{/yey}{-jya}.} names and verbal topic agreement use the marker 
\rayr{y}{ya}. There is no difference made between animate and inanimate 
referents in the locative.

\pex\label{ex:locplain}
\a\label{ex:locnedra}\begingl
	\gla Ang nedraya paray \textbf{hinya}. //
	\glb Ang nedra-ya paray-Ø \textbf{hin-ya} //
	\glc \AgtT{} sit-\TsgM{} cat-\Top{} \textbf{box-\Loc{}} //
	\glft `The cat sits in the box.' //
\endgl

\a\label{ex:locnara}\begingl
	\gla Ang naraya {} Ajān \textbf{ya} \textbf{Kaman}. //
	\glb Ang nara-ya Ø Ajān \textbf{ya} \textbf{Kaman} //
	\glc \AgtT{} speak-\TsgM{} \Top{} Ajān \textbf{\Loc{}} \textbf{Kaman} //
	\glft `Ajān speaks to Kaman.' //
\endgl

\a\label{ex:locmit}\begingl
	\gla \textbf{Ya} mica ang Kaman \textbf{{}} \textbf{Visamhinang}. //
	\glb \textbf{Ya} mit-ya ang Kaman \textbf{Ø} \textbf{Visamhinang} //
	\glc \textbf{\LocT{}} live-\TsgM{} \Aarg{} Kaman \textbf{\Top{}} 
		\textbf{Visamhinang} //
	\glft `Kaman lives in Visamhinang',\\
		or: `Visamhinang is where Kaman lives.' //
\endgl

\xe

The example sentences in (\ref{ex:locplain}) show locative NPs that are not 
further specified by adpositions so that the correct interpretation may be 
dependent on context and the experience of the addressee. Example 
(\ref{ex:locnedra}) is an instance of this circumstance, insofar as experience 
tells that cats like to sit inside boxes, so further specifying the position 
with the preposition \xayr{koNF}{kong}{inside} would be emphasizing that the 
cat is not sitting just anywhere, but really \emph{inside} the box as opposed to 
on top of it, for instance. The following example has the cat sitting on top of 
the box:

\ex\begingl
	\gla Ang nedraya paray ling hinya. //
	\glb Ang nedra-ya paray-Ø ling hin-ya //
	\glc \AgtT{} sit-\TsgM{} cat-\Top{} on.top box-\Loc{} //
	\glft `The cat sits on the box.' //
\endgl\xe

Ayeri also has a number of postpositions, which do not change marking on the 
adpositional object, however:

\ex\begingl
	\gla Ang mican edaya \textbf{tenyanya} tan pesan. //
	\glb Ang mit-yan edaya \textbf{tenyan-ya} tan pesan //
	\glc \AgtT{} live-\TplM{} here \textbf{death-\Loc{}} \TplM{}.\Gen{} 
		until //
	\glft `They lived here until their death.' //
\endgl\xe

\index{cases!locative|)}

\subsubsection{Causative}
\index{cases!causative|(}

The causative marks the cause or causer of an action, the instigator or the 
reason on behalf of which an agent is acting. It is thus similar to the agent 
case, though it does not replace it in Ayeri; verbs do not exhibit person 
agreement with causers even though their action logically supersedes or 
precedes that of the agent in the embedded event. \citet{dixon2000} writes that 
a \textcquote[30]{dixon2000}{causer refers to someone or something (which can 
be an event or state) that initiates or controls the activity. This is the 
defining property of the syntactic--semantic function A (transitive subject)}. 
According to \citet[176]{comrie1989}, the causee---the agent of the event 
controlled by the causer---normally takes the highest place in the hierarchy of 
syntactic constituents that is not already filled, in this case, by the causer. 
This observation, however, is complicated by Ayeri's more or less 
semantics-based case marking as well as topicalization. In the following, I 
will give examples of nominal marking for cause as before; a discussion of the 
morphosyntax of Ayeri's morphological causative constructions will be deferred 
to the section on valency-increasing operations.

Causers or causes are marked by \rayr{/Is}{-isa} for common nouns; names and 
verbal topic agreement use the marker \rayr{saa}{sā}. As stated above, verbs do 
not agree with causers even though they have agent-like semantics. There is no 
animacy distinction in the marking of causers.

\pex
\a\begingl
	\gla Ang rua sarāyn \textbf{seyaranisa}. //
	\glb Ang rua sara=ayn.Ø \textbf{seyaran-isa} //
	\glc \AgtT{} must leave=\Fpl{}.\Top{} \textbf{rain-\Caus{}} //
	\glft `We had to leave due to the rain.' //
\endgl

\a\begingl
	\gla Ang yomāy edaya \textbf{sā} \textbf{Apican}. //
	\glb Ang yoma=ay.Ø edaya \textbf{sā} \textbf{Apican} //
	\glc \AgtT{} be=\Fsg{}.\Top{} here \textbf{\Caus{}} \textbf{Apican} //
	\glft `I am here because of Apican.' //
\endgl

\a\label{ex:caustop}\begingl
	\gla \textbf{Sā} nimpvāng hakasley \textbf{yan}. //
	\glb \textbf{Sā} nimp=vāng hakas-ley \textbf{yan.Ø} //
	\glc \textbf{\CauT{}} run=\Ssg{}.\Aarg{} mile-\PargI{} 
		\textbf{\TplM{}.\Top{}} //
	\glft `You run a mile because of them',\\
		or: `Due to them, you run a mile',\\
		or: `They make you run a mile.' //
\endgl

\xe

Regarding the typological oddities mentioned above, example (\ref{ex:caustop}) 
shows what happens in Ayeri with regards to the marking of causers. 
Essentially, the causer topic was grammaticalized to express a causation 
relationship.

\index{cases!causative|)}

\subsubsection{Instrumental}
\label{subsubsec:instrumental}
\index{cases!instrumental|(}

The instrumental marks the means by which an action is carried out by an agent. 
This can be a tool as well as an animate being by whose help the action is 
brought about. The instrumental thus, in effect, marks secondary agents; verbs, 
however, never show person agreement with instrumental NPs. Common nouns are 
marked by \rayr{/ri}{-ri} when ending in a vowel and with \rayr{/Eri}{-eri} 
when ending in a consonant; names and verbal topic agreement are marked by 
\rayr{ri}{ri}. With nouns ending in \fw{-e}, as well as the plural marker 
\rayr{/ye}{-ye}, there is variation regarding whether \rayr{/ri}{-ri} or 
\rayr{/Eri}{-eri} is used, so that in the case of the plural marker both 
\rayr{/yeri}{-yeri} and \rayr{/yeeri}{-yēri} occur. In passive-like 
constructions, it is not grammatical to reintroduce the agent as an 
instrumental; the agent simply remains in the clause in this case, though as a 
non-topic constituent.

\pex
\a\begingl
	\gla Ang visye {} Pila seygoley \textbf{tihangeri} yena. //
	\glb Ang vis-ye Ø Pila seygo-ley \textbf{tihang-eri} yena. //
	\glc \AgtT{} cut-\TsgF{} \Top{} Pila apple-\PargI{} 
		\textbf{knife-\Ins{}} \TsgF{}.\Gen{} //
	\glft `Pila cuts an apple with her knife.' //
\endgl

\a\begingl
	\gla Ang lihoyya-ma badan \textbf{nihanyeri} \textbf{(nihanyēri)}. //
	\glb Ang liha-oy-ya=ma badan-Ø \textbf{nihan-ye-ri} 
		\textbf{(nihan-ye-eri)} //
	\glc \AgtT{} earn-\Neg{}-\TsgM{}=enough father-\Top{} 
		\textbf{nihan-\Pl{}-\Ins} \textbf{(nihan-\Pl{}-\Ins)} //
	\glft `Father did not earn enough with his fruits.' //
\endgl

\a\begingl
	\gla Ang lingya {} Mindan mehiras \textbf{ri} \textbf{Kadijān}. //
	\glb Ang ling-ya Ø Mindan mehir-as \textbf{ri} \textbf{Kadijān}. //
	\glc \AgtT{} climb.up-\TsgM{} \Top{} Mindan tree-\Parg{} 
		\textbf{\Ins{}} \textbf{Kadijān} //
	\glft `Mindan climbs a tree with Kadijān's help.' //
\endgl

\a\begingl
	\gla \textbf{Ri} tavya gino ang Kan \textbf{nimpur}. //
	\glb \textbf{Ri} tav-ya gino ang Kan \textbf{nimpur-Ø} //
	\glc \textbf{\InsT{}} become-\TsgM{} drunk \Aarg{} Kan 
		\textbf{wine-\Top{}} //
	\glft `Kan becomes drunk on the wine', \\
		or: `The wine, Kan becomes drunk on it.' //
\endgl

\xe

The instrumental may also be used for comitative meanings where the 
instrumental NP describes an attribute of its antecedent, for example:

\ex\begingl
	\gla Ang pegayo sinya kasuley \textbf{bariri} nā? //
	\glb Ang pega-yo sinya-Ø kasu-ley \textbf{bari-ri} nā //
	\glc \AgtT{} steal-\TsgN{} who-\Top{} basket-\PargI{} 
		\textbf{meat-\Ins{}} \Fsg{}.\Gen{} //
	\glft `Who stole my basket of meat?' //
\endgl\xe

In this case, \rayr{bri}{bari} is marked as an instrumental since it is an 
attribute of sorts to \rayr{ksu}{kasu}: the instrumental NP describes what its 
antecedent contains or entails more specifically: it is a basket \fw{with} meat 
in it. Note, however, that this comitative use of the instrumental is different 
from mere accompaniment. Thus, it is not possible to say

\ex\label{ex:wrongcomit}\ljudge* \begingl
	\gla Ang sahaya {} Ajān \textbf{ri} \textbf{Pila}. //
	\glb Ang saha-ya Ø Ajān \textbf{ri} \textbf{Pila} //
	\glc \AgtT{} come-\TsgM{} \Top{} Ajān \textbf{\Ins{}} \textbf{Pila} //
\endgl\xe

\noindent to express `Ajān comes (together) with Pila'. The sentence in 
(\ref{ex:wrongcomit}) would instead imply that Pila helps Ajān to come, for 
example, because he has a sprained ankle and thus needs support to go places. 
To express accompaniment, instead, the preposition \xayr{kjvo}{kayvo}{with, 
along, beside} has to be used; the prepositional object appears in the locative 
case:

\ex\begingl
	\gla Ang sahaya {} Ajān \textbf{kayvo} \textbf{ya} \textbf{Pila}. //
	\glb Ang saha-ya Ø Ajān \textbf{kayvo} \textbf{ya} \textbf{Pila} //
	\glc \AgtT{} come-\TsgM{} \Top{} Ajān \textbf{with} \textbf{\Loc{}} 
		\textbf{Pila} //
	\glft `Ajān comes (together) with Pila.' //
\endgl\xe

Theoretically, it should be possible as well to use the instrumental together 
with prepositions for some kind of prolative meaning. The adposition would 
indicate the place \emph{by way of} a motion is happening:

\ex\begingl
	\gla Ang pukay manga luga \textbf{lahaneri}. //
	\glb Ang puk=ay.Ø manga luga \textbf{lahan-eri} //
	\glc \AgtT{} jump=\Fsg{}.\Top{} \Dyn{} top \textbf{fence-\Ins{}} //
	\glft `I jump over the fence.' //
\endgl\xe

This use of the instrumental is unattested in previous translations into Ayeri, 
however, but could be considered a stylistic alternative---in the case of the 
example above, to a construction with the word for `over', 
\rayr{EjrrY}{eyrarya}:

\ex\begingl
	\gla Ang pukay manga eyrarya lahanya. //
	\glb Ang puk=ay.Ø manga eyrarya lahan-ya //
	\glc \AgtT{} jump=\Fsg{}.\Top{} \Dyn{} over fence-\Loc{} //
	\glft `I jump over the fence.' //
\endgl\xe

A more literal translation of \rayr{mN lug lhneri}{manga luga lahaneri} is `by 
way of the top of the fence', though without the verbosity of the English 
translation, as both ways to express the circumstance are about equally long in 
Ayeri.

\index{cases!instrumental|)}

\subsubsection{Case-unmarked nouns}
\label{subsec:uncased}

Case endings are applied to nouns in Ayeri only if the word is actually in a 
syntactic context where case can be applied. Thus, the unmarked form is the 
citation form, not the one declined for agent. This is the case when addressing 
people---one might speak of an unmarked vocative:

\pex
\a\label{ex:vocnoun}\begingl
	\gla Raypu, \textbf{petāya}! //
	\glb Raypa-u, \textbf{petāya} //
	\glc stop-\Imp{}, \textbf{idiot} //
	\glft `Stop it, you idiot!' //
\endgl

\a\label{ex:vocname}\begingl
	\gla Sahu edaya, \textbf{Diras}! //
	\glb Saha-u edaya, \textbf{Diras} //
	\glc come-\Imp{} here, \textbf{Diras} //
	\glft `Come here, Diras!' //
\endgl
\xe

Imperative forms have underlying second-person agents, so both the `idiot' in 
(\ref{ex:vocnoun}) and Diras in (\ref{ex:vocname}) would be the implied agents 
of their sentences, yet neither the noun nor the name are marked by the agent 
markers \rayr{/ANF}{-ang} and \rayr{ANF}{ang}, respectively. Another case where 
nouns are not necessarily marked for case is attested in translations for the 
prefix \xayr{ku/}{ku-}{like, as though} when the phrase acts as an adverb or an 
object complement:

\pex
\a\label{ex:kuudhr}\begingl
	\gla … nay ang mya rankyon sitanyās \textbf{ku-netu}. //
	\glb … nay ang mya rank=yon.Ø sitanya-as \textbf{ku=netu} //
	\glc … and \AgtT{} be.supposed.to treat=\TplN{}.\Top{} 
		each.other-\Parg{} \textbf{like=brother} //
	\glft `… and they shall treat each other like brothers.'\footnotemark%
	\tc{\citep{benung:udhr}}//
\endgl

\a\label{ex:kukafka}\begingl
	\gla … ang nunaya \textbf{ku-vipin} … //
	\glb … ang nuna=ya.Ø \textbf{ku=vipin} … //
	\glc … \AgtT{} fly=\TsgM{}.\Top{} \textbf{like=bird} … //
	\glft `… he (would) fly like a bird …'%
	\tc{\citep[14]{becker:kafka:imperial}}//
\endgl

\xe

\footnotetext{The original English text this was translated from has 
\textcquote[Article 1]{udhr}{and should act towards one another in a spirit of 
brotherhood}.}

Strikingly, in example (\ref{ex:kuudhr}), \xayr{netu}{netu}{brother} in 
\xayr{ku/netu}{ku-netu}{like brothers} is not even inflected for plural; its 
placement after the object is likely an effect of translation: adverbs 
have a strong tendency to appear right after the verb, and a position 
immediately to the right of the verb is attested for adjectival object 
predicatives as well. In (\ref{ex:kukafka}), on the other hand, 
\xayr{ku/vipinF}{ku-vipin}{like a bird} is feasibly interpretable as an adverb, 
since it follows the verb and acts as a modifier to it, not as a complement.

Nouns may also be unmarked if they act as modifiers in a compound and the head 
is marked for the NP's case and number, for instance:

\ex\begingl
	\gla ralanyeri mapang //
	\glb ralan-ye-ri mapang //
	\glc nail-\Pl{}-\Ins{} finger //
	\glft `with the fingernails' //
\endgl\xe

And lastly and probably most importantly, nouns appear superficially unmarked 
if topicalized, since the topic marker is \fw{-Ø}:

\ex\begingl
	\gla Saru-nama, ang nupoyya \textbf{veney} aruno vās. //
	\glb Sar-u=nama, ang nupa-oy-ya \textbf{veney-Ø} aruno vās //
	\glc go-\Imp{}=just, \AgtT{} hurt-\Neg{}-\TsgM{} \textbf{dog-\Top{}} 
 		brown \Ssg{}.\Parg{} //
	\glft `Just go, the brown dog won't hurt you.' //
\endgl\xe

\index{cases|)}

\subsection{Prefixes on nouns}
\label{subsec:nounpref}
\index{prefixes!on nouns|(}

All of the nominal morphology we have so far dealt with in this section was 
suffixing. As mentioned in the previous section already 
(p.~\pageref{nounprefixes}), however, there are also a number of prefixes that 
can be applied to nouns. I have just given two examples of the prefix 
\xayr{ku/}{ku-}{like, as though} above, but \rayr{ku/}{ku-} applies not only to 
nouns themselves. In fact, it rather attaches to whole NPs, which makes it a 
clitic\index{clitics} according to \citet[117]{klavans1985}, and a special 
clitic in \citeauthor{zwicky1977}'s terminology, since no corresponding full 
form exists in its place, comparable to the English possessive clitic \fw{'s}, 
for instance \parencites[3, 
13]{zwicky1977}[295]{zwicky1985}[510]{zwickypullum1983}. To cite from the Ayeri 
translation of Kafka's short story \enquote{Eine kaiserliche Botschaft} again:

\ex\label{ex:kukafka2}\begingl
	\gla … saylingyāng kovaro naynay, ku-ranyāng palung. //
	\glb … sayling=yāng kovaro naynay, ku=ranya-ang palung //
	\glc … progress=\TsgM{}.\Aarg{} easy also, like=nobody-\Aarg{} else //
	\glft `… he also got on easily, like nobody else.'%
	\tc{\citep[12]{becker:kafka:imperial}}//
\endgl\xe

In this example, we can see \rayr{ku/}{ku-} attaching to a properly inflected 
NP adjunct. The NP is case-marked for agent since it can be understood to refer 
to the verb \xayr{sjliNF/}{sayling-}{progress} in the main clause, insofar 
\xayr{rnYaaNF pluNF}{ranyāng palung}{nobody else} can replace 
\xayr{/yaaNF}{-yāng}{he} in the main clause.

Besides \rayr{ku/}{ku-}, there are also the demonstrative prefixes 
\xayr{d/}{da-}{such}, \xayr{Ed/}{eda-}{this}, and \xayr{Ad/}{ada-}{that}, which 
have already been mentioned in the previous section as well. The demonstrative 
prefixes undergo coalescence with nouns beginning with \fw{a-}, that is, they 
form phonological words with their hosts for all means and purposes. The 
demonstrative prefixes are special clitics as well, since no contemporary free 
form exists.

\pex
\a\begingl
	\gla da-nanga kāryo //
	\glb da=nanga kāryo //
	\glc such=house big //
	\glft `such a big house' //
\endgl

\a\begingl
	\gla edāyon nake //
	\glb eda=ayon nake //
	\glc this=man tall //
	\glft `this tall man' //
\endgl

\a\begingl
	\gla ada-envan alingo //
	\glb ada=envan alingo //
	\glc that=woman clever //
	\glft `that clever woman' //
\endgl

\xe

Moreover, there is a prefix \rayr{me/}{mə-} in complementary distribution with 
the demonstrative prefixes, which adds a meaning along the lines of `just any', 
`whatsoever', `some' to the noun. Note that this prefix is distinct from the 
morpheme indicating an inspecific quantity, \xayr{/ArilF}{-aril}{some}.

\pex
\a\begingl
	\gla Ang lampyo mə-veney kayvo kirinya. //
	\glb Ang lamp-yo mə=veney-Ø kayvo kirin-ya //
	\glc \AgtT{} walk-\TsgN{} some=dog-\Top{} along street-\Loc{} //
	\glft `Some dog is walking along the street.' //
\endgl

\a\begingl
	\gla Ang noyan mēntānley pegamayayam. //
	\glb Ang no=yan mə=entān-ley pegamaya-yam //
	\glc \AgtT{} want=\TsgM{}.\Top{} some=punishment-\PargI{} 
		thief-\Dat{} //
	\glft `They demanded some kind of punishment for the thief.' //
\endgl

\xe

\index{prefixes!on nouns|)}

\subsection{Compounding}
\index{compounds|(}

With regards to the classification of compounds, \citet{bauer2001} gives some 
helpful typological guidelines. Besides the compound types recognized by 
Sanskrit grammarians---endocentric (\fw{tatpuruṣa}), coordinative 
(\fw{dvandva}), adjectival-endo\-cent\-ric (\fw{karmadhāraya}), and 
exocentric (\fw{bahuvrīhi})---he also adds synthetic compounds, which Sanskrit 
did not have \citep[697]{bauer2001}. Overall, he finds that determinative, or 
endocentric, compounds are the most common ones in the languages of the world 
\citep[697]{bauer2001}, especially if the head refers to a location or source 
of sorts \citep[702]{bauer2001}.

\citet{gaeta2008}, then, adds to \citeauthor{bauer2001}'s research, based on a 
larger sample of grammars surveyed, that compounds for the largest part 
correlate with the constituent order of the language, both regarding the order 
of verb and object and that of noun and genitive \citep[129--133]{gaeta2008}. 
Mismatches in headedness occur, but appear to constitute the minority of cases 
and may often be explained through historical changes in syntax; he discerns  
for one that \textcquote[135]{gaeta2008}{morphology is not autonomous from 
syntax}, and that secondly, \textcquote[135]{gaeta2008}{[s]yntax seems to be 
the motor of change, which may be then reflected in compounds}, and that 
thirdly, lexical conservativism causes atavisms to linger on, reflecting the 
syntax of earlier stages of the language \citep[138--139]{gaeta2008}.

\index{typology!of compounds}
For the purpose of gaining at least a little insight into which types of 
compounds Ayeri allows---besides endocentric compounds---I conducted a small 
(non-exhaustive) survey based on 130 compounds from the Ayeri dictionary 
\citep[Dictionary]{benung}; \autoref{tab:comptyp} shows the various compound 
classes and the number of words for each. `Harmonic' and `disharmonic', 
respectively, refer to the order of elements; the order is `harmonic' 
if it is following the normal constituent order of the language and 
`disharmonic' if it is at odds with it \citep{gaeta2008}.

\begin{table}[ht]
\caption[Compounds in the Ayeri dictionary]{Compounds in the Ayeri dictionary 
\citep{benung} and their classification (n\,=\,130)}
\begin{tabu} to \linewidth {X[3.5l] X[c] X[c] X[c] X[c] X[c] X[c]}
\tableheaderfont\toprule
Type
	& \multicolumn2{c}{Harmonic}
	& \multicolumn2{c}{Disharmonic}
	& \multicolumn2{c}{Total}
	\\
\toprule

Endocentric (N\,+\,N)
	& 67
	& 51.54\pct
	& 2
	& 1.54\pct
	& 69
	& 53.08\pct
	\\
	
Endocentric (N\,+\,Adj)
	& 18
	& 13.85\pct
	& 4
	& 3.08\pct
	& 22
	& 16.92\pct
	\\

Synthetic (V\,+\,N)
	& 16
	& 12.31\pct
	& 4
	& 3.08\pct
	& 20
	& 15.38\pct
	\\

Coordinative (N\,+\,N)
	& 9
	& 6.92\pct
	& \multicolumn2{c}{---}
	& 9
	& 6.92\pct
	\\
	
Exocentric (N\,+\,N)
	& 1
	& 0.77\pct
	& 3
	& 2.31\pct
	& 4
	& 3.08\pct
	\\
	
\midrule

Unclear
	& 6
	& 4.62\pct
	& \multicolumn2{c}{---}
	& 6
	& 4.62\pct
	\\
	
\midrule

Total
	& 117
	& 90.00\pct
	& 13
	& 10.00\pct
	& 130
	& 100\pct
	\\
	
\bottomrule
\end{tabu}
\label{tab:comptyp}
\end{table}

Unsurprisingly, the largest number of compound nouns in the sample were 
endocentric compounds of the regular kind, which means that, just like genitive 
attributes follow nouns, noun compounds are headed left. Especially compounds 
with adjectives are interesting insofar as this is also the normal order for 
free adjectives, so to illustrate, some tests will be necessary to show that 
these adjectives form a unit with the head noun and are unable to undergo 
comparison, for instance. Synthetic compounds exist in Ayeri and produce nouns. 
These are compounds in which \textcquote[701]{bauer2001}{the modifying element 
in the compound is (usually) interpreted as an argument of the verb from which 
the head is derived}. There are also a number of coordinative compounds; this 
group, however, is lexicalized and not productive. Exocentric compounds 
constitute the minority of the sample. In the following I will give examples 
for each type.

It needs to be noted that unlike Germanic languages, Ayeri does not allow 
compounds of arbitrary length to be strung together, like in the following 
ridiculous but no less real example from (former) German legislation 
\parencite[see, for instance,][]{sz:rindfleisch}:

\ex\begingl\rc{German}%
	\gla %
Rindfleisch­etikettierungs­überwachungs­aufgabenübertragungsgesetz 
//
	\glb Rind-fleisch-­etikettierung-s-­überwachung-s­-aufgabe-n%
		-übertragung-s-gesetz //
	\glc cow-meat-labeling-\Lnk{}-surveillance-\Lnk{}-duty-\Lnk{}%
		-delegation-\Lnk{}-law//
	\glft `Law on the delegation of duties in the surveillance of beef 
		labeling'//
\endgl\xe

In stark contrast, Ayeri allows only two elements in compounds. Furthermore, 
this section on compounds is located within the section on nouns because Ayeri 
almost only possesses compounds involving nouns, and the majority of these also 
results in a noun.

\subsubsection{Endocentric compounds}
\index{compounds!endocentric|(}

To start with the largest group, endocentric/\fw{tatpuruṣa} compounds, the bulk 
of these compounds combines two nouns, one of which is the head which is 
modified by a dependent noun. As Ayeri exhibits a rather strict head-first word 
order, it comes as no surprise, according to \citet{gaeta2008}, that most of 
these compounds follow this order strictly: the second noun modifies the first, 
which is opposite of how English, for instance, typically 
operates:

\pex\label{ex:endonoun}
	\a \makebox[12.2em][l]{\xayr{\larger betjniMpurF}{betaynimpur}{grape}}
		← \xayr{\larger betj}{betay}{berry}
		+ \xayr{\larger niMpurF}{nimpur}{wine}
	\a \makebox[12.2em][l]{\xayr{\larger krirynF}{karirayan}{vertigo}}
		← \xayr{\larger krF}{kar}{fear}
		+ \xayr{\larger IrynF}{irayan}{height}\footnotemark
	\a \makebox[12.2em][l]{\xayr{\larger pikunMdiNF}{pikunanding}{mustache}}
		← \xayr{\larger piku}{piku}{beard}
		+ \xayr{\larger nMdiNF}{nanding}{lips}
	\a \makebox[12.2em][l]{\xayr{\larger tpjperinF}{tapayperin}{sunblind}}
		← \xayr{\larger tpj}{tapay}{screen}
		+ \xayr{\larger perinF}{perin}{sun}
\xe

\footnotetext{\rayr{IrynF}{irayan}, however, is a transparent nominalization of 
\xayr{Irj}{iray}{high}.}

The example words in (\ref{ex:endonoun}) show that the relationships between 
the modifier and the head are various: a grape is a berry \emph{used} to 
make wine from \parencite[compare][702]{bauer2001}; vertigo is the fear 
\emph{of} height; a mustache is a beard \fw{located} over the lips 
\parencite[702]{bauer2001}; and a sunblind is a screen \fw{against} the 
sun.
% \footnote{Further examples include:
% \xayr{AvnMdirunF}{avanandirun}{square root}, lit. `base-square'; 
% \xayr{bidmihnye}{bidamihanaye}{xylophone}, lit. `block-wood-\Pl{}';
% \xayr{bgmFtupoj}{bagamtupoy}{dragon}, lit. `lizard-fire'; 
% \xayr{binMpdNF}{binampadang}{memory}, lit. `picture-mind'; 
% \xayr{burNu\_in}{buranguina}{elephant}, lit. `animal-nose'; 
% \xayr{dgmiMdoj}{dagamindoy}{menu}, lit. `choose-card'; 
% \xayr{dlMpsiNF}{dalampasing}{giraffe}, lit. `cow-neck'; 
% \xayr{drMdevo}{darandevo}{skull}, lit. `bone-head'; 
% \xayr{deveMthaanF}{deventahān}{alphabet}, lit. `system-writing'; 
% \xayr{glimehirF}{galimehir}{resin, tar}, lit. `juice-tree'; 
% \xayr{koybhisF}{koyabahis}{diary}, lit. `book-day'; 
% \xayr{ltuMkem}{latunkema}{tiger}, lit. `lion-stripe'; 
% \xayr{lonupt}{lonupata}{poultice}, lit. `bandage-mash'; 
% \xayr{mliMkronF}{malinkaron}{coast, seashore}, lit. `shore-sea'; 
% \xayr{mehirFgtNF}{mehirgatang}{ovaries}, lit. `tree-womb'; 
% \xayr{mehisiNj}{mehisingay}{conifer}, lit. `tree-needle'; 
% \xayr{mikYnFsitemF}{micansitem}{electric}, lit. `power-lightning'; 
% \xayr{mirMthnF}{mirantahan}{typeface}, lit. `kind-writing'; 
% \xayr{mirMthaanF}{mirantahān}{spelling}, lit. `way-writing'; 
% \xayr{mitFrmtau}{mitramatau}{pubic hair}, lit. `hair-tangle'; 
% \xayr{mitFrnvsNF}{mitranavasang}{axillary hair}, lit. `hair-sweat'; 
% \xayr{nrMbesuhej}{narambesuhey}{dictionary}, lit. `word-list'; 
% \xayr{niMpurivnF}{nimpurivan}{vinyard}, lit. `wine-mountain'; 
% \xayr{ptyelNF}{patayelang}{concrete}, lit. `mash-stone'; 
% \xayr{pikulkj}{pikulakay}{goatee}, lit. `beard-chin'; 
% \xayr{prihiNumo}{prihingumo}{desk}, lit. `table-work'; 
% \xayr{rgMterFpeNF}{raganterpeng}{diameter}, lit. `line-middle'; 
% \xayr{rlmpNF}{ralamapang}{fingernail}, lit. `nail-finger'; 
% \xayr{ridspj}{ridasapay}{glove}, lit. `sock-hand'; 
% \xayr{sNumospoj}{sangumosapoy}{ticket office}, lit. `office-ticket'; 
% \xayr{svtkNF}{savatakang}{tank}, lit. `cart-armor'; 
% \xayr{sayMprl}{sayamparal}{urine hole}, lit. `hole-penis'; 
% \xayr{syNu\_in}{sayanguina}{nostril}, lit. `hole-nose'; 
% \xayr{syniv}{sayaniva}{eye socket}, lit. `hole-eye'; 
% \xayr{selNblN}{selangbalang}{search engine}, lit. `machine-search'; 
% \xayr{selMkurnF}{selangkuran}{computer}, lit. `machine-counting'; 
% \xayr{sepFrkronF}{seprakaron}{ditch}, lit. `cleft-water'; 
% \xayr{similitnF}{similitan}{borderland}, lit. `land-margin-\Nmlz{}'; 
% \xayr{similitj}{similitay}{republic}, lit. `land-democracy'; 
% \xayr{sirjyil}{sirayyila}{knee}, lit. `joint-foot'; 
% \xayr{sirjtinu}{siraytinu}{elbow}, lit. `joint-arm'; 
% \xayr{sirukronF}{sirukaron}{starfish}, lit. `star-water'; 
% \xayr{sirusitFrmF}{sirusitram}{comet}, lit. `star-tail'; 
% \xayr{sirutj}{sirutay}{night}, lit. `star-time'; 
% \xayr{sitNlugaanF}{sitanglugān}{incest}, lit. `self-entry'; 
% \xayr{trFtrihimF}{tartarihim}{tobacco}, lit. `pipe-weed'; 
% \xayr{tepilFpihaanF}{tepilpihān}{fester}, lit. `sore-pus'; 
% \xayr{tFreMdpNisF}{trendapangis}{bank}, lit. `hall-money'; 
% \xayr{tuptinu}{tupatinu}{fathom}, lit. `length-arm'; 
% \xayr{veb\_osnF}{vebaosan}{slug}, lit. `snail-slime'; 
% \xayr{veMkubesonF}{venkubeson}{navy}, lit. `army-ship'; 
% \xayr{vinimyonF}{vinimayon}{monkey}, lit. `forest-man'; 
% \xayr{yno\_avnF}{yanoavan}{area, region}, lit. `place-ground'; 
% \xayr{yelNFssaanF}{yelangsasān}{cobblestone}, lit. `stone-way'; 
% \xayr{yenukrFdNF}{yenukardang}{classmates}, lit. `group-school'; 
% \xayr{yutnjkonF}{yutanaykon}{foreskin}, lit. `skin-cover'.
% }
\citet{bauer2001} mentions that \textquote{there may be special 
morphophonemic processes which apply between the elements of compounds}, such 
as \textcquote[695]{bauer2001}{phonological merger[s] between the elements of 
the compound}. This also occasionally happens in Ayeri, as the next few example 
words show:

\pex\label{ex:endonounmod}
	\a \xayr{\larger AvrrnF}{avararan}{wetland} \\
		← \xayr{\larger AvnF}{avan}{ground}
		+ \xayr{\larger rro}{raro}{wet}
		+ \rayr{\larger /AnF}{-an} (\Nmlz{})
	\a \xayr{\larger mehimitFrNF}{mehimitrang}{fiber tree} \\
		← \xayr{\larger mehirF}{mehir}{tree}
		+ \xayr{\larger mitFrNF}{mitrang}{hair, fiber}
	\a \xayr{\larger niNMpinmF}{ningampinam}{bedtime story} \\
		← \xayr{\larger niNnF}{ningan}{story}
		+ \xayr{\larger pinmF}{pinam}{bed}
	\a \xayr{\larger pdilmikYnF}{padilamican}{gravitational force} \\
		← \xayr{\larger pdilnF}{padilan}{attraction}
		+ \xayr{\larger mikYnF}{mican}{force, power}
\xe

There is a modicum of alteration happening in all of the heads of the example 
words in (\ref{ex:endonounmod}), mostly nasals assimilating to the stop or 
nasal which the modifier begins with (/n/~+~/p/~→~/mp/, /n/~+~/m/~→~/m/), 
though \rayr{AvrrnF}{avararan} and \rayr{mehimitFrNF}{mehimitrang} even delete 
whole coda segments.
% Stuff may even be mashed together completely, but examples??
\citet[703]{bauer2001} notes that very commonly, genitive and plural markers 
may form linking elements, though he also gives examples of languages which 
allow other case markers on the modifying element in languages with head-right 
order; individual languages may allow even more case inflection. However, this 
appears not to happen in Ayeri. The only element that comes up time and again 
in between the two halves of compounds is the nominalizer \rayr{/AnF}{-an}, 
which signifies that the head is being formed by a nominalized root, such as in 
\rayr{pdilmikYnF}{padilamican}, where \xayr{pdilnF}{padilan}{attraction} is a 
nominalization of \xayr{pdilF/}{padil-}{attract}, or in 
\rayr{niNMpinmF}{ningampinam}, where \xayr{niNnF}{ningan}{story} is derived 
from the verb \xayr{niNF/}{ning-}{tell}. However, since Ayeri is head-first and 
possessive phrases are dependent marking, genitive or other case marking would 
be expected on the second element, not the first. Case marking on a compound, 
however, does not inflect just the modifier, but the whole NP:

\ex\begingl
	\gla Ang ningya sipikanena koyabahisena. //
	\glb Ang ning-ya sipik-an-ena koyabahis-ena //
	\glc \AgtT{} talk-\TsgM{}.\Top{} keep-\Nmlz{}-\Gen{} book.day-\Gen{} //
	\glft `He talks about keeping a journal.' //
\endgl\xe

\rayr{koybhisen}{koyabahisena} in this example is not to be interpreted as 
`book of day(s)' but as `of a day-book'. Inflection between the parts of a 
compound can happen nonetheless, though. In compounds which are formed ad 
hoc or which are otherwise transparent in their composition, inflection often 
is deferred to the noun head noun instead of the edge of the compound as a 
whole; the modifier is treated as an adjunct in this case, and stays 
uninflected:

\ex\label{ex:nouncompdiv}\begingl
	\gla Sa trayeng tipin ralanyeri mapang yena. //
	\glb Sa tra=yeng tipin-Ø ralan-ye-ri mapang yena //
	\glc \PatT{} scratch=\TsgF{}.\Aarg{} itch-\Top{} nail-\Pl{}-\Ins{} 
		finger \TsgF{}.\Gen{} //
	\glft `The itch, she scratches it with her fingernails.' //
\endgl\xe

Besides noun modifiers, there are also compounds where the modifier is an 
adjective. In classical Sanskrit terminology, this type is called 
\fw{karmadhāraya} \citep[698--699]{bauer2001}.\footnote{\citeauthor{bauer2001} 
also mentions that appositional compounds like \fw{maid-servant}, \fw{woman 
doctor} and \fw{fighter-bomber} are counted in this category 
\citep[699]{bauer2001}. Ayeri, however, does not possess such formations in 
particular.} Examples in Ayeri include:%
% \footnote{Further examples include: 
% \xayr{bhisino}{bahisino}{holiday, day off}, lit. `day-free'; 
% \xayr{dikuMtrinF}{dikuntaring}{bureaucracy}, lit. `passion-administrative'; 
% \xayr{leMtMkusNF}{lentankusang}{diphthong}, lit. `sound-double'; 
% \xayr{nNbnY}{nangabanya}{hospital}, lit. `house-sick'; 
% \xayr{naraaMtiynF}{narāntiyan}{conlang}, lit. `language-created-\Nmlz{}'; 
% \xayr{rohMpraanF}{rohamparān}{snack}, lit. `bite-quick-\Nmlz{}'; 
% \xayr{rohMkivo}{rohankivo}{snack}, lit. `bite-small'; 
% \xayr{sNumirj}{sangumiray}{ministry, authority}, lit. `office-high'; 
% \xayr{tbMpehu}{tabampehu}{lower jaw}, lit. `jaw-loose'
% \xayr{tabnikp}{tabanikapa}{upper jaw}, lit. `jaw-attached'.
% }

\pex
	\a \makebox[12.5em][l]{\xayr{\larger krFdNirj}{kardangiray}{university}}
		← \xayr{\larger krFdNF}{kardang}{school}
		+ \xayr{\larger Irj}{iray}{high}
	\a \makebox[12.5em][l]{\xayr{\larger mrsFhri}{marashari}{witticism}}
		← \xayr{\larger mrsF}{maras}{phrase}
		+ \xayr{\larger hri}{hari}{pithy}
	\a \makebox[12.5em][l]{\xayr{\larger silFvniknF}{silvanikan}{overview}}
		← \xayr{\larger silFvnF}{silvan}{view}
		+ \xayr{\larger IknF}{ikan}{whole}
	\a \makebox[12.5em][l]{\xayr{\larger vipimkaarY}{vipimakārya}{crow}}
		← \xayr{\larger vipinF}{vipin}{bird}
		+ \xayr{\larger mkaarY}{makārya}{black}
\xe

In all of these cases, the adjective forms a unified lexeme with the head noun, 
hence it is not comparable, for example:

\pex
\a\ljudge* \begingl
	\gla kardangiray-eng \quad{} kardangiray-vā //
	\glb kardang-iray=eng \quad{} kardang-iray=vā //
	\glc school-high=\Comp{} \quad{} school-high=\Supl{} //
	\glft `*higher-school' {} `*highest-school' //
\endgl

\a\ljudge* \begingl
	\gla marashari-eng \quad{} *marashari-vā //
	\glb maras-hari=eng \quad{} maras-hari=vā //
	\glc phrase-pithy=\Comp{} \quad{} phrase-pithy=\Supl{} //
	\glft `*pithier-phrase' {} `*pithiest-phrase' //
\endgl

\xe

In fact, it is possible to form \rayr{krFdNirj vaa}{kardangiray vā} and 
\rayr{mrsFhti vaa}{marasari vā}, but they mean `most universities' and `most 
witticisms', respectively; \xayr{/ENF}{-eng}{rather} as a quantifier does not 
combine with nouns. Since the meaning composed from noun--adjective compounds 
is often idiomatic, they also cannot be divided as shown above in 
(\ref{ex:nouncompdiv}), since a \xayr{krFdNirj}{kardangiray}{university} is not 
a \xayr{krFdNF}{kardang}{school} which is \xayr{Irj}{iray}{high} in the literal 
sense, but a school of the highest tier. \rayr{krFdNen Irj}{kardangena iray} 
(school-\Gen{} high), then, can only be interpreted in the literal sense, `of 
the high school', but not as `of the university', which thus can only be 
\rayr{krFdNiryen}{kardangirayena}.

In the sample, there were also a few compounds I categorized as noun--noun 
combinations, which look as though they violate head-first order. All of these 
involve \xayr{sitNF}{sitang}{self} as a modifier:

\pex\label{ex:compsitang}
	\a \makebox[16em][l]{\xayr{\larger sitNFleMtnF}{sitanglentan}{vowel}}
		← \xayr{\larger sitNF}{sitang}{self}
		+ \xayr{\larger leMtnF}{lentan}{sound}
	\a \makebox[16em][l]{\xayr{\larger sitNFpronaanF}{sitangparonān}%
		{self-confidence}}
		← \xayr{\larger sitNF}{sitang}{self}
		+ \xayr{\larger pronaanF}{paronān}{faith}
	\a \makebox[16em][l]{\xayr{\larger sitNFtenYnF}{sitangtenyan}%
		{suicide}}
		← \xayr{\larger sitNF}{sitang}{self}
		+ \xayr{\larger tenYnF}{tenyan}{death}
\xe

\rayr{sitNF}{sitang} does not exist as a noun by itself in Ayeri, the word for 
`self' is its nominalization \rayr{sitNnF}{sitangan}. Nonetheless, it looks 
like it could have plausibly been a noun once. However, this noun 
may have been grammaticalized into a reflexive morpheme of a more 
general kind, which in turn birthed the form \rayr{sitNnF}{sitangan} as a 
renovation.\footnote{A little bit of language history would certainly simplify 
things here and lend them credence. Let us simply assume that 
\rayr{sitNF}{sitang} used to be a noun meaning something like `self' at a 
previous stage of Ayeri and was repurposed as a reflexive prefix. 
\citet{lehmann2015} quotes a few examples of what he calls `autophoric' nouns 
that came to be used as reflexive pronouns in their respective language: 
\textcquote[45--46]{lehmann2015}{Typical examples are Sanskrit \fw{tan} 
`body, person' and \fw{ātmán} `breath, soul', Buginese \fw{elena} `body', 
Okinawan \fw{dūna} `body', !Xu \fw{l’esi} `body', Basque \fw{burua} `head', 
Abkhaz \fw{a-xə̀̀} `the head'. In their respective languages, all these nouns 
are translation equivalents of English \fw{self}}. Thus, it would not be out of 
line at all to assume such a grammaticalization path for Ayeri as well.} The 
reflexive \rayr{sitNF}{sitang} is used---as we have seen in the previous 
chapter---as a prefix, so there are two ways to intepret these formations: 
first, \rayr{sitNF}{sitang} may be the reflexive prefix here and thus the 
compound follows the normal syntactic order; or second, the order of elements 
is reversed and thus may reflect an earlier stage of Ayeri where 
\rayr{sitNF}{sitang} was still a noun and modifiers could still appear in front 
of their heads, at least optionally so \citep[133--137]{gaeta2008}.

There are a number of genuinely reversed endocentric compounds as well, 
however, in which the modifier comes first and the head last. Since there are 
only a few of these, I will give all of them in the following example:

\pex
	\a \makebox[15em][l]{\xayr{\larger bript}{baripata}{ground meat}}
		← \xayr{\larger bri}{bari}{meat}
		+ \xayr{\larger pt}{pata}{mash}
	\a \makebox[15em][l]{\xayr{\larger kjvoleMtnF}{kayvolentan}{consonant}}
		← \xayr{\larger kjvo}{kayvo}{with}
		+ \xayr{\larger leMtnF}{lentan}{sound}
	\a \makebox[15em][l]{\xayr{\larger maavgneNF}{māvaganeng}{mother's 
		siblings}}
		← \xayr{\larger maav}{māva}{mother}
		+ \xayr{\larger gneNF}{ganeng}{siblings}
	\a \makebox[15em][l]{\xayr{\larger mtinMdiNF}{matinanding}{labia}}
		← \xayr{\larger mtiknF}{matikan}{hot}
		+ \xayr{\larger nMdiNF}{nanding}{lips}
	\a \makebox[15em][l]{\xayr{\larger muyvirNF}{muyavirang}{brass}}
		← \xayr{\larger muy}{muya}{false}
		+ \xayr{\larger AvirNF}{avirang}{gold}
	\a \makebox[15em][l]{\xayr{\larger 
		tonisjtNF}{tonisaytang}{self-assured}}
		← \xayr{\larger tonis}{tonisa}{assured}
		+ \ques{}\,\xayr{\larger sitNnF}{sitangan}{self}
\xe

Given the discussion of \rayr{sitNF}{sitang} above, one word among the examples 
above that is not too clear is \rayr{tonisjtNF}{tonisaytang}, which appears to 
contain a deviant form of either \rayr{sitNF}{sitang} or 
\rayr{sitNnF}{sitangan}, which is preceded by the adjective 
\xayr{tonis}{tonisa}{assured, ascertained}.

All of the previously mentioned compounds involving nominal elements formed 
nouns, though, there are also a few denominal compounds in the sample. This 
process is not productive, however, and interestingly, only noun–adjective 
combinations appear in this group:

\pex
	\a \xayr{\larger mirMpluj}{mirampaluy}{otherwise} \\
		← \xayr{\larger mirnF}{miran}{way}
		+ \ques{}\,\xayr{\larger pluNF}{palung}{different}
	\a \xayr{\larger pdbnY}{padabanya}{insane} \\
		← \xayr{\larger pdNF}{padang}{mind}
		+ \xayr{\larger bny}{banaya}{sick}
	\a \xayr{\larger teMkris/}{tenkarisa-}{be frightened 
		to death} \\
		← \xayr{\larger tenF}{ten}{life}
		+ \xayr{\larger kris}{karisa}{frightened}
\xe

\rayr{mirMpluj}{mirampaluy} is an adverb, the modifier probably a mangling of 
\rayr{pluNF}{palung}; \rayr{pdbnY}{padabanya} is an adjective meaning `insane' 
rather than the expected `insanity' (instead: \rayr{pdbnYaanF}{padabanyān}); 
and \rayr{teMkris/}{tenkarisa-} acts as a verb, possibly from conversion or 
reinterpretation, since the suffix \rayr{/Is}{-isa} also forms morphological 
causatives of a number of verbs. Besides these irregularities, there is also at 
least one noun compound which uses a postposition as an adjectival modifier:

\ex
	\xayr{\larger silFvMkjvj}{silvankayvay}{blindness} 
	← \xayr{\larger silFvnF}{silvan}{sight}
	+ \xayr{\larger kjvj}{kayvay}{without}
\xe

This compound must be derived from the phrase \xayr{silFvnFy kjvj}{silvanya 
kayvay}{without sight} (see-\Nmlz{}-\Loc{} without), though here as well, the 
word roots are simply juxtaposed, as noted above is the common way to form 
compounds in Ayeri.

\index{compounds!endocentric|)}

\subsubsection{Synthetic compounds}
\index{compounds!synthetic|(}

According to \citet{bauer2001}, (semi-)synthetic compounds, or verbal(-nexus) 
compounds, are compounds that have \textcquote[701]{bauer2001}{been variously 
defined as being based on word-groups or syntactic constructions 
\citep[2]{botha1984}, or as compounds whose head elements are derived from 
verbs \citep[3607]{lieber1994}}. Examples of this type in English would include 
\fw{truck-driver}, \fw{peace-keeping}, and \fw{home-made}. He mentions also 
that synthetic compounds have been mainly discussed with regards to Germanic 
languages, but that according to \citet[3608]{lieber1994}, the phenomenon is 
much more widespread. Ayeri possesses compounds like this as well, and the 
regular case again follows the constituent order, here that of verbs and nouns: 
Ayeri is a VO language, and thus the verb as the head of the compound is 
usually found on the left side with its nominal modifier following it 
\citep[129--133]{gaeta2008}:

\pex
	\a \makebox[14em][l]{\xayr{\larger AnFlaagonnF}{anlāgonan}%
		{pronunciation}}
		← \xayr{\larger AnFlF/}{anl-}{bring}
		+ \xayr{\larger AgonnF}{agonan}{outside}
	\a \makebox[14em][l]{\xayr{\larger npkronF}{napakaron}{acid}}
		← \xayr{\larger npF/}{nap-}{burn}
		+ \xayr{\larger kronF}{karon}{water}
	\a \makebox[14em][l]{\xayr{\larger npperinF}{napaperin}{sunburn}}
		← \xayr{\larger npF/}{nap-}{burn}
		+ \xayr{\larger perinF}{perin}{sun}
	\a \makebox[14em][l]{\xayr{\larger telFbssaanF}{telbasasān}{waysign}}
		← \xayr{\larger telFb/}{telba-}{show}
		+ \xayr{\larger ssaanF}{sasān}{way}
\xe

The relations between the verb and the noun are various again, that is, the 
nominal modifier is not simply the direct object of the verb: to pronounce 
something means to bring it \emph{to} the outside; a sunburn is a burn 
\fw{caused by} the sun; and a waysign shows the way (\rayr{ssaanF}{sasān} is 
the object here). Even though \rayr{kronF}{karon} may serve as an agent (or a 
causer) of the burning effect of acid (similarly for 
\xayr{npperinF}{napaperin}{sunburn}), the verb-first order is justified here as 
well, since verbs always go first in Ayeri sentences, and any other NPs, 
whether actor or undergoer, are following.%
% \footnote{Further examples include:
% \xayr{bimkNnF}{bimakangan}{photo}, lit. `paint-light-\Nmlz{}'; 
% \xayr{IlgonnF}{ilagonan}{edition}, lit. `give-out-\Nmlz{}'; 
% \xayr{lMtmidj}{lantamiday}{diversion}, lit. `lead-around'; 
% \xayr{nbisFmaavy}{nabismāvaya}{motherfucker}, lit. `fuck-mother-\Agtz{}'; 
% \xayr{nrkhu}{narakahu}{phone}, lit. `speak-far'; 
% \xayr{srsjliNF}{sarasayling}{progress}, lit. `go-further'; 
% \xayr{silFvkhu}{silvakahu}{TV}, lit. `see-far'; 
% \xayr{silFvmrinnF}{silvamarinan}{preview}, lit. `see-before-\Nmlz{}'; 
% \xayr{telFbgonnF}{telbagonan}{advertisement}, lit. `show-out'; 
% \xayr{vliktu}{valikatu}{masochist}, lit. `enjoys pain'.
% }

Just as with endocentric compounds, there are a number of seeming exceptions to 
the verb-first order of synthetic compounds as well. These are just as far and 
few between, however, and whether they should all be counted as noun–verb 
combinations is also questionable as they appear to all be formed with 
nominalized verbs. The verbal element may thus be only indirectly verbal for 
the purposes of compounding. If interpreted as noun--noun combinations, the 
nominal first element would reasonably form the head again for some of the 
below example words.

\pex\label{ex:compvbrev}
	\a \xayr{\larger mripuMtymF}{maripuntayam}{spread} \\
		← \xayr{\larger mrinF}{marin}{surface}
		+ \xayr{\larger puMt/}{punta-}{stroke}
		+ \rayr{\larger /ymF}{-yam} (\Dat{})
	\a \xayr{\larger ssnFlekaanF}{sasanlekān}{labyrinth} \\
		← \xayr{\larger ssaanF}{sasān}{way}
		+ \xayr{\larger lek/}{leka-}{guess}
		+ \rayr{\larger /AnF}{-an} (\Nmlz{})
	\a \xayr{\larger selNnunaan}{selangnunān}{plane} \\
		← \xayr{\larger selNF}{selang}{machine}
		+ \xayr{\larger nun/}{nuna-}{fly}
		+ \rayr{\larger /AnF}{-an} (\Nmlz{})
	\a \xayr{\larger siMturaanF}{sinturān}{radio} \\
		← \xayr{\larger siMto}{sinto}{wave}
		+ \xayr{\larger tur/}{tura-}{send}
		+ \rayr{\larger /AnF}{-an} (\Nmlz{})
\xe

\rayr{mripuMtymF}{maripuntayam} is special in that it contains the dative 
suffix \rayr{/ymF}{-yam} which is lexicalized as part of the word: something 
made or intended for spreading on a surface. A few more such verbal derivations 
can be found, though not compounds, among others:

\pex
	\a \makebox[10.5em][l]{\xayr{\larger gFrenYmF}{grenyam}{extremity}}
		← \xayr{\larger gFren/}{gren-}{reach out}
	\a \makebox[10.5em][l]{\xayr{\larger lugymF}{lugayam}{password}}
		← \xayr{\larger lug/}{luga-}{go through} 
	\a \makebox[10.5em][l]{\xayr{\larger shymF}{sahayam}{future}}
		← \xayr{\larger sh/}{saha-}{come}
\xe

There is also \xayr{mripuMt/}{maripunta-}{spread over} as the verb corresonding 
to \rayr{mripuMtymF}{maripuntayam}, though its meaning is less specific. Here 
as well, however, the verbal part is last instead of first. For the other 
example words (\ref{ex:compvbrev}b--d), an interpretation of the second part as 
a deverbal noun is possible: a labyrinth as a way or path which requires 
guessing, a plane a machine for flight, and radio as a sending of waves. In the 
latter case, \rayr{siMturaanF}{sinturān}, however, the head is still on the 
wrong side even if one interprets all of the above examples as noun--noun 
compounds with a deverbal element.

\index{compounds!synthetic|)}

\subsubsection{Coordinative compounds}
\index{compounds!coordinative|(}

Coordinative compounds are a very small group among the sample drawn from the 
dictionary, and not a very productive one. \citet{bauer2001} defines this class 
as having \textcquote[699]{bauer2001}{two or more words in a coordinate 
relationship, such that the entity denoted is the totality of the entities 
denoted by each of the elements}. He cautions that they are very easy to 
confuse with appositional (also \fw{karmadhāraya}) compounds in that both types 
of compound allow inserting an \fw{and} between both elements. The following 
nominal coordinative compounds are included in the dictionary sample:

\pex
	\a \makebox[13em][l]{\xayr{\larger baaːm}{bāmā}{mom-and-dad}}
		← \xayr{\larger baa(baa)}{bā(bā)}{dad}
		+ \xayr{\larger maa(maa)}{mā(mā)}{mom}
	\a \makebox[13em][l]{\xayr{\larger pFrujnpj}{pruynapay}{seasoning}}
		← \xayr{\larger pruj}{pruy}{salt}
		+ \xayr{\larger npj}{napay}{pepper}
	\a \makebox[13em][l]{\xayr{\larger spjyil}{sapayyila}{hands-and-feet}}
		← \xayr{\larger spj}{sapay}{hand}
		+ \xayr{\larger yil}{yila}{foot}
	\a \makebox[13em][l]{\xayr{\larger simileno}{simileno}{horizon}}
		← \xayr{\larger similF}{simil}{country}
		+ \xayr{\larger leno}{leno}{sky}
	\a \makebox[13em][l]{\xayr{\larger sitemFrugonF}{sitemrugon}%
		{thunderstorm}}
		← \xayr{\larger sitemF}{sitem}{lightning}
		+ \xayr{\larger rugonF}{rugon}{thunder}
	\a \makebox[13em][l]{\xayr{\larger vekmFdekej}{vekamdekey}{dishes}}
		← \xayr{\larger vekmF}{vekam}{plate}
		+ \xayr{\larger dekej}{dekey}{fork}
\xe

None of the two elements recognizably forms the head in these examples, but 
both elements are typical exponents of the thing the compound signifies. 
\citet[699]{bauer2001} mentions that coordinative adjective compounds are rare, 
or at least rarely documented in the grammars he surveyed. In the sample I 
took, only the following compound is included, which forms a noun from the 
combination of two adjectives, insofar it is relevant to this section even 
though the component parts are not nouns:

\ex
	\xayr{\larger mkgisu}{makagisu}{twilight}
		← \xayr{\larger mk}{maka}{light}
		+ \xayr{\larger gisu}{gisu}{dark}
\xe

The sample also includes the following two words, however, which are neither 
made up from nouns, nor do they form a noun in combination. Instead, they are 
technically verbs combining to form directional adverbs and have been 
exceptionally included here for completeness:

\pex
	\a \makebox[11em][l]{\xayr{\larger mNsh}{mangasaha}{towards}}
		← \xayr{\larger mN/}{manga-}{move}
		+ \xayr{\larger sh/}{saha-}{come}
	\a \makebox[11em][l]{\xayr{\larger mNsr}{mangasara}{away}}
		← \xayr{\larger mN/}{manga-}{move}
		+ \xayr{\larger sr}{sara-}{go}
\xe

\index{compounds!coordinative|)}

\subsubsection{Exocentric compounds}
\index{compounds!exocentric|(}

In exocentric compounds, the modifier is not a hyponym of its head 
\citep[700]{bauer2001}, which means that the modifier is not 
describing a property that more closely determines its head. So while a \fw{dog 
kennel} is a type of kennel made for dogs, the head of an \fw{egghead} is 
neither for eggs, nor containing eggs, nor made of eggs; instead, it refers to 
an egg-shaped skull metaphorically. And while a \fw{bluecollar} may wear a blue 
shirt professionally, the referent it signifies is not a type of collar, but 
the relationship is metonymical in that the blue collar is part of the 
guise of the signified entity as a whole. The sample from the Ayeri dictionary 
contains a few compounds of this kind as well, though again, it is 
not a very productive group:

\pex
	\a \makebox[11em][l]{\xayr{\larger AvnFyonNF}{avanyonang}{artery}}
		← \xayr{\larger AvnF}{avan}{bottom, down}
		+ \xayr{\larger yonNF}{yonang}{stream}
	\a \makebox[11em][l]{\xayr{\larger bjtMdevo}{baytandevo}{headache}}
		← \xayr{\larger bjtNF}{baytang}{blood}
		+ \xayr{\larger devo}{devo}{head}
	\a \makebox[11em][l]{\xayr{\larger linFyonNF}{linyonang}{vein}}
		← \xayr{\larger liNF}{ling}{top, up}
		+ \xayr{\larger yonNF}{yonang}{steam}
	\a \makebox[11em][l]{\xayr{\larger siMdjnN}{sindaynanga}{address}}
		← \xayr{\larger sindj}{sinday}{number}
		+ \xayr{\larger nN}{nanga}{house}
\xe

What is striking here is that only one out of for examples shows the expected 
head-left order: \rayr{siMdjnN}{sindaynanga}. The other three examples all have 
the head head component on the right side, preceded by a modifier. However, 
what all of these have in common, is that they are only metaphorically or 
metonymically describing the thing they signify: veins and arteries are not 
literally streams going up or down (they are a kind of stream flowing in 
different directions, however, so these are probably on the borderline between 
exocentric and endocentric); a headache is related to the head, but has not 
directly to do with being made of or containing blood (the rationale 
behind this being a superstition that you have too much blood in your head, 
which is said to cause the pain); and a house number may be part of an 
address, but is in a \fw{pars pro toto} relationship to it.

\index{compounds!exocentric|)}

\subsubsection{A few mysterious cases}

The following words from my sample were either undeterminable as to their 
composition due to parts of the word not being clear regarding one of their 
constituent parts, either because I tweaked the constituent so much as to not 
be readily recognizable anymore, or because I forgot to make an entry in the 
dictionary, or even deleted or changed that. The words in question are the 
following:

\pex
	\a \makebox[12em][l]{\xayr{\larger btNimnF}{batangiman}{mosquito}}
		← \xayr{\larger bjtNF}{baytang}{blood}
		+ ?
	\a \makebox[12em][l]{\xayr{\larger kirinlNF}{kirinalang}{avenue}}
		← \xayr{\larger kirinF}{kirin}{street}
		+ ?
	\a \makebox[12em][l]{\xayr{\larger niNMbkrF}{ningambakar}{telltale}}
		← \xayr{\larger niNnF}{ningan}{story}
		+ ?
	\a \makebox[12em][l]{\xayr{\larger rgyesuj}{ragayesuy}{grid}}
		← \xayr{\larger rgnF}{ragan}{line}
		+ ?
	\a \makebox[12em][l]{\xayr{\larger terjmino}{teraymino}{melancholic}}
		← ?
		+ \xayr{\larger mino}{mino}{happy}
	\a \makebox[12em][l]{\xayr{\larger vetjsno}{vetaysano}{fare}}
		← ?
		+ \rayr{\larger ssaanF}{sasān} (earlier \rayr{\larger 
			ssno}{sasano}) `way'
\xe

For all of the components represented by a question mark, there is no 
corresponding dictionary entry. At least in \rayr{bjtNimnF}{baytangiman}, the 
*\rayr{ImnF}{*iman} part looks as though it could be a noun due to the 
\rayr{/AnF}{-an} nominalizer suffix. *\rayr{terj}{*teray} in 
\rayr{terjmino}{teraymino} might also be an adjective supposed to mean `sad' 
(which would make it an adjectival coordinative compound), although the 
dictionary entry for that is \rayr{gidj}{giday}. Even though parts of all 
these words are unclear, they all seem to follow the correct syntactic order, 
judging by those parts that are identifiable. And even in the case of 
\rayr{vetjsno}{vetaysano}, which is missing the first part, it can be 
reasonably assumed that the identifiable part, *\rayr{sno}{*sano}, is the 
modifier, and *\rayr{vetj}{vetay} may have once been intended to mean `money' 
or `fee' or something along these lines.

With the exception of \rayr{niNMbkrF}{ningambakar}, all of the mystery words 
were entered into the dictionary in 2006. Digging through old archives and 
translations, I could determine at least that *\rayr{bkrF}{*bakar} was once 
intended to mean `lie', and *\rayr{terj}{*teray} was indeed meant to 
mean `sad'.

\index{compounds|)}

\subsection{Reduplication}
\index{reduplication|(}

\citet{wiltshiremarantz2000} write that it has been suggested that 
reduplication serves an iconic function, 
\textcquote[561]{wiltshiremarantz2000}{with the repetition of phonological 
material indicating a repetition or intensity in the semantics}, so with 
regards to nouns it mainly serves to indicate plurality of various kinds. 
However, they find that in fact, reduplication serves all kinds of functions, 
also ones without iconic meanings, and mention Agta, an Austronesian language 
of the Philippines, which uses reduplication to form diminutives 
\citep[6--9]{healey1960}. As we have seen in \autoref{subsec:reduplication} 
above, so does Ayeri, and it is justified in doing so since there is 
real-world evidence for this use of reduplication. Examples for diminutive 
reduplication in Ayeri include:

\pex
	\a \makebox[7em][l]{\xayr{\larger limu}{limu}{shirt}}
		→ \xayr{\larger limu/limu}{limu-limu}{little shirt}
	\a \makebox[7em][l]{\xayr{\larger nN}{nanga}{house}}
		→ \xayr{\larger nN/nN}{nanga-nanga}{little house}
	\a \makebox[7em][l]{\xayr{\larger spj}{sapay}{hand}}
		→ \xayr{\larger spj/spj}{sapay-sapay}{little hand}
	\a \makebox[7em][l]{\xayr{\larger venej}{veney}{dog}}
		→ \xayr{\larger venej/venej}{veney-veney}{little dog}
\xe

Diminutive reduplication involves full stem reduplication in Ayeri. 
Besides the productive use of reduplication for diminutive marking, there are 
a number of diminutive formations which have been lexicalized, such as in the 
following examples:

\pex
	\a \makebox[8.5em][l]{\xayr{\larger Agu}{agu}{chicken}}
		→ \xayr{\larger Agu/Agu}{agu-agu}{chick}
	\a \makebox[8.5em][l]{\xayr{\larger gnF}{gan}{child}}
		→ \xayr{\larger gnF/gnF}{gan-gan}{grandchild}
	\a \makebox[8.5em][l]{\xayr{\larger psiNF}{pasing}{tube}}
		→ \xayr{\larger psiNF/psiNF}{pasing-pasing}{straw}
	\a \makebox[8.5em][l]{\xayr{\larger poyu}{poyu}{cheek; bacon}}
		→ \xayr{\larger poyu/poyu}{poyu-poyu}{butt}
\xe

There are also at least two documented cases where the reduplicated root is not 
a noun, but the reduplication results in a noun:

\pex
	\a \makebox[10.5em][l]{\xayr{\larger kusNF}{kusang}{double (adj.)}}
		→ \xayr{\larger kusNF/kusNF}{kusang-kusang}{model}
	\a \makebox[10.5em][l]{\xayr{\larger veh/}{veh-}{build}}
		→ \xayr{\larger veh/veh}{veha-veha}{tinkering}
\xe

Reduplicated nouns behave like regular nouns with regards to inflection, that 
is, they receive prefixes and suffixes just like the simplexes from which they 
are derived:

\ex\begingl
	\gla Puco mino \textbf{veney-veneyang}. //
	\glb Puk-yo mino \textbf{veney\til{}veney-ang} //
	\glc jump-\TsgN{} happily \textbf{\Dim{}\til{}dog-\Aarg{}} //
	\glft `The little dog is jumping happily.' //
\endgl\xe

In this example, the reduplicated noun \rayr{venej/venej}{veney-veney} is 
marked as an agent in that the agent suffix \rayr{/ANF}{-ang} is appended to 
the noun as a unit \emph{after} reduplicating the noun stem. In other words, 
the following formation in which the root is reduplicated along with its 
declension suffix is ungrammatical for the purpose of forming a diminutive:

\ex
	*\rayr{\larger veneyNF/veneyNF}{*veneyang-veneyang}
\xe

Likewise, the reduplicated form is not treated in the way an endocentric 
compound would be, so case and plural marking cannot be appended to the first 
element:

\ex
	*\rayr{\larger veneyNF venej}{*veneyang veney}
\xe

While ordinary nouns undergo full reduplication to form a diminutive, in 
compounds, only the head is reduplicated, unless the compound is strongly 
lexicalized or has an idiomatic meaning going beyond that of its components. 
The following example shows the simple case of a transparent endocentric 
compound:

\ex\begingl
	\gla Ya yomayo mehir-mehirang seygo veno kay pang nanga nana. //
	\glb Ya yoma-yo mehir\til{}mehir-ang seygo veno kay pang nanga-Ø nana //
	\glc \LocT{} be-\TsgN{} \Dim{}\til{}tree-\Aarg{} apple pretty three 
		back house-\Top{} \Fpl{}.\Gen{} //
	\glft `There are three pretty little apple trees behind our house.' //
\endgl\xe

In this example, being endearing or otherwise small is treated as a property of 
the head, \xayr{mehirF}{mehir}{tree}, not of the whole compound 
\xayr{mehirFsejgo}{mehirseygo}{apple tree}, or the dependent, 
\xayr{sejgo}{seygo}{apple}---after all, an apple tree which is small is 
rather a small tree with apples on it than a tree with small apples. The 
avoidance of the fully reduplicated form 
\rayr{mehirFsejgo/mehirFsejgo}{mehirseygo-mehirseygo} is probably related to 
the notion of economy of expression.

\index{reduplication|)}

\subsection{Nominalization}
\index{nominalization|(}

Some accidental ways of deriving nouns have been mentioned above, for instance, 
some reduplicated non-nominal roots like \xayr{kusNF}{kusang}{double} or 
\xayr{veh/}{veha-}{build} may form nouns. However, Ayeri also has some 
dedicated morphology to derive nouns from other parts of speech. The most 
common and highly productive way to derive a noun, is the suffix 
\rayr{/AnF}{-an}. The examples in (\ref{ex:vb-nn}) illustrate some derivations 
from verbs, and (\ref{ex:adj-nn}) shows derivations from adjectives to nouns. 
As \xayr{kuhnF}{kuhan}{oar} shows, the nominalization may have an idiomatic 
meaning.

\pex\label{ex:vb-nn}
	\a \makebox[10.5em][l]{\xayr{\larger blNF/}{balang-}{search (v.)}}
		→ \xayr{\larger blNnF}{balangan}{search (n.)}
	\a \makebox[10.5em][l]{\xayr{\larger kuhF/}{kuh-}{row}}
		→ \xayr{\larger kuhnF}{kuhan}{oar}
	\a \makebox[10.5em][l]{\xayr{\larger rigF/}{rig-}{draw}}
		→ \xayr{\larger rignF}{rigan}{drawing}
	\a \makebox[10.5em][l]{\xayr{\larger vehF/}{veh-}{build}}
		→ \xayr{\larger vehnF}{vehan}{building}
\xe

\pex~\label{ex:adj-nn}
	\a \makebox[10.5em][l]{\xayr{\larger Apitu}{apitu}{pure}}
		→ \xayr{\larger Apitu\_an}{apituan}{purity}
	\a \makebox[10.5em][l]{\xayr{\larger gir}{gira}{urgent}}
		→ \xayr{\larger giraanF}{girān}{hurry}
	\a \makebox[10.5em][l]{\xayr{\larger pkisF}{pakis}{serious}}
		→ \xayr{\larger pkisnF}{pakisan}{seriousness}
	\a \makebox[10.5em][l]{\xayr{\larger vp}{vapa}{skillful}}
		→ \xayr{\larger vpn}{vapan}{skill}
\xe

Occasionally, it may even happen that a noun is derived from a noun with a 
related but sometimes more basic meaning using the nominalizer 
\rayr{/AnF}{-an}. 
This process, however, is not productive, so compared to deverbalization and 
deadjectivization, examples of this derivation strategy are few.

\pex\label{ex:nn-nn}
	\a \makebox[8em][l]{\xayr{\larger AgYmF}{ajam}{toy}}
		→ \xayr{\larger AgYmnF}{ajaman}{game}
	\a \makebox[8em][l]{\xayr{\larger kelNF}{kelang}{chain}}
		→ \xayr{\larger kelNnF}{kelangan}{connection}
	\a \makebox[8em][l]{\xayr{\larger nN}{nanga}{house}}
		→ \xayr{\larger nNaanF}{nangān}{household}
	\a \makebox[8em][l]{\xayr{\larger tenF}{ten}{life}}
		→ \xayr{\larger tennF}{tenan}{soul}
\xe

There are also some apparent nominalizations in \rayr{/AmF}{-am} and 
\rayr{/ANF}{-ang}, although these are irregular and non-productive:

\pex
	\a \makebox[8.5em][l]{\xayr{\larger AgY/}{aja-}{play}}
		→ \xayr{\larger AgYmF}{ajam}{toy}
	\a \makebox[8.5em][l]{\xayr{\larger ginF/}{gin-}{drink}}
		→ \xayr{\larger ginmF}{ginam}{glass}
	\a \makebox[8.5em][l]{\xayr{\larger mikF/}{mik-}{poison (v.)}}
		→ \xayr{\larger mikmF}{mikam}{poison (n.), venom}
	\a \makebox[8.5em][l]{\xayr{\larger nun/}{nuna-}{fly}}
		→ \xayr{\larger nunmF}{nunam}{feather}
\xe

\pex~
	\a \makebox[8em][l]{\xayr{\larger bjh/}{bayha-}{rule}}
		→ \xayr{\larger bjhNF}{bayhang}{government}
	\a \makebox[8em][l]{\xayr{\larger hp}{hapa}{remaining}}
		→ \xayr{\larger hpNF}{hapang}{remainder}
	\a \makebox[8em][l]{\xayr{\larger kd/}{kada-}{collect}}
		→ \xayr{\larger kdNF}{kadang}{committee; alliance}
	\a \makebox[8em][l]{\xayr{\larger mim}{mima}{possible}}
		→ \xayr{\larger mimNF}{mimang}{access}
\xe

Agentive nouns can be formed from regular nouns with the suffix 
\rayr{/my}{-maya}, compare the examples in (\ref{ex:mayaregular}). An 
epenthetic /a/ may be introduced to break up consonant clusters that would 
otherwise be either difficult to pronounce or violating phonotactics. When the 
stem of the word the agentive suffix is attached to ends in a consonant or 
/Ca/, it is also often found fused with the root, sometimes with the first /a/ 
of \fw{-Caya} lengthened, see (\ref{ex:mayairregular}). Specifically feminine 
agentive nouns can be derived with the related suffix \rayr{/vy}{-vaya}; two 
examples are given in (\ref{ex:vaya}).

\pex\label{ex:mayaregular}
	\a \makebox[7em][l]{\xayr{\larger AnFlF/}{anl-}{bring}}
		→ \xayr{\larger AnFlmy}{anlamaya}{waiter}
	\a \makebox[7em][l]{\xayr{\larger hor}{hora}{sin}}
		→ \xayr{\larger hormy}{horamaya}{sinner}
	\a \makebox[7em][l]{\xayr{\larger nsY/}{nasy-}{follow}}
		→ \xayr{\larger nsYmy}{nasyamaya}{follower}
	\a \makebox[7em][l]{\xayr{\larger teb/}{teba-}{bake}}
		→ \xayr{\larger tebmy}{tebamaya}{baker}
\xe

\pex~\label{ex:mayairregular}
	\a \makebox[7em][l]{\xayr{\larger As/}{asa-}{travel}}
		→ \xayr{\larger Asaay}{asāya}{traveler}
	\a \makebox[7em][l]{\xayr{\larger IbutF/}{ibut-}{trade}}
		→ \xayr{\larger Ibuty}{ibutaya}{trader, merchant}
	\a \makebox[7em][l]{\xayr{\larger lMtF/}{lant-}{lead}}
		→ \xayr{\larger lMty}{lantaya}{leader; driver}
	\a \makebox[7em][l]{\xayr{\larger tNF/}{tang-}{listen}}
		→ \xayr{\larger tNy}{tangaya}{listener}
\xe

\pex~\label{ex:vaya}
	\a \makebox[7em][l]{\xayr{\larger gnF}{gan}{child}}
		→ \xayr{\larger gnFvy}{ganvaya}{governess}
	\a \makebox[7em][l]{\xayr{\larger lnY}{lanya}{king}}
		→ \xayr{\larger lnFvy}{lanvaya}{queen}
\xe

Besides these, there is also a derivative suffix for makers of things, 
\rayr{/Ati}{-ati} (contracting to /atʃ/ \fw{-ac} before a vowel), though this 
is not too productive, and sometimes irregular, as 
\xayr{sirFtNti}{sirtangati}{youth} shows:

\pex
	\a \makebox[11.5em][l]{\xayr{\larger giMdi}{gindi}{poem}}
		→ \xayr{\larger giMdti}{gindati}{poet}
	\a \makebox[11.5em][l]{\xayr{\larger sirFtNF}{sirtang}{young}}
		→ \xayr{\larger sirFtNti}{sirtangati}{youth}
	\a \makebox[11.5em][l]{\xayr{\larger thnF/}{tahan-}{write}}
		→ \xayr{\larger thnti}{tahanati}{scribe}
	\a \makebox[11.5em][l]{\xayr{\larger vehimF}{vehim}{piece of clothing}}
		→ \xayr{\larger vehimti}{vehimati}{tailor}
\xe

A few instances also exist where a tool of sorts is derived with a suffix 
\rayr{/(E)rYnF}{-(e)ryan}, which is related to the instrumental suffix 
\rayr{/Eri}{-eri} in combination with the nominalizer \rayr{/AnF}{-an}:

\pex
	\a \makebox[9em][l]{\xayr{\larger gurF/}{gur-}{turn}}
		→ \xayr{\larger gurFynF}{guryan}{coil, cylinder}
	\a \makebox[9em][l]{\xayr{\larger misF/}{mis-}{behave}}
		→ \xayr{\larger miserYnF}{miseryan}{method, strategy}
	\a \makebox[9em][l]{\xayr{\larger npF/}{nap-}{burn}}
		→ \xayr{\larger nperYnF}{naperyan}{tinder}
	\a \makebox[9em][l]{\xayr{\larger pr/}{pra-}{glitter, gleam}}
		→ \xayr{\larger pFrrYnF}{praryan}{spark}
\xe

\index{gerund|(}
While \rayr{/AnF}{-an} derives nouns from verbs to produce nouns that act as 
such in every way, it may sometimes be preferable to refer to the action as 
such by a noun, compare in English:

\pex
	\a\label{ex:devnouneng} Manhattan is famous for its tall 
		\textbf{buildings}.
	\a\label{ex:gerundeng} \textbf{Building} a house is an expensive 
		endeavor.
\xe

In (\ref{ex:devnouneng}), \fw{building} is simply a noun derived from the verb 
\fw{build}. It acts as a noun in every way, for example, it can serve as a 
subject and object, it can be pluralized, it can take determiners, and can be 
modified by adjectives. The form of \fw{building} in (\ref{ex:gerundeng}), 
however, is a gerund, and as such underlies the restriction that it cannot be 
pluralized \citep[35]{payne1997}. As we have seen at the beginning of this 
section on nominalization, Ayeri can derive \xayr{vehnF}{vehan}{building, 
construction} from the verb \xayr{vehF/}{veh-}{build}, which acts like every 
other common noun, much like in the English example in (\ref{ex:devnouneng}):

\pex
\a\label{ex:nomz-sbj-adj}\begingl
	\gla Lesāra maritay \textbf{vehānreng} \textbf{tado}. //
	\glb Lesa-ara maritay \textbf{vehān-reng} \textbf{tado} //
	\glc collapse-\TsgI{} about.to \textbf{building-\AargI{}} \textbf{old}//
	\glft `The old building is about to collapse.' //
\endgl

\a\label{ex:nomz-obj-det}\begingl
	\gla Le vacyang \textbf{eda-vehān}. //
	\glb Le vac=yang \textbf{eda=vehān-Ø} //
	\glc \PatTI{} like=\Fsg{}.\Aarg{} \textbf{this=building-\Top{}} //
	\glft `This building, I like it.' //
\endgl

\a\label{ex:nomz-pl-poss}\begingl
	\gla Ang latayo bayhang \textbf{vehānyeley} \textbf{yona}. //
	\glb Ang lata-yo bayhang-Ø \textbf{vehān-ye-ley} \textbf{yona} //
	\glc \AgtT{} sell-\TsgN{} government-\Top{} 
		\textbf{building-\Pl{}-\PargI{}} \textbf{\TsgN{}.\Gen{}} //
	\glft `The government is selling its buildings.' //
\endgl

\a\label{ex:nomz-qty}\begingl
	\gla Le ming kuysāran \textbf{vehān-kay} dirasyam ran. //
	\glb Le ming kuysa-aran \textbf{vehān-Ø=kay} diras-yam ran //
	\glc \PatTI{} can compare-\TplI{} \textbf{building-\Top=few} 
		splendor-\Dat{} \TsgI{}.\Gen{} //
	\glft `Few buildings can compare to its splendor.' //
\endgl
\xe

The above examples condense several properties into one for illustration. Thus, 
(\ref{ex:nomz-sbj-adj}) shows that \rayr{vehaanF}{vehān} can serve as basically 
a subject of a clause, and that it can as well be modified by an 
adjective---the choice of adjectives is not subject to any distributional 
restrictions other than those imposed by the semantic frame of 
\textsc{house}. In the next example, (\ref{ex:nomz-obj-det}), 
\rayr{vehaanF}{vehān} serves as the object of the clause and is being 
determined by the demonstrative prefix \xayr{Ed/}{eda-}{this}. The third 
example, (\ref{ex:nomz-pl-poss}), shows \rayr{vehaanF}{vehān} both pluralized 
and modified by a possessive pronoun, \xayr{yon}{yona}{of it}. And finally, in 
(\ref{ex:nomz-qty}) we see \rayr{vehaanF}{vehān} quantified by the suffix 
\xayr{/kj}{-kay}{few}.

Similar to the English example in (\ref{ex:gerundeng}), Ayeri can also derive 
nouns from the participle of a verb describing the action as such---a gerund. 
For an example, I will again draw on the Ayeri translation of Kafka's short 
story \enquote{Eine kaiserliche Botschaft} \citep[2, 14]{becker:kafka:imperial}:

\ex\label{ex:kafkagerund}\begingl
	\gla … nay ang pətangongva ankyu \textbf{haruyamanas} nanang megayena 
		yana kunangya vana. //
	\glb … nay ang pə-tang-ong=va.Ø ankyu \textbf{haru-yam-an-as} nanang 
		mega-ye-na yana kunang-ya vana //
	\glc … and \AgtT{} \NFut{}-hear-\Irr{}=\Ssg{}.\Top{} truly 
		\textbf{beat-\Ptcp{}-\Nmlz{}-\Parg{}} great fist-\Pl{}-\Gen{} 
		\TsgM{}.\Gen{} door-\Loc{} \Ssg{}.\Gen{} //
	\glft `… and you would indeed hear his magnifcent beating at your door 
		very soon.' //
\endgl\xe

The annotations to this translation contain a comment on the grammatical 
rules which operate in this passage, more specifically also on the gerund 
derivation \xayr{hruymnF}{haruyaman}{beating}:

\blockcquote[14--15]{becker:kafka:imperial}{Furthermore, I wrote 
\fw{haruyaman} `beating' instead of \fw{haruan} `beat(ing)' because I wanted to 
emphasize the process of beating as an incomplete action. This is possible here 
because the word is not topicalized and neither is it marked as a dative, which 
would also require \fw{haruyamanyam} `beat-\Ptcp{}-\Nmlz{}-\Dat{}' to become 
\fw{haruanyam} `beat-\Nmlz{}-\Dat{}' (the participle marker \fw{-yam} is 
derived from the dative case ending \fw{-yam}).}

We can read from this description that the participle marker in Ayeri has 
possibly been grammaticalized from the dative case marker, or that it is at 
least synchronically homonymous. In order for case marking to operate, this 
formation has to be nominalized, which is done the usual way by appending 
\rayr{/AnF}{-an}, thus yielding the suffix cluster \rayr{/ymnF}{-yaman} for the 
derivation of verbs as gerunds. If the gerund is marked for dative case, the 
suffix cluster *\rayr{/ymnFymF}{*-yamanyam} basically undergoes haplology to 
a simple nominalized form with the suffix cluster \rayr{/AnFymF}{-anyam}:

\ex\begingl
	\gla haru- {} haruyam {} haruyaman {} *haruyamanyam {} haruanyam //
	\glb haru- → haru-yam → haru-yam-an → haru-yam-an-yam → 
		haru-an-yam //
	\glc beat {} beat-\Ptcp{} {} beat-\Ptcp{}-\Nmlz{} {} 
		beat-\Ptcp{}-\Nmlz{}-\Dat{} {} beat-\Nmlz{}-\Dat{} //
\endgl\xe

The comment on the translation also makes a little note on the gerund being 
possible because the word is not topicalized. This is based on an old rule that 
gerunds cannot be topicalized unless nominalized first, however, usage has 
since changed so that earlier, \rayr{hruymF}{haruyam} would have constituted 
the gerund form, while even by the time of translating the short story, it had 
changed to \rayr{hruymnF}{haruyaman}. Thus, it is no surprise to see the 
following example, from the partial translation of Saint-Exupéry's story 
\enquote{Le petit prince} \citep[3, 13]{benung:petitprince}:

\ex\label{ex:exuperygerund}\begingl
	\gla Sa koronyang \textbf{palungyaman} na Baysānterpeng nay na Bayokivo 
		menaneri nivānyena. //
	\glb Sa koron=yang \textbf{palung-yam-an-Ø} na Baysānterpeng nay na 
		Bayokivo menan-eri nivān-ye-na //
	\glc \PatT{} knew=\Fsg{}.\Aarg{} 
		\textbf{distinguish-\Ptcp{}-\Nmlz{}-\Top{}} \Gen{} Realm.Middle 
		and \Gen{} Spring.Little first-\Ins{} glimpse-\Pl{}-\Gen{} //
	\glft `I knew how to distinguish between China and Arizona at first 
		sight.' //
\endgl\xe

A more literal translation of this sentence would be `The distinguishing of 
China and Arizona, I knew it at first sight', so the whole passage 
\rayr{pluNFymnF — n byokivo}{palungyaman … na Bayokivo} forms the topic of the 
sentence here, headed by the gerund 
\xayr{pluNFymnF}{palungyaman}{distinguishing}. According to the old rule 
obliquely quoted in the comment to the passage in (\ref{ex:kafkagerund}), this 
should not be possible. As said before, though, usage has changed.

A rule we can gather from the above example from Saint-Exupéry is that gerunds 
are treated as animate nouns. Since they are impersonal, they trigger neuter 
agreement on verbs. They can also be the objects of sentences. The passage in 
(\ref{ex:kafkagerund}) furthermore illustrates that gerunds can be modified by 
The following example shows a gerund used as an agent---basically a 
subject---as well \citep{benung:scientificmethod}:

\ex\label{ex:scimethgerund}\begingl
	\gla \textbf{Dilayamanang} kalamena bahalanas ayonena … //
	\glb \textbf{Dila-yam-an-ang} kalam-ena bahalan-as ayon-ena … //
	\glc \textbf{find.out-\Ptcp{}-\Nmlz{}-\Aarg{}} truth-\Gen{} 
		goal-\Parg{} man-\Gen{} … //
	\glft `(If) finding out the truth is the goal of the man …' //
\endgl\xe

What all the passages on gerunds quoted before show is that gerunds in Ayeri 
do not behave like transitive verbs as in English. Thus, what would be the 
object of the former verb appears in the genitive case in Ayeri. As in English, 
however, gerunds in Ayeri cannot be pluralized:

\ex\ljudge* \begingl
	\gla Noyo \textbf{vehayamanjang} nangayena. //
	\glb Noyo \textbf{veha-yam-an-ye-ang} nanga-ye-na //
	\glc expensive \textbf{build-\Ptcp{}-\Nmlz{}-\Pl{}-\Aarg{}} 
		house-\Pl{}-\Gen{} //
	\glft `*The buildings of houses are expensive.' //
\endgl\xe

It is possible, however, to quantify gerunds insofar as the quantifier does not 
imply countable quantities of the action. Moreover, it is possible for gerunds 
to be modified by possessors. The following to sentences exemplify this use:

\ex\begingl
	\gla Ang lugayan \textbf{delacamanas-ikan} kayanya pang. //
	\glb Ang luga=yan.Ø \textbf{delak-yam-an-as=ikan} kayan-ya pang //
	\glc \AgtT{} go.through=\TplM{}.\Top{} 
		\textbf{suffer-\Ptcp{}-\Nmlz{}-\Parg{}=much} war-\Loc{} after //
	\glft `They went through a lot of suffering after the war.' //
\endgl\xe

\ex~\begingl
	\gla Krico \textbf{malyyamanang} muya \textbf{tan}. //
	\glb Krit-yo \textbf{maly-yam-an-ang} muya \textbf{tan} //
	\glc annoy-\TsgN{} \textbf{sing-\Ptcp{}-\Nmlz{}-\Aarg{}} wrong 
		\textbf{\TplM{}.\Gen{}} //
	\glft `Their off singing is annoying.' //
\endgl\xe

\index{gerund|)}
\index{nominalization|)}

\index{nouns|)}

\section{Pronouns}
\index{pronouns|(}

Ayeri possesses different kinds of pronouns in the sense that there is a closed 
class of words which contains anaphora of various types---personal pronouns, 
demonstrative pronouns, interrogative pronouns, relative pronouns, as 
well as reflexive and reciprocal expressions. Each class of pronouns will be 
discussed in the following.

\subsection{Personal pronouns}
\label{subsec:perspro}
\index{pronouns!personal|(}

\begin{figure}[tp]\centering
\caption{Personal pronouns}

\begin{tabu} to \linewidth{S X[c] X[c] X[c] X[c] X[c] X[c] X[c] X[c]}
\tableheaderfont\toprule
Person
	& \Top{}
	& \Aarg{}
	& \Parg{}
	& \Dat{}
	& \Gen{}
	& \Loc{}
	& \Caus{}
	& \Ins{}
	\\
\toprule

\Fsg{}
	& ay	% \Top{}
	& yang	% \Aarg{}
	& yas	% \Parg{}
	& yām	% \Dat{}
	& nā	% \Gen{}
	& yā	% \Loc{}
	& sā	% \Caus{}
	& rī	% \Ins{}
	\\
	
\midrule

\Ssg{}
	& va	% \Top{}
	& vāng	% \Aarg{}
	& vās	% \Parg{}
	& vayam	% \Dat{}
	& vana	% \Gen{}
	& vaya	% \Loc{}
	& vasa	% \Caus{}
	& vari	% \Ins{}
	\\

\midrule

\TsgM{}
	& ya	% \Top{}
	& yāng	% \Aarg{}
	& yās	% \Parg{}
	& yayam	% \Dat{}
	& yana	% \Gen{}
	& yāy	% \Loc{}
	& yasa	% \Caus{}
	& yari	% \Ins{}
	\\

\TsgF{}
	& ye	% \Top{}
	& yeng	% \Aarg{}
	& yes	% \Parg{}
	& yeyam	% \Dat{}
	& yena	% \Gen{}
	& yea	% \Loc{}
	& yesa	% \Caus{}
	& yeri	% \Ins{}
	\\

\TsgN{}
	& yo	% \Top{}
	& yong	% \Aarg{}
	& yos	% \Parg{}
	& yoyam	% \Dat{}
	& yona	% \Gen{}
	& yoa	% \Loc{}
	& yosa	% \Caus{}
	& yori	% \Ins{}
	\\

\TsgI{}
	& ra	% \Top{}
	& reng	% \Aarg{}
	& rey	% \Parg{}
	& rayam	% \Dat{}
	& ran	% \Gen{}
	& raya	% \Loc{}
	& rasa	% \Caus{}
	& rari	% \Ins{}
	\\

\midrule

\Fpl{}
	& ayn	% \Top{}
	& nang	% \Aarg{}
	& nas	% \Parg{}
	& nyam	% \Dat{}
	& nana	% \Gen{}
	& nyā	% \Loc{}
	& nisa	% \Caus{}
	& ni	% \Ins{}
	\\
	
\midrule

\Spl{}
	& va	% \Top{}
	& vāng	% \Aarg{}
	& vās	% \Parg{}
	& vayam	% \Dat{}
	& vana	% \Gen{}
	& vaya	% \Loc{}
	& vasa	% \Caus{}
	& vari	% \Ins{}
	\\

\midrule

\TplM{}
	& yan	% \Top{}
	& tang	% \Aarg{}
	& tas	% \Parg{}
	& cam	% \Dat{}
	& tan	% \Gen{}
	& ca	% \Loc{}
	& tis	% \Caus{}
	& ti	% \Ins{}
	\\

\TplF{}
	& yen	% \Top{}
	& teng	% \Aarg{}
	& tes	% \Parg{}
	& teyam	% \Dat{}
	& ten	% \Gen{}
	& teya	% \Loc{}
	& tēs	% \Caus{}
	& teri	% \Ins{}
	\\

\TplN{}
	& yon	% \Top{}
	& tong	% \Aarg{}
	& tos	% \Parg{}
	& toyam	% \Dat{}
	& ton	% \Gen{}
	& toya	% \Loc{}
	& tōs	% \Caus{}
	& tori	% \Ins{}
	\\

\TplI{}
	& ran	% \Top{}
	& teng	% \Aarg{}
	& tey	% \Parg{}
	& racam	% \Dat{}
	& ten	% \Gen{}
	& raca	% \Loc{}
	& ratas	% \Caus{}
	& ray	% \Ins{}
	\\

\bottomrule
\end{tabu}
\label{fig:perspro}
\end{figure}

As \autoref{fig:perspro} shows, Ayeri possesses quite a large number of 
personal pronouns with little syncretism between the different paradigm 
slots overall (the second person is a notable exception); there are also no 
gaps in the paradigm. Ayeri's personal pronouns reflect the grammatical 
features also found in nouns, that is, number, gender, and case, and person is 
added to that. The individual forms range from completely fused to fully 
transparent even within the same case paradigm, for instance, 
\xayr{yaamF}{yām}{(to/for) me} (\Fsg{}.\Dat{}) on the one hand, and 
\xayr{yymF}{yayam}{(to/for) him} (transparently \TsgM{}-\Dat{}) on the other. 
Originally, all pronouns have been regular formations based on the respective 
unmarked pronominal element listed in the \Top{} column of 
\autoref{fig:perspro} declined by adding a case suffix (see 
\autoref{subsec:case}). Use has caused many of these formations to contract and 
erode as grammaticalization progressed:

\pex
\a\begingl
	\gla ayang → yāng //
	\glb ay-ang {} yāng //
	\glc \makebox[\widthof{\Tsg{}-\M{}-\Pl{}-\Gen{}}][l]{\Fsg{}-\Aarg{}} {} 
		\Fsg{}.\Aarg{} //
\endgl

\a\begingl
	\gla iyatena → tan //
	\glb iy-a-t-ena {} tan //
	\glc \Tsg{}-\M{}-\Pl{}-\Gen{} {} \TsgM{}.\Gen{}\footnotemark //
\endgl
\xe

\footnotetext{Strictly speaking, this could as well be glossed as \fw{t<a>n} 
(\Tsg{}.\Gen{}<\M{}>). I chose to gloss the pronoun in the above way, however, 
in order to not overly complicate things.}

The plural series used to be derived by adding \rayr{/nF}{-n} or, in the third 
person, \rayr{/tF/}{\mbox{-t-}} to the pronoun stem, which can still be easily 
observed in the unmarked pronouns as well as in the alternation between 
\rayr{yF/}{y-} and \rayr{tF/}{t-} in the third person pronouns. The same goes 
for the gender-marking thematic vowel in the animate third person pronouns, 
which has been retained as a distinctive feature even in the non-core pronouns 
despite sometimes heavy modifications. A further interesting property of Ayeri 
is that synchronically, singular and plural are distinguished, except for the 
second person, where the forms are the same, basically like in English. 
\citet{lehmann2015} explains, however, that this is not an unusual route for 
languages to take:

\blockcquote[42]{lehmann2015}{New pronouns, especially for the second person 
singular, are often obtained by shifting pronouns around in the paradigm, 
especially by substituting marked forms for unmarked ones. This explains, for 
instance, 
the use of [...] English \fw{you} for the second person singular}

The second person singular subject pronoun of English used to be \fw{thou}, 
cognate to German \fw{du}, which can still be found in Shakespeare, for 
instance. Something along the lines of English \fw{you} as a second 
person plural pronoun replacing second person singular \fw{thou} by way of a 
deferential singular use of a plural pronoun \citep[you, pron., adj., and 
n.]{oed} may have happened in Ayeri as well.

The personal pronouns are used in just the same way as their full-NP 
counterparts would be, also in the non-core cases:

\pex\label{ex:perspro}
\a\label{ex:pronfull}\begingl
	\gla Sa harya ang Paradan tandās kaleri. //
	\glb Sa har-ya ang Paradan tanda-as kal-eri //
	\glc \AgtT{} beat-\TsgM{} \Aarg{} Paradan fly-\Parg{} rag-\Ins{} //
	\glft `Paradan beats the fly with a rag.' //
\endgl

\a\label{ex:pronagt}\begingl
	\gla Sa haryāng tandās kaleri. //
	\glb Sa har=yāng tanda-as kal-eri //
	\glc \AgtT{} beat=\TsgM{}.\Aarg{} fly-\Parg{} rag-\Ins{} //
	\glft `He beats the fly with a rag.' //
\endgl

\a\label{ex:pronpat}\begingl
	\gla Sa harya ang Paradan yos kaleri. //
	\glb Sa har-ya ang Paradan yos kal-eri //
	\glc \AgtT{} beat-\TsgM{} \Aarg{} Paradan \TsgN{}.\Parg{} rag-\Ins{} //
	\glft `Paradan beats it with a rag.' //
\endgl

\a\label{ex:pronins}\begingl
	\gla Sa harya ang Paradan tandās rari. //
	\glb Sa har-ya ang Paradan tanda-as rari //
	\glc \AgtT{} beat-\TsgM{} \Aarg{} Paradan fly-\Parg{} \TsgI{}.\Ins{} //
	\glft `Paradan beats the fly with it.' //
\endgl

\xe

In the above set of examples, (\ref{ex:pronfull}) shows a sentence with full 
NPs, with the agent, \rayr{ANF prdnF}{ang Paradan} replaced by the third person 
singular masculine agent pronoun \rayr{yaaNF}{yāng} in (\ref{ex:pronagt}); in 
(\ref{ex:pronpat}) the patient, \rayr{tMdaasF}{tandās}, is replaced with the 
third person singular neuter patient pronoun \rayr{yosF}{yos}; in 
(\ref{ex:pronins}), lastly, the instrument, \rayr{kleri}{kaleri} is replaced 
with the third person singular inanimate instrumental pronoun \rayr{rri}{rari}.
Furthermore, complex NPs are in complementary distribution, that is, an NP 
which contains an adjective is wholly replaced by an NP containing a 
personal pronoun:

\pex\label{ex:procompldist}
\a\begingl
	\gla Ang ninye vehimley veno. //
	\glb Ang nin=ye.Ø vehim-ley veno //
	\glc \Aarg{} wear=\TsgF{}.\Top{} dress-\PargI{} beautiful //
	\glft `She wears a beautiful dress.' //
\endgl

\a\ljudge* \begingl
	\gla Ang ninye adaley veno. //
	\glb Ang nin=ye.Ø ada-ley veno //
	\glc \Aarg{} wear=\TsgF{}.\Top{} that-\PargI{} beautiful //
	\glft `*She wears a beautiful it.' //
\endgl

\a\begingl
	\gla Ang ninye adaley. //
	\glb Ang nin=ye.Ø ada-ley //
	\glc \Aarg{} wear=\TsgF{}.\Top{} that-\PargI{} //
	\glft `She wears it.' //
\endgl

\xe

Comparing the example sentences in (\ref{ex:perspro}) with the \Top{} column in 
\autoref{fig:perspro} an important property of personal pronouns becomes 
apparent. That is, the `unmarked' (or rather, zero-marked) pronoun forms are 
also the ones showing as verb agreement. An important difference in this 
respect, however, is that the third person singular inanimate verb agreement 
marker is not \rayr{/r}{-ra}, but \rayr{/Ar}{-ara}. The following two examples 
illustrate the parallel more clearly---observe the person marking on the verb 
in (\ref{ex:verbinfl1}) and the corresponding object pronouns in 
(\ref{ex:verbinfl2}):

\pex\label{ex:verbinfl1}
\a\begingl
	\gla Sa man\textbf{ya} ang Ajān {} Pila. //
	\glb Sa man\textbf{-ya} ang ​Ajān Ø ​Pila //
	\glc \PatT{} greet\textbf{-\TsgM{}} \Aarg{} ​Ajān \Top{} ​Pila //
	\glft `Pila, Ajān greets her.' //
\endgl

\a\begingl
	\gla Sa man\textbf{ye} ang Pila {} Ajān. //
	\glb Sa man\textbf{-ye} ang Pila Ø ​Ajān //
	\glc \PatT{} greet\textbf{-\TsgF{}} \Aarg{} Pila \Top{} ​Ajān //
	\glft `Ajān, she greets him.' //
\endgl

\xe

\pex~\label{ex:verbinfl2}
\a\begingl
	\gla Sa manye ang Pila \textbf{ya}. //
	\glb Sa man-ye ang Pila \textbf{ya.Ø} //
	\glc \PatT{} greet-\TsgF{} \Aarg{} Pila \textbf{\TsgM{}.\Top{}} //
	\glft `Pila greets him.' //
\endgl

\a\begingl
	\gla Sa manya ang Ajān \textbf{ye}. //
	\glb Sa man-ya ang ​Ajān \textbf{ye.Ø} //
	\glc \PatT{} greet-\TsgM{} \Aarg{} ​Ajān \textbf{\TsgF{}.\Top{}} //
	\glft `Ajān greets her.' //
\endgl

\xe

Another important property of both pronouns and verbs is that agent pronouns 
(and patient pronouns under certain circumstances) replace person agreement by 
cliticizing to the verb stem. As person agreement morphology is a domain of 
verbs, it will be dealt with in more detail in the chapter on verbs proper. The 
following example illustrates the mainly relevant process, however:

\pex
\a\begingl
	\gla Sa man\textbf{ya} \textbf{ang} \textbf{Ajān} {} Pila. //
	\glb Sa man\textbf{-ya} \textbf{ang} \textbf{​Ajān} Ø ​Pila //
	\glc \PatT{} greet\textbf{-\TsgM{}} \textbf{\Aarg{}} \textbf{​Ajān} 
		\Top{} ​Pila //
	\glft `Pila, Ajān greets her.' //
\endgl

\a\begingl
	\gla Sa man\textbf{yāng} {} Pila. //
	\glb Sa man\textbf{=yāng} Ø ​Pila //
	\glc \PatT{} greet\textbf{=\TsgM{}.\Aarg{}} \Top{} ​Pila //
	\glft `Pila, he greets her.' //
\endgl
\xe

\index{pronouns!personal|)}

% \index{pronouns!possessive|(}
% 
% Possessive pronouns are special compared to regular personal pronouns in that 
% they can act as both personal pronouns proper and possessive adjectives, 
% depending on context. The main case for the pronouns listed above in the 
% genitive column of \autoref{fig:perspro} is that of possessive adjectives, 
% which means that unlike personal pronouns, they do not replace NPs fully 
% (compare (\ref{ex:procompldist})), but can be used as modifiers like, or 
% alongside, adjectives:
% 
% \ex\begingl
% 	\gla nangaya ledo nā //
% 	\glb nanga-ya ledo nā //
% 	\glc house-\Loc{} blue \Fsg{}.\Gen{} //
% 	\glft `in my blue house' //
% \endgl\xe
% 
% \index{pronouns!possessive|)}

\subsection{Demonstrative pronouns}
\label{subsec:dempro}
\index{pronouns!demonstrative|(}

\begin{figure}[tp]\centering
\caption{Demonstrative pronouns}

\begin{tabu} to .75\linewidth{S[2] X[4c] X[4c] X[4c]}
\tableheaderfont\toprule

Case
	& Proximal
	& Distal
	& Indefinite
	\\
\toprule

\Top{}
	& edanya
	& adanya
	& danya
	\\
	
\midrule
	
\Aarg{}
	& edanyāng
	& adanyāng
	& \emph{danyāng}
	\\

\Aarg{}.\Inan{}
	& edareng, \emph{edanyareng}
	& adareng, adanyareng
	& \emph{danyareng}
	\\
	
\Parg{}
	& edanyās
	& adanyās
	& danyās
	\\

\Parg{}.\Inan{}
	& edaley
	& \emph{adaley}
	& danyaley
	\\

\Dat{}
	& \emph{edayam}
	& adayam
	& \emph{danyayam}
	\\

\midrule

\Gen{}
	& edanyana
	& adanyana
	& danyana
	\\
	
\Loc{}
	& \emph{edanyaya}
	& adanyaya
	& \emph{danyaya}
	\\
	
\Caus{}
	& \emph{edanyasa}
	& \emph{adanyasa}
	& \emph{danyasa}
	\\
	
\Ins{}
	& \emph{edanyari}
	& \emph{adanyari}
	& \emph{danyari}
	\\

\bottomrule
\end{tabu}
\label{fig:detpro}
\end{figure}

Demonstrative pronouns in Ayeri are formed with the demonstrative 
prefixes: \xayr{Ed/}{eda-}{this} (proximal), \xayr{Ad/}{ada-}{that} 
(distal), and \xayr{d/}{da-}{such} (indefinite). These are combined with a 
morpheme \rayr{nY}{nya}, which is related to the word for `person', 
\rayr{nYaanF}{nyān}. \autoref{fig:detpro} gives the declined forms for all of 
them. Those forms attested in the corpus gathered from dictionary entries and 
example texts also used for the syllable structure analyses in 
\autoref{sec:phonotactics} appear in upright type, those that should be 
grammatical as well otherwise are given in italic type. The corpus is very 
small, but the prevalence of some forms is possibly reflecting varying degrees 
of grammaticalization at least to some extent. \autoref{tab:detprontokenfq} 
gives the token frequencies of the various attested forms.

\begin{table}[tp]\centering
\caption{Token frequencies of attested demonstrative pronouns}

\begin{tabu} to .75\linewidth {>{\itshape}X[2l] X[2l] X[1c] X[1c]}
\tableheaderfont\toprule

Pronoun
	& Gloss
	& Absolute
	& Relative
	\\

\toprule

edanya
	& this.\Top{}
	& 1
	& 1.69\pct
	\\

adanya
	& that.\Top{}
	& 9
	& 15.25\pct
	\\

danya
	& such.\Top{}
	& 1
	& 1.69\pct
	\\

\midrule

edanyāng
	& this.\Aarg{}
	& 4
	& 6.78\pct
	\\

adanyāng
	& that.\Aarg{}
	& 8
	& 13.56\pct
	\\

edareng
	& this.\AargI{}
	& 3
	& 5.08\pct
	\\

adareng
	& that.\AargI{}
	& 15
	& 25.42\pct
	\\

adanyareng
	& that.\AargI{}
	& 1
	& 1.69\pct
	\\

\midrule

edanyās
	& this.\Parg{}
	& 1
	& 1.69\pct
	\\

adanyās
	& that.\Parg{}
	& 2
	& 3.39\pct
	\\

danyās
	& such.\Parg{}
	& 2
	& 3.39\pct
	\\

edaley
	& this.\PargI{}
	& 2
	& 3.39\pct
	\\

danyaley
	& such.\PargI{}
	& 2
	& 3.39\pct
	\\

\midrule

adayam
	& that.\Dat{}
	& 3
	& 5.08\pct
	\\

\midrule

edanyana
	& this.\Gen{}
	& 1
	& 1.69\pct
	\\

adanyana
	& that.\Gen{}
	& 2
	& 3.39\pct
	\\

danyana
	& such.\Gen{}
	& 1
	& 1.69\pct
	\\

\midrule

adanyaya
	& that.\Loc{}
	& 1
	& 1.69\pct
	\\

\bottomrule

\textup{Total}
	& 
	& 59
	& 100\pct
	\\

\bottomrule
\end{tabu}
\label{tab:detprontokenfq}
\end{table}

Of all the cases, the agent demonstratives have the highest token frequency at 
a combined 52.53\pct{}, especially the distal pronouns are very frequent in the 
sample. 
Moreover, the distal inanimate agent demonstative occurs twice as often as its 
animate counterpart, the shortened form \xayr{AdreNF}{adareng}{that (one)} 
being far more current than the full form \rayr{AdnYreNF}{adanyareng}. 
Interestingly, the shortened form \xayr{EdreNF}{edareng}{this one} is also the 
only one attested for the inanimate proximate agent; similarly, the only dative 
demonstrative attested once is shortened as well: \xayr{AdymF}{adayam}{(to/for) 
that}. For non-core cases, only `long' demonstratives are attested, albeit 
sparingly so.

Regarding the variation between `long' and `short' forms, it is not surprising 
that those demonstratives with a high frequency of use are eroded in some way: 
it seems that Ayeri prefers them to stay trisyllabic, which is achieved by 
dropping the \rayr{nY}{nya} part.\footnote{According to the so-called Zipf's 
law, word length and token frequency correlate in that the most frequently used 
words in a language also tend to be the shortest \citep[25--27]{zipf1935}.} A 
further reason for dropping the \rayr{nY}{nya} part especially in the inanimate 
demonstratives may be that it is perceived as a marker of animacy---it has been 
noted above already that it is related to the word \xayr{nYaanF}{nyān}{person}. 
Both factors, high frequency and semantic mismatch, may thus promote 
contraction.

Still, the question for the reason for the high frequency especially of 
\rayr{AdreNF}{adareng} remains open. It may be explained by looking at a few 
typical examples of this word in context, however.

\pex\label{ex:demexpl}
\a\begingl
	\gla Nay ang nelyo-ikan sungkorankihas, adareng tono. //
	\glb Nay ang nel-yo=ikan sungkorankihas, ada-reng tono //
	\glc and \AgtT{} help-\TsgN{}=much geography, that-\AargI{} certain //
	\glft `And geography, that's for sure, helped me a lot.'%
		\tc{\citep[13]{benung:petitprince}} //
\endgl

\a\begingl
	\gla Adareng merambay-ikan, le sundalvāng sasān vana ... //
	\glb Ada-reng merambay=ikan, le sundal=vāng sasān-Ø vana ... //
	\glc that-\AargI{} useful=very, \PatTI{} lose=\Ssg{}.\Aarg{} way-\Top{} 
		\Ssg{}.\Gen{} ... //
	\glft `It’s very useful if you get lost [...]'%
		\tc{\citep[14]{benung:petitprince}} //
\endgl

\a\begingl
	\gla Adareng danyaley segasena boa tinka. //
	\glb Ada-reng danya-ley segas-ena boa tinka //
	\glc that-\AargI{} such-\PargI{} snake-\Gen{} boa closed //
	\glft `The one of the closed boa snake.'\footnotemark%
		\tc{\citep[22]{benung:petitprince}} //
\endgl

\xe

\footnotetext{More literal translations of this sentence are `That is the one 
of the closed boa snake' or `That is one of a closed boa snake'.}

In all of the example sentences in (\ref{ex:demexpl}), 
\xayr{AdreNF}{adareng}{that (one)} serves as a dummy pronoun together with a 
predicative adjective or NP, which is the main reason why it occurs so 
frequently. This is to say, Ayeri prefers the demonstrative pronoun 
\rayr{AdreNF}{adareng} as the dummy agent in predicative contexts over the 
personal pronoun \xayr{reNF}{reng}{it}. Otherwise, however, demonstrative 
pronouns work regularly as deictic anaphora: `this', `that', and `such (a)', 
except that as nominal elements they are declined for case---but not for number 
or animacy, which is a notable difference between demonstrative pronouns and 
personal pronouns:

\pex
\a\begingl
	\gla Ang vehya {} Ajān nangās. //
	\glb Ang veh-ya Ø Ajān nanga-as //
	\glc \AgtT{} build-\TsgM{} \Top{} Ajān house-\Parg{} //
	\glft `Ajān builds a house.' //
\endgl

\a\begingl
	\gla Nangās? Sa vehyāng may danya. //
	\glb Nanga-as? Sa veh=yāng may danya-Ø //
	\glc house-\Parg{}? \PatT{} build=\TsgM{}.\Aarg{} \Aff{} such-\Top{} //
	\glft `A house? He builds one indeed.' //
\endgl

\xe

\pex~
\a\begingl
	\gla Sā hasuyeng eda-migorayye. //
	\glb Sā hasu=yeng eda=migoray-ye-Ø //
	\glc \CauT{} sneeze=\TsgF{}.\Aarg{} this=flower-\Pl{}-\Top{} //
	\glft `These flowers make her sneeze.' //
\endgl

\a\begingl
	\gla Ang tipinyon nivaye yena adanyari naynay. //
	\glb Ang tipin-yon niva-ye-Ø yena adanya-ri naynay //
	\glc \AgtT{} itch-\TplN{} eye-\Pl{}-\Top{} \TsgF{}.\Gen{} that-\Caus{} 
		as.well //
	\glft `Her eyes are itching due to that/them/those [the flowers] as 
		well.' //
\endgl
\xe

As mentioned in the previous chapter (p.~\pageref{nounprefixes}), the prefix 
\xayr{d/}{da-}{such, so} can combine with a range of syntactic phrase types, 
but most notably NPs, to serve as an indefinite demonstrative:

\ex\begingl
	\gla Adareng da-dipakanas. //
	\glb Adareng da=dipakan-as //
	\glc that-\AargI{} such=pity-\Parg{} //
	\glft `That is such a pity.' //
\endgl\xe

\rayr{d/}{da-} can be used to express English `one' in the sense of a deictic 
anaphora as well. Thus, to express `the X one', if X is an adjective, it is 
strictly speaking necessary to use the full demonstrative pronoun, 
\rayr{dnY}{danya}, since adjectives do not decline, and Ayeri largely avoids 
undeclined NPs:\footnote{See \autoref{subsec:uncased} above for examples of 
situations where nouns regularly do not exhibit case marking.}

\pex
\a\begingl
	\gla Silvyo danyāng kivo ku-mino-ing. //
	\glb Silv-yo danya-ang kivo ku=mino=ing //
	\glc look-\TsgN{} such-\Aarg{} little like=happy=so //
	\glft `The little one looks so happy.' //
\endgl

\a\label{ex:danyatop}\begingl
	\gla Sa noyang danya tuvo. //
	\glb Sa no=yang danya-Ø tuvo //
	\glc \PatT{} want=\Fsg{}.\Aarg{} such-\Top{} red //
	\glft `I want the red one.' //
\endgl

\xe

Nonetheless, in cases like (\ref{ex:danyatop}) where the demonstrative is 
topicalized, the prefixed form may be used, which is possible since 
\rayr{d/}{da-} is a clitic that binds to NPs, rather than nouns. As we have 
seen before, NPs do not exhibit overt case marking if topicalized, so whether 
\rayr{d/}{da-} leans on a superficially unmarked noun or an adjective, 
which is always unmarked for case, does not matter, since both are NPs. The 
sentence presented in (\ref{ex:danyatop}) is thus rather formal; less formally, 
the following is acceptable as well:

\ex\label{ex:redone}\begingl
	\gla Sa noyang da-tuvo. //
	\glb Sa no=yang da=tuvo.Ø //
	\glc \PatT{} want=\Fsg{}.\Aarg{} such=red.\Top{} //
	\glft `I want the red one.' //
\endgl\xe

\index{pronouns!demonstrative|)}

\subsection{Interrogative pronouns}
\index{pronouns!interrogative|(}

The intererrogative pronouns are all formed with \rayr{si/}{si}, combined with 
a lexical element or a case marker; \rayr{si/}{si} is also related to the 
relativizer \rayr{si}{si}. The interrogative pronouns are listed in 
\autoref{fig:interpro}.

\begin{figure}[htp]\centering
\caption{Interrogative pronouns}
\begin{tabu} to \linewidth {X[3] X[9] X[8]}
\tableheaderfont\toprule
Pronoun
	& Literal meaning
	& Idiomatic meaning
	\\

\toprule

\rayr{sinY}{sinya}
	& which one (\xayr{nYaanF}{nyān}{person})
	& `who', `what', `which'
	\\

\midrule

\rayr{siknF}{sikan}
	& how much (\xayr{IknF}{ikan}{much})
	& `how much', `how many'
	\\

\rayr{sikj}{sikay}
	& with what (\xayr{kjvo}{kayvo}{with})
	& `how' (tool, circumstance)
	\\

\rayr{siminF}{simin}
	& which way (\xayr{mirnF}{miran}{way})
	& `how' (way, procedure)
	\\

\rayr{sitdj}{sitaday}
	& which time (\xayr{tdj}{taday}{time})
	& `when'
	\\

\rayr{siynF}{siyan}
	& which place (\xayr{yno}{yano}{place})
	& `where'
	\\

\bottomrule
\end{tabu}
\label{fig:interpro}
\end{figure}

A property which all interrogative pronouns share is that they are placed 
\fw{in 
situ}. That is, they appear in the same position as the phrase they stand in 
for, so there will not be movement of the question word to the front as in 
English. Additionally, impersonal interrogative pronouns cannot be topicalized 
since they also do not inflect for case, which preempts the difference between 
zero-marked topicalized and overtly case-marked untopicalized forms.

\pex
\a\begingl
	\gla Sa petigavāng inun sikan? //
	\glb Sa petiga=vāng inun-Ø sikan //
	\glc \PatT{} catch=\Ssg{}.\Aarg{} fish-\Top{} how.much //
	\glft `How much fish did you catch?' //
\endgl

\a\begingl
	\gla Sa-sahavāng sitaday? //
	\glb Sa\til{}saha=vāng sitaday //
	\glc \Iter{}\til{}come=\Ssg{}.\Aarg{} when //
	\glft `When will you return?' //
\endgl
\xe

In the table on interrogative pronouns above, \xayr{sinY}{sinya}{who, what, 
which} is seperated from the other pronouns because it behaves differently. 
Namely, it can be declined for all cases according to the syntactic or semantic 
role of the NP it replaces, and it can and will often be topicalized: what you 
query about will likely constitute the topic of the sentence and the answer.

\pex
\a\begingl
	\gla Ang yomayo sinya adaya?\footnotemark //
	\glb Ang yoma-yo sinya-Ø adaya //
	\glc \AgtT{} exist-\TsgN{} who-\Top{} there //
	\glft `Who is there?' //
\endgl

\a\begingl
	\gla Sa narayeng sinya? //
	\glb Sa nara=yeng sinya-Ø //
	\glc \PatT{} say=\TsgF{}.\Aarg{} what-\Top{} //
	\glft `What did she say?' //
\endgl

\xe

\footnotetext{This may be shortened to just \xayr{sinYaaNF Ady?}{sinyāng 
adaya?}{who (is) there?} (who-\Aarg{} there).}

\begin{figure}[tp]\centering
\caption{Declension paradigm for \xayr{sinY}{sinya}{who, what}}
\begin{tabu} to \linewidth {X[1] X[3] X[8]}
\tableheaderfont\toprule
Case
	& Pronoun
	& Translation
	\\

\toprule

\Top{}
	& \rayr{sinY}{sinya}
	& `who', `what'
	\\

\midrule

\Aarg{}
	& \rayr{sinYaaNF}{sinyāng}
	& `who', `what'
	\\

\AargI{}
	& \rayr{sinYreNF}{sinyareng}
	& `who', `what'
	\\
\Parg{}
	& \rayr{sinYaasF}{sinyās}
	& `whom', `what'
	\\
\PargI{}
	& \rayr{sinYlej}{sinyaley}
	& `whom', `what'
	\\
\Dat{}
	& \rayr{sinYymF}{sinyayam}
	& `for/to whom', `for/to what'
	\\

\midrule

\Gen{}
	& \rayr{sinYn}{sinyana}
	& `whose', `from whom', `from what'
	\\

\Loc{}
	& \rayr{sinYy}{sinyaya}
	& `in/at/on whom', `in/at/on what'
	\\

\Caus{}
	& \rayr{sinYis}{sinyisa}
	& `due to/because of whom', `due to/because of what'
	\\

\Ins{}
	& \rayr{sinYri}{sinyari}
	& `by whose help', `with what'
	\\

\bottomrule
\end{tabu}
\label{fig:sinya}
\end{figure}

Ayeri does not strictly distinguish animate and inanimate entities in its 
interrogative pronouns, so there is no distinction between `who' and `what'. 
\rayr{sinY}{sinya} and/or the verb will simply inflect according to context and 
to the speaker's expectations or knowledge (compare \autoref{fig:sinya}). Thus, 
there is also no dedicated question word for `why', since in Ayeri one can 
simply ask `due to what/whom' by inflecting \rayr{sinY}{sinya}:

\pex
\a\begingl
	\gla Le kayāng adanya sinyayam? //
	\glb Le ka=yāng adanya-Ø sinya-yam //
	\glc \PatTI{} throw.away=\TsgM{}.\Aarg{} that-\Top{} what-\Dat{} //
	\glft `Why (= what for) did he throw that away?' //
\endgl

\a\begingl
	\gla Ang prantoyva sinyaisa? //
	\glb Ang prant-oy=va.Ø sinya-isa //
	\glc \AgtT{} ask-\Neg{}=\Ssg{}.\Top{} what-\Caus{} //
	\glft `Why (= because of what) did you not ask?' //
\endgl

\xe

While there is no dedicated `why', Ayeri distinguishes between two kinds of 
`how': \rayr{siminF}{simin} asks about the way by which---or the circumstances 
under which---an action is carried out, whereas \rayr{sikj}{sikay} asks for the 
means or tools used to carry out an action:

\pex
\a\label{ex:simin}\begingl
	\gla Le tiyavāng vadisān simin? //
	\glb Le tiya=vāng vadisān-Ø simin //
	\glc \PatTI{} make=\Ssg{}.\Aarg{} bread-\Top{} how //
	\glft `How do you make bread?' //
\endgl

\a\label{ex:sikay}\begingl
	\gla Le peralvāng sagan sikay? //
	\glb Le peral=vāng sagan-Ø sikay //
	\glc \PatTI{} grind=\Ssg{}.\Aarg{} flour-\Top{} how //
	\glft `How do you grind flour?' //
\endgl

\xe

The correct answer to the question in (\ref{ex:simin}) needs to treat the 
process of making bread, since \rayr{siminF}{simin} asks about the way; a 
correct answer to the question in (\ref{ex:sikay}), on the other hand, will 
likely mention grinding utensils, like a mill or a pestle. Even though 
Ayeri possesses an instrumental case which can be used in a comitative way, 
note the conflation of that and the preposition of accompaniment, 
\rayr{kjvo}{kayvo}, in this case (see \autoref{subsubsec:instrumental}).

Comparing Tables \ref{fig:interpro} and \ref{fig:sinya}, it may strike the 
reader's eye that there are two possbilities to express 
`where'---lexical \rayr{siynF}{siyan} and synthetic \rayr{sinYy}{sinyaya}. It 
is important to note, however, that these are not strictly interchangeable, 
even though some variation is to be expected. While \rayr{siynF}{siyan} refers 
to \emph{places} in general, the \rayr{sinY}{sinya} series refers to 
\emph{entities} both animate and inanimate more specifically:

\pex
\a\begingl
	\gla Saravāng siyan? --- Ya Sikatay. //
	\glb Sara=vāng siyan --- Ya Sikatay //
	\glc go=\Ssg{}.\Aarg{} where --- \Loc{} Sikatay //
	\glft `\,\enquote{Where are you going?}---\enquote{To Sikatay.}\,' //
\endgl

\a\begingl
	\gla Ya divvāng sinya? --- Ya Haki. //
	\glb Ya div=vāng sinya-Ø --- Ya Haki //
	\glc \LocT{} stay=\Ssg{}.\Aarg{} who-\Top{} --- \Loc{} Haki //
	\glft `\,\enquote{Where are you staying?}---\enquote{At Haki's}\,' //
\endgl

\xe

\index{pronouns!interrogative|)}

\subsection{Indefinite pronouns}
\index{pronouns!indefinite|(}

\citet[56]{haspelmath1997} notes how descriptions of languages often do not 
document indefinite pronouns---whether they simply do not exist in this 
language or whether they escaped the author's attention remains unknown in 
these cases. It may thus be duly noted here that Ayeri does indeed possess 
indefinite pronouns.\footnote{As it is a fictional language, the value of this 
assertion to linguistic typology remains doubtful, however.} In order to 
classify languages, \citeauthor{haspelmath1997} generalizes the map displayed 
in \autoref{fig:haspeltab} based on a sample of 100 languages from all 
continents, although he notes that this sample has a European bias due to the 
availability of data \citep[2]{haspelmath1997}. Languages typically form 
continguous areas on the map, even though they may carve it up quite 
differently, and with overlaps between the different semantic groupings 1--9.

An interesting question that \citeauthor{haspelmath1997} poses towards the end 
of his book is whether there are any correlations between word order typology 
and the preference for generic nouns (`person', `thing', `place', `time', 
`manner') or, for instance, interrogative-based systems 
\citep[239--241]{haspelmath1997}. While from \citeauthor{haspelmath1997}'s 
concluding statistics it looks as though there is a slight preference of 
languages with which Ayeri shares basic typological traits---such as 
verb-initial, verb–object, and noun–genitive word order, also having 
prepositions---for basing indefinite pronouns on generic nouns, 
\citeauthor{haspelmath1997} concedes that these seeming correlations are skewed 
by areal effects, \textcquote[241]{haspelmath1997}{because indefinite pronouns 
have a strongly areal distribution}.\footnote{The map in \citetitle{wals}, 
\citet{wals46A}, suggests areal clusters at least for generic-noun based 
systems in Africa and Southeast Asia. \citetitle{wals} classifies 194 out of 
326 languages (60\pct) as possessing interrogative-based indefinite pronouns 
to date, with evidence for this type quoted for all continents except Africa. 
The next smaller group, generic-noun based, falls behind at 85 data points 
(26\pct). The curious lack of evidence for the interrogative type in Africa 
despite its being the most frequent one in the data on all other continents may 
be due to the unavailability of data. Crossreferencing the \citetitle{wals} 
data for constituent-order with the map on indefinite-pronoun systems did not 
yield a result which obviously suggested a correlation.} He still presumes, 
however, that word-order typology may have an effect on the formation of 
indefinites insofar as it correlates with grammaticalization more generally 
\citep[239]{haspelmath1997}.

\begin{figure}[tp]\centering
\caption[The implicational map for indefinite pronoun functions]{The 
implicational map for indefinite pronoun functions \citep[4]{haspelmath1997}}

\scalebox{.75}{%
\begin{tikzpicture}[x=5em]
\node (1) at (1,3) {(1)};
\node (2) at (2,3) {(2)};
\node (3) at (3,3) {(3)};
\node (4) at (4,4) {(4)};
\node (5) at (4,2) {(5)};
\node (6) at (5,4) {(6)};
\node (7) at (6,5) {(7)};
\node (8) at (5,2) {(8)};
\node (9) at (6,1) {(9)};
%
\draw (1) -- (2);
\draw (2) -- (3);
\draw (3) -- (4);
\draw (3) -- (5);
\draw (4) -- (5);
\draw (4) -- (6);
\draw (5) -- (8);
\draw (6) -- (7);
\draw (6) -- (8);
\draw (8) -- (9);
%
\node[haspanno] (l1) at (1) {specific known};
\node[haspanno] (l2) at (2) {specific unknown};
\node[haspanno] (l3) at (3) {irrealis non-specific};
\node[haspanno] (l4) at (4) {question};
\node[haspanno] (l5) at (5) {conditional};
\node[haspanno] (l6) at (6) {indirect negation};
\node[haspanno] (l7) at (7) {direct negation};
\node[haspanno] (l8) at (8) {comparative};
\node[haspanno] (l9) at (9) {free choice};
\end{tikzpicture}
}

\label{fig:haspeltab}
\end{figure}

\begin{figure}\centering
\caption{Indefinite pronouns}

\begin{tabu} to \linewidth {C[2] X[3c] X[3c] X[3c]}
\toprule\tableheaderfont

%
	& every
	& some
	& none
	\\
\toprule
	
person
	& enya % every/any
	& arilinya % some
	& ranya % none
	\\
	
thing
	& enya % every/any
	& arilinya, arilya % some
	& ranya % none
	\\
\midrule
	
place
	& yanen % every/any
	& yāril % some
	& yanoy % none
	\\
\midrule
	
time
	& tadayen % every/any
	& tajaril; metay % some
	& tadoy; jānyam % none
	\\
\midrule
	
manner
	& arēn % every/any
	& miranaril % some
	& aremoy % none
	\\
\bottomrule

\end{tabu}

\label{fig:indeftab}
\end{figure}

\citeauthor{haspelmath1997} mentions generic nouns, and these can be combined 
with the quantifying expressions `every', `any', `some', and `none' into an 
array like the one presented in \autoref{fig:indeftab}. Ayeri does not 
distinguish `every' from `any' the way English does; there is also no 
distinction in polarity (affirmative versus negative) the way English has it:

\pex
	\a\ljudge* I don't know something about this.
	\a I don't know anything about this.
\xe

Likewise, Ayeri does not distinguish between animate and inanimate indefinite 
referents---the same pronouns are used for either, although the shortening of 
\rayr{ArilinY}{arilinya}, \rayr{ArilY}{arilya}, can only be used for 
inanimates, similar to the distinction in the demonstrative pronouns between 
\xayr{AdnYaaNF}{adanyāng}{that one} (that.one-\Aarg{}) and 
\xayr{AdnYreNF}{adareng}{that one} (that.one-\AargI{}; see 
\autoref{subsec:dempro}). Two further features stand out, however.

Firstly, most of the pronouns in the chart have a lexical part---Ayeri's 
indefinite pronouns are based on generic nouns. Thus, the pronouns referring to 
people and things all have the \rayr{/nY}{-nya} element in common, which we 
also find in the interrogative and demonstrative pronouns, and which also 
appears in the word \xayr{nYaanF}{nyān}{person}. In the same way, the pronouns 
related to the notion of place have a \rayr{y/}{ya-} or \rayr{ynF/}{yan-} part, 
which we also find in \xayr{yno}{yano}{place}.\footnote{\rayr{yno}{yano} itself 
is an old nominalization and very likely related as a morpheme to the locative 
suffix \rayr{/y}{-ya}.} In a regular continuation of this pattern, the 
indefinite pronouns of time all have an element related to 
\xayr{tdj}{taday}{time} in common, which is obscured somewhat by palatalization 
in \rayr{tdYrilF}{tajaril}. The exception to this series, then, is 
\rayr{dYaanFymF}{jānyam}, which is the multiplicative numeral formed from 
\xayr{dYa}{ja}{zero}, thus means `zero times' or `not once' rather than 
`never', although it can also be used emphatically for the latter. The series 
of manner pronouns is an absolute exception in that it must be a residue from 
an older layer of grammaticalization since \rayr{Arem/}{are-} is not a 
recognizable morpheme in the modern language.\footnote{I probably made this up 
as I was going, many years ago, and without considering systematic 
implications, as I was unaware of them at the time.} \rayr{mirnrilF}{miranaril} 
is a regular formation of \xayr{mirnF}{miran}{way, manner} combined with the 
quantifier (!) for indefinite amounts, \xayr{/ArilF}{-aril}{some}.

This observation leads to the second regular feature, that is, affixes as 
modifiers to generic nouns. The `every' series regularly features the 
morpheme \rayr{EnF}{en}, either prefixed or suffixed, which is related to the 
quantifier \xayr{/henF}{-hen}{every, all, each} and can presumably be found 
even on \rayr{AreenF}{arēn} in spite of its obscure lexical base. In the same 
manner, the series related to inspecific generic-noun referents is marked by 
the affix \rayr{ArilF}{aril} which, as we have just seen above, is otherwise 
used to refer to inspecific quantities, for instance, 
\xayr{vdiːsnF/ArilF}{vadisān-aril}{some bread} (bread=some).

In the case of \rayr{mirnrilF}{miranaril}, the suffix seems somewhat of an 
odd choice, since manner is not a quantifiable variable in the same way 
people, things, locations or moments are. Possibly, it is chosen more in 
analogy with the other pronouns in this series than on semantic grounds. In any 
event, \rayr{metj}{metay} has the semantically more `proper' \rayr{me/}{me-} 
prefix, relating it to absolute inspecificity.\footnote{Compare German 
\fw{irgendjemand} and French \fw{n'importe qui} `no matter who'.} This 
alternation is employed to distinguish between the meaning of `sometime', that 
is, occurring once an unspecified point in time, and 
\xayr{tdYrilF}{tajaril}{sometimes}, that is, occurring recurringly at 
inspecific times. The alternation between \rayr{mirnrilF}{miranaril} and 
regularly derived \rayr{me/mirnF}{mə-miran} can be leveraged to express a 
specificity difference as well. While the former suggests that an action is 
carried out or an event is happening by means of a specific, though unknown 
procedure, the latter suggests just any possible procedure.

Lastly, the negative series is reguarly marked by the negative suffix 
\rayr{/Oj}{-oy}, which also occurs with 
adjectives and verbs (see sections \ref{subsec:adjneg} and 
\ref{subsubsec:verbneg}). An outlier in this series is the person/thing-related 
indefinite pronoun, \rayr{rnY}{ranya}. The etymologic connections of the 
\rayr{r}{ra} part are not presently known, perhaps the postposition 
\xayr{rnF}{ran}{against} is related.

How do these forms fit in with the chart from \citet{haspelmath1997} quoted at 
the beginning of this section? Regarding the functions of indefinite pronouns 
annotated to the numbers on the map, he gives the following examples, which, 
however, mostly only give one example for either the `person' or `thing' 
category at a time. It is up to the reader to generalize from this 
\citep[2--3]{haspelmath1997}:\footnote{These appear here reordered according to 
numerical order. The book lists them according to their logical order as tracing 
the map, the enumeration somewhat confusingly tied in with the running 
enumeration of examples.}

\pex[labeltype=numeric]
\a specific, known to the speaker: \\ % 1
	\textit{\underline{Somebody} called while you were away: guess who!}
	
\a specific, unknown to the speaker: \\ % 2
	\textit{I heard \underline{something}, but I couldn't tell what kind of 
	sound it was.}
	
\a non-specific, irrealis: \\ % 3
	\textit{Please try \underline{somewhere} else.}
	
\a polar question: \\ % 4
	\textit{Did \underline{anybody} tell you anything about it?}
	
\a conditional protasis: \\ % 5
	\textit{If you see \underline{anything}, tell me immediately.}
	
\a indirect negation: \\ % 6
	\textit{I don't think that \underline{anybody} knows the answer.}
	
\a direct negation: \\ % 7
	\textit{\underline{Nobody} knows the answer.}
	
\a standard of comparison: \\ % 8
	\textit{In Freiburg the weather is nicer than \underline{anywhere} in 
	Germany.}
	
\a free choice: \\ % 9
	\textit{\underline{Anybody} can solve this simple problem.}
\xe

\begin{figure}[tp]\centering
\caption[Map of indefinite pronoun functions in Ayeri]{Map of indefinite 
pronoun 
functions in Ayeri}

\scalebox{.75}{%
\begin{tikzpicture}[x=5em]
\node (1) at (1,3) {(1)};
\node (2) at (2,3) {(2)};
\node (3) at (3,3) {(3)};
\node (4) at (4,4) {(4)};
\node (5) at (4,2) {(5)};
\node (6) at (5,4) {(6)};
\node (7) at (6,5) {(7)};
\node (8) at (5,2) {(8)};
\node (9) at (6,1) {(9)};
%
\draw (1) -- (2);
\draw (2) -- (3);
\draw (3) -- (4);
\draw (3) -- (5);
\draw (4) -- (5);
\draw (4) -- (6);
\draw (5) -- (8);
\draw (6) -- (7);
\draw (6) -- (8);
\draw (8) -- (9);
%
\node[haspanno] (l1) at (1) {specific known};
\node[haspanno] (l2) at (2) {specific unknown};
\node[haspanno] (l3) at (3) {irrealis non-specific};
\node[haspanno] (l4) at (4) {question};
\node[haspanno] (l5) at (5) {conditional};
\node[haspanno] (l6) at (6) {indirect negation};
\node[haspanno] (l7) at (7) {direct negation};
\node[haspanno] (l8) at (8) {comparative};
\node[haspanno] (l9) at (9) {free choice};
%
\draw[semithick] (0.5,5.0) -- (3.0,5.0) -- (4.0,5.0) -- (5.5,5.0) 
-- (5.5,3.0) -- (4.5,3.0) -- (4.5,1.0) -- (4.0,1.0) -- (3.0,1.0) -- (0.5,1.0) 
-- (0.5,5.0);
%
\draw[dotted, semithick] (2.75,3.5) -- (3.0,3.5) -- (4.0,4.5) -- (5.25,4.5) -- 
(5.25,3.5) -- (4.0,3.5) -- (3.0,2.5) -- (2.75,2.5) -- (2.75,3.5);
%
\node[draw, loosely dashed, semithick, fit=(1) (2)] {};
%
\node[draw, dashed, semithick, fit=(6) (7)] {};
%
\node[draw, dash dot, semithick, fit=(8) (9)] {};
\end{tikzpicture}
}

\label{fig:haspeltabayr}
\end{figure}

As we have seen in \autoref{fig:indeftab} above, Ayeri does not make a 
difference between `every' and `any', which is why the `some' series can be 
applied to all of (1)--(6). The pronouns from the `none' column, then, are used 
to express direct negation (7). Since double negation---that is, agreement in 
negation between verbs and indefinite pronouns for purposes of emphasis rather 
than double negation in the strictly logical sense---is possible, the `none' 
series may also be employed for indirect negation (6). Moreover, Ayeri uses 
the `every' series for both standard of comparison (8) and free choice (9). 
Besides this, absolute-indefinite \rayr{me/}{me-} can be used for (3)--(6) in 
combination with a (generic) noun to attach to.
%
% Off the top of my head I'm not sure if I've done this before or if it is a 
% new rule I made up here for the purpose of spicing things up a little:
%
For the `specific' categories (1) and (2) it is furthermore possible to use the 
plain generic nouns, \xayr{nYaanF}{nyān}{person}, \xayr{linY}{linya}{thing}, 
\xayr{yno}{yano}{place}, \xayr{tdj}{taday}{time}, \xayr{mirnF}{miran}{way}, 
however. \autoref{fig:haspeltabayr} shows the groupings for Ayeri; 
(\ref{ex:indefex}) gives examples of all types.

\pex[labeltype=numeric,interpartskip=1em]\label{ex:indefex}
\a specific, known to the speaker:\vspace{.5em} % 1
	\beginsubsub
	\b{a.} \begingl
		\gla Ang sahaya \textbf{arilinya}, leku, sinyāng adaley! //
		\glb Ang saha-ya arilinya-Ø lek-u sinya-ang ada-ley //
		\glc \AgtT{} come-\TsgM{} someone-\Top{}, guess-\Imp{} 
			who-\Aarg{} that-\PargI{} //
		\glft `Someone came, guess who it is!' //
		\endgl\vspace{.5em}
		
	\b{b.} \begingl
		\gla Le ilta ningyang \textbf{linya} vayam. //
		\glb Le ilta ning=yang linya-Ø vayam //
		\glc \PatTI{} need tell=\Fsg{}.\Aarg{} thing-\Top{} 
			\Ssg{}.\Dat{} //
		\glft `I need to tell you something.' //
		\endgl
	\endsubsub

\a specific, unknown to the speaker:\vspace{.5em} % 2
	\beginsubsub
	\b{a.} \begingl
		\gla Ang pegaya \textbf{arilinya} pangisley nā. //
		\glb Ang pega-ya arilinya-Ø pangis-ley nā //
		\glc \AgtT{} steal-\TsgM{} someone-\Top{} money-\PargI{} 
			\Fsg{}.\Gen{} //
		\glft `Someone stole my money.' //
		\endgl\vspace{.5em}
		
	\b{b.} \begingl
		\gla Ang saratang \textbf{yanoya} agon. //
		\glb Ang sara=tang yano-ya agon //
		\glc \AgtT{} go=\TplM{}.\Aarg{} place-\Loc{} foreign //
		\glft `They are going somewhere foreign.' //
		\endgl
	\endsubsub
	
\a non-specific, irrealis:\vspace{.5em} % 3
	\beginsubsub
	\b{a.} \begingl
		\gla Pinyan, prantu \textbf{yāril} palung. //
		\glb Pinyan prant-u yāril palung //
		\glc Please ask-\Imp{} somewhere different //
		\glft `Please ask somewhere else.' //
		\endgl\vspace{.5em}
		
	\b{b.} \begingl
		\gla Le ilta miranang adanya \textbf{mə-mirari} 
			palung. //
		\glb Le mira=nang ilta adanya-Ø mə=mira-ri palung //
		\glc \PatTI{} need do=\Fsg{}.\Aarg{} that.one-\Top{} 
			some=way-\Ins{} different //
		\glft `We need to do that in some other way.' //
		\endgl
	\endsubsub
	
\a polar question:\vspace{.5em} % 4
	\beginsubsub
	\b{a.} \begingl
		\gla ... //
		\glb ... //
		\glc ... //
		\glft `...' //
		\endgl\vspace{.5em}
		
	\b{b.} \begingl
		\gla ... //
		\glb ... //
		\glc ... //
		\glft `...' //
		\endgl
	\endsubsub
	
\a conditional protasis:\vspace{.5em} % 5
	\beginsubsub
	\b{a.} \begingl
		\gla ... //
		\glb ... //
		\glc ... //
		\glft `...' //
		\endgl\vspace{.5em}
		
	\b{b.} \begingl
		\gla ... //
		\glb ... //
		\glc ... //
		\glft `...' //
		\endgl
	\endsubsub
	
\a indirect negation:\vspace{.5em} % 6
	\beginsubsub
	\b{a.} \begingl
		\gla ... //
		\glb ... //
		\glc ... //
		\glft `...' //
		\endgl\vspace{.5em}
		
	\b{b.} \begingl
		\gla ... //
		\glb ... //
		\glc ... //
		\glft `...' //
		\endgl
	\endsubsub
	
\a direct negation:\vspace{.5em} % 7
	\beginsubsub
	\b{a.} \begingl
		\gla ... //
		\glb ... //
		\glc ... //
		\glft `...' //
		\endgl\vspace{.5em}
		
	\b{b.} \begingl
		\gla ... //
		\glb ... //
		\glc ... //
		\glft `...' //
		\endgl
	\endsubsub
	
\a standard of comparison:\vspace{.5em} % 8
	\beginsubsub
	\b{a.} \begingl
		\gla ... //
		\glb ... //
		\glc ... //
		\glft `...' //
		\endgl\vspace{.5em}
		
	\b{b.} \begingl
		\gla ... //
		\glb ... //
		\glc ... //
		\glft `...' //
		\endgl
	\endsubsub
	
\a free choice:\vspace{.5em} % 9
	\beginsubsub
	\b{a.} \begingl
		\gla ... //
		\glb ... //
		\glc ... //
		\glft `...' //
		\endgl\vspace{.5em}
		
	\b{b.} \begingl
		\gla ... //
		\glb ... //
		\glc ... //
		\glft `...' //
		\endgl
	\endsubsub
	
\xe

\index{pronouns!indefinite|)}

\subsection{Relative pronouns}
\index{pronouns!relative|(}

\begin{figure}[tp]\centering
\caption{Relative pronouns}

\begin{tabu} to \linewidth {S X[c] X[c] X[c] X[c] X[c] X[c]}
\tableheaderfont\toprule
Case
	& Pronoun
	& \multicolumn{5}{c}{Pronoun with secondary inflection}
	\\

\tablesubheaderfont\cmidrule{3-7}
	& 
	& \Dat{}
	& \Gen{}
	& \Loc{}
	& \Caus{}
	& \Ins{}
	\\
	
\toprule

Ø
	& si % Ø
	& siyām % \Dat{}
	& sinā % \Gen{}
	& siyā % \Loc{}
	& sisā % \Caus{}
	& sirī % \Ins{}
	\\

\midrule

\Aarg{}
	& sang % Ø
	& sangyam % \Dat{}
	& sangena % \Gen{}
	& sangya % \Loc{}
	& sangisa % \Caus{}
	& sangeri % \Ins{}
	\\

\Aarg{}.\Inan{}
	& sireng % Ø
	& sirengyam % \Dat{}
	& sirengena % \Gen{}
	& sirengya % \Loc{}
	& sirengisa % \Caus{}
	& sirengeri % \Ins{}
	\\
	
\Parg{}
	& sas % Ø
	& sasyam % \Dat{}
	& sasena % \Gen{}
	& sasya % \Loc{}
	& sasisa % \Caus{}
	& saseri % \Ins{}
	\\

\Parg{}.\Inan{}
	& siley % Ø
	& sileyyam % \Dat{}
	& sileyena % \Gen{}
	& sileyya % \Loc{}
	& sileyisa % \Caus{}
	& sileyeri % \Ins{}
	\\

\Dat{}
	& siyam % Ø
	& siyamyam % \Dat{}
	& siyamena % \Gen{}
	& siyamya % \Loc{}
	& siyamisa % \Caus{}
	& siyameri % \Ins{}
	\\

\midrule

\Gen{}
	& sina/sena % Ø
	& sinayam % \Dat{}
	& sinana % \Gen{}
	& sinaya % \Loc{}
	& sinaisa % \Caus{}
	& sinari % \Ins{}
	\\
	
\Loc{}
	& siya % Ø
	& siyayam % \Dat{}
	& siyana % \Gen{}
	& siyaya % \Loc{}
	& siyaisa % \Caus{}
	& siyari % \Ins{}
	\\
	
\Caus{}
	& sisa % Ø
	& sisayam % \Dat{}
	& sisana % \Gen{}
	& sisaya % \Loc{}
	& sisaisa % \Caus{}
	& sisari % \Ins{}
	\\
	
\Ins{}
	& seri % Ø
	& seriyam % \Dat{}
	& serina % \Gen{}
	& seriya % \Loc{}
	& serīsa % \Caus{}
	& seriri % \Ins{}
	\\

\bottomrule
\end{tabu}
\label{fig:relpro}
\end{figure}

As has been described before, Ayeri connects relative clauses to main clauses 
with the relativizer \rayr{si}{si}. This relativizer can be declined for case 
in accordance to the relative clause's head in the matrix clause. The 
respective forms can be gathered from \autoref{fig:relpro} (column 
`Pronoun').

\pex
\a\label{ex:n-rel}\begingl
	\gla Eryyo tarela natrangās si tado. //
	\glb Ery-yo tarela natranga-as si tado //
	\glc use-\TsgN{} still temple-\Parg{} \Rel{} old //
	\glft `The temple, which is old, is still being used.' //
\endgl

\a\label{ex:n-adj-rel}\begingl
	\gla Edanyāng ayonas sirtang sas ang sihabaya mondoas nana. //
	\glb Edanya-ang ayon-as sirtang si-as ang sihaba=ya mondo-as nana //
	\glc this-\Aarg{} man-\Parg{} young \Rel{}-\Parg{} 
		ang tend=\TsgM{}.\Top{} garden-\Parg{} \Fpl{}.\Gen{} //
	\glft `This is the young man who tends our garden.' //
\endgl
\xe

As explained in \autoref{sec:markstrat}, if the relativizer is immediately 
following its lexical head, only the base form \rayr{si}{si} is used, which is 
illustrated in (\ref{ex:n-rel}). Here, the head of the relative clause is 
\xayr{ntFrNaasF}{natrangās}{the temple}, which is immediately followed by the 
relative clause. If word material is intervening, however, which is the case in 
(\ref{ex:n-adj-rel}), the relative pronoun may be inflected to agree in case 
with its antecedent in more formal language for referential clarity: 
\rayr{ssF}{sas} agrees in case with \rayr{AyonsF}{ayonas} two words over to the 
left. Relative pronouns do not agree in number with their heads, though, and in 
gender only insofar as it is relevant to nominal case inflection, that is, 
agents and patients are distinguished for animacy.

A special property of the relative pronoun is that it can be declined for its 
role in the relative clause as well to express more complex relationships 
between the main clause and the relative clause. The respective forms can be 
found in the columns titled `Pronoun with secondary inflection' in 
\autoref{fig:relpro}. The token frequency of the actually occurring complex 
relative pronouns in the very small corpus gathered from example texts and 
dictionary entries (see \autoref{sec:phonotactics}) is given in 
\autoref{tab:relprotokenfreq}.

\begin{table}[tp]\centering
\caption{Token frequencies of attested complex relative pronouns}

\begin{tabu} to .75\linewidth {>{\itshape}X[2l] X[2l] X[1c]}
\tableheaderfont\toprule

Pronoun & Gloss & Absolute \\

\toprule

siyā	& \Rel{}.Ø.\Loc{} & 7 \\
sirī	& \Rel{}.Ø.\Ins{} & 3 \\
sinā	& \Rel{}.Ø.\Gen{} & 1 \\
siyām	& \Rel{}.Ø.\Dat{} & 1 \\

\bottomrule

\textup{Total}	& & 12 \\

\bottomrule
\end{tabu}
\label{tab:relprotokenfreq}
\end{table}

Compared to the unmarked relativizer \rayr{si}{si}, which occurs 50 times in 
the sample (all relative pronouns from \autoref{fig:relpro} occur 80 times in 
total), the complex relative pronouns have a very low frequency. This is not 
surprising, since `for whom', `by which', etc. are quite specialized 
expressions. It also seems that those forms unmarked for their antecedent are 
preferred, since those are the only ones attested---the sample is really much 
too small to make actually meaningful judgements here, however. Examples of 
complex relative pronouns are:

\pex
\a\begingl[glspace=.33em]
	\gla Le vacyang koya yana sileyya ang layāy adanyana. //
	\glb Le vac=yang koya-Ø yana si-ley-ya ang laya=ay.Ø adanya-na //
	\glc \PatTI{} like=\Fsg{}.\Aarg{} book-\Top{} \TsgM{}.\Gen{} 
		\Rel{}-\PargI{}-\Loc{} \Aarg{} read=\Fsg{}.\Top{} that-\Gen{} //
	\glft `I like his book in which I read about it.' //
\endgl

\a\label{ex:reldat}\begingl
	\gla Ya saratang yano siyām sarasatang. //
	\glb Ya sara=tang yano-Ø si-Ø-yām sara-asa=tang //
	\glc \LocT{} go=\TplM{}.\Aarg{} place-\Top{} \Rel{}-\Loc{}-\Dat{} 
		go-\Hab{}=\TplM{}.\Aarg{} //
	\glft `They went to the place to which they always went.' //
\endgl
\xe

It needs to be pointed out that a complex relative pronoun cannot form 
the topic of the relative clause even though it is marked for case according to 
the relative clause's syntactic domain. Furthermore, the relative pronoun 
cannot receive inflection for an agent or a patient of the embedded clause. The 
following examples illustrate these points:

\pex\label{ex:reltop}
% \a\begingl
\ljudge* \begingl
	\gla Mica edaya sobayāng {si \textup{(\ques{}\textit{sī})}} na ihayang 
		koyaley. //
	\glb Mit-ya edaya sobaya-ang si-Ø-Ø na iha=yang koya-ley //
	\glc live-\TsgM{} here teacher-\Aarg{} \Rel{}-\Aarg{}-\Top{} \GenT{} 
		borrow=\Fsg{}.\Aarg{} book-\PargI{} //
% \endgl
% 
% \a\begingl
% 	\gla Mica edaya sobayāng sinā ang ihāy koyaley. //
% 	\glb Mit-ya edaya sobaya-ang si-Ø-nā ang iha=ay.Ø koya-ley //
% 	\glc live-\TsgM{} here teacher-\Aarg{} \Rel{}-\Aarg{}-\Gen{} \AgtT{} 
% 		borrow=\Fsg{}.\Top{} book-\PargI{} //
	\glft `Here lives the teacher from whom I borrowed a book.' //
\endgl
\xe

\pex~\label{ex:relagt}
% \a\begingl
\ljudge* \begingl
	\gla Mica edaya sobayāng sāng le sobya payutān yām. //
	\glb Mit-ya edaya sobaya-ang si-Ø-ang le sob-ya payutān-Ø yām //
	\glc live-\TsgM{} here teacher-\Aarg{} \Rel{}-\Aarg{}-\Aarg{} \PatTI{} 
		teach-\TsgM{} math-\Top{} \Fsg{}.\Dat{} //
% \endgl
% 
% \a\begingl
% 	\gla Mica edaya sobayāng si le sobyāng payutān yām. //
% 	\glb Mit-ya edaya sobaya-ang si le sob=yāng payutān-Ø yām //
% 	\glc live-\TsgM{} here teacher-\Aarg{} \Rel{} \PatTI{} 
% 		teach=\TsgM{}.\Aarg{} math-\Top{} \Fsg{}.\Dat{} //
	\glft `Here lives the teacher who taught me math.' //
\endgl
\xe

\pex~\label{ex:relpat}
% \a\begingl
\ljudge* \begingl
	\gla Mica edaya sobayāng sās ya kradasayang kardang. //
	\glb Mit-ya edaya sobaya-ang si-Ø-as ya krad-asa=yang kardang-Ø //
	\glc live-\TsgM{} here teacher-\Aarg{} \Rel{}-\Aarg{}-\Parg{} \LocT{}
		hate-\Hab{}=\Fsg{}.\Aarg{} school-\Top{} //
% \endgl
% 
% \a\begingl
% 	\gla Mica edaya sobayāng si ya kradasayang (yas) kardang. //
% 	\glb Mit-ya edaya sobaya-ang si ya krad-asa=yang (yas) kardang-Ø //
% 	\glc live-\TsgM{} here teacher-\Aarg{} \Rel{} \LocT{}
% 		hate-\Hab{}=\Fsg{}.\Aarg{} (\TsgM{}.\Parg{}) school-\Top{} //
	\glft `Here lives the teacher whom I used to hate in school.' //
\endgl
\xe

Example (\ref{ex:reltop}) displays a sentence in which the relative pronoun 
relative clause: \rayr{n}{na} as a genitive topic is supposed to refer to 
ungrammatically forms the controller of topic agreement on the verb in the 
\rayr{sobyaaNF}{sobayāng} in the matrix clause by way of the relativizer 
\rayr{si}{si} which would then necessarily carry a zero-morpheme topic 
marker. There is no resumptive pronoun in the relative clause, so the relative 
pronoun itself forms the anaphora in the relative clause referring to the 
relativized argument of the matrix clause. This is not possible.

In example (\ref{ex:relagt}), the relative pronoun *\rayr{saaNF}{*sāng} carries 
no overt case agreement as it directly follows its antecedent 
(*\rayr{sNNF}{*sangang} otherwise) but the long vowel shows that it is declined 
as the agent of the relative clause; the verb agrees using \rayr{/y}{-ya} 
accordingly. There is no resumptive agent pronoun, so the relative pronoun 
would stand in for the agent NP that would be necessary if the relative clause 
were an independent sentence. The use of the relative pronoun as an agent-NP 
replacement in this sentence is equally ungrammatical, though, and so is the 
agreement between verb and declined relative pronoun.

Similarly, in (\ref{ex:relpat}), the relative pronoun carries case marking for 
the patient of the relative clause, since the agent of the matrix clause serves 
as the patient NP of the embedded clause. This is not grammatical either.

Altogether, it seems that in Ayeri, core arguments of intransitive and 
transitive clauses---agents and patients---cannot precede the embedded verb of 
a relative clause; the verb firmly forms the head of the embedded clause in 
this regard. The relative pronoun also cannot receive secondary marking for 
agents and patients and stand in directly as the agent and patient NP of the 
relative clause, respectively. It is interesting in this regard that Ayeri 
\emph{does} allow this for recipients, however, maybe since by their nature as 
goals they carry something of a locative connotation (compare 
(\ref{ex:reldat})) and are thus less tightly integrated with verbs, occupying a 
middle ground between core arguments and adverbials like the locative 
proper.\footnote{This would be interesting to explore in terms of 
grammaticalization, as it is very possible that this behavior reflects a stage 
of the language before \rayr{/ymF}{-yam} had been grammaticalized as the dative 
marker. In this respect, it would as well be necessary to explore whether the 
similarity between the dative marker \rayr{/ymF}{-yam} and the locative marker 
\rayr{/y}{-ya} is indeed etymological or merely incidental.}

\index{pronouns!relative|)}

\subsection{Reflexives and reciprocals}
\label{subsec:reflrec}
\index{pronouns!reflexive|(}

As mentioned previously, Ayeri forms its reflexives with the prefix 
\rayr{sitNF/}{sitang-} in combination with a personal pronoun, compare 
(\ref{ex:reflpat}). If the agent of the action is the same as the reflexive 
patient---that is, the agent acts on itself---the reflexive prefix can also 
migrate onto the verb instead, which is demonstrated in (\ref{ex:reflvb}).

\ex\label{ex:reflpat}\begingl
	\gla Ang silvye sitang=yes puluyya. //
	\glb Ang silv=ye.Ø sitang=yes puluy-ya //
	\glc \AgtT{} see=\TsgF{}.\Top{} self=\TsgF{}.\Parg{} mirror-\Loc{} //
	\glft `She sees herself in the mirror.' //
\endgl\xe

\ex~\label{ex:reflvb}\begingl
	\gla Ang sitang-silvye puluyya. //
	\glb Ang sitang=silv=ye.Ø puluy-ya //
	\glc \AgtT{} self=see=\TsgF{}.\Top{} mirror-\Loc{} //
	\glft `She sees herself in the mirror.' //
\endgl\xe

Doing the same with a non-patient pronoun does not work, however, so the 
sentence in (\ref{ex:reflvb}) with the reflexive \rayr{sitNF/}{sitang} marked 
on the verb is not equivalent to the following one, in which 
\rayr{sitNF/}{sitang-} appears together with a personal pronoun in the locative 
case, even though here as well, the agent and the locative pronoun refer to the 
same person:

\ex\label{ex:reflloc}\begingl
	\gla Ang silvye sitang-yea puluyya. //
	\glb Ang silv=ye.Ø sitang=yea puluy-ya //
	\glc \AgtT{} look=\TsgF{}.\Top{} self=\TsgF{}.\Loc{} mirror-\Loc{} //
	\glft `She looks at herself in the mirror.' //
\endgl\xe

It may be noted furthermore that the genitive/possessive pronoun series conveys 
the meaning of `one's own', which is completely regular in meaning (`of 
X-self'), however:

\ex\begingl
	\gla Le no eryongyang pakay sitang-nā. //
	\glb Le no ery-ong=yang pakay-Ø sitang=nā //
	\glc \PatTI{} want use-\Irr{}=\Fsg{}.\Aarg{} umbrella-\Top{} 
		self=\Fsg{}.\Gen{} //
	\glft `I'd like to use my own umbrella.' //
\endgl\xe

\index{pronouns!reflexive|)}

\index{pronouns!reciprocal|(}

Besides reflexive pronouns, Ayeri also has a reciprocal pronoun, 
\xayr{sitnY}{sitanya}{each other}. This pronoun acts the same as other pronouns 
and can be inflected according to its function in the clause:

\pex
\a\begingl
	\gla Ang narayan {} Ajān nay Pila sitanyaya. //
	\glb Ang nara-yan Ø Ajān nay Pila sitanya-ya //
	\glc \AgtT{} talk-\TplM{} \Top{} Ajān and Pila each.other-\Loc{} //
	\glft `Ajān and Pila talk to each other.' //
\endgl

\a\begingl
	\gla Sa ming tangtang sitanya. //
	\glb Sa ming tang=tang sitanya-Ø //
	\glc \PatT{} can hear=\TplM{}.\Aarg{} each.other-\Top{} //
	\glft `They can hear each other.' //
\endgl

\xe

\index{pronouns!reciprocal|)}

\index{pronouns|)}

\section{Adjectives}
\index{adjectives|(}

Adjectives are one of the parts of speech in Ayeri which do not inflect for any 
of the grammatical properties of their heads, that is, there is no agreement 
relation between adjectives and nominal heads. They do inflect for comparison 
under certain circumstances, however, and can also take various affixes that 
modify the meaning of the adjective stem.

\subsection{Comparison}
\index{comparison|(}

In cases where a comparee is left unexpressed or the patient forms the 
standard of comparison, Ayeri uses clitic suffixes on adjectives. The suffixes 
involved are \rayr{/ENF}{-eng} (\Comp{}) and \rayr{/vaa}{-vā} (\Supl{}):

\pex\label{ex:sfxcomp}
\a\label{ex:sfxcomp2}\begingl
	\gla Yeng ganyena men si alingo-eng. //
	\glb Yeng gan-ye-na men si alingo=eng //
	\glc \TsgF{}.\Aarg{} child-\Pl{}-\Gen{} one \Rel{} clever=\Comp{} //
	\glft `She is one of the more clever children.' //
\endgl

\a\label{ex:sfxcomp1}\begingl
	\gla Ang tavya {} Diyan tingracas ban-eng na Maha. //
	\glb Ang tav-ya Ø Diyan tingrati-as ban=eng na Maha //
	\glc \AgtT{} become-\TsgM{} \Top{} Diyan musician-\Parg{} good=\Comp{} 
		\Gen{} Maha //
	\glft `Diyan became a better musician than Maha.' //
\endgl

%%% Try to express this with the analytic construction with eng-, and you'll 
%%% fail?! > Diyan became a better musician than Maha---only possible with a 
%%% relative clause: Diyan became a musician who is better than Maha. (Sa 
%%% tavya ang Diyan tingrati si ang engya ban sa Maha)

\a\label{ex:sfxsupl}\begingl
	\gla Garatang, yāng pokamayās para-vā. //
	\glb Gara=tang, yāng pokamaya-as para=vā //
	\glc name=\TplM{}.\Aarg{}, \TsgM.\Aarg{} shooter-\Parg{} fast=\Supl{} //
	\glft `They named him the fastest shooter.' //
\endgl\xe

In (\ref{ex:sfxcomp2}) the comparee is missing, while in (\ref{ex:sfxcomp1}), 
the quality under comparison, \xayr{tiNrti\_asF bnF/ENF}{tingracas ban-eng}{a 
better musician}, is a patient NP; the standard, \rayr{mh}{Maha}, is 
expressed by an adverbial genitive NP. The example in (\ref{ex:sfxsupl}) 
similarly expresses an absolute without giving a group of entities to draw 
from. 
In all these cases, it is, however, also possible to use a more complex 
analytic construction using verbs which will be covered at a later point.

\index{comparison|)}

\subsection{Negation}
\label{subsec:adjneg}
\index{negation!of adjectives|(}

Adjectives in Ayeri can be negated in two ways: categorially with 
\rayr{/ArY}{-arya}, and pragmatically with \rayr{/Oj}{-oy}. These correspond to 
English \fw{un-}, and \fw{in-, il-, ir-} etc. for categorial negation, and to 
\fw{not} for pragmatic negation. \rayr{/Oj}{-oy} absorbs the vowel of the root 
it is attached to if said root ends in a vowel.

\ex\label{ex:adjarya}\begingl
	\gla Telbaya miseryanang ku-ardārya. //
	\glb Telba-ya miseryan-ang ku=arda-arya //
	\glc show-\TsgM{} method-\Aarg{} like=suitable-\Neg{} //
	\glft `The method proved unsuitable.' //
\endgl\xe

\ex~\label{ex:adjoy}\begingl
	\gla Pakoy eda-yanoreng. //
	\glb Paka-oy eda=yano-reng //
	\glc safe-\Neg{} this=place-\AargI{} //
	\glft `This place is not safe.' //
\endgl\xe

Example (\ref{ex:adjarya}) displays an adjective which carries the categorial 
negation marker \rayr{/ArY}{-arya}; the adjective in (\ref{ex:adjoy}) carries 
the simple, pragmatic negation marker \rayr{/Oj}{-oy}. Which one to use is up 
to the speaker, since both negate the described property. The categorial marker 
puts an emphasis more on expressing a general opposite, while the pragmatic 
marker simply negates, so that it is not necessarily implied that the negative 
state persists. The place that is \xayr{pkoj}{pakoy}{not safe} now is not 
necessarily \xayr{pkaarY}{pakārya}{unsafe} in general, but simply not safe in 
the context of the here and now of the utterance.

Besides \fw{ad hoc} derivation of categorial negatives with \rayr{/ArY}{-arya}, 
there are also a few lexicalized instances. These have an idiomatic meaning and 
the negator or the word itself may be irregularly reduced. Examples are, among 
others:

\pex
	\a \makebox[9.5em][l]{\xayr{\larger bnF}{ban}{good}}
		→ \xayr{\larger bny}{banaya}{ill, sick}
	\a \makebox[9.5em][l]{\xayr{\larger kovro}{kovaro}{easy}}
		→ \xayr{\larger kovrY}{kovarya}{awkward}
	\a \makebox[9.5em][l]{\xayr{\larger sirimNF}{sirimang}{straight}}
		→ \xayr{\larger sirimy}{sirimaya}{passive}
\xe

\index{negation!of adjectives|)}

\subsection{Adjectivization}

Adjectives in Ayeri are very commonly zero derivations, that is, 
there is rather free conversion between nouns and 
adjectives,\footnote{Adjectives and split-off modifiers in noun--noun compounds 
are thus similar at least superficially.} for instance:

\pex
	\a \makebox[9em][l]{\xayr{\larger Ayeri}{Ayeri}{Ayeri}}
		\til{} \xayr{\larger Ayeri}{Ayeri}{Ayeri}
	\a \makebox[9em][l]{\xayr{\larger dis}{disa}{soap, lye}}
		\til{} \xayr{\larger dis}{disa}{soapy, alkaline}
	\a \makebox[9em][l]{\xayr{\larger gino}{gino}{drink}}
		\til{} \xayr{\larger gino}{gino}{drunk}
	\a \makebox[9em][l]{\xayr{\larger phmj}{pahamay}{danger}}
		\til{} \xayr{\larger phmj}{pahamay}{dangerous}
	\a \makebox[9em][l]{\xayr{\larger seMpj}{sempay}{peace}}
		\til{} \xayr{\larger seMpj}{sempay}{peaceful}
\xe

Adjectives can also be derived from verbs with the causative suffix 
\rayr{/Is}{-isa}, which often corresponds to adjectives derived from the 
past participle form---the meaning is often, but not necessarily, relating to 
an achieved state. The suffix may change the last vowel to \rayr{U}{u} or drop 
it; a specific pattern to these changes is not recognizable. The derivations 
may be idiomatic occasionally, as some derivations in the example below show.

\pex
	\a \makebox[10.5em][l]{\xayr{\larger kelNF/}{kelang-}{connect}}
		→ \xayr{\larger kelNisu}{kelangisu}{connected, related}
	\a \makebox[10.5em][l]{\xayr{\larger pluNF/}{palung-}{distinguish}}
		→ \xayr{\larger pluNis}{palungisa}{various}
	\a \makebox[10.5em][l]{\xayr{\larger suMdl/}{sundala-}{lose}}
		→ \xayr{\larger suMdlisu}{sundalisu}{lost}
	\a \makebox[10.5em][l]{\xayr{\larger thnF/}{tahan-}{write}}
		→ \xayr{\larger thnisF}{tahanis}{literary}
	\a \makebox[10.5em][l]{\xayr{\larger ves/}{vesa-}{give birth}}
		→ \xayr{\larger vesis}{vesisa}{native}
\xe

There are also at least two words where an \rayr{/Is}{-isa} adjective is 
derived not from a verb, but a word of a different part of speech---in this 
case, a noun, and another adjective:

\pex
	\a \makebox[6em][l]{\xayr{\larger ApinF}{apin}{luck}}
		→ \xayr{\larger Apinis}{apinisa}{lucky}
	\a \makebox[6em][l]{\xayr{\larger Irj}{iray}{high}}
		→ \xayr{\larger Iryisu}{irayisu}{exalting}
\xe

\subsection{Other affixes}
\label{subsec:adjaffx}

As with nouns, other affixes which can be attached to adjectives as clitic 
hosts, are the prefix \rayr{ku/}{ku-}, expressing semblance, as well as 
quantifying and grading suffixes, of which the suffixes used to express 
comparative and superlative are, essentially, a grammaticalized variety, since 
\rayr{/ENF}{-eng} can also be used like `rather'.

\ex\begingl
	\gla Paray-parayang ku-pikisu //
	\glb Paray\til{}paray-ang ku=pikisu //
	\glc \Dim{}\til{}cat-\Aarg{} like=scared //
	\glft `The kitten is like scared.' //
\endgl\xe

\ex~\begingl
	\gla Eda-prikanreng napay-eng //
	\glb Eda=prikan-reng napay=eng //
	\glc this=soup-\AargI{} spicy=rather //
	\glft `This soup is rather spicy.' //
\endgl\xe

\index{adjectives|)}


\section{Adpositions}
\index{adpositions|(}

Adpositions are another part of speech in Ayeri whose stem itself does not 
inflect. Ayeri's most basic adpositions are derived from relational nouns, 
which is likely the reason why Ayeri mostly employs prepositions, with 
postpositions and ambipositions being less important placement patterns 
\parencites[110--111]{hagege2010}[81~ff.]{lehmann2015}. Adpositions in their 
most basic use trigger locative marking on the governed NP, the prepositional 
object; for allative and ablative meanings, the prepositional object may 
also appear in the dative and the genitive case, respectively, as described in 
\autoref{subsubsec:dative}.\footnote{Even a prolative use together with the 
instrumental is thinkable.} The cognitive metaphor `time equals space' 
with the future conceptualized as lying ahead and the past behind also holds in 
Ayeri, so that some of the words describing locations also double to describe 
temporal relations.

\subsection{Prepositions}
\index{prepositions|(}

\begin{figure}[tp]\centering
\caption{Prepositions (simple)}
\begin{tabu} to \linewidth {I[3] X[4] X[6]}
\tableheaderfont\toprule
\multicolumn{2}{c}{Preposition}
	& Etymology (or related to)
	\\

\toprule

\ayr{\upshape AgonnF}
agonan
	& outside
	& \xayr{AgonnF}{agonan}{outside}
	\\
	
\ayr{\upshape AvnF}
avan
	& bottom, ground
	& \xayr{AvnF}{avan}{ground, bottom; soil}
	\\

% \ayr{\upshape djrinF}
% dayrin
% 	& side
% 	& \xayr{djrinF}{dayrin}{waist}
% 	\\

\ayr{\upshape EjrnF}
eyran
	& under, below
	& \xayr{EjrnF}{eyran}{sole}
	\\

\ayr{\upshape EjrrY}
eyrarya
	& over
	& \xayr{EjrnF}{eyran}{sole} + \rayr{/ArY}{-arya} (\Neg{})
	\\

\ayr{\upshape kjvo}
kayvo
	& with, beside\footnotemark
	& \xayr{kjvF/}{kayv-}{accompany}
	\\

\ayr{\upshape koNF}
kong
	& inside, within
	& \xayr{koNF}{kong}{inside}
	\\
	
\ayr{\upshape liNF}
ling
	& on
	& \xayr{liNF}{ling}{top}
	\\

\ayr{\upshape lug}
luga
	& among, between
	& \xayr{lug/}{luga-}{pass, penetrate}
	\\

\ayr{\upshape mNsh}
mangasaha
	& towards, in\,+\,\emph{time}
	& \xayr{mN sh/}{manga saha-}{coming}
	\\

\ayr{\upshape mNsr}
mangasara
	& away
	& \xayr{mN sr/}{manga sara-}{going}
	\\

\ayr{\upshape mrinF}
marin
	& front, on (walls etc.)
	& \xayr{mrinF}{marin}{face, surface}
	\\

\ayr{\upshape midj}
miday
	& around
	& \xayr{midj/}{miday-}{surround}
	\\

\ayr{\upshape nsj}
nasay
	& near, close
	& \xayr{nsj}{nasay}{proximity}
	\\

\ayr{\upshape nuveNF}
nuveng
	& left
	& \xayr{nuho}{nuho}{liver}
	\\

\ayr{\upshape pNF}
pang
	& behind, ago
	& \xayr{pNF}{pang}{back}
	\\

\ayr{\upshape ptmeNF}
patameng
	& right
	& \xayr{ptmF}{patam}{heart}
	\\

\bottomrule
\end{tabu}

\label{fig:prepos}
\end{figure}

\footnotetext{There is also a preposition \xayr{djrin}{dayrin}{side} listed in 
the dictionary, however, this has never seen much use. Instead, 
\rayr{kjvo}{kayvo} has come to cover `beside, to the side of' as well.}

\autoref{fig:prepos} gives all the words in Ayeri which may be used as 
prepositions. As mentioned above, most of these are derived transparently from 
nouns, so they have probably been grammaticalized relatively recently---their 
non-preposition meaning is still transparent, they are still phonologically 
rather complex, and some of them are even polysyllabic in spite of not being 
composed and covering rather basic meanings.\footnote{Unsurprisingly, 
\citet[129]{hagege2010} references Zipf regarding speech economy and token 
frequency. According to \citet[134--141]{lehmann2015}, the phonological 
integrity of morphemic units reduces as grammaticalization is progressing (with 
token frequency increasing due to increasing obligatoriness). 
\citet{bybeehopper2001b} see the reason for phonological reduction of highly 
frequent phonological material \textcquote[11]{bybeehopper2001b}{in the 
automatization of neuro-motor sequences […]. Such reductions are systematic 
across speakers; that is, they do not respresent \enquote{sloppy} or 
\enquote{lazy} speech}. Hence, for example, English's most basic prepositions 
are extremely short and simple words, for instance, \fw{of, at, in}, which 
derive from the slightly more complex PIE forms \fw{*h₂ep-ó, *h₂ed, *h₁en(-i)}, 
respectively \citep[1, 39, 269]{kroonen2013}. Since adpositions frequently 
grammaticalize into case markers, it may be assumed that the phonologically much 
more simple case affixes of Ayeri constitute an older layer of basic 
adpositions. Their non-suffixed forms may be remnants of this use.} Since these 
nouns have ceased to function as common nouns, however, it is not possible to 
inflect them in the way described in \autoref{sec:nouns}. Thus, for example, 
while it is possible to say (\ref{ex:lingnn}), it is not really possible to say 
(\ref{ex:lingpr}):

\pex
\a\label{ex:lingnn}\begingl
	\gla Le yomareng kanka lingya rivanena. //
	\glb Le yoma=reng kanka-Ø ling-ya rivan-ena //
	\glc \PatTI{} exist=\TsgI{}.\Aarg{} snow-\Top{} top-\Loc{}
		mountain-\Gen{} //
	\glft `There is snow on the top of the mountain.'\footnotemark //
\endgl

\a\label{ex:lingpr}\ljudge* \begingl
	\gla Ang nedraye lingya nedrānena. //
	\glb Ang nedra=ye.Ø ling-ya nedrān-na //
	\glc \AgtT{} sit=\TsgF{}.\Top{} top-\Loc{} chair-\Gen{} //
	\glft `\ques{}She sits on the top of a chair.' //
\endgl

\xe

\footnotetext{The corresponding sentence with a preposition is \xayr{le yomreNF 
kMk liNF rivnFy}{Le yomareng kanka ling rivanya}{There is snow on top of the 
mountain} (\PatTI{} exist=\TsgI{}.\Aarg{} snow-\Top{} top mountain-\Loc{}).}

\noindent Instead, the grammatical way to express (\ref{ex:lingpr}) is the 
following, using \rayr{liNF}{ling} as a preposition with the object in the 
locative case:

\ex\begingl
	\gla Ang nedraye ling nedrānya. //
	\glb Ang nedra=ye.Ø ling nedrān-ya //
	\glc \AgtT{} sit=\TsgF{}.\Top{} top chair-\Loc{} //
	\glft `She sits on a chair.' //
\endgl\xe

In this case, since \emph{on} is the expected position of sitting with regards 
to chairs, the preposition can even be dropped:

\ex\begingl
	\gla Ang nedraye nedrānya. //
	\glb Ang nedra=ye.Ø nedrān-ya //
	\glc \AgtT{} sit=\TsgF{}.\Top{} chair-\Loc{} //
	\glft `She sits on a chair.' //
\endgl\xe

With regards to (\ref{ex:lingnn}) it is also necessary to mention what 
\citeauthor{hagege2010} calls the `Proof by Anachrony Principle' 
\citep[158--159]{hagege2010}. According to this principle, when an adposition 
is very grammaticalized, speakers can use both the adposition and its 
etymological ancestor side by side without taking offense in the double 
occurrence. This is notably not the case in Ayeri, where it is not possible to 
say things like (\ref{ex:behindback1}), where \rayr{pNF}{pang} is used in both 
its meanings so that the preposition \xayr{pNF}{pang}{behind} governs the 
original noun \xayr{pNF}{pang}{back}.

\pex
\a\label{ex:behindback1}\ljudge* \begingl
	\gla Le ranice ang Maha adanya pang pangya yena. //
	\glb Le ranit-ye ang Maha adanya-Ø pang pang-ya yena //
	\glc \PatTI{} hide-\TsgF{} \Aarg{} Maha that-\Top{} back back-\Loc{} 
		\TsgF{}.\Gen{} //
	\glft `*Maha hides it at the back of her back.' //
\endgl

\a\label{ex:behindback2}\begingl
	\gla Le ranice ang Maha adanya pangya yena. //
	\glb Le ranit-ye ang Maha adanya-Ø pang-ya yena //
	\glc \PatTI{} hide-\TsgF{} \Aarg{} Maha that-\Top{} back-\Loc{} 
		\TsgF{}.\Gen{} //
	\glft `Maha hides it at her back,'\\
		or: `Maha hides it behind herself.' //
\endgl

\xe

Examples like (\ref{ex:lingpr}), on the other hand, show that there is 
nonetheless a tendency towards grammaticalization of originally relational 
nouns in Ayeri. Grammaticalization is visible in that formerly relational nouns 
have become restricted in the way they can be used syntactically 
\citep[174]{lehmann2015}. This specialization is also apparent in 
morphology from the fact that prepositions in Ayeri, in spite of their nominal 
origin, cannot be modified by adjectives and relative clauses like regular 
nouns. Thus, for instance, while \rayr{AvnF}{avan} as a noun can mean 
`soil' or `ground' and can be modified by semantically coherent adjectives like 
\xayr{kbu}{kabu}{fertile}, the preposition \rayr{AvnF}{avan} cannot. Again, a 
grammatical way to express (\ref{ex:avanprep}) would have to use 
\rayr{AvnF}{avan} as a relational noun, that is, \xayr{AvnFy kbu 
similen}{avanya kabu similena}{at the fertile bottom of the country} 
(bottom-\Loc{} fertile country-\Gen{}).

\pex
\a\label{ex:avannn}\begingl
	\gla Sa yomareng avan kabu ibangya yana. //
	\glb Sa yoma=reng avan-Ø kabu ibang-ya yana //
	\glc \PatT{} exist=\TsgI.\AargI{} ground-\Top{} fertile field-\Loc{} 
		\TsgM{}.\Gen{} //
	\glft `Fertile ground is on his field.' //
\endgl

\a\label{ex:avanprep}\ljudge* \begingl
	\gla Ang mican avan kabu similya //
	\glb Ang mit=yan.Ø avan kabu simil-ya //
	\glc \AgtT{} live=\TplM{}.\Top{} bottom fertile country-\Loc{} //
	\glft `*They live at the fertile bottom of the country.' //
\endgl

\xe

At the beginning of this section it was shown that prepositions in Ayeri cannot 
receive number and case marking, which are otherwise typical features of nouns. 
What is possible with regards to affixes, however, is adding degree suffixes to 
prepositions, since these suffixes are clitics, selecting phrasal heads as 
their hosts, rather than inflections:

\ex\begingl
	\gla Ang mitasaye pang-ikan mandayya tado. //
	\glb Ang mit-asa=ye.Ø pang=ikan manday-ya tado //
	\glc \AgtT{} live-\Hab{}=\TsgF{}.\Top{} back=much forum-\Loc{} old //
	\glft `She used to live way behind the old forum.' //
\endgl\xe

\begin{figure}[tp]\centering
\caption{Prepositions (dynamic)}
\begin{tabu} to \linewidth {X[2] X[3]}
\tableheaderfont\toprule
Preposition
	& \fw{manga} + \Prep{}
	\\

\toprule

\xayr{AgonnF}{agonan}{outside}
	& out
	\\

\xayr{AvnF}{avan}{at bottom}
	& to the bottom; \textit{with \Dat{}/\Gen{}:} down to/from
	\\

% \xayr{djrinF}{dayrin}{beside}
% 	& to the side
% 	\\

\xayr{EjrnF}{eyran}{under}
	& under
	\\

\xayr{EjrrY}{eyrarya}{over}
	& across, over
	\\

\xayr{kjvo}{kayvo}{with, beside}
	& along
	\\

\xayr{koNF}{kong}{inside}
	& into
	\\

\xayr{liNF}{ling}{on top}
	& onto, while; \textit{with \Dat{}/\Gen{}:} up to/from
	\\

\xayr{lug}{luga}{between}
	& through, during, for\,+\,\textit{time}
	\\

% \ayr{\upshape mNsh}
% mangasaha
% 	& 
% 	\\
% 
% \ayr{\upshape mNsr}
% mangasara
% 	& 
% 	\\

\xayr{mrinF}{marin}{in front}
	& to the front
	\\

\xayr{midj}{miday}{around}
	& circling around
	\\

\xayr{nsj}{nasay}{near}
	& into the near
	\\

\xayr{nuveNF}{nuveng}{left}
	& to the left
	\\

\xayr{pNF}{pang}{behind}
	& behind, to the back
	\\

\xayr{ptmeNF}{patameng}{right}
	& to the right
	\\

\bottomrule
\end{tabu}

\label{fig:preposdyn}
\end{figure}

\phantomsection\label{manga}
As demonstrated before, another quasi-inflection adpositions in Ayeri can 
carry is the dynamic marker \rayr{mN}{manga} (see \autoref{sec:typology}). 
While most of the prepositions in \autoref{fig:prepos} have a static meaning, 
\rayr{mN}{manga} indicates a motion in the direction of the respective 
location, thus \xayr{koNF}{kong}{inside} becomes \xayr{mN koNF}{manga 
kong}{into}, for instance. \autoref{fig:preposdyn} repeats the table of 
prepositions above for the most part and gives the respective dynamic meanings. 
The prepositions \rayr{mNsh}{mangasaha} and \rayr{mNsr}{mangasara} are missing 
from this list and appear in the previous table instead, even though they 
express motion rather than position, because they are only used in this base 
form and cannot be prefixed by \rayr{mN}{manga}, which they already contain. 
Note, however, that \rayr{mNsh}{mangasaha} and \rayr{mNsr}{mangasara} are not 
synonymous to an adjunct in the dative and the genitive case, respectively. 
Rather, the prepositions add a more deliberate or literal meaning:

\pex
\a\begingl
	\gla Ang nimpay kardangyam. //
	\glb Ang nimp=ay.Ø kardang-yam //
	\glc \AgtT{} run=\Fsg{}.\Top{} school-\Dat{} //
	\glft `I'm running to (a/the) school.' \\
		(e.g. for class, or just up to the building) //
\endgl

\a\begingl
	\gla Ang nimpay mangasaha kardangya. //
	\glb Ang nimp=ay.Ø mangasaha kardang-ya //
	\glc \AgtT{} run=\Fsg{}.\Top{} towards school-\Loc{} //
	\glft `I'm running towards (a/the) school.' \\
		(up to the building) //
\endgl
\xe

\pex~
\a\begingl
	\gla Ang lampay kardangena. //
	\glb Ang walk=ay.Ø kardang-ena //
	\glc \AgtT{} walk=\Fsg{}.\Top{} school-\Gen{} //
	\glft `I'm walking from (a/the) school.' \\
		(e.g. home, or somewhere else from there) //
\endgl

\a\begingl
	\gla Ang lampay mangasara kardangya. //
	\glb Ang lamp=ay.Ø mangasara kardang-ya //
	\glc \AgtT{} walk=\Fsg{}.\Top{} away school-\Loc{} //
	\glft `I'm walking away from (a/the) school.' \\
		(away from the building) //
\endgl
\xe

Also note that while Germanic languages like English make frequent use of 
set expressions which combine a verb with a preposition, such as \fw{run 
away, go by, raise up, track down}, sometimes with rather idiomatic meanings, 
this pattern does not occur as frequently in Ayeri. Some exceptions are:

\pex
\a \xayr{\larger IlF/ mNsr}{il- mangasara}{surrender} (give away),
\a \xayr{\larger lMtF/ mNsr}{lant- mangasara}{distract} (lead away),
\a \xayr{\larger niMpF/ mNsr}{nimp- mangasara}{escape} (run away),
\a \xayr{\larger tpY/ djrinF}{tapy- dayrin}{save (valuable assets)} (put aside),
\a \xayr{\larger tpY/ midj}{tapy- miday}{put on} (put around),
\a \xayr{\larger tur/ mNsh}{tura- mangasaha}{forward} (send towards).
\xe

These verbs do not govern a prepositional object in the locative case in their 
idiomatic meaning, as displayed by the next example, in which 
\rayr{btNimnF}{batangiman} and \rayr{s AgYaanF}{sa Ajān} do neither serve as 
arguments of \rayr{lMtYo}{lanco} or \rayr{mNsr}{mangasara}, but of the phrasal 
verb \rayr{lMtF/ mNsr}{lant- mangasara}:\footnote{Colloquially, 
\rayr{mNsh}{mangasaha} and \rayr{mNsr}{mangasara} may be shortened to just 
\rayr{sh}{saha} and \rayr{sr}{sara}, respectively.}

\ex\begingl
	\gla Ang lanco mangasara batangiman sa Ajān. //
	\glb Ang lant-yo mangasara batangiman-Ø sa Ajān //
	\glc \AgtT{} lead-\TsgN{} away mosquito-\Top{} \Parg{} Ajān //
	\glft `The mosquito distracted Ajān.' //
\endgl\xe

Very often, where the verb expression in English contains a preposition, there 
is a separate verb in Ayeri, or the same verb is used in Ayeri for both the 
plain English verb and the one extended by a preposition:

\pex
	\a \xayr{\larger ApMdF/}{apand-}{descend, climb down},
	\a \xayr{\larger dil/}{dila}{figure out, find out},
	\a \xayr{\larger liNF/}{ling-}{ascend, mount, climb up},
	\a \xayr{\larger ng/}{naga-}{watch after},
	\a \xayr{\larger phF/}{pah-}{remove, take away},
	\a \xayr{\larger subFrF/}{subr-}{cease, give up}.
\xe

\pex~
	\a \xayr{\larger k/}{ka-}{throw (away)},
	\a \xayr{\larger mtF/}{mat-}{warm (up)},
	\a \xayr{\larger sikFlF/}{sikl-}{rip (up)}.
\xe

In cases where the preposition does not have a prepositional object otherwise, 
its double nature as a noun comes to the fore in that the preposition word will 
be treated like a noun if it is denominal and carries the appropriate case 
marker itself:

\pex
\a\begingl
	\gla Ang sahayan manga pang nangaya. //
	\glb Ang saha=yan.Ø manga pang nanga-ya //
	\glc \AgtT{} go=\Tpl{}.\Top{} \Dyn{} back house-\Loc{} //
	\glft `They go behind the house.' //
\endgl

\a\begingl
	\gla Ang sahayan pangyam. //
	\glb Ang saha=yan.Ø pangyam //
	\glc \AgtT{} go=\Tpl{}.\Top{} back-\Dat{} //
	\glft `They go behind (it),'\\
		or: `They go to the back.' //
\endgl

\xe

% So far we have only looked at nominal terms governed by prepositions; Ayeri also 
% allows pronominal NPs as prepositional objects, however. In a maybe 
% typologically peculiar way, personal pronouns are not affected phonologically or 
% morphologically by prepositions: \blockcquote[138--139]{hagege2010}{More 
% generally, the governed terms of Adps are not always treated identically 
% depending on whether they are nominal or pronominal. [...] In many languages, as 
% opposed to Abkhaz, Adps do not exhibit the same behaviour with pronouns as with 
% nouns.} Personal pronouns have the same shape as anywhere else if they are 
% governed by prepositions insofar as they appear primarily in the locative case, 
% or alternatively in the dative and genitive case, to specify direction of 
% movement:
% 
% \pex
% \a\begingl
% 	\gla Ang bengya {} Ajān kayvo yea. //
% 	\glb Ang beng-ya {} Ajān kayvo yea //
% 	\glc \AgtT{} stand-\TsgM{} \Top{} Ajān beside \TsgF{}.\Loc{} //
% 	\glft `Ajān stands beside her.' //
% \endgl
% 
% \a\begingl
% 	\gla Sahu manga ling yām! //
% 	\glb Saha-u manga ling yām //
% 	\glc come-\Imp{} \Dyn{} top \Fsg{}.\Dat{} //
% 	\glft `Come up to me!' //
% \endgl
% 
% \xe
% 
% As discussed above, personal pronouns are not the only kind of pronoun Ayeri 
% possesses; there are also question pronouns and demonstrative pronouns, for 
% instance. These as well can feature as governed terms: example 
% (\ref{ex:ppquespro}) shows a question pronoun governed by a preposition to ask 
% for the place when the position is already known (the question pronoun is not 
% cased, though, so it cannot receive topic marking); in example 
% (\ref{ex:ppdempro}), the place is indicated by a demonstrative pronoun. It is, 
% however, not only possible to use demonstrative pronouns as governed terms, but 
% also nouns carrying a demonstrative prefix, which is illustrated in 
% (\ref{ex:ppdemnn}). Ayeri does not impose any restrictions to modification by 
% adjectives or topicalization of governed nominal terms of prepositions either, 
% that is, there are no restrictions on the definiteness of the dependent NP as 
% far as Ayeri can express relative notions of definiteness.
% 
% \ex\label{ex:ppquespro}\begingl
% 	\gla Le tapyavāng adanya eyran siyan? //
% 	\glb Le tapy=vāng adanya-Ø eyran siyan //
% 	\glc \PatTI{} put=\Ssg{}.\Aarg{} that-\Top{} under where //
% 	\glft `The thing, where did you put it under?' //
% \endgl\xe
% 
% \pex~
% \a\label{ex:ppdempro}\begingl
% 	\gla Ya tapyyāng tinkayley eyran adanya. //
% 	\glb Ya tapy=yāng tinkay-ley eyran adanya-Ø //
% 	\glc \LocT{} put=\Fsg{}.\Aarg{} key-\PargI{} under that-\Top{} //
% 	\glft `That thing is where I put the key under.' //
% \endgl
% 
% \a\label{ex:ppdemnn}\begingl
% 	\gla Ya tapyyāng tinkayley eyran ada-prekay. //
% 	\glb Ya tapy=yāng tinkay-ley eyran ada=prekay-Ø //
% 	\glc \LocT{} put=\Fsg{}.\Aarg{} key-\PargI{} under that=pillow-\Top{} //
% 	\glft `That pillow is where I put the key under.' //
% \endgl
% 
% \xe

\index{prepositions|)}

\subsection{Postpositions}
\index{postpositions|(}

\begin{figure}[tp]\centering
\caption{Postpositions}
\begin{tabu} to \linewidth {I[3] X[4] X[6]}
\tableheaderfont\toprule
\multicolumn{2}{c}{Postposition}
	& Etymology (or related to)
	\\

\toprule

\ayr{\upshape d/naarY}
da-nārya
	& despite, in spite of
	& \xayr{d/}{da-}{such} + \xayr{naarY}{nārya}{but}
	\\

\ayr{\upshape kjvj}
kayvay
	& without
	& \xayr{kjvo}{kayvo}{with} + \rayr{/Oj}{-oy} (\Neg{})
	\\

\ayr{\upshape mshtj}
masahatay
	& since
	& \rayr{m/}{mə-} (\Pst{}) + \xayr{sh/}{saha-}{come} + 
		\xayr{tdj}{taday}{time}
	\\

\ayr{\upshape nsYmF}
nasyam
	& according to
	& \xayr{nsYymF}{nasyyam}{following}
	\\

\ayr{\upshape pNF}
pang
	& beyond, after, past
	& \xayr{pNF}{pang}{back}
	\\

\ayr{\upshape pesnF}
pesan
	& until
	& ---
	\\

\ayr{\upshape rnF}
ran
	& against
	& \emph{possibly} \xayr{rnF}{ran}{from it}
	\\

\ayr{\upshape ryu}
rayu
	& diagonally across
	& \xayr{ryu}{rayu}{slanted, oblique, skewed}
	\\
	
\ayr{\upshape ymFv}
yamva
	& instead of
	& ---
	\\

\bottomrule
\end{tabu}

\label{fig:postpos}
\end{figure}

While Ayeri mainly uses prepositions---which is by far the most common order
for VO languages \citep{wals95}---it also uses a number of postpositions, which 
are given in \autoref{fig:postpos}. As can be read from the figure, 
postpositions do not usually have a nominal origin but are derived either from 
other prepositions, from adverbial phrases, or even from an adjective in the 
case of \rayr{ryu}{rayu}. The etymologies of \rayr{pesnF}{pesan} and 
\rayr{ymFv}{yamva} are unclear to date.

The postposition \rayr{pNF}{pang} is special in that it also exist as a 
preposition meaning `behind, in the back of', though as a postposition it 
acquires the related but slightly different meaning `beyond, after, past'. It 
might thus be better treated as a homonym to the preposition rather than as an 
ambiposition \citep[115]{hagege2010}. Example (\ref{ex:pangprep}) illustrates a 
use of \rayr{pNF}{pang} as a preposition, (\ref{ex:pangpost}) one of 
\rayr{pNF}{pang} as a postposition. This is in contrast to typical 
ambipositions like German \fw{wegen} `because of, due to' in (\ref{ex:wegen}), 
which has the same meaning in either position and the position variant is just a 
matter of style.

\pex
\a\label{ex:pangprep}\begingl
	\gla Sa lancāng pel manga pang penungya. //
	\glb Sa lant=yāng pel-Ø manga pang penung-ya //
	\glc \PatT{} lead=\TsgM{}.\Aarg{} horse-\Top{} \Dyn{} back 
		barn-\Loc{} //
	\glft `The horse, he leads it behind the stable.' //
\endgl

\a\label{ex:pangpost}\begingl
	\gla Lesyo pelang si sā nimpyong penungya pang yan. //
	\glb Les-yo pel-ang si sā nimp=yong penung-ya pang yan.Ø //
	\glc fall-\TsgN{} horse-\Aarg{} \Rel{} \CauT{} 
		run=\TsgN{}.\Aarg{} stable-\Loc{} back \Tpl{}.\Top{} //
	\glft `The horse they raced past the barn fell.' //
\endgl

\xe

\pex~\label{ex:wegen}
\a\label{ex:wegenprep}\begingl\rc{German}%
	\gla wegen des schlechten Wetters //
	\glb wegen des schlecht-en Wetter-s //
	\glc because.of \Def{}.\Gen{}.\N{}.\Sg{} bad-\Gen{}.\N{}.\Sg{} 
		weather-\Gen{} //
	\glft `because of the bad weather' //
\endgl

\a\label{ex:wegenpost}\begingl
	\gla des schlechten Wetters wegen //
	\glb des schlecht-en Wetter-s wegen //
	\glc \Def{}.\Gen{}.\N{}.\Sg{} bad-\Gen{}.\N{}.\Sg{} weather-\Gen{} 
		because.of //
	\glft (idem) //
\endgl

\xe

Besides the difference in placement, the morphological properties of 
postpositions are the same as those of prepositions. That is, where 
postpositions are derived from nouns at all, they do not receive case and 
number marking and cannot themselves be modified by adjectives or relative 
clauses. Generally, it is possible for them to be hosts of quantifier clitics 
where semantics permit it.

\index{postpositions|(}

\subsection{Adpositions and time}
\index{adpositions!of time|(}

\begin{figure}[tp]\centering
\caption{Adpositions with temporal meaning}
\begin{tabu} to \linewidth {X X X}
\tableheaderfont\toprule
Adposition
	& Spatial meaning
	& Temporal meaning
	\\

\toprule

\tablesubheaderfont\multicolumn{3}{c}{P r e p o s i t i o n s} \\

\midrule

\rayr{koNF}{kong}
	& inside
	& within
	\\

\rayr{liNF}{ling}
	& on top of
	& while
	\\

\rayr{mrinF}{marin}
	& in front of
	& before
	\\

\rayr{mN lug}{manga luga}
	& through
	& during
	\\

\rayr{mNsh}{mangasaha}
	& towards
	& in + \textit{time}
	\\

\rayr{pN}{pang}
	& behind
	& ago
	\\

\midrule

\tablesubheaderfont\multicolumn{3}{c}{P o s t p o s i t i o n s} \\

\midrule

\rayr{mshtj}{masahatay}
	& ---
	& since
	\\

\rayr{pesnF}{pesan}
	& ---
	& until
	\\

\rayr{pN}{pang}
	& beyond, after
	& after, past
	\\

\bottomrule
\end{tabu}

\label{fig:temppos}
\end{figure}

It has been mentioned above that location also serves as the conceptual 
metaphor for expressing temporal relationships. Notably the prepositions 
\xayr{koNF}{kong}{inside}, \xayr{liNF}{ling}{on}, \xayr{mrinF}{marin}{in 
front of}, \xayr{mN lug}{manga luga}{through}, \xayr{mNsh}{mangasaha}{towards}, 
and \xayr{pNF}{pang}{behind} come to mind as doubling for `within', `while', 
`before', `during', `in + \emph{time}', and `ago', respectively (also see 
\autoref{fig:temppos}). Since postpositions are not primarily derived from 
nouns, there are dedicated forms for expressing temporal relationships, namely, 
\xayr{mshtj}{masahatay}{since}, 
\xayr{pesnF}{pesan}{until}, and as the only form with a double function, 
\xayr{pNF}{pang}{after, past}.

\pex
\a\label{ex:kongtemp}\begingl
	\gla Miranang kong bihanya sam. //
	\glb Mira=nang kong bihan-ya sam //
	\glc do=\Fpl{}.\Aarg{} inside week-\Loc{} two //
	\glft `We will do it within two weeks.' //
\endgl

\a\label{ex:mgshtemp}\begingl
	\gla Girenjang mangasaha pidimya-kay. //
	\glb Girend=yang mangasaha pidim-ya=kay //
	\glc arrive=\TsgM{}.\Aarg{} towards hour-\Loc{}=few //
	\glft `He will arrive in a few hours.' //
\endgl

\a\label{ex:lingtemp}\begingl
	\gla Layaye-ikan ang Pila ling yeng pakur. //
	\glb Laya-ye=ikan ang Pila ling yeng pakur //
	\glc read-\TsgF{}=much \Aarg{} Pila on \TsgF{}.\Aarg{} sick //
	\glft `Pila read a lot while she was sick.' //
\endgl

\xe

Of the examples above, the use of \rayr{koNF}{kong} in (\ref{ex:kongtemp}) is 
probably still closest to a local preposition in that the time span is 
conceptualized as a container, or the distance between two points. The use of 
\rayr{mNsh}{mangasaha} in (\ref{ex:mgshtemp}), on the other hand, is more 
idiomatic. While the prepositions in these two examples each govern an NP, 
example (\ref{ex:lingtemp}) shows that it is also possible for prepositions 
expressing a temporal relationship to govern a subclause. This ability is 
even more prominent with temporal postpositions in that all of the words listed 
above can govern either an NP or a clause, for instance, 
\rayr{mshtj}{masahatay}:

\pex
\a\label{ex:mshtaynp}\begingl
	\gla Ang manga hangya lakayperinya masahatay. //
	\glb Ang manga hang=ya.Ø lakayperin-ya masahatay //
	\glc \AgtT{} \Prog{} stay=\TsgM{}.\Top{} solstice-\Loc{} since //
	\glft He has been staying since the solstice. //
\endgl

\a\label{ex:mshtays}\begingl
	\gla Yeng giday sarayāng masahatay. //
	\glb Yeng giday sara=yāng masahatay //
	\glc \TsgF{}.\Aarg{} sad leave=\TsgM{}.\Aarg{} since //
	\glft `She has been sad since he left.' //
\endgl

\xe

\index{adpositions!of time|)}

\index{adpositions|)}

\section{Verbs}
\index{verbs|(}

Besides nouns, verbs constitute the other main part of speech in Ayeri which 
carries inflections. Verbs show person and number agreement, but may also 
inflect for tense, aspect, mood, and modality as grammatical categories of the 
verb itself. Personal pronouns may furthermore cliticize to the verb stem, and 
the verb phrase is also often marked with a clitic indicating the topic of 
the sentence and the topic NP's role in Ayeri's case system, which can be 
interpreted as a second agreement relation. Further clitics may indicate 
reflexive actions, likeness, logical connection, as well as degree and measure. 
Verbs are thus probably the most versatile part of speech on the one hand, but 
also the one with the heaviest workload on the other. The following sections 
will dissect the morphology of verbs category by category. Since cliticization 
is a phrase-level process \citep{klavans1985}, it will only be touched on 
briefly here. Since verbs inhabit a central position in syntax and exhibit 
agreement morphology, it will be necessary in this section to merge syntax and 
morphology on occasion in order to describe morphosyntactic effects.

\subsection{Person--number marking}
\label{subsec:persnumagr}
\index{agreement!person|(}
\index{agreement!number|(}

\begin{figure}[tp]\centering
\caption[Conjugation paradigm for \xayr{sobF/}{sob-}{learn, teach}]{Conjugation 
paradigm for \xayr{sobF/}{sob-}{learn, teach} (monoconsonantal root)}

\begin{tabu} to \linewidth {X I[2] X[2] I[2] X[2]}
\tableheaderfont\toprule
Person
	& \multicolumn{2}{c}{Singular}
	& \multicolumn{2}{c}{Plural}
	\\

\toprule

\Fsg{}
	& sobay		& `I learn'
	& sobayn	& `we learn'
	\\
	
\Ssg{}
	& sobva		& `you learn'
	& sobva		& `you learn'
	\\
	
\TsgM{}
	& sobya		& `he learns'
	& sobyan	& `they learn'
	\\

\TsgF{}
	& sobye		& `she learns'
	& sobyen	& `they learn'
	\\

\TsgN{}
	& sobyo		& `it learns'
	& sobyon	& `they learn'
	\\

\TsgI{}
	& sobara	& `it learns'
	& sobaran	& `they learn'
	\\

\midrule

\Imp{}
	& sobu!		& \multicolumn{2}{l}{`learn!'}
	\\
	
\Hort{}
	& sobu-sobu!	& \multicolumn{2}{l}{`let's learn!'}
	\\

\Iter{}
	& so-sob-	& \multicolumn{2}{l}{`learn again, relearn'}
	\\
	
\Ptcp{}
	& sobyam	& \multicolumn{2}{l}{`learning'}
	\\
	
\bottomrule

\end{tabu}
\label{fig:monoconsconj}
\end{figure}

\begin{figure}[tp]\centering
\caption[Conjugation paradigm for \xayr{AnFlF/}{anl-}{bring}]{Conjugation 
paradigm for \xayr{AnFlF/}{anl-}{bring} (biconsonantal root)}

\begin{tabu} to \linewidth {X I[2] X[2] I[2] X[2]}
\tableheaderfont\toprule
Person
	& \multicolumn{2}{c}{Singular}
	& \multicolumn{2}{c}{Plural}
	\\

\toprule

\Fsg{}
	& anlay		& `I bring'
	& anlayn	& `we bring'
	\\
	
\Ssg{}
	& anlava	& `you bring'
	& anlava	& `you bring'
	\\
	
\TsgM{}
	& anlya		& `he brings'
	& anlyan	& `they bring'
	\\

\TsgF{}
	& anlye		& `she brings'
	& anlyen	& `they bring'
	\\

\TsgN{}
	& anlyo		& `it brings'
	& anlyon	& `they bring'
	\\

\TsgI{}
	& anlara	& `it brings'
	& anlaran	& `they bring'
	\\

\midrule

\Imp{}
	& anlu!		& \multicolumn{2}{l}{`bring!'}
	\\
	
\Hort{}
	& anlu-anlu!	& \multicolumn{2}{l}{`let's bring!'}
	\\
	
\Iter{}
	& an-anl-	& \multicolumn{2}{l}{`bring again, bring back'}
	\\
	
\Ptcp{}
	& anlyam	& \multicolumn{2}{l}{`bringing'}
	\\
	
\bottomrule

\end{tabu}
\label{fig:biconsconj}
\end{figure}

\begin{figure}[tp]\centering
\caption[Conjugation paradigm for \xayr{no/}{no-}{want}]{Conjugation 
paradigm for \xayr{no/}{no-}{want} (vocalic root)}

\begin{tabu} to \linewidth {X I[2] X[2] I[2] X[2]}
\tableheaderfont\toprule
Person
	& \multicolumn{2}{c}{Singular}
	& \multicolumn{2}{c}{Plural}
	\\

\toprule

\Fsg{}
	& noay		& `I want'
	& noayn		& `we want'
	\\
	
\Ssg{}
	& nova		& `you want'
	& nova		& `you want'
	\\
	
\TsgM{}
	& noya		& `he wants'
	& noyan		& `they want'
	\\

\TsgF{}
	& noye		& `she wants'
	& noyen		& `they want'
	\\

\TsgN{}
	& noyo		& `it wants'
	& noyon		& `they want'
	\\

\TsgI{}
	& noara		& `it wants'
	& noaran	& `they want'
	\\

\midrule

\Imp{}
	& nu!		& \multicolumn{2}{l}{`want!'}
	\\
	
\Hort{}
	& nu-nu!	& \multicolumn{2}{l}{`let's want!'}
	\\
	
\Iter{}
	& no-no-	& \multicolumn{2}{l}{`want again'}
	\\
	
\Ptcp{}
	& noyam		& \multicolumn{2}{l}{`wanting'}
	\\
	
\bottomrule

\end{tabu}
\label{fig:vocconj}
\end{figure}

As described in \autoref{sec:markstrat}, Ayeri conjugates its main verbs, 
canonically in agreement with the agent NP, and verb conjugation as such is 
extremely pervasive. The basic conjugation paradigms are given in Figures 
\ref{fig:monoconsconj}--\ref{fig:vocconj}.\footnote{Due to the agglutinating 
structure of Ayeri, it makes little sense to list the whole paradigm of verb 
inflection for all possible affix combinations here, as the table would be 
unreasonably large. Instead, the various sections below will contain examples of 
use for all affixes.} Agreement causes verbs to reflect grammatical categories 
of nominal entities, thus, verbs show agreement in person (\First{}, \Second{}, 
\Third{}) and number (\Sg{}, \Pl{}); third persons are again differentiated by 
gender (\M{}, \F{}, \N{}, \Inan{}; compare \autoref{subsec:gender}). The 
conjugation suffixes are basically the same as the topic-marked (thus 
superficially unmarked) personal pronouns (see \autoref{subsec:perspro}).

Regarding person--number inflection, verbs may be divided into three classes: 
monoconsonantal, biconsonantal, and vocalic stems. As discussed in 
\autoref{sec:phonotactics}, Ayeri restricts the number of successive non-glide 
consonants to two, which has repercussions in the second person, since the 
conjugation suffix there is \rayr{/v}{-va}. Monoconsonantal roots are 
unaffected by this restriction, however, hence the conjugation suffixes can 
simply be appended as they are; this is illustrated with the verb 
\xayr{sobF/}{sob-}{teach, learn} in \autoref{fig:monoconsconj}. Verb stems 
ending in dental and velar plosives will naturally undergo palatalization in 
the third person animate, so for instance, the third person singular masculine 
of the verb \xayr{gurtF/}{gurat-}{answer} is \xayr{gurtFy}{guraca}{(he) 
answers}, and the third person feminine plural of \xayr{AbgF/}{abag-}{roam, 
wander} is \xayr{AbgFyenF}{abajen}{(they) roam, (they) wander}. Verbs whose 
stem ends in an affricate are treated as monoconsonantal roots as well, since 
the affricate occupies one consonant segment. Thus, the second person of 
\xayr{ItYF/}{ic-}{glide, slide} is not *\rayr{ItYv}{*icava}, but 
\xayr{ItYFv}{icva}{you glide, you slide}.

Since /v/ is neither a vowel nor a glide, as present in the non–second 
person suffixes, an epenthetic \fw{-a-} is inserted between the stem and 
the second-person suffix for verbs whose stem ends in -CC. This is illustrated 
in \autoref{fig:biconsconj} for the verb \xayr{AnFlF/}{anl-}{bring}. The 
second person conjugation of this verb is not *\rayr{AnFlFv}{*anlva}, since 
the cluster \fw{-nlv-} is illegal, but \rayr{AnFlv}{*anlava}. Since Ayeri 
treats two successive instances of the same consonant as a single 
segment---there is no gemination---verbs like \xayr{silFvF/}{silv-}{see} 
conjugate like monoconsonantal roots. That is, the second person of 
\rayr{silFvF/}{silv-} is not *\rayr{silFvv}{*silvava}, as one might expect, but 
\rayr{silFvFv}{silvva}. A further exception to this are verbs ending in -Cs, 
since -Cs-C- is commonly resyllabified as -C-sC- (see \autoref{chap:phonology}, 
\autoref{fn:ssyl}). Thus, the second person conjugation of 
\xayr{krFsF/}{kars-}{freeze} is not *\rayr{krFsv}{*karsava} as expected, but 
\xayr{krFsFv}{karsva}{you freeze}.

Lastly, verb stems may end in a vowel, most commonly \fw{-a}. In these cases as 
well, the conjugation suffixes may simply be appended to the stem. The 
conjugation of this class is illustrated in \autoref{fig:vocconj} with the verb 
\xayr{no/}{no}{want}. Verb stems ending in \fw{-a} undergo the regular vowel 
lengthening process for the first person suffixes, hence, the first person 
singular conjugation of \xayr{Ap/}{apa-}{laugh} is \xayr{Apaaj}{apāy}{I laugh}. 
Verb stems ending in a diphthong in /ɪ/ are essentially treated as a hybrid of 
monoconsonantal and vocalic stems, since the diphthong's final /ɪ/ is treated 
as /j/ before a vowel: \xayr{plyj}{palayay}{I rejoice}, 
\xayr{pljv}{palayva}{you rejoice}.

As mentioned above, the person marking on verbs is essentially the same as the 
topic-marked personal pronouns. This has further ramifications for 
person-marking on verbs, however, insofar as---again, canonically---even fully 
case-marked agent pronouns may act as person marking by means of cliticization. 
Thus, any person-marking on verbs except third-person agreement is, in fact, a 
topicalized pronoun clitic not only by diachronic origin. Unlike English, Ayeri 
does not use agent pronouns with person agreement on verbs. Consider these two 
example sentences in English:

\pex
\a\label{ex:vbagrengnn}\begingl\rc{English}%
	\gla John greets Mary. //
	\glb John greet-s Mary //
	\glc John greet-\Tsg{}.\Prs{} Mary //
\endgl

\a\label{ex:vbagrengpr}\begingl
	\gla He greets Mary. //
	\glb He greet-s Mary //
	\glc \TsgM{} greet-\Tsg{}.\Prs{} Mary //
\endgl

\xe

In these examples, the verb has an agreement suffix \fw{-s} which indicates 
third person singular, present tense, whether the subject of the sentence is a 
noun (\fw{John}) or a pronoun (\fw{he}), which acts as a free morpheme in 
English. Now consider the Ayeri equivalents of these two examples, on the other 
hand:\footnote{Most of the following account is taken nearly verbatim from a 
previously published blog article, \citet{benung:verbagreement}. Some of the 
Ayeri examples used in the following come from a list of samples I provided for 
a bachelor's thesis at the University of Kent in March 2016, in private 
conversation, on request.%
% I do not know what the author made of them---the questionnaire I filled out 
% initially indicated that the thesis was probably on the syntactic typology of 
% fictional languages regarding typical word-order correlations (VO correlating 
% with head-first order etc.). I hope that my reflections here do not preempt or 
% invalidate the author's analyses should they still be in the process of writing 
% or their submitted thesis be in the process of evaluation and grading. I would 
% certainly like to learn about their analysis of my examples.
}

\pex % (1)
\a\label{ex:vbagr}\begingl
	\gla Ang manya {} Ajān sa Pila. //
	\glb Ang man-ya Ø Ajān sa Pila //
	\glc \AgtT{} greet-\TsgM{} \Top{} ​Ajān[\TsgM{}] \Parg{} Pila[\TsgF{}]//
	\glft `Ajān greets Pila.' //
\endgl

\a\label{ex:vbclt}\begingl
	\gla Ang manya sa Pila. //
	\glb Ang man=ya.Ø sa ​Pila //
	\glc \AgtT{} greet=\TsgM{}.\Top{} \Parg{} ​Pila[\TsgF{}] //
	\glft `He greets Pila.' //
\endgl

\xe

It is probably uncontroversial to analyze \rayr{/y}{-ya} in (\ref{ex:vbagr}) as 
person agreement: \rayr{AgYaanF}{Ajān} is a male name in Ayeri while 
\rayr{pil}{Pila} is a feminine one; the verb inflects for a masculine third 
person, which tells us that it agrees with the one doing the greeting, 
Ajān. Ajān is also who this is about, which is shown on the verb by marking for 
an agent topic. In the second case, there is only anaphoric reference to Ajān, 
so you might say that the agent NP is left out, so very broadly, the verb 
marking here seems to be like in Spanish, where you can drop the subject 
pronoun:
% \footnote{However, we will see that it is probably more complicated than 
% this.}

\pex % (2)
\a\label{ex:vbagrspann}\begingl\rc{Spanish}%
	\gla Juan saluda a María. //
	\glb Juan salud-a a María //
	\glc John greet-\Tsg{} \Acc{} Mary //
	\glft `John greets Mary.' //
\endgl

\a\label{ex:vbagrspapr}\begingl
	\gla Saluda a María. //
	\glb Salud-a a María //
	\glc greet-\Tsg{} \Acc{} Mary //
	\glft `He greets Mary.' //
\endgl

\xe

Example (\ref{ex:vbclt}) probably probably does not seem conspicious either, 
except that there is also topic marking for an agent there, the controller of 
which I have so far assumed to be the person inflection on the verb, in analogy 
with examples like the following:

\ex\label{ex:lampyaang} % (3)
\begingl
	\gla Lampyāng. //
	\glb Lamp=yāng //
	\glc walk=\TsgM{}.\Aarg{} //
	\glft `He walks.' //
\endgl\xe

This raises the question whether in Ayeri, there is dropping of an agent 
pronoun involved at all, which is why the person suffix in (\ref{ex:vbclt}) was 
glossed as \emph{-ya.Ø} (\mbox{-\TsgM{}.\Top{}}) rather than just as \emph{-ya} 
(-\TsgM{}). In turn, this question leads us to consider another characteristic 
of Ayeri, namely that the topic morpheme on noun phrases is zero. That is, the 
absence of overt case marking on a nominal element indicates that it is a 
topic; the verb in turn marks the case of the topicalized NP with a (case) 
particle preceding it. Pronouns as well show up in their unmarked form when 
topicalized, which is why I am hesitant to analyze the pronoun in 
(\ref{ex:protop}) as a clitic on the VP rather than an independent 
morpheme:\footnote{Also, perhaps a little untypically, topic NPs in Ayeri are 
not usually pulled to the front of the phrase (at least not in the written 
language; see \citet[120--122]{lehmann2015}), so topic-marked pronouns stay 
in-situ. Which NP constitutes the topic of the phrase is marked on the verb 
right at the head of the clause. How and whether this can be justified in terms 
of grammatical weight (see, for instance, \citet[95--98]{wasow1997}) remains to 
be seen.}

\pex % (4)
\a\label{ex:fullsntc}\begingl
	\gla Sa manya ang Ajān {} Pila. //
	\glb Sa man-ya ang ​Ajān Ø ​Pila //
	\glc \PatT{} greet-\TsgM{} \Aarg{} ​Ajān \Top{} ​Pila //
	\glft `It's Pila that Ajān greets.' //
\endgl

\a\label{ex:protop}\begingl
	\gla Sa manyāng ye. //
	\glb Sa man=yāng ye.Ø //
	\glc \PatT{} greet=\TsgM{}.\Aarg{} \TsgF{}.\Top{} //
	\glft `It's her that he greets.' //
\endgl

\xe

What is remarkable, then, is that \rayr{ye}{ye} (\TsgF{}.\Top{}) is the very 
same form that appears as an agreement morpheme on the verb, just like 
\rayr{/y}{-ya} (\TsgM{}) in various examples above (also compare the examples 
in \autoref{subsec:perspro}):

\ex % (5)
\begingl
	\gla Ang purivaye yāy. //
	\glb Ang puriva=ye.Ø yāy //
	\glc \AgtT{} smile=\TsgF{}.\Top{} \TsgM{}.\Loc{} //
	\glft `She smiles at him.' //
\endgl\xe

This also holds for all other personal pronouns. Moreover, 
\rayr{/yaaNF}{-yāng} as seen in examples (\ref{ex:lampyaang}) and 
(\ref{ex:protop}) may also be used as a free pronoun, as well as other such 
case-marked personal forms:

\pex % (6)
\a\begingl
	\gla Yeng mino. //
	\glb Yeng mino //
	\glc \TsgF{}.\Aarg{} happy //
	\glft `She is happy.' //
\endgl
	
\a\begingl
	\gla Yāng naynay. //
	\glb Yāng naynay. //
	\glc \TsgM{}.\Aarg{} too //
	\glft `He is, too.' //
\endgl

\xe

\phantomsection\label{patagr}
As for case-marked person suffixes on verbs, the assumption so far has been 
that they are essentially clitics, especially since the following marking 
strategy is the grammatical one in absence of an agent NP:

\pex\label{ex:passive} % (7)
\a\label{ex:manye}\begingl
	\gla Manye sa Pila. //
	\glb Man-ye sa ​Pila //
	\glc greet-\TsgF{} \Parg{} ​Pila //
	\glft `Pila is being greeted.' //
\endgl
	
\a\label{ex:manyes}\begingl
	\gla Manyes. //
	\glb Man=yes. //
	\glc greet=\TsgF{}.\Parg{} //
	\glft `She is being greeted.' //
\endgl

\xe

The verb here agrees with the patient---or is it that person agreement 
suffixes on verbs are generally clitics in Ayeri, even where they do not
involve case marking? There seems to be a gradient here between what looks 
like regular verb agreement with the agent on the one hand, and agent or 
patient pronouns just stacked onto the verb stem on the other hand. For an 
overview, compare \autoref{fig:persinfltypes}.

\afterpage{%
\clearpage% Flush earlier floats (otherwise order might not be correct)
\begin{landscape}\centering
\mbox{}\vfill
\captionof{figure}{Verb inflection types in Ayeri}
~\\
\begin{tabu} to \linewidth{H[2l] X[4] X[4] X[4]}
\tableheaderfont\toprule
%
	& Type I: Clitic pronouns
	& Type II: Transitional
	& Type III: Verb agreement
\\

\toprule

Inflectional categories
	& Person\newline
		Number\newline
		Case
	%
	& Person\newline
		Number\newline
		Case/Topic
	%
	& Person\newline
		Number
\\

\midrule

Examples (intransitive)
	& …=yāng\newline
		…=\TsgM{}.\Aarg{}
	%
	& ---
	%
	& …-ya₁ …-ang₁\newline
		…-\TsgM{} …-\Aarg{}
\\

\midrule

Examples (transitive)
	& sa₁ …=yāng …-Ø₁\newline
		\PatT{} …=\TsgM{}.\Aarg{} …-\Top{}
	%
	& ang₁ …=ya.Ø₁ …-as\newline
		\AgtT{} …=\TsgM{}.\Top{} …-\Parg{}
	%
	& \begin{tabu} to \linewidth {X[1] X[9]}
		a.	& ang₁ …-ya₁ …-Ø₁ …-as\newline
			\AgtT{} …-\TsgM{} …-\Top{} …-\Parg{}
		\\
		
		b.	& a₁ …-ya₂ …-ang₂ …-Ø₁\newline
			\PatT{} …-\TsgM{} …-\Aarg{} …-\Top{}
		\\
	\end{tabu}
\\

\bottomrule

\end{tabu}
\label{fig:persinfltypes}
\mbox{}\vfill
\end{landscape}
\clearpage
}

In this figure, especially the middle, transitional category is interesting in 
that what looks like verb agreement superficially can still govern 
topicalization marking, which is indicated in column II by an index `1'. Note 
that this behavior only occurs in transitive contexts; there is no topic 
marking on the verb if the verb only has a single NP dependent. Note that for 
example (b) in the type III transitive cell the question is, whether this should 
not better be analyzed as \AgtT{} …-\TsgM{}.\Top{} …-\Top{} …-\Parg{}, with 
co-indexing of the topic on the person inflection of the verb, making it 
structurally closer to type II.

As for personal pronouns fused with the verb stem like in the first column, 
\citeauthor{corbett2006} points out that

\blockcquote[99--100]{corbett2006}{In terms of syntax, pronominal affixes are 
arguments of the verb; a verb with its pronominal affixes constitutes a full 
sentence, and additional noun phrases are optional. If pronominal affixes are 
the primary arguments, then they agree in the way that anaphoric pronouns agree 
[…] In terms of morphology, pronominal affixes are bound to the verb; typically 
they are obligatory […].}

This seems to be exactly what is going on for instance in (\ref{ex:lampyaang}) 
and (\ref{ex:manyes}), where the verb forms a complete sentence. It needs to be 
pointed out that Corbett includes an example from Tuscarora, a native American 
polysynthetic language, in relation to the above quotation. Ayeri should not be 
considered polysynthetic, however, since its verbs generally do not exhibit 
relations with multiple NPs, at least as far as person and number agreement is 
involved \citep[45--46]{comrie1989}.\footnote{The topic NP marked on the verb 
may be a different from the one with which it agrees in person and number, so 
technically, Ayeri verbs \emph{may} agree with more than one NP in a very 
limited way (compare \autoref{sec:markstrat}). Still, I would not analyze this 
as polypersonal agreement, since there is only canonical verb agreement with 
one constituent. Topic marking should, in my opinion, be viewed as a separate 
agreement relation, as pointed out in the quoted section above.}

Taking everything written above so far into account, it looks much as though 
Ayeri is in the process of grammaticalizing personal pronouns into person 
agreement \parencites[42--45]{lehmann2015}[493--497]{vangelderen2011}. 
\citet[76--77]{corbett2006} illustrates an early stage of such a process:

\pex % (8)
\a\begingl\rc{Skou}%
	\gla Ke móe ke=fue. \quad {\textup{(*}​Ke móe fue.\textup{)}} //
	\glb \TsgM{} fish \TsgM{}=​see.\TsgM{} {} //
	\glft `He saw a fish.' //
\endgl

\a\begingl
	\gla Pe móe pe=fu. \quad {\textup{(*}​Pe móe fu.\textup{)}} //
	\glb \TsgF{} fish \TsgF{}=​see.\TsgF{} {} //
	\glft `She saw a fish.' //
\endgl

\xe

What \citeauthor{vangelderen2011} calls the \emph{subject cycle}, the 
\textcquote[493]{vangelderen2011}{oft-noted cline expressing that pronouns can 
be reanalyzed as clitics and agreement markers} applies here, and as well in 
Ayeri. However, while she continues to say that in 
\textcquote[494]{vangelderen2011}{many languages, the agreement affix resembles 
the emphatic pronoun and derives from it}, Ayeri does at least in part the 
opposite and uses the case-unmarked, unstressed form of personal pronouns for 
what resembles verb agreement most closely. This, however, should not be too 
controversial either, considering that, for instance, semantic bleaching and 
phonetic erosion go hand in hand with grammaticalization 
\parencites[136--137]{lehmann2015}[497]{vangelderen2011}.

As pointed out above in (\ref{ex:passive}), Ayeri usually exhibits verbs as 
agreeing with agents and occasionally patients, not topics as such. This may be 
a little counterintuitive since the relation between topics and subjects is 
close, but is possibly due to the fact that the unmarked word order is VAP. This 
means that agent NPs usually follow the verb, and it strikes me as not too 
unnatural to have an agreement relation between the verb and the closest NP also 
when non-conjoined NPs are involved \citep[180]{corbett2006}. This conveniently 
explains why verbs can agree with patients as well if the agent NP is absent. 
Taking into account that the grammaticalization process is still ongoing so that 
there is still some relative freedom in how morphemes may be used if a paradigm 
has not yet fully settled \citep[148--150]{lehmann2015} also makes this seem 
less strange. Verbs simply become agreement targets of the closest semantically 
plausible nominal constituent. Ayeri seems to be shifting from topics to 
subjects, and as a consequence the bond between agents and verbs is strengthened 
due to their usual adjacency; developing verb agreement with agents may be seen 
as symptomatic of this change.\footnote{When translating things in Ayeri, I find 
myself very often using agent topics, which may be because I am used to subjects 
proper. Supposing that this is also what Ayeri prefers in-universe, it would 
make sense to assume the usual grammaticalization path by which topics become 
subjects, thereby also leading to subject-verb agreement by means of resumptive 
pronouns referring back to left-dislocated topics 
\parencites[121--122]{lehmann2015}[499--500]{vangelderen2011}. 
\citet[120]{lehmann2015} gives colloquial French \fw{Jean, je l'ai vu hier} 
`John, I saw him yesterday' as an example here: the object clitic \fw{l'} (← 
\fw{le} `him') may well develop into an agreement affix (also see 
\citet[498]{vangelderen2011} on a dialect of Spanish in which, she argues, this 
has happened).}

\begin{figure}[tp]\centering
\caption[The syntax and morphology of pronominal affixes]{The syntax and 
morphology of pronominal affixes \citep[101]{corbett2006}}

%\tabulinesep=3pt
\begin{tabu} to \linewidth {H[l,m] | X[c,m] | X[c,m] | X[c,m]}

\tabucline[1pt]{1-4}

Syntax:\bigstrut
	& non-argument\bigstrut
	& \multicolumn{2}{c}{argument\bigstrut}
\\

\hline

Linguistic element:\bigstrut
	& `pure' agreement marker
	& pronominal affix\bigstrut
	& free pronoun\bigstrut
\\

\hline

Morphology:
	& \multicolumn{2}{c|}{inflectional form\bigstrut}
	& free form\bigstrut
\\

\tabucline[1pt]{1-4}

\end{tabu}
\label{ex:typproaffx}
\end{figure}

Signs so far point towards Ayeri's person agreement in fact being more 
likely enclitic pronominal affixes, even what I had been thinking of as 
person agreement before (that is, suffixes on the verb that only encode person 
and number, but not case). The question is, then, how this might be corrobated. 
\citeauthor{corbett2006} offers a typology here, see \autoref{ex:typproaffx}. 
According to this typology, a pronominal affix is syntactically an argument of 
the verb but has the morphology of an inflectional form. If we compare this to 
the gradient given in table 1 above, it becomes evident that I definitely 
fulfills these criteria, and II does so as well, in fact, in that there is no 
agent NP that could serve as a controller if the verb inflection in II were 
`merely' a agreement target. The inflection in III, on the other hand, appears 
to have all hallmarks of agreement in that there is a controller NP that 
triggers it, with the verb serving as an agreement target. Moreover, the person 
marking on the verb is not a syntactic argument of the verb. As example 
(\ref{ex:manye}) shows, however, marking of type III permits the verb to mark 
more than one case role, which makes it slightly atypical, although verbs can 
only carry a single instance of person marking \citep[103]{corbett2006}. 
Regarding referentiality, the person suffixes on the verb in table 1, columns I 
and II are independent means of referring to discourse participants mentioned 
earlier, whereas the person suffix in III needs support from an NP in the same 
clause as a source of morphological features to share:

\pex % (9)
\a\label{ex:agttopclit}\begingl
	\gla Ajān … Ang manya sa Pila. //
	\glb Ajān … Ang man=ya.Ø sa ​Pila //
	\glc Ajān … \AgtT{} greet=\TsgM{}.\Top{} \Parg{} ​Pila //
	\glft `Ajān … He greets Pila.' //
\endgl

\a\label{ex:agtproclit}\begingl
	\gla Ajān … Sa manyāng {} Pila. //
	\glb Ajān … Sa man=yāng Ø ​Pila //
	\glc Ajān … \PatT{} greet=\TsgM{}.\Aarg{} \Top{} ​Pila //
	\glft `Ajān … It's Pila that he greets.' //
\endgl

\a\label{ex:wrongagr}\ljudge* \begingl
	\gla Ajān … Manya sa Pila. //
	\glb Ajān … Man-ya sa ​Pila //
	\glc Ajān … greet-\TsgM{} \Parg{} ​Pila //
\endgl

\xe

Since person marking of the type I and II is \emph{referential}, as shown 
in example (\ref{ex:agttopclit}) and (\ref{ex:agtproclit}), it can be counted as 
a cliticized pronoun \citep[103]{corbett2006}. Pronouns in Ayeri can also refer 
to non-people---there are both a `neuter' gender for non-people considered 
living (or being closely associated with living things), and an `inanimate' 
gender for the whole rest of things (compare \autoref{subsec:gender}). Since 
mere agreement as in type III needs support from an NP within the verb's scope, 
though, it does not have \emph{descriptive/lexical content} of its own. That 
is, it \emph{only} serves a grammatical function \citep[104]{corbett2006}. As 
for \citeauthor{corbett2006}'s \emph{balance of information} criterion, 
\autoref{fig:persinfltypes} also highlights differences in what information is 
provided by the person marking. Nouns in Ayeri inherently bear information on 
person, number, and gender, and all three types of person inflection on verbs 
share these features. However, there are no extra grammatical features indicated 
by the first two inflection types that are not expressed by noun phrases, 
although under a very close understanding of \citeauthor{corbett2006}, the 
following example (\ref{ex:verbplagr}) may still qualify as person-marking on 
the verb realizing a grammatical feature shared with an NP that is not openly 
expressed by the NP. He writes that in the world's languages, this frequently is 
number \citep[105]{corbett2006}. This, however, does not apply to Ayeri because 
the only time that verbs display number not expressed overtly by inflection on a 
noun is in agreement like in type III\,(a):

\ex\label{ex:verbplagr} % (10)
\begingl
	\gla Ang sahayan ayon kay kong nangginoya. //
	\glb Ang saha-yan ayon-Ø kay kong nanggino-ya //
	\glc \AgtT{} come-\TplM{} man-\Top{} three into tavern-\Loc{} //
	\glft `Three men come into a pub.' //
\endgl\xe

As shown above, verb marking of the types I and II is independent as a 
reference, so there is \emph{unirepresentation} of the marked NP. In contrast, 
verb marking of type III requires a controlling NP in the same clause to share 
grammatical features with, so that there is \emph{multirepresentation} typical 
of canonical agreement \citep[106]{corbett2006}. Note that unirepresentation as 
outlined here is probably different from pro-drop, as in this case I would 
expect sentences like (\ref{ex:wrongagr}) to be grammatical 
\citep[107]{corbett2006}. A further property that hinges on types I and II being 
independent pronouns tacked onto verbs as clitics is that they are not 
coreferential with another NP of the same grammatical relation, but in 
complementary distribution, as commonly assumed with pronominals 
\citep[108]{corbett2006}. Hence, either of these two examples is ungrammatical:

\pex % (11)
\a\ljudge* \begingl
	\gla Lampyāng ang Ajān. //
	\glb Lamp=yāng ang ​Ajān //
	\glc walk=\TsgM{}.\Aarg{} \Aarg{} Ajān //
\endgl

\a\ljudge* \begingl	
	\gla Ang lampyāng {} Ajān. //
	\glb Ang lamp=yāng Ø ​Ajān //
	\glc \AgtT{} walk=\TsgM{}.\Aarg{} \Top{} ​Ajān //
\endgl

\xe

However, verb agreement with a pronoun is also not possible even though it 
would be expectable according to \citep[109]{corbett2006}---also compare 
example (\ref{ex:vbagrengpr}) above:

\pex % (12)
\a\begingl
	\gla Lampyāng. //
	\glb Lamp-yāng //
	\glc walk=\TsgM{} //
	\glft `He walks.' //
\endgl

\a\ljudge* \begingl	
	\gla Lampya yāng. //
	\glb Lamp-ya yāng //
	\glc walk-\TsgM{} \TsgM{}.\Aarg{} //
\endgl

\xe

In conclusion, we may assert that Ayeri appears to be in the process of 
grammaticalizing pronouns as verb infletions, however, how far this 
grammaticalization process has progressed is dependent on syntactic context. 
Ayeri displays a full gamut from personal pronouns (usually agents) tacked on 
verbs as clitics to agreement with coreferential NPs that is transparently 
derived from these personal pronouns. With the latter, there is the 
complication that coreferential pronoun NPs are not allowed as one might 
expect, but only properly nominal ones. Slight oddities with regards to 
Austronesian alignment---Ayeri's actors bear more similarities to subjects than 
expected, but still without fully conflating the two notions---can possibly be 
explained by a strengthening of the verb-agent relationship pointed out as a 
grammaticalization process in this article as well. Information on agreement 
with committee nouns and coordinated NPs with incongruent agreement features 
can be found in the section on VPs.

\index{agreement!number|)}
\index{agreement!person|)}

\subsection{Tense}
\index{tense|(}

Tense in Ayeri is often not explicitly marked, but has to be inferred from 
context. However, where marked, Ayeri distinguishes past and future as 
referring to past and future events, respectively. Both past and future 
tenses come with three degrees each: near, recent/impending, and remote. 
Its distinguishing three degrees of both past and future time is a little 
unusual with regards to typology according to the survey conducted by 
\citet[127]{dahl1985}. The decision for which subtier of the past and the 
future to use is up to pragmatics, that is, there are no definitive 
and clear-cut lines. The near-time markers are most commonly used for 
immediate scope, that is, things which have just happened or will happen in a 
moment. The recent/impending-time markers may then be used for anything else 
which does not qualify as remote, that is, a long time into the past or the 
future from the point of view of the speaker.

\citet[117]{dahl1985} further notes that among the languages in the surveyed 
sample, past tenses are mostly marked by suffixes, the marking of this 
category being extremely common in addition. Ayeri may thus be a little unusual 
crosslinguistically again in this regard by exclusively using prefixes for 
tense marking. This makes sense, however, if we assume that historically, the 
tense prefixes once were auxiliary verbs. Ayeri applies head-first word 
order to verbs with verbal complements, as we will see further below, so these 
prefixes may just have begun to \emph{pro}cliticize instead of slipping 
into a position behind their head (that is, Wackernagel's position).

Tense and aspect work closely together also in Ayeri, so this section will only 
cover basic uses of the marked tense categories, and the next section will go 
deeper into aspectual categories and how they interact with tense marking.

\subsubsection{Present tense}
\index{tense!present|(}
Verbs in Ayeri are unmarked for present tense, as it is the normal mode of 
speaking. Besides being used to comment or report on current events, the 
present tense is also used to make statements of general truth:

\ex\begingl
	\gla Sa arapyo tahanyamanang koyana nogalam-ikan. //
	\glb Sa arap-yo tahanyaman-ang koya-na nogalam-Ø=ikan //
	\glc \PatT{} require-\TsgN{} writing-\Aarg{} book-\Gen{} 
		patience-\Top{}=much //
	\glft `Writing a book requires much patience.' //
\endgl\xe

Moreover, Ayeri does not strictly mark its verbs for past tense in narrative 
discourses---verbs may thus appear as though with a present-time reference in 
spite of recounting past events, whether historical or fictional. See the next 
subsection on the past tense.

\index{tense!present|)}

\subsubsection{Past tense}
\index{tense!past|(}
The past tense indicates actions in the past if not further modified. 
The three degrees of past tense are marked with \rayr{k/}{kə-} 
(near/immediate), \rayr{m/}{mə-} (recent), and \rayr{v/}{və-} (remote), which 
attach right in front of a verb root. In spite of the customary spelling of the 
past tense prefixes with \orth{ə}, which reflects pronunciation, they have an 
underlying /a/ vowel in this place. This means that the vowel of the tense 
prefixes coalesces with a following a to form a long vowel (see 
\autoref{subsec:vowels}), which is demonstrated in example (\ref{ex:pst}) 
below:\index{allomorphy}

\pex
\a\label{ex:npst}\begingl
	\gla Ang kəsilvay yes motonya. //
	\glb Ang kə-silv=ay.Ø yes moton-ya //
	\glc \AgtT{} \NPst{}-see=\Fsg{}.\Top{} \TsgF{}.\Parg{} store-\Loc{} //
	\glft `I've just seen her at the store.' //
\endgl

\a\label{ex:pst}\begingl
	\gla Le mādruyāng ikan biratay. //
	\glb Le mə-adru=yāng ikan biratay-Ø //
	\glc \PatTI{} \Pst{}-break=\TsgM{}.\Aarg{} wholly pot-\Top{} //
	\glft `The pot, he completely broke it.' //
\endgl

\a\label{ex:rpst}\begingl
	\gla Vəmittang edaya. //
	\glb Və-mit=tang edaya //
	\glc \RPst{}-live=\TplM{}.\Aarg{} here //
	\glft `They lived here (a long time ago)' //
\endgl

\xe

Note that the recent and the remote past tense are not generally marked if the 
past context is clear, for instance, when a past context has already been 
established in discourse. This may also happen explicitly by using a time 
adverbial such as \xayr{tml}{tamala}{yesterday} or \xayr{perikYnFy menNF 
pNF}{pericanya menang pang}{a hundred years ago}. In the presence of an 
explicit time adverbial, redundant tense marking is also dropped subsequently:

\ex\begingl
	\gla Ang kondayn kadanya terpasānley bihanya sarisa. //
	\glb Ang kond=ayn.Ø kadanya terpasān-ley bihan-ya sarisa //
	\glc \AgtT{} eat=\Fpl{}.\Top{} together lunch-\PargI{} week-\Loc{}
		previous //
	\glft `We had lunch together last week.' //
\endgl\xe

The reference to a past time frame is explicitly given in this example by the 
adverbial phrase \xayr{bihnFy sris}{bihanya sarisa}{last week}, hence the verb 
appears here simply as \rayr{koMdjnF}{kondayn}, rather than with redundant 
past-tense marking as \rayr{mkoMdjnF}{məkondayn}.

Since past tense is often underspecified in Ayeri, the language also does not 
employ epic past forms in narrative contexts like English, among others, 
commonly does: 

\ex\label{ex:neuromancer}
	The sky above the port was the color of television, tuned to a dead 
	channel. \tc{\citep[9]{gibson:neuromancer}}
\xe

This quote is, of course, the first sentence of 
\citeauthor{gibson:neuromancer}'s novel \citetitle{gibson:neuromancer}, which 
never mentions any definite dates, but is clearly set in a future world, maybe 
somewhere around the middle of the twenty-first century. Yet, however, 
\citeauthor{gibson:neuromancer} recounts events which are logically happening in 
an imagined future as having already happened in the past: he uses the past 
tense as a convention of storytelling. What Ayeri, then, does in contrast to 
English is to basically treat stories as though happening in the present; 
adverbials referring to past time may, again, set up the correct time frame if 
required. Ayeri is in good company here, since according to 
\citeauthor{dahl1985} \textcquote[113]{dahl1985}{[m]ore common than marking 
narrative contexts [...] is not marking them---quite a considerable number of 
languages use unmarked verb forms in narrative contexts}. This, however, is yet 
different from a narrative present, that is, the use of present tense 
within a past context, which languages like English may use in narrative 
contexts to increase the feeling of immediacy and thus raise suspense. The 
following example from an Ayeri translation of the well-known Aesopian fable, 
`\citetitle{aesop:northwind}' (compare \cite{aesop:northwind}), 
illustrates Ayeri's non-marking of tense on verbs in narrative contexts:

\ex
\begingl
	\gla Ang manga ranyon adauyi {} Pintemis nay {} Perin, engyo mico 
		sinyāng luga toya, lingya si lugaya asāyāng si sitang-naykonyāng 
		kong tovaya mato. //
	\glb Ang manga ran-yon adauyi Ø  Pintemis nay Ø Perin, eng-yo mico 
		sinya-ang luga toya, ling-ya si luga-ya asāya-ang si 
		sitang=naykon=yāng kong tova-ya mato. //
	\glc \AgtT{} \Prog{} argue-\TplN{} then \Top{} {North Wind} and 
		\Top{} Sun, be.more-\TsgN{} strong who-\Aarg{} among 
		\TplN{}.\Loc{}, while-\Loc{} \Rel{} pass-\TsgM{} 
		traveler-\Aarg{} \Rel{} self=wrap=\TsgM{}.\Aarg{} inside 
		cloak-\Loc{} warm. //
	\glft `The North Wind and the Sun were then arguing which among them is 
		stronger, all the while a traveler passed by who had wrapped 
		himself in a warm cloak.' //
\endgl
\xe

\index{tense!past|)}

\subsubsection{Future tense}
\index{tense!future|(}
Future tense marks explicit reference to future time in Ayeri, that is, future 
\textcquote[103]{dahl1985}{someone's plans, intentions or obligations}, as well 
as predictions. The future prefixes behave analogous to the ones indicating 
past tense: \rayr{p/}{pə-} indicates immediate/near future (\NFut{}), 
\rayr{se/}{sə-} indicates impending future (\Fut{}), and \rayr{ni/}{ni-} 
indicates remote future (\RFut{}). Underlying the reduced vowels in 
\rayr{p/}{pə-} and \rayr{se/}{sə-} are /a/ and /e/, respectively, so that these 
prefixes cause adjacent vowels of the same type to lengthen as 
usual;\index{allomorphy} the same, of course, applies to \rayr{ni/}{ni-} 
regarding /i/. The following examples show the future tense markers in context:

\pex
\a\label{ex:nfut}\begingl
	\gla Pəsahayang! //
	\glb Pə-saha=yang //
	\glc \NFut{}-come=\Fsg{}.\Aarg{} //
	\glft `I'm coming (in a moment)!' //
\endgl

\a\label{ex:fut}\begingl
	\gla Ang səkarsayn kankaya. //
	\glb Ang sə-kars=ayn.Ø kanka-ya //
	\glc \AgtT{} \Fut{}-freeze=\Fsg{}.\Top{} snow-\Loc{} //
	\glft `We will freeze in the snow.' //
\endgl

\a\label{ex:rfut}\begingl
	\gla Paronatang, nisa-sahaya dihakayāng. //
	\glb Parona=tang ni-sa\til{}saha-ya dihakaya-ang //
	\glc believe=\TplM{}.\Aarg{} \RFut{}-\Iter{}\til{}come-\TsgM{} 
		prophet-\Aarg{} //
	\glft `They believe that the prophet will return (one day).' //
\endgl

\xe

Like the past tense, the future is often not explicitly marked if the time 
frame is clear from context or has been clarified with such adverbials as 
\xayr{tsel}{tasela}{tomorrow}, \xayr{mNsh perikYnFy}{mangasaha pericanya}{in 
a year}, or \xayr{metj}{metay}{sometime}:

\ex\begingl
	\gla Ang raypāy vaya bihanya mararya. //
	\glb Ang raypa=ay.Ø vaya bihan-ya mararya //
	\glc \AgtT{} stop=\Fsg{}.\Top{} \Ssg{}.\Loc{} week-\Loc{} next //
	\glft `I'm stopping by you next week.' //
\endgl\xe

It is possible here to explicitly mark the verb for future tense as well, for 
example, to make a promise, or to otherwise emphasize that the future condition 
will come to pass:

\ex\begingl
	\gla Səsidejang tasela, diran. //
	\glb Sə-sideg=yang tasela diran //
	\glc \Fut{}-repair=\Fsg{}.\Aarg{} tomorrow uncle //
	\glft `I \fw{will} repair it tomorrow, uncle.' //
\endgl\xe

\index{tense!future|)}

\index{tense|)}

\subsection{Aspect}
\index{aspect|(}

Aspectually unmarked verb forms indicate general statements, which may be 
completed or ongoing, depending on the meaning of the verb itself. Ayeri 
seems not to make strict formal distinctions with regards to either 
perfectivity, or lexical aspect. It needs to be noted, however, that at least 
to date, it is not entirely clear how Ayeri fares with regards to 
conceptualizing perfectivity, which \citet[76]{dahl1985} in reference to 
\citet[16]{comrie1976} characterizes as being based on the conceptualization of 
actions or events as bounded or otherwise limited wholes, versus a lack of 
closure. \citeauthor{dahl1985} also notes that \textcquote[69]{dahl1985}{it 
seems rather to be a typical situation that even in individual languages, we 
cannot choose one member of the opposition [perfective--imperfective] as being 
clearly unmarked}. He further argues that the 

\blockcquote[73]{dahl1985}{difficulty of deciding wich member of the 
opposition is marked and which is unmarked is connected with the tendency for 
PFV:IPFV to be realized not by affixation or by periphrastic constructions but 
rather by less straightforward morphological processes.}

In other words: it \emph{is} a difficult category to assess, in spite of being 
\textcquote[69]{dahl1985}{often taken to be \enquote{the} category of aspect}, 
mostly since languages often do not reify it by straightforward means. In 
Ayeri, the most tangible way of expressing completeness of an action is 
to use adverbs like \xayr{myis}{mayisa}{ready, done}, \xayr{Iri}{iri}{already}, 
\xayr{IknF}{ikan}{completely, wholly} (also as an adjective); a quantifier 
like \xayr{/henF}{-hen}{all}; verbs like \xayr{smirF/}{samir-}{finish}, 
\xayr{pN/}{panga-}{end}, and \xayr{rjp/}{raypa-}{stop}; or an indefinite pronoun 
like \xayr{EnY}{enya}{everything, everybody}:

\ex\begingl
	\gla Le kondjeng enya. //
	\glb Le kond=yeng enya-Ø //
	\glc \PatTI{} eat=\TsgF{}.\Aarg{} everything //
	\glft `She ate everything.' or: `She ate it all up.' //
\endgl\xe

Apart from the more general dilemma of determining how perfectivity is expressed 
in detail, Ayeri marks verbs openly by morphological means to indicate 
progressive, habitual, and iterative actions---by their nature all 
conceptualizing actions as being composed of a series of two or more related 
actions of the same kind, though not necessarily implying a strong semantic 
connection to the past. The following sections will discuss each of these 
categories.

\subsubsection{Progressive}
\index{aspect!progressive|(}

In order to indicate an ongoing action explicitly, Ayeri employs the marker 
\rayr{mN}{manga}, which we have already seen with dynamic prepositions above 
(\autoref{manga}). This marker is a bound morpheme within the verb phrase and 
precedes the verb word:

\ex\label{ex:presprog}\begingl
	\gla Ang manga ilye karonas nakajyam. //
	\glb Ang manga il=ye.Ø karon-as naka-ye-yam //
	\glc \AgtT{} \Prog{} give=\TsgF{}.\Top{} water-\Parg{} 
		plant-\Pl{}-\Dat{} //
	\glft `She is giving water to the plants.' //
\endgl\xe

Going by the data presented by \citet[91]{dahl1985}, Ayeri is typologically 
unremarkable in marking progressive aspect with a periphrastic construction, 
although it is remarkable in possessing morphological progressive marking at 
all---it only occurs in 27\pct{} of the languages in \citeauthor{dahl1985}'s 
sample. Typical of progressives, this form of the verb is not limited to 
present contexts in Ayeri as exemplified in (\ref{ex:presprog}) above. Instead, 
it is possible to also use the progressive in past (\ref{ex:pastprog}) and 
future (\ref{ex:futprog}) contexts, the latter being probably less typical, 
though:

\pex\label{ex:nonpresprog}
\a\label{ex:pastprog}\begingl
	\gla Ang manga gumya {} Ajān tadayya si ya kongaye ang Pila gumanga 
		tamala. //
	\glb Ang manga gum-ya {} Ajān taday-ya si ya konga-ye ang Pila 
		gumanga-Ø tamala //
	\glc \AgtT{} \Prog{} work-\Tsg{} \Top{} Ajān time-\Loc{} \Rel{} \LocT{} 
		enter-\TsgF{} \Aarg{} Pila workshop-\Top{} yesterday //
	\glft `Ajān was working when Pila entered the workshop yesterday.' //
\endgl

\a\label{ex:futprog}\begingl
	\gla Ang manga nimpay rangya nā tadayya si cunyo bekalang tasela. //
	\glb Ang manga nimp=ay.Ø rang-ya nā taday-ya si cun-yo bekal-ang 
		tasela //
	\glc \AgtT{} \Prog{} run=\Fsg{}.\Top{} home-\Loc{} \Fsg{}.\Gen{} 
		time-\Loc{} \Rel{} begin-\TsgN{} festival-\Aarg{} tomorrow //
	\glft `I will be running home when when the festival starts 
		tomorrow.' //
\endgl

\xe

Ignoring the constructedness of the above examples, the time adverb is located 
in the relative clause in both sentences in this case. Let us still assume that 
a narrative context with the respective time frames has already been 
established. As noted above, Ayeri prefers to not mark every verb for tense 
explicitly when the context is clear already, insofar the argument that 
progressive aspect works independent of \fw{tense} is needs corrobation; the 
question being if constructions like \rayr{mN m/—}{manga mə-...} (\Prog{} 
\Pst{}-...) are possible. Strictly speaking, there is nothing to prevent this 
construction, however, we have to wonder if it is actually \fw{natural} to 
phrase things this way. What can be said at least is that progressive marking 
is possible within a context referring to past or future actions and events 
irrespective of their explicit marking on the verb. Furthermore, the examples 
in (\ref{ex:nonpresprog}) illustrate a very typical use of the progressive as a 
structuring means, that is, an ongoing background action may be expressed using 
a progressive form, while an interrupting action receives no special marking 
(compare the past progressive in English).

\index{aspect!progressive|)}

\subsubsection{Habitual}
\index{aspect!habitual|(}

Unlike the few instances of habitual marking in \citeauthor{dahl1985}'s survey 
\citep[96]{dahl1985}, Ayeri possesses a suffix for marking habitual actions on 
the verb: \rayr{/As}{-asa}, where the first \fw{-a} replaces the terminal vowel 
of a verb stem if present, compare example (\ref{ex:habvwl}) below. The 
habitual aspect in Ayeri stresses that an action is carried out as a habit, that 
is, not just a few times, but with regular frequency. Essentially, verbs marked 
with the habitual in Ayeri can be translated by adding the adverb \fw{usually} 
in English \citep[97]{dahl1985}. Again, the habitual aspect is not restricted to 
present actions or absolute statements like the one in (\ref{ex:habcns}), but 
can also be used in past contexts to express that something \fw{used to} be done 
in the past as, again, in (\ref{ex:habvwl}). While the contexts are probably 
very few, there are no restrictions about using the habitual also in contexts 
relating to future actions which are predicted to be carried out habitually. The 
following sentences illustrate typical contexts in which the habitual may be 
used:

\pex
\a\label{ex:habcns}\begingl
	\gla Le kondasayāng hemaye pruyya nay napayya kayvay. //
	\glb Le kond-asa=yāng hema-ye-Ø pruy-ya nay napay-ya kayvay //
	\glc \PatTI{} eat-\Hab{}=\TsgM{}.\Aarg{} egg-\Pl{}-\Top{} salt-\Loc{} 
		and pepper-\Loc{} without //
	\glft `He always eats his eggs without salt and pepper.' //
\endgl

\a\label{ex:habvwl}\begingl
	\gla Ang ajasāyn ranisungas tadayya si yāng ganas. //
	\glb Ang aja-asa=ayn.Ø ranisung-as taday-ya si yāng gan-as //
	\glc \AgtT{} play-\Hab{}=\Fpl{}.\Top{} hide.and.seek-\Parg{} 
		time-\Loc{} \Rel{} \Fsg{}.\Aarg{} child-\Parg{} //
	\glft `We used to play hide-and-seek when I was a child.' //
\endgl

\xe

Importantly, the verb root with habitual marking forms a new verb stem to which 
affixes may be attached. This is relevant to mood suffixes, which follow 
aspectual marking.

\index{aspect!habitual|)}

\index{aspect|)}

\subsubsection{Iterative}
\index{aspect!iterative|(}

The iterative aspect marks actions that are repeated at least once by 
reduplication. The equivalent in English is to use the adverb \fw{again} or the 
prefix \fw{re-}. Iterative reduplication in Ayeri is only partial, in that only 
the initial CV- or VC- of a verb root is repeated---there are no verb roots 
which consist only of a single consonant or vowel. Complications begin, 
however, if the verb root starts with a consonant cluster (not unusual), or a 
diphtong (rare). In the case of an intial consonant cluster, the cluster is 
simplified to only include the first consonant; for initial diphthongs, there 
is no necessity to include the first available consonant, since the secondary 
vowel of a diphthong can by itself act as a semivowel to make up for the vowel 
hiatus.

\pex
	\a \makebox[9em][l]{\xayr{\larger kut/}{kuta-}{thank}}
		→ \xayr{\larger ku/kut/}{ku-kuta-}{thank again}
	\a \makebox[9em][l]{\xayr{\larger AmNF/}{amang-}{happen}}
		→ \xayr{\larger AmF/AmNF/}{am-amang-}{happen again}
	\a \makebox[9em][l]{\xayr{\larger pFraMtF/}{prant-}{ask}}
		→ \xayr{\larger p/pFrMtF/}{pa-prant-}{ask again}
	\a \makebox[9em][l]{\xayr{\larger AjrinF/}{ayrin-}{set}}
		→ \xayr{\larger Aj/AjrinF/}{ay-ayrin-}{set again}
\xe

The reduplicated stem works as a new stem for other prefixes, that is, no 
morphological material can go between the reduplicated part and the lexical 
stem proper; the following example also shows that there is, again, no 
restriction on the iterative aspect with regards to tense:

\ex\begingl
	\gla Məku-kutayāng. \quad \textup{(*}Ku-məkutayāng\textup{)} //
	\glb Mə-ku\til{}kuta=yāng //
	\glc \Pst{}-\Iter{}\til{}thank=\TsgM{}.\Aarg{} //
	\glft `He thanked again.' //
\endgl\xe

Iterative reduplication is lexicalized at least in one verb, 
\xayr{s/sh/}{sa-saha-}{return}. Besides the meaning of `again', iterative 
reduplication may also indicate the meaning `back', for instance in the 
following example:

\ex\begingl
	\gla Ta-tapyu adaley! //
	\glb Ta\til{}tapy-u ada-ley //
	\glc \Iter{}\til{}put-\Imp{} that-\PargI{} //
	\glft `Put that back!' //
\endgl\xe

In addition to a simple iterative meaning, a frequentative meaning like `run 
around', `cry all the time', or `keep asking' can be achieved by combining the 
iterative and progressive aspects, that is, the verb is both modified by 
\rayr{mN}{manga} for progressive aspect and partial initial reduplication for 
iterative aspect:

\pex
\a\begingl
	\gla Ang manga la-lampay saha-sara manga luga bahisya-hen. //
	\glb Ang manga la\til{}lamp-ay.Ø saha-sara manga luga bahis-ya=hen //
	\glc \AgtT{} \Prog{} \Iter{}\til{}walk=\Fsg{}.\Top{} back.and.forth 
		\Dyn{} while day-\Loc{}=all //
	\glft `I was walking around back and forth all day long.' //
\endgl

\a\begingl
	\gla Ang manga si-sipye kimay sirutayya. //
	\glb Ang manga si\til{}sip-ye kimay.Ø sirutay-ya //
	\glc \AgtT{} \Prog{} \Iter{}\til{}cry-\TsgF{} baby.\Top{} 
		night-\Loc{} //
	\glft `The baby, she is crying all the time at night.' //
\endgl

\a\begingl
	\gla Manga pa-prantu! //
	\glb Manga pa\til{}prant-u //
	\glc \Prog{} \Iter{}\til{}ask-\Imp{} //
	\glft `Keep asking!' //
\endgl

\xe

\index{aspect!iterative|)}

\subsubsection{Lexically marked aspectual categories}

Besides using morphological means, Ayeri expresses some aspectual categories by 
way of lexical items, that is, verbs and adverbs. The relevant words in this 
respect are the adverbs \xayr{mritj}{maritay}{before; be about to} 
(prospective) and \xayr{myis}{mayisa}{ready; be done} (cessative), as well as 
the verb \xayr{kYunF/}{cun-}{begin, start} (inchoative):

\ex\label{ex:prospective}\begingl
	\gla Saratang maritay. //
	\glb Sara=tang maritay //
	\glc leave=\TplM{}.\Aarg{} about.to //
	\glft `They are about to leave.' //
\endgl\xe

\ex~\label{ex:cessative}\begingl
	\gla Konjang mayisa. //
	\glb Kond=yang mayisa //
	\glc eat=\Fsg.\Aarg{} be.done //
	\glft `I am done eating.' //
\endgl\xe

\ex~\label{ex:inchoative}\begingl
	\gla Pəcunreng seyaryam. //
	\glb Pə-cun=reng seyar-yam //
	\glc \NFut{}-begin=\TsgI{}.\Aarg{} rain-\Ptcp{} //
	\glft `It is going to start raining any moment.' //
\endgl\xe

Prospective \rayr{mritj}{maritay} (\ref{ex:prospective}) and cessative 
\rayr{myis}{mayisa} (\ref{ex:cessative}) are expressed by adverbs which are 
regularly following verbs as their heads. They precede other adverbs due to a 
higher amount of semantic bondedness, by tendency, than other descriptive 
adverbs. For this reason, as well as for expressing grammatical function rather 
than lexical meaning with the original meaning still transparent, they appear to 
be on the verge of grammaticalization. In contrast, inchoative 
\rayr{kYunF/}{cun-} (\ref{ex:inchoative}) is expressed by a periphrastic verb 
construction, that is, \rayr{kYunF/}{cun-} requires a content-verb VP as a 
complement rather than an NP. The content/main verb appears in a non-finite 
form marked by \rayr{/ymF}{-yam}, which will be described below.

\subsection{Mood}
\index{mood|(}

Besides various aspects, Ayeri also marks mood other than realis: irrealis, 
imperative, hortative, and negative. These are also expressed by suffixes on the 
verb and typically follow aspectual marking where it is expressed by a sufffix, 
that is, the habitative suffix \rayr{As/}{-asa}. The following subsections will 
discuss each of the modal categories expressed by suffixes; modal verbs proper 
will be discussed in \autoref{subsec:modals}.

\subsubsection{Irrealis}
\index{mood!irrealis|(}

Irrealis marking in Ayeri might also be termed \emph{subjunctive}; either way, 
however, the suffix \rayr{/ONF}{-ong} marks that an action is thought of as 
hypothetical by the speaker, whether he or she expects it to be fulfilled or 
not:

\ex\label{ex:irrealis}\begingl
	\gla Sahongvāng edaya, ming silvongvāng sitang-vāri. //
	\glb Saha-ong=vāng edaya, ming silv-ong=vāng sitang=vāri //
	\glb come-\Irr{}=\Ssg{}.\Aarg{} here, can see-\Irr{}=\Ssg{}.\Aarg{} 
		\Refl{}=\Ssg{}.\Ins{} //
	\glft `If you came/had come here, you could see/have seen it  
		yourself.' //
\endgl\xe

As (\ref{ex:irrealis}) shows, irrealis marking is especially prominent in 
conditional clauses which express a hypothetical cause and effect. Both 
condition/protasis and consequence/apodosis are marked with the irrealis suffix 
in this case. The example sentence also shows that, again, the initial vowel of 
the suffix replaces the last vowel of the verb stem, if there is one, so that 
\rayr{sh/}{saha-} becomes \rayr{shoNF/}{sahong-}, to which further mood suffixes 
may be added, and finally, person marking.

The same suffix, \rayr{/ONF}{-ong} is also used in other contexts expressing 
inactual events, for instance, in reported speech, or complement clauses 
expressing a wish about the actualization of a hypothetical event:

\ex\begingl
	\gla Narayeng, ang menongye demās yena. //
	\glb Nara=yeng ang menu-ong=ye.Ø dema-as yena //
	\glc say=\TsgF{}.\Aarg{} \AgtT{} visit-\Irr{}=\TsgF{}.\Top{} 
		aunt-\Parg{} \TsgF{}.\Gen{} //
	\glft `She said she were visiting her aunt.' //
\endgl\xe

\ex~\begingl
	\gla Hanuyang, koronongyang maritay. //
	\glb Hanu=yang koron-ong=yang maritay //
	\glc wish=\Fsg{}.\Aarg{} know-\Irr{}=\Fsg{}.\Aarg{} before //
	\glft `I wish I had known this before.' //
\endgl\xe

Irrealis marking does not, however, appear in contexts that express 
requirements on or wishes about a third person's actions, that is, typical 
subjuctive contexts; the verb in the complement clause rather appears in the 
indicative in these contexts. To add a sense of expectation of compliance about 
the action, the modal \xayr{mY}{mya}{be supposed to} may be added, see 
\autoref{subsec:modals}.

\pex
\a\ljudge*\begingl
	\gla Arapnang, sa garongyāng hatay. //
	\glb Arap=nang sa gara-ong=yāng hatay-Ø //
	\glc require=\Fpl{}.\Aarg{} \PatT{} call-\Irr{}=\TsgM{}.\Aarg{} 
		police-\Top{} //
\endgl

\a\label{ex:myashall}\begingl
	\gla Arapnang, sa (mya) garayāng hatay. //
	\glb Arap=nang sa (mya) gara=yāng hatay-Ø //
	\glc require=\Fpl{}.\Aarg{} \PatT{} (be.supposed.to) 
		call=\TsgM{}.\Aarg{} police-\Top{} //
	\glft `We require that he call the police.' //
\endgl
\xe

\index{mood!irrealis|)}

\subsubsection{Negative}
\label{subsubsec:verbneg}
\index{mood!negative|(}
\index{negation!of verbs|(}

The negative mood is used to negate verbs, which is separate from irrealis 
marking: negation of verbs is marked by the suffix \rayr{/Oj}{-oy}, which has 
an allomorph \fw{-u} before diphthongs in romanization but also in 
pronunciation. The Tahano Hikamu spelling is more conservative 
here and keeps the spelling \ayr{/Oyj} for /-uay/ 
(-\Neg{}=\Fsg{}.\Top{}).\index{allomorphy} Like the irrealis suffix, the 
negative suffix deletes the last vowel of the verb stem if present, which is 
exemplified in (\ref{ex:negallo}) besides this example showing the \fw{-u} 
allomorph. Moreover, example (\ref{ex:irrneg}) shows that negative marking 
usually follows irrealis marking when suffixes are stacked: \rayr{/ONF}{-ong} + 
\rayr{/Oj}{-oy} → \rayr{/ONoj}{-ongoy}.

\pex
\a\label{ex:negative}\begingl
	\gla Ang silvoyyan nasiyamanas tan. //
	\glb Ang silv-oy=yan.Ø nasi-yam-an-as tan //
	\glc \AgtT{} see-\Neg{}=\TplM{}.\Top{} approach-\Ptcp{}-\Nmlz{}-\Parg{} 
		\TplM{}.\Gen{} //
	\glft `They did not see them approaching.' //
\endgl

\a\label{ex:negallo}\begingl
	\gla Ang peguay kalam adaley!  //
	\glb Ang pega-oy=ay.Ø kalam ada-ley //
	\glc \AgtT{} steal-\Neg{}=\Fsg{}.\Top{} honestly that-\PargI{} //
	\glft `I didn't steal it, honestly!' //
\endgl

\a\label{ex:irrneg}\begingl
	\gla Ang tendongoyva sarayam adaya. //
	\glb Ang tend-ong-oy=va.Ø sara-yam adaya //
	\glc \AgtT{} dare-\Irr{}-\Neg{}=\Ssg{}.\Top{} go-\Ptcp{} there //
	\glft `You would not dare to go there.' //
\endgl

\xe

\index{negation!of verbs|)}
\index{mood!negative|)}

\subsubsection{Imperative}
\index{mood!imperative|(}

The imperative mood is used to mark orders to an unspecified second person, 
that is, imperative verbs do not require an overt second person agent; if an 
addressee is included, it is oblique and unmarked for case, see 
\autoref{subsec:uncased}. Moreover, no distinction is made between singular and 
plural second-person addressees, so that the marker is \rayr{/U}{-u} in either 
case. Like the other mood suffixes, the vowel of the imperative suffix replaces 
the vowel of the verb stem if there is one.

\pex
\a\begingl
	\gla Giru māy! //
	\glb gira-u māy //
	\glc hurry-\Imp{} \Int{} //
	\glft `Hurry up!' //
\endgl

\a\begingl
	\gla Tangu yām, Yan! //
	\glb Tang-u yām Yan //
	\glc listen-\Imp{} \Fsg{}.\Dat{} Yan //
	\glft `Listen to me, Yan!' //
\endgl

\a\begingl
	\gla Tangu yām, ledanye nā! //
	\glb Tang-u yām ledan-ye nā //
	\glc listen-\Imp{} \Fsg{}.\Dat{} friend \Fsg{}.\Gen{} //
	\glft `Listen to me, my friends!' //
\endgl

\xe

It is important to note that the imperative paradigm is defective; imperative 
verbs behave essentially as infinite forms in that they do not exhibit any 
agreement in person, number, gender, and topic, and also cannot act as hosts 
for clitic personal pronouns. Imperative verbs may be marked for negative and 
hortative, however. Hence, for instance, (\ref{ex:negimp}) is grammatical, 
while the examples in (\ref{ex:agrimp}) are not.

\ex\label{ex:negimp}\begingl
	\gla Saroyu yas! //
	\glb Sara-oy-u yas! //
	\glc leave-\Neg{}-\Imp{} \Fsg{}.\Parg{} //
	\glft `Don't leave me!' //
\endgl\xe

\pex~\label{ex:agrimp}
\a\label{ex:topimp}\ljudge*\begingl
	\gla Ya sa-sahu nanga! //
	\glb Ya sa\til{}saha-u nanga-Ø //
	\glc \LocT{} \Iter{}\til{}go-\Imp{} house-\Top{} //
	\glft `Go back to the house!' //
\endgl

\a\label{ex:persimp}\ljudge*\begingl
	\gla Sa sutamuya kohanya tasela! //
	\glb Sa sutam-u=ya.Ø kohan-ya tasela //
	\glc \PatT{} hang-\Imp{}=\TsgM{}.\Top{} sunrise-\Loc{} tomorrow //
	\glft `May he be hanged tomorrow at sunrise!' //
\endgl
\xe

Example (\ref{ex:negimp}) simply expresses a negative command, which is 
unproblematic in terms of logic, since commands may be issued to act in a 
certain way, or to forgo this action. Example (\ref{ex:topimp}) shows the 
imperative verb as preceded by a locative topic marker, which is not logically 
impossible, but unacceptable by convention. Example (\ref{ex:persimp}) takes 
this one step further in displaying a cliticized object pronoun in the fashion 
of morphological passives (compare \autoref{subsec:persnumagr}, page 
\pageref{patagr}).

\index{mood!imperative|)}

\subsubsection{Hortative}
\index{mood!hortative|(}

The hortative is a special kind of imperative, which addresses a group 
including the speaker. Its implied referent is thus first-person plural. 
Again, it is not necessary to mark the verb for the addressee here. As 
the hortative is related in meaning to the imperative, the verb also uses the 
imperative inflection with \rayr{/U}{-u}, but it is fully reduplicated in 
addition to mark the difference. As regards agreement morphology, the same 
restrictions as those of imperatives apply.

\pex
\a\begingl
	\gla Sahu! //
	\glb Saha-u //
	\glc go-\Imp{} //
	\glft `Go!' //
\endgl

\a\begingl
	\gla Sahu-sahu umangya! //
	\glb Sahu\til{}saha-u umang-ya //
	\glc \Hort{}\til{}go-\Imp{} beach-\Loc{} //
	\glft `Let's go to the beach!' //
\endgl
\xe

\index{mood!hortative|)}

\index{mood|)}

\subsection{Modal verbs}
\label{subsec:modals}
\index{modal verbs|(}

\begin{figure}
\caption{Modal verbs and particles}
\begin{tabu} to \linewidth {C[3] X[2] X[2] X[4]}
\tableheaderfont\toprule
Category
	& Verb
	& Particle
	& Translation
	\\
\toprule

ability
	& \ayr{miNF/} \fw{ming-}
	& \ayr{miNF} \fw{ming}
	& `be able to, can'
	\\
	
\midrule
	
desire, intention
	& \ayr{vtYF/} \fw{vac-}
	& \ayr{vtYF} \fw{vaca}
	& `like to'
	\\
	
	& \ayr{no/} \fw{no-}
	& \ayr{no} \fw{no}
	& `want to'
	\\
	
\midrule

permission
	& \ayr{kil/} \fw{kila-}
	& \ayr{kil} \fw{kila}
	& `be allowed to, may'
	\\
	
\midrule

requirement
	& \ayr{IlFt/} \fw{ilta-}
	& \ayr{IlFt} \fw{ilta}
	& `need to'
	\\
	
\midrule

obligation
	& \ayr{mY/} \fw{mya-}
	& \ayr{mY} \fw{mya}
	& `be supposed to, shall'
	\\
	
	& \ayr{ru\_a/} \fw{rua-}
	& \ayr{ru\_a} \fw{rua}
	& `have to, must'
	\\
	
\midrule
	
continuation
	& \ayr{divF/} \fw{div-}
	& \ayr{div} \fw{diva}
	& `stay, remain'
	\\

\bottomrule
\end{tabu}
\label{fig:modverb}
\end{figure}

Modal verbs in Ayeri express the notions of ability, desire, permission, 
requirement, obligation, and also of continuation, as indicated by 
\autoref{fig:modverb}. They can generally act as both fully inflectable 
intransitive verbs, as well as invariable bound morphemes which occur in 
combination with fully inflected content verbs:

\pex
\a\label{ex:modalinvar}\begingl
	\gla Rua bahavāng baho, ang bihanoyya mirampaluy nas. //
	\glb Rua baha=vāng baho, ang bihan-oy=ya.Ø mirampaluy nas //
	\glc must shout=\Ssg{}.\Aarg{} loudly, \AgtT{} 
		understand-\Neg{}=\TsgM{}.\Top{} otherwise \Fpl{}.\Parg{} //
	\glft `You have to shout loudly, otherwise he does not understand 
		us.'//
\endgl

\a\label{ex:modalinfl}\begingl
	\gla Ruasanang. //
	\glb Rua-asa=nang //
	\glc must-\Hab{}=\Fpl{}.\Aarg{} //
	\glft `We usually have to.' //
\endgl

\xe

As (\ref{ex:modalinvar}) shows, the modal does not inflect in combination with 
another verb; it rather acts similarly to a prefix, like the progressive marker 
\rayr{mN}{manga}, which is also presumably deverbal, compare 
\autoref{sec:typology}, \autoref{fn:mangaverb}. In difference to 
\rayr{mN}{manga}, which as a preverbal element only serves a grammatical 
function, the semantic component of the modals is still prevalent, as is shown 
by (\ref{ex:modalinfl}), where \rayr{ru\_a/}{rua-} appears in its function as 
an intransitive verb with the same meaning of strong obligation as in 
(\ref{ex:modalinvar}), though it carries regular person and aspect inflection 
here. Inflecting the modal in the context of cooccurrence with a content verb 
is, however, considered unacceptable:

\ex\ljudge*\begingl
	\gla Ruavāng bahayam baho. //
	\glb Rua=vāng baha-yam baho //
	\glc must=\Ssg{}.\AgtT{} shout-\Ptcp{} loudly //
	\glft `You have to shout loudly.' //
\endgl\xe

Regarding what is basically the opposite case, Ayeri has a verb that generally 
means `do', namely, \rayr{mir/}{mira-}. However, it is not common to use this 
as a dummy verb to carry the inflection instead of the modal verb either. While 
such a construction is not ungrammatical \fw{per se}, it is simply not the 
preferred way to express intransitive modal verbs:

\ex\ljudge\ques\begingl
	\gla Rua mirasanang. //
	\glb Rua mira-asa=nang //
	\glc must do-\Hab{}=\Fpl{}.\Aarg{} //
	\glft `We usually have to.' //
\endgl\xe

While most of the verbs listed in \autoref{fig:modverb} should look 
reasonable to English speakers, Ayeri uses two verbs for modal particles which 
may seem odd: \xayr{vtY/}{vaca}{like to}, to express taking pleasure in doing 
something, and \xayr{div/}{diva}{stay, remain}, to express that the action is 
being prolonged.\footnote{The verb stems indeed end in a consonant while the 
modal particles need an epenthetic \fw{-a} to form permissible words.} The 
latter verb thus also has an aspectual component to its meaning.

\pex\label{ex:vacvaca}
\a\label{ex:vacfull}\begingl
	\gla Ang vacay betayley. //
	\glb Ang vac=ay.Ø betay-ley //
	\glc \AgtT{} like=\Fsg{}.\Top{} berry-\PargI{} //
	\glft `I like berries.' //
\endgl

\a\label{ex:vacamod}\begingl
	\gla Ang vaca konday betayley. //
	\glb Ang vaca kond=ay.Ø betay-ley //
	\glc \AgtT{} like eat=\Fsg{}.\Top{} berry-\PargI{} //
	\glft `I like to eat berries.' //
\endgl
\xe

\pex~\label{ex:divdiva}
\a\label{ex:divfull}\begingl
	\gla Ang divay rangya nā tasela. //
	\glb Ang div=ay.Ø rang-ya nā tasela //
	\glc \AgtT{} stay=\Fsg{}.\Top{} home-\Loc{} \Fsg{}.\Gen{} //
	\glft `I will stay home tomorrow.' //
\endgl

\a\label{ex:divamod}\begingl
	\gla Ang diva bengya ku-danyās kebay. //
	\glb Ang diva beng=ya.Ø ku=danya-as kebay //
	\glc \AgtT{} stay stand=\TsgM{}.\Top{} like=one-\Parg{} alone //
	\glft `He remained standing as the only one.' //
\endgl
\xe

The fact that modal particles in Ayeri retain their verbal semantics in spite 
of shedding verb morphology is probably even more obvious from the above 
examples (\ref{ex:vacvaca}) and (\ref{ex:divdiva}), which show the alternation 
between full-verb use (a) and modal use (b) for both \rayr{vtYF/}{vac-} and 
\rayr{divF/}{div-}. In comparison to the other modal verbs in 
\autoref{fig:modverb}, these two verbs in particular also stand out by virtue 
of their roots ending in a consonant instead of a vowel like in the other 
cases. This suggests that they may have been grammaticalized as modal verbs 
only relatively recently, and there appears to be variation at least for 
\rayr{vtYF/}{vac-}, for instance:

\ex\begingl
	\gla ... yam vacongyang ilisayam eda-koyās gan ... //
	\glb ... yam vac-ong-yang ilisa-yam eda=koya-as gan-Ø ... //
	\glc {} \DatT{} like-\Irr{}-\Fsg{}.\Aarg{} dedicate-\Ptcp{} 		
		this=book-\Parg{} child-\Top{} {} //
	\glft `... I would like to dedicate this book to the child ...' 
		\tc{\citep[1, 8]{benung:petitprince}} //
\endgl\xe

Moreover, as illustrated previously in (\ref{ex:myashall}), \xayr{mY}{mya}{be 
supposed to, shall} can be used to express indirect commands where English may 
use the subjunctive mood; essentially the function of this modal particle is 
that of the jussive mood. For convenience, the above example will be repeated 
here:

\ex\begingl
	\gla Arapnang, sa (mya) garayāng hatay. //
	\glb Arap=nang sa (mya) gara=yāng hatay-Ø //
	\glc require=\Fpl{}.\Aarg{} \PatT{} (be.supposed.to) 
		call=\TsgM{}.\Aarg{} police-\Top{} //
	\glft `We require that he call the police.' //
\endgl\xe

In addition to this use, \rayr{mY}{mya} is also used in commands to third 
persons, whether direct or indirect. English may use \fw{shall} here as an 
equivalent.

\pex
\a\begingl
	\gla Ningu cam, mya saratang. //
	\glb Ning-u cam, mya sara=tang //
	\glc tell-\Imp{} \TplM{}.\Dat{} shall leave=\TplM{}.\Aarg{} //
	\glft `Tell them to leave.' //
\endgl

\a\begingl
	\gla Mya vehara nekanley. //
	\glb Mya veh-ara nekan-ley //
	\glc shall build-\TsgI{} bridge-\PargI{} //
	\glft `A bridge shall be built.' //
\endgl

\a\begingl
	\gla Mya yomāra makangreng. //
	\glb Mya yoma-ara makang-reng //
	\glc shall exist-\TsgI{} light-\AargI{} //
	\glft `Let there be light.' //
\endgl
\xe

% \subsubsection{Ability}
% \index{modal verbs!ability|(}
% ... \xayr{miNF/}{ming-}{be able to, can} ...
% 
% \index{modal verbs!ability|)}
% 
% \subsubsection{Desire}
% \index{modal verbs!desire|(}
% ... \xayr{vkY/}{vaca-}{like to} ...
% ... \xayr{no/}{no-}{want to} ...
% 
% \index{modal verbs!desire|)}
% 
% \subsubsection{Permission}
% \index{modal verbs!permission|(}
% ... \xayr{kil/}{kila-}{be allowed to, may} ...
% 
% \index{modal verbs!permission|)}
% 
% \subsubsection{Requirement}
% \index{modal verbs!requirement|(}
% ... \xayr{IlFt/}{ilta-}{need to} ...
% 
% \index{modal verbs!requirement|)}
% 
% \subsubsection{Obligation}
% \index{modal verbs!obligation|(}
% ... \xayr{mY/}{mya-}{be supposed to, shall} ...
% ... \xayr{ru\_a/}{rua-}{have to, must} ...
% 
% \index{modal verbs!obligation|)}
% 
% \subsubsection{Continuation}
% \index{modal verbs!continuation|(}
% ... \xayr{div/}{diva-}{stay, remain} ...
% 
% \index{modal verbs!continuation|)}
% 
% \index{modal verbs|)}

\subsection{Participle}
\label{subsec:participle}
\index{participle|(}
Besides the imperative---and, by extension, the hortative---Ayeri also 
possesses another infinite form called the participle. This form is marked by 
appending \rayr{/ymF}{-yam} to the verb root. The participle is generally the 
form of verbal complements of intransitive subordinating verbs other than 
modal verbs (compare \autoref{subsec:modals}). For instance,  
\xayr{kYunF/}{cun-}{begin} or \xayr{mnNF/}{manang-}{avoid} both allow 
complementation with another verb:

\pex
\a\label{ex:intrcompl}\begingl
	\gla Cunyo makayam perinang. // 
	\glb Cun-yo maka-yam perin-ang // 
	\glc begin-\TsgN{} shine-\Ptcp{} sun-\Aarg{} //
	\glft `The sun began to shine.' //
\endgl

\a\label{ex:trcompl}\begingl
	\gla Manangyeng pengalyam badanas saha yena. //
	\glb Manang=yeng pengal-yam badan-as saha yena //
	\glc avoid=\TsgF{}.\Aarg{} meet-\Ptcp{} father-\Parg{} in.law 
		\TsgF{}.\Gen{} //
	\glft `She avoids to meet her father-in-law.' //
\endgl
\xe

Since subordinated verbs may be transitive like in (\ref{ex:trcompl}), the 
problem of center-embedding arises when the agent NP of the subordinating verb 
is not simply a cliticized pronoun (see \autoref{subsec:persnumagr}), since 
arguments of the subordinating verb follow the embedded clause as in 
(\ref{ex:intrcompl}):

\pex[*=\ques\ques]
\a\ljudge{\ques}\begingl
	\gla Ang pinyaya \normalfont{[}konjam inunas\normalfont{]} {} Yan sa 
		Pila. //
	\glb Ang pinya-ya kond-yam inun-as Ø Yan sa= Pila //
	\glc \AgtT{} ask-\TsgM{} eat-\Ptcp{} fish-\Parg{} \Top{} Yan \Parg{}
		Pila //
	\glft `Yan asks Pila to eat the fish.' //
\endgl

\a\ljudge{\ques\ques}\begingl
	\gla Ang pinyaya \normalfont{[}ilyam koyaley ledanyam 
		yana\normalfont{]} {} Yan sa Pila. //
	\glb Ang pinya-ya il-yam koya-ley ledan-yam yana Ø Yan sa Pila //
	\glc \AgtT{} ask-\TsgM{} give-\Ptcp{} book-\PargI{} friend-\Dat{} 
		\TsgM{}.\Gen{} \Top{} Yan \Parg{} Pila //
	\glft `Yan asks Pila to give the book to his friend.' //
\endgl
\xe

In order to avoid too much complexity at the expense of ease of composition on 
the speaker's side, and intelligibility on the listener's, it is much 
preferable to express the embedded clause as a complement clause 
instead.\footnote{The German linguist Otto Behaghel (1854--1936) coined a 
number of laws---albeit with German in focus---three of which are relevant to 
information flow: \textcquote[4]{behaghel1932}{Das oberste Gesetz ist dieses, 
daß das geistig eng Zusammengehörige auch eng zusammengestellt wird.} [`The 
supreme law is such that the mentally closely related is also arranged in close 
proximity.']---\textcquote[4]{behaghel1932}{Ein zweites machtvolles Gesetz 
verlangt, daß das Wichtigere später steht als das Unwichtige, dasjenige, was 
zuletzt noch im Ohr klingen soll.} [`A second powerful law demands that more 
important information appear at a later point than what is less important: the 
which is supposed lastly to resonate in the listener's 
ear.']---\textcquote[6]{behaghel1932}{Gesetz der wachsenden Glieder […]; es 
besagt, daß von zwei Gliedern, soweit möglich, das kürzere vorausgeht, das 
längere nachsteht.} [`Law of the growing constituents […]; it signifies that of 
two constituents, if possible, the shorter one precedes, the longer one 
follows.']} The particle \rayr{d/}{da-} may be added to the formerly 
subordinating verb in order to signal that a complement clause is following.

\pex
\a\begingl
	\gla Ang da-pinyaya {} Yan sa Pila, \normalfont{[}le konjeng 
		inun\normalfont{]}. //
	\glb Ang da=pinya-ya Ø Yan sa Pila le kond=yeng inun-Ø //
	\glc \AgtT{} such=ask-\TsgM{} \Top{} Yan \Parg{} Pila \PatTI{} 
		eat=\TsgF{}.\Aarg{} fish-\Top{} //
	\glft `Yan asks Pila to eat the fish.' //
\endgl

\a\begingl
	\gla Ang da-pinyaya {} Yan sa Pila, \normalfont{[}le ilyeng koya 
		ledanyam yana\normalfont{]}. //
	\glb Ang da=pinya-ya Ø Yan sa Pila, le il=yeng koya-Ø ledan-yam
		yana //
	\glc AT such=ask-\TsgM{} \Top{} Yan \Parg{} Pila, \PatTI{} give-\TsgF{} 
		book-\Top{} friend-\Dat{} \TsgM{}.\Gen{} //
	\glft `Yan asks Pila to give the book to his friend.' //
\endgl
\xe

\index{participle|)}

\subsection{Other affixes}

In the section on noun morphology we have already encountered a number of 
clitic prefixes that may attach to noun heads or NPs (see 
\autoref{subsec:nounpref}), and these can also attach to verbs. Furthermore, 
verbs may also be modified by certain adverbial quantifier clitics. The latter 
are dealt with in more detail in the section on adverbs; only a few relevant 
examples will be given here.

\subsubsection{Prefixes}

We have already encountered the prefix \xayr{d/}{da-}{so, such} in the previous 
section, as well as in the section on noun prefixes (see sections
\ref{subsec:nounpref} and \ref{subsec:participle}). With nouns, 
\xayr{d/}{da-}{such} patterns as a demonstrative with the deictic prefixes 
\xayr{Ed/}{eda-}{this} and \xayr{Ad/}{ada-}{that}. Distinguishing between near 
and far is not possible with verbs, but pointing out that something is 
happening `in this way', `so' is still possible, hence \rayr{d/}{da-} is also 
applicable to verbs. \rayr{d/}{da-} can thus act essentially as a pro-verb. As 
a clitic, it leans on the verb, preceding all other inflectional prefixes, that 
is, any tense prefixes that may possibly precede the verb root.

\pex\label{ex:daproverb}
\a\begingl
	\gla Da-mingya ang Diyan. //
	\glb Da=ming-ya ang Diyan. //
	\glc so=can-\TsgM{} \Aarg{} Diyan //
	\glft `Diyan can (do it).' //
\endgl

\a\begingl
	\gla Ang da-məpinyaya {} Yan sa Pila. //
	\glb Ang da=mə-pinya-ya Ø Yan sa Pila //
	\glc \AgtT{} such=\Pst{}-ask-\TsgM{} \Top{} Yan \Parg{} Pila //
	\glft `Yan asked Pila to (do so).' //
\endgl

\xe

Another possible use of the prefix \rayr{da/}{da-} with verbs is related to the 
colloquial abbreviation of \xayr{danY}{danya}{such one} as described in 
\autoref{subsec:dempro}, where the demonstrative part, \rayr{da/}{da-} may be 
split off the pronoun and attached in front of the adjective directly to 
express `the \textsc{adj} one'. This practice has possibly been extended to 
verbs in analogy to the use just illustrated in (\ref{ex:daproverb}). Example 
(\ref{ex:redone}) from the mentioned section is repeated here for the reader's 
convenience:

\ex\begingl
	\gla Sa noyang da-tuvo. //
	\glb Sa no=yang da=tuvo.Ø //
	\glc \PatT{} want=\Fsg{}.\Aarg{} such=red.\Top{} //
	\glft `I want the red one.' //
\endgl\xe

When \rayr{d/}{da-} is used as an abbreviation for \rayr{dnYaasF}{danyās} 
(such.one-\Parg{}) or \rayr{dnYlej}{danyaley} (such.one-\PargI{}), as in the 
following example, it may also appear prefixed to the verb:

\ex\begingl
	\gla Mya da-vehoyyāng. //
	\glb Mya da=veh-oy=yāng //
	\glc supposed.to one=build-\Neg{}=\Tsg.\M{} //
	\glft `He is not supposed to build one.' //
\endgl\xe

As mentioned above, \rayr{d/}{da-} can also be used in an expletive way, to 
express `in this way' or `like that'. It does not encode an anaphoric relation 
in this case, but merely serves as a discourse particle to highlight the action.

\pex
\a\begingl
	\gla Da-sahāra seyaraneng. //
	\glb Da=saha-ara seyaran-eng //
	\glc thus=come-\TsgI{} rain-\AargI{} //
	\glft `Here comes the rain.' //
\endgl

\a\begingl
	\gla Le no da-subroyya ang Hasanjan tiga kaytan yana. //
	\glb Le no da=subr-oy-ya ang Hasanjan tiga kaytan-Ø yana //
	\glc \PatT{} want there=give.up-\Neg{}-\TsgM{} \Aarg{} Hasanjan 
		honorable right-\Top{} \TsgM{}.\Gen{} //
	\glft `Mr. Hasanjan did not want to cease his right just there.' //
\endgl

\xe

Besides \rayr{d/}{da-}, verbs may also take the \xayr{ku/}{ku-}{like} prefix, 
which we have already seen with both nouns and adjectives (compare sections 
\ref{subsec:nounpref} and \ref{subsec:adjaffx}). The English translation in 
context may rather be `as though' than `like' here, but the function is the 
same: expressing alikeness and resemblance.

\ex\begingl
	\gla Misyeng, ang ku-tangoyye yās. //
	\glb Mis=yeng, ang ku=tang-oy=ye.Ø yās //
	\glc act=\TsgF{}.\Aarg{} \AgtT{} like=hear-\Neg{}=\TsgF{}.\Top{} 
		\TsgM{}.\Parg{} //
	\glft `She acts as though she does not hear him.' //
\endgl\xe

As previously described (compare \autoref{subsec:reflrec}), 
\xayr{sitNF/}{sitang}{self}, the reflexive prefix, can appear as a prefix on 
verbs as well. This may be the case when the patient/undergoer of a 
transitive sentence signifies the same entity as the actor. Example 
(\ref{ex:reflvb}) is repeated here for convenience:

\ex\begingl
	\gla Ang sitang-silvye puluyya. //
	\glb Ang sitang=silv=ye.Ø puluy-ya //
	\glc \AgtT{} self=see=\TsgF{}.\Top{} mirror-\Loc{} //
	\glft `She sees herself in the mirror.' //
\endgl\xe

The image of the agent in the mirror is that of the agent herself, so she is 
seeing her own reflection. Both agent and patient thus reference the same 
person, which means that instead of using the reflexive object pronoun 
\xayr{sitNF/yesF}{sitang-yes}{herself} (self=\TsgF{}.\Parg{}), it is possible to 
drop the pronoun and to place the reflexive prefix on the verb instead.

\subsubsection{Suffixes}
Besides taking clitic prefixes, verbs may also take clitic suffixes, namely, 
adverbial suffixes denoting degree, such as \xayr{/Ani}{-ani}{not at all}, 
\xayr{/ENF}{-eng}{rather}, \xayr{/IknF}{-ikan}{much}, 
\xayr{/Iknoj}{-ikanoy}{not much}, \xayr{/kj}{-kay}{a little}, 
\xayr{/nm}{-nama}{just, only, merely}, \xayr{/NsF}{-ngas}{almost}, and 
\xayr{/nYm}{-nyama}{even}---some of these overlap with quantifiers applicable 
to nouns, and all of them are also applicable to adjectives. As enclitics, 
these suffixes lean on the inflected verb:

\pex
\a\begingl
	\gla Ang rua apaya-kay {} Latun adanyaya. //
	\glb Ang rua apa-ya=kay Ø Latun adanya-ya //
	\glc \AgtT{} must laugh=\TsgM{}=a.little \Top{} Latun that.one-\Loc{} //
	\glft `Latun had to laugh a little at that.' //
\endgl

\a\begingl
	\gla Ya no narayang-nama va. //
	\glb Ya no nara=yang=nama va.Ø //
	\glc \LocT{} want speak=\Fsg{}.\Aarg{}=just \Ssg{}.\Top{} //
	\glft `It is you I just want to talk to.' //
\endgl
\xe

\index{verbs|)}
