% kate: word-wrap true;

\chapter{Grammatical categories}

While the previous chapter was about general mechanisms of marking in Ayeri, 
this chapter will dive into the various parts of speech in order to define 
their morphology with a closer look. I will begin with nouns as the main 
carriers of meaning, then deal with other parts of speech that regularly 
feature in or in combination with the noun phrase---pronouns, adjectives, and 
adpositions. Following this, there will be a discussion of verbs and adverbs 
before moving on to numerals and conjunctions.

\section{Nouns}
\index{nouns|(}

Nouns in Ayeri have \emph{gender} and \emph{number} as their inherent 
grammatical properties. Besides common nouns, there are, of course, also proper 
nouns (i.e. names) and nominalizations. Nouns, as the heads of NPs, are also 
assigned \emph{case} by the verb, which is a third grammatical property they 
display. For an illustration of the declension paradigms, compare Figures 
\ref{fig:anideclcons}–\ref{fig:inandeclvow}.

\begin{figure}[ht]
\caption[Declension paradigm for \xayr{bdnF}{badan}{father}]{Declension 
paradigm for \xayr{bdnF}{badan}{father} (animate; consonantal root)}
\begin{tabu} to \linewidth {X[1] I[2] X[4] I[2] X[4]}
\tableheaderfont\toprule

	& \multicolumn2{c}{Singular}
	& \multicolumn2{c}{Plural}
	\\

\midrule
	
\Top{}
	& badan
	& `the father'
	%
	& badanye
	& `the fathers'
	\\

\midrule

\Aarg{}
	& badanang
	& `father'
	%
	& badanjang
	& `fathers'
	\\

\Parg{}
	& badanas
	& `father' (obj.)
	%
	& badanjas
	& `fathers' (obj.)
	\\

\Dat{}
	& badanyam
	& `to the father'
	%
	& badanjyam
	& `to the fathers'
	\\

\midrule

\Gen{}
	& badanena
	& `of the father'
	%
	& badanyena
	& `of the fathers'
	\\
	
\Loc{}
	& badanya
	& `at the father'
	%
	& badanjya
	& `at the fathers'
	\\

\Caus{}
	& badanisa
	& `due to the father'
	%
	& badanjisa
	& `due to the fathers'
	\\

\Ins{}
	& badaneri
	& `with the father'
	%
	& badanyeri
	& `with the fathers'
	\\

\bottomrule
\end{tabu}
\label{fig:anideclcons}
\end{figure}
~
\begin{figure}[ht]
\caption[Declension paradigm for \xayr{maav}{māva}{mother}]{Declension 
paradigm for \xayr{maav}{māva}{mother} (animate; vocalic root)}
\begin{tabu} to \linewidth {X[1] I[2] X[4] I[2] X[4]}
\tableheaderfont\toprule

	& \multicolumn2{c}{Singular}
	& \multicolumn2{c}{Plural}
	\\

\midrule
	
\Top{}
	& māva
	& `the mother'
	%
	& māvaye
	& `the mothers'
	\\

\midrule

\Aarg{}
	& māvāng
	& `mother'
	%
	& māvajang
	& `mothers'
	\\

\Parg{}
	& māvās
	& `mother' (obj.)
	%
	& māvajas
	& `mothers' (obj.)
	\\

\Dat{}
	& māvayam
	& `to the mother'
	%
	& māvajyam
	& `to the mothers'
	\\

\midrule

\Gen{}
	& māvana
	& `of the mother'
	%
	& māvayena
	& `of the mothers'
	\\
	
\Loc{}
	& māvaya
	& `at the mother'
	%
	& māvajya
	& `at the mothers'
	\\

\Caus{}
	& māvaisa
	& `due to the mother'
	%
	& māvajisa
	& `due to the mothers'
	\\

\Ins{}
	& māvari
	& `with the mother'
	%
	& māvayeri
	& `with the mothers'
	\\

\bottomrule
\end{tabu}
\label{fig:anideclvow}
\end{figure}

\begin{figure}[ht]
\caption[Declension paradigm for \xayr{kirinF}{kirin}{street}]{Declension 
paradigm for \xayr{kirinF}{kirin}{street} (inanimate; consonantal root)}
\begin{tabu} to \linewidth {X[1] I[2] X[4] I[2] X[4]}
\tableheaderfont\toprule

	& \multicolumn2{c}{Singular}
	& \multicolumn2{c}{Plural}
	\\

\midrule
	
\Top{}
	& kirin
	& `the street'
	%
	& kirinye
	& `the streets'
	\\

\midrule

\Aarg{}
	& kirinreng
	& `street'
	%
	& kirinyereng
	& `streets'
	\\

\Parg{}
	& kirinley
	& `street' (obj.)
	%
	& kirinyeley
	& `streets' (obj.)
	\\

\Dat{}
	& kirinyam
	& `to the street'
	%
	& kirinjyam
	& `to the streets'
	\\

\midrule

\Gen{}
	& kirinena
	& `of the street'
	%
	& kirinyena
	& `of the streets'
	\\
	
\Loc{}
	& kirinya
	& `at the street'
	%
	& kirinjya
	& `at the streets'
	\\

\Caus{}
	& kirinisa
	& `due to the street'
	%
	& kirinjisa
	& `due to the streets'
	\\

\Ins{}
	& kirineri
	& `with the street'
	%
	& kirinyeri
	& `with the streets'
	\\

\bottomrule
\end{tabu}
\label{fig:inandeclcons}
\end{figure}
~
\begin{figure}[ht]
\caption[Declension paradigm for \xayr{per}{pera}{measure}]{Declension 
paradigm for \xayr{per}{pera}{measure} (inanimate; vocalic root)}
\begin{tabu} to \linewidth {X[1] I[2] X[4] I[2] X[4]}
\tableheaderfont\toprule

	& \multicolumn2{c}{Singular}
	& \multicolumn2{c}{Plural}
	\\

\midrule
	
\Top{}
	& pera
	& `the measure'
	%
	& peraye
	& `the measures'
	\\

\midrule

\Aarg{}
	& perareng
	& `measure'
	%
	& perayereng
	& `measures'
	\\

\Parg{}
	& peraley
	& `measure' (obj.)
	%
	& perayeley
	& `measures' (obj.)
	\\

\Dat{}
	& perayam
	& `to the measure'
	%
	& perajyam
	& `to the measures'
	\\

\midrule

\Gen{}
	& perana
	& `of the measure'
	%
	& perayena
	& `of the measures'
	\\
	
\Loc{}
	& peraya
	& `at the measure'
	%
	& perajya
	& `at the measures'
	\\

\Caus{}
	& peraisa
	& `due to the measure'
	%
	& perajisa
	& `due to the measures'
	\\

\Ins{}
	& perari
	& `with the measure'
	%
	& perayeri
	& `with the measures'
	\\

\bottomrule
\end{tabu}
\label{fig:inandeclvow}
\end{figure}

\subsection{Gender}
\index{gender|(}

\begin{figure}[hb]
\caption{Grammatical genders in Ayeri}\centering
\begin{forest}
where n children=0{tier=word}{}
[grammatical gender
	[animate
		[masculine]
		[feminine]
		[neuter]
	]
	[inanimate]
]
\end{forest}
\label{fig:gramgend}
\end{figure}

Grammatical gender in Ayeri consists of two tiers which are subdivided into 
four classes based on a mixture of semantic and epistemic properties, see 
\autoref{fig:gramgend}. The animate gender refers, broadly speaking, to entities 
that are considered alive or are closely associated with living things, such 
as events, concepts, or activities executed or connected to them. The 
`masculine' and `feminine' subcategories are applied to humans, animals whose 
sex is known (for example on behalf of breeding them or keeping them as pets), 
and gods---basically anything that shows sexual dimorphism or is assumed to be 
an exponent of it as well as nouns referring to such entities in a functional 
way, for instance, \xayr{bdnF}{badan}{father} and \xayr{maav}{māva}{mother}. 
The remainder falls into the `neuter' category---plants, for instance, body 
parts, or animals whose sex is unknown. The `inanimate' category typically 
contains materials and things, such as tools. Furthermore, animals and plants 
change their category to inanimate as well if they serve as food. There are 
exceptions to either group, where elements appear in them for no obviously 
discernable reason. In order to illustrate, here are a few examples for each 
category:

\pex
	\a Animate masculine:\\
		\xayr{\larger bdnF}{badan}{father}, 
		\xayr{\larger netu}{netu}{brother}, 
		\xayr{\larger AguynF}{aguyan}{rooster}, 
		\rayr{\larger AgYaanF}{Ajān}, 
		\rayr{\larger ltunF}{Latun};
		% FIXME: bull? stallion? dog?
	
	\a Animate feminine:\\
		\xayr{\larger maav}{māva}{mother}, 
		\xayr{\larger kin}{kina}{sister}, 
		\xayr{\larger Aguvj}{aguvay}{hen}, 
		\rayr{\larger mh}{Maha}, 
		\rayr{\larger tFraanj}{Trānay};
		% FIXME: cow? mare? bitch?
	
	\a Animate neuter:\\
		\xayr{\larger AdNF}{adang}{palm tree},
		\xayr{\larger bino}{bino}{color},
		\xayr{\larger IkmF}{ikam}{deer},
		\xayr{\larger kdaanF}{kadān}{harvest},
		\xayr{\larger tYaanF}{cān}{love},
		\xayr{\larger nN}{nanga}{house}, 
		\xayr{\larger tMpu}{tampu}{luck},
		\xayr{\larger yil}{yila}{foot};
	
	\a Inanimate:\\
		\xayr{\larger AhlF}{ahal}{sand},
		\xayr{\larger hem}{hema}{egg},
		\xayr{\larger khnF}{kahan}{spear},
		\xayr{\larger meluNF}{melung}{yogurt},
		\xayr{\larger nusaanF}{nusān}{damage},
		\xayr{\larger pyutaanF}{payutān}{mathematics}.
\xe

There are also a number of doublets like French \fw{le livre} `the book' and 
\fw{la livre} `the pound', for instance, \ayr{bnnF} \fw{banan} (an.) `kindness, 
charity' or \ayr{bino} \fw{bino} (an.) `color' on the one hand, and 
\ayr{bnnF} \fw{banan} (inan.) `quality' or \ayr{bino} \emph{bino} (inan.) 
`paint' on the other. Gender is reified by case marking as well as verb 
agreement; it is not possible to read the gender of a noun from its phonological 
makeup. The following example illustrates differences in case marking and 
agreement (inherent information on grammatical features underneath the NPs):

\pex
\a\label{ex:gender1}\begingl
	\gla Ang konja badan hemaley. //
	\glb Ang kond-ya badan-Ø hema-ley //
	\glc {} {} {\tiny [\TsgM{}.\An{}}] {\tiny [\TsgI{}]} //
	\glc \AgtT{}.\An{} eat-\TsgM{}.\An{} father-\Top{} egg-\PargI{} //
	\glft `Father eats an egg.' //
\endgl

\a\label{ex:gender2}\begingl
	\gla Sa tombara kahanreng burang. //
	\glb Sa tomb-ara kahan-reng burang-Ø //
	\glc {} {} {\tiny [\TsgI{}]} {\tiny [\TsgN{}.\An{}]} //
	\glc \PatT{}.\An{} kill-\TsgI{} spear-\AargI{} animal-\Top{} //
	\glft `The animal, the spear kills it.' //
\endgl

\xe

In example (\ref{ex:gender1}), the noun in the agent NP, 
\xayr{bdnF}{badan}{father}, bears the features \textsc{[+\,animate, 
+\,masculine]}, which triggers the animate agent topic agreement marker 
\rayr{ANF}{ang} on the verb, since the agent NP is also topicalized. The verb 
also agrees in person and number with the agent NP by way of the person marker 
\rayr{/y}{-ya} for third person singular masculine. The object of the sentence, 
\xayr{hem}{hema}{egg}, on the other hand bears the feature 
\textsc{[–\,animate]}, so it receives the inanimate patient case marker 
\rayr{/lej}{-ley} rather than its animate counterpart \rayr{/AsF}{-as}.

In (\ref{ex:gender2}), on the other hand, we see an inanimate agent, 
\xayr{khnF}{kahan}{spear}, so the verb receives the marker \rayr{/Ar}{-ara} for 
third person singular inanimate rather than its animate neuter counterpart 
\rayr{/yo}{-yo}. The (non-topicalized) NP's case marking shows that the agent 
of the clause is inanimate: \rayr{khnF}{kahan} carries the marker 
\rayr{/reNF}{-reng}, which marks it as an inanimate agent. The object of the 
sentence, \xayr{burNF}{burang}{animal}, is also the topic, hence topic agreement 
on the verb uses the marker \rayr{s}{sa} according to the NP being animate, 
rather than its inanimate counterpart \rayr{le}{le}.

\index{gender|)}

\subsection{Number}
\index{number|(}

Ayeri only distinguishes singular and plural in nouns, which receive plural 
marking; verbs, then, agree with agent NPs in number in the canonical case. 
Ordinarily, nouns in Ayeri are countable, however, there is also a group of 
uncountable nouns as well as a (small) group of nouns which are always plural. 
As above, I will list a few words from each group for illustration:

\pex
	\a Countable nouns:\label{ex:plurals}\\[0.5\baselineskip]
		\makebox[6.5em][l]{\xayr{\larger AgYmF}{ajam}{toy}}
			\makebox[2em][c]{---}
			\xayr{\larger AgYmFye}{ajamye}{toys}, %
				\\[0.5\baselineskip]
		\makebox[6.5em][l]{\xayr{\larger devo}{devo}{head}}
			\makebox[2em][c]{---}
			\xayr{\larger devoye}{devoye}{heads}, %
				\\[0.5\baselineskip]
		\makebox[6.5em][l]{\xayr{\larger InunF}{inun}{fish}}
			\makebox[2em][c]{---}
			\xayr{\larger InunFye}{inunye}{fish} (pl.),%
				\\[0.5\baselineskip]
		\makebox[6.5em][l]{\xayr{\larger netu}{netu}{brother}}
			\makebox[2em][c]{---}
			\xayr{\larger netuye}{netuye}{brothers};
	
	\a Uncountable nouns:\\
		\xayr{\larger AhlF}{ahal}{sand}, 
		\xayr{\larger bkj}{bakay}{stuff}, 
		\xayr{\larger ghaanF}{gahān}{hope}, 
		\xayr{\larger miNnF}{mingan}{ability};
	
	\a Plurale tantum nouns:\\
		\xayr{\larger burNF}{burang}{lifestock, 
			cattle},\footnotemark~
		\xayr{\larger gneNnF}{ganengan}{siblings}, 
		\xayr{\larger kejnmF}{keynam}{people}, 
		\xayr{\larger tNF}{tang}{ears}.
\xe

\footnotetext{Specifically in this meaning; \rayr{burNF}{burang} can also simply 
mean `animal', in which case there is a plural form 
\xayr{burNFye}{burangye}{animals}.}

Most concrete things that exist as discrete entities are countable, also, for 
instance, animals and lifestock---fish, deer, sheep etc. are thus countable, 
unlike in English; pants, pliers, scissors, glasses, etc. are by default 
singular as well. Uncountable, on the other hand, are materials in general or 
abstract concepts. There are also a number of nouns which are plural by default, 
most notably entities which often occur in groups, but there is as well the odd 
word for which there seems to be no reason to be included in this group, for 
instance, \xayr{\larger bino}{bino}{paint}, and \xayr{\larger 
giMbj}{gimbay}{sorrows}. A few body parts are also plurale tantum nouns, 
especially those which occur in pairs (\xayr{niv}{niva}{eye} is a notable 
exception).

\index{morphophonology!of the plural marker}\index{plural}
As demonstrated in (\ref{ex:plurals}), the noun plural marker is 
\rayr{/ye}{-ye}, which in native orthography also occurs in the variant 
\ayr{*Ye} or \ayr{ʲ*e}. As described above (\autoref{pluralmorph}, 
p.~\pageref{pluralmorph}), the plural marker may also be reduced to [dʒ] 
\orth{-j} before case suffixes that begin with /j/ or with a vowel other than 
/e/, like \rayr{/ANF}{-ang} (\Aarg{}) or \rayr{/ymF}{-yam} (\Dat{}):

\pex
	\a \rayr{\larger dirnFANF}{diranang} (uncle-\Aarg{})
		+ \rayr{\larger /ye}{-ye} (\Pl{}) %\\[0.5\baselineskip]
		→ \rayr{\larger dirnFye\_aNF}{diranjang} (uncle-\Pl{}-\Aarg{}),
	\a \rayr{\larger dirnen}{diranena} (uncle-\Gen{})
		+ \rayr{\larger /ye}{-ye} (\Pl{}) %\\[0.5\baselineskip]
		→ \rayr{\larger dirnFyen}{diranyena} (uncle-\Pl{}-\Gen{}),
	\a \rayr{\larger dirnFymF}{diranyam} (uncle-\Dat{})
		+ \rayr{\larger /ye}{-ye} (\Pl{}) %\\[0.5\baselineskip]
		→ \rayr{\larger dirnFyeymF}{diranjyam} (uncle-\Pl{}-\Dat{}).
\xe

For pluralia tantum, to express a singular entity, it is always possible to 
use a genitive phrase like \xayr{—/En menF}{…-ena men}{one of …} (…-\Gen{} 
one), for instance:

\pex
\a\begingl
	\gla Nupayon tangang nā. //
	\glb Nupa-yon tang-ang nā //
	\glc hurt-\TplN{} ears-\Aarg{} \Fsg{}.\Gen{} //
	\glft `My ears hurt.' //
\endgl

\a\label{ex:gensubj}\begingl
	\gla Na nupareng tang nā men. //
	\glb Na nupa=reng tang-Ø nā men //
	\glc \GenT{} hurt=\TsgI{}.\Aarg{} ears-\Top{} \Fsg{}.\Gen{} one //
	\glft `Of my ears, it hurts one.' //
\endgl
\xe

Number in nouns can also be manipulated by quantifiers which attach to declined 
nouns as suffixes. In this case, when plurality is indicated by the 
quantifier, the noun is not additionally marked for number; the verb, however, 
keeps agreeing in number:

\pex
\a\begingl
	\gla Ajayon ganjang kivo. //
	\glb Aja-yon gan-ye-ang kivo //
	\glc play-\TsgN{} child-\Pl{}-\Aarg{} small //
	\glft `The small children are playing.' //
\endgl
	
\a\begingl
	\gla Ajayon ganang-ikan kivo. //
	\glb Aja-yon gan-ang=ikan kivo. //
	\glc play-\TsgN{} child-\Aarg{}=many small //
	\glft `Many small children are playing.' //
\endgl

\xe

Likewise, when nouns are modified by numerals,\index{numerals} plurality is not 
normally marked again on the noun. In example (\ref{ex:plurnorm}), we see a 
plural noun, \xayr{nN}{nanga}{house}, and in (\ref{ex:plurnum}) the same phrase 
is repeated again with plurality implied by the use of a numeral, 
\xayr{smF}{sam}{two}; the plural noun itself appears unmarked in its singular 
form in this case.

\pex
\a\label{ex:plurnorm}\begingl
	\gla Ang no vehya sitang-yām nangajas veno nay hiro. //
	\glb Ang no veh=ya.Ø sitang=yām nanga-ye-as veno nay hiro //
	\glc \AgtT{} want build-\TsgM.\Top{} self=\TsgM{}.\Dat{} 
		house-\Pl{}-\Parg{} pretty and new //
	\glft `He wants to build himself pretty new houses.' //
\endgl

\a\label{ex:plurnum}\begingl
	\gla Ang no vehya sitang-yām nangās sam veno nay hiro. //
	\glb Ang no veh=ya.Ø sitang=yām nanga-as sam veno nay hiro //
	\glc \AgtT{} want build-\TsgM.\Top{} self=\TsgM{}.\Dat{} house-\Parg{} 
		two pretty and new //
	\glft `He wants to build himself two pretty new houses.' //
\endgl

\xe

An exception to this is the use of words for the numeral powers, 
like \xayr{lnF}{lan}{dozen}, \xayr{menNF}{menang}{gross}, 
\xayr{smNF}{samang}{myriad}, etc. in an unspecified way like `dozens 
of people'. In this case, to convey that the numeral is not to be understood as 
a precise value, the modified noun will appear in the plural---even if 
it is a plurale tantum like \xayr{kejnmF}{keynam}{people}:

\ex\begingl
	\gla Bengyon keynamjang menang. //
	\glb Beng-yon keynam-ye-ang menang //
	\glc attend-\TsgN{} people-\Pl{}-\Aarg{} gross //
	\glft `Hundreds of people attended.' //
\endgl\xe

% This is a new rule; earlier names were treated as countable but still carried 
% special case marking. I found this slightly weird, however so, let us simply 
% assert this new rule, which should make things more consistent. The odd 
% case of a pluralized name could still be explained as individual variation, 
% though I can't think of an example where this was ever an issue.
%
As we have seen in various examples above, proper nouns in Ayeri do not 
receive inflection for case by suffixes as common nouns do, and for the 
purpose of number they are treated as uncountable in Ayeri---they resist 
inflection by suffixation, marking their special status.\footnote{Many common 
names in Ayeri are derived from regular words in the language, so the language 
needs to have a way to distinguish between regular use and use as a name. For 
instance, the name \rayr{ynF}{Yan} also means `boy, son' as a common noun.} 
However, they can still be modified by quantifiers and quantifying suffixes; 
verb agreement as well can be used to indicate plurality:

\pex
\a\begingl
	\gla Sahayan cabo ekeng ang Yan. //
	\glb Saha-yan cabo ekeng ang Yan //
	\glc come-\TplM{} late too \Aarg{} Yan //
	\glft `The Yans are coming too late.' //
\endgl

\a\begingl
	\gla Ang apatang sa Yan-ikan. //
	\glb Ang apa=teng sa Yan=ikan //
	\glc \AgtT{} laugh=\TplF{}.\Aarg{} \Parg{} Yan=all //
	\glft `They laughed at (all) the Yans.' //
\endgl

\xe

\index{number|)}

\subsection{Case}
\index{cases|(}

As demonstrated in the declension tables at the beginning of this section 
(Figures \ref{fig:anideclcons}–\ref{fig:inandeclvow}), Ayeri's NPs are marked 
for case, which is governed by the verb. Since Ayeri uses a split alignment 
system with some additional complications, it is not very straightforward, in my 
opinion, to use the classical labels of nominative (S/A) and accusative (O), or 
of absolutive (S/P) and ergative (O) for the first two core roles. Hence, I will 
be using the terms `agent' and `patient', which I hope brings about some more 
clarity, especially when discussing the mentioned complications later on.


\subsubsection{Agent}
\index{cases!agent|(}

To quote \citet{fillmore1968}, what I call `agent' here is  
\textcquote[46]{fillmore1968}{the case of the typically animate perceived 
instigator of the action identified by the verb}. \citeauthor{fillmore1968} 
himself qualifies this definition, however, in that the \textcquote[46, 
footnote 31]{fillmore1968}{escape qualification `typically' expresses my 
awareness that contexts which I will say require agents are sometimes occupied 
by `inanimate' nouns like robot or `human institution' nouns like nation}. 
\citet{payne1997} summarizes on prototypical agents with regards to 
their topicality that a \textcquote[151]{payne1997}{less technical way of 
expressing this fact is to say that people identify with and like to talk about 
things that act, move, control events, and have power}.

Agents in Ayeri frequently embody the properties quoted by both 
\citeauthor{fillmore1968} and \citeauthor{payne1997} in this regard, including 
\citeauthor{fillmore1968}'s caveat. However, importantly, `agent' in Ayeri is a 
macrorole that may be applied to, for instance, instruments, experiencers, and 
less typical actors as well, specifically, in absence of more prototypical 
candidates for agenthood in a sentence. It thus comes very close to a 
nominative, except that it does not need to be locus of the sentence's 
topic\index{topic}---although agents very typically are topics, as 
\citet[151]{payne1997} goes on to note.\footnote{This is the main reason I spoke 
of `complications' above: Ayeri's notion of `subject' is somewhat problematic 
due to topicalization, which is why I try to avoid complicating terminology by 
using `nominative' for agent topics and `ergative' for agent non-topics, and 
`accusative' for patient non-topics and `absolutive' for patient topics, 
respecitvely.} Thus, the first NP after the verb in all of the following 
examples is treated as an agent; the agent is marked by the suffix 
\rayr{/ANF}{-ang} for animate referents and the suffix \rayr{/reNF}{-reng} for 
inanimate referents; names and verbal topic agreement are marked by 
\rayr{ANF}{ang} and \rayr{ENF}{eng}, respectively:

\pex
\a\begingl
	\gla \textbf{Ang} tinkaya \textbf{{}} \textbf{Yan} kunangley. //
	\glb \textbf{Ang} tinka-ya \textbf{Ø} \textbf{Yan} kunang-ley //
	\glc \textbf{\AgtT{}} open-\TsgM{} \textbf{\Top{}} \textbf{Yan} 
		door-\PargI{} //
	\glft `Yan opens the door.' //
\endgl

\a\begingl
	\gla Le tinkaya \textbf{ayonang} kunang. //
	\glb Le tinka-ya \textbf{ayon-ang} kunang-Ø //
	\glc \PatT{} open-\TsgM{} \textbf{man-\Aarg{}} door-\Top{} //
	\glft `The door is opened by a/the man',\\
		or: `The door, a/the man opens it.' //
\endgl

\a\begingl
	\gla \textbf{Eng} tinkāra \textbf{tinkay} kunangley. //
	\glb \textbf{Eng} tinka-ara \textbf{tinkay-Ø} kunang-ley //
	\glc \textbf{\AgtTI{}} open-\TsgI{} \textbf{key-\Top{}} door-\PargI{} //
	\glft `The key opens the door.' //
\endgl

\a\begingl
	\gla Tinkāra \textbf{kunangreng}. //
	\glb Tinka-ara \textbf{kunang-reng} //
	\glc open-\TsgI{} \textbf{door-\AargI{}} //
	\glft `The door opens.' //
\endgl

\a\begingl
	\gla Sā tinkaya \textbf{ang} \textbf{Yan} kunangley yan. //
	\glb Sā tinka-ya \textbf{ang} \textbf{Yan} kunang-ley yan.Ø //
	\glc \CauT{} open-\TsgM{} \textbf{\Aarg{}} \textbf{Yan} door-\PargI{} 
		\TsgM{}.\Top{} //
	\glft `They make Yan open a/the door',\\
		or: `Because of them, Yan opens the door.' //
\endgl

\xe

In predicative constructions, the constituent which a quality is assigned to or 
about which a judgement is made is also assigned the agent case:

\pex
\a\begingl
	\gla Tado \textbf{tinkayreng}. //
	\glb Tado \textbf{tinkay-reng} //
	\glc old \textbf{key-\AargI{}} //
	\glft `The key is old.' //
\endgl

\a\begingl
	\gla \textbf{Ang} \textbf{Yan} nimpayās ban. //
	\glb \textbf{Ang} \textbf{Yan} nimpaya-as ban //
	\glc \textbf{\Aarg{}} \textbf{Yan} runner-\Parg{} good //
	\glft `Yan is a good runner.' //
\endgl

\xe

\index{cases!agent|)}

With regards to constituents' roles in ditransitive verb frames, donors are 
represented by agents in Ayeri as well, since they are the origin of whatever 
is conceptually passed on to the recipient party:

\ex\begingl
	\gla Le ilya \textbf{ang} \textbf{Yan} tinkay yam Cānlay. //
	\glb Le il-ya \textbf{ang} \textbf{Yan} tinkay-Ø yam Cānlay //
	\glc \PatT{} give-\TsgM{} \textbf{\Aarg{}} \textbf{Yan} key-\Top{} 
		\Dat{} Cānlay //
	\glft `The key, Yan gives it to Cānlay.' //
\endgl\xe

\subsubsection{Patient}
\index{cases!patient|(}

Patients are less of a definitional problem than agents, since in transitive 
sentences, they are very typically undergoers, that is, the constituent that is 
acted on, affected, or produced by the action expressed by the verb. The patient 
case is thus the one assigned by default to direct objects---but also to 
predicative nominals. In ditransitive sentences, the theme is represented by the 
patient. Animate patients are marked by \rayr{/AsF}{-as}, inanimate ones by 
\rayr{/lej}{-ley}; for names and verbal topic agreement, the markers are 
\rayr{s}{sa} and \rayr{le}{le}, respectively:

\pex
\a\begingl
	\gla Ang silvye {} Briha \textbf{sa} \textbf{Taryan}. //
	\glb Ang silv-ye Ø Briha \textbf{sa} \textbf{Taryan} //
	\glc \AgtT{} see-\TsgF{} \Top{} Briha \textbf{\Parg{}} \textbf{Taryan}//
	\glft `Briha sees Taryan.' //
\endgl

\a\begingl
	\gla \textbf{Sa} manye ang Briha \textbf{{}} \textbf{Taryan}. //
	\glb \textbf{Sa} man-ye ang Briha \textbf{Ø} \textbf{Taryan} //
	\glc \textbf{\PatT{}} greet-\TsgF{} \Aarg{} Briha \textbf{\Top{}} 
		\textbf{Taryan} //
	\glft `Taryan is greeted by Briha',\\
		or: `Taryan, Briha greets him.' //
\endgl

\xe

\pex~
\a\begingl
	\gla Ang rimaye {} Briha \textbf{kunangley}. //
	\glb Ang rima-ye Ø Briha \textbf{kunang-ley} //
	\glc \AgtT{} close-\TsgF{} \Top{} Briha \textbf{door-\PargI{}} //
	\glft `Briha closes a/the door.' //
\endgl

\a\begingl
	\gla \textbf{Le} rimaye ang Briha \textbf{kunang}. //
	\glb \textbf{Le} rima-ye ang Briha \textbf{kunang-Ø} //
	\glc \textbf{\PatTI{}} close-\TsgF{} \Aarg{} Briha 
		\textbf{door-\Top{}} //
	\glft `The door is closed by Briha',\\
		or: `The door, Briha closes it.' //
\endgl

\xe

\ex~
\begingl
	\gla Ang ilya {} Taryan \textbf{koyaley} yam Kandan. //
	\glb Ang il-ya Ø Taryan \textbf{koya-ley} yam Kandan //
	\glc \AgtT{} give-\TsgM{} \Top{} Taryan \textbf{book-\PargI{}} \Dat{} 
		Kandan //
	\glft `Taryan gives Kandan a book.' //
\endgl

\xe

As the translations of the examples above show, topicalizing the patient can be 
used to create an effect similar to English's passive voice, except that the 
patient will not become marked by the agent case for logical reasons---this is 
a notable difference from the nominative. Even if the agent NP is omitted, the 
patient NP will not be changed to the agent case, since that would reverse the 
direction of action:

\ex\begingl
	\gla Manya sa Taryan. ≠ Manya ang Taryan. //
	\glb Man-ya sa Taryan {} Man-ya ang Taryan //
	\glc greet-\TsgM{} \Parg{} Taryan {} greet-\TsgM{} \Aarg{} Taryan //
	\glft `Taryan is greeted.' ≠ `Taryan greets.' //
\endgl\xe

This example shows that the case of the NP will not change, however, the 
verb will: it now agrees with the next argument in line, the patient NP. It will 
not do so, however, if the order of arguments is just scrambled, as 
exemplified by (\ref{ex:verbscram}). This is to say that the verb does not 
simply agree with whichever NP follows it, even if it can be assumed that verb 
agreement in Ayeri developed along similar lines in-world, which will become 
especially apparent in the discussion of pronouns.\footnote{Mismatches in 
agreement in connection to scrambling such as exemplified by 
(\ref{ex:scramfalse}) are to be expected, however, since the brain can only 
handle so much information between the controller and the target of an 
agreement relationship. \citet{corbett2006}, notes that with regards to 
agreement in NP conjuncts, \textcquote[62]{corbett2006}{distant agreement is 
rare, and that agreement with the nearest noun phrase or agreement with all 
(resolution) is much more common}. If there were an extensive corpus of 
texts written by Ayeri speakers, it might be interesting to gather statistics 
on the number of words between target and controller in relation to the 
prevalence of agreement mismatches.}

\tikzstyle{every picture}+=[remember picture]
\pex[aboveglftskip=2em]\label{ex:verbscram}
\a\label{ex:scramcorr}\begingl
	\gla Sa manye {} Taryan ang Briha. //
	\glb Sa man-ye Ø Taryan ang Briha //
	\glc \PatT{} greet\tikz\node[na](target){-\TsgF{}}; \Top{} Taryan 
		\Aarg{} \tikz\node[na](controller){Briha}; //
	\glft `Taryan is greeted by Briha',\\
		or: `Taryan, Briha greets him.' //
\endgl

\a\label{ex:scramfalse}\begingl
	\gla *Sa manya {} Taryan ang Briha. //
	\glb Sa man-ya Ø Taryan ang Briha //
	\glc \PatT{} greet\tikz\node[na](target2){-\TsgM{}}; \Top{} 
		\tikz\node[na](controller2){Taryan}; \Aarg{} Briha //
\endgl\xe
\begin{tikzpicture}[overlay]
	\coordinate [below=.25em of controller] (A);
	\coordinate [below=1em of controller] (B);
	\coordinate [below=1em of target] (C);
	\coordinate [below=.25em of target] (D);
	\draw [-latex] (A) -- (B) -- (C) -- (D);
	\node (label) at ($(B)!0.5!(C)$) [below] {\tiny\itshape person 
		agreement};
	
	\coordinate [below=.25em of controller2] (A);
	\coordinate [below=1em of controller2] (B);
	\coordinate [below=1em of target2] (C);
	\coordinate [below=.25em of target2] (D);
	\draw [-latex, dotted] (A) -- (B) -- (C) -- (D);
	\node (label) at ($(B)!0.5!(C)$) [below] {\tiny\itshape *person 
		agreement};
\end{tikzpicture}

Besides being the default case for direct objects, the patient case is also 
assigned to predicative nominals, by analogy with transitive sentences and in 
spite of the likening nature of the construction:

\ex\begingl
	\gla Ang Yan \textbf{nimpayās} ban. //
	\glb Ang Yan \textbf{nimpaya-as} ban //
	\glc \Aarg{} Yan \textbf{runner-\Parg{}} good //
	\glft `Yan is a good runner.' //
\endgl\xe

\index{cases!patient|)}

\subsubsection{Dative}
\index{cases!dative|(}

The most typical use of the dative is for the recipient NP in a ditransitive 
clause; as such, it may be a recipient proper or the entity to whose benefit 
the action is carried out. A number of transitive verbs also use the dative 
for their object, for example, when it is the target of address. The dative can 
furthermore be used to mark movement toward a place. The case suffix for 
datives is \rayr{/ymF}\fw{-yam} for both animate and inanimate entities. Names 
and verbal topic agreement are marked equally by \rayr{ymF}{yam}. Verbs do not 
exhibit person agreement with dative NPs, since experiencers are treated as 
agents.

\pex\label{ex:datregular}
\a\begingl
	\gla Ang ilya {} Taryan koyaley \textbf{ayonyam}. //
	\glb Ang il-ya Ø Taryan koya-ley \textbf{ayon-yam} //
	\glc \AgtT{} give-\TsgM{} \Top{} Taryan book-\PargI{} 
		\textbf{man-\Dat{}} //
	\glft `Taryan gives a book to the man.' //
\endgl

\a\begingl
	\gla Ang ilya {} Taryan koyaley \textbf{yam} \textbf{Kandan}. //
	\glb Ang il-ya Ø Taryan koya-ley \textbf{yam} \textbf{Kandan} //
	\glc \AgtT{} give-\TsgM{} \Top{} Taryan book-\PargI{} \textbf{\Dat{}} 
		\textbf{Kandan} //
	\glft `Taryan gives Kandan a book.' //
\endgl

\a\begingl
	\gla \textbf{Yam} ilya ang Taryan koyaley \textbf{ayon}. //
	\glb \textbf{Yam} il-ya ang Taryan koya-ley \textbf{ayon-Ø} //
	\glc \textbf{\DatT{}} give-\TsgM{} \Aarg{} Taryan book-\PargI{} 
		\textbf{man-\Top{}} //
	\glft `The man is given a book by Taryan',\\
		or: `The man, Taryan gives him a book.' //
\endgl

\xe

The three examples in (\ref{ex:datregular}) show the regular use of the dative 
as the case the recipient of the theme appears in. What distinguishes Ayeri 
from a pure split-S language is that all constituents can serve as topics, not 
just agents and patients with regards to their function as syntactic subjects. 
Thus, it is also possible for dative NPs to appear as topics---person 
agreement is unaffected by this, though. The following example shows the 
addressee of a speech act in the dative case; the message is treated as the 
theme which is passed on:

\ex
\begingl
	\gla Ang ningye māva ninganas \textbf{ganyam} yena. //
	\glb Ang ning-ye māva-Ø ningan-as \textbf{gan-yam} yena //
	\glc \AgtT{} tell-\TsgF{} mother-\Top{} story-\Parg{} 
		\textbf{child-\Dat{}} \TsgF{}.\Gen{} //
	\glft `The mother tells her child a story.' //
\endgl
\xe

As mentioned above, the dative can also take on an allative meaning insofar as 
it marks the target of a motion, as displayed in (\ref{ex:datloc}). As an 
extention of this means, the adpositional object may as well appear in the 
dative, since Ayeri cannot distinguish, for instance, `up' from `to the top of' 
with just the preposition, in this case \xayr{liNF}{ling}{on top of}. With the 
adpositional object in the locative case (see below), the phrase in 
(\ref{ex:datlocprep}) would imply that the man were literally going to the top 
of the temple, that is, possibly ending up on its roof.

\pex
\a\label{ex:datloc}\begingl
	\gla Ang nimpye lay \textbf{māvayam} yena. //
	\glb Ang nimp-ye lay-Ø \textbf{māva-yam} yena //
	\glc \AgtT{} run-\TsgF{} girl-\Top{} \textbf{mother-\Dat{}} 
		\TsgF{}.\Gen{} //
	\glft `The girl runs to her mother.' //
\endgl

\a\label{ex:datlocprep}\begingl
	\gla Ang saraya ayon manga ling \textbf{natrangyam}. //
	\glb Ang sara-ya ayon-Ø manga ling \textbf{natrang-yam} //
	\glc \AgtT{} go-\TsgM{} man-\Top{} \Dyn{} top \textbf{temple-\Dat{}} //
	\glft `The man goes up to the temple.' //
\endgl

\xe

\index{cases!dative|)}

\subsubsection{Genitive}
\index{cases!genitive|(}

The genitive is used to mark possessors; attributive genitives follow the 
possessee. It can also be used for ablative meanings, that is, to mark the place 
from which a motion originates, in analogy to the dative's allative use. 
The genitive is marked on common nouns with the suffix \rayr{/n}{-na}. If a 
noun stem ends in a consonant, the marker becomes \rayr{/En}{-ena}, compare 
Figures \ref{fig:anideclcons}–\ref{fig:inandeclvow} above. Names and verbal 
topic agreement are marked by \rayr{n}{na}. There is no animacy distinction in 
the genitive case.

\pex
\a\begingl
	\gla Pakur ledanang \textbf{netuna} nā. //
	\glb Pakur ledan-ang \textbf{netu-na} nā //
	\glc sick friend-\Aarg{} \textbf{brother-\Gen{}} \Fsg{}.\Gen{} //
	\glft `My brother's friend is sick.' //
\endgl

\a\begingl
	\gla Kopo dilengyereng \textbf{ajānena}. //
	\glb Kopo dileng-ye-reng \textbf{ajān-ena} //
	\glc difficult rule-\Pl{}-\AargI{} \textbf{game-\Gen{}} //
	\glft `The rules of the game are difficult.' //
\endgl

\a\begingl
	\gla Ang nakasyo tamo ibangya \textbf{na} \textbf{Niyas}. //
	\glb Ang nakas-yo tamo-Ø ibang-ya \textbf{na} \textbf{Niyas} //
	\glc \AgtT{} grow-\TsgN{} wheat-\Top{} field-\Loc{} \textbf{\Gen} 
		\textbf{Niyas} //
	\glft `There is wheat growing on Niyas's field.' //
\endgl

\a\begingl
	\gla \textbf{Na} nakasyo tamoang ibangya \textbf{{}} \textbf{Niyas}. //
	\glb \textbf{Na} nakas-yo tamo-ang ibang-ya \textbf{Ø} \textbf{Niyas} //
	\glc \textbf{\GenT{}} grow-\TsgN{} wheat-\Aarg{} field-\Loc{} 
		\textbf{\Top} \textbf{Niyas} //
	\glft `Regarding Niyas, there is wheat growing on his field.' //
\endgl

\xe

Futhermore, Ayeri does not make a distinction between alienable and inalienable 
possession, so that typically inalienable things such as body parts, 
relatives and family members, or personal items and tools are all treated 
as described above. Consider the following example for illustration:

\ex\begingl
	\gla Ang puntaye māva \textbf{nā} mitrangas \textbf{yena} sembari 
		\textbf{yena}. //
	\glb Ang punta-ye māva-Ø \textbf{nā} mitrang-as \textbf{yena} semba-ri 
		\textbf{yena} //
	\glc \AgtT{} brush-\TsgF{} mother-\Top{} \textbf{\Fsg{}.\Gen{}} 
		hair-\Parg{} \textbf{\TsgF{}.\Gen{}} comb-\Ins{} 
		\textbf{\TsgF{}.\Gen{}} //
	\glft `My mother is brushing her hair with her comb.' //
\endgl\xe

The above examples show the regular use of the genitive as a marker of 
possession. The following examples, on the other hand, show the genitive in its 
ablative function, first without qualification by a preposition, then with the 
preposition \xayr{AvnF}{avan}{at the bottom of}, which together with the 
genitive assumes the meaning `down from':

\pex
\a\begingl
	\gla Ang sahaya {} Vetayan \textbf{rimanena}. //
	\glb Ang saha-ya Ø Vetayan \textbf{riman-ena} //
	\glc \AgtT{} come-\TsgM{} \Top{} Vetayan \textbf{city-\Gen{}} //
	\glft `Vetayan comes from the city.' //
\endgl

\a\begingl
	\gla Sahu manga avan \textbf{mehirena}, Niva! //
	\glb Saha-u manga avan \textbf{mehir-ena}, Niva //
	\glc come-\Imp{} \Dyn{} at.bottom \textbf{tree-\Gen{}}, Niva //
	\glft `Come down from the tree, Niva!' //
\endgl

\xe

\index{cases!genitive|)}

\subsubsection{Locative}
\index{cases!locative|(}

The locative marks basic locations, often the default that is associated with a 
verb. It is also the case in which adpositional objects normally appear, 
besides the special cases using the dative and the genitive mentioned above. 
Common nouns are marked by \rayr{/y}{-ya};\footnote{Older texts still exhibit 
an allomorph \rayr{/E\_a}{-ea}, used especially in combination with the plural 
suffix \rayr{/ye}, giving \rayr{/yee\_a}{-yēa}. The modern language uses 
\rayr{/yey}{-jya}.} names and verbal topic agreement use the marker 
\rayr{y}{ya}. There is no difference made between animate and inanimate 
referents in the locative.

\pex\label{ex:locplain}
\a\label{ex:locnedra}\begingl
	\gla Ang nedraya paray \textbf{hinya}. //
	\glb Ang nedra-ya paray-Ø \textbf{hin-ya} //
	\glc \AgtT{} sit-\TsgM{} cat-\Top{} \textbf{box-\Loc{}} //
	\glft `The cat sits in the box.' //
\endgl

\a\label{ex:locnara}\begingl
	\gla Ang naraya {} Ajān \textbf{ya} \textbf{Kaman}. //
	\glb Ang nara-ya Ø Ajān \textbf{ya} \textbf{Kaman} //
	\glc \AgtT{} speak-\TsgM{} \Top{} Ajān \textbf{\Loc{}} \textbf{Kaman} //
	\glft `Ajān speaks to Kaman.' //
\endgl

\a\label{ex:locmit}\begingl
	\gla \textbf{Ya} mica ang Kaman \textbf{{}} \textbf{Visamhinang}. //
	\glb \textbf{Ya} mit-ya ang Kaman \textbf{Ø} \textbf{Visamhinang} //
	\glc \textbf{\LocT{}} live-\TsgM{} \Aarg{} Kaman \textbf{\Top{}} 
		\textbf{Visamhinang} //
	\glft `Kaman lives in Visamhinang',\\
		or: `Visamhinang is where Kaman lives.' //
\endgl

\xe

The example sentences in (\ref{ex:locplain}) show locative NPs that are not 
further specified by adpositions so that the correct interpretation may be 
dependent on context and the experience of the addressee. Example 
(\ref{ex:locnedra}) is an instance of this circumstance, insofar as experience 
tells that cats like to sit inside boxes, so further specifying the position 
with the preposition \xayr{koNF}{kong}{inside} would be emphasizing that the cat 
is not sitting just anywhere, but really \emph{inside} the box as opposed to on 
top of it, for instance. The following example has the cat sitting on top of the 
box:

\ex\begingl
	\gla Ang nedraya paray ling hinya. //
	\glb Ang nedra-ya paray-Ø ling hin-ya //
	\glc \AgtT{} sit-\TsgM{} cat-\Top{} on.top box-\Loc{} //
	\glft `The cat sits on the box.' //
\endgl\xe

Ayeri also has a number of postpositions, which do not change marking on the 
adpositional object, however:

\ex\begingl
	\gla Ang mican edaya \textbf{tenyanya} tan pesan. //
	\glb Ang mit-yan edaya \textbf{tenyan-ya} tan pesan //
	\glc \AgtT{} live-\TplM{} here \textbf{death-\Loc{}} \TplM{}.\Gen{} 
		until //
	\glft `They lived here until their death.' //
\endgl\xe

\index{cases!locative|)}

\subsubsection{Causative}
\index{cases!causative|(}

The causative marks the cause or causer of an action, the instigator or the 
reason on behalf of which an agent is acting. It is thus similar to the agent 
case, though it does not replace it in Ayeri; verbs do not exhibit person 
agreement with causers even though their action logically supersedes or precedes 
that of the agent in the embedded event. \citet{dixon2000} writes that a 
\textcquote[30]{dixon2000}{causer refers to someone or something (which can be 
an event or state) that initiates or controls the activity. This is the defining 
property of the syntactic--semantic function A (transitive subject)}. According 
to \citet[176]{comrie1989}, the causee---the agent of the event controlled by 
the causer---normally takes the highest place in the hierarchy of syntactic 
constituents that is not already filled, in this case, by the causer. This 
observation, however, is complicated by Ayeri's more or less semantics-based 
case marking as well as topicalization. In the following, I will give examples 
of nominal marking for cause as before; a discussion of the morphosyntax of 
Ayeri's morphological causative constructions will be deferred to the section on 
valency-increasing operations.

Causers or causes are marked by \rayr{/Is}{-isa} for common nouns; names and 
verbal topic agreement use the marker \rayr{saa}{sā}. As stated above, verbs do 
not agree with causers even though they have agent-like semantics. There is no 
animacy distinction in the marking of causers.

\pex
\a\begingl
	\gla Ang rua sarāyn \textbf{seyaranisa}. //
	\glb Ang rua sara=ayn.Ø \textbf{seyaran-isa} //
	\glc \AgtT{} must leave=\Fpl{}.\Top{} \textbf{rain-\Caus{}} //
	\glft `We had to leave due to the rain.' //
\endgl

\a\begingl
	\gla Ang yomāy edaya \textbf{sā} \textbf{Apican}. //
	\glb Ang yoma=ay.Ø edaya \textbf{sā} \textbf{Apican} //
	\glc \AgtT{} be=\Fsg{}.\Top{} here \textbf{\Caus{}} \textbf{Apican} //
	\glft `I am here because of Apican.' //
\endgl

\a\label{ex:caustop}\begingl
	\gla \textbf{Sā} nimpvāng hakasley \textbf{yan}. //
	\glb \textbf{Sā} nimp=vāng hakas-ley \textbf{yan.Ø} //
	\glc \textbf{\CauT{}} run=\Ssg{}.\Aarg{} mile-\PargI{} 
		\textbf{\TplM{}.\Top{}} //
	\glft `You run a mile because of them',\\
		or: `Due to them, you run a mile',\\
		or: `They make you run a mile.' //
\endgl

\xe

Regarding the typological oddities mentioned above, example (\ref{ex:caustop}) 
shows what happens in Ayeri with regards to the marking of causers. 
Essentially, the causer topic was grammaticalized to express a causation 
relationship.

\index{cases!causative|)}

\subsubsection{Instrumental}
\label{subsubsec:instrumental}
\index{cases!instrumental|(}

The instrumental marks the means by which an action is carried out by an agent. 
This can be a tool as well as an animate being by whose help the action is 
brought about. The instrumental thus, in effect, marks secondary agents; verbs, 
however, never show person agreement with instrumental NPs. Common nouns are 
marked by \rayr{/ri}{-ri} when ending in a vowel and with \rayr{/Eri}{-eri} 
when ending in a consonant; names and verbal topic agreement are marked by 
\rayr{ri}{ri}. With nouns ending in \fw{-e}, as well as the plural marker 
\rayr{/ye}{-ye}, there is variation regarding whether \rayr{/ri}{-ri} or 
\rayr{/Eri}{-eri} is used, so that in the case of the plural marker both 
\rayr{/yeri}{-yeri} and \rayr{/yeeri}{-yēri} occur. In passive-like 
constructions, it is not grammatical to reintroduce the agent as an 
instrumental; the agent simply remains in the clause in this case, though as a 
non-topic constituent.

\pex
\a\begingl
	\gla Ang visye {} Pila seygoley \textbf{tihangeri} yena. //
	\glb Ang vis-ye Ø Pila seygo-ley \textbf{tihang-eri} yena. //
	\glc \AgtT{} cut-\TsgF{} \Top{} Pila apple-\PargI{} 
		\textbf{knife-\Ins{}} \TsgF{}.\Gen{} //
	\glft `Pila cuts an apple with her knife.' //
\endgl

\a\begingl
	\gla Ang lihoyya-ma badan \textbf{nihanyeri} \textbf{(nihanyēri)}. //
	\glb Ang liha-oy-ya=ma badan-Ø \textbf{nihan-ye-ri} 
		\textbf{(nihan-ye-eri)} //
	\glc \AgtT{} earn-\Neg{}-\TsgM{}=enough father-\Top{} 
		\textbf{nihan-\Pl{}-\Ins} \textbf{(nihan-\Pl{}-\Ins)} //
	\glft `Father did not earn enough with his fruits.' //
\endgl

\a\begingl
	\gla Ang lingya {} Mindan mehiras \textbf{ri} \textbf{Kadijān}. //
	\glb Ang ling-ya Ø Mindan mehir-as \textbf{ri} \textbf{Kadijān}. //
	\glc \AgtT{} climb.up-\TsgM{} \Top{} Mindan tree-\Parg{} 
		\textbf{\Ins{}} \textbf{Kadijān} //
	\glft `Mindan climbs a tree with Kadijān's help.' //
\endgl

\a\begingl
	\gla \textbf{Ri} tavya gino ang Kan \textbf{nimpur}. //
	\glb \textbf{Ri} tav-ya gino ang Kan \textbf{nimpur-Ø} //
	\glc \textbf{\InsT{}} become-\TsgM{} drunk \Aarg{} Kan 
		\textbf{wine-\Top{}} //
	\glft `Kan becomes drunk on the wine', \\
		or: `The wine, Kan becomes drunk on it.' //
\endgl

\xe

The instrumental may also be used for comitative meanings where the 
instrumental NP describes an attribute of its antecedent, for example:

\ex\begingl
	\gla Ang pegayo sinya kasuley \textbf{bariri} nā? //
	\glb Ang pega-yo sinya-Ø kasu-ley \textbf{bari-ri} nā //
	\glc \AgtT{} steal-\TsgN{} who-\Top{} basket-\PargI{} 
		\textbf{meat-\Ins{}} \Fsg{}.\Gen{} //
	\glft `Who stole my basket of meat?' //
\endgl\xe

In this case, \rayr{bri}{bari} is marked as an instrumental since it is an 
attribute of sorts to \rayr{ksu}{kasu}: the instrumental NP describes what its 
antecedent contains or entails more specifically: it is a basket \fw{with} meat 
in it. Note, however, that this comitative use of the instrumental is different 
from mere accompaniment. Thus, it is not possible to say

\ex\label{ex:wrongcomit}\begingl
	\gla *Ang sahaya {} Ajān \textbf{ri} \textbf{Pila}. //
	\glb Ang saha-ya Ø Ajān \textbf{ri} \textbf{Pila} //
	\glc \AgtT{} come-\TsgM{} \Top{} Ajān \textbf{\Ins{}} \textbf{Pila} //
\endgl\xe

\noindent to express `Ajān comes (together) with Pila'. The sentence in 
(\ref{ex:wrongcomit}) would instead imply that Pila helps Ajān to come, for 
example, because he has a sprained ankle and thus needs support to go places. 
To express accompaniment, instead, the preposition \xayr{kjvo}{kayvo}{with, 
along, beside} has to be used; the prepositional object appears in the locative 
case:

\ex\begingl
	\gla Ang sahaya {} Ajān \textbf{kayvo} \textbf{ya} \textbf{Pila}. //
	\glb Ang saha-ya Ø Ajān \textbf{kayvo} \textbf{ya} \textbf{Pila} //
	\glc \AgtT{} come-\TsgM{} \Top{} Ajān \textbf{with} \textbf{\Loc{}} 
		\textbf{Pila} //
	\glft `Ajān comes (together) with Pila.' //
\endgl\xe

Theoretically, it should be possible as well to use the instrumental together 
with prepositions for some kind of prolative meaning. The adposition would 
indicate the place \emph{by way of} a motion is happening:

\ex\begingl
	\gla Ang pukay manga luga \textbf{lahaneri}. //
	\glb Ang puk=ay.Ø manga luga \textbf{lahan-eri} //
	\glc \AgtT{} jump=\Fsg{}.\Top{} \Dyn{} top \textbf{fence-\Ins{}} //
	\glft `I jump over the fence.' //
\endgl\xe

This use of the instrumental is unattested in previous translations into Ayeri, 
however, but could be considered a stylistic alternative---in the case of the 
example above, to a construction with the word for `over', 
\rayr{EjrrY}{eyrarya}:

\ex\begingl
	\gla Ang pukay manga eyrarya lahanya. //
	\glb Ang puk=ay.Ø manga eyrarya lahan-ya //
	\glc \AgtT{} jump=\Fsg{}.\Top{} \Dyn{} over fence-\Loc{} //
	\glft `I jump over the fence.' //
\endgl\xe

A more literal translation of \rayr{mN lug lhneri}{manga luga lahaneri} is `by 
way of the top of the fence', though without the verbosity of the English 
translation, as both ways to express the circumstance are about equally long in 
Ayeri.

\index{cases!instrumental|)}

\subsubsection{Case-unmarked nouns}
\label{subsec:uncased}

Case endings are applied to nouns in Ayeri only if the word is actually in a 
syntactic context where case can be applied. Thus, the unmarked form is the 
citation form, not the one declined for agent. This is the case when addressing 
people---one might speak of an unmarked vocative:

\pex
\a\label{ex:vocnoun}\begingl
	\gla Raypu, \textbf{petāya}! //
	\glb Raypa-u, \textbf{petāya} //
	\glc stop-\Imp{}, \textbf{idiot} //
	\glft `Stop it, you idiot!' //
\endgl

\a\label{ex:vocname}\begingl
	\gla Sahu edaya, \textbf{Diras}! //
	\glb Saha-u edaya, \textbf{Diras} //
	\glc come-\Imp{} here, \textbf{Diras} //
	\glft `Come here, Diras!' //
\endgl
\xe

Imperative forms have underlying second-person agents, so both the `idiot' in 
(\ref{ex:vocnoun}) and Diras in (\ref{ex:vocname}) would be the implied agents 
of their sentences, yet neither the noun nor the name are marked by the agent 
markers \rayr{/ANF}{-ang} and \rayr{ANF}{ang}, respectively. Another case where 
nouns are not necessarily marked for case is attested in translations for the 
prefix \xayr{ku/}{ku-}{like, as though} when the phrase acts as an adverb or an 
object complement:

\pex
\a\label{ex:kuudhr}\begingl
	\gla … nay ang mya rankyon sitanyās \textbf{ku-netu}. //
	\glb … nay ang mya rank=yon.Ø sitanya-as \textbf{ku=netu} //
	\glc … and \AgtT{} be.supposed.to treat=\TplN{}.\Top{} 
		each.other-\Parg{} \textbf{like=brother} //
	\glft `… and they shall treat each other like brothers.'\footnotemark%
	\tc{\citep{benung:udhr}}//
\endgl

\a\label{ex:kukafka}\begingl
	\gla … ang nunaya \textbf{ku-vipin} … //
	\glb … ang nuna=ya.Ø \textbf{ku=vipin} … //
	\glc … \AgtT{} fly=\TsgM{}.\Top{} \textbf{like=bird} … //
	\glft `… he (would) fly like a bird …'%
	\tc{\citep[14]{becker:kafka:imperial}}//
\endgl

\xe

\footnotetext{The original English text this was translated from has 
\textcquote[Article 1]{udhr}{and should act towards one another in a spirit of 
brotherhood}.}

Strikingly, in example (\ref{ex:kuudhr}), \xayr{netu}{netu}{brother} in 
\xayr{ku/netu}{ku-netu}{like brothers} is not even inflected for plural; its 
placement after the object is likely an effect of translation: adverbs 
have a strong tendency to appear right after the verb, and a position 
immediately to the right of the verb is attested for adjectival object 
predicatives as well. In (\ref{ex:kukafka}), on the other hand, 
\xayr{ku/vipinF}{ku-vipin}{like a bird} is feasibly interpretable as an adverb, 
since it follows the verb and acts as a modifier to it, not as a complement.

Nouns may also be unmarked if they act as modifiers in a compound and the head 
is marked for the NP's case and number, for instance:

\ex\begingl
	\gla ralanyeri mapang //
	\glb ralan-ye-ri mapang //
	\glc nail-\Pl{}-\Ins{} finger //
	\glft `with the fingernails' //
\endgl\xe

And lastly and probably most importantly, nouns appear superficially unmarked 
if topicalized, since the topic marker is \fw{-Ø}:

\ex\begingl
	\gla Saru-nama, ang nupoyya \textbf{veney} aruno vās. //
	\glb Sar-u=nama, ang nupa-oy-ya \textbf{veney-Ø} aruno vās //
	\glc go-\Imp{}=just, \AgtT{} hurt-\Neg{}-\TsgM{} \textbf{dog-\Top{}} 
 		brown \Ssg{}.\Parg{} //
	\glft `Just go, the brown dog won't hurt you.' //
\endgl\xe

\index{cases|)}

\subsection{Prefixes on nouns}
\index{prefixes!on nouns|(}

All of the nominal morphology we have so far dealt with in this section was 
suffixing. As mentioned in the previous section already 
(p.~\pageref{nounprefixes}), however, there are also a number of prefixes that 
can be applied to nouns. I have just given two examples of the prefix 
\xayr{ku/}{ku-}{like, as though} above, but \rayr{ku/}{ku-} applies not only to 
nouns themselves. In fact, it rather attaches to whole NPs, which makes it a 
clitic\index{clitics} according to \citet[117]{klavans1985}, and a special 
clitic in \citeauthor{zwicky1977}'s terminology, since no corresponding full 
form exists in its place, comparable to the English possessive clitic \fw{'s}, 
for instance \parencites[3, 
13]{zwicky1977}[295]{zwicky1985}[510]{zwickypullum1983}. To cite from the Ayeri 
translation of Kafka's short story \enquote{Eine kaiserliche Botschaft} again:

\ex\label{ex:kukafka2}\begingl
	\gla … saylingyāng kovaro naynay, ku-ranyāng palung. //
	\glb … sayling=yāng kovaro naynay, ku=ranya-ang palung //
	\glc … progress=\TsgM{}.\Aarg{} easy also, like=nobody-\Aarg{} else //
	\glft `… he also got on easily, like nobody else.'%
	\tc{\citep[12]{becker:kafka:imperial}}//
\endgl\xe

In this example, we can see \rayr{ku/}{ku-} attaching to a properly inflected 
NP adjunct. The NP is case-marked for agent since it can be understood to refer 
to the verb \xayr{sjliNF/}{sayling-}{progress} in the main clause, insofar 
\xayr{rnYaaNF pluNF}{ranyāng palung}{nobody else} can replace 
\xayr{/yaaNF}{-yāng}{he} in the main clause.

Besides \rayr{ku/}{ku-}, there are also the demonstrative prefixes 
\xayr{d/}{da-}{such}, \xayr{Ed/}{eda-}{this}, and \xayr{Ad/}{ada-}{that}, which 
have already been mentioned in the previous section as well. The demonstrative 
prefixes undergo coalescence with nouns beginning with \fw{a-}, that is, they 
form phonological words with their hosts for all means and purposes. The 
demonstrative prefixes are special clitics as well, since no contemporary free 
form exists.

\pex
\a\begingl
	\gla da-nanga kāryo //
	\glb da=nanga kāryo //
	\glc such=house big //
	\glft `such a big house' //
\endgl

\a\begingl
	\gla edāyon nake //
	\glb eda=ayon nake //
	\glc this=man tall //
	\glft `this tall man' //
\endgl

\a\begingl
	\gla ada-envan alingo //
	\glb ada=envan alingo //
	\glc that=woman clever //
	\glft `that clever woman' //
\endgl

\xe

Moreover, there is a prefix \rayr{me/}{mə-} in complementary distribution with 
the demonstrative prefixes, which adds a meaning along the lines of `just any', 
`whatsoever', `some' to the noun. Note that this prefix is distinct from the 
morpheme indicating an inspecific quantity, \xayr{/ArilF}{-aril}{some}.

\pex
\a\begingl
	\gla Ang lampyo mə-veney kayvo kirinya. //
	\glb Ang lamp-yo mə=veney-Ø kayvo kirin-ya //
	\glc \AgtT{} walk-\TsgN{} some=dog-\Top{} along street-\Loc{} //
	\glft `Some dog is walking along the street.' //
\endgl

\a\begingl
	\gla Ang noyan mēntānley pegamayayam. //
	\glb Ang no=yan mə=entān-ley pegamaya-yam //
	\glc \AgtT{} want=\TsgM{}.\Top{} some=punishment-\PargI{} 
		thief-\Dat{} //
	\glft `They demanded some kind of punishment for the thief.' //
\endgl

\xe

\index{prefixes!on nouns|)}

\subsection{Compounding}
\index{compounds|(}

With regards to the classification of compounds, \citet{bauer2001} gives some 
helpful typological guidelines. Besides the compound types recognized by 
Sanskrit grammarians---endocentric (\fw{tatpuruṣa}), coordinative 
(\fw{dvandva}), adjectival-endo\-cent\-ric (\fw{karmadhāraya}), and 
exocentric (\fw{bahuvrīhi})---he also adds synthetic compounds, which Sanskrit 
did not have \citep[697]{bauer2001}. Overall, he finds that determinative, or 
endocentric, compounds are the most common ones in the languages of the world 
\citep[697]{bauer2001}, especially if the head refers to a location or source 
of sorts \citep[702]{bauer2001}.

\citet{gaeta2008}, then, adds to \citeauthor{bauer2001}'s research, based on a 
larger sample of grammars surveyed, that compounds for the largest part 
correlate with the constituent order of the language, both regarding the order 
of verb and object and that of noun and genitive \citep[129--133]{gaeta2008}. 
Mismatches in headedness occur, but appear to constitute the minority of cases 
and may often be explained through historical changes in syntax; he discerns  
for one that \textcquote[135]{gaeta2008}{morphology is not autonomous from 
syntax}, and that secondly, \textcquote[135]{gaeta2008}{[s]yntax seems to be 
the motor of change, which may be then reflected in compounds}, and that 
thirdly, lexical conservativism causes atavisms to linger on, reflecting the 
syntax of earlier stages of the language \citep[138--139]{gaeta2008}.

\index{typology!of compounds}
For the purpose of gaining at least a little insight into which types of 
compounds Ayeri allows---besides endocentric compounds---I conducted a small 
(non-exhaustive) survey based on 130 compounds from the Ayeri dictionary 
\citep[Dictionary]{benung}; \autoref{tab:comptyp} shows the various compound 
classes and the number of words for each. `Harmonic' and `disharmonic', 
respectively, refer to the order of elements; the order is `harmonic' 
if it is following the normal constituent order of the language and 
`disharmonic' if it is at odds with it \citep{gaeta2008}.

\begin{table}[ht]
\caption[Compounds in the Ayeri dictionary]{Compounds in the Ayeri dictionary 
\citep{benung} and their classification (n\,=\,130)}
\begin{tabu} to \linewidth {X[3.5l] X[c] X[c] X[c] X[c] X[c] X[c]}
\toprule\tableheaderfont
Type
	& \multicolumn2{c}{Harmonic}
	& \multicolumn2{c}{Disharmonic}
	& \multicolumn2{c}{Total}
	\\
\toprule

Endocentric (N\,+\,N)
	& 67
	& 51.54\pct
	& 2
	& 1.54\pct
	& 69
	& 53.08\pct
	\\
	
Endocentric (N\,+\,Adj)
	& 18
	& 13.85\pct
	& 4
	& 3.08\pct
	& 22
	& 16.92\pct
	\\

Synthetic (V\,+\,N)
	& 16
	& 12.31\pct
	& 4
	& 3.08\pct
	& 20
	& 15.38\pct
	\\

Coordinative (N\,+\,N)
	& 9
	& 6.92\pct
	& \multicolumn2{c}{---}
	& 9
	& 6.92\pct
	\\
	
Exocentric (N\,+\,N)
	& 1
	& 0.77\pct
	& 3
	& 2.31\pct
	& 4
	& 3.08\pct
	\\
	
\midrule

Unclear
	& 6
	& 4.62\pct
	& \multicolumn2{c}{---}
	& 6
	& 4.62\pct
	\\
	
\midrule

Total
	& 117
	& 90.00\pct
	& 13
	& 10.00\pct
	& 130
	& 100\pct
	\\
	
\bottomrule
\end{tabu}
\label{tab:comptyp}
\end{table}

Unsurprisingly, the largest number of compound nouns in the sample were 
endocentric compounds of the regular kind, which means that, just like genitive 
attributes follow nouns, noun compounds are headed left. Especially compounds 
with adjectives are interesting insofar as this is also the normal order for 
free adjectives, so to illustrate, some tests will be necessary to show that 
these adjectives form a unit with the head noun and are unable to undergo 
comparison, for instance. Synthetic compounds exist in Ayeri and produce nouns. 
These are compounds in which \textcquote[701]{bauer2001}{the modifying element 
in the compound is (usually) interpreted as an argument of the verb from which 
the head is derived}. There are also a number of coordinative compounds; this 
group, however, is lexicalized and not productive. Exocentric compounds 
constitute the minority of the sample. In the following I will give examples for 
each type.

It needs to be noted that unlike Germanic languages, Ayeri does not allow 
compounds of arbitrary length to be strung together, like in the following 
ridiculous but no less real example from (former) German legislation 
\parencite[see, for instance,][]{sz:rindfleisch}:

\ex\begingl\rc{German}%
	\gla %
Rindfleisch­etikettierungs­überwachungs­aufgabenübertragungsgesetz 
//
	\glb Rind-fleisch-­etikettierung-s-­überwachung-s­-aufgabe-n%
		-übertragung-s-gesetz //
	\glc cow-meat-labeling-\Lnk{}-surveillance-\Lnk{}-duty-\Lnk{}%
		-delegation-\Lnk{}-law//
	\glft `Law on the delegation of duties in the surveillance of beef 
		labeling'//
\endgl\xe

In stark contrast, Ayeri allows only two elements in compounds. Furthermore, 
this section on compounds is located within the section on nouns because Ayeri 
almost only possesses compounds involving nouns, and the majority of these also 
results in a noun.

\subsubsection{Endocentric compounds}
\index{compounds!endocentric|(}

To start with the largest group, endocentric/\fw{tatpuruṣa} compounds, the bulk 
of these compounds combines two nouns, one of which is the head which is 
modified by a dependent noun. As Ayeri exhibits a rather strict head-first word 
order, it comes as no surprise, according to \citet{gaeta2008}, that most of 
these compounds follow this order strictly: the second noun modifies the first, 
which is opposite of how English, for instance, typically 
operates:

\pex\label{ex:endonoun}
	\a \makebox[12.2em][l]{\xayr{\larger betjniMpurF}{betaynimpur}{grape}}
		← \xayr{\larger betj}{betay}{berry}
		+ \xayr{\larger niMpurF}{nimpur}{wine}
	\a \makebox[12.2em][l]{\xayr{\larger krirynF}{karirayan}{vertigo}}
		← \xayr{\larger krF}{kar}{fear}
		+ \xayr{\larger IrynF}{irayan}{height}\footnotemark
	\a \makebox[12.2em][l]{\xayr{\larger pikunMdiNF}{pikunanding}{mustache}}
		← \xayr{\larger piku}{piku}{beard}
		+ \xayr{\larger nMdiNF}{nanding}{lips}
	\a \makebox[12.2em][l]{\xayr{\larger tpjperinF}{tapayperin}{sunblind}}
		← \xayr{\larger tpj}{tapay}{screen}
		+ \xayr{\larger perinF}{perin}{sun}
\xe

\footnotetext{\rayr{IrynF}{irayan}, however, is a transparent nominalization of 
\xayr{Irj}{iray}{high}.}

The example words in (\ref{ex:endonoun}) show that the relationships between 
the modifier and the head are various: a grape is a berry \emph{used} to 
make wine from \parencite[compare][702]{bauer2001}; vertigo is the fear 
\emph{of} height; a mustache is a beard \fw{located} over the lips 
\parencite[702]{bauer2001}; and a sunblind is a screen \fw{against} the 
sun.
% \footnote{Further examples include:
% \xayr{AvnMdirunF}{avanandirun}{square root}, lit. `base-square'; 
% \xayr{bidmihnye}{bidamihanaye}{xylophone}, lit. `block-wood-\Pl{}';
% \xayr{bgmFtupoj}{bagamtupoy}{dragon}, lit. `lizard-fire'; 
% \xayr{binMpdNF}{binampadang}{memory}, lit. `picture-mind'; 
% \xayr{burNu\_in}{buranguina}{elephant}, lit. `animal-nose'; 
% \xayr{dgmiMdoj}{dagamindoy}{menu}, lit. `choose-card'; 
% \xayr{dlMpsiNF}{dalampasing}{giraffe}, lit. `cow-neck'; 
% \xayr{drMdevo}{darandevo}{skull}, lit. `bone-head'; 
% \xayr{deveMthaanF}{deventahān}{alphabet}, lit. `system-writing'; 
% \xayr{glimehirF}{galimehir}{resin, tar}, lit. `juice-tree'; 
% \xayr{koybhisF}{koyabahis}{diary}, lit. `book-day'; 
% \xayr{ltuMkem}{latunkema}{tiger}, lit. `lion-stripe'; 
% \xayr{lonupt}{lonupata}{poultice}, lit. `bandage-mash'; 
% \xayr{mliMkronF}{malinkaron}{coast, seashore}, lit. `shore-sea'; 
% \xayr{mehirFgtNF}{mehirgatang}{ovaries}, lit. `tree-womb'; 
% \xayr{mehisiNj}{mehisingay}{conifer}, lit. `tree-needle'; 
% \xayr{mikYnFsitemF}{micansitem}{electric}, lit. `power-lightning'; 
% \xayr{mirMthnF}{mirantahan}{typeface}, lit. `kind-writing'; 
% \xayr{mirMthaanF}{mirantahān}{spelling}, lit. `way-writing'; 
% \xayr{mitFrmtau}{mitramatau}{pubic hair}, lit. `hair-tangle'; 
% \xayr{mitFrnvsNF}{mitranavasang}{axillary hair}, lit. `hair-sweat'; 
% \xayr{nrMbesuhej}{narambesuhey}{dictionary}, lit. `word-list'; 
% \xayr{niMpurivnF}{nimpurivan}{vinyard}, lit. `wine-mountain'; 
% \xayr{ptyelNF}{patayelang}{concrete}, lit. `mash-stone'; 
% \xayr{pikulkj}{pikulakay}{goatee}, lit. `beard-chin'; 
% \xayr{prihiNumo}{prihingumo}{desk}, lit. `table-work'; 
% \xayr{rgMterFpeNF}{raganterpeng}{diameter}, lit. `line-middle'; 
% \xayr{rlmpNF}{ralamapang}{fingernail}, lit. `nail-finger'; 
% \xayr{ridspj}{ridasapay}{glove}, lit. `sock-hand'; 
% \xayr{sNumospoj}{sangumosapoy}{ticket office}, lit. `office-ticket'; 
% \xayr{svtkNF}{savatakang}{tank}, lit. `cart-armor'; 
% \xayr{sayMprl}{sayamparal}{urine hole}, lit. `hole-penis'; 
% \xayr{syNu\_in}{sayanguina}{nostril}, lit. `hole-nose'; 
% \xayr{syniv}{sayaniva}{eye socket}, lit. `hole-eye'; 
% \xayr{selNblN}{selangbalang}{search engine}, lit. `machine-search'; 
% \xayr{selMkurnF}{selangkuran}{computer}, lit. `machine-counting'; 
% \xayr{sepFrkronF}{seprakaron}{ditch}, lit. `cleft-water'; 
% \xayr{similitnF}{similitan}{borderland}, lit. `land-margin-\Nmlz{}'; 
% \xayr{similitj}{similitay}{republic}, lit. `land-democracy'; 
% \xayr{sirjyil}{sirayyila}{knee}, lit. `joint-foot'; 
% \xayr{sirjtinu}{siraytinu}{elbow}, lit. `joint-arm'; 
% \xayr{sirukronF}{sirukaron}{starfish}, lit. `star-water'; 
% \xayr{sirusitFrmF}{sirusitram}{comet}, lit. `star-tail'; 
% \xayr{sirutj}{sirutay}{night}, lit. `star-time'; 
% \xayr{sitNlugaanF}{sitanglugān}{incest}, lit. `self-entry'; 
% \xayr{trFtrihimF}{tartarihim}{tobacco}, lit. `pipe-weed'; 
% \xayr{tepilFpihaanF}{tepilpihān}{fester}, lit. `sore-pus'; 
% \xayr{tFreMdpNisF}{trendapangis}{bank}, lit. `hall-money'; 
% \xayr{tuptinu}{tupatinu}{fathom}, lit. `length-arm'; 
% \xayr{veb\_osnF}{vebaosan}{slug}, lit. `snail-slime'; 
% \xayr{veMkubesonF}{venkubeson}{navy}, lit. `army-ship'; 
% \xayr{vinimyonF}{vinimayon}{monkey}, lit. `forest-man'; 
% \xayr{yno\_avnF}{yanoavan}{area, region}, lit. `place-ground'; 
% \xayr{yelNFssaanF}{yelangsasān}{cobblestone}, lit. `stone-way'; 
% \xayr{yenukrFdNF}{yenukardang}{classmates}, lit. `group-school'; 
% \xayr{yutnjkonF}{yutanaykon}{foreskin}, lit. `skin-cover'.
% }
\citet{bauer2001} mentions that \textquote{there may be special 
morphophonemic processes which apply between the elements of compounds}, such 
as \textcquote[695]{bauer2001}{phonological merger[s] between the elements of 
the compound}. This also occasionally happens in Ayeri, as the next few example 
words show:

\pex\label{ex:endonounmod}
	\a \xayr{\larger AvrrnF}{avararan}{wetland} \\
		← \xayr{\larger AvnF}{avan}{ground}
		+ \xayr{\larger rro}{raro}{wet}
		+ \rayr{\larger /AnF}{-an} (\Nmlz{})
	\a \xayr{\larger mehimitFrNF}{mehimitrang}{fiber tree} \\
		← \xayr{\larger mehirF}{mehir}{tree}
		+ \xayr{\larger mitFrNF}{mitrang}{hair, fiber}
	\a \xayr{\larger niNMpinmF}{ningampinam}{bedtime story} \\
		← \xayr{\larger niNnF}{ningan}{story}
		+ \xayr{\larger pinmF}{pinam}{bed}
	\a \xayr{\larger pdilmikYnF}{padilamican}{gravitational force} \\
		← \xayr{\larger pdilnF}{padilan}{attraction}
		+ \xayr{\larger mikYnF}{mican}{force, power}
\xe

There is a modicum of alteration happening in all of the heads of the example 
words in (\ref{ex:endonounmod}), mostly nasals assimilating to the stop or 
nasal which the modifier begins with (/n/~+~/p/~→~/mp/, /n/~+~/m/~→~/m/), 
though \rayr{AvrrnF}{avararan} and \rayr{mehimitFrNF}{mehimitrang} even delete 
whole coda segments.
% Stuff may even be mashed together completely, but examples??
\citet[703]{bauer2001} notes that very commonly, genitive and plural markers 
may form linking elements, though he also gives examples of languages which 
allow other case markers on the modifying element in languages with head-right 
order; individual languages may allow even more case inflection. However, this 
appears not to happen in Ayeri. The only element that comes up time and again 
in between the two halves of compounds is the nominalizer \rayr{/AnF}{-an}, 
which signifies that the head is being formed by a nominalized root, such as in 
\rayr{pdilmikYnF}{padilamican}, where \xayr{pdilnF}{padilan}{attraction} is a 
nominalization of \xayr{pdilF/}{padil-}{attract}, or in 
\rayr{niNMpinmF}{ningampinam}, where \xayr{niNnF}{ningan}{story} is derived 
from the verb \xayr{niNF/}{ning-}{tell}. However, since Ayeri is head-first and 
possessive phrases are dependent marking, genitive or other case marking would 
be expected on the second element, not the first. Case marking on a compound, 
however, does not inflect just the modifier, but the whole NP:

\ex\begingl
	\gla Ang ningya sipikanena koyabahisena. //
	\glb Ang ning-ya sipik-an-ena koyabahis-ena //
	\glc \AgtT{} talk-\TsgM{}.\Top{} keep-\Nmlz{}-\Gen{} book.day-\Gen{} //
	\glft `He talks about keeping a journal.' //
\endgl\xe

\rayr{koybhisen}{koyabahisena} in this example is not to be interpreted as 
`book of day(s)' but as `of a day-book'. Inflection between the parts of a 
compound can happen nonetheless, though. In compounds which are formed ad 
hoc or which are otherwise transparent in their composition, inflection often 
is deferred to the noun head noun instead of the edge of the compound as a 
whole; the modifier is treated as an adjunct in this case, and stays 
uninflected:

\ex\label{ex:nouncompdiv}\begingl
	\gla Sa trayeng tipin ralanyeri mapang yena. //
	\glb Sa tra=yeng tipin-Ø ralan-ye-ri mapang yena //
	\glc \PatT{} scratch=\TsgF{}.\Aarg{} itch-\Top{} nail-\Pl{}-\Ins{} 
		finger \TsgF{}.\Gen{} //
	\glft `The itch, she scratches it with her fingernails.' //
\endgl\xe

Besides noun modifiers, there are also compounds where the modifier is an 
adjective. In classical Sanskrit terminology, this type is called 
\fw{karmadhāraya} \citep[698--699]{bauer2001}.\footnote{\citeauthor{bauer2001} 
also mentions that appositional compounds like \fw{maid-servant}, \fw{woman 
doctor} and \fw{fighter-bomber} are counted in this category 
\citep[699]{bauer2001}. Ayeri, however, does not possess such formations in 
particular.} Examples in Ayeri include:%
% \footnote{Further examples include: 
% \xayr{bhisino}{bahisino}{holiday, day off}, lit. `day-free'; 
% \xayr{dikuMtrinF}{dikuntaring}{bureaucracy}, lit. `passion-administrative'; 
% \xayr{leMtMkusNF}{lentankusang}{diphthong}, lit. `sound-double'; 
% \xayr{nNbnY}{nangabanya}{hospital}, lit. `house-sick'; 
% \xayr{naraaMtiynF}{narāntiyan}{conlang}, lit. `language-created-\Nmlz{}'; 
% \xayr{rohMpraanF}{rohamparān}{snack}, lit. `bite-quick-\Nmlz{}'; 
% \xayr{rohMkivo}{rohankivo}{snack}, lit. `bite-small'; 
% \xayr{sNumirj}{sangumiray}{ministry, authority}, lit. `office-high'; 
% \xayr{tbMpehu}{tabampehu}{lower jaw}, lit. `jaw-loose'
% \xayr{tabnikp}{tabanikapa}{upper jaw}, lit. `jaw-attached'.
% }

\pex
	\a \makebox[12.5em][l]{\xayr{\larger krFdNirj}{kardangiray}{university}}
		← \xayr{\larger krFdNF}{kardang}{school}
		+ \xayr{\larger Irj}{iray}{high}
	\a \makebox[12.5em][l]{\xayr{\larger mrsFhri}{marashari}{witticism}}
		← \xayr{\larger mrsF}{maras}{phrase}
		+ \xayr{\larger hti}{hati}{pithy}
	\a \makebox[12.5em][l]{\xayr{\larger silFvniknF}{silvanikan}{overview}}
		← \xayr{\larger silFvnF}{silvan}{view}
		+ \xayr{\larger IknF}{ikan}{whole}
	\a \makebox[12.5em][l]{\xayr{\larger vipimkaarY}{vipimakārya}{crow}}
		← \xayr{\larger vipinF}{vipin}{bird}
		+ \xayr{\larger mkaarY}{makārya}{black}
\xe

In all of these cases, the adjective forms a unified lexeme with the head noun, 
hence it is not comparable, for example:

\pex
\a\begingl
	\gla *kardangiray-eng \quad{} kardangiray-vā //
	\glb kardang-iray=eng \quad{} kardang-iray=vā //
	\glc school-high=\Comp{} \quad{} school-high=\Supl{} //
	\glft `*higher-school' {} `*highest-school' //
\endgl

\a\begingl
	\gla *marashati-eng \quad{} *marashati-vā //
	\glb maras-hati=eng \quad{} maras-hati=vā //
	\glc phrase-pithy=\Comp{} \quad{} phrase-pithy=\Supl{} //
	\glft `*pithier-phrase' {} `*pithiest-phrase' //
\endgl

\xe

In fact, it is possible to form \rayr{krFdNirj vaa}{kardangiray vā} and 
\rayr{mrsFhti vaa}{marashati vā}, but they mean `most universities' and `most 
witticisms', respectively; \xayr{/ENF}{-eng}{rather} as a quantifier does not 
combine with nouns. Since the meaning composed from noun--adjective compounds is 
often idiomatic, they also cannot be divided as shown above in 
(\ref{ex:nouncompdiv}), since a \xayr{krFdNirj}{kardangiray}{university} is not 
a \xayr{krFdNF}{kardang}{school} which is \xayr{Irj}{iray}{high} in the literal 
sense, but a school of the highest tier. \rayr{krFdNen Irj}{kardangena iray} 
(school-\Gen{} high), then, can only be interpreted in the literal sense, `of 
the high school', but not as `of the university', which thus can only be 
\rayr{krFdNiryen}{kardangirayena}.

In the sample, there were also a few compounds I categorized as noun--noun 
combinations, which look as though they violate head-first order. All of these 
involve \xayr{sitNF}{sitang}{self} as a modifier:

\pex\label{ex:compsitang}
	\a \makebox[16em][l]{\xayr{\larger sitNFleMtnF}{sitanglentan}{vowel}}
		← \xayr{\larger sitNF}{sitang}{self}
		+ \xayr{\larger leMtnF}{lentan}{sound}
	\a \makebox[16em][l]{\xayr{\larger sitNFpronaanF}{sitangparonān}%
		{self-confidence}}
		← \xayr{\larger sitNF}{sitang}{self}
		+ \xayr{\larger pronaanF}{paronān}{faith}
	\a \makebox[16em][l]{\xayr{\larger sitNFtenYnF}{sitangtenyan}%
		{suicide}}
		← \xayr{\larger sitNF}{sitang}{self}
		+ \xayr{\larger tenYnF}{tenyan}{death}
\xe

\rayr{sitNF}{sitang} does not exist as a noun by itself in Ayeri, the word for 
`self' is its nominalization \rayr{sitNnF}{sitangan}. Nonetheless, it looks 
like it could have plausibly been a noun once. However, this noun 
may have been grammaticalized into a reflexive morpheme of a more 
general kind, which in turn birthed the form \rayr{sitNnF}{sitangan} as a 
renovation.\footnote{A little bit of language history would certainly simplify 
things here and lend them credence. Let us simply assume that 
\rayr{sitNF}{sitang} used to be a noun meaning something like `self' at a 
previous stage of Ayeri and was repurposed as a reflexive prefix. 
\citet{lehmann2015} quotes a few examples of what he calls `autophoric' nouns 
that came to be used as reflexive pronouns in their respective language: 
\textcquote[45--46]{lehmann2015}{Typical examples are Sanskrit \fw{tan} 
`body, person' and \fw{ātmán} `breath, soul', Buginese \fw{elena} `body', 
Okinawan \fw{dūna} `body', !Xu \fw{l’esi} `body', Basque \fw{burua} `head', 
Abkhaz \fw{a-xə̀̀} `the head'. In their respective languages, all these nouns 
are translation equivalents of English \fw{self}}. Thus, it would not be out of 
line at all to assume such a grammaticalization path for Ayeri as well.} The 
reflexive \rayr{sitNF}{sitang} is used---as we have seen in the previous 
chapter---as a prefix, so there are two ways to intepret these formations: 
first, \rayr{sitNF}{sitang} may be the reflexive prefix here and thus the 
compound follows the normal syntactic order; or second, the order of elements 
is reversed and thus may reflect an earlier stage of Ayeri where 
\rayr{sitNF}{sitang} was still a noun and modifiers could still appear in front 
of their heads, at least optionally so \citep[133--137]{gaeta2008}.

There are a number of genuinely reversed endocentric compounds as well, 
however, in which the modifier comes first and the head last. Since there are 
only a few of these, I will give all of them in the following example:

\pex
	\a \makebox[15em][l]{\xayr{\larger bript}{baripata}{ground meat}}
		← \xayr{\larger bri}{bari}{meat}
		+ \xayr{\larger pt}{pata}{mash}
	\a \makebox[15em][l]{\xayr{\larger kjvoleMtnF}{kayvolentan}{consonant}}
		← \xayr{\larger kjvo}{kayvo}{with}
		+ \xayr{\larger leMtnF}{lentan}{sound}
	\a \makebox[15em][l]{\xayr{\larger maavgneNF}{māvaganeng}{mother's 
		siblings}}
		← \xayr{\larger maav}{māva}{mother}
		+ \xayr{\larger gneNF}{ganeng}{siblings}
	\a \makebox[15em][l]{\xayr{\larger mtinMdiNF}{matinanding}{labia}}
		← \xayr{\larger mtiknF}{matikan}{hot}
		+ \xayr{\larger nMdiNF}{nanding}{lips}
	\a \makebox[15em][l]{\xayr{\larger muyvirNF}{muyavirang}{brass}}
		← \xayr{\larger muy}{muya}{false}
		+ \xayr{\larger AvirNF}{avirang}{gold}
	\a \makebox[15em][l]{\xayr{\larger 
		tonisjtNF}{tonisaytang}{self-assured}}
		← \xayr{\larger tonis}{tonisa}{assured}
		+ \ques{}\,\xayr{\larger sitNnF}{sitangan}{self}
\xe

Given the discussion of \rayr{sitNF}{sitang} above, one word among the examples 
above that is not too clear is \rayr{tonisjtNF}{tonisaytang}, which appears to 
contain a deviant form of either \rayr{sitNF}{sitang} or 
\rayr{sitNnF}{sitangan}, which is preceded by the adjective 
\xayr{tonis}{tonisa}{assured, ascertained}.

All of the previously mentioned compounds involving nominal elements formed 
nouns, though, there are also a few denominal compounds in the sample. This 
process is not productive, however, and interestingly, only noun–adjective 
combinations appear in this group:

\pex
	\a \xayr{\larger mirMpluj}{mirampaluy}{otherwise} \\
		← \xayr{\larger mirnF}{miran}{way}
		+ \ques{}\,\xayr{\larger pluNF}{palung}{different}
	\a \xayr{\larger pdbnY}{padabanya}{insane} \\
		← \xayr{\larger pdNF}{padang}{mind}
		+ \xayr{\larger bny}{banaya}{sick}
	\a \xayr{\larger teMkris/}{tenkarisa-}{be frightened 
		to death} \\
		← \xayr{\larger tenF}{ten}{life}
		+ \xayr{\larger kris}{karisa}{frightened}
\xe

\rayr{mirMpluj}{mirampaluy} is an adverb, the modifier probably a mangling of 
\rayr{pluNF}{palung}; \rayr{pdbnY}{padabanya} is an adjective meaning `insane' 
rather than the expected `insanity' (instead: \rayr{pdbnYaanF}{padabanyān}); 
and \rayr{teMkris/}{tenkarisa-} acts as a verb, possibly from conversion or 
reinterpretation, since the suffix \rayr{/Is}{-isa} also forms morphological 
causatives of a number of verbs. Besides these irregularities, there is also at 
least one noun compound which uses a postposition as an adjectival modifier:

\ex
	\xayr{\larger silFvMkjvj}{silvankayvay}{blindness} 
	← \xayr{\larger silFvnF}{silvan}{sight}
	+ \xayr{\larger kjvj}{kayvay}{without}
\xe

This compound must be derived from the phrase \xayr{silFvnFy kjvj}{silvanya 
kayvay}{without sight} (see-\Nmlz{}-\Loc{} without), though here as well, the 
word roots are simply juxtaposed, as noted above is the common way to form 
compounds in Ayeri.

\index{compounds!endocentric|)}

\subsubsection{Synthetic compounds}
\index{compounds!synthetic|(}

According to \citet{bauer2001}, (semi-)synthetic compounds, or verbal(-nexus) 
compounds, are compounds that have \textcquote[701]{bauer2001}{been variously 
defined as being based on word-groups or syntactic constructions 
\citep[2]{botha1984}, or as compounds whose head elements are derived from 
verbs \citep[3607]{lieber1994}}. Examples of this type in English would include 
\fw{truck-driver}, \fw{peace-keeping}, and \fw{home-made}. He mentions also 
that synthetic compounds have been mainly discussed with regards to Germanic 
languages, but that according to \citet[3608]{lieber1994}, the phenomenon is 
much more widespread. Ayeri possesses compounds like this as well, and the 
regular case again follows the constituent order, here that of verbs and nouns: 
Ayeri is a VO language, and thus the verb as the head of the compound is usually 
found on 
the left side with its nominal modifier following it 
\citep[129--133]{gaeta2008}:

\pex
	\a \makebox[14em][l]{\xayr{\larger AnFlaagonnF}{anlāgonan}%
		{pronunciation}}
		← \xayr{\larger AnFlF/}{anl-}{bring}
		+ \xayr{\larger AgonnF}{agonan}{outside}
	\a \makebox[14em][l]{\xayr{\larger npkronF}{napakaron}{acid}}
		← \xayr{\larger npF/}{nap-}{burn}
		+ \xayr{\larger kronF}{karon}{water}
	\a \makebox[14em][l]{\xayr{\larger npperinF}{napaperin}{sunburn}}
		← \xayr{\larger npF/}{nap-}{burn}
		+ \xayr{\larger perinF}{perin}{sun}
	\a \makebox[14em][l]{\xayr{\larger telFbssaanF}{telbasasān}{waysign}}
		← \xayr{\larger telFb/}{telba-}{show}
		+ \xayr{\larger ssaanF}{sasān}{way}
\xe

The relations between the verb and the noun are various again, that is, the 
nominal modifier is not simply the direct object of the verb: to pronounce 
something means to bring it \emph{to} the outside; a sunburn is a burn 
\fw{caused by} the sun; and a waysign shows the way (\rayr{ssaanF}{sasān} is 
the object here). Even though \rayr{kronF}{karon} may serve as an agent (or a 
causer) of the burning effect of acid (similarly for 
\xayr{npperinF}{napaperin}{sunburn}), the verb-first order is justified here as 
well, since verbs always go first in Ayeri sentences, and any other NPs, 
whether actor or undergoer, are following.%
% \footnote{Further examples include:
% \xayr{bimkNnF}{bimakangan}{photo}, lit. `paint-light-\Nmlz{}'; 
% \xayr{IlgonnF}{ilagonan}{edition}, lit. `give-out-\Nmlz{}'; 
% \xayr{lMtmidj}{lantamiday}{diversion}, lit. `lead-around'; 
% \xayr{nbisFmaavy}{nabismāvaya}{motherfucker}, lit. `fuck-mother-\Agtz{}'; 
% \xayr{nrkhu}{narakahu}{phone}, lit. `speak-far'; 
% \xayr{srsjliNF}{sarasayling}{progress}, lit. `go-further'; 
% \xayr{silFvkhu}{silvakahu}{TV}, lit. `see-far'; 
% \xayr{silFvmrinnF}{silvamarinan}{preview}, lit. `see-before-\Nmlz{}'; 
% \xayr{telFbgonnF}{telbagonan}{advertisement}, lit. `show-out'; 
% \xayr{vliktu}{valikatu}{masochist}, lit. `enjoys pain'.
% }

Just as with endocentric compounds, there are a number of seeming exceptions to 
the verb-first order of synthetic compounds as well. These are just as far and 
few between, however, and whether they should all be counted as noun–verb 
combinations is also questionable as they appear to all be formed with 
nominalized verbs. The verbal element may thus be only indirectly verbal for the 
purposes of compounding. If interpreted as noun--noun combinations, the nominal 
first element would reasonably form the head again for some of the below example 
words.

\pex\label{ex:compvbrev}
	\a \xayr{\larger mripuMtymF}{maripuntayam}{spread} \\
		← \xayr{\larger mrinF}{marin}{surface}
		+ \xayr{\larger puMt/}{punta-}{stroke}
		+ \rayr{\larger /ymF}{-yam} (\Dat{})
	\a \xayr{\larger ssnFlekaanF}{sasanlekān}{labyrinth} \\
		← \xayr{\larger ssaanF}{sasān}{way}
		+ \xayr{\larger lek/}{leka-}{guess}
		+ \rayr{\larger /AnF}{-an} (\Nmlz{})
	\a \xayr{\larger selNnunaan}{selangnunān}{plane} \\
		← \xayr{\larger selNF}{selang}{machine}
		+ \xayr{\larger nun/}{nuna-}{fly}
		+ \rayr{\larger /AnF}{-an} (\Nmlz{})
	\a \xayr{\larger siMturaanF}{sinturān}{radio} \\
		← \xayr{\larger siMto}{sinto}{wave}
		+ \xayr{\larger tur/}{tura-}{send}
		+ \rayr{\larger /AnF}{-an} (\Nmlz{})
\xe

\rayr{mripuMtymF}{maripuntayam} is special in that it contains the dative 
suffix \rayr{/ymF}{-yam} which is lexicalized as part of the word: something 
made or intended for spreading on a surface. A few more such verbal derivations 
can be found, though not compounds, among others:

\pex
	\a \makebox[10.5em][l]{\xayr{\larger gFrenYmF}{grenyam}{extremity}}
		← \xayr{\larger gFren/}{gren-}{reach out}
	\a \makebox[10.5em][l]{\xayr{\larger lugymF}{lugayam}{password}}
		← \xayr{\larger lug/}{luga-}{go through} 
	\a \makebox[10.5em][l]{\xayr{\larger shymF}{sahayam}{future}}
		← \xayr{\larger sh/}{saha-}{come}
\xe

There is also \xayr{mripuMt/}{maripunta-}{spread over} as the verb corresonding 
to \rayr{mripuMtymF}{maripuntayam}, though its meaning is less specific. Here 
as well, however, the verbal part is last instead of first. For the other 
example words (\ref{ex:compvbrev}b--d), an interpretation of the second part as 
a deverbal noun is possible: a labyrinth as a way or path which requires 
guessing, a plane a machine for flight, and radio as a sending of waves. In the 
latter case, \rayr{siMturaanF}{sinturān}, however, the head is still on the 
wrong side even if one interprets all of the above examples as noun--noun 
compounds with a deverbal element.

\index{compounds!synthetic|)}

\subsubsection{Coordinative compounds}
\index{compounds!coordinative|(}

Coordinative compounds are a very small group among the sample drawn from the 
dictionary, and not a very productive one. \citet{bauer2001} defines this class 
as having \textcquote[699]{bauer2001}{two or more words in a coordinate 
relationship, such that the entity denoted is the totality of the entities 
denoted by each of the elements}. He cautions that they are very easy to 
confuse with appositional (also \fw{karmadhāraya}) compounds in that both types 
of compound allow inserting an \fw{and} between both elements. The following 
nominal coordinative compounds are included in the dictionary sample:

\pex
	\a \makebox[13em][l]{\xayr{\larger baaːm}{bāmā}{mom-and-dad}}
		← \xayr{\larger baa(baa)}{bā(bā)}{dad}
		+ \xayr{\larger maa(maa)}{mā(mā)}{mom}
	\a \makebox[13em][l]{\xayr{\larger pFrujnpj}{pruynapay}{seasoning}}
		← \xayr{\larger pruj}{pruy}{salt}
		+ \xayr{\larger npj}{napay}{pepper}
	\a \makebox[13em][l]{\xayr{\larger spjyil}{sapayyila}{hands-and-feet}}
		← \xayr{\larger spj}{sapay}{hand}
		+ \xayr{\larger yil}{yila}{foot}
	\a \makebox[13em][l]{\xayr{\larger simileno}{simileno}{horizon}}
		← \xayr{\larger similF}{simil}{country}
		+ \xayr{\larger leno}{leno}{sky}
	\a \makebox[13em][l]{\xayr{\larger sitemFrugonF}{sitemrugon}%
		{thunderstorm}}
		← \xayr{\larger sitemF}{sitem}{lightning}
		+ \xayr{\larger rugonF}{rugon}{thunder}
	\a \makebox[13em][l]{\xayr{\larger vekmFdekej}{vekamdekey}{dishes}}
		← \xayr{\larger vekmF}{vekam}{plate}
		+ \xayr{\larger dekej}{dekey}{fork}
\xe

None of the two elements recognizably forms the head in these examples, but 
both elements are typical exponents of the thing the compound signifies. 
\citet[699]{bauer2001} mentions that coordinative adjective compounds are rare, 
or at least rarely documented in the grammars he surveyed. In the sample I 
took, only the following compound is included, which forms a noun from the 
combination of two adjectives, insofar it is relevant to this section even 
though the component parts are not nouns:

\ex
	\xayr{\larger mkgisu}{makagisu}{twilight}
		← \xayr{\larger mk}{maka}{light}
		+ \xayr{\larger gisu}{gisu}{dark}
\xe

The sample also includes the following two words, however, which are neither 
made up from nouns, nor do they form a noun in combination. Instead, they are 
technically verbs combining to form directional adverbs and have been 
exceptionally included here for completeness:

\pex
	\a \makebox[11em][l]{\xayr{\larger mNsh}{mangasaha}{towards}}
		← \xayr{\larger mN/}{manga-}{move}
		+ \xayr{\larger sh/}{saha-}{come}
	\a \makebox[11em][l]{\xayr{\larger mNsr}{mangasara}{away}}
		← \xayr{\larger mN/}{manga-}{move}
		+ \xayr{\larger sr}{sara-}{go}
\xe

\index{compounds!coordinative|)}

\subsubsection{Exocentric compounds}
\index{compounds!exocentric|(}

In exocentric compounds, the modifier is not a hyponym of its head 
\citep[700]{bauer2001}, which means that the modifier is not 
describing a property that more closely determines its head. So while a \fw{dog 
kennel} is a type of kennel made for dogs, the head of an \fw{egghead} is 
neither for eggs, nor containing eggs, nor made of eggs; instead, it refers to 
an egg-shaped skull metaphorically. And while a \fw{bluecollar} may wear a blue 
shirt professionally, the referent it signifies is not a type of collar, but 
the relationship is metonymical in that the blue collar is part of the 
guise of the signified entity as a whole. The sample from the Ayeri dictionary 
contains a few compounds of this kind as well, though again, it is 
not a very productive group:

\pex
	\a \makebox[11em][l]{\xayr{\larger AvnFyonNF}{avanyonang}{artery}}
		← \xayr{\larger AvnF}{avan}{bottom, down}
		+ \xayr{\larger yonNF}{yonang}{stream}
	\a \makebox[11em][l]{\xayr{\larger bjtMdevo}{baytandevo}{headache}}
		← \xayr{\larger bjtNF}{baytang}{blood}
		+ \xayr{\larger devo}{devo}{head}
	\a \makebox[11em][l]{\xayr{\larger linFyonNF}{linyonang}{vein}}
		← \xayr{\larger liNF}{ling}{top, up}
		+ \xayr{\larger yonNF}{yonang}{steam}
	\a \makebox[11em][l]{\xayr{\larger siMdjnN}{sindaynanga}{address}}
		← \xayr{\larger sindj}{sinday}{number}
		+ \xayr{\larger nN}{nanga}{house}
\xe

What is striking here is that only one out of for examples shows the expected 
head-left order: \rayr{siMdjnN}{sindaynanga}. The other three examples all have 
the head head component on the right side, preceded by a modifier. However, 
what all of these have in common, is that they are only metaphorically or 
metonymically describing the thing they signify: veins and arteries are not 
literally streams going up or down (they are a kind of stream flowing in 
different directions, however, so these are probably on the borderline between 
exocentric and endocentric); a headache is related to the head, but has not 
directly to do with being made of or containing blood (the rationale 
behind this being a superstition that you have too much blood in your head, 
which is said to cause the pain); and a house number may be part of an 
address, but is in a \fw{pars pro toto} relationship to it.

\index{compounds!exocentric|)}

\subsubsection{A few mysterious cases}

The following words from my sample were either undeterminable as to their 
composition due to parts of the word not being clear regarding one of their 
constituent parts, either because I tweaked the constituent so much as to not 
be readily recognizable anymore, or because I forgot to make an entry in the 
dictionary, or even deleted or changed that. The words in question are the 
following:

\pex
	\a \makebox[12em][l]{\xayr{\larger btNimnF}{batangiman}{mosquito}}
		← \xayr{\larger bjtNF}{baytang}{blood}
		+ ?
	\a \makebox[12em][l]{\xayr{\larger kirinlNF}{kirinalang}{avenue}}
		← \xayr{\larger kirinF}{kirin}{street}
		+ ?
	\a \makebox[12em][l]{\xayr{\larger niNMbkrF}{ningambakar}{telltale}}
		← \xayr{\larger niNnF}{ningan}{story}
		+ ?
	\a \makebox[12em][l]{\xayr{\larger rgyesuj}{ragayesuy}{grid}}
		← \xayr{\larger rgnF}{ragan}{line}
		+ ?
	\a \makebox[12em][l]{\xayr{\larger terjmino}{teraymino}{melancholic}}
		← ?
		+ \xayr{\larger mino}{mino}{happy}
	\a \makebox[12em][l]{\xayr{\larger vetjsno}{vetaysano}{fare}}
		← ?
		+ \rayr{\larger ssaanF}{sasān} (earlier \rayr{\larger 
			ssno}{sasano}) `way'
\xe

For all of the components represented by a question mark, there is no 
corresponding dictionary entry. At least in \rayr{bjtNimnF}{baytangiman}, the 
*\rayr{ImnF}{*iman} part looks as though it could be a noun due to the 
\rayr{/AnF}{-an} nominalizer suffix. *\rayr{terj}{*teray} in 
\rayr{terjmino}{teraymino} might also be an adjective supposed to mean `sad' 
(which would make it an adjectival coordinative compound), although the 
dictionary entry for that is \rayr{gidj}{giday}. Even though parts of all 
these words are unclear, they all seem to follow the correct syntactic order, 
judging by those parts that are identifiable. And even in the case of 
\rayr{vetjsno}{vetaysano}, which is missing the first part, it can be 
reasonably assumed that the identifiable part, *\rayr{sno}{*sano}, is the 
modifier, and *\rayr{vetj}{vetay} may have once been intended to mean `money' 
or `fee' or something along these lines.

With the exception of \rayr{niNMbkrF}{ningambakar}, all of the mystery words 
were entered into the dictionary in 2006. Digging through old archives and 
translations, I could determine at least that *\rayr{bkrF}{*bakar} was once 
intended to mean `lie', and *\rayr{terj}{*teray} was indeed meant to 
mean `sad'.

\index{compounds|)}

\subsection{Reduplication}
\index{reduplication|(}

\citet{wiltshiremarantz2000} write that it has been suggested that 
reduplication serves an iconic function, 
\textcquote[561]{wiltshiremarantz2000}{with the repetition of phonological 
material indicating a repetition or intensity in the semantics}, so with 
regards to nouns it mainly serves to indicate plurality of various kinds. 
However, they find that in fact, reduplication serves all kinds of functions, 
also ones without iconic meanings, and mention Agta, an Austronesian language of 
the Philippines, which uses reduplication to form diminutives 
\citep[6--9]{healey1960}. As we have seen in \autoref{subsec:reduplication} 
above, so does Ayeri, and it is justified in doing so since there is 
real-world evidence for this use of reduplication. Examples for diminutive 
reduplication in Ayeri include:

\pex
	\a \makebox[7em][l]{\xayr{\larger limu}{limu}{shirt}}
		→ \xayr{\larger limu/limu}{limu-limu}{little shirt}
	\a \makebox[7em][l]{\xayr{\larger nN}{nanga}{house}}
		→ \xayr{\larger nN/nN}{nanga-nanga}{little house}
	\a \makebox[7em][l]{\xayr{\larger spj}{sapay}{hand}}
		→ \xayr{\larger spj/spj}{sapay-sapay}{little hand}
	\a \makebox[7em][l]{\xayr{\larger venej}{veney}{dog}}
		→ \xayr{\larger venej/venej}{veney-veney}{little dog}
\xe

Diminutive reduplication involves full stem reduplication in Ayeri. 
Besides the productive use of reduplication for diminutive marking, there are 
a number of diminutive formations which have been lexicalized, such as in the 
following examples:

\pex
	\a \makebox[8.5em][l]{\xayr{\larger Agu}{agu}{chicken}}
		→ \xayr{\larger Agu/Agu}{agu-agu}{chick}
	\a \makebox[8.5em][l]{\xayr{\larger gnF}{gan}{child}}
		→ \xayr{\larger gnF/gnF}{gan-gan}{grandchild}
	\a \makebox[8.5em][l]{\xayr{\larger psiNF}{pasing}{tube}}
		→ \xayr{\larger psiNF/psiNF}{pasing-pasing}{straw}
	\a \makebox[8.5em][l]{\xayr{\larger poyu}{poyu}{cheek; bacon}}
		→ \xayr{\larger poyu/poyu}{poyu-poyu}{butt}
\xe

There are also at least two documented cases where the reduplicated root is not 
a noun, but the reduplication results in a noun:

\pex
	\a \makebox[10.5em][l]{\xayr{\larger kusNF}{kusang}{double (adj.)}}
		→ \xayr{\larger kusNF/kusNF}{kusang-kusang}{model}
	\a \makebox[10.5em][l]{\xayr{\larger veh/}{veh-}{build}}
		→ \xayr{\larger veh/veh}{veha-veha}{tinkering}
\xe

Reduplicated nouns behave like regular nouns with regards to inflection, that 
is, they receive prefixes and suffixes just like the simplexes from which they 
are derived:

\ex\begingl
	\gla Puco mino \textbf{veney-veneyang}. //
	\glb Puk-yo mino \textbf{veney\til{}veney-ang} //
	\glc jump-\TsgN{} happily \textbf{\Dim{}\til{}dog-\Aarg{}} //
	\glft `The little dog is jumping happily.' //
\endgl\xe

In this example, the reduplicated noun \rayr{venej/venej}{veney-veney} is 
marked as an agent in that the agent suffix \rayr{/ANF}{-ang} is appended to 
the noun as a unit \emph{after} reduplicating the noun stem. In other words, the 
following formation in which the root is reduplicated along with its declension 
suffix is ungrammatical for the purpose of forming a diminutive:

\ex
	*\rayr{\larger veneyNF/veneyNF}{*veneyang-veneyang}
\xe

Likewise, the reduplicated form is not treated in the way an endocentric 
compound would be, so case and plural marking cannot be appended to the first 
element:

\ex
	*\rayr{\larger veneyNF venej}{*veneyang veney}
\xe

While ordinary nouns undergo full reduplication to form a diminutive, in 
compounds, only the head is reduplicated, unless the compound is strongly 
lexicalized or has an idiomatic meaning going beyond that of its components. 
The following example shows the simple case of a transparent endocentric 
compound:

\ex\begingl
	\gla Ya yomayo mehir-mehirang seygo veno kay pang nanga nana. //
	\glb Ya yoma-yo mehir\til{}mehir-ang seygo veno kay pang nanga-Ø nana //
	\glc \LocT{} be-\TsgN{} \Dim{}\til{}tree-\Aarg{} apple pretty three 
		back house-\Top{} \Fpl{}.\Gen{} //
	\glft `There are three pretty little apple trees behind our house.' //
\endgl\xe

In this example, being endearing or otherwise small is treated as a property of 
the head, \xayr{mehirF}{mehir}{tree}, not of the whole compound 
\xayr{mehirFsejgo}{mehirseygo}{apple tree}, or the dependent, 
\xayr{sejgo}{seygo}{apple}---after all, an apple tree which is small is 
rather a small tree with apples on it than a tree with small apples. The 
avoidance of the fully reduplicated form 
\rayr{mehirFsejgo/mehirFsejgo}{mehirseygo-mehirseygo} is probably related to the 
notion of economy of expression.

\index{reduplication|)}

\subsection{Nominalization}
\index{nominalization|(}

Some accidental ways of deriving nouns have been mentioned above, for instance, 
some reduplicated non-nominal roots like \xayr{kusNF}{kusang}{double} or 
\xayr{veh/}{veha-}{build} may form nouns. However, Ayeri also has some 
dedicated morphology to derive nouns from other parts of speech. The most common 
and highly productive way to derive a noun, is the suffix \rayr{/AnF}{-an}. 
The examples in (\ref{ex:vb-nn}) illustrate some derivations from verbs, and 
(\ref{ex:adj-nn}) shows derivations from adjectives to nouns. As 
\xayr{kuhnF}{kuhan}{oar} shows, the nominalization may have an idiomatic 
meaning.

\pex\label{ex:vb-nn}
	\a \makebox[10.5em][l]{\xayr{\larger blNF/}{balang-}{search (v.)}}
		→ \xayr{\larger blNnF}{balangan}{search (n.)}
	\a \makebox[10.5em][l]{\xayr{\larger kuhF/}{kuh-}{row}}
		→ \xayr{\larger kuhnF}{kuhan}{oar}
	\a \makebox[10.5em][l]{\xayr{\larger rigF/}{rig-}{draw}}
		→ \xayr{\larger rignF}{rigan}{drawing}
	\a \makebox[10.5em][l]{\xayr{\larger vehF/}{veh-}{build}}
		→ \xayr{\larger vehnF}{vehan}{building}
\xe

\pex~\label{ex:adj-nn}
	\a \makebox[10.5em][l]{\xayr{\larger Apitu}{apitu}{pure}}
		→ \xayr{\larger Apitu\_an}{apituan}{purity}
	\a \makebox[10.5em][l]{\xayr{\larger gir}{gira}{urgent}}
		→ \xayr{\larger giraanF}{girān}{hurry}
	\a \makebox[10.5em][l]{\xayr{\larger pkisF}{pakis}{serious}}
		→ \xayr{\larger pkisnF}{pakisan}{seriousness}
	\a \makebox[10.5em][l]{\xayr{\larger vp}{vapa}{skillful}}
		→ \xayr{\larger vpn}{vapan}{skill}
\xe

Occasionally, it may even happen that a noun is derived from a noun with a 
related but sometimes more basic meaning using the nominalizer \rayr{/AnF}{-an}. 
This process, however, is not productive, so compared to deverbalization and 
deadjectivization, examples of this derivation strategy are few.

\pex\label{ex:nn-nn}
	\a \makebox[8em][l]{\xayr{\larger AgYmF}{ajam}{toy}}
		→ \xayr{\larger AgYmnF}{ajaman}{game}
	\a \makebox[8em][l]{\xayr{\larger kelNF}{kelang}{chain}}
		→ \xayr{\larger kelNnF}{kelangan}{connection}
	\a \makebox[8em][l]{\xayr{\larger nN}{nanga}{house}}
		→ \xayr{\larger nNaanF}{nangān}{household}
	\a \makebox[8em][l]{\xayr{\larger tenF}{ten}{life}}
		→ \xayr{\larger tennF}{tenan}{soul}
\xe

There are also some apparent nominalizations in \rayr{/AmF}{-am} and 
\rayr{/ANF}{-ang}, although these are irregular and non-productive:

\pex
	\a \makebox[8.5em][l]{\xayr{\larger AgY/}{aja-}{play}}
		→ \xayr{\larger AgYmF}{ajam}{toy}
	\a \makebox[8.5em][l]{\xayr{\larger ginF/}{gin-}{drink}}
		→ \xayr{\larger ginmF}{ginam}{glass}
	\a \makebox[8.5em][l]{\xayr{\larger mikF/}{mik-}{poison (v.)}}
		→ \xayr{\larger mikmF}{mikam}{poison (n.), venom}
	\a \makebox[8.5em][l]{\xayr{\larger nun/}{nuna-}{fly}}
		→ \xayr{\larger nunmF}{nunam}{feather}
\xe

\pex~
	\a \makebox[8em][l]{\xayr{\larger bjh/}{bayha-}{rule}}
		→ \xayr{\larger bjhNF}{bayhang}{government}
	\a \makebox[8em][l]{\xayr{\larger hp}{hapa}{remaining}}
		→ \xayr{\larger hpNF}{hapang}{remainder}
	\a \makebox[8em][l]{\xayr{\larger kd/}{kada-}{collect}}
		→ \xayr{\larger kdNF}{kadang}{committee; alliance}
	\a \makebox[8em][l]{\xayr{\larger mim}{mima}{possible}}
		→ \xayr{\larger mimNF}{mimang}{access}
\xe

Agentive nouns can be formed from regular nouns with the suffix 
\rayr{/my}{-maya}, compare the examples in (\ref{ex:mayaregular}). An 
epenthetic /a/ may be introduced to break up consonant clusters that would 
otherwise be either difficult to pronounce or violating phonotactics. When the 
stem of the word the agentive suffix is attached to ends in a consonant or 
/Ca/, it is also often found fused with the root, sometimes with the first /a/ 
of \fw{-Caya} lengthened, see (\ref{ex:mayairregular}). Specifically feminine 
agentive nouns can be derived with the related suffix \rayr{/vy}{-vaya}; two 
examples are given in (\ref{ex:vaya}).

\pex\label{ex:mayaregular}
	\a \makebox[7em][l]{\xayr{\larger AnFlF/}{anl-}{bring}}
		→ \xayr{\larger AnFlmy}{anlamaya}{waiter}
	\a \makebox[7em][l]{\xayr{\larger hor}{hora}{sin}}
		→ \xayr{\larger hormy}{horamaya}{sinner}
	\a \makebox[7em][l]{\xayr{\larger nsY/}{nasy-}{follow}}
		→ \xayr{\larger nsYmy}{nasyamaya}{follower}
	\a \makebox[7em][l]{\xayr{\larger teb/}{teba-}{bake}}
		→ \xayr{\larger tebmy}{tebamaya}{baker}
\xe

\pex~\label{ex:mayairregular}
	\a \makebox[7em][l]{\xayr{\larger As/}{asa-}{travel}}
		→ \xayr{\larger Asaay}{asāya}{traveler}
	\a \makebox[7em][l]{\xayr{\larger IbutF/}{ibut-}{trade}}
		→ \xayr{\larger Ibuty}{ibutaya}{trader, merchant}
	\a \makebox[7em][l]{\xayr{\larger lMtF/}{lant-}{lead}}
		→ \xayr{\larger lMty}{lantaya}{leader; driver}
	\a \makebox[7em][l]{\xayr{\larger tNF/}{tang-}{listen}}
		→ \xayr{\larger tNy}{tangaya}{listener}
\xe

\pex~\label{ex:vaya}
	\a \makebox[7em][l]{\xayr{\larger gnF}{gan}{child}}
		→ \xayr{\larger gnFvy}{ganvaya}{governess}
	\a \makebox[7em][l]{\xayr{\larger lnY}{lanya}{king}}
		→ \xayr{\larger lnFvy}{lanvaya}{queen}
\xe

Besides these, there is also a derivative suffix for makers of things, 
\rayr{/Ati}{-ati}, though this is not too productive, and sometimes irregular, 
as \xayr{sirFtNti}{sirtangati}{youth} shows:

\pex
	\a \makebox[11.5em][l]{\xayr{\larger giMdi}{gindi}{poem}}
		→ \xayr{\larger giMdti}{gindati}{poet}
	\a \makebox[11.5em][l]{\xayr{\larger sirFtNF}{sirtang}{young}}
		→ \xayr{\larger sirFtNti}{sirtangati}{youth}
	\a \makebox[11.5em][l]{\xayr{\larger thnF/}{tahan-}{write}}
		→ \xayr{\larger thnti}{tahanati}{scribe}
	\a \makebox[11.5em][l]{\xayr{\larger vehimF}{vehim}{piece of clothing}}
		→ \xayr{\larger vehimti}{vehimati}{tailor}
\xe

A few instances also exist where a tool of sorts is derived with a suffix 
\rayr{/(E)rYnF}{-(e)ryan}, which is related to the instrumental suffix 
\rayr{/Eri}{-eri} in combination with the nominalizer \rayr{/AnF}{-an}:

\pex
	\a \makebox[9em][l]{\xayr{\larger gurF/}{gur-}{turn}}
		→ \xayr{\larger gurFynF}{guryan}{coil, cylinder}
	\a \makebox[9em][l]{\xayr{\larger misF/}{mis-}{behave}}
		→ \xayr{\larger miserYnF}{miseryan}{method, strategy}
	\a \makebox[9em][l]{\xayr{\larger npF/}{nap-}{burn}}
		→ \xayr{\larger nperYnF}{naperyan}{tinder}
	\a \makebox[9em][l]{\xayr{\larger pr/}{pra-}{glitter, gleam}}
		→ \xayr{\larger pFrrYnF}{praryan}{spark}
\xe

\index{gerund|(}
While \rayr{/AnF}{-an} derives nouns from verbs to produce nouns that act as 
such in every way, it may sometimes be preferable to refer to the action as 
such by a noun, compare in English:

\pex
	\a\label{ex:devnouneng} Manhattan is famous for its tall 
		\textbf{buildings}.
	\a\label{ex:gerundeng} \textbf{Building} a house is an expensive 
		endeavor.
\xe

In (\ref{ex:devnouneng}), \fw{building} is simply a noun derived from the verb 
\fw{build}. It acts as a noun in every way, for example, it can serve as a 
subject and object, it can be pluralized, it can take determiners, and can be 
modified by adjectives. The form of \fw{building} in (\ref{ex:gerundeng}), 
however, is a gerund, and as such underlies the restriction that it cannot be 
pluralized \citep[35]{payne1997}. As we have seen at the beginning of this 
section on nominalization, Ayeri can derive \xayr{vehnF}{vehan}{building, 
construction} from the verb \xayr{vehF/}{veh-}{build}, which acts like every 
other common noun, much like in the English example in (\ref{ex:devnouneng}):

\pex
\a\label{ex:nomz-sbj-adj}\begingl
	\gla Lesāra maritay \textbf{vehānreng} \textbf{tado}. //
	\glb Lesa-ara maritay \textbf{vehān-reng} \textbf{tado} //
	\glc collapse-\TsgI{} about.to \textbf{building-\AargI{}} \textbf{old}//
	\glft `The old building is about to collapse.' //
\endgl

\a\label{ex:nomz-obj-det}\begingl
	\gla Le vacyang \textbf{eda-vehān}. //
	\glb Le vac=yang \textbf{eda=vehān-Ø} //
	\glc \PatTI{} like=\Fsg{}.\Aarg{} \textbf{this=building-\Top{}} //
	\glft `This building, I like it.' //
\endgl

\a\label{ex:nomz-pl-poss}\begingl
	\gla Ang latayo bayhang \textbf{vehānyeley} \textbf{yona}. //
	\glb Ang lata-yo bayhang-Ø \textbf{vehān-ye-ley} \textbf{yona} //
	\glc \AgtT{} sell-\TsgN{} government-\Top{} 
		\textbf{building-\Pl{}-\PargI{}} \textbf{\TsgN{}.\Gen{}} //
	\glft `The government is selling its buildings.' //
\endgl

\a\label{ex:nomz-qty}\begingl
	\gla Le ming kuysāran \textbf{vehān-kay} dirasyam ran. //
	\glb Le ming kuysa-aran \textbf{vehān-Ø=kay} diras-yam ran //
	\glc \PatTI{} can compare-\TplI{} \textbf{building-\Top=few} 
		splendor-\Dat{} \TsgI{}.\Gen{} //
	\glft `Few buildings can compare to its splendor.' //
\endgl
\xe

The above examples condense several properties into one for illustration. Thus, 
(\ref{ex:nomz-sbj-adj}) shows that \rayr{vehaanF}{vehān} can serve as basically 
a subject of a clause, and that it can as well be modified by an 
adjective---the choice of adjectives is not subject to any distributional 
restrictions other than those imposed by the semantic frame of 
\textsc{house}. In the next example, (\ref{ex:nomz-obj-det}), 
\rayr{vehaanF}{vehān} serves as the object of the clause and is being determined 
by the demonstrative prefix \xayr{Ed/}{eda-}{this}. The third example, 
(\ref{ex:nomz-pl-poss}), shows \rayr{vehaanF}{vehān} both pluralized and 
modified by a possessive pronoun, \xayr{yon}{yona}{of it}. And finally, in 
(\ref{ex:nomz-qty}) we see \rayr{vehaanF}{vehān} quantified by the suffix 
\xayr{/kj}{-kay}{few}.

Similar to the English example in (\ref{ex:gerundeng}), Ayeri can also derive 
nouns from the participle of a verb describing the action as such---a gerund. 
For an example, I will again draw on the Ayeri translation of Kafka's short 
story \enquote{Eine kaiserliche Botschaft} \citep[2, 14]{becker:kafka:imperial}:

\ex\label{ex:kafkagerund}\begingl
	\gla … nay ang pətangongva ankyu \textbf{haruyamanas} nanang megayena 
		yana kunangya vana. //
	\glb … nay ang pə-tang-ong=va.Ø ankyu \textbf{haru-yam-an-as} nanang 
		mega-ye-na yana kunang-ya vana //
	\glc … and \AgtT{} \NFut{}-hear-\Irr{}=\Ssg{}.\Top{} truly 
		\textbf{beat-\Ptcp{}-\Nmlz{}-\Parg{}} great fist-\Pl{}-\Gen{} 
		\TsgM{}.\Gen{} door-\Loc{} \Ssg{}.\Gen{} //
	\glft `… and you would indeed hear his magnifcent beating at your door 
		very soon.' //
\endgl\xe

The annotations to this translation contain a comment on the grammatical 
rules which operate in this passage, more specifically also on the gerund 
derivation \xayr{hruymnF}{haruyaman}{beating}:

\blockcquote[14--15]{becker:kafka:imperial}{Furthermore, I wrote 
\fw{haruyaman} `beating' instead of \fw{haruan} `beat(ing)' because I wanted to 
emphasize the process of beating as an incomplete action. This is possible here 
because the word is not topicalized and neither is it marked as a dative, which 
would also require \fw{haruyamanyam} `beat-\Ptcp{}-\Nmlz{}-\Dat{}' to become 
\fw{haruanyam} `beat-\Nmlz{}-\Dat{}' (the participle marker \fw{-yam} is 
derived from the dative case ending \fw{-yam}).}

We can read from this description that the participle marker in Ayeri has 
possibly been grammaticalized from the dative case marker, or that it is at 
least synchronically homonymous. In order for case marking to operate, this 
formation has to be nominalized, which is done the usual way by appending 
\rayr{/AnF}{-an}, thus yielding the suffix cluster \rayr{/ymnF}{-yaman} for the 
derivation of verbs as gerunds. If the gerund is marked for dative case, the 
suffix cluster *\rayr{/ymnFymF}{*-yamanyam} basically undergoes haplology to 
a simple nominalized form with the suffix cluster \rayr{/AnFymF}{-anyam}:

\ex\begingl
	\gla haru- {} haruyam {} haruyaman {} *haruyamanyam {} haruanyam //
	\glb haru- → haru-yam → haru-yam-an → haru-yam-an-yam → 
		haru-an-yam //
	\glc beat {} beat-\Ptcp{} {} beat-\Ptcp{}-\Nmlz{} {} 
		beat-\Ptcp{}-\Nmlz{}-\Dat{} {} beat-\Nmlz{}-\Dat{} //
\endgl\xe

The comment on the translation also makes a little note on the gerund being 
possible because the word is not topicalized. This is based on an old rule that 
gerunds cannot be topicalized unless nominalized first, however, usage has 
since changed so that earlier, \rayr{hruymF}{haruyam} would have constituted 
the gerund form, while even by the time of translating the short story, it had 
changed to \rayr{hruymnF}{haruyaman}. Thus, it is no surprise to see the 
following example, from the partial translation of Saint-Exupéry's story 
\enquote{Le petit prince} \citep[3, 13]{benung:petitprince}:

\ex\label{ex:exuperygerund}\begingl
	\gla Sa koronyang \textbf{palungyaman} na Baysānterpeng nay na Bayokivo 
		menaneri nivānyena. //
	\glb Sa koron=yang \textbf{palung-yam-an-Ø} na Baysānterpeng nay na 
		Bayokivo menan-eri nivān-ye-na //
	\glc \PatT{} knew=\Fsg{}.\Aarg{} 
		\textbf{distinguish-\Ptcp{}-\Nmlz{}-\Top{}} \Gen{} Realm.Middle 
		and \Gen{} Spring.Little first-\Ins{} glimpse-\Pl{}-\Gen{} //
	\glft `I knew how to distinguish between China and Arizona at first 
		sight.' //
\endgl\xe

A more literal translation of this sentence would be `The distinguishing of 
China and Arizona, I knew it at first sight', so the whole passage 
\rayr{pluNFymnF — n byokivo}{palungyaman … na Bayokivo} forms the topic of the 
sentence here, headed by the gerund 
\xayr{pluNFymnF}{palungyaman}{distinguishing}. According to the old rule 
obliquely quoted in the comment to the passage in (\ref{ex:kafkagerund}), this 
should not be possible. As said before, though, usage has changed.

A rule we can gather from the above example from Saint-Exupéry is that gerunds 
are treated as animate nouns. Since they are impersonal, they trigger neuter 
agreement on verbs. They can also be the objects of sentences. The passage in 
(\ref{ex:kafkagerund}) furthermore illustrates that gerunds can be modified by 
The following example shows a gerund used as an agent---basically a 
subject---as well \citep{benung:scientificmethod}:

\ex\label{ex:scimethgerund}\begingl
	\gla \textbf{Dilayamanang} kalamena bahalanas ayonena … //
	\glb \textbf{Dila-yam-an-ang} kalam-ena bahalan-as ayon-ena … //
	\glc \textbf{find.out-\Ptcp{}-\Nmlz{}-\Aarg{}} truth-\Gen{} 
		goal-\Parg{} man-\Gen{} … //
	\glft `(If) finding out the truth is the goal of the man …' //
\endgl\xe

What all the passages on gerunds quoted before show is that gerunds in Ayeri 
do not behave like transitive verbs as in English. Thus, what would be the 
object of the former verb appears in the genitive case in Ayeri. As in English, 
however, gerunds in Ayeri cannot be pluralized:

\ex\begingl
	\gla *Noyo \textbf{vehayamanjang} nangayena. //
	\glb Noyo \textbf{veha-yam-an-ye-ang} nanga-ye-na //
	\glc expensive \textbf{build-\Ptcp{}-\Nmlz{}-\Pl{}-\Aarg{}} 
		house-\Pl{}-\Gen{} //
	\glft `*The buildings of houses are expensive.' //
\endgl\xe

It is possible, however, to quantify gerunds insofar as the quantifier does not 
imply countable quantities of the action. Moreover, it is possible for gerunds 
to be modified by possessors. The following to sentences exemplify this use:

\ex\begingl
	\gla Ang lugayan \textbf{delacamanas-ikan} kayanya pang. //
	\glb Ang luga=yan.Ø \textbf{delak-yam-an-as=ikan} kayan-ya pang //
	\glc \AgtT{} go.through=\TplM{}.\Top{} 
		\textbf{suffer-\Ptcp{}-\Nmlz{}-\Parg{}=much} war-\Loc{} after //
	\glft `They went through a lot of suffering after the war.' //
\endgl\xe

\ex~\begingl
	\gla Krico \textbf{malyyamanang} muya \textbf{tan}. //
	\glb Krit-yo \textbf{maly-yam-an-ang} muya \textbf{tan} //
	\glc annoy-\TsgN{} \textbf{sing-\Ptcp{}-\Nmlz{}-\Aarg{}} wrong 
		\textbf{\TplM{}.\Gen{}} //
	\glft `Their off singing is annoying.' //
\endgl\xe

\index{gerund|)}
\index{nominalization|)}
\index{nouns|)}

\section{Pronouns}
\index{pronouns|(}

Ayeri possesses different kinds of pronouns in the sense that there is a closed 
class of words which contains anaphora of various types---personal pronouns, 
demonstrative pronouns, interrogative pronouns, relative pronouns, and other 
assorted pronouns (reflexive, reciprocal, distributive). Each class of pronouns 
will be discussed in the following.

\subsection{Personal pronouns}
\index{pronouns!personal|(}

\begin{figure}[tp]\centering
\caption{Personal pronouns}

\begin{tabu} to \linewidth{S X[c] X[c] X[c] X[c] X[c] X[c] X[c] X[c]}
\tableheaderfont\toprule
Person
	& \Top{}
	& \Aarg{}
	& \Parg{}
	& \Dat{}
	& \Gen{}
	& \Loc{}
	& \Caus{}
	& \Ins{}
	\\
\toprule

\Fsg{}
	& ay	% \Top{}
	& yang	% \Aarg{}
	& yas	% \Parg{}
	& yām	% \Dat{}
	& nā	% \Gen{}
	& yā	% \Loc{}
	& sā	% \Caus{}
	& rī	% \Ins{}
	\\
	
\midrule

\Ssg{}
	& va	% \Top{}
	& vāng	% \Aarg{}
	& vās	% \Parg{}
	& vayam	% \Dat{}
	& vana	% \Gen{}
	& vaya	% \Loc{}
	& vasa	% \Caus{}
	& vari	% \Ins{}
	\\

\midrule

\TsgM{}
	& ya	% \Top{}
	& yāng	% \Aarg{}
	& yās	% \Parg{}
	& yayam	% \Dat{}
	& yana	% \Gen{}
	& yāy	% \Loc{}
	& yasa	% \Caus{}
	& yari	% \Ins{}
	\\

\TsgF{}
	& ye	% \Top{}
	& yeng	% \Aarg{}
	& yes	% \Parg{}
	& yeyam	% \Dat{}
	& yena	% \Gen{}
	& yea	% \Loc{}
	& yesa	% \Caus{}
	& yeri	% \Ins{}
	\\

\TsgN{}
	& yo	% \Top{}
	& yong	% \Aarg{}
	& yos	% \Parg{}
	& yoyam	% \Dat{}
	& yona	% \Gen{}
	& yoa	% \Loc{}
	& yosa	% \Caus{}
	& yori	% \Ins{}
	\\

\TsgI{}
	& ra	% \Top{}
	& reng	% \Aarg{}
	& rey	% \Parg{}
	& rayam	% \Dat{}
	& ran	% \Gen{}
	& raya	% \Loc{}
	& rasa	% \Caus{}
	& rari	% \Ins{}
	\\

\midrule

\Fpl{}
	& ayn	% \Top{}
	& nang	% \Aarg{}
	& nas	% \Parg{}
	& nyam	% \Dat{}
	& nana	% \Gen{}
	& nyā	% \Loc{}
	& nisa	% \Caus{}
	& ni	% \Ins{}
	\\
	
\midrule

\Spl{}
	& va	% \Top{}
	& vāng	% \Aarg{}
	& vās	% \Parg{}
	& vayam	% \Dat{}
	& vana	% \Gen{}
	& vaya	% \Loc{}
	& vasa	% \Caus{}
	& vari	% \Ins{}
	\\

\midrule

\TplM{}
	& yan	% \Top{}
	& tang	% \Aarg{}
	& tas	% \Parg{}
	& cam	% \Dat{}
	& tan	% \Gen{}
	& ca	% \Loc{}
	& tis	% \Caus{}
	& ti	% \Ins{}
	\\

\TplF{}
	& yen	% \Top{}
	& teng	% \Aarg{}
	& tes	% \Parg{}
	& teyam	% \Dat{}
	& ten	% \Gen{}
	& teya	% \Loc{}
	& tēs	% \Caus{}
	& teri	% \Ins{}
	\\

\TplN{}
	& yon	% \Top{}
	& tong	% \Aarg{}
	& tos	% \Parg{}
	& toyam	% \Dat{}
	& ton	% \Gen{}
	& toya	% \Loc{}
	& tōs	% \Caus{}
	& tori	% \Ins{}
	\\

\TplI{}
	& ran	% \Top{}
	& teng	% \Aarg{}
	& tey	% \Parg{}
	& racam	% \Dat{}
	& ten	% \Gen{}
	& raca	% \Loc{}
	& ratas	% \Caus{}
	& ray	% \Ins{}
	\\

\bottomrule
\end{tabu}
\label{fig:perspro}
\end{figure}

As \autoref{fig:perspro} shows, Ayeri possesses quite a large number of 
personal pronouns with little syncretism between the different paradigm 
slots overall (the second person is a notable exception); there are also no 
gaps in the paradigm. Ayeri's personal pronouns reflect the grammatical 
features 
also found in nouns, that is, number, gender, and case, and person is added to 
that. The individual forms range from completely fused to fully transparent 
even 
within in the same case paradigm, for instance, \xayr{yaamF}{yām}{(to/for) me} 
on the one hand, and \xayr{yymF}{yayam}{(to/for) him} on the other. Originally, 
all pronouns have been regular formations based on the respective unmarked 
pronominal element listed in the \Top{} column of \autoref{fig:perspro} 
declined by adding a case suffix. Use has caused many of these formations to 
contract and erode as grammaticalization progressed:

\pex
\a\begingl
	\gla ayang → yāng //
	\glb ay-ang {} yāng //
	\glc \Fsg{}-\Aarg{} {} \Fsg{}.\Aarg{} //
\endgl

\a\begingl
	\gla iyatena → tan //
	\glb iy-a-t-ena {} tan //
	\glc \Tsg{}-\M{}-\Pl{}-\Gen{} {} \TsgM{}.\Gen{}\footnotemark //
\endgl
\xe

\footnotetext{Strictly speaking, this could as well be glossed as \fw{t<a>n} 
(\Tsg{}.\Gen{}<\M{}>). I chose to gloss the pronoun in the above way, however, 
in order to not overly complicate things.}

The plural series used to be derived by adding \rayr{/nF}{-n} or, in the third 
person, \rayr{/tF/}{\mbox{-t-}} to the pronoun stem, which can still be easily 
observed in the unmarked pronouns as well as in the alternation between 
\rayr{yF/}{y-} and \rayr{tF/}{t-} in the third person pronouns. The same goes 
for the gender-marking thematic vowel in the animate third person pronouns, 
which is retained as a distinctive feature even in the non-core pronouns in 
spite of sometimes heavy modifications. A further interesting property of Ayeri 
is that synchronically, singular and plural are distinguished, except for the 
second person, where the forms are the same, basically like in English. 
\citet{lehmann2015} explains, however, that this is not an unusual route to 
take 
for languages:

\blockcquote[42]{lehmann2015}{New pronouns, especially for the second person 
singular, are often obtained by shifting pronouns around in the paradigm, 
especially by substituting marked forms for unmarked ones. This explains, e.g., 
the use of [...] English \fw{you} for the second person singular}

The second person singular subject pronoun of English used to be \fw{thou}, 
cognate to German \fw{du}, which can still be found in Shakespeare, for 
instance. Something along the lines of English \fw{you} as a second 
person plural pronoun replacing second person singular \fw{thou} by way of a 
deferential singular use of a plural pronoun \citep[you, pron., adj., and 
n.]{oed} may have happened in Ayeri as well.

The personal pronouns are used in just the same way as their full-NP 
counterparts would be, also in the non-core cases:

\pex\label{ex:perspro}
\a\label{ex:pronfull}\begingl
	\gla Sa harya ang Paradan tandās kaleri. //
	\glb Sa har-ya ang Paradan tanda-as kal-eri //
	\glc \AgtT{} beat-\TsgM{} \Aarg{} Paradan fly-\Parg{} rag-\Ins{} //
	\glft `Paradan beats the fly with a rag.' //
\endgl

\a\label{ex:pronagt}\begingl
	\gla Sa haryāng tandās kaleri. //
	\glb Sa har=yāng tanda-as kal-eri //
	\glc \AgtT{} beat=\TsgM{}.\Aarg{} fly-\Parg{} rag-\Ins{} //
	\glft `He beats the fly with a rag.' //
\endgl

\a\label{ex:pronpat}\begingl
	\gla Sa harya ang Paradan yos kaleri. //
	\glb Sa har-ya ang Paradan yos kal-eri //
	\glc \AgtT{} beat-\TsgM{} \Aarg{} Paradan \TsgN{}.\Parg{} rag-\Ins{} //
	\glft `Paradan beats it with a rag.' //
\endgl

\a\label{ex:pronins}\begingl
	\gla Sa harya ang Paradan tandās rari. //
	\glb Sa har-ya ang Paradan tanda-as rari //
	\glc \AgtT{} beat-\TsgM{} \Aarg{} Paradan fly-\Parg{} \TsgI{}.\Ins{} //
	\glft `Paradan beats the fly with it.' //
\endgl

\xe

In the above set of examples, (\ref{ex:pronfull}) shows a sentence with full 
NPs, with the agent, \rayr{ANF prdnF}{ang Paradan} replaced by the third person 
singular masculine agent pronoun \rayr{yaaNF}{yāng} in (\ref{ex:pronagt}); in 
(\ref{ex:pronpat}) the patient, \rayr{tMdaasF}{tandās}, is replaced with the 
third person singular neuter patient pronoun \rayr{yosF}{yos}; in 
(\ref{ex:pronins}), lastly, the instrument, \rayr{kleri}{kaleri} is replaced 
with the third person singular inanimate instrumental pronoun \rayr{rari}{rari}.

Comparing the example sentences in (\ref{ex:perspro}) with the \Top{} column 
in \autoref{fig:perspro} another important property of personal pronouns 
shows. 
That is, the `unmarked' (or rather, zero-marked) pronoun forms are also the 
ones 
showing as verb agreement. The only important difference in this respect, 
however, is that the third person singular inanimate verb agreement marker is 
not \rayr{/r}{-ra}, but \rayr{/Ar}{-ara}. The following two examples are 
supposed to illustrate the parallel more clearly---observe the person marking 
on 
the verb in (\ref{ex:verbinfl1}) and the corresponding object pronouns in 
(\ref{ex:verbinfl2}):

\pex\label{ex:verbinfl1}
\a\begingl
	\gla Sa manya ang Ajān {} Pila. //
	\glb Sa man-ya ang ​Ajān Ø ​Pila //
	\glc \PatT{} greet-\TsgM{} \Aarg{} ​Ajān \Top{} ​Pila //
	\glft `Pila, Ajān greets her.' //
\endgl

\a\begingl
	\gla Sa manye ang Pila {} Ajān. //
	\glb Sa man-ye ang Pila Ø ​Ajān //
	\glc \PatT{} greet-\TsgF{} \Aarg{} Pila \Top{} ​Ajān //
	\glft `Ajān, she greets him.' //
\endgl

\xe

\pex~\label{ex:verbinfl2}
\a\begingl
	\gla Sa manye ang Pila ya. //
	\glb Sa man-ye ang Pila ya.Ø //
	\glc \PatT{} greet-\TsgF{} \Aarg{} Pila \TsgM{}.\Top{} //
	\glft `Pila greets him.' //
\endgl

\a\begingl
	\gla Sa manya ang Ajān ye. //
	\glb Sa man-ya ang ​Ajān ye.Ø //
	\glc \PatT{} greet-\TsgM{} \Aarg{} ​Ajān \TsgF{}.\Top{} //
	\glft `Ajān greets her.' //
\endgl

\xe

Another important property of both pronouns and verbs which will be dealt with 
in more detail in the chapter on verbs proper---person agreement morphology is 
a domain of verbs---is that agent pronouns replace person agreement by 
cliticizing to the verb stem:

\pex
\a\begingl
	\gla Sa manya ang Ajān {} Pila. //
	\glb Sa man-ya ang ​Ajān Ø ​Pila //
	\glc \PatT{} greet-\TsgM{} \Aarg{} ​Ajān \Top{} ​Pila //
	\glft `Pila, Ajān greets her.' //
\endgl

\a\begingl
	\gla Sa manyāng {} Pila. //
	\glb Sa man=yāng Ø ​Pila //
	\glc \PatT{} greet=\TsgM{}.\Aarg{} \Top{} ​Pila //
	\glft `Pila, he greets her.' //
\endgl
\xe

\index{pronouns!personal|)}

\subsection{Demonstrative pronouns}
\index{pronouns!demonstrative|(}

\begin{figure}[tp]\centering
\caption{Demonstrative pronouns}

\begin{tabu} to .75\linewidth{S[2] X[4c] X[4c] X[4c]}
\tableheaderfont\toprule

Case
	& \Prox{}
	& \Dist{}
	& \Indf{}
	\\
\toprule

\Top{}
	& edanya
	& adanya
	& danya
	\\
	
\midrule
	
\Aarg{}
	& edanyāng
	& adanyāng
	& \emph{danyāng}
	\\

\Aarg{}.\Inan{}
	& edareng, \emph{edanyareng}
	& adareng, adanyareng
	& \emph{danyareng}
	\\
	
\Parg{}
	& edanyās
	& adanyās
	& danyās
	\\

\Parg{}.\Inan{}
	& edaley
	& \emph{adaley}
	& danyaley
	\\

\Dat{}
	& \emph{edayam}
	& adayam
	& \emph{danyayam}
	\\

\midrule

\Gen{}
	& edanyana
	& adanyana
	& danyana
	\\
	
\Loc{}
	& \emph{edanyaya}
	& adanyaya
	& \emph{danyaya}
	\\
	
\Caus{}
	& \emph{edanyasa}
	& \emph{adanyasa}
	& \emph{danyasa}
	\\
	
\Ins{}
	& \emph{edanyari}
	& \emph{adanyari}
	& \emph{danyari}
	\\

\bottomrule
\end{tabu}
\label{fig:detpro}
\end{figure}

Demonstrative pronouns in Ayeri are formed with one of the demonstrative 
prefixes---\xayr{Ed/}{eda-}{this} (\Prox{}), \xayr{Ad/}{ada-}{that} 
(\Dist{}), and \xayr{d/}{da-}{such} (\Indf{})---combined with a morpheme 
\rayr{nY}{nya}, which is related to the word for `person', \rayr{nYaanF}{nyān}. 
\autoref{fig:detpro} gives the declined forms for both proximal and distal. 
Those forms attested in the corpus gathered from dictionary entries and example 
texts also used for the syllable structure analyses in 
\autoref{sec:phonotactics} appear in upright type, those that should be 
grammatical as well otherwise are given in italic type. The corpus is very 
small, but the prevalence of some forms is possibly reflecting varying degrees 
of grammaticalization at least to some extent. \autoref{tab:detprontokenfq} 
gives the token frequencies of the various attested forms.

\begin{table}[tp]\centering
\caption{Token frequencies of attested demonstrative pronouns}

\begin{tabu} to .75\linewidth {>{\itshape}X[2l] X[2l] X[1c] X[1c]}
\tableheaderfont\toprule

Pronoun
	& Gloss
	& Absolute
	& Relative
	\\

\toprule

edanya
	& this.\Top{}
	& 1
	& 1.69\pct
	\\

adanya
	& that.\Top{}
	& 9
	& 15.25\pct
	\\

danya
	& such.\Top{}
	& 1
	& 1.69\pct
	\\

\midrule

edanyāng
	& this.\Aarg{}
	& 4
	& 6.78\pct
	\\

adanyāng
	& that.\Aarg{}
	& 8
	& 13.56\pct
	\\

edareng
	& this.\AargI{}
	& 3
	& 5.08\pct
	\\

adareng
	& that.\AargI{}
	& 15
	& 25.42\pct
	\\

adanyareng
	& that.\AargI{}
	& 1
	& 1.69\pct
	\\

\midrule

edanyās
	& this.\Parg{}
	& 1
	& 1.69\pct
	\\

adanyās
	& that.\Parg{}
	& 2
	& 3.39\pct
	\\

danyās
	& such.\Parg{}
	& 2
	& 3.39\pct
	\\

edaley
	& this.\PargI{}
	& 2
	& 3.39\pct
	\\

danyaley
	& such.\PargI{}
	& 2
	& 3.39\pct
	\\

\midrule

adayam
	& that.\Dat{}
	& 3
	& 5.08\pct
	\\

\midrule

edanyana
	& this.\Gen{}
	& 1
	& 1.69\pct
	\\

adanyana
	& that.\Gen{}
	& 2
	& 3.39\pct
	\\

danyana
	& such.\Gen{}
	& 1
	& 1.69\pct
	\\

\midrule

adanyaya
	& that.\Loc{}
	& 1
	& 1.69\pct
	\\

\bottomrule

\textup{Total}
	& 
	& 59
	& 100\pct
	\\

\bottomrule
\end{tabu}
\label{tab:detprontokenfq}
\end{table}

Of all the cases, the agent demonstratives have the highest token frequency at 
60.78\pct{}, especially the distal pronouns are very frequent in the sample. 
Moreover, the distal inanimate agent demonstative occurs twice as often as its 
animate counterpart, the shortened form \xayr{AdrENF}{adareng}{that (one)} 
being far more current than the full form \rayr{AdnYreNF}{adanyareng}. 
Interestingly, the shortened form \xayr{EdreNF}{edareng}{this one} is also the 
only one attested for the inanimate proximate agent; similarly, the only dative 
demonstrative attested once is shortened as well: \xayr{AdymF}{adayam}{(to/for) 
that}. For non-core cases, only long demonstratives are sparingly attested.

Given that most verbs should have a valency between 1 and 3, it is not 
surprising that only very few cases of non-core demonstratives are attested. It 
is furthermore not surprising that those demonstratives with a high frequency 
of use are eroded in some way: it seems that Ayeri prefers them to stay 
trisyllabic, which is achieved by dropping the \rayr{nY}{nya} 
part.\footnote{According to Zipf's law, word length and use frequency correlate 
in that the most frequently used words in a language also tend to be the 
shortest \citep[25--27]{zipf1935}.} A further reason for dropping the 
\rayr{nY}{nya} part especially in the inanimate demonstratives may be that it is 
perceived as a marker of animacy---it has been noted above already that it is 
related to the word \xayr{nYaanF}{nyān}{person}. Both factors, high frequency 
and semantic mismatch, may thus promote contraction.

Still, the question for the reason for the high frequency especially of 
\rayr{AdreNF}{adareng} remains open. It may be elucidated by looking at a few 
examples of this word in context, however.

\pex\label{ex:demexpl}
\a\begingl
	\gla Nay ang nelyo-ikan sungkorankihas, adareng tono. //
	\glb Nay ang nel-yo=ikan sungkorankihas, ada-reng tono //
	\glc and \AgtT{} help-\TsgN{}=much geography, that-\AargI{} certain //
	\glft `And geography, that's for sure, helped me a lot.'%
		\tc{\citep[13]{benung:petitprince}} //
\endgl

\a\begingl
	\gla Adareng merambay-ikan, le sundalvāng sasān vana ... //
	\glb Ada-reng merambay=ikan, le sundal=vāng sasān-Ø vana ... //
	\glc that-\AargI{} useful=very, \PatTI{} lose=\Ssg{}.\Aarg{} way-\Top{} 
		\Ssg{}.\Gen{} ... //
	\glft `It’s very useful if you get lost [...]'%
		\tc{\citep[14]{benung:petitprince}} //
\endgl

\a\begingl
	\gla Adareng danyaley segasena boa tinka. //
	\glb Ada-reng danya-ley segas-ena boa tinka //
	\glc that-\AargI{} such-\PargI{} snake-\Gen{} boa closed //
	\glft `The one of the closed boa snake.'\footnotemark%
		\tc{\citep[22]{benung:petitprince}} //
\endgl

\xe

\footnotetext{More literal translations of this sentence are `That is the one 
of the closed boa snake' or `That is one of a closed boa snake'.}

In all of the example sentences in (\ref{ex:demexpl}), 
\xayr{AdreNF}{adareng}{that (one)} serves as a dummy pronoun together with a 
predicative adjective or NP, which is the main reason why it occurs so 
frequently. This is to say, Ayeri prefers the demonstrative pronoun 
\rayr{AdreNF}{adareng} as the dummy agent in predicative contexts over the 
personal pronoun \xayr{reNF}{reng}{it}. Otherwise, however, demonstrative 
pronouns work regularly as deictic anaphora: `this', `that', and `such (a)', 
except that as nominal elements they are declined for case---but not for number 
or animacy, which is a notable difference between demonstrative pronouns and 
personal pronouns:

\pex
\a\begingl
	\gla Ang vehya {} Ajān nangās. //
	\glb Ang veh-ya Ø Ajān nanga-as //
	\glc \AgtT{} build-\TsgM{} \Top{} Ajān house-\Parg{} //
	\glft `Ajān builds a house.' //
\endgl

\a\begingl
	\gla Nangās? Sa vehyāng may danya. //
	\glb Nanga-as? Sa veh=yāng may danya-Ø //
	\glc house-\Parg{}? \PatT{} build=\TsgM{}.\Aarg{} \Aff{} such-\Top{} //
	\glft `A house? He builds one indeed.' //
\endgl

\xe

\pex~
\a\begingl
	\gla Sā hasuyeng eda-migorayye. //
	\glb Sā hasu=yeng eda=migoray-ye-Ø //
	\glc \CauT{} sneeze=\TsgF{}.\Aarg{} this=flower-\Pl{}-\Top{} //
	\glft `These flowers make her sneeze.' //
\endgl

\a\begingl
	\gla Ang tipinyon nivaye yena adanyari naynay. //
	\glb Ang tipin-yon niva-ye-Ø yena adanya-ri naynay //
	\glc \AgtT{} itch-\TplN{} eye-\Pl{}-\Top{} \TsgF{}.\Gen{} that-\Caus{} 
		as.well //
	\glft `Her eyes are itching due to that/them/those as well.' //
\endgl
\xe

As mentioned in the previous chapter (p.~\pageref{nounprefixes}), the prefix 
\xayr{d/}{da-}{such, so} can combine with a range of phrases, but most notably 
NPs, to serve as an indefinite demonstrative:

\ex\begingl
	\gla Adareng da-dipakanas. //
	\glb Adareng da=dipakan-as //
	\glc that-\AargI{} such=pity-\Parg{} //
	\glft `That is such a pity.' //
\endgl\xe

\rayr{d/}{da-} can be used to express English `one' in the sense of a deictic 
anaphora as well. Thus, to express `the X one', if X is an adjective, it is 
strictly speaking necessary to use the full demonstrative pronoun, 
\rayr{dnY}{danya}, since adjectives do not decline, and Ayeri largely avoids 
undeclined NPs:\footnote{See \autoref{subsec:uncased} above for examples of 
situations where nouns regularly do not exhibit case marking.}

\pex
\a\begingl
	\gla Silvyo danyāng kivo ku-mino-ing. //
	\glb Silv-yo danya-ang kivo ku=mino=ing //
	\glc look-\TsgN{} such-\Aarg{} little like=happy=so //
	\glft `The little one looks so happy.' //
\endgl

\a\label{ex:danyatop}\begingl
	\gla Sa noyang danya tuvo. //
	\glb Sa no=yang danya-Ø tuvo //
	\glc \PatT{} want=\Fsg{}.\Aarg{} such-\Top{} red //
	\glft `I want the red one.' //
\endgl

\xe

Nonetheless, in cases like (\ref{ex:danyatop}) where the demonstrative is 
topicalized, the prefixed form may be used, which is possible since 
\rayr{d/}{da-} is a clitic that binds to NPs, rather than nouns. As we have 
seen before, NPs do not exhibit overt case marking if topicalized, so whether 
\rayr{d/}{da-} leans on a superficially unmarked noun or an adjective, 
which is always unmarked for case, does not matter, since both are NPs. The 
sentence presented in (\ref{ex:danyatop}) is thus very formal. Less formally, 
the following is acceptable as well:

\ex\begingl
	\gla Sa noyang da-tuvo. //
	\glb Sa no=yang da=tuvo.Ø //
	\glc \PatT{} want=\Fsg{}.\Aarg{} such=red.\Top{} //
	\glft `I want the red one.' //
\endgl\xe

\index{pronouns!demonstrative|)}

\subsection{Interrogative pronouns}
\index{pronouns!interrogative|(}

The intererrogative pronouns are all formed with \rayr{si/}{si}, often combined 
with a lexical element or a case marker; \rayr{si/}{si} is also related to the 
relativizer \rayr{si}{si}. The interrogative pronouns are listed in 
\autoref{fig:interpro}.

\begin{figure}[htp]\centering
\caption{Interrogative pronouns}
\begin{tabu} to \linewidth {X[3] X[9] X[8]}
\tableheaderfont\toprule
Pronoun
	& Literal meaning
	& Idiomatic meaning
	\\

\toprule

\rayr{sinY}{sinya}
	& which one (\xayr{nYaanF}{nyān}{person})
	& `who', `what', `which'
	\\

\midrule

\rayr{siknF}{sikan}
	& how much (\xayr{IknF}{ikan}{much})
	& `how much', `how many'
	\\

\rayr{sikj}{sikay}
	& with what (\xayr{kjvo}{kayvo}{with})
	& `how' (tool, circumstance)
	\\

\rayr{siminF}{simin}
	& which way (\xayr{mirnF}{miran}{way})
	& `how' (way, procedure)
	\\

\rayr{sitdj}{sitaday}
	& which time (\xayr{tdj}{taday}{time})
	& `when'
	\\

\rayr{siynF}{siyan}
	& which place (\xayr{yno}{yano}{place})
	& `where'
	\\

\bottomrule
\end{tabu}
\label{fig:interpro}
\end{figure}

A property which all interrogative pronouns share, however, is that they are 
placed \fw{in situ}. That is, they appear in the same position as the phrase 
they stand in for, so there will not be movement to the front as in English. 
Additionally, impersonal interrogative pronouns cannot be topicalized since they 
also do not inflect for case, which preempts the distributional difference.

\pex
\a\begingl
	\gla Sa petigavāng inun sikan? //
	\glb Sa petiga=vāng inun-Ø sikan //
	\glc \PatT{} catch=\Ssg{}.\Aarg{} fish-\Top{} how.much //
	\glft `How much fish did you catch?' //
\endgl

\a\begingl
	\gla Sa-sahavāng sitaday? //
	\glb Sa\til{}saha=vāng sitaday //
	\glc \Iter{}\til{}come=\Ssg{}.\Aarg{} when //
	\glft `When will you return?' //
\endgl
\xe

In the table on interrogative pronouns, \xayr{sinY}{sinya}{who, what, which} is 
seperated from the other pronouns because it behaves differently in that it 
can be declined for all cases according to the syntactic or semantic role of 
the NP it replaces and can and will often be topicalized---what you query about 
will likely constitute the topic of the sentence and the answer.

\pex
\a\begingl
	\gla Ang yomayo sinya adaya?\footnotemark //
	\glb Ang yoma-yo sinya-Ø adaya //
	\glc \AgtT{} exist-\TsgN{} who-\Top{} there //
	\glft `Who is there?' //
\endgl

\a\begingl
	\gla Sa narayeng sinya? //
	\glb Sa nara=yeng sinya-Ø //
	\glc \PatT{} say=\TsgF{}.\Aarg{} what-\Top{} //
	\glft `What did she say?' //
\endgl

\xe

\footnotetext{This may be shortened to just \xayr{sinYaaNF Ady?}{sinyāng 
adaya?}{who (is) there?} (who-\Aarg{} there).}

\begin{figure}[tp]\centering
\caption{Declension paradigm for \xayr{sinY}{sinya}{who, what}}
\begin{tabu} to \linewidth {X[1] X[3] X[8]}
\tableheaderfont\toprule
Case
	& Pronoun
	& Translation
	\\

\toprule

\Top{}
	& \rayr{sinY}{sinya}
	& `who', `what'
	\\

\midrule

\Aarg{}
	& \rayr{sinYaaNF}{sinyāng}
	& `who', `what'
	\\

\AargI{}
	& \rayr{sinYreNF}{sinyareng}
	& `who', `what'
	\\
\Parg{}
	& \rayr{sinYaasF}{sinyās}
	& `whom', `what'
	\\
\PargI{}
	& \rayr{sinYlej}{sinyaley}
	& `whom', `what'
	\\
\Dat{}
	& \rayr{sinYymF}{sinyayam}
	& `for/to whom', `for/to what'
	\\

\midrule

\Gen{}
	& \rayr{sinYn}{sinyana}
	& `whose', `from whom', `from what'
	\\

\Loc{}
	& \rayr{sinYy}{sinyaya}
	& `in/at/on whom', `in/at/on what'
	\\

\Caus{}
	& \rayr{sinYis}{sinyisa}
	& `due to/because of whom', `due to/because of what'
	\\

\Ins{}
	& \rayr{sinYri}{sinyari}
	& `by whose help', `with what'
	\\

\bottomrule
\end{tabu}
\label{fig:sinya}
\end{figure}

Ayeri does not strictly distinguish animate and inanimate entities in its 
interrogative pronouns, so there is no distinction between `who' and `what'. 
\rayr{sinY}{sinya} and/or the verb will simply inflect according to context and 
the speaker's expectations or knowledge (compare \autoref{fig:sinya}). Thus, 
there is no dedicated question word for `why', since in Ayeri one can simply ask 
`due to what/whom' by inflecting \rayr{sinY}{sinya}:

\pex
\a\begingl
	\gla Le kayāng adanya sinyayam? //
	\glb Le ka=yāng adanya-Ø sinya-yam //
	\glc \PatTI{} throw.away=\TsgM{}.\Aarg{} that-\Top{} what-\Dat{} //
	\glft `Why (= what for) did he throw that away?' //
\endgl

\a\begingl
	\gla Ang prantoyva sinyaisa? //
	\glb Ang prant-oy=va.Ø sinya-isa //
	\glc \AgtT{} ask-\Neg{}=\Ssg{}.\Top{} what-\Caus{} //
	\glft `Why (= because of what) did you not ask?' //
\endgl

\xe

While there is no dedicated `why', Ayeri distinguishes between two kinds of 
`how': \rayr{siminF}{simin} asks about the way by which---or the circumstances 
under which---an action is carried out, whereas \rayr{sikj}{sikay} asks for the 
means or tools:

\pex
\a\label{ex:simin}\begingl
	\gla Le tiyavāng vadisān simin? //
	\glb Le tiya=vāng vadisān-Ø simin //
	\glc \PatTI{} make=\Ssg{}.\Aarg{} bread-\Top{} how //
	\glft `How do you make bread?' //
\endgl

\a\label{ex:sikay}\begingl
	\gla Le peralvāng sagan sikay? //
	\glb Le peral=vāng sagan-Ø sikay //
	\glc \PatTI{} grind=\Ssg{}.\Aarg{} flour-\Top{} how //
	\glft `How do you grind flour?' //
\endgl

\xe

The correct answer to the question in (\ref{ex:simin}) needs to treat the 
process of making bread, since \rayr{siminF}{simin} asks about the way; a 
correct answer to the question in (\ref{ex:sikay}), on the other hand, will 
likely mention grinding utensils, like a mill or a pestle. Even though 
Ayeri possesses an instrumental case which can be used in a comitative way, 
note the conflation of that and the preposition of accompaniment, 
\rayr{kjvo}{kayvo}, in this case (see \autoref{subsubsec:instrumental}).

Comparing Tables \ref{fig:interpro} and \ref{fig:sinya}, it may strike the 
reader's eye that there are two possbilities to express 
`where'---lexical \rayr{siynF}{siyan} and synthetic \rayr{sinYy}{sinyaya}. It 
is important to note, however, that these are not strictly interchangeable, 
even though some variation is expectable. While \rayr{siynF}{siyan} refers to 
places in general, the \rayr{sinY}{sinya} series refers to \emph{entities} both 
animate and inanimate more specifically:

\pex
\a\begingl
	\gla Saravāng siyan? --- Ya Sikatay. //
	\glb Sara=vāng siyan --- Ya Sikatay //
	\glc go=\Ssg{}.\Aarg{} where --- \Loc{} Sikatay //
	\glft `\enquote{Where are you going?}---\enquote{To Sikatay.}' //
\endgl

\a\begingl
	\gla Ya divvāng sinya? --- Ya Haki. //
	\glb Ya div=vāng sinya-Ø --- Ya Haki //
	\glc \LocT{} stay=\Ssg{}.\Aarg{} who-\Top{} --- \Loc{} Haki //
	\glft `\enquote{Where are you staying?}---\enquote{At Haki's}' //
\endgl

\xe

\index{pronouns!interrogative|)}

\subsection{Relative pronouns}
\index{pronouns!relative|(}

\begin{figure}[tp]\centering
\caption{Relative pronouns}

\begin{tabu} to \linewidth {S X[c] X[c] X[c] X[c] X[c] X[c]}
\tableheaderfont\toprule
Case
	& Pronoun
	& \multicolumn{5}{c}{Pronoun with secondary inflection}
	\\

\tablesubheaderfont\cmidrule{3-7}
	& 
	& \Dat{}
	& \Gen{}
	& \Loc{}
	& \Caus{}
	& \Ins{}
	\\
	
\toprule

Ø
	& si % Ø
	& siyām % \Dat{}
	& sinā % \Gen{}
	& siyā % \Loc{}
	& sisā % \Caus{}
	& sirī % \Ins{}
	\\

\midrule

\Aarg{}
	& sang % Ø
	& sangyam % \Dat{}
	& sangena % \Gen{}
	& sangya % \Loc{}
	& sangisa % \Caus{}
	& sangeri % \Ins{}
	\\

\Aarg{}.\Inan{}
	& sireng % Ø
	& sirengyam % \Dat{}
	& sirengena % \Gen{}
	& sirengya % \Loc{}
	& sirengisa % \Caus{}
	& sirengeri % \Ins{}
	\\
	
\Parg{}
	& sas % Ø
	& sasyam % \Dat{}
	& sasena % \Gen{}
	& sasya % \Loc{}
	& sasisa % \Caus{}
	& saseri % \Ins{}
	\\

\Parg{}.\Inan{}
	& siley % Ø
	& sileyyam % \Dat{}
	& sileyena % \Gen{}
	& sileyya % \Loc{}
	& sileyisa % \Caus{}
	& sileyeri % \Ins{}
	\\

\Dat{}
	& siyam % Ø
	& siyamyam % \Dat{}
	& siyamena % \Gen{}
	& siyamya % \Loc{}
	& siyamisa % \Caus{}
	& siyameri % \Ins{}
	\\

\midrule

\Gen{}
	& sina/sena % Ø
	& sinayam % \Dat{}
	& sinana % \Gen{}
	& sinaya % \Loc{}
	& sinaisa % \Caus{}
	& sinari % \Ins{}
	\\
	
\Loc{}
	& siya % Ø
	& siyayam % \Dat{}
	& siyana % \Gen{}
	& siyaya % \Loc{}
	& siyaisa % \Caus{}
	& siyari % \Ins{}
	\\
	
\Caus{}
	& sisa % Ø
	& sisayam % \Dat{}
	& sisana % \Gen{}
	& sisaya % \Loc{}
	& sisaisa % \Caus{}
	& sisari % \Ins{}
	\\
	
\Ins{}
	& seri % Ø
	& seriyam % \Dat{}
	& serina % \Gen{}
	& seriya % \Loc{}
	& serīsa % \Caus{}
	& seriri % \Ins{}
	\\

\bottomrule
\end{tabu}
\label{fig:relpro}
\end{figure}

As has been described before, Ayeri connects relative clauses to main clauses 
with the relativizer \rayr{si}{si}. This relativizer can be declined for case 
in accordance to the relative clause's head in the matrix clause. The 
respective forms can be gathered from \autoref{fig:relpro} (column 
`Pronoun').

\pex
\a\label{ex:n-rel}\begingl
	\gla Eryyo tarela natrangās si tado. //
	\glb Ery-yo tarela natranga-as si tado //
	\glc use-\TsgN{} still temple-\Parg{} \Rel{} old //
	\glft `The temple, which is old, is still being used.' //
\endgl

\a\label{ex:n-adj-rel}\begingl
	\gla Edanyāng ayonas sirtang sas ang sihabaya mondoas nana. //
	\glb Edanya-ang ayon-as sirtang si-as ang sihaba=ya mondo-as nana //
	\glc this-\Aarg{} man-\Parg{} young \Rel{}-\Parg{} 
		ang tend=\TsgM{}.\Top{} garden-\Parg{} \Fpl{}.\Gen{} //
	\glft `This is the young man who tends our garden.' //
\endgl
\xe

As explained in \autoref{sec:markstrat}, if the relativizer is immediately 
following its lexical head, only the base form \rayr{si}{si} is used, which is 
illustrated in (\ref{ex:n-rel}). Here, the head of the relative clause is 
\xayr{ntFrNaasF}{natrangās}{the temple}, which is immediately followed by the 
relative clause. If word material is intervening, however, which is the case in 
(\ref{ex:n-adj-rel}), the relative pronoun is usually inflected to agree in 
case with its antecedent: \rayr{ssF}{sas} agrees in case with 
\rayr{AyonsF}{ayonas} two words over to the left. Relative pronouns do not 
agree in number with their heads, though, and in gender only insofar as it is 
relevant to nominal case inflection, that is, agents and patients are 
distinguished for animacy.

A special property of the relative pronoun is that it can be declined for its 
role in the relative clause as well to express more complex relationships 
between the main clause and the relative clause. The respective forms can be 
found in the columns titled `Pronoun with secondary inflection' in 
\autoref{fig:relpro}. The token frequency of the actually occurring complex 
relative pronouns in the very small corpus gathered from example texts and 
dictionary entries (see \autoref{sec:phonotactics}) is given in 
\autoref{tab:relprotokenfreq}.

\begin{table}[tp]\centering
\caption{Token frequencies of attested complex relative pronouns (n\,=\,12)}

\begin{tabu} to .75\linewidth {>{\itshape}X[2l] X[2l] X[1c]}
\tableheaderfont\toprule

Pronoun & Gloss & Absolute \\

\toprule

siyā	& \Rel{}.Ø.\Loc{} & 7 \\
sirī	& \Rel{}.Ø.\Ins{} & 3 \\
sinā	& \Rel{}.Ø.\Gen{} & 1 \\
siyām	& \Rel{}.Ø.\Dat{} & 1 \\

\bottomrule
\end{tabu}
\label{tab:relprotokenfreq}
\end{table}

Compared to the unmarked relativizer \rayr{si}{si}, which occurs 50 times in 
the sample (all relative pronouns from \autoref{fig:relpro} occur 80 times in 
total), the complex relative pronouns have a very low frequency. This is not 
surprising, since `for whom', `by which', etc. are quite specialized 
expressions. It also seems that those forms unmarked for their antecedent are 
preferred, since those are the only ones attested---the sample is really much 
too small to make actually meaningful judgements here, however. Examples of 
complex relative pronouns are:

\pex
\a\begingl
	\gla Le vacyang koya yana sileyya adanyana. //
	\glb Le vac=yang koya-Ø yana si-ley-ya adanya-na //
	\glc \PatTI{} like=\Fsg{}.\Aarg{} book-\Top{} \TsgM{}.\Gen{} 
		\Rel{}-\PargI{}-\Loc{} that-\Gen{} //
	\glft `I like his book in which I read about it.' //
\endgl

\a\begingl
	\gla Ya saratang yano siyām sarasatang. //
	\glb Ya sara=tang yano-Ø si-Ø-yām sara-asa=tang //
	\glc \LocT{} go=\TplM{}.\Aarg{} place-\Top{} \Rel{}-\Loc{}-\Dat{} 
		go-\Hab{}=\TplM{}.\Aarg{} //
	\glft `They went to the place to which they always went.' //
\endgl
\xe

\index{pronouns!relative|)}

\index{pronouns|)}
