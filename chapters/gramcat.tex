% kate: word-wrap true;

\chapter{Grammatical categories}

While the previous chapter was about general mechanisms of marking in Ayeri, 
this chapter will dive into the various parts of speech in order to define 
their morphology with a closer look. I will begin with nouns as the main 
carriers of meaning, then deal with other parts of speech that regularly 
feature in the noun phrase---pronouns, adjectives, and adpositions. Following 
this, there will be a discussion of verbs and adverbs before moving on to 
numerals and conjunctions.

\section{Nouns}
\index{nouns|(}

Nouns in Ayeri have \emph{gender} and \emph{number} as their inherent 
grammatical properties. Besides common nouns, there are, of course, also proper 
nouns (i.e. names) and deverbal nouns. Nouns, as the heads of NPs, are also 
assigned \emph{case} by the VP, which is a third grammatical property they 
display. For an illustration of the declension paradigms, compare Figures 
\ref{fig:anideclcons}–\ref{fig:inandeclvow}.

\begin{figure}[ht]
\caption[Declension paradigm for Ayeri \xayr{bdnF}{badan}{father}]{Declension 
paradigm for Ayeri \xayr{bdnF}{badan}{father} (animate; consonantal root)}
\begin{tabu} to \linewidth {X[1] I[2] X[4] I[2] X[4]}
\tableheaderfont\toprule

	& \multicolumn2{c}{Singular}
	& \multicolumn2{c}{Plural}
	\\

\midrule
	
\Top{}
	& badan
	& `the father'
	%
	& badanye
	& `the fathers'
	\\

\midrule

\Aarg{}
	& badanang
	& `father'
	%
	& badanjang
	& `fathers'
	\\

\Parg{}
	& badanas
	& `father' (obj.)
	%
	& badanjas
	& `fathers' (obj.)
	\\

\Dat{}
	& badanyam
	& `to the father'
	%
	& badanjyam
	& `to the fathers'
	\\

\midrule

\Gen{}
	& badanena
	& `of the father'
	%
	& badanyena
	& `of the fathers'
	\\
	
\Loc{}
	& badanya
	& `at the father'
	%
	& badanjya
	& `at the fathers'
	\\

\Caus{}
	& badanisa
	& `due to the father'
	%
	& badanjisa
	& `due to the fathers'
	\\

\Ins{}
	& badaneri
	& `with the father'
	%
	& badanyeri
	& `with the fathers'
	\\

\bottomrule
\end{tabu}
\label{fig:anideclcons}
\end{figure}
~
\begin{figure}[ht]
\caption[Declension paradigm for Ayeri \xayr{mv}{māva}{mother}]{Declension 
paradigm for Ayeri \xayr{mv}{māva}{mother} (animate; vocalic root)}
\begin{tabu} to \linewidth {X[1] I[2] X[4] I[2] X[4]}
\tableheaderfont\toprule

	& \multicolumn2{c}{Singular}
	& \multicolumn2{c}{Plural}
	\\

\midrule
	
\Top{}
	& māva
	& `the mother'
	%
	& māvaye
	& `the mothers'
	\\

\midrule

\Aarg{}
	& māvāng
	& `mother'
	%
	& māvajang
	& `mothers'
	\\

\Parg{}
	& māvās
	& `mother' (obj.)
	%
	& māvajas
	& `mothers' (obj.)
	\\

\Dat{}
	& māvayam
	& `to the mother'
	%
	& māvajyam
	& `to the mothers'
	\\

\midrule

\Gen{}
	& māvana
	& `of the mother'
	%
	& māvayena
	& `of the mothers'
	\\
	
\Loc{}
	& māvaya
	& `at the mother'
	%
	& māvajya
	& `at the mothers'
	\\

\Caus{}
	& māvaisa
	& `due to the mother'
	%
	& māvajisa
	& `due to the mothers'
	\\

\Ins{}
	& māvari
	& `with the mother'
	%
	& māvayeri
	& `with the mothers'
	\\

\bottomrule
\end{tabu}
\label{fig:anideclvow}
\end{figure}

\begin{figure}[ht]
\caption[Declension paradigm for Ayeri \xayr{kirinF}{kirin}{street}]{Declension 
paradigm for Ayeri \xayr{kirinF}{kirin}{street} (inanimate; consonantal root)}
\begin{tabu} to \linewidth {X[1] I[2] X[4] I[2] X[4]}
\tableheaderfont\toprule

	& \multicolumn2{c}{Singular}
	& \multicolumn2{c}{Plural}
	\\

\midrule
	
\Top{}
	& kirin
	& `the street'
	%
	& kirinye
	& `the streets'
	\\

\midrule

\Aarg{}
	& kirinreng
	& `street'
	%
	& kirinyereng
	& `streets'
	\\

\Parg{}
	& kirinley
	& `street' (obj.)
	%
	& kirinyeley
	& `streets' (obj.)
	\\

\Dat{}
	& kirinyam
	& `to the street'
	%
	& kirinjyam
	& `to the streets'
	\\

\midrule

\Gen{}
	& kirinena
	& `of the street'
	%
	& kirinyena
	& `of the streets'
	\\
	
\Loc{}
	& kirinya
	& `at the street'
	%
	& kirinjya
	& `at the streets'
	\\

\Caus{}
	& kirinisa
	& `due to the street'
	%
	& kirinjisa
	& `due to the streets'
	\\

\Ins{}
	& kirineri
	& `with the street'
	%
	& kirinyeri
	& `with the streets'
	\\

\bottomrule
\end{tabu}
\label{fig:inandeclcons}
\end{figure}
~
\begin{figure}[ht]
\caption[Declension paradigm for Ayeri \xayr{per}{pera}{measure}]{Declension 
paradigm for Ayeri \xayr{per}{pera}{measure} (inanimate; vocalic root)}
\begin{tabu} to \linewidth {X[1] I[2] X[4] I[2] X[4]}
\tableheaderfont\toprule

	& \multicolumn2{c}{Singular}
	& \multicolumn2{c}{Plural}
	\\

\midrule
	
\Top{}
	& pera
	& `the measure'
	%
	& peraye
	& `the measures'
	\\

\midrule

\Aarg{}
	& perareng
	& `measure'
	%
	& perayereng
	& `measures'
	\\

\Parg{}
	& peraley
	& `measure' (obj.)
	%
	& perayeley
	& `measures' (obj.)
	\\

\Dat{}
	& perayam
	& `to the measure'
	%
	& perajyam
	& `to the measures'
	\\

\midrule

\Gen{}
	& perana
	& `of the measure'
	%
	& perayena
	& `of the measures'
	\\
	
\Loc{}
	& peraya
	& `at the measure'
	%
	& perajya
	& `at the measures'
	\\

\Caus{}
	& peraisa
	& `due to the measure'
	%
	& perajisa
	& `due to the measures'
	\\

\Ins{}
	& perari
	& `with the measure'
	%
	& perayeri
	& `with the measures'
	\\

\bottomrule
\end{tabu}
\label{fig:inandeclvow}
\end{figure}

\subsection{Gender}
\index{gender|(}

\begin{figure}[hb]
\caption{Grammatical genders in Ayeri}\centering
\begin{forest}
where n children=0{tier=word}{}
[grammatical gender
	[animate
		[masculine]
		[feminine]
		[neuter]
	]
	[inanimate]
]
\end{forest}
\label{fig:gramgend}
\end{figure}

Grammatical gender in Ayeri consists of two tiers which are subdivided into 
four classes based on a mixture of semantic and epistemic properties, see 
\autoref{fig:gramgend}. The animate gender refers, broadly speaking, to entities 
that are considered alive or are closely associated with living entities, such 
as events, concepts, or activities executed by living things. The `masculine' 
and `feminine' subcategories are applied to humans, animals whose sex is known 
(for example on behalf of breeding them or keeping them as pets), and 
gods---basically anything that shows sexual dimorphism or is assumed to be an 
exponent of it as well as nouns referring to such entities in a functional way, 
for instance, \xayr{bdnF}{badan}{father} and \xayr{maav}{māva}{mother}. The 
remainder falls into the `neuter' category---plants, for instance, body parts, 
or animals whose sex is unknown. The `inanimate' category typically contains 
tools or materials. Furthermore, animals and plants change their category to 
inanimate as well if they serve as food. There are exceptions to either group, 
where elements appear in them for no obviously discernable reason. In order to 
illustrate, here are a few examples for each category:

\pex
	\a Animate masculine:\\
		\xayr{\larger bdnF}{badan}{father}, 
		\xayr{\larger netu}{netu}{brother}, 
		\xayr{\larger AguynF}{aguyan}{rooster}, 
		\rayr{\larger AgYaanF}{Ajān}, 
		\rayr{\larger ltunF}{Latun};
		% FIXME: bull? stallion? dog?
	
	\a Animate feminine:\\
		\xayr{\larger maav}{māva}{mother}, 
		\xayr{\larger kin}{kina}{sister}, 
		\xayr{\larger Aguvj}{aguvay}{hen}, 
		\rayr{\larger mh}{Maha}, 
		\rayr{\larger tFraanj}{Trānay};
		% FIXME: cow? mare? bitch?
	
	\a Animate neuter:\\
		\xayr{\larger AdNF}{adang}{palm tree},
		\xayr{\larger bino}{bino}{color},
		\xayr{\larger IkmF}{ikam}{deer},
		\xayr{\larger kdaanF}{kadān}{harvest},
		\xayr{\larger tYaanF}{cān}{love},
		\xayr{\larger nN}{nanga}{house}, 
		\xayr{\larger tMpu}{tampu}{luck},
		\xayr{\larger yil}{yila}{foot};
	
	\a Inanimate:\\
		\xayr{\larger AhlF}{ahal}{sand},
		\xayr{\larger hem}{hema}{egg},
		\xayr{\larger khnF}{kahan}{spear},
		\xayr{\larger meluNF}{melung}{yogurt},
		\xayr{\larger nusaanF}{nusān}{damage},
		\xayr{\larger pyutaanF}{payutān}{mathematics}.
\xe

There are also a number of doublets like French \fw{le livre} `the book' and 
\fw{la livre} `the pound', for instance, \ayr{bnnF} \fw{banan} (an.) `kindness, 
charity' or \ayr{bino} \fw{bino} (an.) `color' on the one hand, and 
\ayr{bnnF} \fw{banan} (inan.) `quality' or \ayr{bino} \emph{bino} (inan.) 
`paint' on the other. Gender is reified by case marking as well as verb 
agreement; it is not possible to read the gender of a noun from its phonological 
makeup. The following example illustrates differences in case marking and 
agreement (inherent information on grammatical features underneath the NPs):

\pex
\a\label{ex:gender1}\begingl
	\gla Ang konja badan hemaley. //
	\glb Ang kond-ya badan-Ø hema-ley //
	\glc {} {} {\tiny [\TsgM{}.\An{}}] {\tiny [\TsgI{}]} //
	\glc \AgtT{}.\An{} eat-\TsgM{}.\An{} father-\Top{} egg-\PargI{} //
	\glft `Father eats an egg.' //
\endgl

\a\label{ex:gender2}\begingl
	\gla Sa tombara kahanreng burang. //
	\glb Sa tomb-ara kahan-reng burang-Ø //
	\glc {} {} {\tiny [\TsgI{}]} {\tiny [\TsgN{}.\An{}]} //
	\glc \PatT{}.\An{} kill-\TsgI{} spear-\AargI{} animal-\Top{} //
	\glft `The animal, the spear kills it.' //
\endgl

\xe

In example (\ref{ex:gender1}), the noun in the agent NP, 
\xayr{bdnF}{badan}{father}, bears the features \textsc{[+\,animate, 
+\,masculine]}, which triggers the animate agent topic agreement marker 
\rayr{ANF}{ang} on the verb, since the agent NP is also topicalized. The verb 
also agrees in person and number with the agent NP by way of the person marker 
\rayr{/y}{-ya} for third person singular masculine. The object of the sentence, 
\xayr{hem}{hema}{egg}, on the other hand bears the feature 
\textsc{[–\,animate]}, so it receives the inanimate patient case marker 
\rayr{/lej}{-ley} rather than its animate counterpart \rayr{/AsF}{-as}.

In (\ref{ex:gender2}), on the other hand, we see an inanimate agent, 
\xayr{khnF}{kahan}{spear}, so the verb receives the marker \rayr{/Ar}{-ara} for 
third person singular inanimate rather than its animate neuter counterpart 
\rayr{/yo}{-yo}. That the agent of the clause is inanimate is also shown by the 
(non-topicalized) NP's case marking: \rayr{khnF}{kahan} carries the marker 
\rayr{/reNF}{-reng}, which marks it as an inanimate agent. The object of the 
sentence, \xayr{burNF}{burang}{animal}, is also the topic, hence topic agreement 
on the verb uses the marker \rayr{s}{sa} according to the NP being animate, 
rather than its inanimate counterpart \rayr{le}{le}.

\index{gender|)}

\subsection{Number}
\index{number|(}

Ayeri only distinguishes singular and plural in nouns, which receive plural 
marking; verbs, then, agree with agent NPs in number in the canonical case. 
Ordinarily, nouns in Ayeri are countable, however, there is also a group of 
uncountable nouns as well as a (small) group of nouns which are always plural. 
As above, I will list a few words from each group in the following example:

\pex
	\a Countable nouns:\label{ex:plurals}\\[0.5\baselineskip]
		\makebox[7em][l]{\xayr{\larger AgYmF}{ajam}{toy}}
			\xayr{\larger AgYmFye}{ajamye}{toys}, %
				\\[0.5\baselineskip]
		\makebox[7em][l]{\xayr{\larger devo}{devo}{head}}
			\xayr{\larger devoye}{devoye}{heads}, %
				\\[0.5\baselineskip]
		\makebox[7em][l]{\xayr{\larger InunF}{inun}{fish}}
			\xayr{\larger InunFye}{inunye}{fish} (pl.),%
				\\[0.5\baselineskip]
		\makebox[7em][l]{\xayr{\larger netu}{netu}{brother}}
			\xayr{\larger netuye}{netuye}{brothers};
	
	\a Uncountable nouns:\\
		\xayr{\larger AhlF}{ahal}{sand}, 
		\xayr{\larger bkj}{bakay}{stuff}, 
		\xayr{\larger ghaanF}{gahān}{hope}, 
		\xayr{\larger miNnF}{mingan}{ability};
	
	\a Plurale tantum nouns:\\
		\xayr{\larger burNF}{burang}{lifestock, cattle},\footnotemark~%
		\xayr{\larger gneNnF}{ganengan}{siblings}, 
		\xayr{\larger kejnmF}{keynam}{people}, 
		\xayr{\larger tNF}{tang}{ears}.
\xe

\footnotetext{Specifically in this meaning; \rayr{burNF}{burang} can also simply 
mean `animal', in which case there is a plural form 
\xayr{burNFye}{burangye}{animals}.}

Most concrete things that exist as clearly separate entities are countable, 
also, for instance, animals and lifestock---fish, deer, sheep etc. are thus 
countable, unlike in English; pants, pliers, scissors, glasses, etc. are by 
default singular as well. Uncountable, on the other hand, are materials in 
general or abstract concepts. There is also a number of nouns which is plural 
by default, most notably entities which often occur in groups, but there is as 
well the odd word for which there seems to be no reason to be included in this 
group, for instance, \xayr{\larger bino}{bino}{paint}, and \xayr{\larger 
giMbj}{gimbay}{sorrows}. A few body parts are also plurale tantum nouns, 
especially those which occur in pairs (\xayr{niv}{niva}{eye} is a notable 
exception).

As demonstrated in (\ref{ex:plurals}), the noun plural marker is 
\rayr{/ye}{-ye}, which in native orthography also occurs in the variant 
\ayr{*Ye} or \ayr{ʲ*e}. As described above (\autoref{pluralmorph}, 
p.~\pageref{pluralmorph}), the plural marker may also be reduced to [dʒ] 
\orth{-j} before case suffixes that begin with a vowel other than /e/ or /j/, 
like \rayr{/ANF}{-ang} (\Aarg{}) or \rayr{/ymF}{-jam} (\Dat{}):

\pex
	\a \rayr{\larger dirnFANF}{diranang} (uncle-\Aarg{})
		+ \rayr{\larger /ye}{-ye} (\Pl{}) %\\[0.5\baselineskip]
		→ \rayr{\larger dirnFye\_aNF}{diranjang} (uncle-\Pl{}-\Aarg{}),
	\a \rayr{\larger dirnen}{diranena} (uncle-\Gen{})
		+ \rayr{\larger /ye}{-ye} (\Pl{}) %\\[0.5\baselineskip]
		→ \rayr{\larger dirnFyen}{diranyena} (uncle-\Pl{}-\Gen{}),
	\a \rayr{\larger dirnFymF}{diranyam} (uncle-\Dat{})
		+ \rayr{\larger /ye}{-ye} (\Pl{}) %\\[0.5\baselineskip]
		→ \rayr{\larger dirnFyeymF}{diranjyam} (uncle-\Pl{}-\Dat{}).
\xe

For pluralia tantum, to express a singular entity, it is always possible to 
use a genitive phrase like \xayr{—/En menF}{…-ena men}{one of …} (…-\Gen{} 
one), for instance:

\pex
\a\begingl
	\gla Nupayon tangang nā. //
	\glb Nupa-yon tang-ang nā //
	\glc hurt-\TplN{} ears-\Aarg{} \Fsg{}.\Gen{} //
	\glft `My ears hurt.' //
\endgl

\a\begingl
	\gla Na nupareng tang nā men. //
	\glb Na nupa-reng tang-Ø nā men //
	\glc \GenT{} hurt-\TsgI{}.\Aarg{} ears-\Top{} \Fsg{}.\Gen{} one //
	\glft `One of my ears, it hurts.' //
\endgl
\xe

Number in nouns can also be manipulated by quantifiers which attach to declined 
nouns as suffixes. In this case, when plurality is indicated by the 
quantifier, the noun is not additionally marked for number; the verb, however, 
keeps agreeing in number:

\pex
\a\begingl
	\gla Ajayon ganjang kivo. //
	\glb Aja-yon gan-ye-ang kivo //
	\glc play-\TsgN{} child-\Pl{}-\Aarg{} small //
	\glft `The small children are playing.' //
\endgl
	
\a\begingl
	\gla Ajayon ganang-ikan kivo. //
	\glb Aja-yon gan-ang-ikan kivo. //
	\glc play-\TsgN{} child-\Aarg{}=many small //
	\glft `Many small children are playing.' //
\endgl

\xe

\index{number|)}

\subsection{Case}
\index{cases|(}

As demonstrated in the declension tables at the beginning of this section, 
Ayeri's NPs are marked for case, which is governed by the verb. Since Ayeri 
uses a split alignment system with some additional complications, it is not 
very straightforward, in my opinion, to use the classical labels of 
nominative (S/A) and accusative (O), or of absolutive (S/P) and ergative (O) 
for the first two core roles. Hence, I will be using the terms `agent' and 
`patient', which I hope brings about some more clarity, especially when 
discussing the mentioned complications later on.


\subsubsection{Agent}
\index{cases!agent|(}

To quote \citet{fillmore1968}, what I call `agent' here is  
\textcquote[46]{fillmore1968}{the case of the typically animate perceived 
instigator of the action identified by the verb}. \citeauthor{fillmore1968} 
himself qualifies this definition, however, in that the \textcquote[46, 
footnote 31]{fillmore1968}{escape qualification `typically' expresses my 
awareness that contexts which I will say require agents are sometimes occupied 
by `inanimate' nouns like robot or `human institution' nouns like nation}. 
\citet{payne1997} summarizes about prototypical agents with regards to 
their topicality that a \textcquote[151]{payne1997}{less technical way of 
expressing this fact is to say that people identify with and like to talk about 
things that act, move, control events, and have power}.

Agents in Ayeri frequently embody the properties quoted by both 
\citeauthor{fillmore1968} and \citeauthor{payne1997} in this regard, including 
\citeauthor{fillmore1968}'s caveat. However, importantly, `agent' in Ayeri is a 
macrorole that may be applied to, for instance, instruments, experiencers, and 
less typical actors as well, namely, in absence of more prototypical 
candidates for agenthood in a sentence. It thus comes very close to a 
nominative, except that it does not need to be locus of the sentence's 
topic---although it very typically is, as \citet[151]{payne1997} goes on to 
note.\footnote{This is the main reason I spoke of `complications' above: Ayeri's 
notion of `subject' is somewhat problematic due to topicalization, which is why 
I try to avoid terminology associated with the nominative--accusative alignment 
system.} Thus, the first NP after the verb in all of the following examples is 
treated as an agent; the agent is marked by the suffix \rayr{/ANF}{-ang} for 
animate referents and the suffix \rayr{/reNF}{-reng} for inanimate referents; 
names and verbal topic agreement are marked by \rayr{ANF}{ang} and 
\rayr{ENF}{eng}, respectively:

\pex
\a\begingl
	\gla \textbf{Ang} tinkaya \textbf{{}} \textbf{Yan} kunangley. //
	\glb \textbf{Ang} tinka-ya \textbf{Ø} \textbf{Yan} kunang-ley //
	\glc \textbf{\AgtT{}} open-\TsgM{} \textbf{\Top{}} \textbf{Yan} 
		door-\PargI{} //
	\glft `Yan opens the door.' //
\endgl

\a\begingl
	\gla Le tinkaya \textbf{ayonang} kunang. //
	\glb Le tinka-ya \textbf{ayon-ang} kunang-Ø //
	\glc \PatT{} open-\TsgM{} \textbf{man-\Aarg{}} door-\Top{} //
	\glft `The door is opened by a/the man,' or: \\
		`The door, a/the man opens it.' //
\endgl

\a\begingl
	\gla \textbf{Eng} tinkāra \textbf{tinkay} kunangley. //
	\glb \textbf{Eng} tinka-ara \textbf{tinkay-Ø} kunang-ley //
	\glc \textbf{\AgtTI{}} open-\TsgI{} \textbf{key-\Top{}} door-\PargI{} //
	\glft `The key opens the door.' //
\endgl

\a\begingl
	\gla Tinkāra \textbf{kunangreng}. //
	\glb Tinka-ara \textbf{kunang-reng} //
	\glc open-\TsgI{} \textbf{door-\AargI{}} //
	\glft `The door opens.' //
\endgl

\a\begingl
	\gla Sā tinkaya \textbf{ang} \textbf{Yan} kunangley yan. //
	\glb Sā tinka-ya \textbf{ang} \textbf{Yan} kunang-ley yan.Ø //
	\glc \CauT{} open-\TsgM{} \textbf{\Aarg{}} \textbf{Yan} door-\PargI{} 
		\TsgM{}.\Top{} //
	\glft `They make Yan open a/the door,' or: \\
		`Because of them, Yan opens the door.' //
\endgl

\xe

In predicative constructions, the constituent which a quality is assigned to or 
about which a judgement is made is also assigned the agent case:

\pex
\a\begingl
	\gla \textbf{Tinkayreng} tado. //
	\glb \textbf{Tinkay-reng} tado //
	\glc \textbf{key-\AargI{}} old //
	\glft `The key is old.' //
\endgl

\a\begingl
	\gla \textbf{Ang} \textbf{Yan} nimpayās ban. //
	\glb \textbf{Ang} \textbf{Yan} nimpaya-as ban //
	\glc \textbf{\Aarg{}} \textbf{Yan} runner-\Parg{} good //
	\glft `Yan is a good runner.' //
\endgl

\xe

\index{cases!agent|)}

With regards to constituents' roles in ditransitive verb frames, donors are 
represented by agents in Ayeri as well, since they are the origin of whatever 
is conceptually passed on to the recipient party:

\ex\begingl
	\gla Le ilya \textbf{ang} \textbf{Yan} tinkay yam Cānlay. //
	\glb Le il-ya \textbf{ang} \textbf{Yan} tinkay-Ø yam Cānlay //
	\glc \PatT{} give-\TsgM{} \textbf{\Aarg{}} \textbf{Yan} key-\Top{} 
		\Dat{} Cānlay //
	\glft `The key, Yan gives it to Cānlay.' //
\endgl\xe

\subsubsection{Patient}
\index{cases!patient|(}

Patients are less of a definitional problem than agents in Ayeri, since in 
transitive sentences, they are very typically undergoers, that is, the 
constituent that is acted on or affected by the action expressed by the verb. 
This is the default case assigned to objects---but also to predicative 
nominals. In ditransitive sentences, the theme is represented by the patient. 
Animate patients are marked with \rayr{/AsF}{-as}, inanimate ones with 
\rayr{/lej}{-ley}; for names and verbal topic agreement, the markers are 
\rayr{s}{sa} and \rayr{le}{le}, respectively:

\pex
\a\begingl
	\gla Ang silvye {} Briha \textbf{sa} \textbf{Taryan}. //
	\glb Ang silv-ye Ø Briha \textbf{sa} \textbf{Taryan} //
	\glc \AgtT{} see-\TsgF{} \Top{} Briha \textbf{\Parg{}} \textbf{Taryan}//
	\glft `Briha sees Taryan.' //
\endgl

\a\begingl
	\gla \textbf{Sa} manye ang Briha \textbf{{}} \textbf{Taryan}. //
	\glb \textbf{Sa} man-ye ang Briha \textbf{Ø} \textbf{Taryan} //
	\glc \textbf{\PatT{}} greet-\TsgF{} \Aarg{} Briha \textbf{\Top{}} 
		\textbf{Taryan} //
	\glft `Taryan is greeted by Briha,' or:\\
		`Taryan, Briha greets him.' //
\endgl

\xe

\pex~
\a\begingl
	\gla Ang rimaye {} Briha \textbf{kunangley} //
	\glb Ang rima-ye Ø Briha \textbf{kunang-ley} //
	\glc \AgtT{} close-\TsgF{} \Top{} Briha \textbf{door-\PargI{}} //
	\glft `Briha closes a/the door.' //
\endgl

\a\begingl
	\gla \textbf{Le} rimaye ang Briha \textbf{kunang} //
	\glb \textbf{Le} rima-ye ang Briha \textbf{kunang-Ø} //
	\glc \textbf{\PatTI{}} close-\TsgF{} \Aarg{} Briha 
		\textbf{door-\Top{}} //
	\glft `The door is closed by Briha,' or:\\
		`The door, Briha closes it.' //
\endgl

\xe

\ex~
\begingl
	\gla Ang ilya {} Taryan \textbf{koyaley} yam Kandan. //
	\glb Ang il-ya Ø Taryan \textbf{koya-ley} yam Kandan //
	\glc \AgtT{} give-\TsgM{} \Top{} Taryan \textbf{book-\PargI{}} \Dat{} 
		Kandan //
	\glft `Taryan gives Kandan a book.' //
\endgl

\xe

As the translations of the examples above show, topicalizing the patient can be 
used to create an effect similar to English's passive voice, except that the 
patient will not become marked by the agent case---this is a notable difference 
from the nominative. Even if the agent NP is omitted, the patient NP will not 
be changed to the agent case, since that would reverse the direction of action:

\ex\begingl
	\gla Manya sa Taryan. ≠ Manya ang Taryan. //
	\glb Man-ya sa Taryan ≠ Man-ya ang Taryan //
	\glc greet-\TsgM{} \Parg{} Taryan ≠ greet-\TsgM{} \Aarg{} Taryan //
	\glft `Taryan is greeted.' ≠ `Taryan greets.' //
\endgl\xe

This examples shows that the case of the NP will not change, however, the 
verb will: it now agrees with the next argument in line, the patient NP. It will 
not do so, however, if the order of arguments is just scrambled. This is to 
say that the verb does not simply agree with whichever NP follows it, even 
if it can be assumed that verb agreement in Ayeri developed along similar 
lines, which will become especially apparent in the discussion of pronouns.

\tikzstyle{every picture}+=[remember picture]
\ex[aboveglftskip=2em]\begingl
	\gla Sa manye {} Taryan ang Briha. //
	\glb Sa man-ye Ø Taryan ang Briha //
	\glc \PatT{} greet\tikz\node[na](target){-\TsgF{}}; \Top{} Taryan 
		\Aarg{} \tikz\node[na](controller){Briha}; //
	\glft `Taryan is greeted by Briha,' or:\\
		`Taryan, Briha greets him.' //
\endgl\xe
\begin{tikzpicture}[overlay]
	\coordinate [below=.25em of controller] (A);
	\coordinate [below=.75em of controller] (B);
	\coordinate [below=.75em of target] (C);
	\coordinate [below=.25em of target] (D);
	\draw [-latex] (A) -- (B) -- (C) -- (D);
	\node (label) at ($(B)!0.5!(C)$) [below] {\tiny\itshape person 
		agreement};
\end{tikzpicture}

Besides being the default case for objects, the patient case is also assigned 
to predicative nominals, by analogy with transitive sentences and in spite of 
the likening nature of the construction:

\ex\begingl
	\gla Ang Yan \textbf{nimpayās} ban. //
	\glb Ang Yan \textbf{nimpaya-as} ban //
	\glc \Aarg{} Yan \textbf{runner-\Parg{}} good //
	\glft `Yan is a good runner.' //
\endgl\xe

\index{cases!patient|)}

\subsubsection{Dative}
\index{cases!dative|(}

...

\index{cases!dative|)}

\index{cases|)}

\index{nouns|)}