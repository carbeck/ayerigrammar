\chapter{Preface}

This book constitutes my latest attempt at writing a grammar of Ayeri, a
fictional language (or \enquote{conlang}) which I have been developing since
December 2003. Getting to work on grammar writing again was triggered in the
summer of 2016 by a growing dissatisfaction with not having a central place of
documentation when the first thing people were looking for on my website was
often the previous iteration of the \tit{Ayeri Grammar}, incomplete as well as
partially inaccurate and outdated as it may have been at that point. In
addition to that, there was a class on fictional languages taught at the
University of Tübingen, Germany, that summer \autocite{buch2016ss}. Ayeri was
one of the languages chosen for students to explore and evaluate.

The student group who worked on Ayeri came to the conclusion that its
documentation is severely lacking in the description of basic elements and
assumptions, since whole chapters of the grammar had been missing to that date
(\cite[12]{boga2016}).\footnote{\xayr{kuːtnsF/IknF}{Kutānas-ikan}{thanks a lot}
to Bella Boga, Madita Breuninger, Thora Daneyko, and Martina Stama-Kirr for
their hard work on making sense of my published materials in spite of
information being scattered all over the place, as well as their providing me
with the presentation concluding their group work.} Even though the formal
documentation of Ayeri's grammar had been lying dormant between 2011 and 2016,
I had written a whole number of blog articles detailing various grammatical
issues in the meantime \autocite[Blog]{benung}. These articles have finally
been taken into consideration here.

This book is, however, not just an extension to previous documentation, but for
the most part has been rewritten completely from scratch. This goes especially
for the syntax chapter, which finally gave me an opportunity to begin
acquainting myself with this field. With this book, you are holding the result
of two years of hard work in your hands. I hope you will have as much joy
reading it as I had researching it.

\begin{flushright}\itshape\footnotesize
\DTMsavenow{currentdate}
Carsten Becker, \DTMenglishmonthname{\DTMfetchmonth{currentdate}} \the\year
\end{flushright}