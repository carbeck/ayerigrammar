% kate: word-wrap true;

\chapter{Preface}

This is my latest attempt to write a grammar of Ayeri, a fictional language 
which I have been developing since December 2003. Getting to work on grammar 
writing again was triggered by a growing dissatisfaction with not having a 
central place of documentation, when the first thing people look for on my 
website is often the grammar, incomplete as well as partially inaccurate and 
outdated as it may be. In addition to that, there was a seminar on fictional 
languages at the University of Tübingen, Germany, in the summer semester of 
2016 \autocite{buch2016ss}. Ayeri was one of the languages that was chosen for 
students to explore and evaluate.

The student group who worked on Ayeri came to the conclusion that its 
documentation is severely lacking in the description of basic elements and 
assumptions, since whole chapters of the grammar had been missing to date 
(\cite[12]{boga2016}).\footnote{\xayr{kuːtns/IknF}{Kutānas-ikan}{thanks a 
lot} to Bella Boga, Madita Breuninger, Thora Daneyko, and Martina Stama-Kirr for 
their hard work on making sense of my published materials in spite of 
information being scattered all over the place, as well as their providing me 
with the presentation concluding their group work.} This is to say that previous 
attempts of writing a full-fledged grammar of Ayeri have been incomplete due to 
loss of enthusiasm and creeping neglect.

Although the \tit{Ayeri Grammar} has so far been lying dormant for five years, 
I have written a whole number of blog articles detailing various grammatical 
issues \autocite[Blog]{benung}. These articles have been taken into 
consideration here. This grammar writing attempt is thus not only a transferral 
to a different typesetting system, but constitutes an extension to previous 
formal documentation as well.

I hope that by transferring my previous grammar writing from LibreOffice to  
\LaTeX{}, combined with using GitHub as a version control system, maintaining 
and editing will become faster, more transparent, and more elegant, since 
\LaTeX{} operates on plain text files, and version control helps in keeping 
track of changes over time.

\begin{flushright}\itshape\footnotesize
Carsten Becker\\
Marburg, \DTMdate{2016-07-18}
\end{flushright}