% kate: word-wrap true;

\chapter{Phrase structures}

While the previous chapter dealt largely with the various parts of speech and 
their various distributive and inflectional properties, the present chapter 
will elaborate on how these words combine into syntactic phrases. Since Ayeri 
is a verb-initial language, it is probably most comfortably analyzed in terms 
of Lexical-Functional Grammar \citep{bresnan2016}, since \lfg{} does not 
require complicated derivations behind the surface structure of sentences. It 
will be assumed here that, even though Ayeri is basically VSO with predicate 
and predication not adjacent to each other, it is configurational in that there 
is a VP which c-commands a number of other constituents as complements in 
transitive sentences. For an overview of the theory and its 
notational formalisms, see \citet{buttking2015}.

In principle, \lfg{} assumes that grammar operates on different structural 
levels: mainly, these are a(rgument) structure, c(onstituent) structure, and 
f(unctional) structure; other layers have been proposed by different 
researchers for different purposes \citep[862--865]{buttking2015}. 
\citet{bresnan2016} define three core design principles for \lfg{}:

\begin{description}
\item[Variability:] \textcquote[41]{bresnan2016}{The principle of variability 
states that \emph{external structures vary across languages}. The formal model 
of external structure in \lfg{} is the \emph{c-structure}, which stands for 
\enquote{constituent structure} or \enquote{categorial structure}}. 
C-structures are commonly represented by context-free phrase-structure rules; 
constituency trees are based on an extended version of \xbar{X} theory 
\citep[42]{bresnan2016}.

\item[Universality:] \textcquote[42]{bresnan2016}{The principle of universality 
states that \emph{internal structures are largely invariant across languages}. 
The formal model of internal structure in \lfg{} is the \emph{f-structure}, 
which stands for \enquote{functional structure}}. The f-structure is depicted 
as an argument-value matrix (\textsc{avm}) which maps the relations between 
`subject' (\Subj{}), `object' (\Obj{}), `predicator' (\Pred{}), etc. as 
functional abstractions of NP, VP, V, etc. \citep[42]{bresnan2016}. Verbs are 
also encoded with their \fw{a-structure} indicated. That is, which arguments 
a verb takes is formally stated \citep[15]{bresnan2016}. The f-structure lists 
semantic features associated with words, such as case (\Case{}), person 
(\Pers{}), number (\Num{}), which are abstract features and as such need not 
have morphological realization \citep[43]{bresnan2016}.

\item[Monotonicity:] \textcquote[43]{bresnan2016}{Constituent structure form is 
simply not the same in all languages [...] In \lfg{} the correspondence mapping 
between internal and external structures does not preserve sameness of form. 
Instead, \emph{it is designed to preserve inclusion relations between the 
information expressed by the external structure and the content of the internal 
structure}}. Passivization, for instance, is assumed to be a lexically 
motivated alternation in predicate structure (\Subj{} is blocked, so 
the nominative is assigned to \Obj{}, and the original \Subj{} is expressed 
by an \Adjc{}), rather than an internal derivational process 
\citep[23\psqq]{bresnan2016}.

\end{description}

To illustrate the different parallel structures in operation, 
\citet[15]{bresnan2016} give the following illustration concerning which part of 
the respective a- and c-structures corresponds (`links', `maps') to which part 
of the f-structure:

\ex\begin{tabular}[t]{l @{\quad} c}
argument (a-)structure:
& \astruct{\tikzmark{verb}verb}{\tikzmark{x}x, \tikzmark{y}y}\bigskip \\

functional (f-)strucutre:
& \tikzmark{fstruct}\begin{avm}
\[
	\quad subj \tikzmark{subj} & \[
		{\enspace}\vdots{\enspace} \\
	\]{\quad} \\
	
	\quad obj \tikzmark{obj} & \[
		{\enspace}\vdots{\enspace} \\
	\] \tikzmark{objval}{\quad} \\
	\quad \tikzmark{pred} pred & \dots \tikzmark{predval} \\
\]
\end{avm}\bigskip\\

constituent (c-)structure:
& \begin{forest} baseline
[\xbar{V}
	[\subnode{V}{V}]
	[NP \tikzmark{NP}
		[\xbar{N} \tikzmark{Nbar}]
	]
]
\end{forest}

\end{tabular}\xe

\begin{tikzpicture}[remember picture, overlay]
\draw [-latex]
	([xshift=1.5ex, yshift=-0.5ex]{pic cs:verb})
	to [out=south, in=north west]
	([yshift=1ex]{pic cs:pred});
	
\draw [-latex]
	([xshift=0.5ex, yshift=-0.5ex]{pic cs:x})
	to [out=south, in=north east]
	([yshift=1ex]{pic cs:subj});
	
\draw [-latex]
	([xshift=0.5ex, yshift=-0.75ex]{pic cs:y})
	to [out=south, in=north east]
	([yshift=1ex]{pic cs:obj});
	
\draw [-latex]
	([yshift=0.5ex]{pic cs:V})
	to [out=west, in=south west]
	([yshift=-9ex]{pic cs:fstruct});
	
\draw [-latex]
	([yshift=0.5ex]{pic cs:NP})
	to [out=east, in=east]
	([yshift=0.5ex]{pic cs:objval});
	
\draw [-latex]
	([yshift=0.5ex]{pic cs:Nbar})
	to [out=east, in=east]
	([yshift=0.5ex]{pic cs:objval});
\end{tikzpicture}

Regarding the grammatical functions distinguished, \lfg{} assumes the following 
functional hierarchy \citep[97]{bresnan2016}:

\pex\label{ex:functions}
\a Grammatical functions (GFs):\\
	$\overbrace{\Subj{} > \Obj{} > \SObj{}}^{core} > 
	\overbrace{\OblqT{} > \XCompl{}, \Compl{} > \Adjc{}}^{noncore}$
\a Nonargument/discoursive functions ($\overline{\mbox{AF}}$s/DFs):\\
	\Top{}, \Foc{}, \Adjc{}
\xe

The elements listed in (\ref{ex:functions}) will also appear in 
phrase-structure rules and c-structure trees together with arrows. These arrows 
symbolize an inheritance from the current level (↓) of the tree to the next 
(↑), so for instance, `\pass{\Subj}' means that the information on the subject 
of the next higher node is equal to the information subsumed by the current 
node.

\section{Determiner- and noun phrases}

...

\ex\begin{minipage}[t]{.5\linewidth}
DP → \anno*{D} (\anno*{NP}) \\
\end{minipage}
% \begin{forest}
% [...
% 	[\anno{\pass{\AF{}/\DF{}}}{DP}
% 		[\anno{\updown}{D}]
% 		[\anno{\updown}{NP}
% 		]
% 	]
% ]
% \end{forest}
\xe

...

\ex\begin{minipage}[t]{.5\linewidth}
NP → \anno*{N} (\anno*{AP*}) (\anno*{CP})
\end{minipage}
% \begin{forest}
% [...
% 	[\anno{\updown}{NP}
% 		[\anno{\updown}{\^P}]
% 		[\anno{\updown}{\xbar{N}}
% 			[\anno{\updown}{N}]
% 			[\anno{\updown}{\xbar{N}}
% 				[\anno{\updown}{AP}]
% 				[\anno{\updown}{CP}]
% 			]
% 		]
% 	]
% ]
% \end{forest}
\xe

...

% \section{Adjective phrases}
%
% \section{Adpositional phrases}
%
% \section{Verb phrases}
%
% \section{Complement phrases}
%
