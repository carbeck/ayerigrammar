% kate: word-wrap true;

\chapter{Phrase structures}

While the previous chapter dealt largely with the various parts of speech and 
their various distributive and inflectional properties, the present chapter 
will elaborate on how these words combine into syntactic phrases. Since Ayeri 
is a verb-initial language, it is probably most comfortably analyzed in terms 
of Lexical-Functional Grammar \citep{bresnan2016}, since \lfg{} does not 
require complicated derivations behind the surface structure of sentences. It 
will be assumed here that, even though Ayeri is basically VSO with predicate 
and predication not adjacent to each other, it is configurational in that there 
is a VP which c-commands a number of other constituents as complements in 
transitive sentences. For an overview of the theory and its 
notational formalisms, see \citep{buttking2015}.

\section{Determiner- and noun phrases}

...

\ex\begin{minipage}[t]{.5\linewidth}
DP → \anno*{D} \anno*{NP}\\
\end{minipage}
\begin{forest}
[...
	[\anno{\pass{\AF{}/\DF{}}}{DP}
		[\anno{\updown}{D}]
		[\anno{\updown}{NP}
		]
	]
]
\end{forest}
\xe

...

\ex\begin{minipage}[t]{.5\linewidth}
NP → (\anno*{\^P}) \anno*{N} (\anno*{AP*}) (\anno*{CP})
\end{minipage}
\begin{forest}
[...
	[\anno{\updown}{NP}
		[\anno{\updown}{\^P}]
		[\anno{\updown}{\xbar{N}}
			[\anno{\updown}{N}]
			[\anno{\updown}{\xbar{N}}
				[\anno{\updown}{AP}]
				[\anno{\updown}{CP}]
			]
		]
	]
]
\end{forest}
\xe

...

% \section{Adjective phrases}
%
% \section{Adpositional phrases}
%
% \section{Verb phrases}
%
% \section{Complement phrases}
%
