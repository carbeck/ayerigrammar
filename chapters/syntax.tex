% kate: word-wrap true;

\chapter{Phrase structures}

While the previous chapter dealt largely with the various parts of speech and
their various distributive and inflectional properties, the present chapter
will elaborate on how these words combine into syntactic phrases. Since Ayeri
is a verb-initial language, it is probably most comfortably analyzed in terms
of Lexical-Functional Grammar (\cite{bresnan1982}~ff.; more recently
\cite{bresnan2016}), since \Lfg{} does not require complicated derivations 
behind the surface structure of  sentences.\footnote{Passivization, for
instance, is assumed to be a lexically  motivated alternation in predicate
structure (\Sbj{} is blocked, so the  nominative is assigned to \Obj{}, and the
original \Sbj{} is expressed  by an \Adjc{}), rather than an internal 
derivational process \citep[23\psqq]{bresnan2016}.} It will be assumed here 
that, even though Ayeri is basically VSO with predicate and predication not 
adjacent to each other, it is configurational in that there is a VP which 
c-commands a number of other constituents as complements in transitive 
sentences.

In principle, \Lfg{} assumes that grammar operates on different structural 
levels: mainly, these are a(rgument) structure, c(onstituent) structure, and 
f(unctional) structure; other layers have been proposed by different 
researchers for different purposes \citep[862--865]{buttking2015}. 
\citet{bresnan2016} define three core design principles for \Lfg{}:

\begin{description}
\item[Variability:] \textcquote[41]{bresnan2016}{The principle of variability 
states that \emph{external structures vary across languages}. The formal model 
of external structure in \Lfg{} is the \emph{c-structure}, which stands for 
\enquote{constituent structure} or \enquote{categorial structure}}. 
C-structures are commonly represented by context-free phrase-structure rules; 
constituency trees are based on an extended version of X-bar theory 
\citep[42]{bresnan2016}.\footnote{The basic recursive rules of X-bar theory 
are observed:
\begin{enumerate}[nosep, leftmargin={2\footnotemargin}]
\item XP → YP, \xbar{X} (specifier rule)
\item \xbar{X} → \xbar{X}, YP (adjunct rule)
\item \xbar{X} → \xhead{X}, YP (complement rule)
\end{enumerate}

The principle of economy of expression furthermore dictates that essentially, 
trees be pruned of empty terminal nodes and non-branching preterminal nodes, 
since these do not provide structurally or semantically relevant information 
\citep[119--128]{bresnan2016}.}

\item[Universality:] \textcquote[42]{bresnan2016}{The principle of universality 
states that \emph{internal structures are largely invariant across languages}. 
The formal model of internal structure in \Lfg{} is the \emph{f-structure}, 
which stands for \enquote{functional structure}}. The f-structure is depicted 
as an argument-value matrix (\Avm{}) which maps the relations between 
`subject' (\Sbj{}), `object' (\Obj{}), `predicator' (\Pred{}), etc. as 
functional abstractions of NP, VP, V, etc. \citep[42]{bresnan2016}. Verbs 
are also presented with their \fw{a-structure} spelled out. That is, which 
arguments a verb has relations to is formally stated \citep[15]{bresnan2016}. 
The f-structure collates semantic features associated with heads of grammatical 
functions (GFs), such as case (\Case{}), person (\Pers{}), number (\Num{}), 
which are abstract features and as such need not have morphological realization 
\citep[43]{bresnan2016}.

\item[Monotonicity:] \textcquote[43]{bresnan2016}{Constituent structure form is 
simply not the same in all languages [...] In \Lfg{} the correspondence mapping 
between internal and external structures does not preserve sameness of form. 
Instead, \emph{it is designed to preserve inclusion relations between the 
information expressed by the external structure and the content of the internal 
structure}}. Due to the principle of monorepresentation, 
information distributed over different morphemes which logically 
belongs to a single grammatical function is presented in the f-structure as 
unified.

\end{description}

\begin{figure}[t]\centering
\caption[F-structure mappings]{F-structure mappings \citep[15]{bresnan2016}}

\begin{tabular}[t]{l @{\quad\quad} c}
argument (a-)structure:
& \astruct{\tikzmark{verb}verb}{\tikzmark{x}x, \tikzmark{y}y}\bigskip \\

functional (f-)strucutre:
& \tikzmark{fstruct}{\smaller\begin{avm}
\[
	\quad sbj \tikzmark{subj} & \[
		{\enspace}\vdots{\enspace} \\
	\]{\quad} \\
	
	\quad obj \tikzmark{obj} & \[
		{\enspace}\vdots{\enspace} \\
	\] \tikzmark{objval}{\quad} \\
	\quad \tikzmark{pred} pred & \dots \tikzmark{predval} \\
\]
\end{avm}}\bigskip\\

constituent (c-)structure:
& \begin{forest} baseline
[\xbar{V}
	[\subnode{V}{V}]
	[NP \tikzmark{NP}
		[\xbar{N} \tikzmark{Nbar}]
	]
]
\end{forest}
%
\end{tabular}
\begin{tikzpicture}[remember picture, overlay]
\draw [-latex]
	([xshift=1.5ex, yshift=-0.5ex]{pic cs:verb})
	to [out=south, in=north west]
	([yshift=1ex]{pic cs:pred});
	
\draw [-latex]
	([xshift=0.5ex, yshift=-0.5ex]{pic cs:x})
	to [out=south, in=north east]
	([yshift=1ex]{pic cs:subj});
	
\draw [-latex]
	([xshift=0.5ex, yshift=-0.75ex]{pic cs:y})
	to [out=south, in=north east]
	([yshift=1ex]{pic cs:obj});
	
\draw [-latex]
	([yshift=0.5ex]{pic cs:V})
	to [out=west, in=south west]
	([yshift=-10ex]{pic cs:fstruct});
	
\draw [-latex]
	([yshift=0.5ex]{pic cs:NP})
	to [out=east, in=east]
	([yshift=0.5ex]{pic cs:objval});
	
\draw [-latex]
	([yshift=0.5ex]{pic cs:Nbar})
	to [out=east, in=east]
	([yshift=0.5ex]{pic cs:objval});
\end{tikzpicture}
\label{fig:phimap}
\end{figure}

To illustrate the different parallel structures in operation, 
\citet[15]{bresnan2016} give the schema in \autoref{fig:phimap} to demonstrate
which part of the a- and c-structure respectively corresponds (`links', `maps') 
to which part of the f-structure.
% \footnote{\citet{bresnan2016} use \textsc{`subj'} for `subject'.
% ; for consistency with the above I will use 
% `\Sbj{}' in the following. I will also divergently use \Compl{} and 
% \XCompl{} for \textsc{(x)comp}, since \textsc{comp} has already been used for 
% `comparative' above.} 
Regarding the different functions distinguished, \Lfg{} 
assumes the following hierarchies \citep[97, 100]{bresnan2016}:

\pex\label{ex:functions}
\a\label{ex:gfs} Grammatical functions (GFs):\\
	$\overbrace{\Sbj{} > \Obj{} > \SObj{}}^{\text{core}} > 
	\overbrace{\OblqT{} > \XCompl{}, \Compl{} > \Adjc{}}^{\text{noncore}}$
\a\label{ex:nonafs} (Non)argument functions (AFs/$\overline{\mbox{AF}}$s):\\
	$\underbrace{\Top{}\: \Foc{}}_{\text{non-a-fns}}\; 
	\overbrace{\Sbj{}\: \Obj{}\: \SObj{}\: \OblqT{}\: \XCompl{}\: 
		\Compl{}}^{\text{a-fns}}\; 
	\underbrace{\Adjc{}}_{\text{non-a-fns}}$
\a\label{ex:dfs} Discourse functions (DFs):\\
	$\overbrace{\Top{}\: \Foc{}\: \Sbj{}}^{\text{d-fns}}\;  
	\underbrace{\Obj{}\: \SObj{}\: \OblqT{}\: \XCompl{}\: \Compl{}\: 
		\Adjc{}}_{\text{non-d-fns}}$
\xe

The elements listed in (\ref{ex:functions}) will also appear in 
phrase-structure rules and c-structure trees together with arrows. These arrows 
symbolize inheritance of feature information from the current level (↓) of the 
tree to the next (↑), so for instance, `\pass{\Sbj}' means that the information 
subsumed by the current node (`down') is passed on as the subject function of 
the next higher node (`up') in the tree. Concise information on notational 
formalisms of \Lfg{} can be found, for instance, in \citet{buttking2015}.

\section{Determiner phrases and noun phrases}
\label{sec:dps-nps}

Noun phrases (NPs), and determiner phrases (DPs) as their functional 
counterpart, fulfill the functions of subject (\Sbj{}), object (\Obj{}), 
secondary object (\SObj{}), as well as oblique location (\Oblq{loc}), and 
various adjuncts (\Adjc{}). DPs and NPs can also constitute topics (\Top{}). 
Which DP or NP receives which function is selected by the a-structure of the 
verb---this also has repercussions on case- and topic marking.

Generally, DPs and NPs can be described with the phrase-structure formulas 
given in (\ref{ex:dpnpstruct}), which list the various parts that can occur in 
them; parentheses indicate the optionality of a term, that is, the respective 
element may occur but is not constitutive.
% ; an asterisk stands for `zero or 
% more' occurrences of the respective element

\pex\label{ex:dpnpstruct}
\a\label{ex:dpdef} DP → \anno*{\xhead{D}} (\anno*{XP})
\a\label{ex:npdef} NP → \anno*{\xhead{N}} (\anno*[\pass{\Adjc}]{XP})
\xe

Since the constituent containing the various non-verbal elements in Ayeri is 
likely exocentric and constituents may move around within it in a restricted 
way, we have to assume that not constituent structure, but case marking 
identifies the grammatical functions of the various arguments of verbs. Thus, 
the following lexicocentric conditions operate on both DP and NP as exponents 
of case:

\ex\label{ex:dpnpcasemap}\labels
\begin{tabular}[t]{@{} l @{\quad} l @{$\implies$} l}
\tl\quad & \downs{\Case} = \Aarg	& \pass{\Sbj} \\
\tl\quad & \downs{\Case} = \Parg	& \pass{\Obj} \\
\tl\quad & \downs{\Case} = \Dat		& \pass{\SObj} \logor{}
						\pass{\Oblq{loc}} \\
\tl\quad & \downs{\Case} = \Gen		& \pass{\Oblq{poss}} \logor{} 
						\pass{\Oblq{loc}} \\
\tl\quad & \downs{\Case} = \Loc		& \pass{\Oblq{loc}} \\
\tl\quad & \downs{\Case} = \Caus	& \pass{\Oblq{caus}} \\
\tl\quad & \downs{\Case} = \Ins		& \pass{\Oblq{ins}} \logor{} 
						\pass{\Adjc} \\
\end{tabular}
\xe

The rules in (\ref{ex:dpnpcasemap}) determine the typical mappings between case 
marking and grammatical functions, which are not always unambiguous. As 
explained above (compare \autoref{subsec:case}), the dative case does not only 
indicate that something is done to this referent or to their benefit, but it 
may also indicate motion towards this referent. Likewise, the genitive case 
does not only indicate possession, but also origin, and motion from this 
referent. Nominal adjuncts to nouns which specify what the noun consists also 
appear in the instrumental case, besides the instrumental being used to 
indicate the means or the circumstance by which an action comes about. 
Moreover, DPs or NPs may also lack case marking, which indicates that the 
respective phrase is a part of the topic of the verb, which is what 
(\ref{ex:topicrule}) describes:

\ex\label{ex:topicrule}
¬\,\downs{\Case} $\implies$ \elem{\Top}
\xe

Instead of case marking on the DP or NP, there is a marker in front of the verb 
which provides information on the case and, if \AgtT{} or \PatT{}, also about 
the animacy of the topicalized phrase. This means that grammatical information 
about the topic of a phrase is spread over two discontinuous locations. This 
issue does not pose a problem to an \Lfg{}-based analysis, however, since both 
locations unify their information content in the f-structure feature \Top{}. 
Since information located in multiple places is jointly feeding this feature, I 
am using the annotation `\elem{\Top}' for each location rather than simple 
`\pass{\Top}'. Note that only one NP among the arguments of a verb may be the 
topic of the phrase, and a topic can only be marked if the verb is finite and 
the number of arguments to the verb is greater than one.

\subsection{Determiner phrases}
\label{subsec:dps}

The functional heads of DPs are formed by determiners. The c-structure of a DP 
corresponding to the linear description in (\ref{ex:dpdef}) is depicted in 
(\ref{ex:dpcstruct}); annotations about feature inheritance for 
c-to-f-structure conversion have been added. If the DP has a dependent phrase 
providing lexical content, it is in the form of an NP complement, whose head 
forms a lexical co-head of \xhead{D}. This NP must be annotated with `\updown{}' 
to signify that its features are inherited to DP wholesale. Alternatively, a DP 
containing a pronoun may also be modified by an NP adjunct, a possessive NP, an 
AP or a relative-clause CP, which all need to be annotated with `\pass{\Adj}', 
since they are not complements.

\ex\label{ex:dpcstruct}
\begin{forest}
[{\anno[\pass{df} \logor{} \pass{gf}]{DP}}
	[\anno{\xbar{D}}
		[\anno{\xhead{D}}]
		[{$\left(\anno[\updown{} \logor{} \pass{\Adj}]{XP}\right)$}]
	]
]
\end{forest}
\xe

Ayeri does not possess articles as such, though it does have a variety of 
morphemes which occur as clitic prefixes on nouns which can be analyzed to 
fulfill this function. These clitics are the demonstrative prefixes 
\xayr{Ed/}{eda-}{this}, \xayr{Ad/}{ada-}{that}, and \xayr{d/}{da-}{such (a)}, 
as well as the inspecificity prefix \xayr{me/}{mə-}{some} (compare 
\autoref{subsec:nounpref}). These prefixes can only be used with nouns, but not 
with personal pronouns (compare \autoref{subsec:perspro}), since personal 
pronouns as well have the status of determiners. Pronouns, as pro-forms, are in 
complementary distribution with NPs containing a noun. Due to affix-order 
considerations concerning the demonstrative prefixes, it also seems advisable 
to analyze the case markers of proper nouns as determiners. In order to capture 
the fact that an NP is facultative with the demonstrative and inspecificity 
prefixes while it must not occur with pronouns, `NP' appears in parentheses 
since a phrasal complement it is not always present. We can define the 
following morpholexic rules for determiners (\xhead{D}):

\pex
\a\label{ex:dmorphlex-dem}Demonstrative prefixes:\medskip

	\begin{tabu} to \linewidth {@{} X[15l] X[5l] X[80l]}
	\savetabu{morphlex}
	\rayr{Ed/}{eda-} (`this')
		& D
		& \begin{tabular}[t]{l l l}
% 			\ups{\Pred} & = & `this' \\
			\ups{\Pers} & \req{} & 3 \\
			\ups{\Prox} & = & $+$ \\
			\ups{\Dist} & = & $-$ \\
			\ups{\Def} & = & $+$ \\
			\ups{\Spec} & = & $+$ \\
		\end{tabular}
	\end{tabu}\medskip

	\begin{tabu} {\usetabu{morphlex}}
	\rayr{Ad/}{ada-} (`that')
		& D
		& \begin{tabular}[t]{l l l}
% 			\ups{\Pred} & = & `that' \\
			\ups{\Pers} & \req{} & 3 \\
			\ups{\Prox} & = & $-$ \\
			\ups{\Dist} & = & $+$ \\
			\ups{\Def} & = & $+$ \\
			\ups{\Spec} & = & $+$ \\
		\end{tabular}
	\end{tabu}\medskip
	
	\begin{tabu} {\usetabu{morphlex}}
	\rayr{d/}{da-} \mbox{(`such a')}
		& D
		& \begin{tabular}[t]{l l l}
% 			\ups{\Pred} & = & `such' \\
			\ups{\Pers} & \req{} & 3 \\
			\ups{\Def} & = & $-$ \\
			\ups{\Spec} & = & $-$ \\
		\end{tabular}
	\end{tabu}
	
\needspace{3\baselineskip}
\a Inspecificity prefix:\medskip

	\begin{tabu} {\usetabu{morphlex}}
	\rayr{me/}{mə-} (`whichever')
		& D
		& \begin{tabular}[t]{l l l}
			\ups{\Def} & = & $-$ \\
			\ups{\Spec} & = & $-$ \\
			\ups{\Pers} & \req{} & 3 \\
		\end{tabular}
	\end{tabu}

\needspace{3\baselineskip}
\a Personal pronouns:\medskip

	\begin{tabu} {\usetabu{morphlex}}
	(various)
		& N
		& \begin{tabular}[t]{l l l}
			\ups{\Pred} & = & `pro' \\
			\ups{\Pers} & = & \{\First{}, \Second{}, \Third{}\} \\
			\ups{\Refl} & = & $\pm$ \\
			\ups{\Num} & = & \{\Sg{}, \Pl{}\} \\
			\ups{\Gend} & = & \{\M{}, \F{}, \N{}\} \\
			\ups{\Anim} & = & $\pm$ \\
			\ups{\Case} & = & \{\Aarg{}, \Parg{}, \Dat{}, \Gen{}, 
				\Loc{}, \Ins{}, \Caus{}\} \\
			\ups{\Def} & = & $+$ \\
			\ups{\Spec} & = & $+$ \\
		\end{tabular}
	\end{tabu}
	
\needspace{3\baselineskip}
\a Demonstrative pronouns:\medskip

	\begin{tabu} {\usetabu{morphlex}}
	(various)
		& N
		& \begin{tabular}[t]{l l l}
			\ups{\Pred} & = & `pro' \\
			\ups{\Pers} & = & 3 \\
			\ups{\Prox} & = & $\pm$ \\
			\ups{\Dist} & = & $\pm$ \\
			\ups{\Def} & = & $\pm$ \\
			\ups{\Spec} & = & $+$ \\
			\ups{\Anim} & = & $\pm$ \\
			\ups{\Case} & = & \{\Aarg{}, \Parg{}, \Dat{}, \Gen{}, 
				\Loc{}, \Ins{}, \Caus{}\} \\
		\end{tabular}
	\end{tabu}
	
\needspace{3\baselineskip}
\a\label{ex:dmorphlex-propn}Case markers of proper nouns:\medskip	
	
	\begin{tabu} {\usetabu{morphlex}}
	(various)
		& D
		& \begin{tabular}[t]{l l l}
% 			\ups{\Pers} & \req{} & \Third \\
% A 3rd person must not be *required* if the topic particle is in fact a D head 
% in Spec NP, since pronouns as well can serve as topics without being 3rd 
% persons!
			\ups{\Anim} & \req{} & $\pm$ \\
			\ups{\Case} & = & \{\Aarg{}, \Parg{}, \Dat{}, \Gen{}, 
				\Loc{}, \Ins{}, \Caus{}\} \\
		\end{tabular}
	\end{tabu}\medskip
	
\xe

Demonstrative prefixes, the inspecificity prefix \rayr{me/}{mə-}, as well as 
proper-noun case markers cannot be without an NP complement, which is what 
`\req{}' is supposed to express in the feature specification: the demonstrative 
prefixes require that an element exist which encodes a third-person referent to 
are assumed to encode third person by default.\footnote{A simple `=' 
\emph{defines} a value; a subscript `c' indicates that the morpheme 
\emph{requires} that this value be present. Thus `\req{}' expresses a 
\fw{constraining} equation \citep[59--61]{bresnan2016}.} On the other hand, 
personal and demonstrative pronouns inherently define information on person 
(\Pers{}). Since there is a great number of both personal and demonstrative 
pronouns, only the various values they can assume are indicated, without 
ensuring that the combinations are actually possible (compare 
\autoref{subsec:perspro}). Personal pronouns, for instance, only distinguish 
\Inan{} as set against $\{\M{}, \F{}, \N{}\}$, which are subgroups of \An{}. 
Gender is also not distinguished in all persons, but only in the third.

\subsection{Noun phrases}

Regarding NPs proper, it is necessary to distinguish morphologically between 
those containing common nouns and those containing proper nouns (that is,  
names), as we have seen in the previous chapter (compare \autoref{sec:nouns}): 
common nouns indicate case by a suffix, while proper nouns receive case marking 
by a particle preceding the noun (see \autoref{subsec:dps} and below). Nouns 
can also be modified by adjectives, adverbs, adjunct nouns, or relative 
clauses, which are typically following the noun. The c-structure of NPs can be 
assumed to look like given in (\ref{ex:npcstruct}). Since modifiers of 
\xhead{N} (\xhead{N}'s sister node) can consist of different phrase types (AP, 
NP, DP, CP), they are indicated generalized as `XP' in the diagrams.

\ex\label{ex:npcstruct}\labels
\begin{minipage}[t]{.5\linewidth}%
\tl\quad Common nouns:\\

\quad\begin{forest}
[{\anno[\pass{df} \logor{} \pass{gf} \logor{} \updown]{NP}}
	[\anno{\xbar{N}}
		[\anno{\xhead{N}}
			[\anno{N\tsub{stem}}]
			[\anno{\mbox{N\tsub{infl}}}]
		]
		[{$\left(\anno[\pass{\Adjc}]{XP}\right)$}]
	]
]
\end{forest}
\end{minipage}
%
\begin{minipage}[t]{.5\linewidth}%
\tl\quad Proper nouns:\\

\quad\begin{forest} shorter edges,
[{\anno[\pass{df} \logor{} \pass{gf}]{DP}}
	[\anno{\xbar{D}}
		[\anno{\xhead{D}}]
		[{\anno[\updown]{NP}}
			[\anno{\xbar{N}}
				[\anno{\xhead{N}}]
				[{$\left(\anno[\pass{\Adjc}]{XP}\right)$}]
			]
		]
	]
]
\end{forest}
\end{minipage}
\xe

In the following, I will again give a list of morpholexic specifications 
which give an overview of the different semantic and morphological features 
nouns provide (also compare \autoref{sec:nouns}):

\needspace{3\baselineskip}
\pex
\a Common nouns:\\

	\begin{tabu} {\usetabu{morphlex}}
	...
		& N
		& \begin{tabular}[t]{l l l}
			\ups{\Pred} & = & `...' \\
			\ups{\Pers} & = & 3 \\
			\ups{\Num} & = & \{\Sg{}, \Pl{}\} \\
			\ups{\Gend} & = & \{\M{}, \F{}, \N{}\} \\
			\ups{\Anim} & = & $\pm$ \\
			\ups{\Case} & = & \{\Aarg{}, \Parg{}, \Dat{}, \Gen{}, 
				\Loc{}, \Ins{}, \Caus{}\} \\
		\end{tabular}
	\end{tabu}

\a Proper nouns:\\

	\begin{tabu} {\usetabu{morphlex}}
	...
		& N
		& \begin{tabular}[t]{l l l}
			\ups{\Pred} & = & `...' \\
			\ups{\Pers} & = & 3 \\
			\ups{\Num} & = & \{\Sg{}, \Pl{}\} \\
			\ups{\Gend} & = & \{\M{}, \F{}, \N{}\} \\
			\ups{\Anim} & = & $\pm$ \\
		\end{tabular}
	\end{tabu}

\xe

Generally, both common and proper nouns provide information about person, 
number, gender, and animacy. As mentioned above, proper nouns require a 
particle to precede them which carries information on case. Case markers of 
nouns distinguish animacy for \Aarg{} and \Parg{} and are thus required to be 
consistent with the noun's animacy feature:\footnote{For clarity, I will only 
list the features relevant to the current discussion.}

\ex\labels%
\begin{minipage}[t]{.5\linewidth}
\tl\label{ex:animok}\quad %
\begin{forest}
[{%
\xhead{N} \\
\ups{\Anim} = $+$ \\
\ups{\Case} = \Parg{}
}
	[N\tsub{stem}
		[{%
			gan \\
			\ups{\Anim} = $+$ \\
		}]
	]
	[-N\tsub{infl}
		[{%
			-as \\
			\ups{\Anim} = $+$ \\
			\ups{\Case} = \Parg{} \\
		}]
	]
]
\end{forest}
\end{minipage}
\begin{minipage}[t]{.5\linewidth}
\tl\label{ex:animclash}\quad %
\ljudge*\begin{forest}
[{%
\xhead{N} \\
\ups{\Anim} = \err{} \\
\ups{\Case} = \Parg{}
}
	[N\tsub{stem}
		[{%
			gan \\
			\ups{\Anim} = $+$ \\
		}]
	]
	[N\tsub{infl}
		[{%
			-ley \\
			\ups{\Anim} = $-$ \\
			\ups{\Case} = \Parg{} \\
		}]
	]
]
\end{forest}
\end{minipage}
\xe

Example (\ref{ex:animok}a) shows a well-formed construction: the noun, 
\xayr{gnF}{gan}{child}, is animate, hence the case particle also needs to be 
animate---the case particle must thus be \rayr{/As}{-as} to be coherent. In 
contrast to this, example (\ref{ex:animclash}b) is not well-formed in that the 
noun is animate but the case particle, \rayr{/lej}{-ley}, signals that 
it is inanimate: the \Anim{} values of the noun stem and its suffix 
clash and cannot be conclusively unified for \xhead{N} itself. The 
same principle of coherence is, of course, also true for proper nouns, which 
receive a case-marking particle. These are defined in (\ref{ex:dmorphlex-propn}) 
as demanding that a noun with the correct animacy value be within their scope.

As previously mentioned, nouns may be modified by a number of adjuncts of
various phrase types: adjectives and adverbs (AP), genitive attributes (NP/DP), 
other nominal adjuncts (NP), and relative clauses (CP):

\pex
\a\begingl
	\glpreamble With an adjective: //
	\gla Ya manga nu-nunāran \textbf{baloyyereng} \textbf{tiru} venya 
		satitay. //
	\glb Ya manga nu\til{}nuna-aran baloy-ye-reng tiru ven-ya satitay-Ø //
	\glc \LocT{} \Prog{} \Iter{}\til{}fly-\TplI{} leaf-\Pl{}-\AargI{} dry 
		air-\Loc{} winter-\Top{} //
	\glft `In winter, dry leaves are flying around in the air.' //
\endgl
\medskip

	\begin{forest}, shorter edges
	[{\anno[\pass{\Sbj}]{NP}}
		[\anno{\xbar{N}}
			[\anno{\xhead{N}}
				[baloyyereng]
			]
			[{\anno[\pass{\Adjc}]{AP}}
				[{tiru}, roof]
			]
		]
	]
	\end{forest}
	\quad
	\begin{avm}
	\[
		\Sbj	& \[
			\Pred	& `leaf' \\
% 			\Pers	& \Third \\
			\Num	& \Pl \\
			\Anim	& $-$ \\
			\Case	& \Aarg \\
			\Adjc	& \{\[
				\Pred	& `dry' \\
				\]\} \\
			\] \\
	\]
	\end{avm}

\a\begingl
	\glpreamble With a genitive attribute: //
	\gla Adanyāng \textbf{pacanas} \textbf{na} \textbf{Tuvo}. //
	\glb Adanya-ang pacan-as na Tuvo //
	\glc that.one-\Aarg{} cousin-\Parg{} \Gen{} Tuvo //
	\glft `That is Tuvo's cousin.' //
\endgl
\medskip

	\begin{forest}, shorter edges
	[{\anno[\pass{\XCompl}]{NP}}
		[\anno{\xbar{N}}
			[\anno{\xhead{N}}
				[pacanas]
			]
			[{\anno[\pass{\Adjc}]{NP}}
				[{na Tuvo}, roof]
			]
		]
	]
	\end{forest}
	\quad
	\begin{avm}
	\[
		\XCompl	& \[
			\Pred	& `cousin' \\
% 			\Pers	& \Third \\
			\Num	& \Sg \\
			\Anim	& $+$ \\
			\Case	& \Parg \\
			\Adjc	& \{ \Oblq{poss} & \[
					\Pred	& `Tuvo' \\
% 					\Pers	& \Third \\
% 					\Anim	& $+$ \\
					\Case	& \Gen \\
					\] \\
				\} \\
			\] \\
	\]
	\end{avm}

\a\begingl
	\glpreamble With a noun adjunct: //
	\gla Le pegaya sinyāng \textbf{kasu} \textbf{bariri}? //
	\glb Le pega-ya sinya-ang kasu-Ø bari-ri //
	\glc \PargI{} steal-\TsgM{} who-\Aarg{} basket-\Top{} meat-\Ins{} //
	\glft `The basket of meat, who stole it?' //
\endgl
\medskip

	\begin{forest}, shorter edges
	[{\anno[\elem{\Top}]{NP}}
		[\anno{\xbar{N}}
			[\anno{\xhead{N}}
				[kasu]
			]
			[{\anno[\pass{\Adjc}]{NP}}
				[{bariri}, roof]
			]
		]
	]
	\end{forest}
	\quad
	\begin{avm}
	\[
		\Top	& \[
			\Pred	& `basket' \\
% 			\Pers	& \Third \\
			\Num	& \Sg \\
			\Anim	& $-$ \\
			\Case	& \Parg \\
			\Adjc	& \{ \Oblq{ins} & \[
					\Pred	& `meat' \\
% 					\Pers	& \Third \\
% 					\Anim	& $-$ \\
					\Case	& \Ins \\
					\] \\
				\} \\
			\] \\
	\]
	\end{avm}

\a\begingl
	\glpreamble With a relative clause: //
	\gla Ang konja \textbf{seygoley} \textbf{si} \textbf{tuvo} \textbf{nay} 
		\textbf{sepra}. //
	\glb Ang kond=ya.Ø seygo-ley si tuvo nay sepra //
	\glc \AgtT{} eat=\Fsg{}.\Top{} apple-\PargI{} \Rel{} red and crunchy //
	\glft `I ate an apple which was red and crunchy.' //
\endgl
\medskip

	\begin{forest}, shorter edges
	[{\anno[\pass{\Obj}]{NP}}
		[\anno{\xbar{N}}
			[\anno{\xhead{N}}
				[seygoley]
			]
			[{\anno[\pass{\Adjc}]{CP}}
				[{si tuvo nay sepra}, roof]
			]
		]
	]
	\end{forest}
	\quad
	\begin{avm}
	\[
		\Obj	& \[
			\Pred	& `apple' \\
% 			\Pers	& \Third \\
			\Num	& \Sg \\
			\Anim	& $-$ \\
			\Case	& \Parg \\
			\Adjc	& \{
					\[
						\Pred	& `be red ...' \\
					\] \\
				\} \\
			\] \\
	\]
	\end{avm}

\xe

Of course, it is also possible to combine these nominal modifiers. In this 
case, a genitive NP always follows an adjective and an adjunct noun, while a 
relative clause always forms the last element in an NP. For one, the relative 
grammatical weight of a relative clause is very high, whether one counts 
\textcquote[102]{wasow1997}{words, nodes, or phrasal nodes} as indicators of 
grammatical weight. Relative clauses also trail whenever possible 
since placing one before a lighter element may make an utterance hard to 
construct coherently for the speaker, and hence also hard to understand for the 
listener. The following example (\ref{ex:nounmodord}) illustrates the unmarked 
order of modifiers.

\ex\label{ex:nounmodord}
\begingl
	\gla diranang caban nā si ang mica ya Kārvisam //
	\glb diranang caban nā si ang mit=ya.Ø ya Kārvisam //
	\glc uncle-\Aarg{} favorite \Fsg{}.\Gen{} \Rel{} \AgtT{} 
		live=\TsgM{}.\Top{} \Loc{} Kārvisam //
	\glft `my favorite uncle who lives in Kārvisam' //
\endgl
\medskip

\begin{multicols}{2}
% \begin{forest} shorter edges, narrower nodes,
% [{\anno[\pass{\Sbj}]{NP}}
% 	[\anno{\xbar{N}}
% 		[\anno{\xhead{N}}
% 			[diranang]
% 		]
% 		[\anno{\xbar{N}}
% 			[{\anno[\elem{\Adj}]{AP}}
% % 				[\anno{\xhead{A}}
% 					[{caban}, roof]
% % 				]
% 			]
% 			[\anno{\xbar{N}}
% 				[{\anno[\elem{\Adj}]{DP}}
% % 					[\anno{\xhead{D}}
% 						[{nā}, roof]
% % 					]
% 				]
% 				[\anno{\xbar{N}}
% 					[{\anno[\elem{\Adj}]{CP}}
% 						[{si ang mica ...}, roof]
% 					]
% 				]
% 			]
% 		]
% 	]
% ]
% \end{forest}

\begin{forest} shorter edges, narrower nodes,
[{\anno[\pass{\Sbj}]{NP}}
	[\anno{\xbar{N}}
		[\anno{\xbar{N}}
			[\anno{\xbar{N}}
				[\anno{\xhead{N}}
					[diranang]
				]
				[{\anno[\elem{\Adj}]{AP}}
					[\anno{\xhead{A}}
						[caban]
					]
				]
			]
			[{\anno[\elem{\Adj}]{DP}}
				[\anno{\xhead{D}}
					[nā]
				]
			]
		]
		[{\anno[\elem{\Adj}]{CP}}
			[{si ang mica ...}, roof]
		]
	]
]
\end{forest}

\begin{avm}
\[
\Sbj	& \[
	\Pred	& `uncle' \\
	\Case	& \Aarg \\
	\Anim	& $+$ \\
	\Adjc	& \{
			\[
				\Pred	& `favorite' \\
			\], \\
			\[
				\Pers	& \First \\
				\Num	& \Sg \\
				\Case	& \Gen \\
			\], \\
			\[
				\Pred	& `lives in ...' \\
			\] \\
		\} \\
	\] \\
\]
\end{avm}
\end{multicols}
\xe

As the c-structure tree in (\ref{ex:nounmodord}) shows, Ayeri prefers 
head--dependent word order with exceeding consistency. As previous examples have
shown, both adjuncts and complements are, for the most part, consistently 
appended to the right of their heads, which means that Ayeri may be classified
as a rather consistently right-branching language. However, a certain number of
postpositions form an exception to this classification 
(\autoref{subsec:postpos}). In the light of word order typology, we can 
formulate the following generalizations:

\pex
\a Order of noun and adjective: N Adj
\a Order of noun and genitive: N Gen
\a Order of noun and relative clause: N Rel
\xe

\subsection{Interactions between DP and NP}
\label{subsec:dpnpinteract}

With common nouns, a determiner which requires a complement (like the
demonstrative prefixes, for instance) simply cliticizes to the next \xhead{N}
it commands, which is illustrated in (\ref{ex:det+comn}). If a proper noun is
involved, however, there need to be two \xhead{D} slots, one for the case
marker, and another for the demonstrative prefix, which (\ref{ex:det+propn})
gives an example of: \xbar{D} is simply doubled to accommodate for another
\xhead{D}. In both cases, the determiner prefix is unstressed and combines 
phonetically with the NP modifier, as shown in (\ref{ex:det+comn}). If there is
no coalescence of vowels, as in (\ref{ex:det+propn}), the prefix remains
unstressed: \rayr{Ad/YnF}{ada-Yan} [a.da.ˈjan].

\begin{multicols}{2}
\pex\label{ex:det+comn}
\a\begingl
	\gla ayonang //
	\glb ayon-ang //
	\glc man-\Aarg{} //
	\glft `the/a man' //
\endgl

\a\begingl
	\gla adāyonang //
	\glb ada=ayon-ang //
	\glc that=man-\Aarg{} //
	\glft `that man' //
\endgl

\xe

\pex~\label{ex:det+propn}
\a\begingl
	\gla ang Diyan //
	\glb ang Diyan //
	\glc \Aarg{} Diyan //
	\glft `Diyan' //
\endgl

\a\label{ex:case+det+propn}\begingl
	\gla ang ada-Diyan //
	\glb ang ada=Diyan //
	\glc \Aarg{} that=Diyan //
	\glft `that Diyan' //
\endgl

\xe

\end{multicols}

\begin{multicols}{2}
\pex~
\a \begin{minipage}[t]{\linewidth}
C-structure for (\ref{ex:det+comn}):\medskip

\quad {\begin{forest} shorter edges, baseline,
[DP
	[\xbar{D}
		[\xhead{D}
			[ada-]
		]
		[NP
			[\xhead{N}
				[ayonang]
			]
		]
	]
]
\end{forest}
}
\end{minipage}

\a \begin{minipage}[t]{\linewidth}
C-structure for (\ref{ex:det+propn}):\medskip

\quad {\begin{forest} shorter edges, baseline,
[DP
	[\xbar{D}
		[\xhead{D}
			[ang]
		]
		[\xbar{D}
			[\xhead{D}
				[ada-]
			]
			[NP
				[\xhead{N}
					[Diyan]
				]
			]
		]
	]
]
\end{forest}
}
\end{minipage}
\xe
\end{multicols}

Furthermore, as mentioned before in \autoref{subsubsec:verbprefixes}, it is 
possible to abbreviate \xayr{dnY}{danya}{such one} + adjective as 
\rayr{d/}{da-} + adjective. This affixing of the indefinite demonstrative 
essentially derives a complex anaphora. The trees charted in (\ref{ex:adjpron}) 
illustrate pronominalized adjectives and possessive pronouns.

\ex\label{ex:adjpron}\labels
\parbox[t]{.5\linewidth}{
\tl\quad\begingl
	\gla da-kivo //
	\glb da=kivo //
	\glc one=small //
	\glft `the small one' //
\endgl}%\medskip
%
\begin{tabular}[t]{@{} l @{\quad=\quad} l}
\begin{forest} shorter edges,
[DP
	[\anno{\xbar{D}}
		[\anno{\xhead{D}}
			[danya]
		]
		[{\anno[\pass{\Adj}]{AP}}
			[\anno{\xhead{A}}
				[kivo]
			]
		]
	]
]
\end{forest}

&

\begin{forest} shorter edges,
[DP
	[\anno{\xbar{D}}
		[\anno{\xhead{D}}
			[da-]
		]
		[{\anno[\pass{\Adj}]{AP}}
			[\anno{\xhead{A}}
				[kivo]
			]
		]
	]
]
\end{forest}
\end{tabular}

\parbox[t]{.5\linewidth}{
\tl\quad\label{ex:posspron}
\begingl
	\gla da-vana //
	\glb da=vana //
	\glc such=\Second{}.\Gen{} //
	\glft `yours' //
\endgl}%\medskip
%
\begin{tabular}[t]{@{} l @{\quad=\quad} l}
\begin{forest} shorter edges,
[DP
	[\anno{\xbar{D}}
		[\anno{\xhead{D}}
			[danya]
		]
		[{\anno[\pass{\Adj}]{DP}}
			[\anno{\xhead{D}}
				[vana]
			]
		]
	]
]
\end{forest}

&

\begin{forest} shorter edges,
[DP
	[\anno{\xbar{D}}
		[\anno{\xhead{D}}
			[da-]
		]
		[{\anno[\pass{\Adj}]{DP}}
			[\anno{\xhead{D}}
				[vana]
			]
		]
	]
]
\end{forest}
\end{tabular}
\xe

\rayr{d/}{da-} in this case looks like the indefinite demonstrative prefix,
defined in (\ref{ex:dmorphlex-dem}) above. However, \rayr{d/} in this case does
not have a strictly deictic meaning, but an anaphoric one---together with the
adjunct it forms a pronoun of sorts, though unlike \rayr{dnY}{danya} and like
the indefinite demonstrative, it cannot stand alone, but needs another word to
lean on as a clitic. Thus, we have to distinguish between the demonstrative
prefix \rayr{d/}{da-} and the anaphoric prefix \rayr{d/}{da-}.
\autoref{fig:da-s} compares their morpholexical features.

\begin{figure}[t]\centering
\caption{Comparison between two functions of \rayr{d/}{da-}}
\begin{tabu} to \linewidth {X[1] X[12] X[1] X[12]}
\toprule\tableheaderfont
\multicolumn{2}{c}{Demonstrative \rayr{d/}{da-}}
& \multicolumn{2}{c}{Anaphoric \rayr{d/}{da-}}
\\
\toprule
D
& \begin{tabular}[t]{l l l}
	\ups{\Pers} & \req{} & 3 \\
	\ups{\Def} & = & $-$ \\
	\ups{\Spec} & = & $-$ \\
\end{tabular}

&

N
& \begin{tabular}[t]{l l l}
	\ups{\Pred} & = & `pro' \\
	(\ups{\Pers} & = & 3) \\
	(\ups{\Def} & = & $+$) \\
	\quad\downs{\Index{} \Pers} & = & 3 \\
	\quad\downs{\Index{} \Num} & = & \{\Sg{}, \Pl\} \\
	\quad\downs{\Index{} \Gend} & = & \{\M{}, \F{}, \N{}\} \\
	\quad\downs{\Index{} \Anim} & = & $\pm$ \\
\end{tabular}
\\

\bottomrule

\end{tabu}
\label{fig:da-s}
\end{figure}

The most striking difference in features should be that anaphoric 
\rayr{d/}{da-} lists a predicator which designates it as a pro-form, whereas
demonstrative \rayr{d/}{da-} is, essentially, a functional morpheme only, 
without lexical content. As a pro-form, the anaphoric prefix is able to
bind another NP, while demonstrative prefix cannot. Furthermore, the 
demonstrative prefix requires the presence of a third-person term, while the
anaphoric prefix defines one. Also, the definiteness and specificity values 
differs in that the demonstrative prefix with its meaning `such a' refers to
one, but no particular item of a set of alike items, while the anaphoric prefix 
refers to a certain item, that is, the referent it binds.

As shown in (\ref{ex:adjpron}), the host of the anaphoric \rayr{d/}{da-} clitic
may be an adjective or a possessive pronoun. In contrast to NPs, which are
complements of \xhead{D}, their maximal projections (AP and DP, respectively)
are adjuncts and thus need to be annotated with `\pass{\Adj}' rather than 
`\updown{}'. Complex anaphora with anaphoric \rayr{d/}{da-} may also receive
case marking like any other nominal, whether DP or NP. The variable person
features---\Num{}, \Gend{}, and \Anim{}---are inherited from the bound NP, so
that the respective \Aarg{} and \Parg{} case markers are selected in accordance
with the referred-to NP's \Anim{} feature. Verbs agree in number, gender, and
animacy accordingly as well if relevant.

\pex
\a\begingl
	\gla Le ming eryavāng {\normalfont [\tsup{NP}} @ limu nā @ 
			{\normalfont ]\tsub{i}}. //
	\glb Le ming ery=vāng {} limu-Ø nā {} //
	\glc \PatTI{} can use=\Second{}.\Aarg{} {} shirt-\Top{} {} //
	\glft `You can use my shirt.' //
\endgl

\a\begingl
	\gla Da-nāreng\tsub{i} apitu. //
	\glb Da=nā-reng apitu //
	\glc one=\Fsg{}.\Gen{}-\AargI{} clean //
	\glft `Mine is clean.' //
\endgl

\xe

Note, however, that in predicative statements, if a possessive pronoun is in the
second position, it can take \rayr{d/}{da-}, but it acts similarly to an
adjective. This means that in this exceptional case, there is no case marking
for agent or patient, which would otherwise be required of NPs. This is the 
reason why in \autoref{fig:da-s}, the \Pers{} and \Def{} values are in 
parenthesis---these features are not encoded by adjectives.

\pex
\a\begingl
	\gla Adanyāng {\normalfont [\tsup{NP}} @ nangās nā @ 
		{\normalfont ]\tsub{j}}. //
	\glb Adanya-ang {} nanga-as nā {} //
	\glc that-\Aarg{} {} house-\Parg{} \Fsg{}.\Gen{} {} //
	\glft `That is my house'. //
\endgl

\a\begingl
	\gla Adanyāng da-nā\tsub{j}. //
	\glb Adanya-ang da=nā //
	\glc that-\Aarg{} one=\Fsg{}.\Gen{} //
	\glft `It is mine.' //
\endgl

\a\begingl
	\gla Apitu da-nā\tsub{i}. //
	\glb Apitu da=nā //
	\glc clean one=\Fsg{}.\Gen{} //
	\glft `Mine is clean.' //
\endgl

\xe

While it is possible to derive complex anaphora from adjectives and possessive
pronouns this way, the process as described above only works for single 
words, but not for complex phrases. Thus, for instance, the following examples
are unacceptable:

\pex
\a\label{ex:suchadjp}\ljudge*\begingl
	\gla Tahayang {\normalfont [\tsup{DP}} @ da- @ {\normalfont 
		[\tsup{AP}} @ dano nay kema @ {\normalfont]]}. //
	\glb Taha=yang {} da= {} dano nay kema {} //
	\glc have=\Fsg{}.\Aarg{} {} one= {} green and striped {} //
	\glft `I have a/the green-and-striped one.' //
\endgl

\a\label{ex:suchrelc}\ljudge*\begingl
	\gla Vāng {\normalfont [\tsup{DP}} @ da- @ {\normalfont[\tsup{CP}} 
		@ si ya silvasayang ren @ {\normalfont ]]}. //
	\glb Vāng {} da= {} si ya silv-asa=yang ren-Ø {} //
	\glc \Second.\Aarg{} {} one= {} \Rel{} \LocT{} 
		see-\Hab{}=\Fsg{}.\Aarg{} market-\Top{} {} //
	\glft `You are the who-I-always-see-at-the-market one.' //
\endgl
\xe

Example (\ref{ex:suchadjp}) shows an AP consisting of two coordinated 
adjectives which are supposed to be nominalized here. The DP would normally 
have to be a patient, since \xayr{th/}{taha-}{have} requires two 
arguments---a possessor and a possessee, which are conventionally mapped to 
\Sbj{}/\Aarg{} and \Obj{}/\Parg{}, respectively. We have seen, however, that 
Ayeri allows nominal phrases unmarked for case only under very special 
circumstances. The question is, thus, whether a complex phrase can be 
nominalized as a whole in the first place. The next questions are, then, where 
the case marker should go:

\begin{enumerate}[noitemsep]
	\item only on the first head,
	\item on each conjunct (would \rayr{d/}{da-} also need to be 
		distributed, then?),
	\item or, essentially, as a clitic at the very end?
\end{enumerate}

\noindent If marking went on each conjunct, this would coordinate two APs 
instead of two \xhead{A}, however: `the green one and the striped one', which
is not necessarily the same as `the green and striped one'. With the relative
clause in (\ref{ex:suchrelc}) only suggestions 1 and 3 are possible, with an
awkward result as well. The solution in either case is to remember that 
\rayr{d/}{da-} + non-noun is an abbreviation of the full indefinite 
demonstrative, which basically behaves like a noun in terms of adjuncts:

\pex
\a\label{ex:danyaadjp}\begingl
	\gla Le tahayang {\normalfont [\tsup{DP}} @ danya {\normalfont 
		[\tsup{AP}} @ dano nay kema @ {\normalfont]]}. //
	\glb Le taha=yang {} danya-Ø {} dano nay kema {} //
	\glc \PatTI{} have=\Fsg{}.\Aarg{} {} one-\Top{} {} green and striped {} //
	\glft `I have the green and striped one.' //
\endgl

\a\begingl
	\gla Vāng {\normalfont [\tsup{DP}} @ danyās {\normalfont[\tsup{CP}} 
		@ si ya silvasayang ren @ {\normalfont ]]}. //
	\glb Vāng {} danya-as {} si ya silv-asa=yang ren-Ø {} //
	\glc \Second.\Aarg{} {} one-\Parg{} {} \Rel{} \LocT{} 
		see-\Hab{}=\Fsg{}.\Aarg{} market-\Top{} {} //
	\glft `You are the one who I always see at the market.' //
\endgl
\xe

Instead of the construction in (\ref{ex:danyaadjp}) it is also possible to
pronominalize the first adjective and to include the second one in a relative 
clause, or to turn the second adjective into a noun adjunct if there is an
equivalent noun. Since \xayr{kem}{kema}{striped} is derived from the noun
\xayr{kem}{kema}{stripe}, this strategy is possible here.

\pex
\a\begingl
	\gla da-danoley si kema //
	\glb da=dano-ley si kema //
	\glb one=green-\PargI{} \Rel{} striped //
	\glft `the green one which is striped' //
\endgl

\a\begingl
	\gla da-danoley kemayeri //
	\glb da=dano-ley kema-ye-ri //
	\glc one=green-\PargI{} stripe-\Pl{}-\Ins{} //
	\glft `the green one with stripes' //
\endgl

\xe

\section{Adjective phrases and adverbial phrases}

Adjectives describe properties of nouns; adverbs describe properties of actions,
but may also provide additional information about place, time, manner,
quantity, degree etc. more generally by not only modifying verbs, but also other
parts of speech, or even whole clauses. The morphology of adjectives has been 
elaborated in \autoref{sec:adjectives}; different uses of adverbs have been
described in \autoref{sec:adverbs}. In this section, I will not deal with those
adverbials consisting of an adpositional phrase, that is, ones which primarily
provide information on time and place.

Adjective phrases are headed by an adjective (\xhead{A}), which may either be
modified by an adverb as an adjunct (\ref{ex:advadj}), or by a phrasal
complement, for instance, an NP (\ref{ex:advcompl}):

\pex
\a\label{ex:advadj}\begingl
	\gla viyu \textbf{cuyam} //
	\glb viyu cuyam //
	\glc proud indeed //
	\glft `indeed proud' //
\endgl

\a\label{ex:advcompl}\begingl
	\gla viyu \textbf{yaneri} \textbf{tan} //
	\glb viyu yan-eri tan //
	\glc proud son-\Ins{} \TplM{}.\Gen{} //
	\glft `proud of their son' //
\endgl

\xe

The phrase structure of an AP thus can be generalized as shown in 
(\ref{ex:apstruct}), which corresponds to the c-structure illustrated in 
(\ref{ex:apcstruct}). I am assuming here that adjectives and adverbs share the 
same structures, hence I will not distinguish between them formally.

\ex\label{ex:apstruct}
AP → \xhead{A} (\anno*[\pass{\Adjc{}} \logor{} \pass{\XCompl{}}]{XP})
\xe

\ex~\label{ex:apcstruct}
\begin{forest}
[{\anno[\pass{\Adj} \logor{} \pass{\XCompl}]{AP}}
	[\anno{\xbar{A}}
		[\anno{\xhead{A}}]
		[{$\left(\anno[\pass{\Adjc{}} \logor{} \pass{\XCompl{}}]{XP}\right)$}]
	]
]
\end{forest}
\xe

Adjectives and adverbs in Ayeri can be interpreted as not inflecting, with even
the comparative suffixes construed as adverbial clitics meaning `rather' and
`most'. The only feature property of both adjectives and adverbs would be 
\Pred{}, then. A more progressive and strictly lexicalist interpretation would
treat these markers as inflections, however, so that adjectives and adverbs 
would both additionally have to have a \textsc{compar} feature (compare 
\cite[Feature Table]{pargram}) taking the values \Comp{} and \Supl{}.\footnote{%
In this context, \Compl{} is short for `comparative', not `complement'!} Both
hypotheses should be easily comprehensible, since clitics usually are 
grammaticalized as inflection. This makes sense, especially since the 
suffixes---even though they are still existing as separate lexical
items, for which reason I am glossing them as clitics rather than
as inflection---mainly serve a functional purpose here. The prototype of a 
table of morpholexic definitions for \xhead{A} accordingly looks like this:

\needspace{3\baselineskip}
\ex \begin{tabu}[t] {\usetabu{morphlex}}
	...
		& A
		& \begin{tabular}[t]{l l l}
			\ups{\Pred} & = & `...' \\
			\ups{\Compar} & = & \{\Comp{}, \Supl{}\} \\
		\end{tabular}
	\end{tabu}

\xe

As the following example shows, adjuncts and complements to \xhead{A} may be
combined as well: \xayr{/IknF}{ikan}{very} freely modifies 
\xayr{viyu}{viyu}{proud}, and \xayr{kYuymF}{cuyam}{indeed, really}
may be interpreted as being an adjunct to \xayr{/IknF}{-ikan}{very}. The NP
\xayr{yneri tnF}{yaneri tan}{of their son} (literally, `by their son'), however,
serves as a complement of \rayr{viyu}{viyu}, which is optionally transitive
here. From the structure of the AP we know that it cannot merely be an adjunct
to an NP, but rather, that it must be a predicative AP, and as such a
complement to a subject NP. Since the phrase does not contain a finite verb, it
is an \XCompl{} rather than a \Compl{}.

\ex
\begingl
	\gla viyu-ikan cuyam yaneri tan //
	\glb viyu=ikan cuyam yan-eri tan //
	\glc proud=very indeed son-\Ins{} \TplM{}.\Gen{} //
	\glft `indeed very proud of their son' //
\endgl\medskip

% \begin{forest} shorter edges,
% [{\anno[\pass{\XCompl}]{AP}}
% 	[\anno{\xbar{A}}
% 		[\anno{\xhead{A}}
% 			[viyu]
% 		]
% 		[\anno{\xbar{A}}
% 			[{\anno[\pass{\Adjc}]{AP}}
% 				[\anno{\xbar{A}}
% 					[\anno{\xhead{A}}
% 						[-ikan]
% 					]
% 					[{\anno[\pass{\Adjc}]{AP}}
% 						[\anno{\xhead{A}}
% 							[cuyam]
% 						]
% 					]
% 				]
% 			]
% 			[\anno{\xbar{A}}
% 				[{\anno[\pass{\XCompl}]{NP}}
% 					[\anno{\xbar{N}}
% 						[\anno{\xhead{N}}
% 							[yaneri]
% 						]
% 						[{\anno[\pass{\Adjc}]{DP}}
% 							[\anno{\xhead{D}}
% 								[tan]
% 							]
% 						]
% 					]
% 				]
% 			]
% 		]
% 	]
% ]
% \end{forest}

\begin{forest} shorter edges,
[{\anno[\pass{\XCompl}]{AP}}
	[\anno{\xbar{A}}
		[\anno{\xbar{A}}
			[\anno{\xbar{A}}
				[\anno{\xhead{A}}
					[viyu]
				]
				[{\anno[\pass{\Adjc}]{AP}}
					[\anno{\xhead{A}}
						[-ikan]
					]
				]
			]
			[{\anno[\pass{\Adjc}]{AP}}
				[\anno{\xhead{A}}
					[cuyam]
				]
			]
		]
		[{\anno[\pass{\XCompl}]{NP}}
			[\anno{\xbar{N}}
				[\anno{\xhead{N}}
					[yaneri]
				]
				[{\anno[\pass{\Adjc}]{DP}}
					[\anno{\xhead{D}}
						[tan]
					]
				]
			]
		]
	]
]
\end{forest}

\xe

% \section{Adpositional phrases}
%
% \section{Inflectional, verb, and complementizer phrases}
%
% * Ayeri's unmarked word order is essentially VSO:
%
%      IP
%     /  \
%   (DP)  I'
%        /  \
%       I°   S
%           / \
%          NP XP*
% 
%   → Analysis similar to the one suggested for Irish by ??? and ??? 
%     (Kroeger 1991, 1993; Bresnan 2016)
%   → Probably one has to distinguish clauses with IP from `small clauses' 
% 	  (cf. exposition to Kroeger 1991) with only S
%   → S is likely an exocentric category
% 	  → support/tests for this?
% 	  → Kroeger 1991 should list some criteria in this respect
% * Subject vs. Topic vs. Actor vs. Nominative in Ayeri, in relation to
%   superficial (?) similarities with Austronesian alignment (Kroeger 1991,
%   1993, 2007; Schachter 2015)
%   → Promoting patient in whatever way does not demote logical subject etc.
%   → Ayeri's notion of A = logical subject (also what LFG calls `SUBJ', right?)
%     probably stronger than Tagalog's
%   → PROBLEM: terminology all about the place---may have inspired differences 
%     between Ayeri and real-world AA + my only now beginning to learn more
%     about syntax and stuff
%     → IMPORTANT: how do I use the respective terms and why?
% * Bresnan (2016) on extended head principle relevant w/r/t placement of 
%   finite verb in I°?
%   → Are there any tests for whether there's a covert VP after all? Nothing
%     verbal ever seems to occur in S'
%   → Ayeri should treat logical subjects and objects similarly, so no 
%     hierarchy between them?
% * Interesting problem: subject pronoun citics as a specifier of S?
%   → Funny stuff happens when adverbs are present
%   → Especially with patient-pronoun subjects (if that term is appropriate)
% * Patient subjects, causative constructions, no secundative, avoidance (?) of 
%   raising all need to be investigated with regards to proto-role mapping in 
%   a-structure ([±r, ±o])
% * Existential vs. predicative statements
%   → also, object predicatives!
%   → also, comparative verbs: kama-, eng-, va- and why they're weird
% * Agreement with conjoined/disjoined NPs
%   → due to the way verb agreement in general works, closest-conjunct agreement
%     is the most likely strategy for gender mismatches; yet number resolution.
%   → default gender for anaphoric reference to mixed-gender NPs (also animacy
%     mismatches!)
%
% \citet[153--156, 165]{bresnan2016} → pronouns to clitics to agreement
% \citet[157]{bresnan2016} → evidence criteria for IP
% \citet[151\psqq]{dixon2012} → subjects
% \citet[116\psqq]{comrie1989} → subjects
% \citet[26--54]{kroeger1991} → subjects (and actually the whole thesis)
%
