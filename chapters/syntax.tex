% kate: word-wrap true;

\chapter{Phrase structures}

While the previous chapter dealt largely with the various parts of speech and 
their various distributive and inflectional properties, the present chapter 
will elaborate on how these words combine into syntactic phrases. Since Ayeri 
is a verb-initial language, it is probably most comfortably analyzed in terms 
of Lexical-Functional Grammar \citep{bresnan2016}, since \Lfg{} does not 
require complicated derivations behind the surface structure of 
sentences.\footnote{Passivization, for instance, is assumed to be a lexically 
motivated alternation in predicate structure (\Sbj{} is blocked, so the 
nominative is assigned to \Obj{}, and the original \Sbj{} is expressed 
by an \Adjc{}), rather than an internal derivational process 
\citep[23\psqq]{bresnan2016}.} It will be assumed here that, even though Ayeri 
is basically VSO with predicate and predication not adjacent to each other, it 
is configurational in that there is a VP which c-commands a number of other 
constituents as complements in transitive sentences.

In principle, \Lfg{} assumes that grammar operates on different structural 
levels: mainly, these are a(rgument) structure, c(onstituent) structure, and 
f(unctional) structure; other layers have been proposed by different 
researchers for different purposes \citep[862--865]{buttking2015}. 
\citet{bresnan2016} define three core design principles for \Lfg{}:

\begin{description}
\item[Variability:] \textcquote[41]{bresnan2016}{The principle of variability 
states that \emph{external structures vary across languages}. The formal model 
of external structure in \Lfg{} is the \emph{c-structure}, which stands for 
\enquote{constituent structure} or \enquote{categorial structure}}. 
C-structures are commonly represented by context-free phrase-structure rules; 
constituency trees are based on an extended version of X-bar theory 
\citep[42]{bresnan2016}.\footnote{The basic recursive rules of X-bar theory 
are observed:
\begin{enumerate}[nosep, leftmargin={2\footnotemargin}]
\item \begin{enumerate}
	\item \xbar{X} → \xhead{X}, YP
	\item \xbar{X} → \xbar{X}, YP
\end{enumerate}\medskip
\item XP → YP, \xbar{X}
\end{enumerate}

The principle of economy of expression furthermore dictates that trees be 
pruned of empty nodes, and projection levels be omitted if they are not 
branching \citep[119--128]{bresnan2016}.}

\item[Universality:] \textcquote[42]{bresnan2016}{The principle of universality 
states that \emph{internal structures are largely invariant across languages}. 
The formal model of internal structure in \Lfg{} is the \emph{f-structure}, 
which stands for \enquote{functional structure}}. The f-structure is depicted 
as an argument-value matrix (\Avm{}) which maps the relations between 
`subject' (\Sbj{}), `object' (\Obj{}), `predicator' (\Pred{}), etc. as 
functional abstractions of NP, VP, V, etc. \citep[42]{bresnan2016}. Verbs 
are also presented with their \fw{a-structure} spelled out. That is, which 
arguments a verb has relations to is formally stated \citep[15]{bresnan2016}. 
The f-structure collates semantic features associated with heads of grammatical 
functions (GFs), such as case (\Case{}), person (\Pers{}), number (\Num{}), 
which are abstract features and as such need not have morphological realization 
\citep[43]{bresnan2016}.

\item[Monotonicity:] \textcquote[43]{bresnan2016}{Constituent structure form is 
simply not the same in all languages [...] In \Lfg{} the correspondence mapping 
between internal and external structures does not preserve sameness of form. 
Instead, \emph{it is designed to preserve inclusion relations between the 
information expressed by the external structure and the content of the internal 
structure}}. Due to the principle of monorepresentation, 
information distributed over different morphemes which logically 
belongs to a single grammatical function is presented in the f-structure as 
unified.

\end{description}

\begin{figure}[t]\centering
\caption[F-structure mappings]{F-structure mappings \citep[15]{bresnan2016}}

\begin{tabular}[t]{l @{\quad\quad} c}
argument (a-)structure:
& \astruct{\tikzmark{verb}verb}{\tikzmark{x}x, \tikzmark{y}y}\bigskip \\

functional (f-)strucutre:
& \tikzmark{fstruct}\begin{avm}
\[
	\quad sbj \tikzmark{subj} & \[
		{\enspace}\vdots{\enspace} \\
	\]{\quad} \\
	
	\quad obj \tikzmark{obj} & \[
		{\enspace}\vdots{\enspace} \\
	\] \tikzmark{objval}{\quad} \\
	\quad \tikzmark{pred} pred & \dots \tikzmark{predval} \\
\]
\end{avm}\bigskip\\

constituent (c-)structure:
& \begin{forest} baseline
[\xbar{V}
	[\subnode{V}{V}]
	[NP \tikzmark{NP}
		[\xbar{N} \tikzmark{Nbar}]
	]
]
\end{forest}
%
\end{tabular}
\begin{tikzpicture}[remember picture, overlay]
\draw [-latex]
	([xshift=1.5ex, yshift=-0.5ex]{pic cs:verb})
	to [out=south, in=north west]
	([yshift=1ex]{pic cs:pred});
	
\draw [-latex]
	([xshift=0.5ex, yshift=-0.5ex]{pic cs:x})
	to [out=south, in=north east]
	([yshift=1ex]{pic cs:subj});
	
\draw [-latex]
	([xshift=0.5ex, yshift=-0.75ex]{pic cs:y})
	to [out=south, in=north east]
	([yshift=1ex]{pic cs:obj});
	
\draw [-latex]
	([yshift=0.5ex]{pic cs:V})
	to [out=west, in=south west]
	([yshift=-10ex]{pic cs:fstruct});
	
\draw [-latex]
	([yshift=0.5ex]{pic cs:NP})
	to [out=east, in=east]
	([yshift=0.5ex]{pic cs:objval});
	
\draw [-latex]
	([yshift=0.5ex]{pic cs:Nbar})
	to [out=east, in=east]
	([yshift=0.5ex]{pic cs:objval});
\end{tikzpicture}
\label{fig:phimap}
\end{figure}

To illustrate the different parallel structures in operation, 
\citet[15]{bresnan2016} give the schema in \autoref{fig:phimap} to demonstrate
which part of the a- and c-structure respectively corresponds (`links', `maps') 
to which part of the f-structure.\footnote{\citet{bresnan2016} use 
\textsc{`subj'} for `subject'; for consistency with the above I will use 
`\Sbj{}' in the following. I will also divergently use \Compl{} and 
\XCompl{} for \textsc{(x)comp}, since \textsc{comp} has already been used for 
`comparative' above.} Regarding the different functions distinguished, \Lfg{} 
assumes the following hierarchies \citep[97, 100]{bresnan2016}:

\pex\label{ex:functions}
\a\label{ex:gfs} Grammatical functions (GFs):\\
	$\overbrace{\Sbj{} > \Obj{} > \SObj{}}^{\text{core}} > 
	\overbrace{\OblqT{} > \XCompl{}, \Compl{} > \Adjc{}}^{\text{noncore}}$
\a\label{ex:nonafs} (Non)argument functions (AFs/$\overline{\mbox{AF}}$s):\\
	$\underbrace{\Top{}\: \Foc{}}_{\text{non-a-fns}}\; 
	\overbrace{\Sbj{}\: \Obj{}\: \SObj{}\: \OblqT{}\: \XCompl{}\: 
		\Compl{}}^{\text{a-fns}}\; 
	\underbrace{\Adjc{}}_{\text{non-a-fns}}$
\a\label{ex:dfs} Discourse functions (DFs):\\
	$\overbrace{\Top{}\: \Foc{}\: \Sbj{}}^{\text{d-fns}}\;  
	\underbrace{\Obj{}\: \SObj{}\: \OblqT{}\: \XCompl{}\: \Compl{}\: 
		\Adjc{}}_{\text{non-d-fns}}$
\xe

The elements listed in (\ref{ex:functions}) will also appear in 
phrase-structure rules and c-structure trees together with arrows. These arrows 
symbolize inheritance of feature information from the current level (↓) of the 
tree to the next (↑), so for instance, `\pass{\Sbj}' means that the information 
subsumed by the current node (`down') is passed on as the subject function of 
the next higher node (`up') in the tree. Concise information on notational 
formalisms of \Lfg{} can be found, for instance, in \citet{buttking2015}.

\section{Determiner- and noun phrases}

Noun phrases (NPs), and determiner phrases (DPs) as their functional 
counterpart, fulfill the functions of subject (\Sbj{}), object (\Obj{}), 
secondary object (\SObj{}), as well as oblique location (\Oblq{loc}), and 
various adjuncts (\Adjc{}). DPs and NPs can also constitute topics (\Top{}). 
Which DP or NP receives which function is selected by the a-structure of the 
verb---this also has repercussions on case- and topic marking.

Generally, DPs and NPs can be described with the phrase-structure formulas 
given in (\ref{ex:dpnpstruct}), which list the various parts that can occur in 
them; parentheses indicate the optionality of a term, that is, the respective 
element may occur but is not constitutive; an asterisk stands for `zero or 
more' occurrences of the respective element.

\pex\label{ex:dpnpstruct}
\a\label{ex:dpdef} DP → \anno*{\xhead{D}} (\anno*{NP})
\a\label{ex:npdef} NP → \anno*{\xhead{N}} (\anno*[\pass{\Adjc}]{XP}) 
% 	(\anno*[\pass{\Adjc}]{NP})
\a\label{ex:ndef} \xhead{N} → \anno*{N\tsub{stem}} \anno*{\mbox{-N\tsub{infl}}} 
			\logor{} (\anno*{\^P}) \anno*{\xhead{N}}
\xe

\subsection{Determiner phrases}

The functional heads of DPs are formed by determiners. The c-structure of a DP 
corresponding to the linear description in (\ref{ex:dpdef}) is depicted in 
(\ref{ex:dpcstruct}); annotations about feature inheritance for 
c-to-f-structure conversion have been added.

\ex\label{ex:dpcstruct}
\begin{forest}
[{\anno[\pass{df} \logor{} \pass{gf}]{DP}}
	[\anno{\xbar{D}}
		[\anno{\xhead{D}}]
		[$\left(\anno{NP}\right)$]
	]
]
\end{forest}
\xe

Ayeri does not possess articles as such, though it does have a variety of 
morphemes which occur as clitic prefixes on nouns which can be analyzed to 
fulfill this function. These clitics are the demonstrative prefixes 
\xayr{Ed/}{eda-}{this}, \xayr{Ad/}{ada-}{that}, and \xayr{d/}{da-}{such (a)}, 
as well as the inspecificity prefix \xayr{me/}{mə-}{some} (compare 
\autoref{subsec:nounpref}). These prefixes can only be used with nouns, but not 
with personal pronouns (compare \autoref{subsec:perspro}), since personal 
pronouns as well have the status of determiners. Pronouns, as pro-forms, are in 
complementary distribution with NPs containing a noun. In order to capture the 
fact that an NP is facultative with the demonstrative and inspecificity 
prefixes while it must not occur with pronouns, `NP' appears in parentheses 
since it is not always present. We can define the following lexicosemantic 
rules for determiners (\xhead{D}):

\pex
\a Demonstrative prefixes:\medskip

	\begin{tabu} to \linewidth {@{} X[15l] X[5l] X[80l]}
	\savetabu{lexsem}
	\rayr{Ed/}{eda-}
		& P
		& \begin{tabular}[t]{l l l}
			\ups{\Pred} & = & `this' \\
			\ups{\Pers} & \req{} & 3 \\
% 			\ups{\Prox} & = & $+$ \\
% 			\ups{\Dist} & = & $-$ \\
% 			\ups{\Def} & = & $+$ \\
			\ups{\Spec} & = & $+$ \\
		\end{tabular}
	\end{tabu}\medskip

	\begin{tabu} {\usetabu{lexsem}}
	\rayr{Ad/}{ada-}
		& P
		& \begin{tabular}[t]{l l l}
			\ups{\Pred} & = & `that' \\
			\ups{\Pers} & \req{} & 3 \\
% 			\ups{\Prox} & = & $-$ \\
% 			\ups{\Dist} & = & $+$ \\
% 			\ups{\Def} & = & $+$ \\
			\ups{\Spec} & = & $+$ \\
		\end{tabular}
	\end{tabu}\medskip
	
	\begin{tabu} {\usetabu{lexsem}}
	\rayr{d/}{da-}
		& P
		& \begin{tabular}[t]{l l l}
			\ups{\Pred} & = & `such' \\
			\ups{\Pers} & \req{} & 3 \\
% 			\ups{\Prox} & = & $+$ \\
% 			\ups{\Dist} & = & $-$ \\
% 			\ups{\Def} & = & $-$ \\
			\ups{\Spec} & = & $+$ \\
		\end{tabular}
	\end{tabu}
	
\needspace{3\baselineskip}
\a Inspecificity prefix:\medskip

	\begin{tabu} {\usetabu{lexsem}}
	\rayr{me/}{mə-}
		& P
		& \begin{tabular}[t]{l l l}
			\ups{\Spec} & = & $-$ \\
			\ups{\Pers} & \req{} & 3 \\
		\end{tabular}
	\end{tabu}

\needspace{3\baselineskip}
\a Personal pronouns:\medskip

	\begin{tabu} {\usetabu{lexsem}}
	(various)
		& N
		& \begin{tabular}[t]{l l l}
			\ups{\Pred} & = & `pro' \\
			\ups{\Pers} & = & \{\First{}, \Second{}, \Third{}\} \\
			\ups{\Refl} & = & $\pm$ \\
			\ups{\Num} & = & \{\Sg{}, \Pl{}\} \\
			\ups{\Gend} & = & \{\M{}, \F{}, \N{}\} \\
			\ups{\Anim} & = & $\pm$ \\
			\ups{\Case} & = & \{\Aarg{}, \Parg{}, \Dat{}, \Gen{}, 
				\Loc{}, \Ins{}, \Caus{}\} \\
			\ups{\Spec} & = & $+$ \\
		\end{tabular}
	\end{tabu}
	
\needspace{3\baselineskip}
\a Demonstrative pronouns:\medskip

	\begin{tabu} {\usetabu{lexsem}}
	(various)
		& N
		& \begin{tabular}[t]{l l l}
			\ups{\Pred} & = & `this/that/such' \\
			\ups{\Pers} & = & 3 \\
			\ups{\Anim} & = & $\pm$ \\
			\ups{\Case} & = & \{\Aarg{}, \Parg{}, \Dat{}, \Gen{}, 
				\Loc{}, \Ins{}, \Caus{}\} \\
			\ups{\Spec} & = & $+$ \\
		\end{tabular}
	\end{tabu}
	
\xe

Demonstrative prefixes as well as the inspecificity prefix \rayr{me/}{mə-} 
cannot be without an NP complement, which is what `\req{}' is supposed to 
express in the feature specification: the demonstrative prefixes require that 
an element exist which encodes a third-person referent to bind to; nouns 
are assumed to encode third person by default.\footnote{A simple `=' 
\emph{defines} a value; a subscript `c' indicates that the morpheme 
\emph{requires} that this value be present. Thus `\req{}' expresses a 
\fw{constraining} equation \citep[59--61]{bresnan2016}.} On the other hand, 
personal and demonstrative pronouns inherently define information on person 
(\Pers{}). Since there is a great number of both personal and demonstrative 
pronouns, only the various values they can assume are indicated, without 
ensuring that the combinations are actually possible (compare 
\autoref{subsec:perspro}). Personal pronouns, for instance, only distinguish 
\Inan{} as set against $\{\M{}, \F{}, \N{}\}$, which are subgroups of \An{}. 
Gender is also not distinguished in all persons, but only in the third.

\subsection{Noun phrases}

Regarding NPs proper, it is necessary to distinguish morphologically between 
those containing common nouns and those containing proper nouns (that is,  
names), as we have seen in the previous chapter (compare \autoref{sec:nouns}): 
common nouns indicate case by a suffix, while proper nouns receive case marking 
by a particle preceding the noun. For this reason, the phrase-structure formula 
in (\ref{ex:ndef}), defining nominal heads (\xhead{N}), has two halves 
connected by \textsc{or}. In the second half, the case particle before names is 
indicated by `\^P' (`P-roof'), a non-projecting particle 
\citep[116--117]{bresnan2016}. Nouns can also be modified by adjectives, 
adjunct nouns, or relative clauses, which are typically following the noun. The 
c-structure of NPs can be assumed to look like given in (\ref{ex:npcstruct}). 
Since modifiers of \xhead{N} (\xhead{N}'s sister node) can consist of different 
phrase types (AP, NP, CP), they are indicated generalized as `XP' in the 
diagrams.
% The NP sister node of \xbar{N} is reserved for genitive-NP attributes 
% to \xhead{N}.

\ex\label{ex:npcstruct}\labels
\begin{minipage}[t]{.5\linewidth}%
\tl\quad Common nouns:\\

\quad\begin{forest}
[{\anno[\pass{df} \logor{} \pass{gf} \logor{} \updown]{NP}}
	[\anno{\xbar{N}}
		[\anno{\xhead{N}}
			[\anno{N\tsub{stem}}]
			[\anno{\mbox{-N\tsub{infl}}}]
		]
		[{$\left(\anno[\pass{\Adjc}]{XP}\right)$}]
	]
% 	[{$\left(\anno[\pass{\Adjc}]{NP}\right)$}]
]
\end{forest}
\end{minipage}
%
\begin{minipage}[t]{.5\linewidth}%
\tl\quad Proper nouns:\\

\quad\begin{forest}
[{\anno[\pass{df} \logor{} \pass{gf} \logor{} \updown]{NP}}
	[\anno{\xbar{N}}
		[\anno{\xhead{N}}
			[\anno{\^P}]
			[\anno{\xhead{N}}]
		]
		[{$\left(\anno[\pass{\Adjc}]{XP}\right)$}]
	]
% 	[{$\left(\anno[\pass{\Adjc}]{NP}\right)$}]
]
\end{forest}
\end{minipage}
\xe

In the following, I will again give a list of lexicosemantic specifications 
which give an overview of the different semantic and morphological features 
nouns provide (also compare \autoref{sec:nouns}):

\needspace{3\baselineskip}
\pex
\a Common nouns:\\

	\begin{tabu} {\usetabu{lexsem}}
	...
		& N
		& \begin{tabular}[t]{l l l}
			\ups{\Pred} & = & `...' \\
			\ups{\Pers} & = & 3 \\
			\ups{\Num} & = & \{\Sg{}, \Pl{}\} \\
			\ups{\Gend} & = & \{\M{}, \F{}, \N{}\} \\
			\ups{\Anim} & = & $\pm$ \\
			\ups{\Case} & = & \{\Aarg{}, \Parg{}, \Dat{}, \Gen{}, 
				\Loc{}, \Ins{}, \Caus{}\} \\
		\end{tabular}
	\end{tabu}

\a Proper nouns:\\

	\begin{tabu} {\usetabu{lexsem}}
	(various)
		& P
		& \begin{tabular}[t]{l l l}
			\ups{\Anim} & \req{} & $\pm$ \\
			\ups{\Case} & = & \{\Aarg{}, \Parg{}, \Dat{}, \Gen{}, 
				\Loc{}, \Ins{}, \Caus{}\} \\
		\end{tabular}
	\end{tabu}\medskip

	\begin{tabu} {\usetabu{lexsem}}
	...
		& N
		& \begin{tabular}[t]{l l l}
			\ups{\Pred} & = & `...' \\
			\ups{\Pers} & = & 3 \\
			\ups{\Num} & = & \{\Sg{}, \Pl{}\} \\
			\ups{\Gend} & = & \{\M{}, \F{}, \N{}\} \\
			\ups{\Anim} & = & $\pm$ \\
		\end{tabular}
	\end{tabu}

\xe

Generally, both command and proper nouns provide information about person, 
number, gender, and animacy. As mentioned above, proper nouns require a 
particle to precede them which carries information on case. This particle 
distinguishes animacy for \Aarg{} and \Parg{} and is thus required to be 
consistent with the noun's animacy feature:\footnote{For clarity, I will only 
list the features relevant to the current discussion.}

\ex\labels%
\begin{minipage}[t]{.5\linewidth}
\tl\label{ex:animok}\quad %
\begin{forest}
[{%
\xhead{N} \\
\ups{\Anim} = $+$ \\
\ups{\Case} = \Parg{}
}
	[\^P
		[{%
			sa \\
			\ups{\Anim} \req{} $+$ \\
			\ups{\Case} = \Parg{}
		}]
	]
	[\xhead{N}
		[{%
			Ajān \\
			\ups{\Anim} = $+$
		}]
	]
]
\end{forest}
\end{minipage}
\begin{minipage}[t]{.5\linewidth}
\tl\label{ex:animclash}\quad %
\ljudge*\begin{forest}
[{%
\xhead{N} \\
\ups{\Anim} = \err{} \\
\ups{\Case} = \Parg{}
}
	[\^P
		[{%
			le \\
			\ups{\Anim} \req{} $-$ \\
			\ups{\Case} = \Parg{}
		}]
	]
	[\xhead{N}
		[{%
			Ajān \\
			\ups{\Anim} = $+$
		}]
	]
]
\end{forest}
\end{minipage}
\xe

Example (\ref{ex:animok}a) shows a well-formed construction: the proper noun, 
\rayr{AgYaanF}{Ajān}, is animate, hence the case particle also needs to be 
animate---the case particle, \rayr{s}{sa}, constrains the existence of an 
animate complement in its domain. In contrast to this, example 
(\ref{ex:animclash}b) is not well-formed in that the proper noun is animate but 
the case particle, \rayr{le}{le}, signals that \xhead{N} is inanimate: the 
\Anim{} values of \^P and its \xhead{N} sister clash and cannot be conclusively 
unified at the governing \xhead{N} node. The same principle of coherence is, of 
course, also true for common nouns, which receive a case-marking suffix.

As previously mentioned, nouns may be modified by a number of adjuncts of
various phrase types: adjectives (AP), adnominal genitives (NP), other nominal 
adjuncts (NP), and relative clauses (CP):

\pex
\a\begingl
	\glpreamble With an adjective: //
	\gla Ya manga nu-nunāran \textbf{baloyyereng} \textbf{tiru} venya 
		satitay. //
	\glb Ya manga nu\til{}nuna-aran baloy-ye-reng tiru ven-ya satitay //
	\glc \LocT{} \Prog{} \Iter{}\til{}fly-\TplI{} leaf-\Pl{}-\AargI{} dry 
		air-\Loc{} dry.season-\Top{} //
	\glft `In the dry season, the dry leaves are flying around in the 
		air.' //
\endgl
\medskip

	\begin{forest}
	[{\anno[\pass{\Sbj}]{NP}}
		[\anno{\xbar{N}}
			[\anno{\xhead{N}}
				[baloyyereng]
			]
			[{\anno[\pass{\Adjc}]{AP}}
				[{tiru}, roof]
			]
		]
	]
	\end{forest}
	\hfill
	{\larger\begin{avm}
	\[
		\Pred	& `...' \\
		\Sbj	& \[
			\Pred	& `leaf' \\
% 			\Pers	& \Third \\
			\Num	& \Pl \\
			\Anim	& $-$ \\
			\Case	& \Aarg \\
			\Adjc	& \{\[
				\Pred	& `dry' \\
				\]\} \\
			\] \\
		...
	\]
	\end{avm}}

\a\begingl
	\glpreamble With an adnominal genitive: //
	\gla Adanyāng \textbf{pacanas} \textbf{na} \textbf{Tuvo}. //
	\glb Adanya-ang pacan-as na Tuvo //
	\glc that.one-\Aarg{} cousin-\Parg{} \Gen{} Tuvo //
	\glft `That is Tuvo's cousin.' //
\endgl
\medskip

	\begin{forest}
	[{\anno[\pass{\XCompl}]{NP}}
		[\anno{\xbar{N}}
			[\anno{\xhead{N}}
				[pacanas]
			]
			[{\anno[\pass{\Adjc}]{NP}}
				[{na Tuvo}, roof]
			]
		]
	]
	\end{forest}
	\hfill
	{\larger\begin{avm}
	\[
% 		\Pred	& `...' \\
		... \\
		\XCompl	& \[
			\Pred	& `cousin' \\
% 			\Pers	& \Third \\
			\Num	& \Sg \\
			\Anim	& $+$ \\
			\Case	& \Parg \\
			\Adjc	& \{ \Oblq{gen} & \[
					\Pred	& `Tuvo' \\
% 					\Pers	& \Third \\
% 					\Anim	& $+$ \\
					\Case	& \Gen \\
					\] \\
				\} \\
			\] \\
	\]
	\end{avm}}

\a\begingl
	\glpreamble With a noun adjunct: //
	\gla Le pegaya sinyāng \textbf{kasu} \textbf{bariri}? //
	\glb Le pega-ya sinya-ang kasu-Ø bari-ri //
	\glc \PargI{} steal-\TsgM{} who-\Aarg{} basket-\Top{} meat-\Ins{} //
	\glft `The basket of meat, who stole it?' //
\endgl
\medskip

	\begin{forest}
	[{\anno[\elem{\Top}]{NP}}
		[\anno{\xbar{N}}
			[\anno{\xhead{N}}
				[kasu]
			]
			[{\anno[\pass{\Adjc}]{NP}}
				[{bariri}, roof]
			]
		]
	]
	\end{forest}
	\hfill
	{\larger\begin{avm}
	\[
		\Pred	& `...' \\
		\Top	& \[
			\Pred	& `basket' \\
% 			\Pers	& \Third \\
			\Num	& \Sg \\
			\Anim	& $-$ \\
			\Case	& \Parg \\
			\Adjc	& \{ \Oblq{ins} & \[
					\Pred	& `meat' \\
% 					\Pers	& \Third \\
% 					\Anim	& $-$ \\
					\Case	& \Ins \\
					\] \\
				\} \\
			\] \\
		...
	\]
	\end{avm}}

\a\begingl
	\glpreamble With a relative clause: //
	\gla Ang konja \textbf{seygoley} \textbf{si} \textbf{tuvo} \textbf{nay} 
		\textbf{sepra}. //
	\glb Ang kond=ya.Ø seygo-ley si tuvo nay sepra //
	\glc \AgtT{} eat=\Fsg{}.\Top{} apple-\PargI{} \Rel{} red and crunchy //
	\glft `I ate an apple which was red and crunchy.' //
\endgl
\medskip

	\begin{forest}
	[{\anno[\pass{\Obj}]{NP}}
		[\anno{\xbar{N}}
			[\anno{\xhead{N}}
				[seygoley]
			]
			[{\anno[\pass{\Adjc}]{CP}}
				[{si tuvo nay sepra}, roof]
			]
		]
	]
	\end{forest}
	\hfill
	{\larger\begin{avm}
	$f_1$: \[
		\Pred	& `...' \\
		\Obj	& \[
			\Pred	& `apple' \\
% 			\Pers	& \Third \\
			\Num	& \Sg \\
			\Anim	& $-$ \\
			\Case	& \Parg \\
			\Adjc	& \{
					\[
						\Pred	& `be red ...' \\
					\] \\
				\} \\
			\] \\
		...
	\]
	\end{avm}}

\xe

Of course, it is also possible to combine these nominal modifiers. In this 
case, a genitive NP always follows an adjective, while a relative clause 
always forms the last element in an NP. For one, the relative grammatical 
weight of a relative clause is very high, whether one counts 
\textcquote[102]{wasow1997}{words, nodes, or phrasal nodes} as indicators of 
grammatical weight. Relative clauses also trail whenever possible 
since placing one before a lighter element may make an utterance hard to 
construct coherently for the speaker, and hence also hard to understand for the 
listener. The following example illustrates the unmarked order of modifiers:

\ex
\begingl
	\gla diranang caban nā si ang mica ya Kārvisam //
	\glb diranang caban nā si ang mit=ya.Ø ya Kārvisam //
	\glc uncle-\Aarg{} favorite \Fsg{}.\Gen{} \Rel{} \AgtT{} 
		live=\TsgM{}.\Top{} \Loc{} Kārvisam //
	\glft `my favorite uncle who lives in Kārvisam' //
\endgl
\medskip

\begin{forest}
[{\anno[\pass{\Sbj}]{NP}}
	[\anno{\xbar{N}}
		[\anno{\xhead{N}}
			[diranang]
		]
		[\anno{\xbar{N}}
			[{\anno[\elem{\Adj}]{AP}}
% 				[\anno{\xhead{A}}
					[{caban}, roof]
% 				]
			]
			[\anno{\xbar{N}}
				[{\anno[\elem{\Adj}]{DP}}
% 					[\anno{\xhead{D}}
						[{nā}, roof]
% 					]
				]
				[\anno{\xbar{N}}
					[{\anno[\elem{\Adj}]{CP}}
						[{si ang mica ...}, roof]
					]
				]
			]
		]
	]
]
\end{forest}
\hfill
{\larger\begin{avm}
\[
\Sbj	& \[
	\Pred	& `uncle' \\
	\Case	& \Aarg \\
	\Anim	& $+$ \\
	\Adjc	& \{
			\[
				\Pred	& `favorite' \\
			\], \\
			\[
				\Pers	& \First \\
				\Num	& \Sg \\
				\Case	& \Gen \\
			\], \\
			\[
				\Pred	& `lives in ...' \\
			\] \\
		\} \\
	\] \\
\]
\end{avm}}

\xe

% \section{Adjective phrases}
%
% \section{Adpositional phrases}
%
% \section{Verb phrases}
%
% \section{Complement phrases}
%
