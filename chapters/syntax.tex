% kate: word-wrap true;

\chapter{Phrase structures}

While the previous chapter dealt largely with the various parts of speech and 
their various distributive and inflectional properties, the present chapter 
will elaborate on how these words combine into syntactic phrases. Since Ayeri 
is a verb-initial language, it is probably most comfortably analyzed in terms 
of Lexical-Functional Grammar \citep{bresnan2016}, since \Lfg{} does not 
require complicated derivations behind the surface structure of 
sentences.\footnote{Passivization, for instance, is assumed to be a lexically 
motivated alternation in predicate structure (\Sbj{} is blocked, so the 
nominative is assigned to \Obj{}, and the original \Sbj{} is expressed 
by an \Adjc{}), rather than an internal derivational process 
\citep[23\psqq]{bresnan2016}.} It will be assumed here that, even though Ayeri 
is basically VSO with predicate and predication not adjacent to each other, it 
is configurational in that there is a VP which c-commands a number of other 
constituents as complements in transitive sentences.

In principle, \Lfg{} assumes that grammar operates on different structural 
levels: mainly, these are a(rgument) structure, c(onstituent) structure, and 
f(unctional) structure; other layers have been proposed by different 
researchers for different purposes \citep[862--865]{buttking2015}. 
\citet{bresnan2016} define three core design principles for \Lfg{}:

\begin{description}
\item[Variability:] \textcquote[41]{bresnan2016}{The principle of variability 
states that \emph{external structures vary across languages}. The formal model 
of external structure in \Lfg{} is the \emph{c-structure}, which stands for 
\enquote{constituent structure} or \enquote{categorial structure}}. 
C-structures are commonly represented by context-free phrase-structure rules; 
constituency trees are based on an extended version of X-bar theory 
\citep[42]{bresnan2016}.\footnote{The basic recursive rules of X-bar theory 
are observed:
\begin{enumerate}[nosep, leftmargin={2\footnotemargin}]
\item \xbar{X} → \xhead{X}, YP
\item XP → YP, \xbar{X}
\end{enumerate}

The principle of economy of expression furthermore dictates that trees be 
pruned of empty nodes, and projection levels be omitted if they are not 
branching \citep[119--128]{bresnan2016}.}

\item[Universality:] \textcquote[42]{bresnan2016}{The principle of universality 
states that \emph{internal structures are largely invariant across languages}. 
The formal model of internal structure in \Lfg{} is the \emph{f-structure}, 
which stands for \enquote{functional structure}}. The f-structure is depicted 
as an argument-value matrix (\Avm{}) which maps the relations between 
`subject' (\Sbj{}), `object' (\Obj{}), `predicator' (\Pred{}), etc. as 
functional abstractions of NP, VP, V, etc. \citep[42]{bresnan2016}. Verbs 
are also presented with their \fw{a-structure} spelled out. That is, which 
arguments a verb has relations to is formally stated \citep[15]{bresnan2016}. 
The f-structure lists semantic features associated with words, such as case 
(\Case{}), person (\Pers{}), number (\Num{}), which are abstract features and 
as such need not have morphological realization \citep[43]{bresnan2016}.

\item[Monotonicity:] \textcquote[43]{bresnan2016}{Constituent structure form is 
simply not the same in all languages [...] In \Lfg{} the correspondence mapping 
between internal and external structures does not preserve sameness of form. 
Instead, \emph{it is designed to preserve inclusion relations between the 
information expressed by the external structure and the content of the internal 
structure}}. Due to the principle of monorepresentation, 
information distributed over different morphemes which logically 
belongs to a single grammatical function is presented in the f-structure as 
unified.

\end{description}

To illustrate the different parallel structures in operation, 
\citet[15]{bresnan2016} give the following schema concerning which part 
of the a- and c-structure respectively corresponds (`links', `maps') to which 
part of the f-structure:\footnote{\citet{bresnan2016} use \textsc{`subj'} for 
`subject'; for consistency with the above I will use `\Sbj{}' in the following.}

\ex\begin{tabular}[t]{l @{\quad} c}
argument (a-)structure:
& \astruct{\tikzmark{verb}verb}{\tikzmark{x}x, \tikzmark{y}y}\bigskip \\

functional (f-)strucutre:
& \tikzmark{fstruct}\begin{avm}
\[
	\quad sbj \tikzmark{subj} & \[
		{\enspace}\vdots{\enspace} \\
	\]{\quad} \\
	
	\quad obj \tikzmark{obj} & \[
		{\enspace}\vdots{\enspace} \\
	\] \tikzmark{objval}{\quad} \\
	\quad \tikzmark{pred} pred & \dots \tikzmark{predval} \\
\]
\end{avm}\bigskip\\

constituent (c-)structure:
& \begin{forest} baseline
[\xbar{V}
	[\subnode{V}{V}]
	[NP \tikzmark{NP}
		[\xbar{N} \tikzmark{Nbar}]
	]
]
\end{forest}
%
\end{tabular}\xe
%
\begin{tikzpicture}[remember picture, overlay]
\draw [-latex]
	([xshift=1.5ex, yshift=-0.5ex]{pic cs:verb})
	to [out=south, in=north west]
	([yshift=1ex]{pic cs:pred});
	
\draw [-latex]
	([xshift=0.5ex, yshift=-0.5ex]{pic cs:x})
	to [out=south, in=north east]
	([yshift=1ex]{pic cs:subj});
	
\draw [-latex]
	([xshift=0.5ex, yshift=-0.75ex]{pic cs:y})
	to [out=south, in=north east]
	([yshift=1ex]{pic cs:obj});
	
\draw [-latex]
	([yshift=0.5ex]{pic cs:V})
	to [out=west, in=south west]
	([yshift=-9ex]{pic cs:fstruct});
	
\draw [-latex]
	([yshift=0.5ex]{pic cs:NP})
	to [out=east, in=east]
	([yshift=0.5ex]{pic cs:objval});
	
\draw [-latex]
	([yshift=0.5ex]{pic cs:Nbar})
	to [out=east, in=east]
	([yshift=0.5ex]{pic cs:objval});
\end{tikzpicture}

Regarding the different functions distinguished, \Lfg{} assumes the following 
hierarchies \citep[97, 100]{bresnan2016}:

\pex\label{ex:functions}
\a\label{ex:gfs} Grammatical functions (GFs):\\
	$\overbrace{\Sbj{} > \Obj{} > \SObj{}}^{\text{core}} > 
	\overbrace{\OblqT{} > \XCompl{}, \Compl{} > \Adjc{}}^{\text{noncore}}$
\a\label{ex:nonafs} (Non)argument functions (AFs/$\overline{\mbox{AF}}$s):\\
	$\underbrace{\Top{}\: \Foc{}}_{\text{non-a-fns}}\; 
	\overbrace{\Sbj{}\: \Obj{}\: \SObj{}\: \OblqT{}\: \XCompl{}\: 
		\Compl{}}^{\text{a-fns}}\; 
	\underbrace{\Adjc{}}_{\text{non-a-fns}}$
\a\label{ex:dfs} Discourse functions (DFs):\\
	$\overbrace{\Top{}\: \Foc{}\: \Sbj{}}_{\text{d-fns}}\;  
	\underbrace{\Obj{}\: \SObj{}\: \OblqT{}\: \XCompl{}\: \Compl{}\: 
		\Adjc{}}_{\text{non-d-fns}}$
\xe

The elements listed in (\ref{ex:functions}) will also appear in 
phrase-structure rules and c-structure trees together with arrows. These arrows 
symbolize inheritance of feature information from the current level (↓) of the 
tree to the next (↑), so for instance, `\pass{\Sbj}' means that the information 
subsumed by the current node (`down') is passed on as the subject function of 
the next higher node (`up') in the tree. Concise information on notational 
formalisms of \Lfg{} can be found, for instance, in \citet{buttking2015}.

\section{Determiner- and noun phrases}

Noun phrases (NPs), and determiner phrases (DPs) as their functional 
counterpart, fulfill the functions of subject (\Sbj{}), object (\Obj{}), 
secondary object (\SObj{}), as well as oblique location (\Oblq{loc}), and 
various adjuncts (\Adjc{}). DPs and NPs can also constitute topics (\Top{}). 
Which DP or NP receives which function is selected by the a-structure of the 
verb---this also has repercussions on case- and topic marking.

Generally, DPs and NPs can be described with the phrase-structure formulas 
given in (\ref{ex:dpnpstruct}), which list the various parts that can occur in 
them; parentheses indicate the optionality of a term, that is, the respective 
element may occur but is not constitutive; an asterisk stands for `zero or 
more' occurrences of the respective element.

\pex\label{ex:dpnpstruct}
\a\label{ex:dpdef} DP → \anno*{\xhead{D}} (\anno*{NP})
\a\label{ex:npdef} NP → \anno*{\xhead{N}} (\anno*{AP*}) (\anno*{CP})
\a\label{ex:ndef} \xhead{N} → \anno*{N\tsub{\itshape stem}} 
\anno*{\mbox{-N\tsub{\itshape infl}}} $\lor$ (\anno*{\^P}) \anno*{\xhead{N}}
\xe

\subsection{Determiner phrases}

The functional heads of DPs are formed by determiners. Ayeri does not possess 
articles as such, though it does have a variety of morphemes which occur 
as clitic prefixes on nouns which can be analyzed to fulfill this function. 
These clitics are the demonstrative prefixes \xayr{Ed/}{eda-}{this}, 
\xayr{Ad/}{ada-}{that}, and \xayr{d/}{da-}{such (a)}, as well as the 
inspecificity prefix \xayr{me/}{mə-}{some} (compare \autoref{subsec:nounpref}). 
These prefixes can only be used with nouns, but not with personal pronouns 
(compare \autoref{subsec:perspro}), since personal pronouns as well have the 
status of determiners. Pronouns, as pro-forms, are in complementary 
distribution with NPs containing a noun. In order to capture the fact that an 
NP is facultative with the demonstrative and inspecificity prefixes while it 
must not occur with pronouns, `NP' appears in parentheses in (\ref{ex:dpdef}), 
since it is not always present. We can define the following lexicosemantic 
rules for determiners:

\pex
\a Demonstrative prefixes:\medskip

	\begin{tabu} to \linewidth {X[15l] X[5l] X[80l]}
	\savetabu{lexsem}
	\rayr{Ed/}{eda-}
		& P
		& \begin{tabular}[t]{l l l}
			\ups{\Pred} & = & `this' \\
			\ups{\Pers} & \req{} & 3 \\
% 			\ups{\Prox} & = & $+$ \\
% 			\ups{\Dist} & = & $-$ \\
% 			\ups{\Def} & = & $+$ \\
% 			\ups{\Spec} & = & $+$ \\
		\end{tabular}
	\end{tabu}\medskip

	\begin{tabu} {\usetabu{lexsem}}
	\rayr{Ad/}{ada-}
		& P
		& \begin{tabular}[t]{l l l}
			\ups{\Pred} & = & `that' \\
			\ups{\Pers} & \req{} & 3 \\
% 			\ups{\Prox} & = & $-$ \\
% 			\ups{\Dist} & = & $+$ \\
% 			\ups{\Def} & = & $+$ \\
% 			\ups{\Spec} & = & $+$ \\
		\end{tabular}
	\end{tabu}\medskip
	
	\begin{tabu} {\usetabu{lexsem}}
	\rayr{d/}{da-}
		& P
		& \begin{tabular}[t]{l l l}
			\ups{\Pred} & = & `such' \\
			\ups{\Pers} & \req{} & 3 \\
% 			\ups{\Prox} & = & $+$ \\
% 			\ups{\Dist} & = & $-$ \\
% 			\ups{\Def} & = & $-$ \\
% 			\ups{\Spec} & = & $+$ \\
		\end{tabular}
	\end{tabu}
	
\a Specifiers:\medskip

	\begin{tabu} {\usetabu{lexsem}}
	\rayr{me/}{mə-}
		& P
		& \begin{tabular}[t]{l l l}
			\ups{\Spec} & = & $-$ \\
			\ups{\Pers} & \req{} & 3 \\
		\end{tabular}
	\end{tabu}
	
\a Personal pronouns:\medskip

	\begin{tabu} {\usetabu{lexsem}}
	(various)
		& N
		& \begin{tabular}[t]{l l l}
			\ups{\Pred} & = & `pro' \\
			\ups{\Pers} & = & \{\First{}, \Second{}, \Third{}\} \\
			\ups{\Refl} & = & $\pm$ \\
			\ups{\Num} & = & \{\Sg{}, \Pl{}\} \\
			\ups{\Gend} & = & \{\M{}, \F{}, \N{}\} \\
			\ups{\Anim} & = & $\pm$ \\
			\ups{\Case} & = & \{\Aarg{}, \Parg{}, \Dat{}, \Gen{}, 
				\Loc{}, \Ins{}, \Caus{}\} \\
		\end{tabular}
	\end{tabu}
	
\a Demonstrative pronouns:\medskip

	\begin{tabu} {\usetabu{lexsem}}
	(various)
		& N
		& \begin{tabular}[t]{l l l}
			\ups{\Pred} & = & `this/that/such' \\
			\ups{\Pers} & = & 3 \\
			\ups{\Anim} & = & $\pm$ \\
			\ups{\Case} & = & \{\Aarg{}, \Parg{}, \Dat{}, \Gen{}, 
				\Loc{}, \Ins{}, \Caus{}\} \\
		\end{tabular}
	\end{tabu}
\xe

Demonstrative prefixes as well as the inspecificity prefix \rayr{me/}{mə-} 
cannot be without an NP complement, which is what `\req{}' is supposed to 
express in the feature specification: the demonstrative prefixes require that 
an element which encodes a third-person referent to bind to exist; nouns are 
assumed to encode third person by default.\footnote{A simple `=' \emph{defines} 
a value; a subscript `c' indicates that the morpheme \emph{requires} that this 
value be present. Thus `\req{}' expresses a \fw{constraining} equation 
\citep[59--61]{bresnan2016}.} On the other hand, personal and demonstrative 
pronouns inherently define information on person (\Pers{}). Since there is a 
great number of both personal and demonstrative pronouns, only the various 
values they can assume are indicated, without ensuring that the combinations are 
actually possible (compare \autoref{subsec:perspro}). Personal pronouns, for 
instance, only distinguish \Inan{} as set against $\{\M{}, \F{}, \N{}\}$, which 
are subgroups of \An{}. Gender is also not distinguished in all persons, but 
only in the third.

\subsection{Noun phrases}

Regarding NPs proper, it is necessary to distinguish morphologically between 
those containing common nouns and those containing proper nouns (that is,  
names), as we have seen in the previous chapter (compare \autoref{sec:nouns}): 
common nouns indicate case by a suffix, while proper nouns receive case marking 
by a particle preceding the noun. For this reason, the phrase-structure formula 
in (\ref{ex:ndef}), defining nominal heads (\xhead{N}), has two halves 
connected by \textsc{or}. In the second half, the case particle before names is 
indicated by `\^P' (`P-roof'), a non-projecting particle 
\citep[116--117]{bresnan2016}. Nouns can also be modified by a number of 
adjectives, as well as relative clauses (here indicated by `CP'), both of which 
are typically following their heads. For nouns, the following general 
lexicosemantic properties can be assumed:

\pex
\a Common nouns:\\

	\begin{tabu} {\usetabu{lexsem}}
	(various)
		& N
		& \begin{tabular}[t]{l l l}
			\ups{\Pred} & = & `...' \\
			\ups{\Num} & = & \{\Sg{}, \Pl{}\} \\
			\ups{\Gend} & = & \{\M{}, \F{}, \N{}\} \\
			\ups{\Anim} & = & $\pm$ \\
			\ups{\Case} & = & \{\Aarg{}, \Parg{}, \Dat{}, \Gen{}, 
				\Loc{}, \Ins{}, \Caus{}\} \\
		\end{tabular}
	\end{tabu}

\a Proper nouns:\\

	\begin{tabu} {\usetabu{lexsem}}
	(various)
		& P
		& \begin{tabular}[t]{l l l}
			\ups{\Anim} & = & $\pm$ \\
			\ups{\Case} & = & \{\Aarg{}, \Parg{}, \Dat{}, \Gen{}, 
				\Loc{}, \Ins{}, \Caus{}\} \\
		\end{tabular}
	\end{tabu}\medskip

	\begin{tabu} {\usetabu{lexsem}}
	(various)
		& N
		& \begin{tabular}[t]{l l l}
			\ups{\Pred} & = & `...' \\
			\quad\downs{index \Pers} & = & 3 \\
			\quad\downs{index \Num} & = & \{\Sg{}, \Pl{}\} \\
			\quad\downs{index \Gend} & = & \{\M{}, \F{}, \N{}\} \\
			\quad\downs{index \Anim} & = & $\pm$ \\
		\end{tabular}
	\end{tabu}

\xe

...

% \section{Adjective phrases}
%
% \section{Adpositional phrases}
%
% \section{Verb phrases}
%
% \section{Complement phrases}
%
