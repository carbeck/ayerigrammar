% kate: word-wrap true;

% IMPORTANT NOTE: I've never dealt much with syntax before, at least not in a
% way as formal as I'm trying to come to terms with it here. If there's
% anything wrong with my analyses, it's because I don't know enough yet. Things
% may be subject to change *especially* in this chapter.

\chapter{Phrase structures}

While the previous chapter dealt largely with the various parts of speech and
their various distributive and inflectional properties, the present chapter
will elaborate on how these words combine into syntactic phrases. Since Ayeri
is a verb-initial language, it is probably most comfortably analyzed in terms
of Lexical-Functional Grammar (\cite{bresnan1982}~ff.; more recently
\cite{bresnan2016}), since \Lfg{} does not require complicated derivations
behind the surface structure of  sentences.\footnote{Passivization, for
instance, is assumed to be a lexically  motivated alternation in predicate
structure (\Sbj{} is blocked, so the  nominative is assigned to \Obj{}, and the
original \Sbj{} is expressed  by an \Adjc{}), rather than an internal
derivational process \citep[23\psqq]{bresnan2016}.} It will be assumed here
that, even though Ayeri is basically VSO with predicate and predication not
adjacent to each other, it is configurational in that there is a VP which
c-commands a number of other constituents as complements in transitive
sentences.

In principle, \Lfg{} assumes that grammar operates on different structural 
levels: mainly, these are a(rgument) structure, c(onstituent) structure, and 
f(unctional) structure; other layers have been proposed by different 
researchers for different purposes \citep[862--865]{buttking2015}. 
\citet{bresnan2016} define three core design principles for \Lfg{}:

\begin{description}
\item[Variability:] \textcquote[41]{bresnan2016}{The principle of variability
states that \emph{external structures vary across languages}. The formal model
of external structure in \Lfg{} is the \emph{c-structure}, which stands for
\enquote{constituent structure} or \enquote{categorial structure}}. 
C-structures are commonly represented by context-free phrase-structure rules; 
constituency trees are based on an extended version of X-bar theory 
\citep[42]{bresnan2016}.\footnote{The basic recursive rules of X-bar theory 
are observed:
\begin{enumerate}[nosep, leftmargin={2\footnotemargin}]
\item XP → YP, \xbar{X} (specifier rule)
\item \xbar{X} → \xbar{X}, YP (adjunct rule)
\item \xbar{X} → \xhead{X}, YP (complement rule)
\end{enumerate}

The principle of economy of expression furthermore dictates that essentially, 
trees be pruned of empty terminal nodes and non-branching preterminal nodes, 
since these do not provide structurally or semantically relevant information 
\citep[119--128]{bresnan2016}.}

\item[Universality:] \textcquote[42]{bresnan2016}{The principle of universality
states that \emph{internal structures are largely invariant across languages}.
The formal model of internal structure in \Lfg{} is the \emph{f-structure},
which stands for \enquote{functional structure}}. The f-structure is depicted
as an argument-value matrix (\Avm{}) which maps the relations between `subject'
(\Sbj{}), `object' (\Obj{}), `predicator' (\Pred{}), etc. as functional
abstractions of NP, VP, V, etc. \citep[42]{bresnan2016}. Verbs are also
presented with their \fw{a-structure} spelled out. That is, which arguments a
verb has relations to is formally stated \citep[15]{bresnan2016}. The
f-structure collates semantic features associated with heads of grammatical
functions (GFs), such as case (\Case{}), person (\Pers{}), number (\Num{}),
which are abstract features and as such need not have morphological realization
\citep[43]{bresnan2016}.

\item[Monotonicity:] \textcquote[43]{bresnan2016}{Constituent structure form is
simply not the same in all languages [...] In \Lfg{} the correspondence mapping
between internal and external structures does not preserve sameness of form.
Instead, \emph{it is designed to preserve inclusion relations between the
information expressed by the external structure and the content of the internal
structure}}. Due to the principle of monorepresentation, information
distributed over different morphemes which logically belongs to a single
grammatical function is presented in the f-structure as unified.

\end{description}

\begin{figure}[t]\centering
\caption[F-structure mappings]{F-structure mappings \citep[15]{bresnan2016}}

\begin{tabular}[t]{l @{\quad\quad} c}
argument (a-)structure:
& \astruct{\tikzmark{verb}verb}{\tikzmark{x}x, \tikzmark{y}y}\bigskip \\

functional (f-)strucutre:
& \tikzmark{fstruct}{\smaller\begin{avm}
\[
	\quad sbj \tikzmark{subj} & \[
		{\enspace}\vdots{\enspace} \\
	\]{\quad} \\
	
	\quad obj \tikzmark{obj} & \[
		{\enspace}\vdots{\enspace} \\
	\] \tikzmark{objval}{\quad} \\
	\quad \tikzmark{pred} pred & \dots \tikzmark{predval} \\
\]
\end{avm}}\bigskip\\

constituent (c-)structure:
& \begin{forest} baseline
[\xbar{V}
	[\subnode{V}{V}]
	[NP \tikzmark{NP}
		[\xbar{N} \tikzmark{Nbar}]
	]
]
\end{forest}
%
\end{tabular}
\begin{tikzpicture}[remember picture, overlay]
\draw [-latex]
	([xshift=1.5ex, yshift=-0.5ex]{pic cs:verb})
	to [out=south, in=north west]
	([yshift=1ex]{pic cs:pred});
	
\draw [-latex]
	([xshift=0.5ex, yshift=-0.5ex]{pic cs:x})
	to [out=south, in=north east]
	([yshift=1ex]{pic cs:subj});
	
\draw [-latex]
	([xshift=0.5ex, yshift=-0.75ex]{pic cs:y})
	to [out=south, in=north east]
	([yshift=1ex]{pic cs:obj});
	
\draw [-latex]
	([yshift=0.5ex]{pic cs:V})
	to [out=west, in=south west]
	([yshift=-10ex]{pic cs:fstruct});
	
\draw [-latex]
	([yshift=0.5ex]{pic cs:NP})
	to [out=east, in=east]
	([yshift=0.5ex]{pic cs:objval});
	
\draw [-latex]
	([yshift=0.5ex]{pic cs:Nbar})
	to [out=east, in=east]
	([yshift=0.5ex]{pic cs:objval});
\end{tikzpicture}
\label{fig:phimap}
\end{figure}

To illustrate the different parallel structures in operation,
\citet[15]{bresnan2016} give the schema in \autoref{fig:phimap} to demonstrate
which part of the a- and c-structure respectively corresponds (`links', `maps')
to which part of the f-structure.
% \footnote{\citet{bresnan2016} use \textsc{`subj'} for `subject'. ; for
% consistency with the above I will use `\Sbj{}' in the following. I will also
% divergently use \Compl{} and \XCompl{} for \textsc{(x)comp}, since
% \textsc{comp} has already been used for `comparative' above.}
Regarding the different functions distinguished, \Lfg{} 
assumes the following hierarchies \citep[97, 100]{bresnan2016}:

\pex\label{ex:functions}
\a\label{ex:gfs} Grammatical functions (GFs):\\
	$\overbrace{\Sbj{} > \Obj{} > \SObj{}}^{\text{core}} > 
	\overbrace{\OblqT{} > \XCompl{}, \Compl{} > \Adjc{}}^{\text{noncore}}$
\a\label{ex:nonafs} (Non)argument functions (AFs/$\overline{\mbox{AF}}$s):\\
	$\underbrace{\Top{}\: \Foc{}}_{\text{non-a-fns}}\; 
	\overbrace{\Sbj{}\: \Obj{}\: \SObj{}\: \OblqT{}\: \XCompl{}\: 
		\Compl{}}^{\text{a-fns}}\; 
	\underbrace{\Adjc{}}_{\text{non-a-fns}}$
\a\label{ex:dfs} Discourse functions (DFs):\\
	$\overbrace{\Top{}\: \Foc{}\: \Sbj{}}^{\text{d-fns}}\;  
	\underbrace{\Obj{}\: \SObj{}\: \OblqT{}\: \XCompl{}\: \Compl{}\: 
		\Adjc{}}_{\text{non-d-fns}}$
\xe

The elements listed in (\ref{ex:functions}) will also appear in
phrase-structure rules and c-structure trees together with arrows. These arrows
symbolize inheritance of feature information from the current level (↓) of the
tree to the next (↑), so for instance, `\pass{\Sbj}' means that the information
subsumed by the current node (`down') is passed on as the subject function of
the next higher node (`up') in the tree. Concise information on notational
formalisms of \Lfg{} can be found, for instance, in \citet{buttking2015}.

\section{Noun phrases and determiner phrases}
\label{sec:nps-dps}

Noun phrases (NPs), and determiner phrases (DPs) as their functional 
counterpart, fulfill the functions of subject (\Sbj{}), object (\Obj{}), 
secondary object (\SObj{}), as well as various oblique constituents (\OblqT;
see (\ref{ex:dpnpcasemap}) below), and they can also form adjuncts (\Adjc{}).
DPs and NPs can also constitute topics (\Top{}). Which DP or NP receives which 
function is selected by the a-structure of the verb---this also has 
repercussions on case- and topic marking.

Since the constituent containing the various non-verbal elements in Ayeri is 
likely exocentric and constituents may move around within it in a restricted 
way, we have to assume that not constituent structure, but case marking 
identifies the grammatical functions of the various arguments of verbs. Thus, 
the following lexicocentric conditions operate on both DP and NP as exponents 
of case:

\ex\label{ex:dpnpcasemap}\labels
\begin{tabular}[t]{@{} l @{\quad} l @{$\implies$} l}
\tl\quad & \downs{\Case} = \Aarg	& \pass{\Sbj} \\
\tl\quad & \downs{\Case} = \Parg	& \pass{\Obj} \\
\tl\quad & \downs{\Case} = \Dat		& \pass{\SObj} \logor{}
										\pass{\Oblq{loc}} \\
\tl\quad & \downs{\Case} = \Gen		& \pass{\Oblq{poss}} \logor{} 
						\pass{\Oblq{loc}} \\
\tl\quad & \downs{\Case} = \Loc		& \pass{\Oblq{loc}} \\
\tl\quad & \downs{\Case} = \Caus	& \pass{\Oblq{caus}} \\
\tl\quad & \downs{\Case} = \Ins		& \pass{\Oblq{ins}} \logor{} 
										\pass{\Adjc} \\
\end{tabular}
\xe

The rules in (\ref{ex:dpnpcasemap}) determine the typical mappings between case
marking and grammatical functions, which are not always unambiguous. As
explained above (compare \autoref{subsec:case}), the dative case does not only
indicate that something is done to this referent or to their benefit, but it
may also indicate motion towards this referent. Likewise, the genitive case
does not only indicate possession, but also origin, and motion from this
referent. Nominal adjuncts to nouns which specify what the noun consists also
appear in the instrumental case, besides the instrumental being used to
indicate the means or the circumstance by which an action comes about.
Moreover, DPs or NPs may also lack case marking, which indicates that the
respective phrase is a part of the topic of the verb, which is what
(\ref{ex:topicrule}) describes:

\ex\label{ex:topicrule}
¬\,\downs{\Case} $\implies$ \elem{\Top}
\xe

Instead of case marking on the DP or NP, there is a marker in front of the verb
which provides information on the case and, if \AgtT{} or \PatT{}, also about
the animacy of the topicalized phrase. This means that grammatical information
about the topic of a phrase is spread over two discontinuous locations. This
issue does not pose a problem to an \Lfg{}-based analysis, however, since both
locations unify their information content in the f-structure feature \Top{}.
Since information located in multiple places is jointly feeding this feature, I
am using the annotation `\elem{\Top}' for each location rather than simple
`\pass{\Top}'. Note that only one NP among the arguments of a verb may be the
topic of the phrase, and a topic can only be marked if the verb is finite and
the number of arguments to the verb is greater than one.

\subsection{Noun phrases}

\subsubsection{Constituent order within noun phrases}

Nouns are one of the main parts of speech of Ayeri, and nouns can be modified
by a number of other free elements, as we have seen previously---adjectives,
possessive adjectives, as well as relative clauses and nominal adjuncts. These
typically follow nouns. It was also described before that Ayeri's nouns may
host a number of clitics, among which are deictic prefixes and quantifiers, as
well as enclitic case markers in the case of proper nouns. These clitics,
however, will not be treated as targets of syntactic operations, since \Lfg{}
follows the approach of lexical integrity. Thus, bound elements like affixes
and clitics are assumed not to be reflected or affected by syntax itself. The
phrase structure of NPs should thus look like depicted in (\ref{ex:npstruct}).

\pex\label{ex:npstruct}
\a NP → \anno*{\xbar{N}}
\a \xbar{N} → \anno*{\xhead{N}} $\left(\anno*[{\pass{\Adjc}}]{XP}\right)$
\xe

This rule defines that NPs have a lexical head which is on the left side,
followed optionally by modifiers which act as adjuncts to the noun. The
description leaves open which kind of phrases the adjuncts can consist of, but
as mentioned above, these are commonly adjectives (forming adjective phrases,
APs), nominal adjuncts (NPs), as well as relative clauses (forming
complementizer phrases, CPs). This can be represented as a constituent-structure tree in the way described in (\ref{ex:npcstruct}).

\ex\label{ex:npcstruct}\labels
\begin{forest}
[{\anno[\{\pass{df} | \pass{gf}\}]{NP}}
	[\anno{\xbar{N}}
		[\anno{\xhead{N}}]
		[{$\left(\anno[{%
				\pass{\Adjc}%
			}]{XP}\right)$
		}]
	]
]
\end{forest}
\xe

Here as well, we can see a nominal head on the left which may be modified by
adjuncts of various types. The maximal projection of \xhead{N} (that is, NP) is
annotated very generally for the function of the NP---basically, an NP can act
as either a discourse function (DF) or a grammatical function (GF). 
(\ref{ex:nounmods}) gives an example of each kind of modifier. Since there is 
no grammatical context given, NP is unmarked for function in these examples.

\pex\label{ex:nounmods}
\a %
	\begin{minipage}[t]{.5\linewidth}
	\begingl
		\glpreamble noun + adjective: //
		\gla ningan hiro //
		\glb ningan hiro //
		\glc story new //
		\glft `new story' //
	\endgl
	\end{minipage}
	~
	\begin{forest} shorter edges,
	[NP
		[\anno{\xbar{N}}
			[\anno{\xhead{N}}
				[{ningan}]
			]
			[{\anno[\pass{\Adjc}]{AP}}
				[{hiro}, roof]
			]
		]
	]
	\end{forest}

\a %
	\begin{minipage}[t]{.5\linewidth}
	\begingl
		\glpreamble noun + genitive attribute: //
		\gla kegan ayonena //
		\glb kegan ayon-ena //
		\glc hat man-\Gen{} //
		\glft `the man's hat' //
	\endgl
	\end{minipage}
	~
	\begin{forest} shorter edges,
	[NP
		[\anno{\xbar{N}}
			[\anno{\xhead{N}}
				[{kegan}]
			]
			[{\anno[\pass{\Adjc}]{NP}}
				[{ayonena}, roof]
			]
		]
	]
	\end{forest}

\a %
	\begin{minipage}[t]{.5\linewidth}
	\begingl
		\glpreamble noun + instrumental attribute: //
		\gla kasu bariri //
		\glb kasu bari-ri //
		\glc basket meat-\Ins{} //
		\glft `basket of meat' //
	\endgl
	\end{minipage}
	~
	\begin{forest} shorter edges,
	[NP
		[\anno{\xbar{N}}
			[\anno{\xhead{N}}
				[{kasu}]
			]
			[{\anno[\pass{\Adjc}]{NP}}
				[{bariri}, roof]
			]
		]
	]
	\end{forest}

\a %
	\begin{minipage}[t]{.5\linewidth}
	\begingl
		\glpreamble noun + relative clause: //
		\gla nanga si incāng //
		\glb nanga si int=yāng //
		\glc house \Rel{} buy=\TsgM{}.\Aarg{} //
		\glft `the house he bought' //
	\endgl
	\end{minipage}
	~
	\begin{forest} shorter edges,
	[NP
		[\anno{\xbar{N}}
			[\anno{\xhead{N}}
				[{nanga}]
			]
			[{\anno[\pass{\Adjc}]{CP}}
				[{si incāng}, roof]
			]
		]
	]
	\end{forest}

\xe

As described before (compare \autoref{subsec:clitics}), nouns can be modified
by a number of clitics which are not represented through syntax. Since it is
not possible for these clitic elements to be divided from their phonological
hosts, they should be treated as being an integral part of the word they attach
to. Hence, \xhead{N} is given in (\ref{ex:nouncltree}) as split into `Cl' and
\xhead{N}.

\pex\label{ex:nouncltree}
\a %
	\begin{minipage}[t]{.5\linewidth}
	\begingl
		\glpreamble noun + deictic prefix: //
		\gla eda- @ nanga //
		\glb eda= nanga //
		\glc this= house //
		\glft `this house' //
	\endgl
	\end{minipage}
	~
	\begin{forest} shorter edges,
	[\anno{\xhead{N}}
		[\anno{Cl}
			[eda-]
		]
		[\anno{\xhead{N}}
			[nanga]
		]
	]
	\end{forest}

\a %
	\begin{minipage}[t]{.5\linewidth}
	\begingl
		\glpreamble noun + quantifier: //
		\gla nangās @ -ikan //
		\glb nanga-as =ikan //
		\glc house-\Parg{} =many //
		\glft `many houses' //
	\endgl
	\end{minipage}
	~
	\begin{forest} shorter edges,
	[\anno{\xhead{N}}
		[\anno{\xhead{N}}
			[nangās]
		]
		[\anno{Cl}
			[-ikan]
		]
	]
	\end{forest}

\a %
	\begin{minipage}[t]{.5\linewidth}
	\begingl
		\glpreamble proper noun + case: //
		\gla ang @ Diyan //
		\glb ang= Diyan //
		\glc \Aarg{}= Diyan //
		\glft `Diyan' //
	\endgl
	\end{minipage}
	~
	\begin{forest} shorter edges,
	[\anno{\xhead{N}}
		[\anno{Cl}
			[ang]
		]
		[\anno{\xhead{N}}
			[Diyan]
		]
	]
	\end{forest}

\xe

Of course, it is also possible to combine these nominal modifiers. In this
case, there is a certain hierarchy, presumably based on Behaghel's first law,
\textquote{Das oberste Gesetz ist dieses, daß das geistig eng Zusammengehörige 
auch eng zusammengestellt wird} (\cite[4]{behaghel1932}; `The supreme law is
such that the mentally closely related is also arranged in close proximity.'), 
and also grammatical weight \citep{wasow1997}:

\begin{enumerate}[noitemsep]
	\item instrumental NP indicating what the head consists of,
	\item APs (also cardinals) and other NPs describing attributes,
	\item possessive genitive NPs,
	\item relative clauses.
\end{enumerate}

\citet{wasow1997} writes that \textcquote[102]{wasow1997}{[i]t is very hard to
distinguish among various structural weight measures as predictors of weight
effects. Counting words, nodes, or phrasal nodes all work well}, which means
that no single metric can be used to describe the order of constituents in a
phrase. However, for instance, relative clauses trail whenever possible
presumably since they tend to contain whole subclauses and thus a lot of
information. It seems advisable not to put an element with much less 
information content after them, especially when it refers to a different head
than all the things inside the relative clause. The following example 
(\ref{ex:nounmodord}) illustrates the unmarked order of modifiers.

\pex\label{ex:nounmodord}
\a\begingl
	\gla diranang caban nā si ang mica ya @ Kārvisam //
	\glb diran-ang caban nā si ang mit=ya.Ø ya= Kārvisam //
	\glc uncle-\Aarg{} favorite \Fsg{}.\Gen{} \Rel{} \AgtT{} 
		live=\TsgM{}.\Top{} \Loc{}= Kārvisam //
	\glft `my favorite uncle who lives in Kārvisam' //
\endgl
\medskip

\a\begin{forest} shorter edges, narrower nodes,
[{\anno[\pass{\Sbj}]{NP}}
	[\anno{\xbar{N}}
		[\anno{\xbar{N}}
			[\anno{\xbar{N}}
				[\anno{\xhead{N}}
					[{diranang}]
				]
				[{\anno[\elem{\Adj}]{AP}}
					[{caban}, roof]
				]
			]
			[{\anno[\elem{\Adj}]{AP}}
				[{nā}, roof]
			]
		]
		[{\anno[\elem{\Adj}]{CP}}
			[{si ang mica ...}, roof]
		]
	]
]
\end{forest}

\xe

As the c-structure tree in (\ref{ex:nounmodord}) shows, Ayeri prefers
head--dependent word order with exceeding consistency. As illustrated by
previous examples, both adjuncts and complements are, for the most part,
consistently appended to the right of their heads, which means that Ayeri may
be classified as a rather consistently right-branching language. However, a
certain number of postpositions form an exception to this classification
(\autoref{subsec:postpos}). In the light of word order typology, we can
formulate the following generalizations:

\pex
\a Order of noun and adjective: N Adj
\a Order of noun and genitive: N Gen
\a Order of noun and relative clause: N Rel
\xe

More important to \Lfg{} than c-structure trees, however, are
function-structure matrices which gather all information in a given utterance
and gather potentially disparate information into semantically coherent units
within the matrix.\footnote{Essentially, c-structure is similar to the tree
hierarchy of paragraphs, images, tables etc. in an \textsc{html} file, while
f-structure describes semantic properties of elements in the tree similar to
how \textsc{css} defines the layout properties of these elements.} In the
following, I will thus give a list of morpholexic specifications which give an
overview of the different semantic and morphological features nouns basically
provide (also compare \autoref{sec:nouns}). These also form the basis for
f-structure matrices of the kind already shown in (\ref{ex:clitics_43}),
\autoref{clitics_postverb_person}, p.~\pageref{ex:clitics_43}.

\ex\label{ex:nounmorphlex}%
\adjustbox{valign=t}{%
	\begin{tabu} {\usetabu{morphlexnarrow}}
	...
		& N
		& \begin{tabular}[t]{l l l}
			\ups{\Pred} & = & `...' \\
			\ups{\Anim} & = & $\pm$ \\
			\ups{\Case} & = & \{\Aarg{}, \Parg{}, \Dat{}, \Gen{}, 
				\Loc{}, \Ins{}, \Caus{}\} \\
			\ups{\Gend} & = & \{\M{}, \F{}, \N{}, \Inan{}\} \\
			\ups{\Num} & = & \{\Sg{}, \Pl{}\} \\
			\ups{\Pers} & = & 3 \\
		\end{tabular}
	\end{tabu}%
}
\xe

Nouns generally imply a third-person reference; they distinguish number, gender
and animacy, as well as case. Clitics, however, may also add information about
deixis (\ref{ex:deixisfeat}); likeness and quantity might be interpreted
conveniently as adding to the list of a noun's \Adjc{} feature 
(\ref{ex:nounqadj}).

\ex\label{ex:deixisfeat}
%\adjustbox{valign=t}{%
	\begin{tabu} {\usetabu{morphlexnarrow}}
	\hphantom{...}
		& \hphantom{N}
		& \begin{tabular}[t]{l l l}
			\ups{\Deix} & = & \{$this$, $that$, $such$\} \\
		\end{tabular}
	\end{tabu}%
%}
\xe

\pex~\label{ex:nounqadj}
\a\begingl
	\gla ganang-hen mino //
	\glb gan-ang=hen mino //
	\glc child-\Aarg{}=all happy //
	\glft `all happy children' //
\endgl

\a\begin{avm}
\[
	\Subj{}	&	\[
					\Pred	&	`child' \\
					\Anim	&	$+$ \\
					\Case	&	\Aarg \\
					\Adjc	&	\{
									\[
										\Pred	&	`all' \\
										\Num	&	\Pl \\
									\], \\
									\[
										\Pred	&	`happy' \\
									\] \\
								\} \\
				\] \\
\]
\end{avm}
\xe

\subsubsection{Morphosyntactic operations within the noun phrase}

It has been pointed out above that nouns encode animacy. This has repercussions
in the choice of case markers of the agent and patient cases, which need to
agree with the lexical head they attach to. An example of this is given in 
(\ref{ex:animcaseagr}).

\pex\label{ex:animcaseagr}
\a\label{ex:animok} %
\begin{forest}
[\anno{\xhead{N}}
	[\anno{N\tsub{stem}}
		[{%
			gan \\
			\ups{\Anim} = $+$ \\
		}]
	]
	[\anno{-N\tsub{infl}}
		[{%
			-ang \\
			\ups{\Anim} = $+$ \\
			\ups{\Case} = \Aarg{} \\
		}]
	]
]
\end{forest}

\a\label{ex:animclash} %
\ljudge*\begin{forest}
[\anno{\xhead{N}}
	[\anno{N\tsub{stem}}
		[{%
			gan \\
			\ups{\Anim} = $+$ \\
		}]
	]
	[\anno{N\tsub{infl}}
		[{%
			-reng \\
			\ups{\Anim} = $-$ \\
			\ups{\Case} = \Aarg{} \\
		}]
	]
]
\end{forest}
\xe

Example (\ref{ex:animok}) shows a well-formed construction: the noun,
\xayr{gnF}{gan}{child}, is animate, hence the case particle also needs to be 
animate---the case particle must thus be \rayr{/As}{-ang} to be coherent. In
contrast to this, example (\ref{ex:animclash}) is not well-formed in that the
noun is animate but the case particle, \rayr{/reNF}{-reng}, signals that it is
inanimate: the \Anim{} values of the noun stem and its suffix clash and cannot
be conclusively unified for \xhead{N} itself. The same principle of coherence
is, of course, also true for proper nouns, which receive a case-marking
particle:

\pex\label{ex:animcaseagrname}
\a\label{ex:animokname} %
\begin{forest}
[\anno{\xhead{N}}
	[\anno{Cl}
		[{%
			ang \\
			\ups{\Anim} = $+$ \\
			\ups{\Case} = \Aarg{} \\
		}]
	]
	[\anno{\xhead{N}}
		[{%
			Dita \\
			\ups{\Anim} = $+$ \\
		}]
	]
]
\end{forest}

\a\label{ex:animclashname} %
\ljudge*\begin{forest}
[\anno{\xhead{N}}
	[\anno{Cl}
		[{%
			eng \\
			\ups{\Anim} = $-$ \\
			\ups{\Case} = \Aarg{} \\
		}]
	]
	[\anno{\xhead{N}}
		[{%
			Dita \\
			\ups{\Anim} = $+$ \\
		}]
	]
]
\end{forest}
\xe

Furthermore, example (\ref{ex:nounqadj}) already showed that nouns may be
modified by quantifiers, whether these are clitic suffixes
(\autoref{subsec:quantifiers}) or numerals (\autoref{sec:numerals}). In these
cases, plural marking on the noun is suppressed by the presence of the modifier
which supplies the information by itself so that further morphological plural
marking by the suffix \rayr{/ye}{-ye} on the noun stem itself would be
redundant. As shown in \autoref{sec:numerals} (p.~\pageref{hundreds}), however,
there are very limited occasions where a noun may be marked for plural in spite
of the presence of a numeral, for instance:

\ex\begingl
	\gla Ang bengyon keynamye menang kanānya {desay iray}. //
	\glb ang beng-yon keynam-ye-Ø menang kanān-ya {desay iray} //
	\glc \AgtT{} attend-\TplN{} people-\Pl{}-\Top{} hundred wedding-\Loc{} 
		royal //
	\glft `Hundreds of people attended the royal wedding.' //
\endgl\xe

Here, the noun \xayr{kejnmF}{keynam}{people} is marked additionally for plural
by the nominal plural suffix \rayr{/ye}{-ye} in spite of being a \fw{plurale
tantum} and in spite of the presence of the numeral \rayr{menNF}{menang}
{hundred}. Without plural morphology, the meaning of \rayr{kejnmF menNF}{keynam
menang} would be `a hundred people', not generic `hundreds'.

\subsection{Determiner phrases}
\label{subsec:dps}

Determiner phrases (DPs) are the functional equivalent of NPs; determiners as
their heads (\xhead{D}) are a closed class of function words \citep[102]
{bresnan2016}. In English, for instance, articles and pronouns are counted
among them \citep[208--211]{carnie2013}. Ayeri, however, probably does not
possess articles as such. The preposed case markers of proper nouns bear a
superficial similarity to cased articles like in German
(\ref{ex:artcasesimil}), and their presence or absence is morphosyntactically
controlled by topicalization and thus also interacts with definiteness.
However, as we will see below, the distribution of these case particles differs
from that of articles in languages like English or German.

\ex\label{ex:artcasesimil}\labels
\begin{tabular}[t]{@{} l l >{\itshape\bfseries}l @{~} >{\itshape}l l
l}
\tl\quad

	& \Aarg
	& ang & Sān
	& `Sān'
	\\

	& \Parg
	& sa & Sān
	& `Sān'
	\\

	& \Dat	
	& yam & Sān
	& `to Sān'
	\\
	
	& \Gen
	& na & Sān
	& `Sān's'
	\\
	
	& \Loc
	& ya & Sān
	& `at Sān'
	\\
	
	& \Caus
	& sā & Sān
	& `due to Sān'
	\\
	
	& \Ins
	& ri & Sān
	& `with/by Sān'
	\medskip \\

\rc{German}\tl\quad%

	& \Nom
	& der & Mann
	& `the man'
	\\

	& \Acc
	& den & Mann
	& `the man'
	\\

	& \Dat
	& dem & Mann
	& `to the man'
	\\

	& \Gen
	& des & Mannes
	& `of the man'
	\\
\end{tabular}
\xe

While in German an article and a demonstrative pronoun, or also a possessive
pronoun, cannot co-occur, this appears not to be a problem in Ayeri. As argued
in \autoref{subsec:clitics}, both case markers and deictic/demonstrative
prefixes in Ayeri are clitics; the similarity between possessive pronouns and
adjectives has also been noted in \autoref{phsec:possadj}
(p.~\pageref{phsec:possadj}). Furthermore, the preposed case markers of nouns
are an exception compared to the much more frequent occurrence of case-marking
suffixes on generic nouns. It thus does not seem straightforward to analyze the
case markers as heads of DPs.

\pex\label{ex:germandetdist}
	\a\rc{German}
	\begingl
		\gla das Haus //
		\glb das Haus //
		\glc \Def{}.\Nom{}.\Sg{}.\N{} house //
		\glft `the house' //
	\endgl

	\a\begingl
		\gla dieses Haus //
		\glb dies-es Haus //
		\glc this-\Nom{}.\Sg{}.\N{}.\St{} Haus //
		\glft `this house' //
	\endgl

	\a\begingl
		\gla mein Haus //
		\glb mein-Ø Haus //
		\glc \Fsg{}.\Gen{}-\Nom{}.\Sg{}.\N{}.\St{} house //
		\glft `my house' //
	\endgl

	\a\ljudge*\begingl
		\gla das diese Haus //
		\glb das dies-e Haus //
		\glc \Def{}.\Nom{}.\Sg{}.\N{} this-\Nom{}.\Sg{}.\N{}.\Wk{} house //
		\glft `the this house' //
	\endgl

	\a\ljudge*\begingl
		\gla das meine Haus //
		\glb das mein-e Haus //
		\glc \Def{}.\Nom{}.\Sg{}.\N{} \Fsg{}.\Gen{}-\Nom{}.\Sg{}.\N{}.\Wk{} 
			man //
		\glft `the my house' //
	\endgl

	\a\ljudge*\label{ex:germandemposswk}\begingl
		\gla dieses meine Haus //
		\glb dies-es mein-e Haus //
		\glc this-\Nom{}.\Sg{}.\N{}.\St{} 
			\Fsg{}.\Gen{}-\Nom{}.\Sg{}.\N{}.\Wk{} house //
		\glft `this my house' //
	\endgl

	\a\ljudge\hash\label{ex:germandemposs}\begingl
		\gla dieses mein Haus //
		\glb dies-es mein-Ø Haus //
		\glc this-\Nom{}.\Sg{}.\N{}.\St{} 
			\Fsg{}.\Gen{}-\Nom{}.\Sg{}.\N{}.\St{} house //
		\glft `this house of mine' //
	\endgl
\xe

The examples in (\ref{ex:germandetdist}) show that determining elements such as
a definite article (\fw{der} `the'), a demonstrative pronoun (\fw{dieser}
`this') and a possessive pronoun (\fw{mein} `my') are in complementary
distribution for some combinations. The only exception to this is the
combination of demonstrative and possessive in (\ref{ex:germandemposs}), which
is grammatically marked, however.\footnote{Example (\ref{ex:germandemposswk})
differs from (\ref{ex:germandemposs}) in the declension paradigm of the
adjective: (\ref{ex:germandemposswk}) uses the `weak' (\Wk) declension
regularly, since a determiner with strong (\St) declension precedes.
(\ref{ex:germandemposs}) appears to be an exception in permitting two
determiners of the strong declension. \citet[160--161, 203--205]{demske2001}
notes that, according to \citet{plank1992}, possessive pronouns may apparently
still act as modifiers, not only determiners, under certain circumstances. In
modern Standard German, this construction is strongly marked, however. It is
probably a remnant of earlier stages of German where there was no such
restriction on the co-ocurrence of demonstrative and possessive pronouns yet
\citep[173]{demske2001}.} On this phenomenon of complementary distribution of
determiners, which also holds true for English, \citet[208]{carnie2013} writes,
\textquote{One thing to note about determiners is that they are typically
heads. Normally, there can only be one of them in an NP,} at least in English
(and German). \citet[9--22]{demske2001} elaborates on this point for German as
well. Ayeri, however, seems to behave differently in treating case markers and
demonstrative elements as clitics:

\pex\label{ex:ayericasenoart}
	\a
	\begingl
		\gla ang @ Sān //
		\glb ang= Sān //
		\glc \Aarg{}= Sān //
		\glft `Sān' //
	\endgl

	\a\begingl
		\gla ang @ eda- @ Sān //
		\glb ang= eda= Sān //
		\glc \Aarg{}= this= Sān //
		\glft `this Sān' //
	\endgl

	\a\label{ex:naaadj}\begingl
		\gla ang @ Sān nā //
		\glb ang= Sān nā //
		\glc \Aarg{}= Sān \Fsg{}.\Gen{} //
		\glft `my Sān' //
	\endgl

	\a\ljudge\ques\begingl
		\gla ang @ eda- @ Sān nā //
		\glb ang= eda= Sān nā //
		\glc \Aarg{}= this= Sān nā //
		\glft `this Sān of mine' //
	\endgl
\xe

In all cases listed in (\ref{ex:ayericasenoart}), the case marker is present
and marks the NP simply for agent case, irrespective of other elements.
Characteristically, neither the deictic prefixes, nor the possessive
pronoun/adjective in Ayeri mark case, while they do in German. The case marker
thus cannot be simply left out, because the information it provides is not
redundant, strictly speaking. Where it \emph{is} left out, it marks the NP as
topicalized and it is required, then, that the verb mark the topicalized NP's
case. The same is also true of generic nouns:

\pex
	\a
	\begingl
		\gla veneyang //
		\glb veney-ang //
		\glc dog-\Aarg{} //
		\glft `a/the dog' //
	\endgl

	\a\begingl
		\gla eda- @ veneyang //
		\glb eda= veney-ang //
		\glc this= dog-\Aarg{} //
		\glft `this dog' //
	\endgl

	\a\begingl
		\gla veneyang nā //
		\glb veney-ang nā //
		\glc dog-\Aarg{} \Fsg{}.\Gen{} //
		\glft `my dog' //
	\endgl

	\a\begingl
		\gla eda- @ veneyang nā //
		\glb eda= veney-ang nā //
		\glc this= dog-\Aarg{} \Fsg{}.\Gen{} //
		\glft `this dog of mine' //
	\endgl
\xe

While it has been argued that Ayeri does not possess articles, it does possess
a large variety of pronouns. These, as pro-forms, appear in complementary
distribution with NPs. Since they encode morphosyntactic functions rather than
semantic content, they are ideal candidates for heads of DP. If DPs in Ayeri
only consist of pronouns, which replace NPs whole, they should not take any
modifiers---with the notable exception of relative clauses, CP being another
functional category. Thus, the phrase structure of DPs should look as
illustrated in (\ref{ex:dpstruct}) and (\ref{ex:dpcstruct}). In both cases, the
CP constituent is in brackets since it is optional, that is, it is an adjunct
rather than a complement. The functional annotation identifies it as such.

\pex\label{ex:dpstruct}
\a DP → \anno*{\xbar{D}}
\a \xbar{D} → \anno*{\xhead{D}} $\left(\anno*[{\pass{\Adjc}}]{CP}\right)$
\xe

\ex~\label{ex:dpcstruct}
\begin{forest}
[{\anno[\{\pass{df} | \pass{gf}\}]{DP}}
	[\anno{\xbar{D}}
		[\anno{\xhead{D}}]
		[{$\left(\anno[{%
				\pass{\Adjc}%
			}]{CP}\right)$
		}]
	]
]
\end{forest}
\xe

\subsubsection{Personal pronouns}

The morpholexic specifications for personal pronouns are given in 
(\ref{ex:perspromorphlex}). Personal pronouns, as a functional category, are a
closed class of words; the chart of personal pronouns in Ayeri is given in
\autoref{subsec:perspro}. Since personal pronouns are pro-forms, they do not
have lexical content for a predicator, but only `pro'. Pronouns distinguish, of
course, all grammatical categories of nouns---number, gender and animacy, and
case. In addition to this, there is person. The reflexive clitic
\rayr{sitNF/}{sitang-} also adds reflexivity as a semantic feature.

\ex\label{ex:perspromorphlex}
\adjustbox{valign=t}{%
\begin{tabu} {\usetabu{morphlexnarrow}}
...
	& N
	& \begin{tabular}[t]{l l l}
		\ups{\Pred} & = & `pro' \\
		\ups{\Anim} & = & $\pm$ \\
		\ups{\Case} & = & \{\Aarg{}, \Parg{}, \Dat{}, \Gen{}, \Loc{}, \Ins{}, 
			\Caus{}\} \\
		\ups{\Gend} & = & \{\M{}, \F{}, \N{}, \Inan{}\} \\
		\ups{\Num} & = & \{\Sg{}, \Pl{}\} \\
		\ups{\Pers} & = & \{\First{}, \Second{}, \Third{}\} \\
		\ups{\Refl} & = & $\pm$ \\
	\end{tabular}
\end{tabu}%
}
\xe

Personal pronouns cannot be modified by adjectives and most other adjuncts
nouns can be modified by, however, they may still be modified by relative
clauses as well as clitic quantifiers. Modification by a quantifier clitic is
only possible for personal pronouns when they are free morphemes, though;
pronominal suffixes are exempt from this case, as described in
\autoref{subsec:reflrec}. The examples in (\ref{ex:nonounmods}) show what is 
not possible as compared to nouns.

\pex\label{ex:nonounmods}
\a\ljudge* %
	\begin{minipage}[t]{.5\linewidth}
	\begingl
		\glpreamble pronoun + adjective: //
		\gla yāng hiro //
		\glb yāng hiro //
		\glc \TsgM{}.\Aarg{} new //
		\glft `new he' //
	\endgl
	\end{minipage}
	~
	\begin{forest} shorter edges,
	[DP
		[\anno{\xbar{D}}
			[\anno{\xhead{D}}
				[{yāng}]
			]
			[{\anno[\pass{\Adjc}]{AP}}
				[{hiro}, roof]
			]
		]
	]
	\end{forest}

\a\ljudge* %
	\begin{minipage}[t]{.5\linewidth}
	\begingl
		\glpreamble pronoun + genitive attribute: //
		\gla reng ayonena //
		\glb reng ayon-ena //
		\glc \TsgI{}.\Aarg{} man-\Gen{} //
		\glft `the man's it' //
	\endgl
	\end{minipage}
	~
	\begin{forest} shorter edges,
	[DP
		[\anno{\xbar{D}}
			[\anno{\xhead{D}}
				[{reng}]
			]
			[{\anno[\pass{\Adjc}]{NP}}
				[{ayonena}, roof]
			]
		]
	]
	\end{forest}

\a\ljudge* %
	\begin{minipage}[t]{.5\linewidth}
	\begingl
		\glpreamble pronoun + instrumental attribute: //
		\gla nang bariri //
		\glb nang bari-ri //
		\glc \Fpl{}.\Aarg{} meat-\Ins{} //
		\glft `we of meat' //
	\endgl
	\end{minipage}
	~
	\begin{forest} shorter edges,
	[DP
		[\anno{\xbar{D}}
			[\anno{\xhead{D}}
				[{nang}]
			]
			[{\anno[\pass{\Adjc}]{NP}}
				[{bariri}, roof]
			]
		]
	]
	\end{forest}

\xe

As mentioned above, it is possible for pronouns to be modified by relative
clauses, which is illustrated by (\ref{ex:pronounrelc}). In this example, the
pronoun \xayr{yeNF}{yeng}{she} is modified by the relative clause \xayr{si
mino}{si mino}{who is happy}.

\ex\label{ex:pronounrelc} %
	\begin{minipage}[t]{.5\linewidth}
	\begingl
		\glpreamble pronoun + relative clause: //
		\gla yeng si mino //
		\glb yeng si mino //
		\glc \TsgF{}.\Aarg{} \Rel{} happy //
		\glft `she who is happy' //
	\endgl
	\end{minipage}
	~
	\begin{forest} shorter edges,
	[DP
		[\anno{\xbar{D}}
			[\anno{\xhead{D}}
				[{yeng}]
			]
			[{\anno[\pass{\Adjc}]{CP}}
				[{si mino}, roof]
			]
		]
	]
	\end{forest}
\xe

\subsubsection{Demonstrative pronouns}

...

\subsubsection{Indefinite pronouns}

...

\subsubsection{Reciprocal pronouns}

...
	
% \needspace{3\baselineskip}
% \a Demonstrative pronouns:\medskip

% 	\begin{tabu} {\usetabu{morphlex}}
% 	(various)
% 		& N
% 		& \begin{tabular}[t]{l l l}
% 			\ups{\Pred} & = & `pro' \\
% 			\ups{\Pers} & = & 3 \\
% 			\ups{\Prox} & = & $\pm$ \\
% 			\ups{\Dist} & = & $\pm$ \\
% 			\ups{\Def} & = & $\pm$ \\
% 			\ups{\Spec} & = & $+$ \\
% 			\ups{\Anim} & = & $\pm$ \\
% 			\ups{\Case} & = & \{\Aarg{}, \Parg{}, \Dat{}, \Gen{}, 
% 				\Loc{}, \Ins{}, \Caus{}\} \\
% 		\end{tabular}
% 	\end{tabu}

% \needspace{3\baselineskip}
% \a Indefinite pronouns:\medskip

% 	\begin{tabu} {\usetabu{morphlex}}
% 	(various)
% 		& N
% 		& \begin{tabular}[t]{l l l}
% 			\ups{\Pred} & = & `pro' \\
% 			\ups{\Pers} & = & 3 \\
% 			\ups{\Def} & = & $-$ \\
% 			\ups{\Spec} & = & $-$ \\
% 			\ups{\Anim} & = & $\pm$ \\
% 			\ups{\Case} & = & \{\Aarg{}, \Parg{}, \Dat{}, \Gen{}, 
% 				\Loc{}, \Ins{}, \Caus{}\} \\
% 		\end{tabular}
% 	\end{tabu}
	
% % \needspace{3\baselineskip}
% % \a\label{ex:dmorphlex-propn}Case markers of proper nouns:\medskip	
	
% % 	\begin{tabu} {\usetabu{morphlex}}
% % 	(various)
% % 		& D
% % 		& \begin{tabular}[t]{l l l}
% % % 			\ups{\Pers} & \req{} & \Third \\
% % % A 3rd person must not be *required* if the topic particle is in fact a D head 
% % % in Spec NP, since pronouns as well can serve as topics without being 3rd 
% % % persons!
% % 			\ups{\Anim} & \req{} & $\pm$ \\
% % 			\ups{\Case} & = & \{\Aarg{}, \Parg{}, \Dat{}, \Gen{}, 
% % 				\Loc{}, \Ins{}, \Caus{}\} \\
% % 		\end{tabular}
% % 	\end{tabu}\medskip
	
% \xe

% Demonstrative prefixes, the inspecificity prefix \rayr{me/}{mə-}, as well as
% proper-noun case markers cannot be without an NP complement, which is what
% `\req{}' is supposed to express in the feature specification: the demonstrative
% prefixes require that an element exist which encodes a third-person referent to
% are assumed to encode third person by default.\footnote{A simple `='
% \emph{defines} a value; a subscript `c' indicates that the morpheme
% \emph{requires} that this value be present. Thus `\req{}' expresses a
% \fw{constraining} equation \citep[59--61]{bresnan2016}.} On the other hand, 
% personal and demonstrative pronouns inherently define information on person 
% (\Pers{}). Since there is a great number of both personal and demonstrative 
% pronouns, only the various values they can assume are indicated, without 
% ensuring that the combinations are actually possible (compare 
% \autoref{subsec:perspro}). Personal pronouns, for instance, only distinguish 
% \Inan{} as set against $\{\M{}, \F{}, \N{}\}$, which are subgroups of \An{}. 
% Gender is also not distinguished in all persons, but only in the third.

% \section{Adpositional phrases}
%
% \section{Inflectional, verb, and complementizer phrases}
%
% * Ayeri's unmarked word order is essentially VSO, maaaaaaaybe like this???
%
%   IP
%     \
%      I'
%     /  \
%    D°   I'
%        /  \
%       I°   S
%           / \
%          NP XP*
% 
%   → Analysis similar to the one suggested for Irish by ??? and ??? 
%     (Kroeger 1991, 1993; Bresnan 2016)
%   → Probably one has to distinguish clauses with IP from `small clauses' 
% 	  (cf. exposition to Kroeger 1991) with only S
%   → S is likely an exocentric category
% 	  → support/tests for this?
% 	  → Kroeger 1991 should list some criteria in this respect
% * Subject vs. Topic vs. Actor vs. Nominative in Ayeri, in relation to
%   superficial (?) similarities with Austronesian alignment (Kroeger 1991,
%   1993, 2007; Schachter 2015)
%   → Promoting patient in whatever way does not demote logical subject etc.
%   → Ayeri's notion of A = logical subject (also what LFG calls `SUBJ', right?)
%     probably stronger than Tagalog's
%   → PROBLEM: terminology all about the place---may have inspired differences 
%     between Ayeri and real-world AA + my only now beginning to learn more
%     about syntax and stuff
%     → IMPORTANT: how do I use the respective terms and why?
% * Bresnan (2016) on extended head principle relevant w/r/t placement of 
%   finite verb in I°?
%   → Are there any tests for whether there's a covert VP after all? Nothing
%     verbal ever seems to occur in S'
%   → Ayeri should treat logical subjects and objects similarly, so no 
%     hierarchy between them?
% * Interesting problem: subject pronoun citics as a specifier of S?
%   → Funny stuff happens when adverbs are present
%   → Especially with patient-pronoun subjects (if that term is appropriate)
% * Patient subjects, causative constructions, no secundative, avoidance (?) of 
%   raising all need to be investigated with regards to proto-role mapping in 
%   a-structure ([±r, ±o])
% * Existential vs. predicative statements
%   → also, object predicatives!
%   → also, comparative verbs: kama-, eng-, va- and why they're weird
% * Agreement with conjoined/disjoined NPs
%   → due to the way verb agreement in general works, closest-conjunct agreement
%     is the most likely strategy for gender mismatches; yet number resolution.
%   → default gender for anaphoric reference to mixed-gender NPs (also animacy
%     mismatches!)
%
% \citet[153--156, 165]{bresnan2016} → pronouns to clitics to agreement
% \citet[157]{bresnan2016} → evidence criteria for IP
% \citet[151\psqq]{dixon2012} → subjects
% \citet[116\psqq]{comrie1989} → subjects
% \citet[26--54]{kroeger1991} → subjects (and actually the whole thesis)
%
