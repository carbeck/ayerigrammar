% kate: word-wrap true;

% \begin{pycode}[env1]
% import time, hashlib
% m = hashlib.md5()
% m.update(str(time.time()).encode('utf-8'))
% c = m.hexdigest()[1:8]
% \end{pycode}

\begin{minipage}[b][\textheight][b]{0.67\textwidth}\small
\ccbysa~Carsten Becker, \the\year.\\
Published under \textsc{cc-by-sa} 4.0 license.\\
Last edited: \today{}.\\[.5\baselineskip]

%Set in Junicode and {\sffamily Fira Sans} with \XeTeX{}.\\[.5\baselineskip]

Ayeri is a fictional language spoken by fictional people in a fictional setting, 
and as such is not related to any naturally existing languages. It is thus not 
to be confused with \emph{Azeri}, a Turkic language spoken in Azerbaijan and its 
surrounding countries. Ayeri’s vocabulary is entirely a priori, this means, no 
real-world languages have been used specifically as sources of vocabulary. Due 
to the language’s sound and spelling aesthetic being inspired by Austronesian 
languages, it is not surprising if overlaps with existing words in those 
languages happen accidentally.\\[.5\baselineskip]

\makebox[1.5em][c]{\faicon{globe}}~\href{http://benung.nfshost.com}{%
http://benung.nfshost.com}\\
%
\makebox[1.5em][c]{\faicon{cogs}}~\href{https://github.com/carbeck/ayerigrammar}
{https://github.com/carbeck/ayerigrammar/}\\
%
\makebox[1.5em][c]{\faicon{balance-scale}}~\href{%
https://creativecommons.org/licenses/by-sa/4.0/}%
{https://creativecommons.org/licenses/by-sa/4.0/}%\\

% \begin{center}\tiny
% \texttt{\pyc[env1]{print(c.strip())}}
% \end{center}
\end{minipage}
