% kate: word-wrap true;

~\vfill
{\setlength\parindent{0pt}
\ccbysa~Carsten Becker, 2017.\\
Published under \textsc{cc-by-sa} 4.0 license.\\
Last edited: \today{}.\\[.5\baselineskip]

Set in Junicode and {\sffamily Fira Sans} with \XeTeX{}.\\[.5\baselineskip]

Ayeri is a fictional language spoken by fictional people in a fictional
setting, and as such is not related to any naturally existing languages. It is
thus not to be confused with \emph{Azeri}, a Turkic language spoken in
Azerbaijan and its surrounding countries. Ayeri’s vocabulary is entirely \fw{a
priori}, this means, no real-world languages have been used specifically as
sources of vocabulary. Ayeri is also not derived from any specific real-world
language family by means of sound changes. Due to Ayeri's sound and spelling
aesthetic being inspired by Austronesian languages, however, occasional
overlaps with words existing in these may happen, but only accidentally
so.\\[.5\baselineskip]

\begin{tabular}{@{} c @{\enspace} l}
\faicon{globe}
& \href{https://ayeri.de}{https://ayeri.de}\\
\faicon{cogs}
& \href{https://github.com/carbeck/ayerigrammar}
	{https://github.com/carbeck/ayerigrammar/}\\
\faicon{balance-scale}
& \href{https://creativecommons.org/licenses/by-sa/4.0/}%
	{https://creativecommons.org/licenses/by-sa/4.0/}%\\
\end{tabular}
}
