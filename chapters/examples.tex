\chapter{Example Texts}

\section{The North Wind and the Sun}
\citep[From][]{becker:northwind}

\blockcquote[39]{ipa2007}{The North Wind and the Sun were disputing which was the stronger, when a traveler came along wrapped in a warm cloak. They agreed that the one who first succeeded in making the traveller take his cloak off should be considered stronger than the other. Then the North Wind blew as hard as he could, but the more he blew the more closely did the traveller fold his cloak aroand him; and at last the North Wind gave up the attempt. Then the Sun shone out warmly, and immediately the traveller took off his cloak. And so the North Wind was obliged to confess that the Sun was the stronger of the two.}

\noindent{\Tagati ANF mN rnFyonF Adauyi piMtemisF nj perinF, ENFyo mikYo
sinYaaNF lug toy, liNFy si lugy AsaaːyNF si sitNF/njkonFyaaNF koNF tovy mto.
skMtoNF, ENoNFyo mikYo dnYaasF pluNF mennNF sirii ANF phoNFy Asaay tovlej yn.
ANF gihyo piM\-temisF miNneri/henF yon. nj gi\-h\-yoNF mikYo nj mikYo/ENF, nj
ANF d/njkonFy rdo nj rdo/ENF Asaay tovlej yn. su\-bFrFyo dermFymF ANF
piMtemisF. kYunFyo mkymF mto EpNF ANF perinF, nj ANF phFy EdauyiknF Asaay
tovlej yn. kd ru\_a beNFyo ANF piMtemisF, ANF ENFyo mikYo kYuymF perinF lug toy
smF.}\medskip

{\itshape \noindent Ang manga ranyon adauyi Pintemis nay Perin, engyo mico
sinyāng luga toya, lingya si lugaya asāyāng si sitang-naykonyāng kong tovaya
mato. Sakantong, engongyo mico danyās palung menanang sirī ang pahongya asāya
tovaley yana. Ang gihayo Pintemis minganeri-hen yona. Nay gihayong mico nay
mico-eng, nay ang da-naykonya rado nay rado-eng asāya tovaley yana. Subryo
deramyam ang Pintemis. Cunyo makayam mato epang ang Perin, nay ang pahya
edauyikan asāya tovaley yana. Kada rua bengyo ang Pintemis, ang engyo mico
cuyam Perin luga toya sam.}\medskip

\ex % The North Wind and the Sun were disputing ...
\begingl
	\gla Ang manga ranyon adauyi {} Pintemis nay Perin, engyo mico sinyāng luga toya, lingya si lugaya asāyāng si sitang-naykonyāng kong tovaya mato. //
	\glb ang manga ran-yon adauyi Ø Pintemis nay Perin, eng-yo mico sinya-ang luga toya, ling-ya si luga-ya asāya-ang si sitang=naykon-yāng kong tova-ya mato. //
	\glc \AgtT{} \Prog{} argue-\Tpl{}.\N{} then \Top{} North.Wind and Sun, be.more-\Tsg{}.\N{} strong who-\Aarg{} among \Tpl{}.\N{}.\Loc{}, while-\Loc{} \Rel{} pass-\Tsg{}.\M{} traveler-\Aarg{} \Rel{} self=wrap-\Tsg{}.\M{}.\Aarg{} inside cloak-\Loc{} warm. //
	\glft `The North Wind and the Sun were then arguing which among them is stronger, all the while a traveler passed by who had wrapped himself in a warm cloak.' //
\endgl
\xe

\ex~ % They agreed that the one ...
\begingl
	\gla Sakantong, engongyo mico danyās palung menanang sirī ang pahongya asāya tovaley yana. //
	\glb sakan=tong eng-ong-yo mico danya-as palung menan-ang si-ri<i> ang pah-ong-ya asāya-Ø tova-ley yana //
	\glc agree=\Tpl{}.\N{} be.more-\Irr{}-\Tsg{}.\N{} strong one-\Parg{} other first-\Aarg{} \Rel{}<-\Aarg{}>-\Caus{} \AgtT{} remove-\Irr{}-\Tsg{}.\M{} traveler-\Top{} cloak-\Parg{}.\Inan{} \Tsg{}.\M{}.\Gen{} //
	\glft `They agreed that the first one due to whom the traveller would take off his cloak would be stronger than the other.' //
\endgl
\xe

\pex~ % Then the North Wind blew ...
\a \begingl
	\gla Ang gihayo {} Pintemis minganeri-hen yona. //
	\glb ang giha-yo Ø Pintemis mingan-eri=hen yona //
	\glc \AgtT{} blow-\Tsg{}.\N{} \Top{} North.Wind ability-\Ins{}=all \Tsg{}.\N{}.\Gen{} //
	\glft `The North Wind blew with all of his might.' //
\endgl
\a \begingl
	\gla Nay gihayong mico nay mico-eng, nay ang da-naykonya rado nay rado-eng asāya tovaley yana. //
	\glb nay giha=yong mico nay mico=eng nay ang da=naykon-ya rado nay rado=eng asāya-Ø tova-ley yana //
	\glc and blow=\Tsg.\N{}.\Aarg{} strong and strong=\Comp{} and \AgtT{} so=wrap-\Tsg{}.\M{} tight and tight=\Comp{} traveler-\Top{} cloak-\Parg{}.\Inan{} \Tsg{}.\M{}.\Gen{}. //
	\glft `And it blew harder and harder, and the traveller so wrapped his cloak tighter and tighter.' //
\endgl

\a \begingl
	\gla Subryo deramyam ang Pintemis. //
	\glb Subr-yo deramyam ang Pintemis //
	\glc give.up-\Tsg{}.\N{} {after.all} \Aarg{} North.Wind //
	\glft `The North Wind gave up after all.' //
\endgl
\xe

\ex~ % Then the Sun shined out warmly ...
\begingl
	\gla Cunyo makayam mato epang ang Perin, nay ang pahya edauyikan asāya tovaley yana. //
	\glb cun-yo maka-yam mato epang ang Perin nay ang pah-ya edauyikan asāya-Ø tova-ley yana //
	\glc begin-\Tsg{}.\N{} shine-\Ptcp{} warm next \Aarg{} Sun and \AgtT{} remove-\Tsg{}.\M{} immediately traveler-\Top{} cloak-\Parg{}.\Inan{} \Tsg{}.\M.\Gen{} //
	\glft `Next, the Sun began to shine warmly, and the traveler immediately took off his cloak.' //
\endgl
\xe

\ex~ % And so the North Wind was obliged ...
\begingl
	\gla Kada rua bengyo ang Pintemis, ang engyo mico cuyam {} Perin luga toya sam. //
	\glb kada rua beng-yo ang Pintemis ang eng-yo mico cuyam Ø Perin luga toya sam //
	\glc thus must admit-\Tsg{} \Aarg{} North.Wind \AgtT{} be.more-\Tsg{}.\N{} strong indeed \Top{} Sun among \Tpl{}.\N{}.\Loc{} two //
	\glft `Thus the North Wind had to admit that the Sun was indeed the stronger among both of them.' //
\endgl
\xe

\section{The Fox and the Rooster}
\excnt=1

\citep[Adapted from][]{becker:uebersetzungsaufgabe}\medskip

\blockquote{Once upon a time, a hungry Fox came to a village. He said to the
Rooster: \enquote{Let me hear your beautiful voice!} The proud Rooster closed
his eyes and crowed loudly. There the Fox grabbed him and and carried him into
the forest. When the farmers noticed this, they ran after the Fox and cried
\enquote{The Fox is carrying away our Rooster!} There the Rooster said to the
Fox: \enquote{Tell them, \enquote{I am carrying my own Rooster and not yours!}}
The Fox released the Rooster from his mouth and called \enquote{I am carrying
my own Rooster and not yours!} There, however, the Rooster quickly flew onto a
tree. The Fox chided himself to be a fool and trotted off. (After Aesop)}

\noindent{\Tagati me/bhisFy, ANF shy runj mbo miMkjy. ANF nry AguynFy — gru, s
miNF tNFyNF kdaare sekj veno vn! ANF rimy AguynF viyu nivye\_asF yn nj gryaaNF
bho. s d/ktYisy runyNF y nj s ninFyaaNF y mN koNF vinimFy. tdjy si ANF keNFynF
bedNFye Adlej, ANF niMpFynF mN pNF runjy nj bhtNF — ANF mN phFy runj AguynsF
nn! nj ANF nry AguynF runjy – niNu tYmF – s ninFyNF AguynF naa – ninojyNF
d/vn. ANF bomFy runj AguynsF bMtn yn nj gryaaNF – s ninFyNF AguynF naa –
ninojyNF d/vn. ANF nuny pr naarY AguynF mN liNF mehirFy. sitNF/gsiy runyNF,
yaaNF depNsF, nj lMpFyaaNF mNsr.}\medskip

{\itshape \noindent Mə-bahisya, ang sahaya runay mabo minkayya. Ang naraya
aguyanya: \enquote{Garu, sa ming tangyang kadāre sekay veno vana!} Ang rimaya
aguyan viyu nivajas yana nay garayāng baho. Sa da-kacisaya runayang ya nay sa
ninyāng ya manga kong vi\-nim\-ya. Tadayya si ang kengyan bedangye adaley, ang
nimpyan manga pang runayya nay bahatang: \enquote{Ang manga pahya runay
aguyanas nana!} Nay ang naraya aguyan runayya: \enquote{Ningu cam: \enquote{Sa
ninyang aguyan nā; ninoyyang da-vana.}} Ang bomya runay aguyanas bantana yana
nay garayāng: \enquote{Sa ninyang aguyan nā; ninoyyang da-vana.} Ang nunaya
para nārya aguyan manga ling mehirya. Sitang-gasiya runayang, yāng depangas,
nay lampyāng mangasara.}\medskip

\pex % 1
\a\begingl
	\gla Mə-bahisya, ang sahaya runay mabo minkayya. //
	\glb mə=bahis-ya ang saha-ya runay-Ø mabo minkay-ya //
	\glc some=day-\Loc{} \AgtT{} come-\TsgM{} Fox-\Top{} hungry village-\Loc{} //
	\glft \enquote{Some day a hungry Fox came to a village.} //
\endgl

\a\begingl
	\gla Ang naraya aguyanya: Garu, sa ming tangyang kadāre sekay veno vana! //
	\glb ang nara=ya.Ø aguyan-ya gara-u sa ming tang=yang kadāre sekay-Ø veno vana //
	\glc \AgtT{} speak=\TsgM{}.\Top{} Rooster-\Loc{} call-\Imp{} \PatT{} can hear=\Fsg{}.\Aarg{} so.that voice-\Top{} beautiful \Second{}.\Gen{} //
	\glft \enquote{He spoke to a Rooster: \enquote{Call, so that I can hear your beautiful voice!}} //
\endgl

\xe

\pex~ % 2
\a\begingl
	\gla Ang rimaya aguyan viyu nivajas yana nay garayāng baho. //
	\glb ang rima-ya aguyan-Ø viyu niva-ye-as yana nay gara=yāng baho //
	\glc \AgtT{} close-\TsgM{} Rooster-\Top{} proud eye-\Pl{}-\Parg{} \TsgM{}.\Gen{} and call=\TsgM{}.\Aarg{} loudly //
	\glft \enquote{The proud Rooster closed his eyes and crowed loudly.} //
\endgl

\a\begingl
	\gla Sa da-kacisaya runayang ya nay sa ninyāng ya manga kong vinimya. //
	\glb sa da=kacisa-ya runay-ang ya.Ø nay sa nin=yāng ya.Ø manga kong vinim-ya //
	\glc \PatT{} so=grab-\TsgM{} Fox-\Aarg{} \TsgM{}.\Top{} and \PatT{} carry=\TsgM{}.\Aarg{} \TsgM{}.\Top{} \Dyn{} in forest-\Loc{} //
	\glft \enquote{There he was grabbed by the Fox and carried to the forest by him.} //
\endgl

\xe

\pex~ % 3
\a\begingl
	\gla Tadayya si ang kengyan bedangye adaley, ang nimpyan manga pang runayya nay bahatang: //
	\glb taday-ya si ang keng-yan bedang-ye-Ø ada-ley ang nimp=yan.Ø manga pang runay-ya nay nay baha=tang //
	\glc time-\Loc{} \Rel{} \AgtT{} notice-\TplM{} farmer-\Pl{}-\Top{} that-\PargI{} \AgtT{} rennen=\TplM{}.\Top{} \Dyn{} hinter Fox-\Loc{} and cry.out=\TplM{}.\Aarg{} //
	\glft \enquote{As the farmers noticed, they ran after the Fox and cried out:} //
\endgl

\a\begingl
	\gla Ang manga pahya runay aguyanas nana! //
	\glb ang manga pah-ya runay-Ø aguyan-as nana //
	\glc \AgtT{} \Prog{} take.away-\TsgM{} Fox-\Top{} Rooster-\Parg{} \Fsg{}.\Gen{} //
	\glft \enquote{The Fox is taking our Rooster away!} //
\endgl

\xe

\pex~ % 4
\a\begingl
	\gla Nay ang naraya aguyan runayya: Ningu cam: //
	\glb nay ang nara-ya aguyan-Ø runay-ya Ning-u cam //
	\glc and \AgtT{} speak-\TsgM{} Rooster-\Top{} Fox-\Loc{} say-\Imp{} \TplM{}.\Dat{} //
	\glft \enquote{And the Rooster said to the Fox: \enquote{Tell them:}} //
\endgl

\a\label{ex:negativbindung}\begingl
	\gla Sa ninyang aguyan nā; ninoyyang da-vana. //
	\glb sa nin=yang aguyan-Ø nā nin-oy=yang da=vana //
	\glc \PatT{} carry=\Fsg{}.\Aarg{} Rooster-\Top{} \Fsg{}.\Gen{} carry-\Neg{}=\Fsg{}.\Aarg{} so=\Spl{}.\Gen{} //
	\glft \enquote{I am carrying my own Rooster; I am not carrying yours.} //
\endgl

\xe

\pex~ % 5
\a\begingl
	\gla Ang bomya runay aguyanas bantana yana nay garayāng: //
	\glb ang bom-ya runay-Ø aguyan-as banta-na yana nay gara=yāng //
	\glc \AgtT{} release-\TsgM{} Fox-\Top{} Rooster-\Parg{} mouth-\Gen{} \TsgM{}.\Gen{} and call=\TsgM{}.\Aarg{} //
	\glft \enquote{The Fox released the Rooster from his mouth and called:} //
\endgl

\a\begingl
	\gla Sa ninyang aguyan nā; ninoyyang da-vana. //
	\glb sa nin=yang aguyan-Ø nā nin-oy=yang da=vana //
	\glc \PatT{} carry=\Fsg{}.\Aarg{} Rooster-\Top{} \Fsg{}.\Gen{} carry-\Neg{}=\Fsg{}.\Aarg{} so=\Spl{}.\Gen{} //
	\glft \enquote{I am carrying my own Rooster; I am not carrying yours.} //
\endgl

\xe

\pex~ % 6
\a\begingl
	\gla Ang nunaya para nārya aguyan manga ling mehirya. //
	\glb ang nuna-ya para nārya aguyan-Ø manga ling mehir-ya //
	\glc \AgtT{} fly-\TsgM{} quickly though Rooster-\Top{} \Dyn{} on tree-\Loc{} //
	\glft \enquote{The Rooster, though, quickly flew onto a tree.} //
\endgl

\a\label{ex:objpred}\begingl
	\gla Sitang-gasiya runayang, yāng depangas, nay lampyāng mangasara. //
	\glb sitang=gasi-ya runay-ang yāng depang-as nay lamp=yāng mangasara //
	\glc \Refl{}=chide-\TsgM{} Fox-\Aarg{} \TsgM{}.\Aarg{} fool-\Parg{} and walk=\TsgM{}.\Aarg{} away //
	\glft \enquote{The Fox chided himself that he would be a fool and walked away.} //
\endgl

\xe
