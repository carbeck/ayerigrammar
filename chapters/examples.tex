\chapter{Example Texts}

\section{The North Wind and the Sun}
\label{sec:northwind}
\citep[From][]{becker:northwind}

\blockcquote[After Aesop;][39]{ipa2007}{The North Wind and the Sun were
disputing which was the stronger, when a traveller came along wrapped in a warm
cloak. They agreed that the one who first succeeded in making the traveller
take his cloak off should be considered stronger than the other. Then the North
Wind blew as hard as he could, but the more he blew the more closely did the
traveller fold his cloak aroand him; and at last the North Wind gave up the
attempt. Then the Sun shone out warmly, and immediately the traveller took off
his cloak. And so the North Wind was obliged to confess that the Sun was the
stronger of the two.}

\begin{flushleft}
\noindent\Tagati ANF mN rnFyonF Adauyi piMtemisF nj perinF, ENFyo mikYo
sinYaaNF lug toy, liNFy si lugy AsaaːyNF si sitNF/njkonFyaaNF koNF tovy mto.
skMtoNF, ENoNFyo mikYo dnYaasF pluNF mennNF sirii ANF phoNFy Asaay tovlej yn.
ANF gihyo piM\-temisF miNneri/henF yon. nj gi\-h\-yoNF mikYo nj mikYo/ENF, nj
ANF d/njkonFy rdo nj rdo/ENF Asaay tovlej yn. su\-bFrFyo dermFymF ANF
piMtemisF. kYunFyo mkymF mto EpNF ANF perinF, nj ANF phFy EdauyiknF Asaay
tovlej yn. kd ru\_a beNFyo ANF piMtemisF, ANF ENFyo mikYo kYuymF perinF lug toy
smF.
\end{flushleft}

\noindent {\itshape Ang manga ranyon adauyi Pintemis nay Perin, engyo mico
sinyāng luga toya, lingya si lugaya asāyāng si sitang-naykonyāng kong tovaya
mato. Sakantong, engongyo mico danyās palung menanang sirī ang pahongya asāya
tovaley yana. Ang gihayo Pintemis minganeri-hen yona. Nay gihayong mico nay
mico-eng, nay ang da-naykonya rado nay rado-eng asāya tovaley yana. Subryo
deramyam ang Pintemis. Cunyo makayam mato epang ang Perin, nay ang pahya
edauyikan asāya tovaley yana. Kada rua bengyo ang Pintemis, ang engyo mico
cuyam Perin luga toya sam.}\medskip

\ex % The North Wind and the Sun were disputing ...
\begingl
	\gla Ang manga ranyon adauyi {} Pintemis nay Perin, engyo mico sinyāng 
		luga toya, lingya si lugaya asāyāng si sitang-naykonyāng kong tovaya
		mato. //
	\glb ang manga ran-yon adauyi Ø Pintemis nay Perin eng-yo mico sinya-ang
		luga toya ling-ya si luga-ya asāya-ang si sitang=naykon-yāng kong
		tova-ya mato //
	\glc \AgtT{} \Prog{} argue-\TplN{} then \Top{} North.Wind and Sun
		be.more-\TsgN{} strong who-\Aarg{} among \TplN{}.\Loc{}
		while-\Loc{} \Rel{} pass-\TsgM{} traveler-\Aarg{} \Rel{}
		self=wrap-\TsgM{}.\Aarg{} inside cloak-\Loc{} warm //
	\glft `The North Wind and the Sun were then arguing which among them is
		stronger, all the while a traveler passed by who had wrapped himself in
		a warm cloak.' //
\endgl
\xe

\ex~ % They agreed that the one ...
\begingl
	\gla Sakantong, engongyo mico danyās palung menanang sirī ang pahongya
		asāya tovaley yana. //
	\glb sakan=tong eng-ong-yo mico danya-as palung menan-ang si-ri<i> ang pah-
		ong-ya asāya-Ø tova-ley yana //
	\glc agree=\TplN{} be.more-\Irr{}-\TsgN{} strong one-\Parg{} other
		first-\Aarg{} \Rel{}<-\Aarg{}>-\Caus{} \AgtT{}
		remove-\Irr{}-\TsgM{} traveler-\Top{} cloak-\PargI{}
		\TsgM{}.\Gen{} //
	\glft `They agreed that the first one due to whom the traveler would take
		off his cloak would be stronger than the other.' //
\endgl
\xe

\pex~ % Then the North Wind blew ...
\a \begingl
	\gla Ang gihayo {} Pintemis minganeri-hen yona. //
	\glb ang giha-yo Ø Pintemis mingan-eri=hen yona //
	\glc \AgtT{} blow-\TsgN{} \Top{} North.Wind ability-\Ins{}=all
		\TsgN{}.\Gen{} //
	\glft `The North Wind blew with all of his might.' //
\endgl
\a \begingl
	\gla Nay gihayong mico nay mico-eng, nay ang da-naykonya rado nay rado-eng
		asāya tovaley yana. //
	\glb nay giha=yong mico nay mico=eng nay ang da=naykon-ya rado nay rado=eng
		asāya-Ø tova-ley yana //
	\glc and blow=\TsgN{}.\Aarg{} strong and strong=\Comp{} and \AgtT{}
		so=wrap-\TsgM{} tight and tight=\Comp{} traveler-\Top{}
		cloak-\PargI{} \TsgM{}.\Gen{}. //
	\glft `And it blew harder and harder, and the traveler so wrapped his cloak
		tighter and tighter.' //
\endgl

\a \begingl
	\gla Subryo deramyam ang Pintemis. //
	\glb Subr-yo deramyam ang Pintemis //
	\glc give.up-\TsgN{} {after.all} \Aarg{} North.Wind //
	\glft `The North Wind gave up after all.' //
\endgl
\xe

\ex~ % Then the Sun shined out warmly ...
\begingl
	\gla Cunyo makayam mato epang ang Perin, nay ang pahya edauyikan asāya
		tovaley yana. //
	\glb cun-yo maka-yam mato epang ang Perin nay ang pah-ya edauyikan asāya-Ø
		tova-ley yana //
	\glc begin-\TsgN{} shine-\Ptcp{} warm next \Aarg{} Sun and \AgtT{}
		remove-\TsgM{} immediately traveler-\Top{} cloak-\PargI{} 
		\TsgM{}.\Gen{} //
	\glft `Next, the Sun began to shine warmly, and the traveler immediately
		took off his cloak.' //
\endgl
\xe

\ex~ % And so the North Wind was obliged ...
\begingl
	\gla Kada rua bengyo ang Pintemis, ang engyo mico cuyam {} Perin luga toya
		sam. //
	\glb kada rua beng-yo ang Pintemis ang eng-yo mico cuyam Ø Perin luga toya
		sam //
	\glc thus must admit-\Tsg{} \Aarg{} North.Wind \AgtT{} be.more-\TsgN{}
		strong indeed \Top{} Sun among \TplN{}.\Loc{} two //
	\glft `Thus the North Wind had to admit that the Sun was indeed the
		stronger among both of them.' //
\endgl
\xe

\section{The Fox and the Rooster}
\label{sec:foxrooster}
\excnt=1

\citep[Adapted from][]{becker:uebersetzungsaufgabe}\medskip

\blockquote{Once upon a time, a hungry fox came to a village. He said to the
rooster: \enquote{Let me hear your beautiful voice!} The proud rooster closed
his eyes and crowed loudly. There the fox grabbed him and and carried him into
the forest. When the farmers noticed this, they ran after the fox and cried
\enquote{The fox is carrying away our rooster!} There the rooster said to the
fox: \enquote{Tell them, \enquote{I am carrying my rooster and not yours!}} The
fox released the rooster from his mouth and called \enquote{I am carrying my
rooster and not yours!} There, however, the rooster quickly flew onto a tree.
The fox called himself a fool and trotted off. (After Aesop)}

\begin{flushleft}
\Tagati me/bhisFy, ANF shy runj mbo miMkjy. ANF nry AguynFy — gru, s
miNF tNFyNF kdaare sekj veno vn! ANF rimy AguynF viyu nivye\_asF yn nj gryaaNF
bho. s d/ktYisy runyNF y nj s ninFyaaNF y mN koNF vinimFy. tdjy si ANF keNFynF
bedNFye Adlej, ANF niMpFynF mN pNF runjy nj bhtNF — ANF mN phFy runj AguynsF
nn! nj ANF nry AguynF runjy – niNu tYmF – s ninFyNF AguynF naa – ninojyNF
d/vn. ANF bomFy runj AguynsF bMtn yn nj gryaaNF – s ninFyNF AguynF naa –
ninojyNF d/vn. ANF nuny pr naarY AguynF mN liNF mehirFy. sitNF/gsiy runyNF,
yaaNF depNsF, nj lMpFyaaNF mNsr.
\end{flushleft}

\noindent {\itshape Mə-bahisya, ang sahaya runay mabo minkayya. Ang naraya
aguyanya: \enquote{Garu, sa ming tangyang kadāre sekay veno vana!} Ang rimaya
aguyan viyu nivajas yana nay garayāng baho. Sa da-kacisaya runayang ya nay sa
ninyāng ya manga kong vi\-nim\-ya. Tadayya si ang kengyan bedangye adaley, ang
nimpyan manga pang runayya nay bahatang: \enquote{Ang manga pahya runay
aguyanas nana!} Nay ang naraya aguyan runayya: \enquote{Ningu cam: \enquote{Sa
ninyang aguyan nā; ninoyyang da-vana.}} Ang bomya runay aguyanas bantana yana
nay garayāng: \enquote{Sa ninyang aguyan nā; ninoyyang da-vana.} Ang nunaya
para nārya aguyan manga ling mehirya. Sitang-gasiya runayang, yāng depangas,
nay lampyāng mangasara.}\medskip

\pex % 1
\a\begingl
	\gla Mə-bahisya, ang sahaya runay mabo minkayya. //
	\glb mə=bahis-ya ang saha-ya runay-Ø mabo minkay-ya //
	\glc some=day-\Loc{} \AgtT{} come-\TsgM{} fox-\Top{} hungry
		village-\Loc{} //
	\glft `Some day a hungry fox came to a village.' //
\endgl

\a\begingl
	\gla Ang naraya aguyanya: Garu, sa ming tangyang kadāre sekay veno vana! //
	\glb ang nara=ya.Ø aguyan-ya gara-u sa ming tang=yang kadāre sekay-Ø veno
		vana //
	\glc \AgtT{} speak=\TsgM{}.\Top{} rooster-\Loc{} call-\Imp{} \PatT{} can
		hear=\Fsg{}.\Aarg{} so.that voice-\Top{} beautiful \Second{}.\Gen{} //
	\glft `He spoke to a rooster: \enquote{Call, so that I can hear
		your beautiful voice!}' //
\endgl

\xe

\pex~ % 2
\a\begingl
	\gla Ang rimaya aguyan viyu nivajas yana nay garayāng baho. //
	\glb ang rima-ya aguyan-Ø viyu niva-ye-as yana nay gara=yāng baho //
	\glc \AgtT{} close-\TsgM{} rooster-\Top{} proud eye-\Pl{}-\Parg{}
		\TsgM{}.\Gen{} and call=\TsgM{}.\Aarg{} loudly //
	\glft `The proud rooster closed his eyes and crowed loudly.' //
\endgl

\a\begingl
	\gla Sa da-kacisaya runayang ya nay sa ninyāng ya manga kong vinimya. //
	\glb sa da=kacisa-ya runay-ang ya.Ø nay sa nin=yāng ya.Ø manga kong vinim-
		ya //
	\glc \PatT{} so=grab-\TsgM{} fox-\Aarg{} \TsgM{}.\Top{} and \PatT{}
		carry=\TsgM{}.\Aarg{} \TsgM{}.\Top{} \Dyn{} in forest-\Loc{} //
	\glft `There he was grabbed by the fox and carried to the forest by him.' //
\endgl

\xe

\pex~ % 3
\a\begingl
	\gla Tadayya si ang kengyan bedangye adaley, ang nimpyan manga pang runayya
		nay bahatang: //
	\glb taday-ya si ang keng-yan bedang-ye-Ø ada-ley ang nimp=yan.Ø manga pang
		runay-ya nay nay baha=tang //
	\glc time-\Loc{} \Rel{} \AgtT{} notice-\TplM{} farmer-\Pl{}-\Top{}
		that-\PargI{} \AgtT{} run=\TplM{}.\Top{} \Dyn{} behind fox-\Loc{} and
		cry.out=\TplM{}.\Aarg{} //
	\glft `As the farmers noticed, they ran after the fox and cried out:' //
\endgl

\a\begingl
	\gla Ang manga pahya runay aguyanas nana! //
	\glb ang manga pah-ya runay-Ø aguyan-as nana //
	\glc \AgtT{} \Prog{} take.away-\TsgM{} fox-\Top{} rooster-\Parg{}
		\Fsg{}.\Gen{} //
	\glft `The fox is taking our rooster away!' //
\endgl

\xe

\pex~ % 4
\a\begingl
	\gla Nay ang naraya aguyan runayya: Ningu cam: //
	\glb nay ang nara-ya aguyan-Ø runay-ya ning-u cam //
	\glc and \AgtT{} speak-\TsgM{} rooster-\Top{} fox-\Loc{} say-\Imp{}
		\TplM{}.\Dat{} //
	\glft `And the rooster said to the fox: \enquote{Tell them:}' //
\endgl

\a\label{ex:negativbindung}\begingl
	\gla Sa ninyang aguyan nā; ninoyyang da-vana. //
	\glb sa nin=yang aguyan-Ø nā nin-oy=yang da=vana //
	\glc \PatT{} carry=\Fsg{}.\Aarg{} rooster-\Top{} \Fsg{}.\Gen{}
		carry-\Neg{}=\Fsg{}.\Aarg{} so=\Spl{}.\Gen{} //
	\glft `I am carrying my own rooster; I am not carrying yours.' //
\endgl

\xe

\pex~ % 5
\a\begingl
	\gla Ang bomya runay aguyanas bantana yana nay garayāng: //
	\glb ang bom-ya runay-Ø aguyan-as banta-na yana nay gara=yāng //
	\glc \AgtT{} release-\TsgM{} fox-\Top{} rooster-\Parg{} mouth-\Gen{}
		\TsgM{}.\Gen{} and call=\TsgM{}.\Aarg{} //
	\glft `The fox released the rooster from his mouth and called:' //
\endgl

\a\begingl
	\gla Sa ninyang aguyan nā; ninoyyang da-vana. //
	\glb sa nin=yang aguyan-Ø nā nin-oy=yang da=vana //
	\glc \PatT{} carry=\Fsg{}.\Aarg{} rooster-\Top{} \Fsg{}.\Gen{}
		carry-\Neg{}=\Fsg{}.\Aarg{} so=\Spl{}.\Gen{} //
	\glft `I am carrying my own rooster; I am not carrying yours.' //
\endgl

\xe

\pex~ % 6
\a\begingl
	\gla Ang nunaya para nārya aguyan manga ling mehirya. //
	\glb ang nuna-ya para nārya aguyan-Ø manga ling mehir-ya //
	\glc \AgtT{} fly-\TsgM{} quickly though rooster-\Top{} \Dyn{} on
		tree-\Loc{} //
	\glft `The rooster, though, quickly flew onto a tree.' //
\endgl

\a\label{ex:objpred}\begingl
	\gla Sitang-gasiya runayang, yāng depangas, nay lampyāng mangasara. //
	\glb sitang=gasi-ya runay-ang yāng depang-as nay lamp=yāng mangasara //
	\glc \Refl{}=scold-\TsgM{} fox-\Aarg{} \TsgM{}.\Aarg{} fool-\Parg{} and
		walk=\TsgM{}.\Aarg{} away //
	\glft `The fox scolded himself, that he were a fool, and walked 
		away.'\footnotemark //
\endgl

\xe

\footnotetext{This sentence was translated rather literally from the German 
\fw{der Fuchs schalt sich einen Narren}, literally `the fox scolded himself a fool', with \fw{einen Narren} `a fool' as an object-predicative nominal.}

\section{Ozymandias}
\label{sec:ozymandias}

\excnt=1

\citep[Adapted from][]{becker:ozymandias}\medskip

\blockcquote{shelley:ozymandias}{
	\textbf{Ozymandias}\medskip

	I met a traveller from an antique land, \\
	Who said -- ``two vast and trunkless legs of stone \\
	Stand in the desert ... near them, on the sand, \\
	Half sunk a shattered visage lies, whose frown, \\
	And wrinkled lips, and sneer of cold command, \\
	Tell that its sculptor well those passions read \\
	Which yet survive, stamped on these lifeless things, \\
	The hand that mocked them, and the heart that fed; \\
	And on the pedestal these words appear: \\
	My name is Ozymandias, King of Kings, \\
	Look on my Works ye Mighty, and despair! \\
	Nothing beside remains. Round the decay \\
	Of that colossal Wreck, boundless and bare \\
	The lone and level sands stretch far away.'' -- \medskip \\
}

\begin{flushleft}
\Tagati {\large \textbf{simMdYsF}} \\

s peNlYNF Asno similen tdo, ANF \\
nry – nmaaNF smF kaarYo nj trYnFkj \\
beNYonF AdaahlY. y hemyoNF kiyis \\
nsj AdanYF, AhlY, mrinsF Avnu/NsF. \\
ANF niNYonF IgaanF nj nMdiNF diʲgisuF yon \\
nosaansF kilisrY nj sgoymnsF – \\
s lyy bn/IknF tiynYaaNF d/dikunF \\
si telugFyoNF trel, y spFryosF linYye – \\
spysF si sgoyoNF – pdNsF si koMdiʲsoNF. \\
nj s thnYo Ed/ nraanF beNYmnY – \\
grnNF naa \textbf{simMdYsF}, bjhi\_aNF bjhiyen – \\
s silFvu gumo naa, nj pFrisu, vaaNF si lit! \\
hNr rnYreNF pluNF. le ApnisreNF \\
AhlF/nm kebj, pFrj, soy, litoy kjvj, \\
midj nerFnnYee\_a Ed/kiynen nke. \\
\end{flushleft}
\bigskip

\noindent {\itshape 
\textbf{Simanjas}\\

\noindent Sa pengalyang asano similena tado, ang \\ 
naraya: Namāng sam kāryo nay taryankay \\ 
bengyon adāhalya. Ya hemayong kiyisa \\ 
nasay adany', ahalya, marinas avanu-ngas. \\ 
Ang ningyon igān nay nanding dijisu yona \\ 
nosānas kilisarya nay sagoyamanas: \\ 
Sa layaya ban-ikan tiyanyāng da-dikun \\ 
si telujong tarela, ya saprayos linyaye:\footnote{Originally 
	\ques\xayr{telugFtoNF}{\ques{}telugtong}{survive} 
	(survive=\TplN{}.\Aarg{}), which at least presently is a mistake since 
	\xayr{dikunF}{dikun}{passion} is listed in the dictionary as a 
	\fw{singulare tantum}. The same goes for \rayr{spFryosF}{saprayos}, which 
	was plural \ques\xayr{spFrtosF}{\ques{}sapratos}{they are stamped} 
	(stamp=\TplN{}.\Parg{}) before. One word of the line in the original poem 
	was not translated as it did not fit the meter in Ayeri, ``lifeless,'' 
	which may be translated as \xayr{si tenrY}{si tenarya}{which are unalive} 
	(\Rel{} unalive).} \\ 
sapayas si sagoyong; padangas si kondis'yong. \\
Nay sa tahanyo eda-narān bengyamanya: \\ 
Garanang nā \textsc{simanjas,} bayhiang bayhiyena: \\ 
Sa silvu gumo nā, nay prisu, vāng si lita! \\ 
Hangara ranyareng palung. Le apanisareng \\ 
ahal-nama kebay, pray, soya, litoya kayvay, \\ 
miday nernanyēa eda-kiyanena nake. \\
}

\ex\begingl
	\gla Sa pengalyang asano similena tado, ang //
	\glb sa pengal=yang asano-Ø simil-ena tado ang //
	\glc \PatT{} meet=\Fsg{} traveler-\Top{} country-\Gen{} old \AgtT{} //
	\glft `I met a traveler from an old country,' //
\endgl\xe

\ex~\begingl
	\gla naraya: Namāng sam kāryo nay taryankay //
	\glb nara=ya.Ø nama-ang sam kāryo nay taryan-kay //
	\glc say=\TsgM{}.\Top{} leg-\Aarg{} two big and torso-less //
	\glft `he said: Two big and torsoless legs' //
\endgl\xe

\ex~\begingl
	\gla bengyon adāhalya. Ya hemayong kiyisa //
	\glb beng-yon ada=ahal-ya ya hema=yong kiyisa //
	\glc stand-\TplN{} that=desert-\Loc{} \LocT{} lie-\TsgN{}.\Aarg{} 
		shattered //
	\glft `stand in that desert. There lies shattered' //
\endgl\xe

\ex~\begingl
	\gla nasay adany', ahalya, marinas avanu-ngas. //
	\glb nasay adanya-Ø ahal-ya marin-as avanu=ngas //
	\glc near.of that.one-\Top{} sand-\Loc{} face-\Parg{} sunken=almost //
	\glft `close to there, in the sand, an almost-sunken face.' //
\endgl\xe

\ex~\begingl
	\gla Ang ningyon igān nay nanding dijisu yona //
	\glb ang ning-yon igān-Ø nay nanding-Ø dijisu yona //
	\glc \AgtT{} tell-\TplN{} frown-\Top{} and lips-\Top{} twisted 
		\TsgN{}.\Gen{} //
	\glft `Its frown and twisted lips tell' //
\endgl\xe

\ex~\begingl
	\gla nosānas kilisarya nay sagoyamanas: //
	\glb nosān-as kilisarya nay sagoyaman-as //
	\glc command-\Parg{} strict and mocking-\Parg{} //
	\glft `of strict command and mockery' //
\endgl\xe

\ex~\begingl
	\gla Sa layaya ban-ikan tiyanyāng da-dikun //
	\glb sa laya-ya ban=ikan tiyanya-ang da=dikun-Ø //
	\glc \PatT{} read-\TsgM{} well=very creator-\Aarg{} such=passion-\Top{} //
	\glft `Very well did the creator read such passion' //
\endgl\xe

\ex~\begingl
	\gla si telujong tarela, ya saprayos linyaye: //
	\glb si telug=yong tarela ya sapra=yos linya-ye-Ø //
	\glc \Rel{} survive=\TsgN{}.\Aarg{} still \LocT{} stamp=\TsgN{}.\Parg{} 
		thing-\Pl{}-Top{} //
	\glft `which still survives, stamped into the things:' //
\endgl\xe

\ex~\begingl
	\gla sapayas si sagoyong; padangas si kondis'yong. //
	\glb sapay-as si sago=yong padang-as si kondisa=yong //
	\glc hand-\Parg{} \Rel{} mock=\TsgN{}.\Aarg{} heart-\Parg{} \Rel{} 
		feed=\TsgN{}.\Aarg{} //
	\glft `the hand that mocks; the heart that feeds.' //
\endgl\xe

\ex~\begingl
	\gla Nay sa tahanyo eda-narān bengyamanya: //
	\glb nay sa tahan-yo eda=narān-Ø bengyaman-ya //
	\glc and \PatT{} write-\TsgN{} this=word-\Top{} pedestal-\Loc{} //
	\glft `And this word is written on the pedestal:' //
\endgl\xe

\ex~\begingl
	\gla Garanang nā Simanjas, bayhiang bayhiyena: //
	\glb garan-ang nā Simanjas bayhi-ang bayhi-ye-na //
	\glc name-\Aarg{} \Fsg{}.\Gen{} Ozymandias ruler-\Aarg{} 
			ruler-\Pl{}-\Gen{} //
	\glft `My name is Ozymandias, the king of kings:' //
\endgl\xe

\ex~\begingl
	\gla Sa silvu gumo nā, nay prisu, vāng si lita! //
	\glb sa silv-u gumo-Ø nā nay pris-u vāng si lita //
	\glc \PatT{} see-\Imp{} work-\Top{} \Fsg{}.\Gen{} and tremble-\Imp{}
		\Second{}.\Aarg{} \Rel{} mighty //
	\glft `Behold my work and tremble, you who are mighty!' //
\endgl\xe

\ex~\begingl
	\gla Hangara ranyareng palung. Le apanisareng //
	\glb hang-ara ranya-reng palung le apanisa=reng //
	\glc remain-\TsgI{} nothing-\AargI{} else \PatTI{} 
		stretch=\TsgI{}.\Aarg{} //
	\glft `Nothing else remains. It stretches' //
\endgl\xe

\ex~\begingl
	\gla ahal-nama kebay, pray, soya, litoya kayvay, //
	\glb ahal-Ø=nama kebay pray soya lito-ya kayvay //
	\glc sand-\Top{}=only lonely smooth empty border-\Top{} without //
	\glft `only the lonely, smooth, empty sand, without borders,' //
\endgl\xe

\ex~\begingl
	\gla miday nernanyēa eda-kiyanena nake. //
	\glb miday nernan-ye-ya eda=kiyan-ena nake //
	\glc around part-\Pl{}-\Loc{} this=wreckage-\Gen{} large //
	\glft `around the pieces of this large wreckage.' //
\endgl\xe
